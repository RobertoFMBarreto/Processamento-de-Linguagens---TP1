
\begin{itemize}
\item {Proveniência: }
\end{itemize}\documentclass{article}
\usepackage[portuguese]{babel}
\title{I}
\begin{document}
Família de mammíferos roedores, que tem por typo o porco-espinho.
\section{Ião}
\begin{itemize}
\item {Grp. gram.:m.}
\end{itemize}
\begin{itemize}
\item {Utilização:Chím.}
\end{itemize}
\begin{itemize}
\item {Proveniência:(Fr. \textunderscore ion\textunderscore )}
\end{itemize}
Cada um dos dois corpos, que uma corrente eléctrica desaggregou.
\section{Iberização}
\begin{itemize}
\item {Grp. gram.:f.}
\end{itemize}
Acto ou effeito de iberizar.
\section{Iberizar}
\begin{itemize}
\item {Grp. gram.:v. t.}
\end{itemize}
Tornar ibérico; dar carácter ibérico a. Cf. R. Jorge, \textunderscore El Greco\textunderscore , 29.
\section{Ichthyphállico}
\begin{itemize}
\item {Grp. gram.:adj.}
\end{itemize}
\begin{itemize}
\item {Proveniência:(Do gr. \textunderscore ikhthus\textunderscore  + \textunderscore phallos\textunderscore )}
\end{itemize}
Que tem fórma de peixe e de phallo, (falando-se de um ídolo egýpcio).
\section{Ictifálico}
\begin{itemize}
\item {Grp. gram.:adj.}
\end{itemize}
\begin{itemize}
\item {Proveniência:(Do gr. \textunderscore ikhthus\textunderscore  + \textunderscore phallos\textunderscore )}
\end{itemize}
Que tem fórma de peixe e de falo, (falando-se de um ídolo egípcio).
\section{Identificável}
\begin{itemize}
\item {Grp. gram.:adj.}
\end{itemize}
Que se póde identificar.
\section{Igleja}
\begin{itemize}
\item {Grp. gram.:f.}
\end{itemize}
\begin{itemize}
\item {Utilização:Ant.}
\end{itemize}
O mesmo que \textunderscore igreja\textunderscore .
\section{Inchadura}
\begin{itemize}
\item {Grp. gram.:f.}
\end{itemize}
\begin{itemize}
\item {Utilização:Des.}
\end{itemize}
O mesmo que \textunderscore inchação\textunderscore .
\section{Ingeneroso}
\begin{itemize}
\item {Grp. gram.:adj.}
\end{itemize}
\begin{itemize}
\item {Utilização:Neol.}
\end{itemize}
\begin{itemize}
\item {Proveniência:(De \textunderscore in...\textunderscore  + \textunderscore generoso\textunderscore )}
\end{itemize}
Que não é generoso.
Em que não há generosidade:«\textunderscore allusões ingenerosas.\textunderscore »Camillo, \textunderscore Rom. de um Hom. Rico\textunderscore , 149, (ed. 1890).
\section{Ingesta}
\begin{itemize}
\item {Grp. gram.:f.}
\end{itemize}
\begin{itemize}
\item {Utilização:Med.}
\end{itemize}
\begin{itemize}
\item {Proveniência:(Lat. \textunderscore ingesta\textunderscore , pl. de \textunderscore ingestus\textunderscore )}
\end{itemize}
Designação genérica de todos os alimentos sólidos ou líquidos.
\section{Intervalarmente}
\begin{itemize}
\item {Grp. gram.:adv.}
\end{itemize}
\begin{itemize}
\item {Proveniência:(De \textunderscore intervalar\textunderscore ^2)}
\end{itemize}
Com intervalos, interruptamente. Cf. R. Jorge, \textunderscore El Greco\textunderscore , 33.
\section{Intervallarmente}
\begin{itemize}
\item {Grp. gram.:adv.}
\end{itemize}
\begin{itemize}
\item {Proveniência:(De \textunderscore intervallar\textunderscore ^2)}
\end{itemize}
Com intervallos, interruptamente. Cf. R. Jorge, \textunderscore El Greco\textunderscore , 33.
\section{I}
\begin{itemize}
\item {Grp. gram.:m.}
\end{itemize}
\begin{itemize}
\item {Grp. gram.:Adj.}
\end{itemize}
\begin{itemize}
\item {Utilização:Mús.}
\end{itemize}
\begin{itemize}
\item {Grp. gram.:Loc.}
\end{itemize}
\begin{itemize}
\item {Utilização:fam.}
\end{itemize}
Nona letra do alphabeto português.
Que numa série occupa o nono lugar: \textunderscore livro I\textunderscore ; \textunderscore fôlha I...\textunderscore 
Em numeração romana, sígnifica \textunderscore um\textunderscore .
Em Chimica, designa o \textunderscore iodo\textunderscore .
Oitavo grau da escala, na antiga notação alphabética.
\textunderscore Pôr os pontos nos ii\textunderscore , dízer tudo claramente, pôr tudo em pratos límpos, contar minuciosamente.
\section{Ia!}
\begin{itemize}
\item {Grp. gram.:interj.}
\end{itemize}
(Para fazer andar as cavalgaduras). (Colhido na Bairrada)
\section{...ia}
\begin{itemize}
\item {Grp. gram.:suf. f.}
\end{itemize}
(designativo de estado, profissão, indústria, collectividade, etc.: \textunderscore alegria\textunderscore , \textunderscore advocacia...\textunderscore )
\section{Iaca}
\begin{itemize}
\item {Grp. gram.:f.}
\end{itemize}
\begin{itemize}
\item {Utilização:Bras}
\end{itemize}
O mesmo que \textunderscore inhaca\textunderscore .
Arbusto americano, de cujo suco se servem os indígenas para narcotizar e pescar peixes.
\section{Iaçá}
\begin{itemize}
\item {Grp. gram.:m.}
\end{itemize}
\begin{itemize}
\item {Utilização:Bras}
\end{itemize}
Espécie de tracajá pequeno.
\section{Iacas}
\begin{itemize}
\item {Grp. gram.:m. pl.}
\end{itemize}
O mesmo que [[jacas|jaca]].
\section{Iacotinga}
\begin{itemize}
\item {Grp. gram.:f.}
\end{itemize}
Rocha aurífera do Brasil, quartzoza, de estructura laminosa.
\section{Iacu}
\begin{itemize}
\item {Grp. gram.:m.}
\end{itemize}
O mesmo que \textunderscore sacupema\textunderscore .
\section{Iacuto}
\begin{itemize}
\item {Grp. gram.:m.}
\end{itemize}
Idioma do grupo tartárico, vernáculo na Rússia asiática.
\section{Ia-ia}
\begin{itemize}
\item {Grp. gram.:f.}
\end{itemize}
\begin{itemize}
\item {Utilização:Bras}
\end{itemize}
Menina; sinhá.
\section{Iambes}
\begin{itemize}
\item {Grp. gram.:m. pl.}
\end{itemize}
Antigo povo da Malásia. Cf. \textunderscore Peregrinação\textunderscore , c. XVI.
\section{Iambo}
\textunderscore m.\textunderscore  (e der.)
O mesmo que \textunderscore jambo\textunderscore ^2, etc.
\section{Iamologia}
\begin{itemize}
\item {Grp. gram.:f.}
\end{itemize}
\begin{itemize}
\item {Proveniência:(Do gr. \textunderscore iama\textunderscore  + \textunderscore logos\textunderscore )}
\end{itemize}
Tratado dos medicamentos.
\section{Iamológico}
\begin{itemize}
\item {Grp. gram.:adj.}
\end{itemize}
Relativo á iamologia.
\section{Iamotéchnia}
\begin{itemize}
\item {Grp. gram.:f.}
\end{itemize}
\begin{itemize}
\item {Proveniência:(De gr. \textunderscore iama\textunderscore  + \textunderscore tekhne\textunderscore )}
\end{itemize}
Arte de preparar medicamentos.
\section{Iamotéchnico}
\begin{itemize}
\item {Grp. gram.:adj.}
\end{itemize}
Relativo á iamotechnia.
\section{Iamotecnia}
\begin{itemize}
\item {Grp. gram.:f.}
\end{itemize}
\begin{itemize}
\item {Proveniência:(De gr. \textunderscore iama\textunderscore  + \textunderscore tekhne\textunderscore )}
\end{itemize}
Arte de preparar medicamentos.
\section{Iamotecnico}
\begin{itemize}
\item {Grp. gram.:adj.}
\end{itemize}
Relativo á iamotecnia.
\section{Iande}
\begin{itemize}
\item {Grp. gram.:m.}
\end{itemize}
(V.olha-a-água)
\section{Iândom}
\begin{itemize}
\item {Grp. gram.:m.}
\end{itemize}
Espécie de avestruz americano.
\section{Iânthino}
\begin{itemize}
\item {Grp. gram.:adj.}
\end{itemize}
\begin{itemize}
\item {Proveniência:(Lat. \textunderscore ianthinus\textunderscore )}
\end{itemize}
Que tem côr de violetas, mais ou menos brilhante.
\section{Iântino}
\begin{itemize}
\item {Grp. gram.:adj.}
\end{itemize}
\begin{itemize}
\item {Proveniência:(Lat. \textunderscore ianthinus\textunderscore )}
\end{itemize}
Que tem côr de violetas, mais ou menos brilhante.
\section{Iapiruara}
\begin{itemize}
\item {Grp. gram.:m.}
\end{itemize}
\begin{itemize}
\item {Utilização:Bras. do N}
\end{itemize}
Indivíduo sertanejo.
\section{Iapoque}
\begin{itemize}
\item {Grp. gram.:m.}
\end{itemize}
Mammífero marsupial da América do Sul.
\section{Iapu}
\begin{itemize}
\item {Grp. gram.:m.}
\end{itemize}
Pássaro amarelo do Brasil.
\section{Iapuçá}
\begin{itemize}
\item {Grp. gram.:m.}
\end{itemize}
Pequeno macaco do Amazonas.
\section{Iaque}
\begin{itemize}
\item {Grp. gram.:m.}
\end{itemize}
Búfalo, com cauda de cavallo, e originário do Thibet.
\section{Iatagan}
\begin{itemize}
\item {Grp. gram.:m.}
\end{itemize}
\begin{itemize}
\item {Proveniência:(T. turco)}
\end{itemize}
Arma offensiva, de que se servem os Turcos e outros povos orientaes, e é semelhante ao punhal, mas maior do que elle.
\section{Iatai}
\begin{itemize}
\item {Grp. gram.:m.}
\end{itemize}
Espécie de coqueiro do Brasil.
\section{Iatralipta}
\begin{itemize}
\item {Grp. gram.:m.}
\end{itemize}
\begin{itemize}
\item {Proveniência:(Lat. \textunderscore iatralipta\textunderscore )}
\end{itemize}
Médico, que trata os doentes com unturas e fricções.
\section{Iatralíptica}
\begin{itemize}
\item {Grp. gram.:f.}
\end{itemize}
\begin{itemize}
\item {Proveniência:(Lat. \textunderscore iatraliptice\textunderscore )}
\end{itemize}
Parte da Medicina, que trata os doentes com medicamentos externos, como fricções, emplastos, etc.
\section{Iatralipticamente}
\begin{itemize}
\item {Grp. gram.:adv.}
\end{itemize}
De modo iatralíptico.
\section{Iatralíptico}
\begin{itemize}
\item {Grp. gram.:adj.}
\end{itemize}
Relativo á iatralíptica.
\section{Iatria}
\begin{itemize}
\item {Grp. gram.:f.}
\end{itemize}
\begin{itemize}
\item {Utilização:Des.}
\end{itemize}
\begin{itemize}
\item {Proveniência:(Lat. \textunderscore iatria\textunderscore )}
\end{itemize}
Tratamento médico; méthodo de curar.
\section{Iátrica}
\begin{itemize}
\item {Grp. gram.:f.}
\end{itemize}
\begin{itemize}
\item {Utilização:Des.}
\end{itemize}
\begin{itemize}
\item {Proveniência:(Gr. \textunderscore iatrike\textunderscore )}
\end{itemize}
Arte de curar; Medicina.
\section{Iátrico}
\begin{itemize}
\item {Grp. gram.:adj.}
\end{itemize}
Relativo á iátrica.
\section{Iátrion}
\begin{itemize}
\item {Grp. gram.:m.}
\end{itemize}
\begin{itemize}
\item {Proveniência:(Gr. \textunderscore iatrion\textunderscore )}
\end{itemize}
Lugar, onde, na antiguidade, o médico tinha os seus aparelhos, fazia operações ou dava consultas.
\section{Iatro...}
\begin{itemize}
\item {Grp. gram.:pref.}
\end{itemize}
\begin{itemize}
\item {Proveniência:(Gr. \textunderscore iatros\textunderscore )}
\end{itemize}
(significativo de \textunderscore médico\textunderscore  ou \textunderscore relativo á medicina\textunderscore )
\section{Iatrochímica}
\begin{itemize}
\item {fónica:qui}
\end{itemize}
\begin{itemize}
\item {Grp. gram.:f.}
\end{itemize}
\begin{itemize}
\item {Proveniência:(Do gr. \textunderscore iatros\textunderscore  + \textunderscore khumos\textunderscore )}
\end{itemize}
Chímica applicada á Medicina; Chímica médica.
\section{Iatrochímico}
\begin{itemize}
\item {fónica:qui}
\end{itemize}
\begin{itemize}
\item {Grp. gram.:adj.}
\end{itemize}
\begin{itemize}
\item {Grp. gram.:M.}
\end{itemize}
Relativo á iatrochímica.
Aquelle que exerce a iatrochímica.
\section{Iátrofa}
\begin{itemize}
\item {Grp. gram.:f.}
\end{itemize}
\begin{itemize}
\item {Proveniência:(Do gr. \textunderscore iatros\textunderscore  + \textunderscore phagein\textunderscore ?)}
\end{itemize}
Gênero de plantas euforbiáceas.
\section{Iatrofísica}
\begin{itemize}
\item {Grp. gram.:f.}
\end{itemize}
\begin{itemize}
\item {Proveniência:(De \textunderscore iatro...\textunderscore  + \textunderscore física\textunderscore )}
\end{itemize}
Física médica.
\section{Iatrofísico}
\begin{itemize}
\item {Grp. gram.:adj.}
\end{itemize}
Relativo á iatrofísica.
\section{Iatrologia}
\begin{itemize}
\item {Grp. gram.:f.}
\end{itemize}
\begin{itemize}
\item {Proveniência:(Do gr. \textunderscore iatros\textunderscore  + \textunderscore logos\textunderscore )}
\end{itemize}
Estudo ou sciência do tratamento das doenças.
\section{Iatromatemática}
\begin{itemize}
\item {Grp. gram.:f.}
\end{itemize}
\begin{itemize}
\item {Proveniência:(De \textunderscore iatro...\textunderscore  + \textunderscore matemática\textunderscore )}
\end{itemize}
Antigo sistema patológico, em que se procurava explicar por cálculos matemáticos os fenómenos mórbidos, como resultantes da Hidráulica e da Mecânica.
\section{Iatromatemático}
\begin{itemize}
\item {Grp. gram.:m.}
\end{itemize}
\begin{itemize}
\item {Proveniência:(De \textunderscore iatro...\textunderscore  + \textunderscore matemático\textunderscore )}
\end{itemize}
Médico, que aplicava a iatromatemática.
\section{Iatromathemática}
\begin{itemize}
\item {Grp. gram.:f.}
\end{itemize}
\begin{itemize}
\item {Proveniência:(De \textunderscore iatro...\textunderscore  + \textunderscore mathemática\textunderscore )}
\end{itemize}
Antigo systema pathológico, em que se procurava explicar por cálculos mathemáticos os phenómenos mórbidos, como resultantes da Hydráulica e da Mecânica.
\section{Iatromathemático}
\begin{itemize}
\item {Grp. gram.:m.}
\end{itemize}
\begin{itemize}
\item {Proveniência:(De \textunderscore iatro...\textunderscore  + \textunderscore mathemático\textunderscore )}
\end{itemize}
Médico, que applicava a iatromathemática.
\section{Iatromecânica}
\begin{itemize}
\item {Grp. gram.:f.}
\end{itemize}
\begin{itemize}
\item {Proveniência:(De \textunderscore iatro...\textunderscore  + \textunderscore mecânica\textunderscore )}
\end{itemize}
Systema médico, que refere todas as fôrças vitaes a acções mecânicas; iatromathemática.
\section{Iatromecânico}
\begin{itemize}
\item {Grp. gram.:m.}
\end{itemize}
Sectário da iatromecânica.
\section{Iátropha}
\begin{itemize}
\item {Grp. gram.:f.}
\end{itemize}
\begin{itemize}
\item {Proveniência:(Do gr. \textunderscore iatros\textunderscore  + \textunderscore phagein\textunderscore ?)}
\end{itemize}
Gênero de plantas euphorbiáceas.
\section{Iatrophýsica}
\begin{itemize}
\item {Grp. gram.:f.}
\end{itemize}
\begin{itemize}
\item {Proveniência:(De \textunderscore iatro...\textunderscore  + \textunderscore phýsica\textunderscore )}
\end{itemize}
Phýsica médica.
\section{Iatrophýsico}
\begin{itemize}
\item {Grp. gram.:adj.}
\end{itemize}
Relativo á iatrophýsica.
\section{Iatroquímica}
\begin{itemize}
\item {Grp. gram.:f.}
\end{itemize}
\begin{itemize}
\item {Proveniência:(Do gr. \textunderscore iatros\textunderscore  + \textunderscore khumos\textunderscore )}
\end{itemize}
Química aplicada á Medicina; Química médica.
\section{Iatroquímico}
\begin{itemize}
\item {Grp. gram.:adj.}
\end{itemize}
\begin{itemize}
\item {Grp. gram.:M.}
\end{itemize}
Relativo á iatroquímica.
Aquele que exerce a iatroquímica.
\section{Iba}
\begin{itemize}
\item {Grp. gram.:f.}
\end{itemize}
Fruto da oba, árvore africana.
\section{Ibabiraba}
\begin{itemize}
\item {Grp. gram.:f.}
\end{itemize}
Árvore myrtácea do Brasil.
\section{Ibacurupari}
\begin{itemize}
\item {Grp. gram.:f.}
\end{itemize}
Árvore fructífera do Brasil.
\section{Ibairiba}
\begin{itemize}
\item {fónica:ba-i}
\end{itemize}
\begin{itemize}
\item {Grp. gram.:f.}
\end{itemize}
Árvore leguminosa do Brasil, (\textunderscore andira rosea\textunderscore ).
\section{Ibamerato}
\begin{itemize}
\item {Grp. gram.:m.}
\end{itemize}
Designação genérica do coqueiro, no Brasil.
\section{Ibapuringa}
\begin{itemize}
\item {Grp. gram.:f.}
\end{itemize}
Árvore rhamnácea do Brasil.
\section{Iberíaco}
\begin{itemize}
\item {Grp. gram.:adj.}
\end{itemize}
\begin{itemize}
\item {Proveniência:(Lat. \textunderscore iberiacus\textunderscore )}
\end{itemize}
O mesmo que \textunderscore ibérico\textunderscore .
\section{Ibérico}
\begin{itemize}
\item {Grp. gram.:adj.}
\end{itemize}
\begin{itemize}
\item {Grp. gram.:M.}
\end{itemize}
\begin{itemize}
\item {Proveniência:(Lat. \textunderscore ibericus\textunderscore )}
\end{itemize}
Relativo á Ibéria.
Relativo á peninsula hispânica.
Partidário da união política de Portugal com a Espanha.
\section{Iberino}
\begin{itemize}
\item {Grp. gram.:adj.}
\end{itemize}
O mesmo que \textunderscore ibérico\textunderscore .
\section{Ibéris}
\begin{itemize}
\item {Grp. gram.:f.}
\end{itemize}
\begin{itemize}
\item {Proveniência:(Gr. \textunderscore iberis\textunderscore )}
\end{itemize}
Gênero de plantas crucíferas.
\section{Iberismo}
\begin{itemize}
\item {Grp. gram.:m.}
\end{itemize}
Partido dos que pretendem a união política de Portugal com a Espanha.
(Cp. \textunderscore ibérico\textunderscore )
\section{Ibero}
\begin{itemize}
\item {Grp. gram.:adj.}
\end{itemize}
\begin{itemize}
\item {Grp. gram.:M. pl.}
\end{itemize}
\begin{itemize}
\item {Proveniência:(Lat. \textunderscore iberus\textunderscore )}
\end{itemize}
Relativo á Ibéria.
Antigos povoadores da Espanha. Cf. Latino, \textunderscore Elogios\textunderscore , 67 e 72.
\section{Ibero...}
Elemento, que entra na formação de algumas palavras, com a significação de \textunderscore ibérico\textunderscore .
\section{Ibero-americano}
\begin{itemize}
\item {Grp. gram.:adj.}
\end{itemize}
Relativo aos povos americanos, que procederam da península ibérica.
\section{Ibero-céltico}
\begin{itemize}
\item {Grp. gram.:adj.}
\end{itemize}
O mesmo que \textunderscore celtibérico\textunderscore .
\section{Ibibora}
\begin{itemize}
\item {Grp. gram.:f.}
\end{itemize}
Espécie de serpente do Brasil.
\section{Ibipitanga}
\begin{itemize}
\item {Grp. gram.:f.}
\end{itemize}
(V.pitangueira)
\section{Ibiquara}
\begin{itemize}
\item {Grp. gram.:f.}
\end{itemize}
\begin{itemize}
\item {Utilização:Bras. de Minas}
\end{itemize}
Designação de qualquer sarcóphago indiano.
\section{Ibiquiba}
\begin{itemize}
\item {fónica:cu-i}
\end{itemize}
\begin{itemize}
\item {Grp. gram.:f.}
\end{itemize}
Fruto brasileiro, do feitio da bolota.
\section{Ibira}
\begin{itemize}
\item {Grp. gram.:f.}
\end{itemize}
Arbusto anonáceo do Brasil.
\section{Ibiracém}
\begin{itemize}
\item {Grp. gram.:m.}
\end{itemize}
Arbusto solâneo, (\textunderscore liquirita silvestris\textunderscore ).
\section{Ibirapitanga}
\begin{itemize}
\item {Grp. gram.:f.}
\end{itemize}
O mesmo que \textunderscore pau-brasil\textunderscore .
\section{Ibirarema}
\begin{itemize}
\item {Grp. gram.:f.}
\end{itemize}
Planta phytolácea, originária da América.
\section{Ibirizateia}
\begin{itemize}
\item {Grp. gram.:f.}
\end{itemize}
Árvore brasileira, de cerne preto e duríssimo.
\section{Ibirubá}
\begin{itemize}
\item {Grp. gram.:m.}
\end{itemize}
Pitangueira do mato.
\section{Ibirube}
\begin{itemize}
\item {Grp. gram.:f.}
\end{itemize}
\begin{itemize}
\item {Utilização:Bras}
\end{itemize}
(V.jaracatiá)
\section{Íbis}
\begin{itemize}
\item {Grp. gram.:m.  e  f.}
\end{itemize}
\begin{itemize}
\item {Proveniência:(Lat. \textunderscore ibis\textunderscore )}
\end{itemize}
Espécie de pequena cegonha.
Ave, também pernalta, a que os Egýpcios prestaram culto especial.
\section{Ibixuma}
\begin{itemize}
\item {Grp. gram.:f.}
\end{itemize}
O mesmo que \textunderscore mutamba\textunderscore .
\section{Iboga}
\begin{itemize}
\item {Grp. gram.:f.}
\end{itemize}
Planta do Congo, que os indígenas usam como excitante, de effeitos análogos aos do álcool.
\section{Ibogaína}
\begin{itemize}
\item {Grp. gram.:f.}
\end{itemize}
Alcaloide da iboga, cuja ácção se exerce especialmente no systema bulbo-rachidiano. Cf. \textunderscore Diário Official\textunderscore , do Brasil, de 21-II-902.
\section{Ibondeiro}
\begin{itemize}
\item {Grp. gram.:m.}
\end{itemize}
(V.imbondeiro)
\section{Iça}
\begin{itemize}
\item {Grp. gram.:f.}
\end{itemize}
\begin{itemize}
\item {Utilização:Ant.}
\end{itemize}
O mesmo que \textunderscore concubina\textunderscore .
\section{Içá}
\begin{itemize}
\item {Grp. gram.:f.}
\end{itemize}
\begin{itemize}
\item {Grp. gram.:M.}
\end{itemize}
Formiga avermelhada das roças do Brasil.
Espécie de macaco do Amazonas.
\section{Icacína}
\begin{itemize}
\item {Grp. gram.:f.}
\end{itemize}
Gênero de plantas oleáceas.
\section{Icamiabas}
\begin{itemize}
\item {Grp. gram.:m. pl.}
\end{itemize}
Cabildas de Indios, que habitavam na Guiana brasileira.
\section{Içar}
\begin{itemize}
\item {Grp. gram.:v. t.}
\end{itemize}
\begin{itemize}
\item {Grp. gram.:V. i.}
\end{itemize}
\begin{itemize}
\item {Utilização:Náut.}
\end{itemize}
Levantar, erguer.
\textunderscore Içar a beijo\textunderscore , içar a tope.
(Do germ.: al. \textunderscore hissen\textunderscore , suéco \textunderscore iça\textunderscore , din. \textunderscore heise\textunderscore , devendo por isso escrever-se \textunderscore issar\textunderscore  ou \textunderscore hissar\textunderscore )
\section{Içara}
\begin{itemize}
\item {Grp. gram.:f.}
\end{itemize}
\begin{itemize}
\item {Utilização:Bras}
\end{itemize}
Palmeira do sertão.
\section{Icariba}
\begin{itemize}
\item {Grp. gram.:f.}
\end{itemize}
(V.icica)
\section{Icário}
\begin{itemize}
\item {Grp. gram.:adj.}
\end{itemize}
\begin{itemize}
\item {Proveniência:(Lat. \textunderscore icarius\textunderscore )}
\end{itemize}
Relativo a ícaro, próprio de ícaro.
\section{Ícaro}
\begin{itemize}
\item {Grp. gram.:m.}
\end{itemize}
\begin{itemize}
\item {Utilização:Fig.}
\end{itemize}
\begin{itemize}
\item {Proveniência:(De \textunderscore Icaro\textunderscore , n. p. myth.)}
\end{itemize}
Indivíduo, a quem foram funestas as suas elevadas pretensões ou ambições.
\section{Içás}
\begin{itemize}
\item {Grp. gram.:m. pl.}
\end{itemize}
Indígenas brasileiros das margens do Japurá.
\section{Icástico}
\begin{itemize}
\item {Grp. gram.:adj.}
\end{itemize}
\begin{itemize}
\item {Proveniência:(Gr. \textunderscore eikastikos\textunderscore )}
\end{itemize}
Natural.
Que não tem artifícios.
Que representa claramente uma ideia.
\section{Icebergue}
\begin{itemize}
\item {Grp. gram.:m.}
\end{itemize}
\begin{itemize}
\item {Proveniência:(Ingl. \textunderscore iceberg\textunderscore )}
\end{itemize}
Grande massa de gêlo que, desprendendo-se do Oceano polar do norte, fluctua impellida pelas correntes marítimas.
\section{Icéria}
\begin{itemize}
\item {Grp. gram.:f.}
\end{itemize}
Espécie de cochinilha, que é originária da Austrália e ataca as árvores, sugando-lhes a seiva das fôlhas, (\textunderscore iceria purchasi\textunderscore , Maskell).
\section{Icérya}
\begin{itemize}
\item {Grp. gram.:f.}
\end{itemize}
Espécie de cochinilha, que é originária da Austrália e ataca as árvores, sugando-lhes a seiva das fôlhas, (\textunderscore iceria purchasi\textunderscore , Maskell).
\section{Ichacorvar}
\begin{itemize}
\item {Grp. gram.:v. i.}
\end{itemize}
\begin{itemize}
\item {Utilização:Des.}
\end{itemize}
\begin{itemize}
\item {Proveniência:(De \textunderscore ichacorvos\textunderscore )}
\end{itemize}
Andar prègando ao povo pelas aldeias.
\section{Ichacorvos}
\begin{itemize}
\item {Grp. gram.:m.}
\end{itemize}
(V.echacorvos)
\section{Ichão}
\begin{itemize}
\item {Grp. gram.:m.}
\end{itemize}
\begin{itemize}
\item {Utilização:Ant.}
\end{itemize}
Medida itinerária asiática, equivalente a seis léguas antigas de Portugal.
\section{Ichneumon}
\begin{itemize}
\item {Grp. gram.:m.}
\end{itemize}
\begin{itemize}
\item {Proveniência:(Lat. \textunderscore ichneumon\textunderscore )}
\end{itemize}
O mesmo que \textunderscore mangusto\textunderscore ^1.
Gênero de insectos hymenópteros.
\section{Ichnographia}
\begin{itemize}
\item {Grp. gram.:f.}
\end{itemize}
\begin{itemize}
\item {Proveniência:(Lat. \textunderscore ichnographía\textunderscore )}
\end{itemize}
Plano horizontal ou planta de um edifício.
Arte de fazer êsses planos ou plantas.
\section{Ichnographicamente}
\begin{itemize}
\item {Grp. gram.:adv.}
\end{itemize}
De modo ichnográphico.
\section{Ichnográphico}
\begin{itemize}
\item {Grp. gram.:adj.}
\end{itemize}
Relativo á ichnographia.
\section{Ichnógrapho}
\begin{itemize}
\item {Grp. gram.:m.}
\end{itemize}
\begin{itemize}
\item {Proveniência:(Do gr. \textunderscore ikhnos\textunderscore  + \textunderscore graphein\textunderscore )}
\end{itemize}
Aquelle que faz plantas ou planos de edifícios.
Engenheiro versado em ichnographia.
\section{Ichó}
\begin{itemize}
\item {Grp. gram.:m.  e  f.}
\end{itemize}
\begin{itemize}
\item {Proveniência:(Do lat. \textunderscore ustiolum\textunderscore )}
\end{itemize}
Armadilha para coêlhos ou perdizes.
\section{Ichor}
\begin{itemize}
\item {fónica:côr}
\end{itemize}
\begin{itemize}
\item {Grp. gram.:m.}
\end{itemize}
\begin{itemize}
\item {Proveniência:(Gr. \textunderscore ikhor\textunderscore )}
\end{itemize}
Humor purulento, que escorre de certas úlceras.
\section{Ichoroso}
\begin{itemize}
\item {fónica:co}
\end{itemize}
\begin{itemize}
\item {Grp. gram.:adj.}
\end{itemize}
Que tem ichor ou é da natureza do ichor.
\section{Ichós}
\begin{itemize}
\item {Grp. gram.:m.  e  f.}
\end{itemize}
O mesmo que \textunderscore ichó\textunderscore .
\section{Ichthýaco}
\begin{itemize}
\item {Grp. gram.:adj.}
\end{itemize}
(V.ichthýico)
\section{Ichthýico}
\begin{itemize}
\item {Grp. gram.:adj.}
\end{itemize}
\begin{itemize}
\item {Proveniência:(Gr. \textunderscore ikhthuikos\textunderscore )}
\end{itemize}
Relativo a peixe; próprio de peixe.
\section{Ichthyo...}
\begin{itemize}
\item {Proveniência:(Do gr. \textunderscore ikhthus\textunderscore )}
\end{itemize}
Elemento, que entra na formação de algumas palavras, com a significação de \textunderscore peixe\textunderscore  ou \textunderscore relativo a peixe\textunderscore .
\section{Ichthyocolla}
\begin{itemize}
\item {Grp. gram.:f.}
\end{itemize}
\begin{itemize}
\item {Proveniência:(Lat. \textunderscore ichthyocolla\textunderscore )}
\end{itemize}
Colla de peixe.
\section{Ichthyodonte}
\begin{itemize}
\item {Grp. gram.:adj.}
\end{itemize}
\begin{itemize}
\item {Proveniência:(Do gr. \textunderscore ikthus\textunderscore  + \textunderscore odous\textunderscore )}
\end{itemize}
Dente fóssil de peixe.
\section{Ichthyodorýlitho}
\begin{itemize}
\item {Grp. gram.:m.}
\end{itemize}
\begin{itemize}
\item {Proveniência:(Do gr. \textunderscore ikhthus\textunderscore  + \textunderscore dorus\textunderscore  + \textunderscore lithos\textunderscore )}
\end{itemize}
Substância fóssil, cónica e alongada, que se suppõe serem espinhos das barbatanas de certos peixes cartilagíneos.
\section{Ichthyographia}
\begin{itemize}
\item {Grp. gram.:f.}
\end{itemize}
Descripção dos peixes.
(Cp. \textunderscore ichthyógrapho\textunderscore )
\section{Ichthyográphico}
\begin{itemize}
\item {Grp. gram.:adj.}
\end{itemize}
Relativo á ichthyographia.
\section{Ichthyógrapho}
\begin{itemize}
\item {Grp. gram.:m.}
\end{itemize}
\begin{itemize}
\item {Proveniência:(Do gr. \textunderscore ikhthus\textunderscore  + \textunderscore graphein\textunderscore )}
\end{itemize}
Aquelle que é versado em ichthyographia.
\section{Ichthyoide}
\begin{itemize}
\item {Grp. gram.:adj.}
\end{itemize}
\begin{itemize}
\item {Proveniência:(Do gr. \textunderscore ikhthus\textunderscore  + \textunderscore eidos\textunderscore )}
\end{itemize}
Semelhante a um peixe.
\section{Ichthyoídeo}
\begin{itemize}
\item {Grp. gram.:adj.}
\end{itemize}
\begin{itemize}
\item {Proveniência:(Do gr. \textunderscore ikhthus\textunderscore  + \textunderscore eidos\textunderscore )}
\end{itemize}
Semelhante a um peixe.
\section{Ichthyol}
\begin{itemize}
\item {Grp. gram.:m.}
\end{itemize}
Producto medicinal da destillação de uma rocha bituminosa do Tirol.
\section{Ichthyólitho}
\begin{itemize}
\item {Grp. gram.:m.}
\end{itemize}
\begin{itemize}
\item {Proveniência:(Do gr. \textunderscore ikhthus\textunderscore  + \textunderscore lithos\textunderscore )}
\end{itemize}
Peixe fóssil.
\section{Ichthyologia}
\begin{itemize}
\item {Grp. gram.:f.}
\end{itemize}
Parte da Zoologia, que se occupa dos peixes.
(Cp. \textunderscore ichthyólogo\textunderscore )
\section{Icneumon}
\begin{itemize}
\item {Grp. gram.:m.}
\end{itemize}
\begin{itemize}
\item {Proveniência:(Lat. \textunderscore ichneumon\textunderscore )}
\end{itemize}
O mesmo que \textunderscore mangusto\textunderscore ^1.
Gênero de insectos himenópteros.
\section{Icnografia}
\begin{itemize}
\item {Grp. gram.:f.}
\end{itemize}
\begin{itemize}
\item {Proveniência:(Lat. \textunderscore ichnographía\textunderscore )}
\end{itemize}
Plano horizontal ou planta de um edifício.
Arte de fazer êsses planos ou plantas.
\section{Icnograficamente}
\begin{itemize}
\item {Grp. gram.:adv.}
\end{itemize}
De modo icnográfico.
\section{Icnográfico}
\begin{itemize}
\item {Grp. gram.:adj.}
\end{itemize}
Relativo á icnografia.
\section{Icnógrafo}
\begin{itemize}
\item {Grp. gram.:m.}
\end{itemize}
\begin{itemize}
\item {Proveniência:(Do gr. \textunderscore ikhnos\textunderscore  + \textunderscore graphein\textunderscore )}
\end{itemize}
Aquele que faz plantas ou planos de edifícios.
Engenheiro versado em icnografia.
\section{Icor}
\begin{itemize}
\item {Grp. gram.:m.}
\end{itemize}
\begin{itemize}
\item {Proveniência:(Gr. \textunderscore ikhor\textunderscore )}
\end{itemize}
Humor purulento, que escorre de certas úlceras.
\section{Icoroso}
\begin{itemize}
\item {Grp. gram.:adj.}
\end{itemize}
Que tem icor ou é da natureza do icor.
\section{Ictíaco}
\begin{itemize}
\item {Grp. gram.:adj.}
\end{itemize}
(V.ictíico)
\section{Ictíico}
\begin{itemize}
\item {Grp. gram.:adj.}
\end{itemize}
\begin{itemize}
\item {Proveniência:(Gr. \textunderscore ikhthuikos\textunderscore )}
\end{itemize}
Relativo a peixe; próprio de peixe.
\section{Ictio...}
\begin{itemize}
\item {Proveniência:(Do gr. \textunderscore ikhthus\textunderscore )}
\end{itemize}
Elemento, que entra na formação de algumas palavras, com a significação de \textunderscore peixe\textunderscore  ou \textunderscore relativo a peixe\textunderscore .
\section{Ictiocola}
\begin{itemize}
\item {Grp. gram.:f.}
\end{itemize}
\begin{itemize}
\item {Proveniência:(Lat. \textunderscore ichthyocolla\textunderscore )}
\end{itemize}
Cola de peixe.
\section{Ictiodonte}
\begin{itemize}
\item {Grp. gram.:adj.}
\end{itemize}
\begin{itemize}
\item {Proveniência:(Do gr. \textunderscore ikthus\textunderscore  + \textunderscore odous\textunderscore )}
\end{itemize}
Dente fóssil de peixe.
\section{Ictiodorílito}
\begin{itemize}
\item {Grp. gram.:m.}
\end{itemize}
\begin{itemize}
\item {Proveniência:(Do gr. \textunderscore ikhthus\textunderscore  + \textunderscore dorus\textunderscore  + \textunderscore lithos\textunderscore )}
\end{itemize}
Substância fóssil, cónica e alongada, que se supõe serem espinhos das barbatanas de certos peixes cartilagíneos.
\section{Ictiografia}
\begin{itemize}
\item {Grp. gram.:f.}
\end{itemize}
Descripção dos peixes.
(Cp. \textunderscore ictiógrafo\textunderscore )
\section{Ictiográfico}
\begin{itemize}
\item {Grp. gram.:adj.}
\end{itemize}
Relativo á ictiografia.
\section{Ictiógrafo}
\begin{itemize}
\item {Grp. gram.:m.}
\end{itemize}
\begin{itemize}
\item {Proveniência:(Do gr. \textunderscore ikhthus\textunderscore  + \textunderscore graphein\textunderscore )}
\end{itemize}
Aquele que é versado em ictiografia.
\section{Ictioide}
\begin{itemize}
\item {Grp. gram.:adj.}
\end{itemize}
\begin{itemize}
\item {Proveniência:(Do gr. \textunderscore ikhthus\textunderscore  + \textunderscore eidos\textunderscore )}
\end{itemize}
Semelhante a um peixe.
\section{Ictioídeo}
\begin{itemize}
\item {Grp. gram.:adj.}
\end{itemize}
\begin{itemize}
\item {Proveniência:(Do gr. \textunderscore ikhthus\textunderscore  + \textunderscore eidos\textunderscore )}
\end{itemize}
Semelhante a um peixe.
\section{Ictiol}
\begin{itemize}
\item {Grp. gram.:m.}
\end{itemize}
Producto medicinal da destilação de uma rocha bituminosa do Tirol.
\section{Ictiólito}
\begin{itemize}
\item {Grp. gram.:m.}
\end{itemize}
\begin{itemize}
\item {Proveniência:(Do gr. \textunderscore ikhthus\textunderscore  + \textunderscore lithos\textunderscore )}
\end{itemize}
Peixe fóssil.
\section{Ictiologia}
\begin{itemize}
\item {Grp. gram.:f.}
\end{itemize}
Parte da Zoologia, que se ocupa dos peixes.
(Cp. \textunderscore ictiólogo\textunderscore )
\section{Ichthyológico}
\begin{itemize}
\item {Grp. gram.:adj.}
\end{itemize}
Relativo á ichthyologia.
\section{Ichthyólogo}
\begin{itemize}
\item {Grp. gram.:m.}
\end{itemize}
\begin{itemize}
\item {Proveniência:(Do gr. \textunderscore ikhthus\textunderscore  + \textunderscore logos\textunderscore )}
\end{itemize}
Naturalista, que trata de ichthyologia.
\section{Ichthyophagia}
\begin{itemize}
\item {Grp. gram.:f.}
\end{itemize}
Hábito de se alimentar de peixes.
(Cp. \textunderscore ichthyóphago\textunderscore )
\section{Ichthyophágico}
\begin{itemize}
\item {Grp. gram.:adj.}
\end{itemize}
Relativo á ichthyophagia.
\section{Ichthyóphago}
\begin{itemize}
\item {Grp. gram.:adj.}
\end{itemize}
\begin{itemize}
\item {Grp. gram.:M.}
\end{itemize}
\begin{itemize}
\item {Proveniência:(Do gr. \textunderscore ikhthus\textunderscore  + \textunderscore phagein\textunderscore )}
\end{itemize}
Relativo á ichthyophagia.
Aquelle que se alimenta de peixes.
\section{Ichthyopsophose}
\begin{itemize}
\item {Grp. gram.:f.}
\end{itemize}
\begin{itemize}
\item {Proveniência:(Do gr. \textunderscore ikhthus\textunderscore  + \textunderscore psophos\textunderscore )}
\end{itemize}
Rumor, produzido pelos peixes debaixo da água, o qual parece devido á vibração dos músculos da vesícula pulmonar.
\section{Ichthyosáurio}
\begin{itemize}
\item {fónica:sáu}
\end{itemize}
\begin{itemize}
\item {Grp. gram.:m.}
\end{itemize}
\begin{itemize}
\item {Proveniência:(De \textunderscore ichthyo...\textunderscore  + \textunderscore sáurio\textunderscore )}
\end{itemize}
Reptil marinho, hoje fóssil, pertencente ao segundo período geológico.
\section{Ichthyose}
\begin{itemize}
\item {Grp. gram.:f.}
\end{itemize}
\begin{itemize}
\item {Proveniência:(Do gr. \textunderscore ikhthus\textunderscore )}
\end{itemize}
Doença cutânea, caracterizada por escamas.
\section{Ichthyospôndylo}
\begin{itemize}
\item {Grp. gram.:m.}
\end{itemize}
\begin{itemize}
\item {Proveniência:(Do gr. \textunderscore ikhthus\textunderscore  + \textunderscore spondulos\textunderscore )}
\end{itemize}
Vértebra fóssil de peixe.
\section{Icica}
\begin{itemize}
\item {Grp. gram.:f.}
\end{itemize}
Gênero de plantas terebintháceas do Brasil.
Variedade de cipó.
\section{Icicana}
\begin{itemize}
\item {Grp. gram.:f.}
\end{itemize}
\begin{itemize}
\item {Utilização:Chím.}
\end{itemize}
Corpo crystallizável, extrahido da resina de uma espécie de icica.
\section{Icicariba}
\begin{itemize}
\item {Grp. gram.:f.}
\end{itemize}
(V.icica)
\section{Icipó}
\begin{itemize}
\item {Grp. gram.:m.}
\end{itemize}
Arbusto dilleniáceo do Brasil.
\section{Icó}
\begin{itemize}
\item {Grp. gram.:m.}
\end{itemize}
Planta capparídea do Brasil.
\section{Icónico}
\begin{itemize}
\item {Grp. gram.:adj.}
\end{itemize}
\begin{itemize}
\item {Proveniência:(Lat. \textunderscore iconicus\textunderscore )}
\end{itemize}
O mesmo que \textunderscore icástico\textunderscore .
Dizia-se principalmente das estatuas de tamanho natural, erguidas aos que três vezes tinham sido vencedores nos jogos sagrados.
\section{Iconista}
\begin{itemize}
\item {Grp. gram.:m.}
\end{itemize}
\begin{itemize}
\item {Proveniência:(Do gr. \textunderscore eikon\textunderscore , imagem)}
\end{itemize}
Autor de imagens ou estátuas.
\section{Iconoclasmo}
\begin{itemize}
\item {Grp. gram.:m.}
\end{itemize}
Doutrina dos iconoclastas.
(Cp. \textunderscore iconoclasta\textunderscore )
\section{Iconoclasta}
\begin{itemize}
\item {Grp. gram.:m.  e  adj.}
\end{itemize}
\begin{itemize}
\item {Utilização:Ext.}
\end{itemize}
\begin{itemize}
\item {Utilização:Fig.}
\end{itemize}
\begin{itemize}
\item {Proveniência:(Gr. \textunderscore eikonoklastes\textunderscore )}
\end{itemize}
Destruidor de imagens religiosas ou de ídolos.
Adversário da representação de coisas ou pessôas divinas.
Aquelle que não respeita tradições e monumentos.
Aquelle que mina o crédito ou a reputação alheia.
\section{Iconófilo}
\begin{itemize}
\item {Grp. gram.:m.}
\end{itemize}
\begin{itemize}
\item {Proveniência:(Do gr. \textunderscore eikon\textunderscore  + \textunderscore philos\textunderscore )}
\end{itemize}
Aquele que gosta das imagens ou quadros.
\section{Iconografia}
\begin{itemize}
\item {Grp. gram.:f.}
\end{itemize}
\begin{itemize}
\item {Proveniência:(Lat. \textunderscore iconographia\textunderscore )}
\end{itemize}
Conhecimento e descripção de imagens, estátuas, monumentos antigos, etc.
\section{Iconográfico}
\begin{itemize}
\item {Grp. gram.:adj.}
\end{itemize}
Relativo á iconografia.
\section{Iconógrafo}
\begin{itemize}
\item {Grp. gram.:m.}
\end{itemize}
\begin{itemize}
\item {Proveniência:(Do gr. \textunderscore eikon\textunderscore  + gr. \textunderscore graphein\textunderscore )}
\end{itemize}
Aquele que é versado em iconografia ou que se ocupa dela.
\section{Iconographia}
\begin{itemize}
\item {Grp. gram.:f.}
\end{itemize}
\begin{itemize}
\item {Proveniência:(Lat. \textunderscore iconographia\textunderscore )}
\end{itemize}
Conhecimento e descripção de imagens, estátuas, monumentos antigos, etc.
\section{Iconográphico}
\begin{itemize}
\item {Grp. gram.:adj.}
\end{itemize}
Relativo á iconographia.
\section{Iconógrapho}
\begin{itemize}
\item {Grp. gram.:m.}
\end{itemize}
\begin{itemize}
\item {Proveniência:(Do gr. \textunderscore eikon\textunderscore  + gr. \textunderscore graphein\textunderscore )}
\end{itemize}
Aquelle que é versado em iconographia ou que se occupa della.
\section{Iconólatra}
\begin{itemize}
\item {Grp. gram.:m.}
\end{itemize}
Aquelle que pratica a iconolatria.
\section{Iconolatria}
\begin{itemize}
\item {Grp. gram.:f.}
\end{itemize}
\begin{itemize}
\item {Proveniência:(Do gr. \textunderscore eikon\textunderscore  + \textunderscore latreia\textunderscore )}
\end{itemize}
Adoração das imagens.
\section{Iconologia}
\begin{itemize}
\item {Grp. gram.:f.}
\end{itemize}
Explicação das figuras allegóricas e dos seus attributos.
Explicação de imagens ou monumentos antigos.
Representação de entidades moraes, por emblemas ou figuras allegóricas.
(Cp. \textunderscore iconólogo\textunderscore )
\section{Iconológico}
\begin{itemize}
\item {Grp. gram.:adj.}
\end{itemize}
Relativo á iconologia.
\section{Iconologista}
\begin{itemize}
\item {Grp. gram.:m.}
\end{itemize}
O mesmo que \textunderscore iconólogo\textunderscore .
\section{Iconólogo}
\begin{itemize}
\item {Grp. gram.:m.}
\end{itemize}
\begin{itemize}
\item {Proveniência:(Do gr. \textunderscore eikon\textunderscore  + \textunderscore logos\textunderscore )}
\end{itemize}
Aquelle que é versado em iconologia ou que se occupa della.
\section{Iconómaco}
\begin{itemize}
\item {Grp. gram.:m.}
\end{itemize}
\begin{itemize}
\item {Proveniência:(Gr. \textunderscore eikonomakos\textunderscore )}
\end{itemize}
Aquelle que combate o culto das imagens.
\section{Iconomania}
\begin{itemize}
\item {Grp. gram.:f.}
\end{itemize}
\begin{itemize}
\item {Proveniência:(Do gr. \textunderscore eikon\textunderscore  + \textunderscore mania\textunderscore )}
\end{itemize}
Paixão por imagens ou quadros.
\section{Iconómetro}
\begin{itemize}
\item {Grp. gram.:m.}
\end{itemize}
\begin{itemize}
\item {Proveniência:(Do gr. \textunderscore eikon\textunderscore  + \textunderscore metron\textunderscore )}
\end{itemize}
Apparelho, destinado a fixar o ponto de vista nas photographias, fóra do laboratório.
\section{Iconóphilo}
\begin{itemize}
\item {Grp. gram.:m.}
\end{itemize}
\begin{itemize}
\item {Proveniência:(Do gr. \textunderscore eikon\textunderscore  + \textunderscore philos\textunderscore )}
\end{itemize}
Aquelle que gosta das imagens ou quadros.
\section{Iconóstase}
\begin{itemize}
\item {Grp. gram.:f.}
\end{itemize}
\begin{itemize}
\item {Proveniência:(Do gr. \textunderscore eikon\textunderscore  + \textunderscore stasis\textunderscore )}
\end{itemize}
Construcção, que é uma espécie de grande biombo ou anteparo, com três portas, e carregado de imagens, usado nas igrejas da religião grega.
\section{Iconóstrofo}
\begin{itemize}
\item {Grp. gram.:m.}
\end{itemize}
\begin{itemize}
\item {Proveniência:(Do gr. \textunderscore eikon\textunderscore  + \textunderscore strephein\textunderscore )}
\end{itemize}
Instrumento de óptica, que inverte os objectos á vista, servindo aos gravadores na cópia do modêlo.
\section{Iconóstropho}
\begin{itemize}
\item {Grp. gram.:m.}
\end{itemize}
\begin{itemize}
\item {Proveniência:(Do gr. \textunderscore eikon\textunderscore  + \textunderscore strephein\textunderscore )}
\end{itemize}
Instrumento de óptica, que inverte os objectos á vista, servindo aos gravadores na cópia do modêlo.
\section{Icós}
\begin{itemize}
\item {Grp. gram.:m. pl.}
\end{itemize}
Tríbo de Índios do Brasil, de que há restos no Rio-Grande-do-Norte.
\section{Icosaédro}
\begin{itemize}
\item {Grp. gram.:m.}
\end{itemize}
\begin{itemize}
\item {Proveniência:(Lat. \textunderscore icosahedrum\textunderscore )}
\end{itemize}
Polyedro de vinte faces ou bases.
\section{Icosandra}
\begin{itemize}
\item {Grp. gram.:f.}
\end{itemize}
Gênero de sapotáceas que, em Bornéu, fornecem gutapercha.
(Cp. \textunderscore icosandro\textunderscore )
\section{Icosandria}
\begin{itemize}
\item {Grp. gram.:f.}
\end{itemize}
Qualidade de icosandro.
Classe das plantas, que têm vinte ou mais estames inseridos na parede interna do cálice.
\section{Icosândrico}
\begin{itemize}
\item {Grp. gram.:adj.}
\end{itemize}
\begin{itemize}
\item {Proveniência:(Do gr. \textunderscore eikosi\textunderscore  + \textunderscore aner\textunderscore , \textunderscore andros\textunderscore )}
\end{itemize}
Diz-se dos vegetaes, que têm vinte ou mais estames inseridos no cálice.
\section{Icosandro}
\begin{itemize}
\item {Grp. gram.:adj.}
\end{itemize}
\begin{itemize}
\item {Proveniência:(Do gr. \textunderscore eikosi\textunderscore  + \textunderscore aner\textunderscore , \textunderscore andros\textunderscore )}
\end{itemize}
Diz-se dos vegetaes, que têm vinte ou mais estames inseridos no cálice.
\section{Icositetraédro}
\begin{itemize}
\item {Grp. gram.:m.}
\end{itemize}
Polyedro, limitado por vinte deltoides iguaes.
\section{Ictéria}
\begin{itemize}
\item {Grp. gram.:f.}
\end{itemize}
\begin{itemize}
\item {Proveniência:(Do lat. \textunderscore icterus\textunderscore )}
\end{itemize}
Gênero de aves, cuja única espécie vive na América do Norte.
\section{Icterícia}
\begin{itemize}
\item {Grp. gram.:f.}
\end{itemize}
\begin{itemize}
\item {Proveniência:(Do lat. \textunderscore icterus\textunderscore )}
\end{itemize}
Enfermidade, caracterizada por certa amarelidão na pelle e nas escleróticas, e produzida pela mistura da parte còrante da bílis com o sangue.
\section{Ictérico}
\begin{itemize}
\item {Grp. gram.:adj.}
\end{itemize}
\begin{itemize}
\item {Grp. gram.:M.}
\end{itemize}
\begin{itemize}
\item {Proveniência:(Lat. \textunderscore ictericus\textunderscore )}
\end{itemize}
Que padece icterícia.
O doente de icterícia.
\section{Icterocéfalo}
\begin{itemize}
\item {Grp. gram.:adj.}
\end{itemize}
\begin{itemize}
\item {Utilização:Zool.}
\end{itemize}
\begin{itemize}
\item {Proveniência:(Do gr. \textunderscore ikteros\textunderscore  + \textunderscore kephale\textunderscore )}
\end{itemize}
Que tem cabeça amarela.
\section{Icterocéphalo}
\begin{itemize}
\item {Grp. gram.:adj.}
\end{itemize}
\begin{itemize}
\item {Utilização:Zool.}
\end{itemize}
\begin{itemize}
\item {Proveniência:(Do gr. \textunderscore ikteros\textunderscore  + \textunderscore kephale\textunderscore )}
\end{itemize}
Que tem cabeça amarela.
\section{Icterode}
\begin{itemize}
\item {Grp. gram.:adj.}
\end{itemize}
\begin{itemize}
\item {Proveniência:(Gr. \textunderscore icterodes\textunderscore )}
\end{itemize}
Diz-se de uma febre ou typho, que, segundo alguns, se confunde com a febre amarela.
\section{Icteroide}
\begin{itemize}
\item {Grp. gram.:m.}
\end{itemize}
\begin{itemize}
\item {Proveniência:(Do gr. \textunderscore ikteros\textunderscore  + \textunderscore eidos\textunderscore )}
\end{itemize}
Micróbio da febre amarela, segundo as recentes communicações do Dr. Stannarelli, de Montevideu.
\section{Icterópode}
\begin{itemize}
\item {Grp. gram.:adj.}
\end{itemize}
\begin{itemize}
\item {Utilização:Zool.}
\end{itemize}
\begin{itemize}
\item {Proveniência:(Do gr. \textunderscore ikteros\textunderscore  + \textunderscore pous\textunderscore )}
\end{itemize}
Que tem patas amarelas.
\section{Ictínia}
\begin{itemize}
\item {Grp. gram.:f.}
\end{itemize}
\begin{itemize}
\item {Proveniência:(Do gr. \textunderscore iktinos\textunderscore )}
\end{itemize}
Gênero de aves falconídeas.
\section{Ictiofagia}
\begin{itemize}
\item {Grp. gram.:f.}
\end{itemize}
Hábito de se alimentar de peixes.
(Cp. \textunderscore ictiófago\textunderscore )
\section{Ictiofágico}
\begin{itemize}
\item {Grp. gram.:adj.}
\end{itemize}
Relativo á ictofagia.
\section{Ictiófago}
\begin{itemize}
\item {Grp. gram.:adj.}
\end{itemize}
\begin{itemize}
\item {Grp. gram.:M.}
\end{itemize}
\begin{itemize}
\item {Proveniência:(Do gr. \textunderscore ikhthus\textunderscore  + \textunderscore phagein\textunderscore )}
\end{itemize}
Relativo á ictiofagia.
Aquele que se alimenta de peixes.
\section{Ictiológico}
\begin{itemize}
\item {Grp. gram.:adj.}
\end{itemize}
Relativo á ictiologia.
\section{Ictiólogo}
\begin{itemize}
\item {Grp. gram.:m.}
\end{itemize}
\begin{itemize}
\item {Proveniência:(Do gr. \textunderscore ikhthus\textunderscore  + \textunderscore logos\textunderscore )}
\end{itemize}
Naturalista, que trata de ictiologia.
\section{Ictiopsofose}
\begin{itemize}
\item {Grp. gram.:f.}
\end{itemize}
\begin{itemize}
\item {Proveniência:(Do gr. \textunderscore ikhthus\textunderscore  + \textunderscore psophos\textunderscore )}
\end{itemize}
Rumor, produzido pelos peixes debaixo da água, o qual parece devido á vibração dos músculos da vesícula pulmonar.
\section{Ictiose}
\begin{itemize}
\item {Grp. gram.:f.}
\end{itemize}
\begin{itemize}
\item {Proveniência:(Do gr. \textunderscore ikhthus\textunderscore )}
\end{itemize}
Doença cutânea, caracterizada por escamas.
\section{Ictiospôndilo}
\begin{itemize}
\item {Grp. gram.:m.}
\end{itemize}
\begin{itemize}
\item {Proveniência:(Do gr. \textunderscore ikhthus\textunderscore  + \textunderscore spondulos\textunderscore )}
\end{itemize}
Vértebra fóssil de peixe.
\section{Ictiossáurio}
\begin{itemize}
\item {Grp. gram.:m.}
\end{itemize}
\begin{itemize}
\item {Proveniência:(De \textunderscore ictio...\textunderscore  + \textunderscore sáurio\textunderscore )}
\end{itemize}
Reptil marinho, hoje fóssil, pertencente ao segundo período geológico.
\section{Icto}
\begin{itemize}
\item {Grp. gram.:m.}
\end{itemize}
\begin{itemize}
\item {Utilização:Philol.}
\end{itemize}
\begin{itemize}
\item {Proveniência:(Lat. \textunderscore ictus\textunderscore )}
\end{itemize}
A maior energia de expiração de uma sýllaba, com relação ás demais do vocábulo ou da phrase.
Accento tónico.
\section{Íctus}
\begin{itemize}
\item {Grp. gram.:m.}
\end{itemize}
\begin{itemize}
\item {Utilização:Des.}
\end{itemize}
O mesmo que \textunderscore icto\textunderscore .
\section{Icum-mucungo}
\begin{itemize}
\item {Grp. gram.:m.}
\end{itemize}
Arbusto africano, annual, de fôlhas pubescentes e flôres miúdas, branco-roxeadas.
\section{Ida}
\begin{itemize}
\item {Grp. gram.:f.}
\end{itemize}
\begin{itemize}
\item {Proveniência:(Do lat. \textunderscore ilus\textunderscore )}
\end{itemize}
Acto de ir.
Jornada; partida.
Série fiada: \textunderscore desde o tanque até ao silvado há uma ida de pedras\textunderscore .
\section{Idade}
\begin{itemize}
\item {Grp. gram.:f.}
\end{itemize}
\begin{itemize}
\item {Grp. gram.:Loc. adv.}
\end{itemize}
\begin{itemize}
\item {Proveniência:(Do lat. \textunderscore aetas\textunderscore , \textunderscore aetatis\textunderscore )}
\end{itemize}
Duração ordinária da vida.
Número de annos de alguém: \textunderscore que idade tens tu\textunderscore ?
Época da vida, com referência a certos fins ou actos.
Vida.
Época histórica: \textunderscore a Idade-Média\textunderscore .
Espaço considerável de tempo, durante o qual se realizam factos de natureza connexa.
Cada um dos graus, mais ou menos distintos que a vida humana atravessa: \textunderscore a idade das paixões\textunderscore .
Velhice: \textunderscore um homem já de idade\textunderscore .
Tempo.
\textunderscore Sôbre idade\textunderscore , em idade avançada; na velhice. Cf. Camillo, \textunderscore Sc. Innocentes\textunderscore , 106.
\section{Idálico}
\begin{itemize}
\item {Grp. gram.:adj.}
\end{itemize}
O mesmo que \textunderscore idálio\textunderscore .
\section{Idálio}
\begin{itemize}
\item {Grp. gram.:adj.}
\end{itemize}
\begin{itemize}
\item {Proveniência:(Lat. \textunderscore idalius\textunderscore )}
\end{itemize}
Relativo ao monte Idálio, consagrado a Vênus, em Chire.
\section{Idéa}
\begin{itemize}
\item {Grp. gram.:f.}
\end{itemize}
\begin{itemize}
\item {Proveniência:(Lat. \textunderscore idea\textunderscore )}
\end{itemize}
Representação, feita no espírito, de uma coisa que existe fóra ou longe delle ou que é puramente intellectual.
Concepção intellectual.
Typo eterno das coisas.
Engenho.
Imagem: \textunderscore aquelle patife dá idéa de um monstro\textunderscore .
Imaginação.
Opinião, juizo: \textunderscore formar má idéa de alguém\textunderscore .
Conhecimento.
Lembrança: \textunderscore não tenho idéa dêsse caso\textunderscore .
Systema.
Projecto; invenção.
\section{Ideação}
\begin{itemize}
\item {Grp. gram.:f.}
\end{itemize}
Acto ou effeito de idear.
\section{Ideal}
\begin{itemize}
\item {Grp. gram.:adj.}
\end{itemize}
\begin{itemize}
\item {Grp. gram.:M.}
\end{itemize}
\begin{itemize}
\item {Proveniência:(Lat. \textunderscore idealis\textunderscore )}
\end{itemize}
Que só existe na ideia; imaginário: \textunderscore venturas ideaes\textunderscore .
Em que há toda a perfeição que se póde conceber: \textunderscore mulher ideal\textunderscore .
Reunião abstracta de perfeições imaginárias ou que não podem têr realização completa.
A mais elevada e mais viva aspiração: \textunderscore sacrificar-se a um ideal\textunderscore .
\section{Idealidade}
\begin{itemize}
\item {Grp. gram.:f.}
\end{itemize}
Qualidade daquillo que é ideal.
Fantasia; imaginação; devaneio.
\section{Idealismo}
\begin{itemize}
\item {Grp. gram.:m.}
\end{itemize}
\begin{itemize}
\item {Proveniência:(De \textunderscore ideal\textunderscore )}
\end{itemize}
Doutrina philosóphica, em que a ideia é o princípio do conhecimento, ou do conhecimento e do sêr.
Systema mýstico dos que consideram o céu como real, e o mundo como apparente.
Devaneio.
Tendência para o ideal.
\section{Idealista}
\begin{itemize}
\item {Grp. gram.:adj.}
\end{itemize}
\begin{itemize}
\item {Grp. gram.:M.}
\end{itemize}
\begin{itemize}
\item {Proveniência:(De \textunderscore ideal\textunderscore )}
\end{itemize}
Relativo ao idealismo.
Partidário do idealismo.
Fantasiador.
Aquelle que devaneia.
\section{Idealístico}
\begin{itemize}
\item {Grp. gram.:adj.}
\end{itemize}
\begin{itemize}
\item {Proveniência:(De \textunderscore idealista\textunderscore )}
\end{itemize}
Relativo ao idealismo.
\section{Idealização}
\begin{itemize}
\item {Grp. gram.:f.}
\end{itemize}
Acto ou effeito de idealizar.
\section{Idealizador}
\begin{itemize}
\item {Grp. gram.:adj.}
\end{itemize}
Que idealiza.
\section{Idealizar}
\begin{itemize}
\item {Grp. gram.:v. t.}
\end{itemize}
Dar carácter ideal a.
Divinizar.
Fantasiar, imaginar, criar na imaginação.
\section{Idealmente}
\begin{itemize}
\item {Grp. gram.:adv.}
\end{itemize}
De modo ideal; imaginariamente.
\section{Idear}
\begin{itemize}
\item {Grp. gram.:v. t.}
\end{itemize}
Criar na ideia; fantasiar.
Planear; projectar.
\section{Ideável}
\begin{itemize}
\item {Grp. gram.:adj.}
\end{itemize}
Que se póde idear.
\section{Ideia}
\begin{itemize}
\item {Grp. gram.:f.}
\end{itemize}
\begin{itemize}
\item {Proveniência:(Lat. \textunderscore idea\textunderscore )}
\end{itemize}
Representação, feita no espírito, de uma coisa que existe fóra ou longe delle ou que é puramente intellectual.
Concepção intellectual.
Typo eterno das coisas.
Engenho.
Imagem: \textunderscore aquelle patife dá ideia de um monstro\textunderscore .
Imaginação.
Opinião, juizo: \textunderscore formar má ideia de alguém\textunderscore .
Conhecimento.
Lembrança: \textunderscore não tenho ideia dêsse caso\textunderscore .
Systema.
Projecto; invenção.
\section{Identicamente}
\begin{itemize}
\item {Grp. gram.:adv.}
\end{itemize}
De modo idêntico; semelhantemente.
\section{Idêntico}
\begin{itemize}
\item {Grp. gram.:adj.}
\end{itemize}
\begin{itemize}
\item {Proveniência:(Do rad. de \textunderscore identidade\textunderscore )}
\end{itemize}
Que é o mesmo que outro ou outros.
Perfeitamente igual.
Consubstanciado.
Análogo.
\section{Identidade}
\begin{itemize}
\item {Grp. gram.:f.}
\end{itemize}
\begin{itemize}
\item {Proveniência:(Lat. \textunderscore identitas\textunderscore )}
\end{itemize}
Qualidade daquillo que é idêntico.
Qualidade de uma coisa, que é o mesmo que outra.
Qualidade de duas ou mais coisas que fazem uma só.
Circunstância de que um indivíduo é o mesmo que se pretende ou que se presume sêr.
Circunstância de que um cadáver ou um esqueleto é o de determinado indivíduo.
Equação algébrica, em que os dois membros são identicamente os mesmos.
Consciência de si próprio.
\section{Identificação}
\begin{itemize}
\item {Grp. gram.:f.}
\end{itemize}
Acto ou effeito de identificar.
\section{Identificar}
\begin{itemize}
\item {Grp. gram.:v. t.}
\end{itemize}
\begin{itemize}
\item {Grp. gram.:V. p.}
\end{itemize}
\begin{itemize}
\item {Proveniência:(De \textunderscore idêntico\textunderscore  + lat. \textunderscore facere\textunderscore )}
\end{itemize}
Tornar idêntico.
Reconhecer como idêntico: \textunderscore identificar um cadáver\textunderscore .
Confundir o que é seu com o alheio.
Compenetrar-se do que outrem sente ou pensa; conformar-se.
\section{Ideogenia}
\begin{itemize}
\item {Grp. gram.:f.}
\end{itemize}
\begin{itemize}
\item {Proveniência:(Do gr. \textunderscore idea\textunderscore  + \textunderscore genea\textunderscore )}
\end{itemize}
Sciência, que se occupa da origem das ideias.
\section{Ideogênico}
\begin{itemize}
\item {Grp. gram.:adj.}
\end{itemize}
Relativo á ideogenia.
\section{Ideografia}
\begin{itemize}
\item {Grp. gram.:f.}
\end{itemize}
Representação das ideias por sinaes, que são a imagem figurada do objecto.
(Cp. \textunderscore ideógrafo\textunderscore )
\section{Ideográfico}
\begin{itemize}
\item {Grp. gram.:adj.}
\end{itemize}
Relativo á ideografia.
\section{Ideografismo}
\begin{itemize}
\item {Grp. gram.:m.}
\end{itemize}
\begin{itemize}
\item {Proveniência:(De \textunderscore ideógrafo\textunderscore )}
\end{itemize}
Aplicação do sistema ideográfico.
\section{Ideógrafo}
\begin{itemize}
\item {Grp. gram.:m.}
\end{itemize}
\begin{itemize}
\item {Proveniência:(Do gr. \textunderscore idea\textunderscore  + \textunderscore graphein\textunderscore )}
\end{itemize}
Aquele que se ocupa de ideografia.
\section{Ideograma}
\begin{itemize}
\item {Grp. gram.:m.}
\end{itemize}
\begin{itemize}
\item {Proveniência:(Do gr. \textunderscore idea\textunderscore  + \textunderscore gramma\textunderscore )}
\end{itemize}
Sinal, que não exprime letra ou som, mas directamente uma ideia, como os algarismos.
\section{Ideogramma}
\begin{itemize}
\item {Grp. gram.:m.}
\end{itemize}
\begin{itemize}
\item {Proveniência:(Do gr. \textunderscore idea\textunderscore  + \textunderscore gramma\textunderscore )}
\end{itemize}
Sinal, que não exprime letra ou som, mas directamente uma ideia, como os algarismos.
\section{Ideographia}
\begin{itemize}
\item {Grp. gram.:f.}
\end{itemize}
Representação das ideias por sinaes, que são a imagem figurada do objecto.
(Cp. \textunderscore ideógrapho\textunderscore )
\section{Ideográphico}
\begin{itemize}
\item {Grp. gram.:adj.}
\end{itemize}
Relativo á ideographia.
\section{Ideographismo}
\begin{itemize}
\item {Grp. gram.:m.}
\end{itemize}
\begin{itemize}
\item {Proveniência:(De \textunderscore ideógrapho\textunderscore )}
\end{itemize}
Applicação do systema ideográphico.
\section{Ideógrapho}
\begin{itemize}
\item {Grp. gram.:m.}
\end{itemize}
\begin{itemize}
\item {Proveniência:(Do gr. \textunderscore idea\textunderscore  + \textunderscore graphein\textunderscore )}
\end{itemize}
Aquelle que se occupa de ideographia.
\section{Ideologia}
\begin{itemize}
\item {Grp. gram.:f.}
\end{itemize}
Sciência das ideias, consideradas em si mesmas.
Sciência da formação das ideias.
Systema philosóphico, em que a sensação é a única origem dos nossos conhecimentos.
(Cp. \textunderscore ideólogo\textunderscore )
\section{Ideológico}
\begin{itemize}
\item {Grp. gram.:adj.}
\end{itemize}
Relativo á ideologia.
\section{Ideólogo}
\begin{itemize}
\item {Grp. gram.:m.}
\end{itemize}
\begin{itemize}
\item {Utilização:Fig.}
\end{itemize}
\begin{itemize}
\item {Proveniência:(Do gr. \textunderscore idea\textunderscore  + \textunderscore logos\textunderscore )}
\end{itemize}
Aquelle que é versado em ideologia.
Idealista; fantasista; devaneador.
\section{Idésia}
\begin{itemize}
\item {Grp. gram.:f.}
\end{itemize}
Gênero de plantas flacurtiáceas.
\section{Idi}
\begin{itemize}
\item {Grp. gram.:m.}
\end{itemize}
Sacerdote gentílico da Ásia.
\section{Idioelectricidade}
\begin{itemize}
\item {Grp. gram.:f.}
\end{itemize}
Qualidade daquillo que é idioeléctrico.
\section{Idioeléctrico}
\begin{itemize}
\item {Grp. gram.:adj.}
\end{itemize}
\begin{itemize}
\item {Proveniência:(De \textunderscore idios\textunderscore  gr. + \textunderscore eléctrico\textunderscore )}
\end{itemize}
Que póde adquirir propriedades eléctricas por meio de fricção.
\section{Idiógino}
\begin{itemize}
\item {Grp. gram.:adj.}
\end{itemize}
\begin{itemize}
\item {Utilização:Bot.}
\end{itemize}
\begin{itemize}
\item {Proveniência:(Do gr. \textunderscore idios\textunderscore  + \textunderscore gune\textunderscore )}
\end{itemize}
Diz-se das plantas, em que os estames não estão reunidos com o pistilo na mesma flôr.
\section{Idiógyno}
\begin{itemize}
\item {Grp. gram.:adj.}
\end{itemize}
\begin{itemize}
\item {Utilização:Bot.}
\end{itemize}
\begin{itemize}
\item {Proveniência:(Do gr. \textunderscore idios\textunderscore  + \textunderscore gune\textunderscore )}
\end{itemize}
Diz-se das plantas, em que os estames não estão reunidos com o pistillo na mesma flôr.
\section{Idiólatra}
\begin{itemize}
\item {Grp. gram.:m.}
\end{itemize}
Aquelle que se adora a si próprio.
(Cp. \textunderscore idiolatria\textunderscore )
\section{Idiolatria}
\begin{itemize}
\item {Grp. gram.:f.}
\end{itemize}
\begin{itemize}
\item {Proveniência:(Do gr. \textunderscore idios\textunderscore  + \textunderscore latreia\textunderscore )}
\end{itemize}
Adoração de si próprio.
\section{Idioma}
\begin{itemize}
\item {Grp. gram.:m.}
\end{itemize}
\begin{itemize}
\item {Utilização:Ext.}
\end{itemize}
\begin{itemize}
\item {Proveniência:(Lat. \textunderscore idioma\textunderscore )}
\end{itemize}
Língua de um povo, considerada nos seus caracteres especiaes.
Expressão.
\section{Idiomático}
\begin{itemize}
\item {Grp. gram.:adj.}
\end{itemize}
\begin{itemize}
\item {Proveniência:(Gr. \textunderscore idiomatikos\textunderscore )}
\end{itemize}
Relativo a idioma.
\section{Idiometálico}
\begin{itemize}
\item {Grp. gram.:adj.}
\end{itemize}
\begin{itemize}
\item {Proveniência:(De \textunderscore ideos\textunderscore , gr. + \textunderscore metálico\textunderscore )}
\end{itemize}
Diz-se dos fenómenos eléctricos, que se revelam pelo contacto de dois metaes.
\section{Idiometállico}
\begin{itemize}
\item {Grp. gram.:adj.}
\end{itemize}
\begin{itemize}
\item {Proveniência:(De \textunderscore ideos\textunderscore , gr. + \textunderscore metállico\textunderscore )}
\end{itemize}
Diz-se dos phenómenos eléctricos, que se revelam pelo contacto de dois metaes.
\section{Idiomográfico}
\begin{itemize}
\item {Grp. gram.:adj.}
\end{itemize}
Relativo á idiomografia.
\section{Idiomographia}
\begin{itemize}
\item {Grp. gram.:f.}
\end{itemize}
\begin{itemize}
\item {Proveniência:(Do gr. \textunderscore idioma\textunderscore  + \textunderscore graphein\textunderscore )}
\end{itemize}
Sciência, que trata da descripção e classificação dos idiomas.
\section{Idiomográphico}
\begin{itemize}
\item {Grp. gram.:adj.}
\end{itemize}
Relativo á idiomographia.
\section{Idiomorfo}
\begin{itemize}
\item {Grp. gram.:m.  e  adj.}
\end{itemize}
\begin{itemize}
\item {Proveniência:(Do gr. \textunderscore idios\textunderscore  + \textunderscore morphe\textunderscore )}
\end{itemize}
Fóssil, proveniente de animaes ou vegetaes.
\section{Idiomorpho}
\begin{itemize}
\item {Grp. gram.:m.  e  adj.}
\end{itemize}
\begin{itemize}
\item {Proveniência:(Do gr. \textunderscore idios\textunderscore  + \textunderscore morphe\textunderscore )}
\end{itemize}
Fóssil, proveniente de animaes ou vegetaes.
\section{Idionomia}
\begin{itemize}
\item {Grp. gram.:f.}
\end{itemize}
\begin{itemize}
\item {Utilização:Neol.}
\end{itemize}
\begin{itemize}
\item {Proveniência:(Do gr. \textunderscore idios\textunderscore  + \textunderscore nomos\textunderscore )}
\end{itemize}
Estado daquillo que é dirigido por leis próprias ou privativas. Cf. Beviláqua, \textunderscore Direito de Fam.\textunderscore 
\section{Idiopathia}
\begin{itemize}
\item {Grp. gram.:f.}
\end{itemize}
\begin{itemize}
\item {Proveniência:(Do gr. \textunderscore idios\textunderscore  + \textunderscore pathos\textunderscore )}
\end{itemize}
Enfermidade, que existe de per si, ou que não depende de outra affecção.
Propensão especial, predilecção.
\section{Idiopáthico}
\begin{itemize}
\item {Grp. gram.:adj.}
\end{itemize}
Relativo á idiopathia.
\section{Idiopatia}
\begin{itemize}
\item {Grp. gram.:f.}
\end{itemize}
\begin{itemize}
\item {Proveniência:(Do gr. \textunderscore idios\textunderscore  + \textunderscore pathos\textunderscore )}
\end{itemize}
Enfermidade, que existe de per si, ou que não depende de outra afecção.
Propensão especial, predilecção.
\section{Idiopático}
\begin{itemize}
\item {Grp. gram.:adj.}
\end{itemize}
Relativo á idiopatia.
\section{Idioscópico}
\begin{itemize}
\item {Grp. gram.:adj.}
\end{itemize}
\begin{itemize}
\item {Proveniência:(Do gr. \textunderscore idios\textunderscore  + \textunderscore skopein\textunderscore )}
\end{itemize}
Relativo ás propriedades, que pertencem particularmente a certos seres.
\section{Idiosincrasia}
\begin{itemize}
\item {Grp. gram.:f.}
\end{itemize}
\begin{itemize}
\item {Proveniência:(Do gr. \textunderscore idios\textunderscore  + \textunderscore sunkrasis\textunderscore )}
\end{itemize}
Disposição ou temperamento, que faz que um indivíduo sinta de uma fórma especial e privativa dele a influência de diversos agentes.
\section{Idiosyncrasia}
\begin{itemize}
\item {Grp. gram.:f.}
\end{itemize}
\begin{itemize}
\item {Proveniência:(Do gr. \textunderscore idios\textunderscore  + \textunderscore sunkrasis\textunderscore )}
\end{itemize}
Disposição ou temperamento, que faz que um indivíduo sinta de uma fórma especial e privativa delle a influência de diversos agentes.
\section{Idiosyncrásico}
\begin{itemize}
\item {Grp. gram.:adj.}
\end{itemize}
Relativo á idiosyncrasia.
\section{Idiota}
\begin{itemize}
\item {Grp. gram.:m.  e  adj.}
\end{itemize}
\begin{itemize}
\item {Grp. gram.:M.}
\end{itemize}
\begin{itemize}
\item {Utilização:Ant.}
\end{itemize}
\begin{itemize}
\item {Proveniência:(Lat. \textunderscore idiota\textunderscore )}
\end{itemize}
O que não tem instrucção.
Que não é intelligente; pateta; parvo.
Indivíduo, que exerce uma sciência ou profissão, sem diploma para êsse effeito; curioso. Cf. B. Pereira, \textunderscore Prosódia\textunderscore , vb. \textunderscore idiota\textunderscore ; G. Barreiros, \textunderscore Corogr.\textunderscore , 1.^a ed., 60, 64 e 191; \textunderscore Luz e Calor\textunderscore , etc.
\section{Idiotez}
\begin{itemize}
\item {Grp. gram.:f.}
\end{itemize}
Estado de idiota, idiotismo.
\section{Idiotia}
\begin{itemize}
\item {Grp. gram.:f.}
\end{itemize}
(V.idiotismo)
\section{Idiótico}
\begin{itemize}
\item {Grp. gram.:adj.}
\end{itemize}
\begin{itemize}
\item {Proveniência:(Lat. \textunderscore idioticus\textunderscore )}
\end{itemize}
Relativo a idiota ou a idiotismo.
\section{Idiotismo}
\begin{itemize}
\item {Grp. gram.:m.}
\end{itemize}
\begin{itemize}
\item {Utilização:Philol.}
\end{itemize}
\begin{itemize}
\item {Proveniência:(Lat. \textunderscore idiotismus\textunderscore )}
\end{itemize}
Estado de quem é idiota.
Locução ou construcção própria de uma língua, e ordináriamente familiar ou vulgar.
\section{Idílico}
\begin{itemize}
\item {Grp. gram.:adj.}
\end{itemize}
Relativo a idílio.
Suavemente amoroso.
\section{Idílio}
\begin{itemize}
\item {Grp. gram.:m.}
\end{itemize}
\begin{itemize}
\item {Utilização:Fig.}
\end{itemize}
\begin{itemize}
\item {Proveniência:(Lat. \textunderscore idyllium\textunderscore )}
\end{itemize}
Pequena composição poética, ordinariamente campestre ou pastoril.
Diversão bucólica.
Amor suave e tranquilo.
Devaneio; fantasia.
\section{Idilista}
\begin{itemize}
\item {Grp. gram.:m.}
\end{itemize}
Aquele que faz idílios.
Fantasista; devaneador.
\section{Idiotizar}
\begin{itemize}
\item {Grp. gram.:v. t.}
\end{itemize}
Tornar idiota.
\section{Idoiro}
\begin{itemize}
\item {Grp. gram.:m.  e  adj.}
\end{itemize}
\begin{itemize}
\item {Utilização:Ant.}
\end{itemize}
\begin{itemize}
\item {Proveniência:(Do lat. \textunderscore iturus\textunderscore )}
\end{itemize}
O que há de ir ou desapparecer. Cf. Frei Fortun., \textunderscore Inéd.\textunderscore , I, 308.
\section{Ídola}
\begin{itemize}
\item {Grp. gram.:f.}
\end{itemize}
\begin{itemize}
\item {Utilização:Ant.}
\end{itemize}
\begin{itemize}
\item {Proveniência:(De \textunderscore ídolo\textunderscore )}
\end{itemize}
Mulher amada, idolatrada.
Estátua de divindade feminina:«\textunderscore essa ídola come meninos\textunderscore ». \textunderscore Eufrosina\textunderscore , act. I, sc. 1.
\section{Idólatra}
\begin{itemize}
\item {Grp. gram.:adj.}
\end{itemize}
\begin{itemize}
\item {Grp. gram.:M.}
\end{itemize}
Que adora os ídolos.
Relativo á idolatria.
Que presta culto divino a criaturas.
Apaixonado.
Aquelle que adora ídolos.--Nos \textunderscore Lusíadas\textunderscore , lê-se \textunderscore idolátra\textunderscore . Exigênio da metrificação?
(Contr. de \textunderscore idolólatra\textunderscore )
\section{Idolatradamente}
\begin{itemize}
\item {Grp. gram.:adv.}
\end{itemize}
\begin{itemize}
\item {Proveniência:(De \textunderscore idolatrar\textunderscore )}
\end{itemize}
Com idolatria.
\section{Idolatrar}
\begin{itemize}
\item {Grp. gram.:v. t.}
\end{itemize}
\begin{itemize}
\item {Grp. gram.:V. i.}
\end{itemize}
\begin{itemize}
\item {Proveniência:(De \textunderscore idólatra\textunderscore )}
\end{itemize}
Tributar idolatria a; amar cegamente.
Adorar ídolos. Cf. Rui Barb., \textunderscore Réplica\textunderscore , 160.
\section{Idólatre}
\begin{itemize}
\item {Grp. gram.:m.}
\end{itemize}
\begin{itemize}
\item {Utilização:Des.}
\end{itemize}
O mesmo que \textunderscore idólatra\textunderscore . Cf. Usque, 117, v.^o
\section{Idolatria}
\begin{itemize}
\item {Grp. gram.:f.}
\end{itemize}
\begin{itemize}
\item {Proveniência:(Lat. \textunderscore idolatria\textunderscore )}
\end{itemize}
Adoração dos ídolos.
Acto de prestar culto divino a criaturas.
Amor excessivo, apaixonado.
\section{Idolatricamente}
\begin{itemize}
\item {Grp. gram.:adv.}
\end{itemize}
De modo idolátrico; como idólatra:«\textunderscore amar idolatricamente.\textunderscore »Camillo, \textunderscore Judeu\textunderscore , XXXV.
\section{Idolátrico}
\begin{itemize}
\item {Grp. gram.:adj.}
\end{itemize}
Relativo á idolatria.
\section{Idolatrizar}
\begin{itemize}
\item {Grp. gram.:v. t.}
\end{itemize}
Tornar idólatra, ou idolatricamente apaixonado.
\section{Ídolo}
\begin{itemize}
\item {Grp. gram.:m.}
\end{itemize}
\begin{itemize}
\item {Utilização:Fig.}
\end{itemize}
\begin{itemize}
\item {Proveniência:(Lat. \textunderscore idolum\textunderscore )}
\end{itemize}
Figura, que, representando uma divindade, é objecto de culto.
Pessôa, a quem se tributa extraordinário respeito ou excessivo affecto.
\section{Idololatra}
\begin{itemize}
\item {Grp. gram.:m.}
\end{itemize}
\begin{itemize}
\item {Utilização:Des.}
\end{itemize}
\begin{itemize}
\item {Proveniência:(Lat. \textunderscore idololatra\textunderscore )}
\end{itemize}
O mesmo que \textunderscore idólatra\textunderscore :«\textunderscore levando o idololatra e o Mouro preso.\textunderscore »\textunderscore Lusiadas\textunderscore , II, 54.
\section{Idolólatra}
\begin{itemize}
\item {Grp. gram.:m.}
\end{itemize}
\begin{itemize}
\item {Utilização:Des.}
\end{itemize}
\begin{itemize}
\item {Proveniência:(Lat. \textunderscore idololatra\textunderscore )}
\end{itemize}
O mesmo que \textunderscore idólatra\textunderscore :«\textunderscore levando o idolólatra e o Mouro preso.\textunderscore »\textunderscore Lusiadas\textunderscore , II, 54.
\section{Idolopeia}
\begin{itemize}
\item {Grp. gram.:f.}
\end{itemize}
\begin{itemize}
\item {Proveniência:(Do gr. \textunderscore eidolon\textunderscore  + \textunderscore ops\textunderscore )}
\end{itemize}
Figura de pensamento, com que no discurso se introduzem falando falsas divindades ou pessôas fallecidas.
\section{Idoneamente}
\begin{itemize}
\item {Grp. gram.:adv.}
\end{itemize}
De modo idóneo.
\section{Idoneidade}
\begin{itemize}
\item {Grp. gram.:f.}
\end{itemize}
\begin{itemize}
\item {Proveniência:(Lat. \textunderscore idoneitas\textunderscore )}
\end{itemize}
Qualidade daquelle ou daquillo que é idóneo.
\section{Idóneo}
\begin{itemize}
\item {Grp. gram.:adj.}
\end{itemize}
\begin{itemize}
\item {Proveniência:(Lat. \textunderscore idoneus\textunderscore )}
\end{itemize}
Apropriado.
Conveniente.
Apto.
Que tem condições para bem desempenhar certos cargos.
\section{Idos}
\begin{itemize}
\item {Grp. gram.:m. pl.}
\end{itemize}
\begin{itemize}
\item {Proveniência:(Lat. \textunderscore idus\textunderscore )}
\end{itemize}
O dia 15 de Março, Maio, Julho e Outubro, e o dia 13 dos outros meses, no antigo calendário romano.
\section{Idoscópico}
\begin{itemize}
\item {Grp. gram.:adj.}
\end{itemize}
\begin{itemize}
\item {Utilização:Zool.}
\end{itemize}
\begin{itemize}
\item {Proveniência:(Do gr. \textunderscore eidos\textunderscore  + \textunderscore skopein\textunderscore )}
\end{itemize}
Diz-se dos olhos dos invertebrados, em que se reflectem as imagens.
\section{Idoso}
\begin{itemize}
\item {Grp. gram.:adj.}
\end{itemize}
\begin{itemize}
\item {Proveniência:(De \textunderscore idade\textunderscore , ou por \textunderscore dioso\textunderscore , de dia)}
\end{itemize}
Velho; que tem muita idade: \textunderscore homem idoso\textunderscore .
\section{Idouro}
\begin{itemize}
\item {Grp. gram.:m.  e  adj.}
\end{itemize}
\begin{itemize}
\item {Utilização:Ant.}
\end{itemize}
\begin{itemize}
\item {Proveniência:(Do lat. \textunderscore iturus\textunderscore )}
\end{itemize}
O que há de ir ou desapparecer. Cf. Frei Fortun., \textunderscore Inéd.\textunderscore , I, 308.
\section{Idrialina}
\begin{itemize}
\item {Grp. gram.:f.}
\end{itemize}
Substância, que se extrai das minas mercuriaes de Ídria.
\section{Idrol}
\begin{itemize}
\item {Grp. gram.:m.}
\end{itemize}
Citrato de prata.
\section{Idumeu}
\begin{itemize}
\item {Grp. gram.:adj.}
\end{itemize}
\begin{itemize}
\item {Grp. gram.:M.}
\end{itemize}
Relativo á Idumeia.
Habitante da Idumeia.
\section{Idýllico}
\begin{itemize}
\item {Grp. gram.:adj.}
\end{itemize}
Relativo a idýllio.
Suavemente amoroso.
\section{Idýllio}
\begin{itemize}
\item {Grp. gram.:m.}
\end{itemize}
\begin{itemize}
\item {Utilização:Fig.}
\end{itemize}
\begin{itemize}
\item {Proveniência:(Lat. \textunderscore idyllium\textunderscore )}
\end{itemize}
Pequena composição poética, ordinariamente campestre ou pastoril.
Diversão bucólica.
Amor suave e tranquillo.
Devaneio; fantasia.
\section{Idyllista}
\begin{itemize}
\item {Grp. gram.:m.}
\end{itemize}
Aquelle que faz idýllios.
Fantasista; devaneador.
\section{I. é.}
Abrev.
(de \textunderscore isto é\textunderscore )
\section{Iei}
\begin{itemize}
\item {Grp. gram.:m.}
\end{itemize}
O mesmo que \textunderscore icica\textunderscore .
\section{Ieixa}
\begin{itemize}
\item {Grp. gram.:f.}
\end{itemize}
\begin{itemize}
\item {Utilização:T. de Turquel}
\end{itemize}
Planta herbácea, parecida com a mostarda.
\section{Ienite}
\begin{itemize}
\item {Grp. gram.:f.}
\end{itemize}
\begin{itemize}
\item {Utilização:Miner.}
\end{itemize}
Variedade de pedra dura e escura da Córsega.
\section{Iento}
\begin{itemize}
\item {Grp. gram.:m.}
\end{itemize}
\begin{itemize}
\item {Utilização:Ant.}
\end{itemize}
Herdade cultivada e fructífera.--É do século XIV.
\section{Ieramá}
\begin{itemize}
\item {Grp. gram.:adv.}
\end{itemize}
\begin{itemize}
\item {Utilização:Ant.}
\end{itemize}
Em má hora:«\textunderscore corremos a ieramá.\textunderscore »G. Vicente, \textunderscore Inês Pereira\textunderscore . Cf. \textunderscore Eufrosína\textunderscore .
\section{Ietim}
\begin{itemize}
\item {Grp. gram.:m.}
\end{itemize}
Mosquito do Brasil.
\section{Ifante}
\begin{itemize}
\item {Grp. gram.:m.}
\end{itemize}
\begin{itemize}
\item {Utilização:Ant.}
\end{itemize}
O mesmo que \textunderscore infante\textunderscore ^1. Cf. Fern. Lopes.
\section{Iffante}
\begin{itemize}
\item {Grp. gram.:m.}
\end{itemize}
\begin{itemize}
\item {Utilização:Ant.}
\end{itemize}
O mesmo que \textunderscore infante\textunderscore ^1. Cf. Fern. Lopes.
\section{Ifol}
\begin{itemize}
\item {Grp. gram.:m.}
\end{itemize}
Arvoreta da Índia portuguesa.
\section{Igaçaba}
\begin{itemize}
\item {Grp. gram.:f.}
\end{itemize}
\begin{itemize}
\item {Utilização:Bras}
\end{itemize}
\begin{itemize}
\item {Proveniência:(Do guar. \textunderscore igaçana\textunderscore )}
\end{itemize}
Grande talha para água.
\section{Igapó}
\begin{itemize}
\item {Grp. gram.:m.}
\end{itemize}
\begin{itemize}
\item {Utilização:Bras. do N}
\end{itemize}
\begin{itemize}
\item {Proveniência:(T. tupi)}
\end{itemize}
Pedaço de floresta, invadido por enchente.
Mata, cercada de âgua.
Pântano dentro de mata.
\section{Igara}
\begin{itemize}
\item {Grp. gram.:f.}
\end{itemize}
\begin{itemize}
\item {Utilização:Bras}
\end{itemize}
\begin{itemize}
\item {Proveniência:(Do guar. \textunderscore igara\textunderscore )}
\end{itemize}
Pequena canôa, feita geralmente de um tronco de árvore escavada.
\section{Igarapé}
\begin{itemize}
\item {Grp. gram.:m.}
\end{itemize}
\begin{itemize}
\item {Utilização:Bras}
\end{itemize}
\begin{itemize}
\item {Proveniência:(T. tupi)}
\end{itemize}
Pequeno canal, que apenas dá passagem a igaras ou a outros pequenos barcos.
\section{Igaratim}
\begin{itemize}
\item {Grp. gram.:m.}
\end{itemize}
\begin{itemize}
\item {Utilização:Bras}
\end{itemize}
Canôa, em que embarcavam os chefes índios.
\section{Igarité}
\begin{itemize}
\item {Grp. gram.:m.}
\end{itemize}
\begin{itemize}
\item {Utilização:Bras}
\end{itemize}
Canôa, feita de um só tronco.
Galeota com tolda de madeira.
Cp. \textunderscore igara\textunderscore .
(Do tupi)
\section{Igaruana}
\begin{itemize}
\item {Grp. gram.:f.}
\end{itemize}
\begin{itemize}
\item {Utilização:Bras. do N}
\end{itemize}
Navegante.
Cp. \textunderscore igara\textunderscore .
(Do tupi)
\section{Igarucu}
\begin{itemize}
\item {Grp. gram.:m.}
\end{itemize}
\begin{itemize}
\item {Utilização:Bras}
\end{itemize}
Grande canôa, entre os Tupis.
\section{Igarvana}
\begin{itemize}
\item {Grp. gram.:m.}
\end{itemize}
(Fórma errada, com que alguns diccionaristas escrevem \textunderscore igaruana\textunderscore )
\section{Igasol}
\begin{itemize}
\item {Grp. gram.:m.}
\end{itemize}
Novo medicamento contra a tuberculose.
(Cp. \textunderscore igasúrico\textunderscore )
\section{Igasurato}
\begin{itemize}
\item {Grp. gram.:m.}
\end{itemize}
Combinação do ácido igasúrico com uma base.
(Cp. \textunderscore igasúrico\textunderscore )
\section{Igasúrico}
\begin{itemize}
\item {Grp. gram.:adj.}
\end{itemize}
\begin{itemize}
\item {Proveniência:(Do mal. \textunderscore igasur\textunderscore , fava de Santo-Inácio)}
\end{itemize}
Diz-se de um ácido, que se acha em combinação com a estrychnina em a noz vómica.
\section{Igasurina}
\begin{itemize}
\item {Grp. gram.:f.}
\end{itemize}
Princípio, extrahido da estrychnina.
(Cp. \textunderscore igasúrico\textunderscore )
\section{Ignácia}
\begin{itemize}
\item {Grp. gram.:f.}
\end{itemize}
O mesmo que \textunderscore ignaciana\textunderscore .
\section{Ignaciana}
\begin{itemize}
\item {Grp. gram.:f.}
\end{itemize}
\begin{itemize}
\item {Proveniência:(De \textunderscore Ignácio\textunderscore , ou \textunderscore Inácio\textunderscore , n. p.)}
\end{itemize}
Árvore loganiácea, (\textunderscore ignacia amara\textunderscore ), que produz a chamada \textunderscore fava de Santo-Inácio\textunderscore .
\section{Ignaciano}
\begin{itemize}
\item {Grp. gram.:m.}
\end{itemize}
\begin{itemize}
\item {Utilização:deprec.}
\end{itemize}
\begin{itemize}
\item {Utilização:Ant.}
\end{itemize}
\begin{itemize}
\item {Proveniência:(De \textunderscore Ignácio\textunderscore , ou \textunderscore Inácio\textunderscore , n. p.)}
\end{itemize}
O mesmo que \textunderscore jesuíta\textunderscore .
\section{Ignaramente}
\begin{itemize}
\item {Grp. gram.:adv.}
\end{itemize}
De modo ignaro.
Com froixidão, com indolência.
\section{Ignaro}
\begin{itemize}
\item {Grp. gram.:adj.}
\end{itemize}
\begin{itemize}
\item {Proveniência:(Lat. \textunderscore ignarus\textunderscore )}
\end{itemize}
O mesmo que \textunderscore ignorante\textunderscore ; estúpido; idiota.
\section{Ignávia}
\begin{itemize}
\item {Grp. gram.:f.}
\end{itemize}
\begin{itemize}
\item {Proveniência:(Lat. \textunderscore ignavia\textunderscore )}
\end{itemize}
Qualidade de quem é ignavo.
\section{Ignavo}
\begin{itemize}
\item {Grp. gram.:adj.}
\end{itemize}
\begin{itemize}
\item {Proveniência:(Lat. \textunderscore ignavus\textunderscore )}
\end{itemize}
Indolente; fraco; pusillânime.
\section{Ígneo}
\begin{itemize}
\item {Grp. gram.:adj.}
\end{itemize}
\begin{itemize}
\item {Proveniência:(Lat. \textunderscore igneus\textunderscore )}
\end{itemize}
Relativo ao fogo.
Que é de fogo.
Que tem a côr do fogo.
Que é produzido pela acção do fogo.
\section{Ignescência}
\begin{itemize}
\item {Grp. gram.:f.}
\end{itemize}
Estado daquillo que é ignescente.
\section{Ignescente}
\begin{itemize}
\item {Grp. gram.:adj.}
\end{itemize}
\begin{itemize}
\item {Proveniência:(Lat. \textunderscore ignescens\textunderscore )}
\end{itemize}
Que está ardendo; que está em combustão; que se inflamma.
\section{Ignição}
\begin{itemize}
\item {Grp. gram.:f.}
\end{itemize}
\begin{itemize}
\item {Proveniência:(Do lat. \textunderscore ignitus\textunderscore )}
\end{itemize}
Estado de um corpo ignescente; ignescência.
\section{Ignicola}
\begin{itemize}
\item {Grp. gram.:m.  e  adj.}
\end{itemize}
\begin{itemize}
\item {Proveniência:(Do lat. \textunderscore ignis\textunderscore  + \textunderscore colere\textunderscore )}
\end{itemize}
Aquelle que adora o fogo.
\section{Ignífero}
\begin{itemize}
\item {Grp. gram.:adj.}
\end{itemize}
\begin{itemize}
\item {Proveniência:(Lat. \textunderscore ignifer\textunderscore )}
\end{itemize}
Que traz fogo; em que há fogo.
\section{Ignificação}
\begin{itemize}
\item {Grp. gram.:f.}
\end{itemize}
\begin{itemize}
\item {Proveniência:(Do lat. \textunderscore ignis\textunderscore  + \textunderscore facere\textunderscore )}
\end{itemize}
O mesmo que \textunderscore combustão\textunderscore .
\section{Igniflamante}
\begin{itemize}
\item {Grp. gram.:adj.}
\end{itemize}
\begin{itemize}
\item {Proveniência:(Do lat. \textunderscore ignis\textunderscore  + \textunderscore flamma\textunderscore )}
\end{itemize}
Que brilha como o fogo. Cf. Filinto, IX, 220; XVI, 262.
\section{Igniflammante}
\begin{itemize}
\item {Grp. gram.:adj.}
\end{itemize}
\begin{itemize}
\item {Proveniência:(Do lat. \textunderscore ignis\textunderscore  + \textunderscore flamma\textunderscore )}
\end{itemize}
Que brilha como o fogo. Cf. Filinto, IX, 220; XVI, 262.
\section{Ignifremente}
\begin{itemize}
\item {Grp. gram.:adj.}
\end{itemize}
\begin{itemize}
\item {Proveniência:(Do lat. \textunderscore ignis\textunderscore  + \textunderscore fremens\textunderscore )}
\end{itemize}
Que crepita ou ruge, como o fogo. Cf. Rui Barb., \textunderscore Réplica\textunderscore , II, 157.
\section{Ignífugo}
\begin{itemize}
\item {Grp. gram.:adj.}
\end{itemize}
\begin{itemize}
\item {Proveniência:(Do lat. \textunderscore ignis\textunderscore  + \textunderscore fugere\textunderscore )}
\end{itemize}
Que serve para evitar incêndio; que afugenta o fogo. Cf. \textunderscore Jorn. do Comm.\textunderscore , do Rio, do 6-XI-904.
\section{Ignígeno}
\begin{itemize}
\item {Grp. gram.:adj.}
\end{itemize}
\begin{itemize}
\item {Utilização:Des.}
\end{itemize}
\begin{itemize}
\item {Proveniência:(Lat. \textunderscore ignigenus\textunderscore )}
\end{itemize}
Que produz fogo.
\section{Ignígero}
\begin{itemize}
\item {Grp. gram.:adj.}
\end{itemize}
\begin{itemize}
\item {Proveniência:(Do lat. \textunderscore ignis\textunderscore  + \textunderscore gerere\textunderscore )}
\end{itemize}
O mesmo que \textunderscore ignífero\textunderscore . Cf. Castilho, \textunderscore Metam.\textunderscore , 65.
\section{Ignípede}
\begin{itemize}
\item {Grp. gram.:adj.}
\end{itemize}
\begin{itemize}
\item {Utilização:Poét.}
\end{itemize}
\begin{itemize}
\item {Proveniência:(Lat. \textunderscore ignipes\textunderscore )}
\end{itemize}
Que tem pés de fogo.
Cujos pés, ferindo o solo, produzem fogo, (falando-se de cavallos).
\section{Ignipotente}
\begin{itemize}
\item {Grp. gram.:adj.}
\end{itemize}
\begin{itemize}
\item {Utilização:Poét.}
\end{itemize}
\begin{itemize}
\item {Utilização:Fig.}
\end{itemize}
\begin{itemize}
\item {Proveniência:(Do lat. \textunderscore ignis\textunderscore  + \textunderscore potens\textunderscore )}
\end{itemize}
Senhor do fogo, (falando-se de Vulcano).
Fogoso.
\section{Ignipunctura}
\begin{itemize}
\item {Grp. gram.:f.}
\end{itemize}
\begin{itemize}
\item {Proveniência:(Do lat. \textunderscore ignis\textunderscore  + \textunderscore punctura\textunderscore )}
\end{itemize}
Operação cirúrgica, que consiste em embeber em differentes pontos dos tecidos mórbidos a agulha em brasa de um pequeno cautério.
\section{Ignispício}
\begin{itemize}
\item {Grp. gram.:m.}
\end{itemize}
\begin{itemize}
\item {Proveniência:(Lat. \textunderscore ignispicium\textunderscore )}
\end{itemize}
Supposta arte de adivinhar por meio do fogo; pyromancia.
\section{Ignívago}
\begin{itemize}
\item {Grp. gram.:adj.}
\end{itemize}
\begin{itemize}
\item {Proveniência:(Lat. \textunderscore ignivagus\textunderscore )}
\end{itemize}
Que se espalha ou se propaga, como o fogo.
\section{Ignívomo}
\begin{itemize}
\item {Grp. gram.:adj.}
\end{itemize}
\begin{itemize}
\item {Utilização:Poét.}
\end{itemize}
\begin{itemize}
\item {Proveniência:(Do lat. \textunderscore ignis\textunderscore  + \textunderscore vomere\textunderscore )}
\end{itemize}
Que vomita fogo.
Que expelle chammas.
\section{Ignívoro}
\begin{itemize}
\item {Grp. gram.:adj.}
\end{itemize}
\begin{itemize}
\item {Proveniência:(Do lat. \textunderscore ignis\textunderscore  + \textunderscore vorare\textunderscore )}
\end{itemize}
Que engole ou finge engulir matérias em combustão.
\section{Ignizar-se}
\begin{itemize}
\item {Grp. gram.:v. p.}
\end{itemize}
\begin{itemize}
\item {Utilização:Poét.}
\end{itemize}
\begin{itemize}
\item {Proveniência:(Do lat. \textunderscore ignis\textunderscore )}
\end{itemize}
Converter-se em fogo; inflammar-se.
\section{Ignóbil}
\begin{itemize}
\item {Grp. gram.:adj.}
\end{itemize}
\begin{itemize}
\item {Proveniência:(Lat. \textunderscore ignobilis\textunderscore )}
\end{itemize}
Que não tem nobreza.
Desprezível; vil; baixo: \textunderscore procedimento ignóbil\textunderscore .
\section{Ignóbile}
\begin{itemize}
\item {Grp. gram.:adj.}
\end{itemize}
\begin{itemize}
\item {Utilização:Des.}
\end{itemize}
O mesmo que \textunderscore ignóbil\textunderscore . Cf. Filinto, XIV, 181.
\section{Ignobilidade}
\begin{itemize}
\item {Grp. gram.:f.}
\end{itemize}
\begin{itemize}
\item {Proveniência:(Lat. \textunderscore ignobilitas\textunderscore )}
\end{itemize}
Qualidade daquelle ou daquillo que é ignóbil.
\section{Ignobilmente}
\begin{itemize}
\item {Grp. gram.:adv.}
\end{itemize}
De modo ignóbil.
\section{Ignomínia}
\begin{itemize}
\item {Grp. gram.:f.}
\end{itemize}
\begin{itemize}
\item {Proveniência:(Lat. \textunderscore ignominia\textunderscore )}
\end{itemize}
Grande deshonra; oppróbrio; infâmia.
\section{Ignominiar}
\begin{itemize}
\item {Grp. gram.:v. t.}
\end{itemize}
\begin{itemize}
\item {Proveniência:(Lat. \textunderscore ignominiare\textunderscore )}
\end{itemize}
Infamar.
Cobrir de oppróbrios.
\section{Ignominiosamente}
\begin{itemize}
\item {Grp. gram.:adv.}
\end{itemize}
De modo ignominioso.
\section{Ignominioso}
\begin{itemize}
\item {Grp. gram.:adj.}
\end{itemize}
\begin{itemize}
\item {Proveniência:(Lat. \textunderscore ignominiosus\textunderscore )}
\end{itemize}
Que produz ignomínia; digno de repulsão; infame.
\section{Ignoração}
\begin{itemize}
\item {Grp. gram.:f.}
\end{itemize}
O mesmo que \textunderscore ignorância\textunderscore .
\section{Ignorância}
\begin{itemize}
\item {Grp. gram.:f.}
\end{itemize}
\begin{itemize}
\item {Proveniência:(Lat. \textunderscore ignorantia\textunderscore )}
\end{itemize}
Estado de quem é ignorante.
Falta de saber.
Facto, que denota imperícia.
\section{Ignorantão}
\begin{itemize}
\item {Grp. gram.:m.  e  adj.}
\end{itemize}
\begin{itemize}
\item {Utilização:Pop.}
\end{itemize}
Muito ignorante.
\section{Ignorante}
\begin{itemize}
\item {Grp. gram.:m.  e  adj.}
\end{itemize}
\begin{itemize}
\item {Proveniência:(Lat. \textunderscore ignorans\textunderscore )}
\end{itemize}
O que ignora; que não tem instrucção.
Estúpido.
Analphabeto.
\section{Ignorantemente}
\begin{itemize}
\item {Grp. gram.:adv.}
\end{itemize}
\begin{itemize}
\item {Proveniência:(De \textunderscore ignorante\textunderscore )}
\end{itemize}
Com ignorância.
\section{Ignorantinhos}
\begin{itemize}
\item {Grp. gram.:m. pl.}
\end{itemize}
\begin{itemize}
\item {Proveniência:(De \textunderscore ignorante\textunderscore )}
\end{itemize}
Ordem religiosa, fundada em Portugal por San-João de Deus em 1495, e destinada primeiro ao tratamento dos doentes pobres e, depois, á educação de crianças pobres.
\section{Ignorantismo}
\begin{itemize}
\item {Grp. gram.:m.}
\end{itemize}
\begin{itemize}
\item {Proveniência:(De \textunderscore ignorante\textunderscore )}
\end{itemize}
Systema dos que defendem as vantagens da ignorância, sustentando que a sciência é contrária á moral e ao bem social.
\section{Ignorantista}
\begin{itemize}
\item {Grp. gram.:m.}
\end{itemize}
\begin{itemize}
\item {Utilização:Neol.}
\end{itemize}
\begin{itemize}
\item {Proveniência:(De \textunderscore ignorante\textunderscore )}
\end{itemize}
Partidário do ignorantismo.
\section{Ignorar}
\begin{itemize}
\item {Grp. gram.:v. t.}
\end{itemize}
\begin{itemize}
\item {Utilização:Pop.}
\end{itemize}
\begin{itemize}
\item {Proveniência:(Lat. \textunderscore ignorare\textunderscore )}
\end{itemize}
Não conhecer.
Não saber: \textunderscore ignorar a grammática\textunderscore .
Não têr, (falando-se de qualidades).
Estranhar, fazer reparo em.
\section{Ignoscência}
\begin{itemize}
\item {Grp. gram.:f.}
\end{itemize}
\begin{itemize}
\item {Utilização:Des.}
\end{itemize}
\begin{itemize}
\item {Proveniência:(Lat. \textunderscore ignoscentia\textunderscore )}
\end{itemize}
Remissão de culpa.
Perdão que se concede.
\section{Ignoscente}
\begin{itemize}
\item {Grp. gram.:adj.}
\end{itemize}
\begin{itemize}
\item {Proveniência:(Lat. \textunderscore ignoscens\textunderscore )}
\end{itemize}
Que perdôa.
Indulgente.
\section{Ignotícia}
\begin{itemize}
\item {Grp. gram.:f.}
\end{itemize}
\begin{itemize}
\item {Utilização:Des.}
\end{itemize}
\begin{itemize}
\item {Proveniência:(Lat. \textunderscore ignotitia\textunderscore )}
\end{itemize}
Ignorância de qualquer coisa.
\section{Ignoto}
\begin{itemize}
\item {Grp. gram.:adj.}
\end{itemize}
\begin{itemize}
\item {Proveniência:(Lat. \textunderscore ignotus\textunderscore )}
\end{itemize}
Desconhecido.
Obscuro; humilde.
\section{Igorrotes}
\begin{itemize}
\item {Grp. gram.:m. pl.}
\end{itemize}
Indígenas da província de Abra, nas Filippinas.
\section{Igreja}
\begin{itemize}
\item {Grp. gram.:f.}
\end{itemize}
Communidade dos Christãos.
Catholicismo.
Conjunto dos fiéis, ligados pela mesma fé e sujeitos aos mesmos chefes espirituaes.
Autoridade ecclesiástica.
Estado ecclesiástico.
Clerezia.
Qualquer templo christão.
(Port. ant. \textunderscore eigreja\textunderscore , do lat. \textunderscore ecclesia\textunderscore )
\section{Igrejário}
\begin{itemize}
\item {Grp. gram.:m.}
\end{itemize}
\begin{itemize}
\item {Utilização:Des.}
\end{itemize}
\begin{itemize}
\item {Proveniência:(De \textunderscore igreja\textunderscore )}
\end{itemize}
Ermida, pequena igreja.
Conjunto das igrejas de uma circunscripção ecclesiástica.
\section{Igrejeiro}
\begin{itemize}
\item {Grp. gram.:adj.}
\end{itemize}
\begin{itemize}
\item {Utilização:Pop.}
\end{itemize}
Próprio de igrejas.
Frequentador de igrejas.
Beato; santarrão.
\section{Igrejinha}
\begin{itemize}
\item {Grp. gram.:f.}
\end{itemize}
Pequena igreja.
Cilada; conluio.
\section{Igrejó}
\begin{itemize}
\item {Grp. gram.:m.}
\end{itemize}
\begin{itemize}
\item {Utilização:Ant.}
\end{itemize}
Pequena paróchia; pequena igreja.
\section{Igrejola}
\begin{itemize}
\item {Grp. gram.:f.}
\end{itemize}
O mesmo que \textunderscore igrejó\textunderscore .
\section{Igrejório}
\begin{itemize}
\item {Grp. gram.:m.}
\end{itemize}
O mesmo que \textunderscore igrejó\textunderscore .
\section{Igual}
\begin{itemize}
\item {Grp. gram.:adj.}
\end{itemize}
\begin{itemize}
\item {Grp. gram.:M.  e  f.}
\end{itemize}
\begin{itemize}
\item {Proveniência:(Lat. \textunderscore aequalis\textunderscore )}
\end{itemize}
Que tem a mesma quantidade, qualidade, valor, fórma ou dimensão que outro.
Idêntico.
Liso.
Uniforme, que não soffre alteração.
Pessôa que tem a mesma natureza, o mesmo modo de sêr, o mesmo estado ou categoria: \textunderscore os meus iguaes\textunderscore .
\section{Igualação}
\begin{itemize}
\item {Grp. gram.:f.}
\end{itemize}
Acto ou effeito de igualar.
\section{Igualadro}
\begin{itemize}
\item {Grp. gram.:m.  e  adj.}
\end{itemize}
O que iguala.
\section{Igualamento}
\begin{itemize}
\item {Grp. gram.:m.}
\end{itemize}
Acto de igualar.
Qualidade de sêr igual.
\section{Igualança}
\begin{itemize}
\item {Grp. gram.:f.}
\end{itemize}
\begin{itemize}
\item {Utilização:Ant.}
\end{itemize}
\begin{itemize}
\item {Proveniência:(De \textunderscore igualar\textunderscore )}
\end{itemize}
O mesmo que \textunderscore igualdade\textunderscore .
\section{Igualar}
\begin{itemize}
\item {Grp. gram.:v. t.}
\end{itemize}
\begin{itemize}
\item {Grp. gram.:V. i.}
\end{itemize}
Tornar igual; tornar plano, liso.
Sêr igual.
\section{Igualável}
\begin{itemize}
\item {Grp. gram.:adj.}
\end{itemize}
Que se póde igualar.
\section{Igualdação}
\begin{itemize}
\item {Grp. gram.:f.}
\end{itemize}
\begin{itemize}
\item {Utilização:Ant.}
\end{itemize}
Acto de igualdar.
\section{Igualdade}
\begin{itemize}
\item {Grp. gram.:f.}
\end{itemize}
\begin{itemize}
\item {Utilização:Mathem.}
\end{itemize}
\begin{itemize}
\item {Proveniência:(Lat. \textunderscore aequalitas\textunderscore )}
\end{itemize}
Qualidade daquelle ou daquillo que é igual; uniformidade.
Equação.
Em Algebra, expressão de duas quantidades, que têm o mesmo valor.
\section{Igualdança}
\begin{itemize}
\item {Grp. gram.:f.}
\end{itemize}
(V.igualança)
\section{Igualdar}
\textunderscore v. t.\textunderscore  (e der.)
O mesmo que \textunderscore igualar\textunderscore , etc.
\section{Igualeza}
\begin{itemize}
\item {Grp. gram.:f.}
\end{itemize}
\begin{itemize}
\item {Utilização:Ant.}
\end{itemize}
O mesmo que \textunderscore igualdade\textunderscore .
\section{Igualha}
\begin{itemize}
\item {Grp. gram.:f.}
\end{itemize}
Identidade de posição social: \textunderscore os patifes da sua igualha\textunderscore .
(Provavelmente do lat. \textunderscore aequalia\textunderscore , pl. de \textunderscore aequalis\textunderscore )
\section{Igualhar}
\begin{itemize}
\item {Grp. gram.:v. t.}
\end{itemize}
\begin{itemize}
\item {Utilização:Pop.}
\end{itemize}
O mesmo que \textunderscore igualar\textunderscore . Cf. Garrett, \textunderscore Romanceiro\textunderscore , II, 89 e 91.
\section{Igualismo}
\begin{itemize}
\item {Grp. gram.:m.}
\end{itemize}
\begin{itemize}
\item {Proveniência:(De \textunderscore igual\textunderscore )}
\end{itemize}
O mesmo que \textunderscore igualitarismo\textunderscore . Cf. Rui Barb., \textunderscore Réplica\textunderscore , II, 157.
\section{Igualitário}
\begin{itemize}
\item {Grp. gram.:m.  e  adj.}
\end{itemize}
\begin{itemize}
\item {Proveniência:(Do lat. \textunderscore aequalitas\textunderscore )}
\end{itemize}
Sectário do igualitarismo.
\section{Igualitarismo}
\begin{itemize}
\item {Grp. gram.:m.}
\end{itemize}
\begin{itemize}
\item {Proveniência:(De \textunderscore igualitário\textunderscore )}
\end{itemize}
Systema dos que proclamam a igualdade social.
\section{Igualmente}
\begin{itemize}
\item {Grp. gram.:adv.}
\end{itemize}
De modo igual; com igualdade.
Semelhantemente.
\section{Iguana}
\begin{itemize}
\item {Grp. gram.:f.}
\end{itemize}
O mesmo que \textunderscore iguano\textunderscore .
\section{Iguano}
\begin{itemize}
\item {Grp. gram.:m.}
\end{itemize}
\begin{itemize}
\item {Proveniência:(Do car. \textunderscore yana\textunderscore )}
\end{itemize}
Gênero de reptis sáurios.
\section{Iguanodonte}
\begin{itemize}
\item {Grp. gram.:m.}
\end{itemize}
\begin{itemize}
\item {Proveniência:(De \textunderscore iguano\textunderscore  + gr. \textunderscore odous\textunderscore )}
\end{itemize}
Gênero de reptis.
\section{Iguar}
\begin{itemize}
\item {Proveniência:(Do lat. \textunderscore aequare\textunderscore )}
\end{itemize}
\textunderscore v. t. Ant.\textunderscore  (e ainda hoje pop. na Beira)
O mesmo que \textunderscore igualar\textunderscore .
\section{Iguaria}
\begin{itemize}
\item {Grp. gram.:f.}
\end{itemize}
\begin{itemize}
\item {Utilização:Fig.}
\end{itemize}
\begin{itemize}
\item {Utilização:Açor}
\end{itemize}
Manjar delicado, appetitoso.
Qualquer manjar.
Comida; prato de comida.
Objecto de motejo.
Prenda, para se arrematar, nas festas do Espírito-Santo.
\section{Ih!}
\begin{itemize}
\item {Grp. gram.:interj.}
\end{itemize}
(designativa de admiração ou de ironia). Cf. B. Moreno, \textunderscore Com. do Campo\textunderscore , II, 147.
\section{Ilação}
\begin{itemize}
\item {Grp. gram.:f.}
\end{itemize}
\begin{itemize}
\item {Proveniência:(Lat. \textunderscore illatio\textunderscore )}
\end{itemize}
Aquilo que se infere de certos factos, princípios ou premissas.
Conclusão; deducção.
\section{Ilacerado}
\begin{itemize}
\item {Grp. gram.:adj.}
\end{itemize}
\begin{itemize}
\item {Proveniência:(De \textunderscore in...\textunderscore  + \textunderscore lacerado\textunderscore )}
\end{itemize}
Que não é lacerado.
\section{Ilacerável}
\begin{itemize}
\item {Grp. gram.:adj.}
\end{itemize}
\begin{itemize}
\item {Proveniência:(Lat. \textunderscore illacerabilis\textunderscore )}
\end{itemize}
Que se não póde lacerar.
\section{Ilacrimável}
\begin{itemize}
\item {Grp. gram.:adj.}
\end{itemize}
\begin{itemize}
\item {Proveniência:(Lat. \textunderscore illacrimabilis\textunderscore )}
\end{itemize}
Que não cede a lágrimas; implacável; inexorável.
\section{Ilama}
\begin{itemize}
\item {Grp. gram.:f.}
\end{itemize}
Fruto de uma árvore anonácea do México.
\section{Ilapso}
\begin{itemize}
\item {Grp. gram.:m.}
\end{itemize}
\begin{itemize}
\item {Proveniência:(Lat. \textunderscore illapsus\textunderscore )}
\end{itemize}
Influxo, com que Deus se põe em comunicação com a alma de alguém, segundo a opinião dos ascetas:«\textunderscore aqui são os ilapsos e comunicações de Deus.\textunderscore »\textunderscore Luz e Calor\textunderscore , 38.
\section{Ilaquear}
\begin{itemize}
\item {Grp. gram.:v. t.}
\end{itemize}
\begin{itemize}
\item {Utilização:Fig.}
\end{itemize}
\begin{itemize}
\item {Proveniência:(Lat. \textunderscore illaqueare\textunderscore )}
\end{itemize}
Enlaçar.
Prender; pear.
Fazer caír em lôgro, em tentação, etc.
\section{Ílaro}
\begin{itemize}
\item {Grp. gram.:m.}
\end{itemize}
\begin{itemize}
\item {Proveniência:(Do gr. \textunderscore eilar\textunderscore )}
\end{itemize}
Gênero de inseptos lepidópteros.
\section{Ilativo}
\begin{itemize}
\item {Grp. gram.:adj.}
\end{itemize}
\begin{itemize}
\item {Proveniência:(Lat. \textunderscore illativus\textunderscore )}
\end{itemize}
Em que há ilação; conclusivo.
\section{Ileáceas}
\begin{itemize}
\item {Grp. gram.:f. pl.}
\end{itemize}
\begin{itemize}
\item {Proveniência:(De \textunderscore ileáceo\textunderscore )}
\end{itemize}
Família de plantas, que tem por typo o azevinho.
\section{Ileáceo}
\begin{itemize}
\item {Grp. gram.:adj.}
\end{itemize}
\begin{itemize}
\item {Proveniência:(Do lat. \textunderscore ilex\textunderscore )}
\end{itemize}
Relativo ou semelhante ao azevinho.
\section{Ileadelfo}
\begin{itemize}
\item {Grp. gram.:m.}
\end{itemize}
\begin{itemize}
\item {Proveniência:(Do gr. \textunderscore ileon\textunderscore  + \textunderscore adelphos\textunderscore )}
\end{itemize}
Monstro, que é duplo, da bacia para baixo.
\section{Ileadelpho}
\begin{itemize}
\item {Grp. gram.:m.}
\end{itemize}
\begin{itemize}
\item {Proveniência:(Do gr. \textunderscore ileon\textunderscore  + \textunderscore adelphos\textunderscore )}
\end{itemize}
Monstro, que é duplo, da bacia para baixo.
\section{Ilecebras}
\begin{itemize}
\item {Grp. gram.:f. pl.}
\end{itemize}
\begin{itemize}
\item {Utilização:Des.}
\end{itemize}
\begin{itemize}
\item {Proveniência:(Lat. \textunderscore illecebrae\textunderscore )}
\end{itemize}
Tudo que se faz para atrair.
Blandícias.
Seduções.
\section{Ilécebro}
\begin{itemize}
\item {Grp. gram.:m.}
\end{itemize}
\begin{itemize}
\item {Proveniência:(Do gr. \textunderscore illekebrein\textunderscore , encantar)}
\end{itemize}
Gênero de plantas cariofiláceas.
\section{Ilegal}
\begin{itemize}
\item {Grp. gram.:adj.}
\end{itemize}
\begin{itemize}
\item {Proveniência:(De \textunderscore in...\textunderscore  + \textunderscore legal\textunderscore )}
\end{itemize}
Que não é legal; ilegítimo; ilícito.
\section{Ilegalidade}
\begin{itemize}
\item {Grp. gram.:f.}
\end{itemize}
Qualidade daquilo que é ilegal.
\section{Ilegalmente}
\begin{itemize}
\item {Grp. gram.:adv.}
\end{itemize}
De modo ilegal.
\section{Ilegibilidade}
\begin{itemize}
\item {Grp. gram.:f.}
\end{itemize}
Qualidade ilegível.
\section{Ilegitimamente}
\begin{itemize}
\item {Grp. gram.:adv.}
\end{itemize}
De modo ilegítimo.
\section{Ilegitimidade}
\begin{itemize}
\item {Grp. gram.:f.}
\end{itemize}
Qualidade daquilo que é ilegítimo.
\section{Ilegítimo}
\begin{itemize}
\item {Grp. gram.:adj.}
\end{itemize}
\begin{itemize}
\item {Proveniência:(Lat. \textunderscore illegitimus\textunderscore )}
\end{itemize}
Que não é legítimo; injusto.
\section{Ilegível}
\begin{itemize}
\item {Grp. gram.:adj.}
\end{itemize}
\begin{itemize}
\item {Proveniência:(De \textunderscore in...\textunderscore  + \textunderscore legível\textunderscore )}
\end{itemize}
Que não é legível, que se não póde lêr: \textunderscore uma inscripção ilegivel\textunderscore .
\section{Ileíte}
\begin{itemize}
\item {Grp. gram.:f.}
\end{itemize}
\begin{itemize}
\item {Proveniência:(De \textunderscore íleo\textunderscore ^2)}
\end{itemize}
Inflammação da membrana mucosa do íleo.
\section{Íleo}
\begin{itemize}
\item {Grp. gram.:m.}
\end{itemize}
\begin{itemize}
\item {Proveniência:(Lat. \textunderscore ileum\textunderscore )}
\end{itemize}
O mesmo que \textunderscore vólvolo\textunderscore .
\section{Íleo}
\begin{itemize}
\item {Grp. gram.:m.}
\end{itemize}
\begin{itemize}
\item {Utilização:Anat.}
\end{itemize}
\begin{itemize}
\item {Proveniência:(Do lat. \textunderscore ileos\textunderscore )}
\end{itemize}
Última parte do intestino delgado.
\section{Ileo-cecal}
\begin{itemize}
\item {Grp. gram.:adj.}
\end{itemize}
Relativo ao íleo e ao céco.
\section{Ileo-diclidite}
\begin{itemize}
\item {Grp. gram.:f.}
\end{itemize}
\begin{itemize}
\item {Proveniência:(Do gr. \textunderscore ileon\textunderscore  + \textunderscore diklis\textunderscore )}
\end{itemize}
Inflammação do íleo e da válvula ileo-cecal.
Nome, que alguns dão á febre typhoide.
\section{Ileo-dólico}
\begin{itemize}
\item {Grp. gram.:adj.}
\end{itemize}
Relativo ao íleo e ao cólon.
\section{Ileologia}
\begin{itemize}
\item {Grp. gram.:f.}
\end{itemize}
\begin{itemize}
\item {Proveniência:(Do gr. \textunderscore eilein\textunderscore  + \textunderscore logos\textunderscore )}
\end{itemize}
Tratado á cêrca dos intestinos.
\section{Ileológico}
\begin{itemize}
\item {Grp. gram.:adj.}
\end{itemize}
Relativo á ileologia.
\section{Ileologista}
\begin{itemize}
\item {Grp. gram.:m.}
\end{itemize}
Aquelle que se occupa da ileologia.
\section{Íleon}
\begin{itemize}
\item {Grp. gram.:m.}
\end{itemize}
(V. \textunderscore íleo\textunderscore ^2)
\section{Íleos}
\begin{itemize}
\item {Grp. gram.:m.}
\end{itemize}
\begin{itemize}
\item {Utilização:Des.}
\end{itemize}
\begin{itemize}
\item {Proveniência:(Lat. \textunderscore ileos\textunderscore )}
\end{itemize}
Dôr ilíaca.
Dôr rheumática do intestino.
O mesmo que \textunderscore íleo\textunderscore ^2.
\section{Ileose}
\begin{itemize}
\item {Grp. gram.:f.}
\end{itemize}
Doença do íleo.
\section{Ileoso}
\begin{itemize}
\item {Grp. gram.:adj.}
\end{itemize}
\begin{itemize}
\item {Utilização:Des.}
\end{itemize}
Que padece íleos.
\section{Ileso}
\begin{itemize}
\item {Grp. gram.:adj.}
\end{itemize}
\begin{itemize}
\item {Proveniência:(Lat. \textunderscore illaesus\textunderscore )}
\end{itemize}
Que não é ou não está leso; que ficou incólume.
\section{Iletrado}
\begin{itemize}
\item {Grp. gram.:m.  e  adj.}
\end{itemize}
\begin{itemize}
\item {Proveniência:(Lat. \textunderscore illiteratus\textunderscore )}
\end{itemize}
O que não é letrado.
Analfabeto.
\section{Ilha}
\begin{itemize}
\item {Grp. gram.:f.}
\end{itemize}
\begin{itemize}
\item {Grp. gram.:Pl.}
\end{itemize}
\begin{itemize}
\item {Proveniência:(Do lat. \textunderscore insula\textunderscore )}
\end{itemize}
Espaço de terra, cercado de água por todos os lados.
Grupo de casas, insulado de outras habitações e cercado de ruas por todos os lados.
Pátio, cercado de habitações pobres.
O mesmo que \textunderscore archipélago\textunderscore .
\section{Ilhal}
\begin{itemize}
\item {Grp. gram.:m.}
\end{itemize}
\begin{itemize}
\item {Proveniência:(Do lat. \textunderscore ilia\textunderscore )}
\end{itemize}
Cada uma das duas partes da rês, situadas entre a última costella, a ponta da alcatra e o lombo.
Cada uma das depressões lateraes, por baixo dos lombos do cavallo.
\section{Ilhapa}
\begin{itemize}
\item {Grp. gram.:f.}
\end{itemize}
\begin{itemize}
\item {Utilização:Bras. do S}
\end{itemize}
Parte mais grossa do laço, com que se apprehendem os animaes no campo.
\section{Ilhar}
\begin{itemize}
\item {Grp. gram.:v. t.}
\end{itemize}
\begin{itemize}
\item {Proveniência:(De \textunderscore ilha\textunderscore )}
\end{itemize}
Separar por todos os lados, tornar incommunicável, insular.
\section{Ilharga}
\begin{itemize}
\item {Grp. gram.:f.}
\end{itemize}
\begin{itemize}
\item {Utilização:Fig.}
\end{itemize}
\begin{itemize}
\item {Utilização:Prov.}
\end{itemize}
\begin{itemize}
\item {Utilização:trasm.}
\end{itemize}
\begin{itemize}
\item {Grp. gram.:Pl. Loc. adv.}
\end{itemize}
\begin{itemize}
\item {Proveniência:(Do b. lat. \textunderscore iliarica\textunderscore )}
\end{itemize}
Cada uma das partes lateraes e inferiores do baixo ventre.
Lado de um corpo.
O mesmo que \textunderscore ilhal\textunderscore .
Esteio, apoio.
Pessôa, que protege.
O mesmo que \textunderscore empenha\textunderscore .
\textunderscore Ás ilhargas\textunderscore , ao lado, a par. Cf. R. Lobo, \textunderscore Côrte na Ald.\textunderscore , I, 58.
\section{Ilhargada}
\begin{itemize}
\item {Grp. gram.:f.}
\end{itemize}
A região da ilharga; o lado da ilharga. Cf. G. Vicente, \textunderscore Triunfo do Inverno\textunderscore .
\section{Ilhargado}
\begin{itemize}
\item {Grp. gram.:adj.}
\end{itemize}
\begin{itemize}
\item {Utilização:Ant.}
\end{itemize}
Relativo á ilharga.
Que era da ilharga de um animal, (falando-se do coiro).
\section{Ilhargueiro}
\begin{itemize}
\item {Grp. gram.:adj.}
\end{itemize}
\begin{itemize}
\item {Utilização:Ant.}
\end{itemize}
\begin{itemize}
\item {Proveniência:(De \textunderscore ilharga\textunderscore )}
\end{itemize}
O mesmo que \textunderscore collateral\textunderscore .
\section{Ilhéo}
\begin{itemize}
\item {Grp. gram.:adj.}
\end{itemize}
\begin{itemize}
\item {Grp. gram.:M.}
\end{itemize}
\begin{itemize}
\item {Proveniência:(Do rad. de \textunderscore ilha\textunderscore )}
\end{itemize}
Relativo a ilhas.
Aquelle que é natural das ilhas.
Rochedo no mar, ilhota.
\section{Ilheta}
\begin{itemize}
\item {fónica:lhê}
\end{itemize}
\begin{itemize}
\item {Grp. gram.:f.}
\end{itemize}
(V.ilhota)
\section{Ilhéu}
\begin{itemize}
\item {Grp. gram.:adj.}
\end{itemize}
\begin{itemize}
\item {Grp. gram.:M.}
\end{itemize}
\begin{itemize}
\item {Proveniência:(Do rad. de \textunderscore ilha\textunderscore )}
\end{itemize}
Relativo a ilhas.
Aquelle que é natural das ilhas.
Rochedo no mar, ilhota.
\section{Ilho}
\begin{itemize}
\item {Grp. gram.:m.}
\end{itemize}
\begin{itemize}
\item {Utilização:Prov.}
\end{itemize}
\begin{itemize}
\item {Utilização:alent.}
\end{itemize}
Homem do Norte do país, o qual se emprega na pesca ou em fazer recovagem com os seus saveiros.
(Provavelmente, de \textunderscore Ílhavo\textunderscore , n. p.)
\section{Ilhó}
\begin{itemize}
\item {Grp. gram.:m.  e  f.}
\end{itemize}
Orifício circular, por onde se enfia um atacador, fita, etc.
Pequeno orifício em pano, cartão, coiro, etc.
Aro de metal, para debruar um ilhó.
(Por \textunderscore olhó\textunderscore , do lat. hyp. \textunderscore oculiculus\textunderscore , de \textunderscore oculus\textunderscore )
\section{Ilhôa}
\begin{itemize}
\item {Grp. gram.:adj.}
\end{itemize}
\begin{itemize}
\item {Grp. gram.:F.}
\end{itemize}
\begin{itemize}
\item {Utilização:Bras. de Minas}
\end{itemize}
Diz-se da mulher, que nasceu ou vive em ilha.
Mulher, que é natural de uma ilha.
Fruto da limeira, lima.
(Fem. de \textunderscore ilhéu\textunderscore )
\section{Ilhoco}
\begin{itemize}
\item {fónica:lhô}
\end{itemize}
\begin{itemize}
\item {Grp. gram.:m.}
\end{itemize}
\begin{itemize}
\item {Utilização:Ant.}
\end{itemize}
Ilhota.
Ilhéu. Cf. \textunderscore Ethióp. Or.\textunderscore , I, 275.
\section{Ilhós}
\begin{itemize}
\item {Grp. gram.:m.  e  f.}
\end{itemize}
\begin{itemize}
\item {Utilização:Gír.}
\end{itemize}
O mesmo que \textunderscore ilhó\textunderscore .
O ânus.
\section{Ilhota}
\begin{itemize}
\item {Grp. gram.:f.}
\end{itemize}
Pequena ilha.
\section{Ilhote}
\begin{itemize}
\item {Grp. gram.:m.}
\end{itemize}
(V.ilhota)
\section{Ilíaco}
\begin{itemize}
\item {Grp. gram.:adj.}
\end{itemize}
\begin{itemize}
\item {Proveniência:(Lat. \textunderscore iliacus\textunderscore )}
\end{itemize}
Relativo á bacia, no tronco humano.
Que faz parte dessa região: \textunderscore osso ilíaco\textunderscore .
\section{Ilíada}
\begin{itemize}
\item {Grp. gram.:f.}
\end{itemize}
\begin{itemize}
\item {Utilização:Fig.}
\end{itemize}
\begin{itemize}
\item {Proveniência:(De \textunderscore Ilíada\textunderscore , poema attribuído a Homero)}
\end{itemize}
Série de aventuras ou feitos heróicos.
\section{Ilibação}
\begin{itemize}
\item {Grp. gram.:f.}
\end{itemize}
Acto ou efeito de ilibar.
\section{Ilibar}
\begin{itemize}
\item {Grp. gram.:v. t.}
\end{itemize}
\begin{itemize}
\item {Proveniência:(Do lat. \textunderscore in...\textunderscore  + \textunderscore libare\textunderscore )}
\end{itemize}
Tirar mancha a.
Tornar puro; rehabilitar: \textunderscore ilibar os créditos de alguém\textunderscore .
\section{Iliberal}
\begin{itemize}
\item {Grp. gram.:adj.}
\end{itemize}
\begin{itemize}
\item {Proveniência:(Lat. \textunderscore illiberalis\textunderscore )}
\end{itemize}
Que não é liberal; somítico.
Amigo do despotismo.
Adversário da liberdade.
\section{Iliberalidade}
\begin{itemize}
\item {Grp. gram.:f.}
\end{itemize}
\begin{itemize}
\item {Proveniência:(Lat. \textunderscore illiberalitas\textunderscore )}
\end{itemize}
Qualidade de quem é iliberal.
\section{Iliberalismo}
\begin{itemize}
\item {Grp. gram.:m.}
\end{itemize}
\begin{itemize}
\item {Proveniência:(De \textunderscore iliberal\textunderscore )}
\end{itemize}
Sistema ou opinião contrária ao liberalismo político.
\section{Iliberalmente}
\begin{itemize}
\item {Grp. gram.:adv.}
\end{itemize}
De modo iliberal.
\section{Iliçador}
\begin{itemize}
\item {Grp. gram.:m.}
\end{itemize}
Aquele que iliça.
\section{Ilição}
\begin{itemize}
\item {Grp. gram.:f.}
\end{itemize}
\begin{itemize}
\item {Utilização:Med.}
\end{itemize}
\begin{itemize}
\item {Utilização:Des.}
\end{itemize}
\begin{itemize}
\item {Proveniência:(Do lat. \textunderscore illitum\textunderscore )}
\end{itemize}
Untura, fomentação.
\section{Iliçar}
\begin{itemize}
\item {Grp. gram.:v. t.}
\end{itemize}
\begin{itemize}
\item {Utilização:Ant.}
\end{itemize}
\begin{itemize}
\item {Proveniência:(Do lat. \textunderscore illicere\textunderscore )}
\end{itemize}
Burlar, vendendo ou obrigando, como seus, bens que lhe não pertencem.
\section{Ilíceas}
\begin{itemize}
\item {Grp. gram.:f. pl.}
\end{itemize}
\begin{itemize}
\item {Proveniência:(Do lat. \textunderscore ilex\textunderscore )}
\end{itemize}
Tríbo de plantas magnoliáceas, no systema de De-Candolle.
\section{Ilicina}
\begin{itemize}
\item {Grp. gram.:f.}
\end{itemize}
\begin{itemize}
\item {Proveniência:(Do lat. \textunderscore ilex\textunderscore )}
\end{itemize}
Substância còrante do azevinho.
\section{Ilicíneas}
\begin{itemize}
\item {Grp. gram.:f. pl.}
\end{itemize}
O mesmo que \textunderscore ileáceas\textunderscore .
\section{Ilício}
\begin{itemize}
\item {Grp. gram.:m.}
\end{itemize}
\begin{itemize}
\item {Proveniência:(Lat. \textunderscore illicium\textunderscore )}
\end{itemize}
Acto ou crime de iliçar.
\section{Ilicitamente}
\begin{itemize}
\item {Grp. gram.:adv.}
\end{itemize}
De modo ilícito.
\section{Ilícito}
\begin{itemize}
\item {Grp. gram.:adj.}
\end{itemize}
\begin{itemize}
\item {Proveniência:(Lat. \textunderscore illicitus\textunderscore )}
\end{itemize}
Que não é lícito; contrário ás leis ou á moral; ilegítimo: \textunderscore relações ilícitas\textunderscore .
\section{Ilídimo}
\begin{itemize}
\item {Grp. gram.:adj.}
\end{itemize}
\begin{itemize}
\item {Utilização:Ant.}
\end{itemize}
\begin{itemize}
\item {Proveniência:(De \textunderscore in...\textunderscore  + \textunderscore lídimo\textunderscore )}
\end{itemize}
Que não é lídimo.
\section{Ilidir}
\begin{itemize}
\item {Grp. gram.:v. t.}
\end{itemize}
\begin{itemize}
\item {Proveniência:(Lat. \textunderscore illidere\textunderscore )}
\end{itemize}
Rebater; refutar.
\section{Ilidível}
\begin{itemize}
\item {Grp. gram.:adj.}
\end{itemize}
Que se póde ilidir.
\section{Iligar}
\begin{itemize}
\item {Grp. gram.:v. t.}
\end{itemize}
\begin{itemize}
\item {Proveniência:(Lat. \textunderscore illigare\textunderscore )}
\end{itemize}
Ligar; atar. Cf. Latino, \textunderscore Hist. Pol. e Mil.\textunderscore , I, 58.
\section{Ilimitado}
\begin{itemize}
\item {Grp. gram.:adj.}
\end{itemize}
\begin{itemize}
\item {Proveniência:(De \textunderscore in...\textunderscore  + \textunderscore limitado\textunderscore )}
\end{itemize}
Que não é limitado.
Imenso: \textunderscore no espaço ilimitado...\textunderscore 
\section{Ilimitável}
\begin{itemize}
\item {Grp. gram.:adj.}
\end{itemize}
\begin{itemize}
\item {Proveniência:(De \textunderscore in...\textunderscore  + \textunderscore limitar\textunderscore )}
\end{itemize}
Que se não póde limitar.
Indefinido; imenso.
\section{Ílio}
\begin{itemize}
\item {Grp. gram.:m.}
\end{itemize}
\begin{itemize}
\item {Proveniência:(Do rad. do lat. \textunderscore ilia\textunderscore )}
\end{itemize}
A maior das três partes, em que se divide o osso ilíaco.
\section{Ilio...}
\begin{itemize}
\item {Proveniência:(Do lat. \textunderscore ilia\textunderscore )}
\end{itemize}
Elemento, que entra na composição de algumas palavras, com a significação de \textunderscore relativo ao ílio\textunderscore .
\section{Ilio-abdominal}
\begin{itemize}
\item {Grp. gram.:adj.}
\end{itemize}
Diz-se de um músculo, que entra na formação das paredes abdominaes.
\section{Ilio-costal}
\begin{itemize}
\item {Grp. gram.:adj.}
\end{itemize}
Diz-se de um músculo, que vai da última costella ao osso ilíaco.
\section{Ilio-femoral}
\begin{itemize}
\item {Grp. gram.:adj.}
\end{itemize}
Diz-se de um músculo delgado da parte anterior da coxa.
\section{Ilio-hypogástrico}
\begin{itemize}
\item {Grp. gram.:adj.}
\end{itemize}
Relativo ao osso ilíaco e ao hypogástrio.
\section{Ilio-inguinal}
\begin{itemize}
\item {Grp. gram.:adj.}
\end{itemize}
Diz-se do vaso nervoso, que se liga ao músculo oblíquo ascendente do abdome.
\section{Ilio-lombar}
\begin{itemize}
\item {Grp. gram.:adj.}
\end{itemize}
Que vai do osso ilíaco á região lombar.
\section{Ílion}
\begin{itemize}
\item {Grp. gram.:m.}
\end{itemize}
(V. \textunderscore ílio\textunderscore ^1)
\section{Ílio-pectineo}
\begin{itemize}
\item {Grp. gram.:adj.}
\end{itemize}
Relativo ao osso ilíaco e ao púbis.
\section{Ilio-rotuliano}
\begin{itemize}
\item {Grp. gram.:adj.}
\end{itemize}
Relativo ao osso ilíaco e á rótula.
\section{Ilio-sacro}
\begin{itemize}
\item {Grp. gram.:adj.}
\end{itemize}
Relativo ao osso ilíaco e ao sacro.
\section{Ilíquido}
\begin{itemize}
\item {Grp. gram.:adj.}
\end{itemize}
\begin{itemize}
\item {Proveniência:(De \textunderscore in...\textunderscore  + \textunderscore líquido\textunderscore )}
\end{itemize}
Que não é ou não está líquido.
Confuso.
\section{Iliterado}
\begin{itemize}
\item {Grp. gram.:adj.}
\end{itemize}
O mesmo que \textunderscore iliterato\textunderscore .
\section{Iliterato}
\begin{itemize}
\item {Grp. gram.:adj.}
\end{itemize}
\begin{itemize}
\item {Grp. gram.:M.}
\end{itemize}
O mesmo que \textunderscore iletrado\textunderscore .
Aquele que é iletrado. Cf. Garrett, \textunderscore Camões\textunderscore , 236.
\section{Illação}
\begin{itemize}
\item {Grp. gram.:f.}
\end{itemize}
\begin{itemize}
\item {Proveniência:(Lat. \textunderscore illatio\textunderscore )}
\end{itemize}
Aquillo que se infere de certos factos, princípios ou premissas.
Conclusão; deducção.
\section{Illacerado}
\begin{itemize}
\item {Grp. gram.:adj.}
\end{itemize}
\begin{itemize}
\item {Proveniência:(De \textunderscore in...\textunderscore  + \textunderscore lacerado\textunderscore )}
\end{itemize}
Que não é lacerado.
\section{Illacerável}
\begin{itemize}
\item {Grp. gram.:adj.}
\end{itemize}
\begin{itemize}
\item {Proveniência:(Lat. \textunderscore illacerabilis\textunderscore )}
\end{itemize}
Que se não póde lacerar.
\section{Illacrimável}
\begin{itemize}
\item {Grp. gram.:adj.}
\end{itemize}
\begin{itemize}
\item {Proveniência:(Lat. \textunderscore illacrimabilis\textunderscore )}
\end{itemize}
Que não cede a lágrimas; implacável; inexorável.
\section{Illapso}
\begin{itemize}
\item {Grp. gram.:m.}
\end{itemize}
\begin{itemize}
\item {Proveniência:(Lat. \textunderscore illapsus\textunderscore )}
\end{itemize}
Influxo, com que Deus se põe em communicação com a alma de alguém, segundo a opinião dos ascetas:«\textunderscore aqui são os illapsos e communicações de Deus.\textunderscore »\textunderscore Luz e Calor\textunderscore , 38.
\section{Illaquear}
\begin{itemize}
\item {Grp. gram.:v. t.}
\end{itemize}
\begin{itemize}
\item {Utilização:Fig.}
\end{itemize}
\begin{itemize}
\item {Proveniência:(Lat. \textunderscore illaqueare\textunderscore )}
\end{itemize}
Enlaçar.
Prender; pear.
Fazer caír em lôgro, em tentação, etc.
\section{Illativo}
\begin{itemize}
\item {Grp. gram.:adj.}
\end{itemize}
\begin{itemize}
\item {Proveniência:(Lat. \textunderscore illativus\textunderscore )}
\end{itemize}
Em que há illação; conclusivo.
\section{Illécebras}
\begin{itemize}
\item {Grp. gram.:f. pl.}
\end{itemize}
\begin{itemize}
\item {Utilização:Des.}
\end{itemize}
\begin{itemize}
\item {Proveniência:(Lat. \textunderscore illecebrae\textunderscore )}
\end{itemize}
Tudo que se faz para attrahir.
Blandícias.
Seducções.
\section{Illécebro}
\begin{itemize}
\item {Grp. gram.:m.}
\end{itemize}
\begin{itemize}
\item {Proveniência:(Do gr. \textunderscore illekebrein\textunderscore , encantar)}
\end{itemize}
Gênero de plantas caryophylláceas.
\section{Illegal}
\begin{itemize}
\item {Grp. gram.:adj.}
\end{itemize}
\begin{itemize}
\item {Proveniência:(De \textunderscore in...\textunderscore  + \textunderscore legal\textunderscore )}
\end{itemize}
Que não é legal; illegítimo; illícito.
\section{Illegalidade}
\begin{itemize}
\item {Grp. gram.:f.}
\end{itemize}
Qualidade daquillo que é illegal.
\section{Illegalmente}
\begin{itemize}
\item {Grp. gram.:adv.}
\end{itemize}
De modo illegal.
\section{Illegibilidade}
\begin{itemize}
\item {Grp. gram.:f.}
\end{itemize}
Qualidade illegível.
\section{Illegitimamente}
\begin{itemize}
\item {Grp. gram.:adv.}
\end{itemize}
De modo illegítimo.
\section{Illegitimidade}
\begin{itemize}
\item {Grp. gram.:f.}
\end{itemize}
Qualidade daquillo que é illegítimo.
\section{Illegítimo}
\begin{itemize}
\item {Grp. gram.:adj.}
\end{itemize}
\begin{itemize}
\item {Proveniência:(Lat. \textunderscore illegitimus\textunderscore )}
\end{itemize}
Que não é legítimo; injusto.
\section{Illegível}
\begin{itemize}
\item {Grp. gram.:adj.}
\end{itemize}
\begin{itemize}
\item {Proveniência:(De \textunderscore in...\textunderscore  + \textunderscore legível\textunderscore )}
\end{itemize}
Que não é legível, que se não póde lêr: \textunderscore uma inscripção illegivel\textunderscore .
\section{Illeso}
\begin{itemize}
\item {Grp. gram.:adj.}
\end{itemize}
\begin{itemize}
\item {Proveniência:(Lat. \textunderscore illaesus\textunderscore )}
\end{itemize}
Que não é ou não está leso; que ficou incólume.
\section{Illetrado}
\begin{itemize}
\item {Grp. gram.:m.  e  adj.}
\end{itemize}
\begin{itemize}
\item {Proveniência:(Lat. \textunderscore illiteratus\textunderscore )}
\end{itemize}
O que não é letrado.
Analphabeto.
\section{Illibação}
\begin{itemize}
\item {Grp. gram.:f.}
\end{itemize}
Acto ou effeito de illibar.
\section{Illibar}
\begin{itemize}
\item {Grp. gram.:v. t.}
\end{itemize}
\begin{itemize}
\item {Proveniência:(Do lat. \textunderscore in...\textunderscore  + \textunderscore libare\textunderscore )}
\end{itemize}
Tirar mancha a.
Tornar puro; rehabilitar: \textunderscore illibar os créditos de alguém\textunderscore .
\section{Illiberal}
\begin{itemize}
\item {Grp. gram.:adj.}
\end{itemize}
\begin{itemize}
\item {Proveniência:(Lat. \textunderscore illiberalis\textunderscore )}
\end{itemize}
Que não é liberal; somítico.
Amigo do despotismo.
Adversário da liberdade.
\section{Illiberalidade}
\begin{itemize}
\item {Grp. gram.:f.}
\end{itemize}
\begin{itemize}
\item {Proveniência:(Lat. \textunderscore illiberalitas\textunderscore )}
\end{itemize}
Qualidade de quem é illiberal.
\section{Illiberalismo}
\begin{itemize}
\item {Grp. gram.:m.}
\end{itemize}
\begin{itemize}
\item {Proveniência:(De \textunderscore illiberal\textunderscore )}
\end{itemize}
Systema ou opinião contrária ao liberalismo político.
\section{Illiberalmente}
\begin{itemize}
\item {Grp. gram.:adv.}
\end{itemize}
De modo illiberal.
\section{Illiçador}
\begin{itemize}
\item {Grp. gram.:m.}
\end{itemize}
Aquelle que illiça.
\section{Illição}
\begin{itemize}
\item {Grp. gram.:f.}
\end{itemize}
\begin{itemize}
\item {Utilização:Med.}
\end{itemize}
\begin{itemize}
\item {Utilização:Des.}
\end{itemize}
\begin{itemize}
\item {Proveniência:(Do lat. \textunderscore illitum\textunderscore )}
\end{itemize}
Untura, fomentação.
\section{Illiçar}
\begin{itemize}
\item {Grp. gram.:v. t.}
\end{itemize}
\begin{itemize}
\item {Utilização:Ant.}
\end{itemize}
\begin{itemize}
\item {Proveniência:(Do lat. \textunderscore illicere\textunderscore )}
\end{itemize}
Burlar, vendendo ou obrigando, como seus, bens que lhe não pertencem.
\section{Illício}
\begin{itemize}
\item {Grp. gram.:m.}
\end{itemize}
\begin{itemize}
\item {Proveniência:(Lat. \textunderscore illicium\textunderscore )}
\end{itemize}
Acto ou crime de illiçar.
\section{Illicitamente}
\begin{itemize}
\item {Grp. gram.:adv.}
\end{itemize}
De modo illícito.
\section{Illícito}
\begin{itemize}
\item {Grp. gram.:adj.}
\end{itemize}
\begin{itemize}
\item {Proveniência:(Lat. \textunderscore illicitus\textunderscore )}
\end{itemize}
Que não é lícito; contrário ás leis ou á moral; illegítimo: \textunderscore relações illícitas\textunderscore .
\section{Illídimo}
\begin{itemize}
\item {Grp. gram.:adj.}
\end{itemize}
\begin{itemize}
\item {Utilização:Ant.}
\end{itemize}
\begin{itemize}
\item {Proveniência:(De \textunderscore in...\textunderscore  + \textunderscore lídimo\textunderscore )}
\end{itemize}
Que não é lídimo.
\section{Illidir}
\begin{itemize}
\item {Grp. gram.:v. t.}
\end{itemize}
\begin{itemize}
\item {Proveniência:(Lat. \textunderscore illidere\textunderscore )}
\end{itemize}
Rebater; refutar.
\section{Illidível}
\begin{itemize}
\item {Grp. gram.:adj.}
\end{itemize}
Que se póde illidir.
\section{Illigar}
\begin{itemize}
\item {Grp. gram.:v. t.}
\end{itemize}
\begin{itemize}
\item {Proveniência:(Lat. \textunderscore illigare\textunderscore )}
\end{itemize}
Ligar; atar. Cf. Latino, \textunderscore Hist. Pol. e Mil.\textunderscore , I, 58.
\section{Illimitado}
\begin{itemize}
\item {Grp. gram.:adj.}
\end{itemize}
\begin{itemize}
\item {Proveniência:(De \textunderscore in...\textunderscore  + \textunderscore limitado\textunderscore )}
\end{itemize}
Que não é limitado.
Immenso: \textunderscore no espaço illimitado...\textunderscore 
\section{Illimitável}
\begin{itemize}
\item {Grp. gram.:adj.}
\end{itemize}
\begin{itemize}
\item {Proveniência:(De \textunderscore in...\textunderscore  + \textunderscore limitar\textunderscore )}
\end{itemize}
Que se não póde limitar.
Indefinido; immenso.
\section{Illíquido}
\begin{itemize}
\item {Grp. gram.:adj.}
\end{itemize}
\begin{itemize}
\item {Proveniência:(De \textunderscore in...\textunderscore  + \textunderscore líquido\textunderscore )}
\end{itemize}
Que não é ou não está líquido.
Confuso.
\section{Illiterado}
\begin{itemize}
\item {Grp. gram.:adj.}
\end{itemize}
O mesmo que \textunderscore illiterato\textunderscore .
\section{Illiterato}
\begin{itemize}
\item {Grp. gram.:adj.}
\end{itemize}
\begin{itemize}
\item {Grp. gram.:M.}
\end{itemize}
O mesmo que \textunderscore illetrado\textunderscore .
Aquelle que é illetrado. Cf. Garrett, \textunderscore Camões\textunderscore , 236.
\section{Illocano}
\begin{itemize}
\item {Grp. gram.:m.}
\end{itemize}
Um dos dialectos das Filippinas.
\section{Illocável}
\begin{itemize}
\item {Grp. gram.:adj.}
\end{itemize}
\begin{itemize}
\item {Proveniência:(Lat. \textunderscore illocabilis\textunderscore )}
\end{itemize}
Que se não póde collocar.
Que não occupa lugar.
\section{Illogicamente}
\begin{itemize}
\item {Grp. gram.:adv.}
\end{itemize}
De modo illógico.
\section{Illógico}
\begin{itemize}
\item {Grp. gram.:adj.}
\end{itemize}
\begin{itemize}
\item {Proveniência:(De \textunderscore in...\textunderscore  + \textunderscore lógico\textunderscore )}
\end{itemize}
Que não é logico.
Absurdo; incoherente.
\section{Illogismo}
\begin{itemize}
\item {Grp. gram.:m.}
\end{itemize}
\begin{itemize}
\item {Proveniência:(De \textunderscore illógico\textunderscore )}
\end{itemize}
Falta de lógica.
\section{Illudente}
\begin{itemize}
\item {Grp. gram.:adj.}
\end{itemize}
\begin{itemize}
\item {Proveniência:(Lat. \textunderscore illudens\textunderscore )}
\end{itemize}
Que illude.
\section{Illudir}
\begin{itemize}
\item {Grp. gram.:v. t.}
\end{itemize}
\begin{itemize}
\item {Proveniência:(Lat. \textunderscore illudere\textunderscore )}
\end{itemize}
Causar illusão a.
Enganar.
Lograr; burlar.
Sophismar.
\section{Illudivel}
\begin{itemize}
\item {Grp. gram.:adj.}
\end{itemize}
\begin{itemize}
\item {Proveniência:(De \textunderscore illudir\textunderscore )}
\end{itemize}
Que póde sêr illudido; que se póde induzir em êrro.
Em que póde haver illusão.
\section{Illudivelmente}
\begin{itemize}
\item {Grp. gram.:adv.}
\end{itemize}
De modo illudivel.
\section{Illuminação}
\begin{itemize}
\item {Grp. gram.:f.}
\end{itemize}
\begin{itemize}
\item {Proveniência:(Lat. \textunderscore illuminatio\textunderscore )}
\end{itemize}
Acto ou effeito de illuminar.
Irradiação.
Conjunto de luzes: \textunderscore illuminação a gás\textunderscore .
Estado daquillo que é illuminado.
\section{Illumminado}
\begin{itemize}
\item {Grp. gram.:adj.}
\end{itemize}
\begin{itemize}
\item {Grp. gram.:M.}
\end{itemize}
\begin{itemize}
\item {Proveniência:(De \textunderscore illuminar\textunderscore )}
\end{itemize}
Que tem illuminuras.
Colorido.
Que recebe luz: \textunderscore rua illuminada\textunderscore .
Instruido.
Sectário do illuminismo.
Visionário em questões religiosas.
Membro de algumas seitas religiosas.
\section{Illuminador}
\begin{itemize}
\item {Grp. gram.:adj.}
\end{itemize}
\begin{itemize}
\item {Grp. gram.:M.}
\end{itemize}
\begin{itemize}
\item {Proveniência:(Lat. \textunderscore illuminator\textunderscore )}
\end{itemize}
Que illumina.
Aquelle que illumina; aquelle que faz illuminuras.
\section{Illuminadura}
\begin{itemize}
\item {Grp. gram.:f.}
\end{itemize}
O mesmo que \textunderscore illuminação\textunderscore .
\section{Illuminante}
\begin{itemize}
\item {Grp. gram.:adj.}
\end{itemize}
\begin{itemize}
\item {Proveniência:(Lat. \textunderscore illuminans\textunderscore )}
\end{itemize}
Que illumina.
\section{Illuminar}
\begin{itemize}
\item {Grp. gram.:v. t.}
\end{itemize}
\begin{itemize}
\item {Utilização:Fig.}
\end{itemize}
\begin{itemize}
\item {Proveniência:(Lat. \textunderscore illuminare\textunderscore )}
\end{itemize}
Tornar luminoso, claro.
Diffundir luz em ou sôbre.
Abrilhantar.
Esclarecer.
Ensinar.
Illustrar.
Adornar; pintar com illuminuras.
\section{Illuminativo}
\begin{itemize}
\item {Grp. gram.:adj.}
\end{itemize}
O mesmo que \textunderscore illuminante\textunderscore .
\section{Illuminismo}
\begin{itemize}
\item {Grp. gram.:m.}
\end{itemize}
\begin{itemize}
\item {Proveniência:(De \textunderscore illuminar\textunderscore )}
\end{itemize}
Conjunto de opiniões, preconizadas no seculo XVIII, sôbre a existência de uma inspiração sobrenatural.
\section{Illuminista}
\begin{itemize}
\item {Grp. gram.:m.}
\end{itemize}
\begin{itemize}
\item {Proveniência:(De \textunderscore illuminar\textunderscore )}
\end{itemize}
Partidário do illuminismo.
\section{Illuminura}
\begin{itemize}
\item {Grp. gram.:f.}
\end{itemize}
\begin{itemize}
\item {Proveniência:(De \textunderscore illuminar\textunderscore )}
\end{itemize}
Pintura a côres, nos livros e manuscritos da Idade-Média.
Applicação de côres vivas a uma estampa.
Colorido sôbre marfim ou pergaminho.
\section{Illusão}
\begin{itemize}
\item {Grp. gram.:f.}
\end{itemize}
\begin{itemize}
\item {Utilização:Fig.}
\end{itemize}
\begin{itemize}
\item {Proveniência:(Lat. \textunderscore illusio\textunderscore )}
\end{itemize}
Engano dos sentidos ou da intelligência.
Errada interpretação de um facto.
Aquillo que dura pouco: \textunderscore a vida é uma illusão\textunderscore .
Zombaria.
Estado da alma, em que nós, dominados por um trabalho artístico, attribuimos realidade àquillo que sabemos não sêr verdadeiro: \textunderscore perder illusões\textunderscore .
\section{Illusionismo}
\begin{itemize}
\item {Grp. gram.:m.}
\end{itemize}
\begin{itemize}
\item {Proveniência:(Do lat. \textunderscore illusio\textunderscore )}
\end{itemize}
O mesmo que \textunderscore prestidigitação\textunderscore .
\section{Illusionista}
\begin{itemize}
\item {Grp. gram.:m.}
\end{itemize}
O mesmo que \textunderscore prestidigitador\textunderscore .
(Cp. \textunderscore illusionismo\textunderscore )
\section{Illusir}
\begin{itemize}
\item {Grp. gram.:v. t.}
\end{itemize}
\begin{itemize}
\item {Utilização:Prov.}
\end{itemize}
\begin{itemize}
\item {Utilização:trasm.}
\end{itemize}
\begin{itemize}
\item {Proveniência:(De \textunderscore illuso\textunderscore )}
\end{itemize}
O mesmo que \textunderscore illudir\textunderscore .
\section{Illusivo}
\begin{itemize}
\item {Grp. gram.:adj.}
\end{itemize}
O mesmo que \textunderscore illusório\textunderscore .
\section{Illuso}
\begin{itemize}
\item {Grp. gram.:adj.}
\end{itemize}
\begin{itemize}
\item {Proveniência:(Lat. \textunderscore illusus\textunderscore )}
\end{itemize}
O mesmo que [[illudido|illudir]]:«\textunderscore andais illuso do demónio\textunderscore ». \textunderscore Luz e Calor\textunderscore , 154 e 219. Cf. Castilho, \textunderscore Tartufo\textunderscore , 183.
\section{Illusor}
\begin{itemize}
\item {Grp. gram.:m.  e  adj.}
\end{itemize}
\begin{itemize}
\item {Proveniência:(Lat. \textunderscore illusor\textunderscore )}
\end{itemize}
O que illude.
\section{Illusoriamente}
\begin{itemize}
\item {Grp. gram.:adv.}
\end{itemize}
De modo illusório.
\section{Ilírico}
\begin{itemize}
\item {Grp. gram.:m.}
\end{itemize}
\begin{itemize}
\item {Proveniência:(De \textunderscore Ilíria\textunderscore , n. p.)}
\end{itemize}
Um dos idiomas esclavónicos.
\section{Illusório}
\begin{itemize}
\item {Grp. gram.:adj.}
\end{itemize}
\begin{itemize}
\item {Proveniência:(Lat. \textunderscore illusorius\textunderscore )}
\end{itemize}
Que produz illusão.
Enganoso; vão.
\section{Illustração}
\begin{itemize}
\item {Grp. gram.:f.}
\end{itemize}
\begin{itemize}
\item {Proveniência:(Lat. \textunderscore illustratio\textunderscore )}
\end{itemize}
Acto ou effeito de illustrar.
Conjunto de conhecimentos.
Sabedoria.
Publicação periódica com estampas: \textunderscore em Lisbôa publicam-se poucas illustrações\textunderscore .
\section{Illustradamente}
\begin{itemize}
\item {Grp. gram.:adv.}
\end{itemize}
De modo illustrado.
Á maneira de gente illustrada.
\section{Illustrado}
\begin{itemize}
\item {Grp. gram.:adj.}
\end{itemize}
\begin{itemize}
\item {Utilização:Neol.}
\end{itemize}
\begin{itemize}
\item {Proveniência:(De \textunderscore illustrar\textunderscore )}
\end{itemize}
Que tem muita instrucção: \textunderscore homem illustrado\textunderscore .
Que tem gravuras ou desenhos: \textunderscore jornal illustrado\textunderscore .
\section{Illustrador}
\begin{itemize}
\item {Grp. gram.:m.  e  adj.}
\end{itemize}
\begin{itemize}
\item {Proveniência:(Lat. \textunderscore illustrator\textunderscore )}
\end{itemize}
O que illustra.
\section{Illustrar}
\begin{itemize}
\item {Grp. gram.:v. t.}
\end{itemize}
\begin{itemize}
\item {Proveniência:(Lat. \textunderscore illustrare\textunderscore )}
\end{itemize}
Tornar illustre ou célebre.
Instruir, esclarecer.
Ornar com gravuras, estampas, desenhos, (um trabalho impresso, ou destinado á imprensa).
\section{Illustrativo}
\begin{itemize}
\item {Grp. gram.:adj.}
\end{itemize}
\begin{itemize}
\item {Proveniência:(Do lat. \textunderscore illustratus\textunderscore )}
\end{itemize}
Que illustra.
Que esclarece, que elucida.
\section{Illustre}
\begin{itemize}
\item {Grp. gram.:adj.}
\end{itemize}
\begin{itemize}
\item {Proveniência:(Lat. \textunderscore illustris\textunderscore )}
\end{itemize}
Que brilha ou se distingue por qualidades louváveis, (falando-se de pessôa).
Célebre.
Nobre.
Esclarecido; distinto; notável: \textunderscore o illustre professor\textunderscore .
\section{Illustríssimo}
\begin{itemize}
\item {Grp. gram.:adj.}
\end{itemize}
\begin{itemize}
\item {Proveniência:(De \textunderscore illustre\textunderscore )}
\end{itemize}
Muito illustre.
Tratamento ceremonioso, que se dá a pessôas de certa consideração, em cartas principalmente.
\section{Illutação}
\begin{itemize}
\item {Grp. gram.:f.}
\end{itemize}
\begin{itemize}
\item {Proveniência:(Do lat. \textunderscore in\textunderscore  + \textunderscore lutum\textunderscore )}
\end{itemize}
Acto de cobrir de lodo uma parte do corpo, para fins therapêuticos.
\section{Illýrico}
\begin{itemize}
\item {Grp. gram.:m.}
\end{itemize}
\begin{itemize}
\item {Proveniência:(De \textunderscore Illýria\textunderscore , n. p.)}
\end{itemize}
Um dos idiomas esclavónicos.
\section{Ilocano}
\begin{itemize}
\item {Grp. gram.:m.}
\end{itemize}
Um dos dialectos das Filipinas.
\section{Ilocável}
\begin{itemize}
\item {Grp. gram.:adj.}
\end{itemize}
\begin{itemize}
\item {Proveniência:(Lat. \textunderscore illocabilis\textunderscore )}
\end{itemize}
Que se não póde colocar.
Que não ocupa lugar.
\section{Ilogicamente}
\begin{itemize}
\item {Grp. gram.:adv.}
\end{itemize}
De modo ilógico.
\section{Ilógico}
\begin{itemize}
\item {Grp. gram.:adj.}
\end{itemize}
\begin{itemize}
\item {Proveniência:(De \textunderscore in...\textunderscore  + \textunderscore lógico\textunderscore )}
\end{itemize}
Que não é logico.
Absurdo; incoerente.
\section{Ilogismo}
\begin{itemize}
\item {Grp. gram.:m.}
\end{itemize}
\begin{itemize}
\item {Proveniência:(De \textunderscore ilógico\textunderscore )}
\end{itemize}
Falta de lógica.
\section{Ilota}
\begin{itemize}
\item {Grp. gram.:m.}
\end{itemize}
\begin{itemize}
\item {Utilização:Fig.}
\end{itemize}
\begin{itemize}
\item {Grp. gram.:Adj.}
\end{itemize}
\begin{itemize}
\item {Proveniência:(Do gr. \textunderscore heilotes\textunderscore )}
\end{itemize}
Escravo, que em Esparta cultivava o campo de seu senhor.
Pessôa da mais baixa condição social.
Próprio do indivíduo inculto:«\textunderscore oh que rudeza ilota!\textunderscore »Castilho, \textunderscore Sabichonas\textunderscore , 74.
\section{Ilote}
\begin{itemize}
\item {Grp. gram.:m.}
\end{itemize}
O mesmo que \textunderscore ilota\textunderscore .
\section{Ilotismo}
\begin{itemize}
\item {Grp. gram.:m.}
\end{itemize}
Condição ou qualidade de ilota.
\section{Iludente}
\begin{itemize}
\item {Grp. gram.:adj.}
\end{itemize}
\begin{itemize}
\item {Proveniência:(Lat. \textunderscore illudens\textunderscore )}
\end{itemize}
Que ilude.
\section{Iludir}
\begin{itemize}
\item {Grp. gram.:v. t.}
\end{itemize}
\begin{itemize}
\item {Proveniência:(Lat. \textunderscore illudere\textunderscore )}
\end{itemize}
Causar ilusão a.
Enganar.
Lograr; burlar.
Sofismar.
\section{Iludivel}
\begin{itemize}
\item {Grp. gram.:adj.}
\end{itemize}
\begin{itemize}
\item {Proveniência:(De \textunderscore iludir\textunderscore )}
\end{itemize}
Que póde sêr iludido; que se póde induzir em êrro.
Em que póde haver ilusão.
\section{Iludivelmente}
\begin{itemize}
\item {Grp. gram.:adv.}
\end{itemize}
De modo iludivel.
\section{Iluminação}
\begin{itemize}
\item {Grp. gram.:f.}
\end{itemize}
\begin{itemize}
\item {Proveniência:(Lat. \textunderscore illuminatio\textunderscore )}
\end{itemize}
Acto ou efeito de iluminar.
Irradiação.
Conjunto de luzes: \textunderscore iluminação a gás\textunderscore .
Estado daquilo que é iluminado.
\section{Iluminado}
\begin{itemize}
\item {Grp. gram.:adj.}
\end{itemize}
\begin{itemize}
\item {Grp. gram.:M.}
\end{itemize}
\begin{itemize}
\item {Proveniência:(De \textunderscore iluminar\textunderscore )}
\end{itemize}
Que tem iluminuras.
Colorido.
Que recebe luz: \textunderscore rua iluminada\textunderscore .
Instruido.
Sectário do iluminismo.
Visionário em questões religiosas.
Membro de algumas seitas religiosas.
\section{Iluminador}
\begin{itemize}
\item {Grp. gram.:adj.}
\end{itemize}
\begin{itemize}
\item {Grp. gram.:M.}
\end{itemize}
\begin{itemize}
\item {Proveniência:(Lat. \textunderscore illuminator\textunderscore )}
\end{itemize}
Que ilumina.
Aquele que ilumina; aquele que faz iluminuras.
\section{Iluminadura}
\begin{itemize}
\item {Grp. gram.:f.}
\end{itemize}
O mesmo que \textunderscore iluminação\textunderscore .
\section{Iluminante}
\begin{itemize}
\item {Grp. gram.:adj.}
\end{itemize}
\begin{itemize}
\item {Proveniência:(Lat. \textunderscore illuminans\textunderscore )}
\end{itemize}
Que ilumina.
\section{Iluminar}
\begin{itemize}
\item {Grp. gram.:v. t.}
\end{itemize}
\begin{itemize}
\item {Utilização:Fig.}
\end{itemize}
\begin{itemize}
\item {Proveniência:(Lat. \textunderscore illuminare\textunderscore )}
\end{itemize}
Tornar luminoso, claro.
Difundir luz em ou sôbre.
Abrilhantar.
Esclarecer.
Ensinar.
Ilustrar.
Adornar; pintar com iluminuras.
\section{Iluminativo}
\begin{itemize}
\item {Grp. gram.:adj.}
\end{itemize}
O mesmo que \textunderscore iluminante\textunderscore .
\section{Iluminismo}
\begin{itemize}
\item {Grp. gram.:m.}
\end{itemize}
\begin{itemize}
\item {Proveniência:(De \textunderscore iluminar\textunderscore )}
\end{itemize}
Conjunto de opiniões, preconizadas no seculo XVIII, sôbre a existência de uma inspiração sobrenatural.
\section{Iluminista}
\begin{itemize}
\item {Grp. gram.:m.}
\end{itemize}
\begin{itemize}
\item {Proveniência:(De \textunderscore iluminar\textunderscore )}
\end{itemize}
Partidário do iluminismo.
\section{Iluminura}
\begin{itemize}
\item {Grp. gram.:f.}
\end{itemize}
\begin{itemize}
\item {Proveniência:(De \textunderscore iluminar\textunderscore )}
\end{itemize}
Pintura a côres, nos livros e manuscritos da Idade-Média.
Aplicação de côres vivas a uma estampa.
Colorido sôbre marfim ou pergaminho.
\section{Ilusão}
\begin{itemize}
\item {Grp. gram.:f.}
\end{itemize}
\begin{itemize}
\item {Utilização:Fig.}
\end{itemize}
\begin{itemize}
\item {Proveniência:(Lat. \textunderscore illusio\textunderscore )}
\end{itemize}
Engano dos sentidos ou da inteligência.
Errada interpretação de um facto.
Aquilo que dura pouco: \textunderscore a vida é uma ilusão\textunderscore .
Zombaria.
Estado da alma, em que nós, dominados por um trabalho artístico, atribuimos realidade àquilo que sabemos não sêr verdadeiro: \textunderscore perder ilusões\textunderscore .
\section{Ilusionismo}
\begin{itemize}
\item {Grp. gram.:m.}
\end{itemize}
\begin{itemize}
\item {Proveniência:(Do lat. \textunderscore illusio\textunderscore )}
\end{itemize}
O mesmo que \textunderscore prestidigitação\textunderscore .
\section{Ilusionista}
\begin{itemize}
\item {Grp. gram.:m.}
\end{itemize}
O mesmo que \textunderscore prestidigitador\textunderscore .
(Cp. \textunderscore illusionismo\textunderscore )
\section{Ilusir}
\begin{itemize}
\item {Grp. gram.:v. t.}
\end{itemize}
\begin{itemize}
\item {Utilização:Prov.}
\end{itemize}
\begin{itemize}
\item {Utilização:trasm.}
\end{itemize}
\begin{itemize}
\item {Proveniência:(De \textunderscore iluso\textunderscore )}
\end{itemize}
O mesmo que \textunderscore iludir\textunderscore .
\section{Ilusivo}
\begin{itemize}
\item {Grp. gram.:adj.}
\end{itemize}
O mesmo que \textunderscore ilusório\textunderscore .
\section{Iluso}
\begin{itemize}
\item {Grp. gram.:adj.}
\end{itemize}
\begin{itemize}
\item {Proveniência:(Lat. \textunderscore illusus\textunderscore )}
\end{itemize}
O mesmo que [[iludido|iludir]]:«\textunderscore andais iluso do demónio\textunderscore ». \textunderscore Luz e Calor\textunderscore , 154 e 219. Cf. Castilho, \textunderscore Tartufo\textunderscore , 183.
\section{Ilusor}
\begin{itemize}
\item {Grp. gram.:m.  e  adj.}
\end{itemize}
\begin{itemize}
\item {Proveniência:(Lat. \textunderscore illusor\textunderscore )}
\end{itemize}
O que ilude.
\section{Ilusoriamente}
\begin{itemize}
\item {Grp. gram.:adv.}
\end{itemize}
De modo ilusório.
\section{Ilusório}
\begin{itemize}
\item {Grp. gram.:adj.}
\end{itemize}
\begin{itemize}
\item {Proveniência:(Lat. \textunderscore illusorius\textunderscore )}
\end{itemize}
Que produz ilusão.
Enganoso; vão.
\section{Ilustração}
\begin{itemize}
\item {Grp. gram.:f.}
\end{itemize}
\begin{itemize}
\item {Proveniência:(Lat. \textunderscore illustratio\textunderscore )}
\end{itemize}
Acto ou efeito de ilustrar.
Conjunto de conhecimentos.
Sabedoria.
Publicação periódica com estampas: \textunderscore em Lisbôa publicam-se poucas ilustrações\textunderscore .
\section{Ilustradamente}
\begin{itemize}
\item {Grp. gram.:adv.}
\end{itemize}
De modo ilustrado.
Á maneira de gente ilustrada.
\section{Ilustrado}
\begin{itemize}
\item {Grp. gram.:adj.}
\end{itemize}
\begin{itemize}
\item {Utilização:Neol.}
\end{itemize}
\begin{itemize}
\item {Proveniência:(De \textunderscore ilustrar\textunderscore )}
\end{itemize}
Que tem muita instrucção: \textunderscore homem ilustrado\textunderscore .
Que tem gravuras ou desenhos: \textunderscore jornal ilustrado\textunderscore .
\section{Ilustrador}
\begin{itemize}
\item {Grp. gram.:m.  e  adj.}
\end{itemize}
\begin{itemize}
\item {Proveniência:(Lat. \textunderscore illustrator\textunderscore )}
\end{itemize}
O que ilustra.
\section{Ilustrar}
\begin{itemize}
\item {Grp. gram.:v. t.}
\end{itemize}
\begin{itemize}
\item {Proveniência:(Lat. \textunderscore illustrare\textunderscore )}
\end{itemize}
Tornar ilustre ou célebre.
Instruir, esclarecer.
Ornar com gravuras, estampas, desenhos, (um trabalho impresso, ou destinado á imprensa).
\section{Ilustrativo}
\begin{itemize}
\item {Grp. gram.:adj.}
\end{itemize}
\begin{itemize}
\item {Proveniência:(Do lat. \textunderscore illustratus\textunderscore )}
\end{itemize}
Que ilustra.
Que esclarece, que elucida.
\section{Ilustre}
\begin{itemize}
\item {Grp. gram.:adj.}
\end{itemize}
\begin{itemize}
\item {Proveniência:(Lat. \textunderscore illustris\textunderscore )}
\end{itemize}
Que brilha ou se distingue por qualidades louváveis, (falando-se de pessôa).
Célebre.
Nobre.
Esclarecido; distinto; notável: \textunderscore o ilustre professor\textunderscore .
\section{Ilustríssimo}
\begin{itemize}
\item {Grp. gram.:adj.}
\end{itemize}
\begin{itemize}
\item {Proveniência:(De \textunderscore ilustre\textunderscore )}
\end{itemize}
Muito ilustre.
Tratamento ceremonioso, que se dá a pessôas de certa consideração, em cartas principalmente.
\section{Ilutação}
\begin{itemize}
\item {Grp. gram.:f.}
\end{itemize}
\begin{itemize}
\item {Proveniência:(Do lat. \textunderscore in\textunderscore  + \textunderscore lutum\textunderscore )}
\end{itemize}
Acto de cobrir de lodo uma parte do corpo, para fins terapêuticos.
\section{Im...}
\begin{itemize}
\item {Grp. gram.:pref.}
\end{itemize}
(usado, em logar de \textunderscore in...\textunderscore , antes dos radicaes que começam por \textunderscore m\textunderscore , \textunderscore b\textunderscore  ou \textunderscore p\textunderscore )
\section{Imã}
\begin{itemize}
\item {Grp. gram.:m.}
\end{itemize}
\begin{itemize}
\item {Utilização:Fig.}
\end{itemize}
\begin{itemize}
\item {Proveniência:(Do fr. \textunderscore aimant\textunderscore )}
\end{itemize}
Magnete ou ferro magnético.
Qualidade daquillo que attrai.
\section{Imã}
\begin{itemize}
\item {Grp. gram.:m.}
\end{itemize}
\begin{itemize}
\item {Proveniência:(Do ár. \textunderscore imam\textunderscore )}
\end{itemize}
Sacerdote muçulmano.
Título de alguns chefes, em povos independentes da Arábia.
\section{Imaculabilidade}
\begin{itemize}
\item {Grp. gram.:f.}
\end{itemize}
Qualidade daquele ou daquilo que é imaculável.
\section{Imaculadidade}
\begin{itemize}
\item {Grp. gram.:f.}
\end{itemize}
Qualidade de imaculado:«\textunderscore extravagâncias dogmáticas da imaculadidade e da infalibilidade.\textunderscore »Herculano.
\section{Imaculado}
\begin{itemize}
\item {Grp. gram.:adj.}
\end{itemize}
\begin{itemize}
\item {Proveniência:(Lat. \textunderscore immaculatus\textunderscore )}
\end{itemize}
Que não tem mácula; puro; inocente.
\section{Imaculatismo}
\begin{itemize}
\item {Grp. gram.:m.}
\end{itemize}
\begin{itemize}
\item {Proveniência:(Do lat. \textunderscore immaculatus\textunderscore )}
\end{itemize}
Doutrina religiosa da Conceição imaculada. Cf. Herculano, \textunderscore Quest. Públ.\textunderscore , I, 269.
\section{Imaculável}
\begin{itemize}
\item {Grp. gram.:adj.}
\end{itemize}
\begin{itemize}
\item {Proveniência:(Lat. \textunderscore immaculabilis\textunderscore )}
\end{itemize}
Que se não póde macular.
Impeccável.
Que não é susceptível de mancha ou culpa.
\section{Imagem}
\begin{itemize}
\item {Grp. gram.:f.}
\end{itemize}
\begin{itemize}
\item {Utilização:Fam.}
\end{itemize}
\begin{itemize}
\item {Proveniência:(Lat. \textunderscore imago\textunderscore )}
\end{itemize}
Aquillo que imita pessôa ou coisa.
Representação por desenho, gravura ou esculptura.
Semelhança.
Representação.
Reflexo de um objecto na água, num espelho, etc.
Reproducção na memória.
Sýmbolo.
Impressão de um objecto no espírito.
Estampa, que representa assunto religioso.
Estampa ou esculptura, que representa divindade fabulosa, ou personagem santificada entre os Christãos.
Descripção.
Reproducção, por meio de phenómenos luminosos.
Pessôa formosa.
\section{Imaginação}
\begin{itemize}
\item {Grp. gram.:f.}
\end{itemize}
\begin{itemize}
\item {Proveniência:(Lat. \textunderscore imaginatio\textunderscore )}
\end{itemize}
Faculdade de imaginar.
Faculdade de conceber ou criar, em literatura, representando vivamente as concepções.
Fantasia.
Coisa imaginada.
Superstição.
Crença ou opinião, procedente só da fantasia.
Apprehensão; scisma.
\section{Imaginador}
\begin{itemize}
\item {Grp. gram.:adj.}
\end{itemize}
\begin{itemize}
\item {Grp. gram.:M.}
\end{itemize}
\begin{itemize}
\item {Utilização:Ant.}
\end{itemize}
Que imagina.
Aquelle que imagina.
Aquelle que faz imagens de santos; santeiro. Cf. Fuschini, \textunderscore Architect. Rel.\textunderscore , 288.
\section{Imaginante}
\begin{itemize}
\item {Grp. gram.:adj.}
\end{itemize}
\begin{itemize}
\item {Proveniência:(Lat. \textunderscore imaginans\textunderscore )}
\end{itemize}
Que imagina.
\section{Imaginar}
\begin{itemize}
\item {Grp. gram.:v. t.}
\end{itemize}
\begin{itemize}
\item {Grp. gram.:V. i.}
\end{itemize}
\begin{itemize}
\item {Proveniência:(Lat. \textunderscore imaginari\textunderscore )}
\end{itemize}
Representar no espirito.
Fantasiar.
Criar na imaginação.
Idear.
Inventar.
Conjecturar: \textunderscore ninguém imagina o que tu és\textunderscore .
Têr scismas, apprehensões.
\section{Imaginária}
\begin{itemize}
\item {Grp. gram.:f.}
\end{itemize}
\begin{itemize}
\item {Utilização:Des.}
\end{itemize}
O mesmo que \textunderscore estatuária\textunderscore .
Figuras humanas, bordadas ou pintadas. Cf. Sousa, \textunderscore Vida do Arceb.\textunderscore , III, 257.
(Cp. \textunderscore imaginário\textunderscore )
\section{Imaginário}
\begin{itemize}
\item {Grp. gram.:adj.}
\end{itemize}
\begin{itemize}
\item {Utilização:Mathem.}
\end{itemize}
\begin{itemize}
\item {Grp. gram.:M.}
\end{itemize}
\begin{itemize}
\item {Utilização:Des.}
\end{itemize}
\begin{itemize}
\item {Proveniência:(Lat. \textunderscore imaginarius\textunderscore )}
\end{itemize}
Que está só na imaginação, que não é real.
Illusório.
Phantástico.
Indefinido.
Diz-se de uma quantidade irracional, em que o radical affecta uma quantidade negativa.
Aquelle que faz estátuas; santeiro.
\section{Imaginativa}
\begin{itemize}
\item {Grp. gram.:f.}
\end{itemize}
\begin{itemize}
\item {Proveniência:(De \textunderscore imaginativo\textunderscore )}
\end{itemize}
Faculdade de imaginar.
\section{Imaginativo}
\begin{itemize}
\item {Grp. gram.:adj.}
\end{itemize}
\begin{itemize}
\item {Utilização:Fig.}
\end{itemize}
\begin{itemize}
\item {Proveniência:(De \textunderscore imaginar\textunderscore )}
\end{itemize}
Que tem a imaginação fértil; que imagina com facilidade.
Misanthrópico, apprehensivo.
\section{Imaginável}
\begin{itemize}
\item {Grp. gram.:adj.}
\end{itemize}
\begin{itemize}
\item {Proveniência:(Lat. \textunderscore imaginabilis\textunderscore )}
\end{itemize}
Que se póde imaginar.
\section{Imaginoso}
\begin{itemize}
\item {Grp. gram.:adj.}
\end{itemize}
\begin{itemize}
\item {Proveniência:(De \textunderscore imaginar\textunderscore )}
\end{itemize}
Que revela imaginação ou faculdades inventivas: \textunderscore theorias imaginosas\textunderscore .
Imaginário; phantástico.
\section{Imaleabilidade}
\begin{itemize}
\item {Grp. gram.:f.}
\end{itemize}
Qualidade daquilo que é imaleável.
\section{Imaleável}
\begin{itemize}
\item {Grp. gram.:adj.}
\end{itemize}
\begin{itemize}
\item {Proveniência:(De \textunderscore im...\textunderscore  + \textunderscore maleável\textunderscore )}
\end{itemize}
Que não é maleável.
\section{Iman}
\begin{itemize}
\item {Grp. gram.:m.}
\end{itemize}
\begin{itemize}
\item {Utilização:Fig.}
\end{itemize}
\begin{itemize}
\item {Proveniência:(Do fr. \textunderscore aimant\textunderscore )}
\end{itemize}
Magnete ou ferro magnético.
Qualidade daquillo que attrai.
\section{Iman}
\begin{itemize}
\item {Grp. gram.:m.}
\end{itemize}
\begin{itemize}
\item {Proveniência:(Do ár. \textunderscore imam\textunderscore )}
\end{itemize}
Sacerdote muçulmano.
Título de alguns chefes, em povos independentes da Arábia.
\section{Imanado}
\begin{itemize}
\item {Grp. gram.:m.}
\end{itemize}
Dignidade do íman^2.
Território governado por um íman^2.
\section{Imanar}
\begin{itemize}
\item {Grp. gram.:v. t.}
\end{itemize}
\begin{itemize}
\item {Proveniência:(De \textunderscore íman\textunderscore ^1)}
\end{itemize}
Magnetizar, communicar propriedades magnéticas a um corpo.
\section{Imane}
\begin{itemize}
\item {Grp. gram.:adj.}
\end{itemize}
\begin{itemize}
\item {Utilização:Fig.}
\end{itemize}
\begin{itemize}
\item {Proveniência:(Lat. \textunderscore immanis\textunderscore )}
\end{itemize}
Muito grande; desmedido.
Feroz.
\section{Imanência}
\begin{itemize}
\item {Grp. gram.:f.}
\end{itemize}
Qualidade daquilo que é imanente; permanência; persistência.
\section{Imanente}
\begin{itemize}
\item {Grp. gram.:adj.}
\end{itemize}
\begin{itemize}
\item {Proveniência:(Lat. \textunderscore immanens\textunderscore )}
\end{itemize}
Perdurável; permanente.
Privativo de um sujeito ou objecto.
\section{Imanentemente}
\begin{itemize}
\item {Grp. gram.:adv.}
\end{itemize}
De modo imanente.
\section{Imanidade}
\begin{itemize}
\item {Grp. gram.:f.}
\end{itemize}
\begin{itemize}
\item {Proveniência:(Lat. \textunderscore immanitas\textunderscore )}
\end{itemize}
Qualidade daquilo que é imane.
\section{Imano}
\begin{itemize}
\item {Grp. gram.:adj.}
\end{itemize}
(V.imane)
\section{Imarcescibilidade}
\begin{itemize}
\item {Grp. gram.:f.}
\end{itemize}
Qualidade daquilo que é imarcescível.
\section{Imarcescível}
\begin{itemize}
\item {Grp. gram.:adj.}
\end{itemize}
\begin{itemize}
\item {Proveniência:(Lat. \textunderscore immarcescibilis\textunderscore )}
\end{itemize}
Que não murcha.
Incorruptível.
\section{Imarginado}
\begin{itemize}
\item {Grp. gram.:adj.}
\end{itemize}
\begin{itemize}
\item {Utilização:Bot.}
\end{itemize}
\begin{itemize}
\item {Proveniência:(De \textunderscore im...\textunderscore  + \textunderscore marginado\textunderscore )}
\end{itemize}
Que não tem margens ou bordos, (falando-se de certas sementes, líchens, etc.).
\section{Imaterial}
\begin{itemize}
\item {Grp. gram.:adj.}
\end{itemize}
\begin{itemize}
\item {Grp. gram.:M.}
\end{itemize}
\begin{itemize}
\item {Proveniência:(De \textunderscore im...\textunderscore  + \textunderscore material\textunderscore )}
\end{itemize}
Que não é material; que não tem a natureza da matéria.
Que é impalpável.
Aquilo que é immaterial.
\section{Imaterialidade}
\begin{itemize}
\item {Grp. gram.:f.}
\end{itemize}
Qualidade daquilo que é imaterial.
\section{Imaterialismo}
\begin{itemize}
\item {Grp. gram.:m.}
\end{itemize}
\begin{itemize}
\item {Proveniência:(De \textunderscore imaterial\textunderscore )}
\end{itemize}
Sistema dos que negam a existência da matéria.
\section{Imaterialista}
\begin{itemize}
\item {Grp. gram.:m.}
\end{itemize}
\begin{itemize}
\item {Proveniência:(De \textunderscore imaterial\textunderscore )}
\end{itemize}
Sectário do imaterialismo.
\section{Imantócera}
\begin{itemize}
\item {Grp. gram.:f.}
\end{itemize}
\begin{itemize}
\item {Proveniência:(Do gr. \textunderscore imas\textunderscore , \textunderscore imantos\textunderscore  + \textunderscore keras\textunderscore )}
\end{itemize}
Gênero de coleópteros tetrâmeros.
\section{Imaturidade}
\begin{itemize}
\item {Grp. gram.:f.}
\end{itemize}
\begin{itemize}
\item {Proveniência:(Lat. \textunderscore immaturitas\textunderscore )}
\end{itemize}
Qualidade daquilo que é imaturo.
\section{Imaturo}
\begin{itemize}
\item {Grp. gram.:adj.}
\end{itemize}
\begin{itemize}
\item {Proveniência:(Lat. \textunderscore immaturus\textunderscore )}
\end{itemize}
Que não é maduro.
Prematuro.
Precoce; temperão; antecipado.
\section{Imbaíba}
\begin{itemize}
\item {Grp. gram.:f.}
\end{itemize}
(V.umbaúba)
\section{Imbangalas}
\begin{itemize}
\item {Grp. gram.:m. pl.}
\end{itemize}
O mesmo que \textunderscore cassanges\textunderscore .
\section{Imbanteque}
\begin{itemize}
\item {Grp. gram.:m.}
\end{itemize}
Ave palmípede africana.
\section{Imbé}
\begin{itemize}
\item {Grp. gram.:m.}
\end{itemize}
\begin{itemize}
\item {Utilização:Bras}
\end{itemize}
Nome de algumas plantas, da fam. das aráceas.
\section{Imbecil}
\begin{itemize}
\item {Grp. gram.:adj.}
\end{itemize}
\begin{itemize}
\item {Utilização:Fig.}
\end{itemize}
\begin{itemize}
\item {Proveniência:(Lat. \textunderscore imbecillis\textunderscore )}
\end{itemize}
Fraco de corpo e de espírito. Cf. Sousa, \textunderscore Vidas do Arceb.\textunderscore , I, 17.
Lânguido.
Parvo.
Que revela tolice ou fraqueza de espirito: \textunderscore um dito imbecil\textunderscore .
Cobarde.
\section{Imbecilidade}
\begin{itemize}
\item {Grp. gram.:f.}
\end{itemize}
\begin{itemize}
\item {Proveniência:(Lat. \textunderscore imbecillitas\textunderscore )}
\end{itemize}
Qualidade de imbecil.
\section{Imbecilismo}
\begin{itemize}
\item {Grp. gram.:m.}
\end{itemize}
O mesmo que \textunderscore imbecilidade\textunderscore . Cf. Camillo, \textunderscore Bohêmia\textunderscore , 261.
\section{Imbecilitar}
\begin{itemize}
\item {Grp. gram.:v. t.}
\end{itemize}
\begin{itemize}
\item {Utilização:Neol.}
\end{itemize}
Tornado imbecil ou idiota. Cf. Camillo, \textunderscore Brasileira\textunderscore , 75.
\section{Imbecilmente}
\begin{itemize}
\item {Grp. gram.:adv.}
\end{itemize}
De modo imbecil.
\section{Imbele}
\begin{itemize}
\item {Grp. gram.:adj.}
\end{itemize}
\begin{itemize}
\item {Utilização:Fig.}
\end{itemize}
\begin{itemize}
\item {Proveniência:(Lat. \textunderscore imbellis\textunderscore )}
\end{itemize}
Que não é belicoso.
Timido.
Débil.
Cobarde.
\section{Imbelicar}
\begin{itemize}
\item {Grp. gram.:v. i.}
\end{itemize}
\begin{itemize}
\item {Utilização:Prov.}
\end{itemize}
\begin{itemize}
\item {Utilização:minh.}
\end{itemize}
Dirigir provocações.
Contender.
(Corr. de \textunderscore implicar\textunderscore )
\section{Imbelle}
\begin{itemize}
\item {Grp. gram.:adj.}
\end{itemize}
\begin{itemize}
\item {Utilização:Fig.}
\end{itemize}
\begin{itemize}
\item {Proveniência:(Lat. \textunderscore imbellis\textunderscore )}
\end{itemize}
Que não é bellicoso.
Timido.
Débil.
Cobarde.
\section{Imberbe}
\begin{itemize}
\item {Grp. gram.:adj.}
\end{itemize}
\begin{itemize}
\item {Proveniência:(Lat. \textunderscore imberbis\textunderscore )}
\end{itemize}
Que não tem barbas; que ainda é moço.
\section{Imberi}
\begin{itemize}
\item {Grp. gram.:m.}
\end{itemize}
Planta canácea do Brasil.
\section{Imbibição}
\begin{itemize}
\item {Grp. gram.:f.}
\end{itemize}
\begin{itemize}
\item {Utilização:Bot.}
\end{itemize}
\begin{itemize}
\item {Proveniência:(Do lat. \textunderscore imbibere\textunderscore )}
\end{itemize}
Acto de se embeber de um liquido (qualquer corpo vegetal).
\section{Imbicar}
\begin{itemize}
\item {Grp. gram.:v. t.}
\end{itemize}
\begin{itemize}
\item {Utilização:Bras}
\end{itemize}
O mesmo que \textunderscore abicar\textunderscore , \textunderscore aportar\textunderscore .
\section{Imbila}
\begin{itemize}
\item {Grp. gram.:f.}
\end{itemize}
Árvore de Moçambique.
\section{Imbira}
\begin{itemize}
\item {Grp. gram.:f.}
\end{itemize}
(V.ibira)
\section{Imbiri}
\begin{itemize}
\item {Grp. gram.:m.}
\end{itemize}
Planta canácea, (\textunderscore canna glauca\textunderscore ).
\section{Imbiriçu}
\begin{itemize}
\item {Grp. gram.:m.}
\end{itemize}
Planta bombácea, (\textunderscore bombax hexaphillum\textunderscore ).
\section{Imbondeiro}
\begin{itemize}
\item {Grp. gram.:m.}
\end{itemize}
O mesmo que \textunderscore adansónia\textunderscore .
\section{Imbondo}
\begin{itemize}
\item {Grp. gram.:m.}
\end{itemize}
\begin{itemize}
\item {Utilização:Bras}
\end{itemize}
Obstáculo.
Difficuldade.
\section{Imbricação}
\begin{itemize}
\item {Grp. gram.:f.}
\end{itemize}
\begin{itemize}
\item {Proveniência:(De \textunderscore imbricar\textunderscore )}
\end{itemize}
Disposição de objectos, sobrepondo-se em parte uns aos outros, á maneira de telhas.
\section{Imbricado}
\begin{itemize}
\item {Grp. gram.:adj.}
\end{itemize}
\begin{itemize}
\item {Proveniência:(De \textunderscore imbricar\textunderscore )}
\end{itemize}
Disposto á maneira de escama ou de telhas que se sobrepõem a outras.
\section{Imbricante}
\begin{itemize}
\item {Grp. gram.:adj.}
\end{itemize}
\begin{itemize}
\item {Utilização:Bot.}
\end{itemize}
\begin{itemize}
\item {Proveniência:(Lat. \textunderscore imbricans\textunderscore )}
\end{itemize}
Diz-se das fôlhas, em que, como nas da sensitiva, os foliolos se dirigem todos para o ápice da fôlha e se applicam contra o ápice do peciolo, cobrindo umas ás outras.
\section{Imbricar}
\begin{itemize}
\item {Grp. gram.:v. t.}
\end{itemize}
\begin{itemize}
\item {Proveniência:(Lat. \textunderscore imbricare\textunderscore )}
\end{itemize}
Collocar em imbricação, tornar imbricado.
\section{Imbrífero}
\begin{itemize}
\item {Grp. gram.:adj.}
\end{itemize}
\begin{itemize}
\item {Utilização:Poét.}
\end{itemize}
\begin{itemize}
\item {Proveniência:(Lat. \textunderscore imbrifer\textunderscore )}
\end{itemize}
Que traz chuvas.
Que inunda.
\section{Imbrincado}
\begin{itemize}
\item {Grp. gram.:adj.}
\end{itemize}
Cheio de feitios bonitos e caprichosos. Us. por Camillo.
(Talvez por \textunderscore imbricado\textunderscore , sob a infl. de \textunderscore brinco\textunderscore )
\section{Imbróglio}
\begin{itemize}
\item {Grp. gram.:m.}
\end{itemize}
\begin{itemize}
\item {Utilização:Fam.}
\end{itemize}
\begin{itemize}
\item {Proveniência:(It. \textunderscore imbroglio\textunderscore )}
\end{itemize}
Trapalhada, confusão.
\section{Imbu}
\begin{itemize}
\item {Grp. gram.:m.}
\end{itemize}
Fruto do imbuzeiro.
\section{Imbuá}
\begin{itemize}
\item {Grp. gram.:m.}
\end{itemize}
Pequeno verme do Brasil.
\section{Imbuia}
\begin{itemize}
\item {Grp. gram.:f.}
\end{itemize}
Madeira preciosa do Paraná, usada em marcenaria.
\section{Imbuir}
\begin{itemize}
\item {Grp. gram.:v. t.}
\end{itemize}
\begin{itemize}
\item {Utilização:Fig.}
\end{itemize}
\begin{itemize}
\item {Proveniência:(Lat. \textunderscore imbuere\textunderscore )}
\end{itemize}
Mergulhar em liquido.
Embeber.
Embutir; arreigar.
Suggerir.
\section{Imbúnde}
\begin{itemize}
\item {Grp. gram.:m.}
\end{itemize}
Planta herbácea, africana, cuja raiz tem matéria saccharina, que se aproveita para uma bebida refrigerante.
\section{Imbúndi}
\begin{itemize}
\item {Grp. gram.:m.}
\end{itemize}
Planta herbácea, africana, cuja raiz tem matéria saccharina, que se aproveita para uma bebida refrigerante.
\section{Imburana}
\begin{itemize}
\item {Grp. gram.:f.}
\end{itemize}
Árvore terebinthácea do Brasil.
\section{Imburguês}
\begin{itemize}
\item {Grp. gram.:m.}
\end{itemize}
\begin{itemize}
\item {Utilização:Açor}
\end{itemize}
Bigorrilhas; troca-tintas.
(Por \textunderscore hamburguês\textunderscore )
\section{Imburi}
\begin{itemize}
\item {Grp. gram.:adj.}
\end{itemize}
\begin{itemize}
\item {Utilização:Bras}
\end{itemize}
Variedade de coqueiro, (\textunderscore coccus canadensis\textunderscore ).
\section{Imbuzada}
\begin{itemize}
\item {Grp. gram.:f.}
\end{itemize}
\begin{itemize}
\item {Utilização:Bras. do N}
\end{itemize}
Iguaria, feita de leite, misturado com o sumo do imbu.
\section{Imbuzeiro}
\begin{itemize}
\item {Grp. gram.:m.}
\end{itemize}
Árvore terebinthácea da América, (\textunderscore spondias tuberosa\textunderscore ).
\section{Imediação}
\begin{itemize}
\item {Grp. gram.:f.}
\end{itemize}
\begin{itemize}
\item {Proveniência:(De \textunderscore im...\textunderscore  + \textunderscore mediação\textunderscore )}
\end{itemize}
Facto de sêr imediato.
Proximidade, vizinhança: \textunderscore nas imediações de Lisbôa\textunderscore .
\section{Imediatamente}
\begin{itemize}
\item {Grp. gram.:adv.}
\end{itemize}
De modo imediato; em seguida; logo.
\section{Imediato}
\begin{itemize}
\item {Grp. gram.:adj.}
\end{itemize}
\begin{itemize}
\item {Grp. gram.:M.}
\end{itemize}
\begin{itemize}
\item {Proveniência:(Lat. \textunderscore immediatus\textunderscore )}
\end{itemize}
Próximo; contíguo.
Que não tem nada de permeio.
Instantâneo: \textunderscore resposta imediata\textunderscore .
Que depende só de um superior.
Funcionário, cuja categoria fica logo abaixo da do chefe, em cuja falta êlle funciona: \textunderscore o imediato de um navio\textunderscore .
\section{Imedicável}
\begin{itemize}
\item {Grp. gram.:adj.}
\end{itemize}
\begin{itemize}
\item {Proveniência:(De \textunderscore im...\textunderscore  + \textunderscore medicável\textunderscore )}
\end{itemize}
Que se não póde medicar.
\section{Imemorado}
\begin{itemize}
\item {Grp. gram.:adj.}
\end{itemize}
\begin{itemize}
\item {Proveniência:(Lat. \textunderscore immemoratus\textunderscore )}
\end{itemize}
Que não foi memorado.
\section{Imemorável}
\begin{itemize}
\item {Grp. gram.:adj.}
\end{itemize}
\begin{itemize}
\item {Proveniência:(Lat. \textunderscore immemorabilis\textunderscore )}
\end{itemize}
Que se não deve ou não se póde memorar; imemorial.
\section{Imemoravelmente}
\begin{itemize}
\item {Grp. gram.:adv.}
\end{itemize}
De modo imemorável.
\section{Imémore}
\begin{itemize}
\item {Grp. gram.:adj.}
\end{itemize}
\begin{itemize}
\item {Utilização:Poét.}
\end{itemize}
\begin{itemize}
\item {Proveniência:(Lat. \textunderscore immemor\textunderscore )}
\end{itemize}
Que se não recorda; esquecido.
\section{Imemorial}
\begin{itemize}
\item {Grp. gram.:adj.}
\end{itemize}
\begin{itemize}
\item {Proveniência:(Do lat. \textunderscore immemoria\textunderscore )}
\end{itemize}
De que não há memória; antiquíssimo.
\section{Imemoriável}
\begin{itemize}
\item {Grp. gram.:adj.}
\end{itemize}
O mesmo que \textunderscore imemorial\textunderscore .
\section{Imensamente}
\begin{itemize}
\item {Grp. gram.:adv.}
\end{itemize}
De modo imenso; desmedidamente: \textunderscore imensamente rico\textunderscore .
Sem termo.
\section{Imensidade}
\begin{itemize}
\item {Grp. gram.:f.}
\end{itemize}
\begin{itemize}
\item {Proveniência:(Lat. \textunderscore immensitas\textunderscore )}
\end{itemize}
Qualidade daquilo que é imenso.
Extensão desmedida; espaço imenso, o infinito.
\section{Imensidão}
\begin{itemize}
\item {Grp. gram.:f.}
\end{itemize}
O mesmo que \textunderscore imensidade\textunderscore .
\section{Imenso}
\begin{itemize}
\item {Grp. gram.:adj.}
\end{itemize}
\begin{itemize}
\item {Proveniência:(Lat. \textunderscore immensus\textunderscore )}
\end{itemize}
Que se não póde medir; ilimitado.
Indefinível.
Infinito.
Numeroso: \textunderscore multidão imensa\textunderscore .
Enorme; muito extenso: \textunderscore mar imenso\textunderscore .
\section{Imido}
\begin{itemize}
\item {Grp. gram.:m.}
\end{itemize}
\begin{itemize}
\item {Utilização:Chím.}
\end{itemize}
Substituição, na molécula do amoníaco, de dois átomos de hydrogênio pelo radical diatómico de um ácido diatómico bibásico.--É um monamido secundário, de constituição particular.
\section{Imigo}
\begin{itemize}
\item {Grp. gram.:adj.}
\end{itemize}
\begin{itemize}
\item {Utilização:Poét.}
\end{itemize}
\begin{itemize}
\item {Proveniência:(Do lat. \textunderscore inimicus\textunderscore , de \textunderscore inimigo\textunderscore )}
\end{itemize}
O mesmo que \textunderscore inimigo\textunderscore .
\section{Imitação}
\begin{itemize}
\item {Grp. gram.:f.}
\end{itemize}
\begin{itemize}
\item {Proveniência:(Lat. \textunderscore imitatio\textunderscore )}
\end{itemize}
Acto ou effeito de imitar.
Obra literária, em que se procura imitar outra.
Contrafacção ou producto industrial, com que se procura imitar outro, sem intuito de lôgro.
\section{Imitador}
\begin{itemize}
\item {Grp. gram.:m.  e  adj.}
\end{itemize}
\begin{itemize}
\item {Proveniência:(Lat. \textunderscore imitator\textunderscore )}
\end{itemize}
O que imita.
\section{Imitante}
\begin{itemize}
\item {Grp. gram.:adj.}
\end{itemize}
\begin{itemize}
\item {Proveniência:(Lat. \textunderscore imitans\textunderscore )}
\end{itemize}
Que imita.
\section{Imitar}
\begin{itemize}
\item {Grp. gram.:v. t.}
\end{itemize}
\begin{itemize}
\item {Proveniência:(Lat. \textunderscore imitari\textunderscore )}
\end{itemize}
Procura reproduzir (o que outrem fez).
Tomar por modêlo.
Assemelhar-se a.
Arremedar: \textunderscore imitar a voz do melro\textunderscore .
Reproduzir.
Falsificar: \textunderscore imitar o vinho do Pôrto\textunderscore .
\section{Imitativo}
\begin{itemize}
\item {Grp. gram.:adj.}
\end{itemize}
\begin{itemize}
\item {Proveniência:(Lat. \textunderscore imitativus\textunderscore )}
\end{itemize}
Imitante.
\section{Imitável}
\begin{itemize}
\item {Grp. gram.:adj.}
\end{itemize}
\begin{itemize}
\item {Proveniência:(Lat. \textunderscore imitabilis\textunderscore )}
\end{itemize}
Que se póde ou se deve imitar.
\section{Imizade}
\begin{itemize}
\item {Grp. gram.:f.}
\end{itemize}
\begin{itemize}
\item {Utilização:Des.}
\end{itemize}
O mesmo que \textunderscore inimizade\textunderscore .
\section{Immaculabilidade}
\begin{itemize}
\item {Grp. gram.:f.}
\end{itemize}
Qualidade daquelle ou daquillo que é immaculável.
\section{Immaculadidade}
\begin{itemize}
\item {Grp. gram.:f.}
\end{itemize}
Qualidade de immaculado:«\textunderscore extravagâncias dogmáticas da immaculadidade e da infallibilidade.\textunderscore »Herculano.
\section{Immaculado}
\begin{itemize}
\item {Grp. gram.:adj.}
\end{itemize}
\begin{itemize}
\item {Proveniência:(Lat. \textunderscore immaculatus\textunderscore )}
\end{itemize}
Que não tem mácula; puro; innocente.
\section{Immaculatismo}
\begin{itemize}
\item {Grp. gram.:m.}
\end{itemize}
\begin{itemize}
\item {Proveniência:(Do lat. \textunderscore immaculatus\textunderscore )}
\end{itemize}
Doutrina religiosa da Conceição immaculada. Cf. Herculano, \textunderscore Quest. Públ.\textunderscore , I, 269.
\section{Immaculável}
\begin{itemize}
\item {Grp. gram.:adj.}
\end{itemize}
\begin{itemize}
\item {Proveniência:(Lat. \textunderscore immaculabilis\textunderscore )}
\end{itemize}
Que se não póde macular.
Impeccável.
Que não é susceptível de mancha ou culpa.
\section{Immalleabilidade}
\begin{itemize}
\item {Grp. gram.:f.}
\end{itemize}
Qualidade daquillo que é immalleável.
\section{Immalleável}
\begin{itemize}
\item {Grp. gram.:adj.}
\end{itemize}
\begin{itemize}
\item {Proveniência:(De \textunderscore im...\textunderscore  + \textunderscore malleável\textunderscore )}
\end{itemize}
Que não é malleável.
\section{Immane}
\begin{itemize}
\item {Grp. gram.:adj.}
\end{itemize}
\begin{itemize}
\item {Utilização:Fig.}
\end{itemize}
\begin{itemize}
\item {Proveniência:(Lat. \textunderscore immanis\textunderscore )}
\end{itemize}
Muito grande; desmedido.
Feroz.
\section{Immanência}
\begin{itemize}
\item {Grp. gram.:f.}
\end{itemize}
Qualidade daquillo que é immanente; permanência; persistência.
\section{Immanente}
\begin{itemize}
\item {Grp. gram.:adj.}
\end{itemize}
\begin{itemize}
\item {Proveniência:(Lat. \textunderscore immanens\textunderscore )}
\end{itemize}
Perdurável; permanente.
Privativo de um sujeito ou objecto.
\section{Immanentemente}
\begin{itemize}
\item {Grp. gram.:adv.}
\end{itemize}
De modo immanente.
\section{Immanidade}
\begin{itemize}
\item {Grp. gram.:f.}
\end{itemize}
\begin{itemize}
\item {Proveniência:(Lat. \textunderscore immanitas\textunderscore )}
\end{itemize}
Qualidade daquillo que é immane.
\section{Immano}
\begin{itemize}
\item {Grp. gram.:adj.}
\end{itemize}
(V.immane)
\section{Immarcescibilidade}
\begin{itemize}
\item {Grp. gram.:f.}
\end{itemize}
Qualidade daquillo que é immarcescível.
\section{Immarcescível}
\begin{itemize}
\item {Grp. gram.:adj.}
\end{itemize}
\begin{itemize}
\item {Proveniência:(Lat. \textunderscore immarcescibilis\textunderscore )}
\end{itemize}
Que não murcha.
Incorruptível.
\section{Immarginado}
\begin{itemize}
\item {Grp. gram.:adj.}
\end{itemize}
\begin{itemize}
\item {Utilização:Bot.}
\end{itemize}
\begin{itemize}
\item {Proveniência:(De \textunderscore im...\textunderscore  + \textunderscore marginado\textunderscore )}
\end{itemize}
Que não tem margens ou bordos, (falando-se de certas sementes, líchens, etc.).
\section{Immaterial}
\begin{itemize}
\item {Grp. gram.:adj.}
\end{itemize}
\begin{itemize}
\item {Grp. gram.:M.}
\end{itemize}
\begin{itemize}
\item {Proveniência:(De \textunderscore im...\textunderscore  + \textunderscore material\textunderscore )}
\end{itemize}
Que não é material; que não tem a natureza da matéria.
Que é impalpável.
Aquillo que é immaterial.
\section{Immaterialidade}
\begin{itemize}
\item {Grp. gram.:f.}
\end{itemize}
Qualidade daquillo que é immaterial.
\section{Immaterialismo}
\begin{itemize}
\item {Grp. gram.:m.}
\end{itemize}
\begin{itemize}
\item {Proveniência:(De \textunderscore immaterial\textunderscore )}
\end{itemize}
Systema dos que negam a existência da matéria.
\section{Immaterialista}
\begin{itemize}
\item {Grp. gram.:m.}
\end{itemize}
\begin{itemize}
\item {Proveniência:(De \textunderscore immaterial\textunderscore )}
\end{itemize}
Sectário do immaterialismo.
\section{Immaturidade}
\begin{itemize}
\item {Grp. gram.:f.}
\end{itemize}
\begin{itemize}
\item {Proveniência:(Lat. \textunderscore immaturitas\textunderscore )}
\end{itemize}
Qualidade daquillo que é immaturo.
\section{Immaturo}
\begin{itemize}
\item {Grp. gram.:adj.}
\end{itemize}
\begin{itemize}
\item {Proveniência:(Lat. \textunderscore immaturus\textunderscore )}
\end{itemize}
Que não é maduro.
Prematuro.
Precoce; temperão; antecipado.
\section{Immediação}
\begin{itemize}
\item {Grp. gram.:f.}
\end{itemize}
\begin{itemize}
\item {Proveniência:(De \textunderscore im...\textunderscore  + \textunderscore mediação\textunderscore )}
\end{itemize}
Facto de sêr immediato.
Proximidade, vizinhança: \textunderscore nas immediações de Lisbôa\textunderscore .
\section{Immediatamente}
\begin{itemize}
\item {Grp. gram.:adv.}
\end{itemize}
De modo immediato; em seguida; logo.
\section{Immediato}
\begin{itemize}
\item {Grp. gram.:adj.}
\end{itemize}
\begin{itemize}
\item {Grp. gram.:M.}
\end{itemize}
\begin{itemize}
\item {Proveniência:(Lat. \textunderscore immediatus\textunderscore )}
\end{itemize}
Próximo; contíguo.
Que não tem nada de permeio.
Instantâneo: \textunderscore resposta immediata\textunderscore .
Que depende só de um superior.
Funccionário, cuja categoria fica logo abaixo da do chefe, em cuja falta êlle funcciona: \textunderscore o immediato de um navio\textunderscore .
\section{Immedicável}
\begin{itemize}
\item {Grp. gram.:adj.}
\end{itemize}
\begin{itemize}
\item {Proveniência:(De \textunderscore im...\textunderscore  + \textunderscore medicável\textunderscore )}
\end{itemize}
Que se não póde medicar.
\section{Immemorado}
\begin{itemize}
\item {Grp. gram.:adj.}
\end{itemize}
\begin{itemize}
\item {Proveniência:(Lat. \textunderscore immemoratus\textunderscore )}
\end{itemize}
Que não foi memorado.
\section{Immemorável}
\begin{itemize}
\item {Grp. gram.:adj.}
\end{itemize}
\begin{itemize}
\item {Proveniência:(Lat. \textunderscore immemorabilis\textunderscore )}
\end{itemize}
Que se não deve ou não se póde memorar; immemorial.
\section{Immemoravelmente}
\begin{itemize}
\item {Grp. gram.:adv.}
\end{itemize}
De modo immemorável.
\section{Immémore}
\begin{itemize}
\item {Grp. gram.:adj.}
\end{itemize}
\begin{itemize}
\item {Utilização:Poét.}
\end{itemize}
\begin{itemize}
\item {Proveniência:(Lat. \textunderscore immemor\textunderscore )}
\end{itemize}
Que se não recorda; esquecido.
\section{Immemorial}
\begin{itemize}
\item {Grp. gram.:adj.}
\end{itemize}
\begin{itemize}
\item {Proveniência:(Do lat. \textunderscore immemoria\textunderscore )}
\end{itemize}
De que não há memória; antiquíssimo.
\section{Immemoriável}
\begin{itemize}
\item {Grp. gram.:adj.}
\end{itemize}
O mesmo que \textunderscore immemorial\textunderscore .
\section{Immensamente}
\begin{itemize}
\item {Grp. gram.:adv.}
\end{itemize}
De modo immenso; desmedidamente: \textunderscore immensamente rico\textunderscore .
Sem termo.
\section{Immensidade}
\begin{itemize}
\item {Grp. gram.:f.}
\end{itemize}
\begin{itemize}
\item {Proveniência:(Lat. \textunderscore immensitas\textunderscore )}
\end{itemize}
Qualidade daquillo que é immenso.
Extensão desmedida; espaço immenso, o infinito.
\section{Immensidão}
\begin{itemize}
\item {Grp. gram.:f.}
\end{itemize}
O mesmo que \textunderscore immensidade\textunderscore .
\section{Immenso}
\begin{itemize}
\item {Grp. gram.:adj.}
\end{itemize}
\begin{itemize}
\item {Proveniência:(Lat. \textunderscore immensus\textunderscore )}
\end{itemize}
Que se não póde medir; illimitado.
Indefinível.
Infinito.
Numeroso: \textunderscore multidão immensa\textunderscore .
Enorme; muito extenso: \textunderscore mar immenso\textunderscore .
\section{Imensurável}
\begin{itemize}
\item {Grp. gram.:adj.}
\end{itemize}
\begin{itemize}
\item {Proveniência:(Lat. \textunderscore immensurabilis\textunderscore )}
\end{itemize}
Que se não póde medir; imenso.
\section{Imensuravelmente}
\begin{itemize}
\item {Grp. gram.:adv.}
\end{itemize}
De modo imensurável.
\section{Imerecidamente}
\begin{itemize}
\item {Grp. gram.:adv.}
\end{itemize}
\begin{itemize}
\item {Proveniência:(De \textunderscore imerecido\textunderscore )}
\end{itemize}
Sem merecimento.
Sem direito: \textunderscore absolvido imerecidamente\textunderscore .
\section{Imerecido}
\begin{itemize}
\item {Grp. gram.:adj.}
\end{itemize}
\begin{itemize}
\item {Proveniência:(De \textunderscore im...\textunderscore  + \textunderscore merecido\textunderscore )}
\end{itemize}
Que não é merecido.
\section{Imergente}
\begin{itemize}
\item {Grp. gram.:adj.}
\end{itemize}
\begin{itemize}
\item {Proveniência:(Lat. \textunderscore immergens\textunderscore )}
\end{itemize}
Que imerge.
\section{Imergir}
\begin{itemize}
\item {Grp. gram.:v. t.}
\end{itemize}
\begin{itemize}
\item {Grp. gram.:V. i.}
\end{itemize}
\begin{itemize}
\item {Proveniência:(Lat. \textunderscore immergere\textunderscore )}
\end{itemize}
Fazer mergulhar; afundar.
Penetrar.
\section{Imeritamente}
\begin{itemize}
\item {Grp. gram.:adv.}
\end{itemize}
O mesmo que \textunderscore imerecidamente\textunderscore .
\section{Imérito}
\begin{itemize}
\item {Grp. gram.:adj.}
\end{itemize}
\begin{itemize}
\item {Proveniência:(Lat. \textunderscore immeritus\textunderscore )}
\end{itemize}
O mesmo que \textunderscore imerecido\textunderscore .
\section{Imersão}
\begin{itemize}
\item {Grp. gram.:f.}
\end{itemize}
\begin{itemize}
\item {Proveniência:(Lat. \textunderscore immersio\textunderscore )}
\end{itemize}
Acto ou efeito de imergir.
Princípio de um eclípse.
\section{Imersivamente}
\begin{itemize}
\item {Grp. gram.:adv.}
\end{itemize}
De modo imersivo; com imersão.
\section{Imersível}
\begin{itemize}
\item {Grp. gram.:adj.}
\end{itemize}
\begin{itemize}
\item {Proveniência:(Do lat. \textunderscore immersus\textunderscore )}
\end{itemize}
Que se póde afundar, que póde mergulhar.
\section{Imersivo}
\begin{itemize}
\item {Grp. gram.:adj.}
\end{itemize}
\begin{itemize}
\item {Proveniência:(De \textunderscore immerso\textunderscore )}
\end{itemize}
Próprio para fazer imergir.
Que faz imergir.
Que se realiza por imersão: \textunderscore um banho imersivo\textunderscore .
\section{Imerso}
\begin{itemize}
\item {Grp. gram.:adj.}
\end{itemize}
\begin{itemize}
\item {Proveniência:(Lat. \textunderscore immersus\textunderscore )}
\end{itemize}
Mergulhado.
Abismado.
Concentrado: \textunderscore imerso em tristezas\textunderscore .
\section{Imersor}
\begin{itemize}
\item {Grp. gram.:m.  e  adj.}
\end{itemize}
\begin{itemize}
\item {Proveniência:(De \textunderscore imerso\textunderscore )}
\end{itemize}
O que faz imergir.
\section{Imigração}
\begin{itemize}
\item {Grp. gram.:f.}
\end{itemize}
Acto ou efeito de imigrar.
\section{Imigrado}
\begin{itemize}
\item {Grp. gram.:adj.}
\end{itemize}
\begin{itemize}
\item {Grp. gram.:M.}
\end{itemize}
Que se estabeleceu num país, vindo de outro.
Aquele que imigrou.
\section{Imigrante}
\begin{itemize}
\item {Grp. gram.:m.  e  adj.}
\end{itemize}
\begin{itemize}
\item {Proveniência:(Lat. \textunderscore immigrans\textunderscore )}
\end{itemize}
O que imigra.
\section{Imigrar}
\begin{itemize}
\item {Grp. gram.:v. i.}
\end{itemize}
\begin{itemize}
\item {Proveniência:(Lat. \textunderscore immigrare\textunderscore )}
\end{itemize}
Entrar num país estranho, para nele se estabelecer.
\section{Imigratório}
\begin{itemize}
\item {Grp. gram.:adj.}
\end{itemize}
\begin{itemize}
\item {Proveniência:(De \textunderscore imigrar\textunderscore )}
\end{itemize}
Relativo á imigração ou aos imigrantes: \textunderscore estatística imigratória\textunderscore .
\section{Iminência}
\begin{itemize}
\item {Grp. gram.:f.}
\end{itemize}
\begin{itemize}
\item {Proveniência:(Lat. \textunderscore imminentia\textunderscore )}
\end{itemize}
Qualidade daquilo que está iminente.
\section{Iminente}
\begin{itemize}
\item {Grp. gram.:adj.}
\end{itemize}
\begin{itemize}
\item {Proveniência:(Lat. \textunderscore imminens\textunderscore )}
\end{itemize}
Sobranceiro; impendente.
Que ameaça caír sôbre alguém ou sôbre alguma coisa: \textunderscore desgraça iminente\textunderscore .
\section{Imiscibilidade}
\begin{itemize}
\item {Grp. gram.:f.}
\end{itemize}
Qualidade daquilo que é imiscível.
\section{Imiscível}
\begin{itemize}
\item {Grp. gram.:adj.}
\end{itemize}
\begin{itemize}
\item {Proveniência:(Lat. \textunderscore immiscibilis\textunderscore )}
\end{itemize}
Que se não póde misturar.
\section{Imisericordiosamente}
\begin{itemize}
\item {Grp. gram.:adv.}
\end{itemize}
De modo imisericordioso.
Inexoravelmente; sem compaixão.
\section{Imisericordioso}
\begin{itemize}
\item {Grp. gram.:adj.}
\end{itemize}
\begin{itemize}
\item {Proveniência:(De \textunderscore im...\textunderscore  + \textunderscore misericordioso\textunderscore )}
\end{itemize}
Que não é misericordioso; impiedoso; deshumano.
\section{Ímite}
\begin{itemize}
\item {Grp. gram.:adj.}
\end{itemize}
\begin{itemize}
\item {Utilização:Des.}
\end{itemize}
\begin{itemize}
\item {Proveniência:(Lat. \textunderscore immitis\textunderscore )}
\end{itemize}
Que não amadureceu ainda; verde.
O mesmo que \textunderscore cruel\textunderscore . Cf. \textunderscore Aff. Africano\textunderscore , 15.
\section{Immensurável}
\begin{itemize}
\item {Grp. gram.:adj.}
\end{itemize}
\begin{itemize}
\item {Proveniência:(Lat. \textunderscore immensurabilis\textunderscore )}
\end{itemize}
Que se não póde medir; immenso.
\section{Immensuravelmente}
\begin{itemize}
\item {Grp. gram.:adv.}
\end{itemize}
De modo immensurável.
\section{Immerecidamente}
\begin{itemize}
\item {Grp. gram.:adv.}
\end{itemize}
\begin{itemize}
\item {Proveniência:(De \textunderscore immerecido\textunderscore )}
\end{itemize}
Sem merecimento.
Sem direito: \textunderscore absolvido immerecidamente\textunderscore .
\section{Immerecido}
\begin{itemize}
\item {Grp. gram.:adj.}
\end{itemize}
\begin{itemize}
\item {Proveniência:(De \textunderscore im...\textunderscore  + \textunderscore merecido\textunderscore )}
\end{itemize}
Que não é merecido.
\section{Immergente}
\begin{itemize}
\item {Grp. gram.:adj.}
\end{itemize}
\begin{itemize}
\item {Proveniência:(Lat. \textunderscore immergens\textunderscore )}
\end{itemize}
Que immerge.
\section{Immergir}
\begin{itemize}
\item {Grp. gram.:v. t.}
\end{itemize}
\begin{itemize}
\item {Grp. gram.:V. i.}
\end{itemize}
\begin{itemize}
\item {Proveniência:(Lat. \textunderscore immergere\textunderscore )}
\end{itemize}
Fazer mergulhar; afundar.
Penetrar.
\section{Immeritamente}
\begin{itemize}
\item {Grp. gram.:adv.}
\end{itemize}
O mesmo que \textunderscore immerecidamente\textunderscore .
\section{Immérito}
\begin{itemize}
\item {Grp. gram.:adj.}
\end{itemize}
\begin{itemize}
\item {Proveniência:(Lat. \textunderscore immeritus\textunderscore )}
\end{itemize}
O mesmo que \textunderscore immerecido\textunderscore .
\section{Immersão}
\begin{itemize}
\item {Grp. gram.:f.}
\end{itemize}
\begin{itemize}
\item {Proveniência:(Lat. \textunderscore immersio\textunderscore )}
\end{itemize}
Acto ou effeito de immergir.
Princípio de um eclípse.
\section{Immersivamente}
\begin{itemize}
\item {Grp. gram.:adv.}
\end{itemize}
De modo immersivo; com immersão.
\section{Immersível}
\begin{itemize}
\item {Grp. gram.:adj.}
\end{itemize}
\begin{itemize}
\item {Proveniência:(Do lat. \textunderscore immersus\textunderscore )}
\end{itemize}
Que se póde afundar, que póde mergulhar.
\section{Immersivo}
\begin{itemize}
\item {Grp. gram.:adj.}
\end{itemize}
\begin{itemize}
\item {Proveniência:(De \textunderscore immerso\textunderscore )}
\end{itemize}
Próprio para fazer immergir.
Que faz immergir.
Que se realiza por immersão: \textunderscore um banho immersivo\textunderscore .
\section{Immerso}
\begin{itemize}
\item {Grp. gram.:adj.}
\end{itemize}
\begin{itemize}
\item {Proveniência:(Lat. \textunderscore immersus\textunderscore )}
\end{itemize}
Mergulhado.
Abysmado.
Concentrado: \textunderscore immerso em tristezas\textunderscore .
\section{Immersor}
\begin{itemize}
\item {Grp. gram.:m.  e  adj.}
\end{itemize}
\begin{itemize}
\item {Proveniência:(De \textunderscore immerso\textunderscore )}
\end{itemize}
O que faz immergir.
\section{Immigração}
\begin{itemize}
\item {Grp. gram.:f.}
\end{itemize}
Acto ou effeito de immigrar.
\section{Immigrado}
\begin{itemize}
\item {Grp. gram.:adj.}
\end{itemize}
\begin{itemize}
\item {Grp. gram.:M.}
\end{itemize}
Que se estabeleceu num país, vindo de outro.
Aquelle que immigrou.
\section{Immigrante}
\begin{itemize}
\item {Grp. gram.:m.  e  adj.}
\end{itemize}
\begin{itemize}
\item {Proveniência:(Lat. \textunderscore immigrans\textunderscore )}
\end{itemize}
O que immigra.
\section{Immigrar}
\begin{itemize}
\item {Grp. gram.:v. i.}
\end{itemize}
\begin{itemize}
\item {Proveniência:(Lat. \textunderscore immigrare\textunderscore )}
\end{itemize}
Entrar num país estranho, para nelle se estabelecer.
\section{Immigratório}
\begin{itemize}
\item {Grp. gram.:adj.}
\end{itemize}
\begin{itemize}
\item {Proveniência:(De \textunderscore immigrar\textunderscore )}
\end{itemize}
Relativo á immigração ou aos immigrantes: \textunderscore estatística immigratória\textunderscore .
\section{Imminência}
\begin{itemize}
\item {Grp. gram.:f.}
\end{itemize}
\begin{itemize}
\item {Proveniência:(Lat. \textunderscore imminentia\textunderscore )}
\end{itemize}
Qualidade daquillo que está imminente.
\section{Imminente}
\begin{itemize}
\item {Grp. gram.:adj.}
\end{itemize}
\begin{itemize}
\item {Proveniência:(Lat. \textunderscore imminens\textunderscore )}
\end{itemize}
Sobranceiro; impendente.
Que ameaça caír sôbre alguém ou sôbre alguma coisa: \textunderscore desgraça imminente\textunderscore .
\section{Immiscibilidade}
\begin{itemize}
\item {Grp. gram.:f.}
\end{itemize}
Qualidade daquillo que é immiscível.
\section{Immiscível}
\begin{itemize}
\item {Grp. gram.:adj.}
\end{itemize}
\begin{itemize}
\item {Proveniência:(Lat. \textunderscore immiscibilis\textunderscore )}
\end{itemize}
Que se não póde misturar.
\section{Immisericordiosamente}
\begin{itemize}
\item {Grp. gram.:adv.}
\end{itemize}
De modo immisericordioso.
Inexoravelmente; sem compaixão.
\section{Immisericordioso}
\begin{itemize}
\item {Grp. gram.:adj.}
\end{itemize}
\begin{itemize}
\item {Proveniência:(De \textunderscore im...\textunderscore  + \textunderscore misericordioso\textunderscore )}
\end{itemize}
Que não é misericordioso; impiedoso; deshumano.
\section{Ímmite}
\begin{itemize}
\item {Grp. gram.:adj.}
\end{itemize}
\begin{itemize}
\item {Utilização:Des.}
\end{itemize}
\begin{itemize}
\item {Proveniência:(Lat. \textunderscore immitis\textunderscore )}
\end{itemize}
Que não amadureceu ainda; verde.
O mesmo que \textunderscore cruel\textunderscore . Cf. \textunderscore Aff. Africano\textunderscore , 15.
\section{Immóbil}
\begin{itemize}
\item {Grp. gram.:adj.}
\end{itemize}
\begin{itemize}
\item {Utilização:Des.}
\end{itemize}
\begin{itemize}
\item {Proveniência:(Lat. \textunderscore immobilis\textunderscore )}
\end{itemize}
O mesmo que \textunderscore immóvel\textunderscore .
\section{Immobiliariamente}
\begin{itemize}
\item {Grp. gram.:adv.}
\end{itemize}
\begin{itemize}
\item {Proveniência:(De \textunderscore immobiliário\textunderscore )}
\end{itemize}
Relativamente a bens immóveis.
\section{Immobiliário}
\begin{itemize}
\item {Grp. gram.:adj.}
\end{itemize}
\begin{itemize}
\item {Proveniência:(De \textunderscore im...\textunderscore  + \textunderscore mobiliário\textunderscore )}
\end{itemize}
Diz-se de bens que são immóveis por natureza ou por disposição da lei.
\section{Immobilidade}
\begin{itemize}
\item {Grp. gram.:f.}
\end{itemize}
\begin{itemize}
\item {Proveniência:(Lat. \textunderscore immobilitas\textunderscore )}
\end{itemize}
Qualidade ou estado daquillo que é immóvel.
Estacionamento.
Imperturbabilidade.
Difficuldade de movimentos ou rigidez dos músculos locomotores, no cavallo.
\section{Immobilismo}
\begin{itemize}
\item {Grp. gram.:m.}
\end{itemize}
\begin{itemize}
\item {Utilização:Neol.}
\end{itemize}
\begin{itemize}
\item {Proveniência:(Do lat. \textunderscore immobilis\textunderscore )}
\end{itemize}
Aversão ao progresso e paixão pelas instituições antigas.
\section{Immobilista}
\begin{itemize}
\item {Grp. gram.:m.  e  adj.}
\end{itemize}
\begin{itemize}
\item {Proveniência:(Do lat. \textunderscore immobilis\textunderscore )}
\end{itemize}
Sectário do immobilismo.
\section{Immobilização}
\begin{itemize}
\item {Grp. gram.:f.}
\end{itemize}
Acto ou effeito de immobilizar.
\section{Immobilizar}
\begin{itemize}
\item {Grp. gram.:v. t.}
\end{itemize}
\begin{itemize}
\item {Proveniência:(Do lat. \textunderscore immobilis\textunderscore )}
\end{itemize}
Tornar immóvel.
Impedir os movimentos de.
Fazer parar; não deixar progredir.
\section{Immoderação}
\begin{itemize}
\item {Grp. gram.:f.}
\end{itemize}
\begin{itemize}
\item {Proveniência:(De \textunderscore im...\textunderscore  + \textunderscore moderação\textunderscore )}
\end{itemize}
Falta de moderação; descommedimento.
\section{Immoderadamente}
\begin{itemize}
\item {Grp. gram.:adv.}
\end{itemize}
De modo immoderado; desmedidamente; com excesso: \textunderscore comer immoderadamente\textunderscore .
\section{Immoderado}
\begin{itemize}
\item {Grp. gram.:adj.}
\end{itemize}
\begin{itemize}
\item {Proveniência:(Lat. \textunderscore immoderatus\textunderscore )}
\end{itemize}
Que não é moderado, que não tem moderação; descommedido; exaggerado; excessivo.
\section{Immoderato}
\begin{itemize}
\item {Grp. gram.:adj.}
\end{itemize}
(V.immoderado)
\section{Immodestamente}
\begin{itemize}
\item {Grp. gram.:adv.}
\end{itemize}
De modo immodesto; sem modéstia; com vaidade.
\section{Immodéstia}
\begin{itemize}
\item {Grp. gram.:f.}
\end{itemize}
\begin{itemize}
\item {Proveniência:(Lat. \textunderscore immodestia\textunderscore )}
\end{itemize}
Falta de modéstia; falta de pudor.
Desenvoltura.
Orgulho, philáucia.
\section{Immodesto}
\begin{itemize}
\item {Grp. gram.:adj.}
\end{itemize}
\begin{itemize}
\item {Proveniência:(Lat. \textunderscore immodestus\textunderscore )}
\end{itemize}
Que não tem modéstia; desenvolto, impudico.
Jactancioso; presumido.
\section{Immodicidade}
\begin{itemize}
\item {Grp. gram.:f.}
\end{itemize}
Qualidade daquillo que é immódico.
\section{Immódico}
\begin{itemize}
\item {Grp. gram.:adj.}
\end{itemize}
\begin{itemize}
\item {Proveniência:(Lat. \textunderscore immodicus\textunderscore )}
\end{itemize}
Que não é módico; excessivo; exaggerado; elevado: \textunderscore preços immódicos\textunderscore .
\section{Immodificável}
\begin{itemize}
\item {Grp. gram.:adj.}
\end{itemize}
\begin{itemize}
\item {Proveniência:(De \textunderscore im...\textunderscore  + \textunderscore modificável\textunderscore )}
\end{itemize}
Que se não póde modificar.
\section{Immoirar}
\begin{itemize}
\item {Grp. gram.:v.}
\end{itemize}
\begin{itemize}
\item {Utilização:t. Marn.}
\end{itemize}
Passar para um compartimento (o líquido reservado em compartimento superior).
(Cp. \textunderscore salmoira\textunderscore )
\section{Immolação}
\begin{itemize}
\item {Grp. gram.:f.}
\end{itemize}
\begin{itemize}
\item {Proveniência:(Lat. \textunderscore immolatio\textunderscore )}
\end{itemize}
Acto ou effeito de immolar.
\section{Immolador}
\begin{itemize}
\item {Grp. gram.:m.  e  adj.}
\end{itemize}
\begin{itemize}
\item {Proveniência:(Lat. \textunderscore immolator\textunderscore )}
\end{itemize}
O que immola.
Sacrificador.
\section{Immolando}
\begin{itemize}
\item {Grp. gram.:adj.}
\end{itemize}
\begin{itemize}
\item {Proveniência:(Lat. \textunderscore immolandus\textunderscore )}
\end{itemize}
Que tem de ser immolado.
\section{Immolante}
\begin{itemize}
\item {Grp. gram.:adj.}
\end{itemize}
\begin{itemize}
\item {Utilização:Poét.}
\end{itemize}
\begin{itemize}
\item {Proveniência:(Lat. \textunderscore immolans\textunderscore )}
\end{itemize}
Que immola.
\section{Immolar}
\begin{itemize}
\item {Grp. gram.:v. t.}
\end{itemize}
\begin{itemize}
\item {Proveniência:(Lat. \textunderscore immolare\textunderscore )}
\end{itemize}
Sacrificar, degollando.
Sacrificar, matando; sacrificar.
\section{Immoral}
\begin{itemize}
\item {Grp. gram.:adj.}
\end{itemize}
\begin{itemize}
\item {Proveniência:(De \textunderscore im...\textunderscore  + \textunderscore moral\textunderscore )}
\end{itemize}
Que não é moral; opposto á moral.
Que tem maus costumes; libertino.
\section{Immoralidade}
\begin{itemize}
\item {Grp. gram.:f.}
\end{itemize}
Qualidade daquelle ou daquillo que é immoral.
Desregramento; devassidão.
Prática de maus costumes.
\section{Immorigerado}
\begin{itemize}
\item {Grp. gram.:adj.}
\end{itemize}
\begin{itemize}
\item {Proveniência:(De \textunderscore im...\textunderscore  + \textunderscore morigerado\textunderscore )}
\end{itemize}
Que não é bem morigerado.
Devasso; libertino.
\section{Immorredoiro}
\begin{itemize}
\item {Grp. gram.:adj.}
\end{itemize}
\begin{itemize}
\item {Proveniência:(De \textunderscore im...\textunderscore  + \textunderscore morredoiro\textunderscore )}
\end{itemize}
Que não é morredoiro.
Que não acaba; imperecível.
\section{Immortal}
\begin{itemize}
\item {Grp. gram.:adj.}
\end{itemize}
\begin{itemize}
\item {Proveniência:(Lat. \textunderscore immortalis\textunderscore )}
\end{itemize}
Que não morre.
Que não acaba.
Immorredoiro: \textunderscore a alma é immortal\textunderscore .
Que não será esquecido.
Glorioso: \textunderscore poeta immortal\textunderscore .
\section{Immortalidade}
\begin{itemize}
\item {Grp. gram.:f.}
\end{itemize}
\begin{itemize}
\item {Proveniência:(Lat. \textunderscore immortalitas\textunderscore )}
\end{itemize}
Qualidade daquillo que é immortal.
A vida eterna.
\section{Immortalização}
\begin{itemize}
\item {Grp. gram.:f.}
\end{itemize}
Acto ou effeito de immortalizar.
\section{Immortalizador}
\begin{itemize}
\item {Grp. gram.:adj.}
\end{itemize}
\begin{itemize}
\item {Grp. gram.:M.}
\end{itemize}
Que immortaliza.
Aquelle que immortaliza.
\section{Immortalizar}
\begin{itemize}
\item {Grp. gram.:v. t.}
\end{itemize}
Tornar immortal.
Tornar famoso ou célebre: \textunderscore os«Lusíadas»immortalizaram Camões\textunderscore .
\section{Immortificação}
\begin{itemize}
\item {Grp. gram.:f.}
\end{itemize}
Acto ou estado de immortificado.
\section{Immortificado}
\begin{itemize}
\item {Grp. gram.:adj.}
\end{itemize}
Não mortificado; alliviado de mortificação. Cf. \textunderscore Luz e Calor\textunderscore , 275.
\section{Immotiva}
\begin{itemize}
\item {Grp. gram.:adj. f.}
\end{itemize}
\begin{itemize}
\item {Utilização:Bot.}
\end{itemize}
\begin{itemize}
\item {Proveniência:(Do lat. \textunderscore immotus\textunderscore )}
\end{itemize}
Diz-se da germinação, quando se effectua sem deslocação do episperma.
\section{Immoto}
\begin{itemize}
\item {Grp. gram.:adj.}
\end{itemize}
\begin{itemize}
\item {Proveniência:(Lat. \textunderscore immotus\textunderscore )}
\end{itemize}
O mesmo que \textunderscore immóvel\textunderscore .
\section{Immóvel}
\begin{itemize}
\item {Grp. gram.:adj.}
\end{itemize}
\begin{itemize}
\item {Grp. gram.:Pl. m.  e  adj.}
\end{itemize}
\begin{itemize}
\item {Proveniência:(Lat. \textunderscore immobilis\textunderscore )}
\end{itemize}
Que se não move; inalterável.
Immutável.
Prédios rústicos ou urbanos e aquelles valores que, não sendo immóveis por natureza, são por lei declarados taes, como os frutos dos prédios, direitos inherentes a prédios e os fundos consolidados.
\section{Immovelmente}
\begin{itemize}
\item {Grp. gram.:adv.}
\end{itemize}
\begin{itemize}
\item {Proveniência:(De \textunderscore immóvel\textunderscore )}
\end{itemize}
Sem movimento.
\section{Immudável}
\begin{itemize}
\item {Grp. gram.:adj.}
\end{itemize}
\begin{itemize}
\item {Proveniência:(De \textunderscore im...\textunderscore  + \textunderscore mudável\textunderscore )}
\end{itemize}
Que não é mudável; immutável; immóvel; permanente; constante.
\section{Immundice}
\begin{itemize}
\item {Grp. gram.:f.}
\end{itemize}
\begin{itemize}
\item {Utilização:Bras. de Minas}
\end{itemize}
Grande porção: \textunderscore uma immundice de dinheiro\textunderscore .
(Por \textunderscore inundice\textunderscore , de \textunderscore inundar\textunderscore ?)
\section{Immundícia}
\begin{itemize}
\item {Grp. gram.:f.}
\end{itemize}
\begin{itemize}
\item {Proveniência:(Lat. \textunderscore immunditia\textunderscore )}
\end{itemize}
Falta de asseio; sujidade; lixo; impureza.
\section{Immundície}
\begin{itemize}
\item {Grp. gram.:f.}
\end{itemize}
\begin{itemize}
\item {Utilização:Bras}
\end{itemize}
\begin{itemize}
\item {Proveniência:(Lat. \textunderscore immundities\textunderscore )}
\end{itemize}
O mesmo que \textunderscore immundícia\textunderscore .
Caça miúda de pêlo.
\section{Immundo}
\begin{itemize}
\item {Grp. gram.:adj.}
\end{itemize}
\begin{itemize}
\item {Proveniência:(Lat. \textunderscore immundus\textunderscore )}
\end{itemize}
Que não é limpo; sujo.
Sórdido.
Immoral; obsceno.
\section{Immundo}
\begin{itemize}
\item {Grp. gram.:adj.}
\end{itemize}
\begin{itemize}
\item {Utilização:Prov.}
\end{itemize}
\begin{itemize}
\item {Utilização:trasm.}
\end{itemize}
\begin{itemize}
\item {Proveniência:(Do lat. \textunderscore in...\textunderscore  + \textunderscore mundus\textunderscore , adj.? Ou por \textunderscore immuto\textunderscore , de \textunderscore immutar\textunderscore ?)}
\end{itemize}
Absorto, alheado, estranho ao mundo.
\section{Immune}
\begin{itemize}
\item {Grp. gram.:adj.}
\end{itemize}
\begin{itemize}
\item {Proveniência:(Lat. \textunderscore immunis\textunderscore )}
\end{itemize}
Que tem immunidade; isento; livre.
\section{Immunidade}
\begin{itemize}
\item {Grp. gram.:f.}
\end{itemize}
\begin{itemize}
\item {Proveniência:(Lat. \textunderscore immunitas\textunderscore )}
\end{itemize}
Isenção de algum encargo.
Privilégio.
Predisposição orgânica, pela qual alguns indivíduos estão isentos de moléstias que atacam outros, collocados em meio idêntico.
\section{Immunização}
\begin{itemize}
\item {Grp. gram.:f.}
\end{itemize}
Acto de immunizar.
\section{Immunizar}
\begin{itemize}
\item {Grp. gram.:v. t.}
\end{itemize}
Tornar immune.
\section{Immutabilidade}
\begin{itemize}
\item {Grp. gram.:f.}
\end{itemize}
\begin{itemize}
\item {Proveniência:(Lat. \textunderscore immutabilitas\textunderscore )}
\end{itemize}
Qualidade daquillo que é immutável.
\section{Immutação}
\begin{itemize}
\item {Grp. gram.:f.}
\end{itemize}
\begin{itemize}
\item {Proveniência:(Lat. \textunderscore immutatio\textunderscore )}
\end{itemize}
Acto de immutar.
\section{Immutar}
\begin{itemize}
\item {Grp. gram.:v. t.}
\end{itemize}
\begin{itemize}
\item {Utilização:Des.}
\end{itemize}
\begin{itemize}
\item {Proveniência:(Lat. \textunderscore immutare\textunderscore )}
\end{itemize}
Transmudar, mudar completamente.
\section{Immutável}
\begin{itemize}
\item {Grp. gram.:adj.}
\end{itemize}
\begin{itemize}
\item {Proveniência:(Lat. \textunderscore immutabilis\textunderscore )}
\end{itemize}
O mesmo que \textunderscore immudável\textunderscore .
\section{Imo}
\begin{itemize}
\item {Grp. gram.:adj.}
\end{itemize}
\begin{itemize}
\item {Proveniência:(Lat. \textunderscore imus\textunderscore )}
\end{itemize}
Que está no lugar mais fundo ou mais baixo; íntimo.
\section{Imóbil}
\begin{itemize}
\item {Grp. gram.:adj.}
\end{itemize}
\begin{itemize}
\item {Utilização:Des.}
\end{itemize}
\begin{itemize}
\item {Proveniência:(Lat. \textunderscore immobilis\textunderscore )}
\end{itemize}
O mesmo que \textunderscore imóvel\textunderscore .
\section{Imobiliariamente}
\begin{itemize}
\item {Grp. gram.:adv.}
\end{itemize}
\begin{itemize}
\item {Proveniência:(De \textunderscore imobiliário\textunderscore )}
\end{itemize}
Relativamente a bens imóveis.
\section{Imobiliário}
\begin{itemize}
\item {Grp. gram.:adj.}
\end{itemize}
\begin{itemize}
\item {Proveniência:(De \textunderscore im...\textunderscore  + \textunderscore mobiliário\textunderscore )}
\end{itemize}
Diz-se de bens que são imóveis por natureza ou por disposição da lei.
\section{Imobilidade}
\begin{itemize}
\item {Grp. gram.:f.}
\end{itemize}
\begin{itemize}
\item {Proveniência:(Lat. \textunderscore immobilitas\textunderscore )}
\end{itemize}
Qualidade ou estado daquilo que é imóvel.
Estacionamento.
Imperturbabilidade.
Dificuldade de movimentos ou rigidez dos músculos locomotores, no cavalo.
\section{Imobilismo}
\begin{itemize}
\item {Grp. gram.:m.}
\end{itemize}
\begin{itemize}
\item {Utilização:Neol.}
\end{itemize}
\begin{itemize}
\item {Proveniência:(Do lat. \textunderscore immobilis\textunderscore )}
\end{itemize}
Aversão ao progresso e paixão pelas instituições antigas.
\section{Imobilista}
\begin{itemize}
\item {Grp. gram.:m.  e  adj.}
\end{itemize}
\begin{itemize}
\item {Proveniência:(Do lat. \textunderscore immobilis\textunderscore )}
\end{itemize}
Sectário do imobilismo.
\section{Imobilização}
\begin{itemize}
\item {Grp. gram.:f.}
\end{itemize}
Acto ou efeito de imobilizar.
\section{Imobilizar}
\begin{itemize}
\item {Grp. gram.:v. t.}
\end{itemize}
\begin{itemize}
\item {Proveniência:(Do lat. \textunderscore immobilis\textunderscore )}
\end{itemize}
Tornar imóvel.
Impedir os movimentos de.
Fazer parar; não deixar progredir.
\section{Imoderação}
\begin{itemize}
\item {Grp. gram.:f.}
\end{itemize}
\begin{itemize}
\item {Proveniência:(De \textunderscore im...\textunderscore  + \textunderscore moderação\textunderscore )}
\end{itemize}
Falta de moderação; descomedimento.
\section{Imoderadamente}
\begin{itemize}
\item {Grp. gram.:adv.}
\end{itemize}
De modo imoderado; desmedidamente; com excesso: \textunderscore comer imoderadamente\textunderscore .
\section{Imoderado}
\begin{itemize}
\item {Grp. gram.:adj.}
\end{itemize}
\begin{itemize}
\item {Proveniência:(Lat. \textunderscore immoderatus\textunderscore )}
\end{itemize}
Que não é moderado, que não tem moderação; descomedido; exagerado; excessivo.
\section{Imoderato}
\begin{itemize}
\item {Grp. gram.:adj.}
\end{itemize}
(V.imoderado)
\section{Imodestamente}
\begin{itemize}
\item {Grp. gram.:adv.}
\end{itemize}
De modo imodesto; sem modéstia; com vaidade.
\section{Imodéstia}
\begin{itemize}
\item {Grp. gram.:f.}
\end{itemize}
\begin{itemize}
\item {Proveniência:(Lat. \textunderscore immodestia\textunderscore )}
\end{itemize}
Falta de modéstia; falta de pudor.
Desenvoltura.
Orgulho, filáucia.
\section{Imodesto}
\begin{itemize}
\item {Grp. gram.:adj.}
\end{itemize}
\begin{itemize}
\item {Proveniência:(Lat. \textunderscore immodestus\textunderscore )}
\end{itemize}
Que não tem modéstia; desenvolto, impudico.
Jactancioso; presumido.
\section{Imodicidade}
\begin{itemize}
\item {Grp. gram.:f.}
\end{itemize}
Qualidade daquilo que é imódico.
\section{Imódico}
\begin{itemize}
\item {Grp. gram.:adj.}
\end{itemize}
\begin{itemize}
\item {Proveniência:(Lat. \textunderscore immodicus\textunderscore )}
\end{itemize}
Que não é módico; excessivo; exagerado; elevado: \textunderscore preços imódicos\textunderscore .
\section{Imodificável}
\begin{itemize}
\item {Grp. gram.:adj.}
\end{itemize}
\begin{itemize}
\item {Proveniência:(De \textunderscore im...\textunderscore  + \textunderscore modificável\textunderscore )}
\end{itemize}
Que se não póde modificar.
\section{Imogênio}
\begin{itemize}
\item {Grp. gram.:m.}
\end{itemize}
Substância chímica, usada em photographia, e de reacção ligeiramente ácida e apenas oxydável lentamente ao ar.
\section{Imoirar}
\begin{itemize}
\item {Grp. gram.:v.}
\end{itemize}
\begin{itemize}
\item {Utilização:t. Marn.}
\end{itemize}
Passar para um compartimento (o líquido reservado em compartimento superior).
(Cp. \textunderscore salmoira\textunderscore )
\section{Imolação}
\begin{itemize}
\item {Grp. gram.:f.}
\end{itemize}
\begin{itemize}
\item {Proveniência:(Lat. \textunderscore immolatio\textunderscore )}
\end{itemize}
Acto ou efeito de imolar.
\section{Imolador}
\begin{itemize}
\item {Grp. gram.:m.  e  adj.}
\end{itemize}
\begin{itemize}
\item {Proveniência:(Lat. \textunderscore immolator\textunderscore )}
\end{itemize}
O que imola.
Sacrificador.
\section{Imolando}
\begin{itemize}
\item {Grp. gram.:adj.}
\end{itemize}
\begin{itemize}
\item {Proveniência:(Lat. \textunderscore immolandus\textunderscore )}
\end{itemize}
Que tem de ser imolado.
\section{Imolante}
\begin{itemize}
\item {Grp. gram.:adj.}
\end{itemize}
\begin{itemize}
\item {Utilização:Poét.}
\end{itemize}
\begin{itemize}
\item {Proveniência:(Lat. \textunderscore immolans\textunderscore )}
\end{itemize}
Que imola.
\section{Imolar}
\begin{itemize}
\item {Grp. gram.:v. t.}
\end{itemize}
\begin{itemize}
\item {Proveniência:(Lat. \textunderscore immolare\textunderscore )}
\end{itemize}
Sacrificar, degolando.
Sacrificar, matando; sacrificar.
\section{Imoral}
\begin{itemize}
\item {Grp. gram.:adj.}
\end{itemize}
\begin{itemize}
\item {Proveniência:(De \textunderscore im...\textunderscore  + \textunderscore moral\textunderscore )}
\end{itemize}
Que não é moral; opposto á moral.
Que tem maus costumes; libertino.
\section{Imoralidade}
\begin{itemize}
\item {Grp. gram.:f.}
\end{itemize}
Qualidade daquele ou daquilo que é imoral.
Desregramento; devassidão.
Prática de maus costumes.
\section{Imorigerado}
\begin{itemize}
\item {Grp. gram.:adj.}
\end{itemize}
\begin{itemize}
\item {Proveniência:(De \textunderscore im...\textunderscore  + \textunderscore morigerado\textunderscore )}
\end{itemize}
Que não é bem morigerado.
Devasso; libertino.
\section{Imorredouro}
\begin{itemize}
\item {Grp. gram.:adj.}
\end{itemize}
\begin{itemize}
\item {Proveniência:(De \textunderscore im...\textunderscore  + \textunderscore morredouro\textunderscore )}
\end{itemize}
Que não é morredouro.
Que não acaba; imperecível.
\section{Imortal}
\begin{itemize}
\item {Grp. gram.:adj.}
\end{itemize}
\begin{itemize}
\item {Proveniência:(Lat. \textunderscore immortalis\textunderscore )}
\end{itemize}
Que não morre.
Que não acaba.
Immorredoiro: \textunderscore a alma é imortal\textunderscore .
Que não será esquecido.
Glorioso: \textunderscore poeta imortal\textunderscore .
\section{Imortalidade}
\begin{itemize}
\item {Grp. gram.:f.}
\end{itemize}
\begin{itemize}
\item {Proveniência:(Lat. \textunderscore immortalitas\textunderscore )}
\end{itemize}
Qualidade daquilo que é imortal.
A vida eterna.
\section{Imortalização}
\begin{itemize}
\item {Grp. gram.:f.}
\end{itemize}
Acto ou efeito de imortalizar.
\section{Imortalizador}
\begin{itemize}
\item {Grp. gram.:adj.}
\end{itemize}
\begin{itemize}
\item {Grp. gram.:M.}
\end{itemize}
Que imortaliza.
Aquele que imortaliza.
\section{Imortalizar}
\begin{itemize}
\item {Grp. gram.:v. t.}
\end{itemize}
Tornar imortal.
Tornar famoso ou célebre: \textunderscore os«Lusíadas»imortalizaram Camões\textunderscore .
\section{Imortificação}
\begin{itemize}
\item {Grp. gram.:f.}
\end{itemize}
Acto ou estado de imortificado.
\section{Imortificado}
\begin{itemize}
\item {Grp. gram.:adj.}
\end{itemize}
Não mortificado; aliviado de mortificação. Cf. \textunderscore Luz e Calor\textunderscore , 275.
\section{Imoscapo}
\begin{itemize}
\item {Grp. gram.:m.}
\end{itemize}
\begin{itemize}
\item {Proveniência:(Do lat. \textunderscore imus\textunderscore  + \textunderscore scapus\textunderscore )}
\end{itemize}
Diâmetro inferior da columna.
\section{Imotiva}
\begin{itemize}
\item {Grp. gram.:adj. f.}
\end{itemize}
\begin{itemize}
\item {Utilização:Bot.}
\end{itemize}
\begin{itemize}
\item {Proveniência:(Do lat. \textunderscore immotus\textunderscore )}
\end{itemize}
Diz-se da germinação, quando se efectua sem deslocação do episperma.
\section{Imoto}
\begin{itemize}
\item {Grp. gram.:adj.}
\end{itemize}
\begin{itemize}
\item {Proveniência:(Lat. \textunderscore immotus\textunderscore )}
\end{itemize}
O mesmo que \textunderscore imóvel\textunderscore .
\section{Imóvel}
\begin{itemize}
\item {Grp. gram.:adj.}
\end{itemize}
\begin{itemize}
\item {Grp. gram.:Pl. m.  e  adj.}
\end{itemize}
\begin{itemize}
\item {Proveniência:(Lat. \textunderscore immobilis\textunderscore )}
\end{itemize}
Que se não move; inalterável.
Immutável.
Prédios rústicos ou urbanos e aqueles valores que, não sendo imóveis por natureza, são por lei declarados taes, como os frutos dos prédios, direitos inerentes a prédios e os fundos consolidados.
\section{Imovelmente}
\begin{itemize}
\item {Grp. gram.:adv.}
\end{itemize}
\begin{itemize}
\item {Proveniência:(De \textunderscore imóvel\textunderscore )}
\end{itemize}
Sem movimento.
\section{Impaca}
\begin{itemize}
\item {Grp. gram.:f.}
\end{itemize}
Corpulento animal angolense, espécie de veado. Cf. Flaviense, \textunderscore Diccion. Geogr.\textunderscore 
\section{Impacaceiro}
\textunderscore m.\textunderscore  Talvez \textunderscore des.\textunderscore 
Soldado negro de Angola, espécie de ordenança. Cf. Flaviense, \textunderscore Diccion. Geogr.\textunderscore 
\section{Impaciência}
\begin{itemize}
\item {Grp. gram.:f.}
\end{itemize}
\begin{itemize}
\item {Proveniência:(Lat. \textunderscore impacientia\textunderscore )}
\end{itemize}
Falta de paciência.
Frenesi.
Pressa; soffreguidão.
Desespêro.
Rabujice.
\section{Impacientar}
\begin{itemize}
\item {Grp. gram.:v. t.}
\end{itemize}
\begin{itemize}
\item {Grp. gram.:V. p.}
\end{itemize}
\begin{itemize}
\item {Proveniência:(De \textunderscore impaciente\textunderscore )}
\end{itemize}
Tornar impaciente.
Importunar.
Irritar.
Perder a paciência; agastar-se.
\section{Impaciente}
\begin{itemize}
\item {Grp. gram.:adj.}
\end{itemize}
\begin{itemize}
\item {Proveniência:(Lat. \textunderscore impatiens\textunderscore )}
\end{itemize}
Que não é paciente.
Precipitado, apressado; sôffrego.
Perturbado; frenético.
\section{Impacientemente}
\begin{itemize}
\item {Grp. gram.:adv.}
\end{itemize}
De modo impaciente; com impaciência.
\section{Impacto}
\begin{itemize}
\item {Grp. gram.:adj.}
\end{itemize}
\begin{itemize}
\item {Proveniência:(Lat. \textunderscore impactus\textunderscore )}
\end{itemize}
Impellido; metido á fôrça.
\section{Ímpado}
\begin{itemize}
\item {Grp. gram.:m.}
\end{itemize}
\begin{itemize}
\item {Utilização:Prov.}
\end{itemize}
\begin{itemize}
\item {Utilização:trasm.}
\end{itemize}
O mesmo que \textunderscore impo\textunderscore .
\section{Impagável}
\begin{itemize}
\item {Grp. gram.:adj.}
\end{itemize}
\begin{itemize}
\item {Utilização:Fig.}
\end{itemize}
\begin{itemize}
\item {Utilização:Fam.}
\end{itemize}
\begin{itemize}
\item {Proveniência:(De \textunderscore im...\textunderscore  + \textunderscore pagável\textunderscore )}
\end{itemize}
Que se não póde ou não se deve pagar.
Inestimável; precioso.
Esquisito, extraordinário.
\section{Impalanca}
\begin{itemize}
\item {Grp. gram.:f.}
\end{itemize}
O mesmo que \textunderscore palanca\textunderscore ^2.
\section{Impalpabilidade}
\begin{itemize}
\item {Grp. gram.:f.}
\end{itemize}
\begin{itemize}
\item {Proveniência:(Do lat. \textunderscore impalpabilis\textunderscore )}
\end{itemize}
Qualidade daquillo que é impalpável.
\section{Impalpável}
\begin{itemize}
\item {Grp. gram.:adj.}
\end{itemize}
\begin{itemize}
\item {Proveniência:(Lat. \textunderscore impalpabilis\textunderscore )}
\end{itemize}
Que não é palpável.
Immaterial.
\section{Impalpavelmente}
\begin{itemize}
\item {Grp. gram.:adv.}
\end{itemize}
De modo impalpável.
\section{Impaludação}
\begin{itemize}
\item {Grp. gram.:f.}
\end{itemize}
Acto ou effeito de impaludar.
\section{Impaludar}
\begin{itemize}
\item {Grp. gram.:v. t.}
\end{itemize}
\begin{itemize}
\item {Proveniência:(Do lat. \textunderscore palus\textunderscore , \textunderscore paludis\textunderscore )}
\end{itemize}
Infeccionar ou atacar com febre palustre.
\section{Impaludismo}
\begin{itemize}
\item {Grp. gram.:m.}
\end{itemize}
\begin{itemize}
\item {Proveniência:(De \textunderscore impaludar\textunderscore )}
\end{itemize}
Malária, resultante da picada de certos insectos.
Doença duradoira ou chrónica, resultante de se têr vivido em regiões pantanosas ou na vizinhança de águas estagnadas; sezonismo.
\section{Ímpar}
\begin{itemize}
\item {Grp. gram.:adj.}
\end{itemize}
\begin{itemize}
\item {Proveniência:(Lat. \textunderscore impar\textunderscore )}
\end{itemize}
Que não é par; desigual; que é único; díspar.
\section{Impár}
\begin{itemize}
\item {Grp. gram.:v. i.}
\end{itemize}
\begin{itemize}
\item {Utilização:Fig.}
\end{itemize}
\begin{itemize}
\item {Proveniência:(Do cast. \textunderscore hipar\textunderscore )}
\end{itemize}
Soluçar.
Arquejar; respirar com difficuldade.
Abarrotar-se com comida ou bebida.
Mostrar-se soberbo ou desdenhoso.
\section{Imparcial}
\begin{itemize}
\item {Grp. gram.:adj.}
\end{itemize}
\begin{itemize}
\item {Proveniência:(De \textunderscore im...\textunderscore  + \textunderscore parcial\textunderscore )}
\end{itemize}
Que não é parcial.
Que revela imparcialidade.
Que julga sem paixão.
Recto.
Que não sacrifica a sua opinião á própria conveniência nem ás conveniências alheias.
\section{Imparcialidade}
\begin{itemize}
\item {Grp. gram.:f.}
\end{itemize}
Qualidade daquelle ou daquillo que é imparcial.
\section{Imparcializar}
\begin{itemize}
\item {Grp. gram.:v. t.}
\end{itemize}
Tornar imparcial.
\section{Imparcialmente}
\begin{itemize}
\item {Grp. gram.:adv.}
\end{itemize}
De modo imparcial.
\section{Imparidade}
\begin{itemize}
\item {Grp. gram.:f.}
\end{itemize}
\begin{itemize}
\item {Proveniência:(Lat. \textunderscore imparitas\textunderscore )}
\end{itemize}
Qualidade daquillo que é ímpar.
\section{Imparinervado}
\begin{itemize}
\item {Grp. gram.:adj.}
\end{itemize}
\begin{itemize}
\item {Utilização:Bot.}
\end{itemize}
\begin{itemize}
\item {Proveniência:(De \textunderscore ímpar\textunderscore  + \textunderscore nervo\textunderscore )}
\end{itemize}
Que tem uma nervura média, sem outras nervuras lateraes.
\section{Imparipinnulado}
\begin{itemize}
\item {Grp. gram.:adj.}
\end{itemize}
\begin{itemize}
\item {Utilização:Bot.}
\end{itemize}
\begin{itemize}
\item {Proveniência:(De \textunderscore ímpar\textunderscore  + \textunderscore pínnula\textunderscore )}
\end{itemize}
Diz-se das fôlhas, que terminam por um folíolo ímpar, como na roseira, na nogueira, etc.
\section{Imparipinulado}
\begin{itemize}
\item {Grp. gram.:adj.}
\end{itemize}
\begin{itemize}
\item {Utilização:Bot.}
\end{itemize}
\begin{itemize}
\item {Proveniência:(De \textunderscore ímpar\textunderscore  + \textunderscore pínnula\textunderscore )}
\end{itemize}
Diz-se das fôlhas, que terminam por um folíolo ímpar, como na roseira, na nogueira, etc.
\section{Imparissilábico}
\begin{itemize}
\item {Grp. gram.:adj.}
\end{itemize}
O mesmo que \textunderscore imparissílabo\textunderscore .
\section{Imparissilabismo}
\begin{itemize}
\item {Grp. gram.:m.}
\end{itemize}
\begin{itemize}
\item {Utilização:Gram.}
\end{itemize}
\begin{itemize}
\item {Proveniência:(De \textunderscore imparisílabo\textunderscore )}
\end{itemize}
Diferença entre o número de sílabas do nominativo e as do acusativo, deslocando-se o acento tónico: \textunderscore látro\textunderscore , \textunderscore latrónem\textunderscore ; ou não se deslocando: \textunderscore córpus\textunderscore , \textunderscore côrporis\textunderscore .
\section{Imparisyllábico}
\begin{itemize}
\item {fónica:si}
\end{itemize}
\begin{itemize}
\item {Grp. gram.:adj.}
\end{itemize}
O mesmo que \textunderscore imparisýllabo\textunderscore .
\section{Imparisyllabismo}
\begin{itemize}
\item {fónica:si}
\end{itemize}
\begin{itemize}
\item {Grp. gram.:m.}
\end{itemize}
\begin{itemize}
\item {Utilização:Gram.}
\end{itemize}
\begin{itemize}
\item {Proveniência:(De \textunderscore imparisýllabo\textunderscore )}
\end{itemize}
Differença entre o número de sýllabas do nominativo e as do accusativo, deslocando-se o accento tónico: \textunderscore látro\textunderscore , \textunderscore latrónem\textunderscore ; ou não se deslocando: \textunderscore córpus\textunderscore , \textunderscore côrporis\textunderscore .
\section{Imparissílabo}
\begin{itemize}
\item {Grp. gram.:adj.}
\end{itemize}
\begin{itemize}
\item {Utilização:Gram.}
\end{itemize}
\begin{itemize}
\item {Proveniência:(De \textunderscore ímpar\textunderscore  + \textunderscore sílaba\textunderscore )}
\end{itemize}
Diz-se das palavras em que há imparisilabismo.
\section{Imparisýllabo}
\begin{itemize}
\item {fónica:si}
\end{itemize}
\begin{itemize}
\item {Grp. gram.:adj.}
\end{itemize}
\begin{itemize}
\item {Utilização:Gram.}
\end{itemize}
\begin{itemize}
\item {Proveniência:(De \textunderscore ímpar\textunderscore  + \textunderscore sýllaba\textunderscore )}
\end{itemize}
Diz-se das palavras em que há imparisyllabismo.
\section{Impartibilidade}
\begin{itemize}
\item {Grp. gram.:f.}
\end{itemize}
\begin{itemize}
\item {Utilização:Jur.}
\end{itemize}
\begin{itemize}
\item {Utilização:ant.}
\end{itemize}
Qualidade daquillo que é impartível.
\section{Impartível}
\begin{itemize}
\item {Grp. gram.:adj.}
\end{itemize}
\begin{itemize}
\item {Proveniência:(De \textunderscore im...\textunderscore  + \textunderscore partível\textunderscore )}
\end{itemize}
Que se não póde partir; indivisível.
\section{Impassibilidade}
\begin{itemize}
\item {Grp. gram.:f.}
\end{itemize}
\begin{itemize}
\item {Proveniência:(Do lat. \textunderscore impassibilis\textunderscore )}
\end{itemize}
Qualidade de quem é impassível.
\section{Impassibilizar}
\begin{itemize}
\item {Grp. gram.:v. t.}
\end{itemize}
\begin{itemize}
\item {Proveniência:(Do lat. \textunderscore impassibilis\textunderscore )}
\end{itemize}
Tornar impassível.
\section{Impassível}
\begin{itemize}
\item {Grp. gram.:adj.}
\end{itemize}
\begin{itemize}
\item {Proveniência:(Lat. \textunderscore impassibilis\textunderscore )}
\end{itemize}
Que não é susceptível de padecer.
Imperturbável.
Sereno; indifferente.
\section{Impassivelmente}
\begin{itemize}
\item {Grp. gram.:adv.}
\end{itemize}
De modo impassível.
\section{Impatrioticamente}
\begin{itemize}
\item {Grp. gram.:adv.}
\end{itemize}
De modo impatriótico.
\section{Impatriótico}
\begin{itemize}
\item {Grp. gram.:adj.}
\end{itemize}
\begin{itemize}
\item {Proveniência:(De \textunderscore im...\textunderscore  + \textunderscore patriótico\textunderscore )}
\end{itemize}
Que não tem ou em que não há patriotismo.
Opposto ao patriotismo.
\section{Impatriotismo}
\begin{itemize}
\item {Grp. gram.:m.}
\end{itemize}
\begin{itemize}
\item {Utilização:Neol.}
\end{itemize}
Falta de patriotismo.
\section{Impavidamente}
\begin{itemize}
\item {Grp. gram.:adv.}
\end{itemize}
De modo impávido.
\section{Impavidez}
\begin{itemize}
\item {Grp. gram.:f.}
\end{itemize}
Qualidade daquelle ou daquillo que é impávido.
\section{Impávido}
\begin{itemize}
\item {Grp. gram.:adj.}
\end{itemize}
\begin{itemize}
\item {Proveniência:(Lat. \textunderscore impavidus\textunderscore )}
\end{itemize}
Que não tem pavor.
Arrojado; destemido; intrépido.
\section{Impecabilidade}
\begin{itemize}
\item {Grp. gram.:f.}
\end{itemize}
Qualidade daquele ou daquilo que é impecável.
\section{Impecável}
\begin{itemize}
\item {Grp. gram.:adj.}
\end{itemize}
\begin{itemize}
\item {Proveniência:(Lat. \textunderscore impeccabilis\textunderscore )}
\end{itemize}
Que não póde pecar.
Imaculável.
\section{Impecavelmente}
\begin{itemize}
\item {Grp. gram.:adv.}
\end{itemize}
De modo impecável.
\section{Impeccabilidade}
\begin{itemize}
\item {Grp. gram.:f.}
\end{itemize}
Qualidade daquelle ou daquillo que é impeccável.
\section{Impeccável}
\begin{itemize}
\item {Grp. gram.:adj.}
\end{itemize}
\begin{itemize}
\item {Proveniência:(Lat. \textunderscore impeccabilis\textunderscore )}
\end{itemize}
Que não póde peccar.
Immaculável.
\section{Impeccavelmente}
\begin{itemize}
\item {Grp. gram.:adv.}
\end{itemize}
De modo impeccável.
\section{Impecunioso}
\begin{itemize}
\item {Grp. gram.:adj.}
\end{itemize}
\begin{itemize}
\item {Proveniência:(De \textunderscore im...\textunderscore  + \textunderscore pecunioso\textunderscore )}
\end{itemize}
Que não é endinheirado, que não é rico. Cf. C. Lobo, \textunderscore Sát.\textunderscore , I, 87.
\section{Impedição}
\begin{itemize}
\item {Grp. gram.:f.}
\end{itemize}
\begin{itemize}
\item {Proveniência:(Lat. \textunderscore impeditio\textunderscore )}
\end{itemize}
O mesmo que \textunderscore impedimento\textunderscore .
\section{Impedido}
\begin{itemize}
\item {Grp. gram.:m.}
\end{itemize}
Soldado ou official, que, com autorização superior, está dispensado do serviço que lhe compete e encarregado de outro.
Soldado, ao serviço especial de um official.
\section{Impedidor}
\begin{itemize}
\item {Grp. gram.:adj.}
\end{itemize}
\begin{itemize}
\item {Grp. gram.:M.}
\end{itemize}
Que impede.
Aquelle que impede.
\section{Impediência}
\begin{itemize}
\item {Grp. gram.:f.}
\end{itemize}
Qualidade de impediente.
\section{Impediente}
\begin{itemize}
\item {Grp. gram.:adj.}
\end{itemize}
\begin{itemize}
\item {Proveniência:(Lat. \textunderscore impediens\textunderscore )}
\end{itemize}
Que impede.
\section{Impedimento}
\begin{itemize}
\item {Grp. gram.:m.}
\end{itemize}
\begin{itemize}
\item {Grp. gram.:Pl.}
\end{itemize}
\begin{itemize}
\item {Utilização:Des.}
\end{itemize}
\begin{itemize}
\item {Proveniência:(Lat. \textunderscore impedimentum\textunderscore )}
\end{itemize}
Aquillo que impede.
Acto ou effeito de impedir.
Bagagens de um exército.
\section{Impedir}
\begin{itemize}
\item {Grp. gram.:v. t.}
\end{itemize}
\begin{itemize}
\item {Proveniência:(Lat. \textunderscore impedire\textunderscore )}
\end{itemize}
Prender pelos pés; pear; embaraçar.
Não permittir.
Atalhar; interromper.
Prohibir: \textunderscore os regulamentos impedem a caça, em certa época do anno\textunderscore .
Obstruír.
Atravancar; obstar a: \textunderscore impediu-lhe a passagem\textunderscore .
\section{Impeditivo}
\begin{itemize}
\item {Grp. gram.:adj.}
\end{itemize}
\begin{itemize}
\item {Proveniência:(Do lat. \textunderscore impeditus\textunderscore )}
\end{itemize}
O mesmo que \textunderscore impediente\textunderscore .
\section{Impelente}
\begin{itemize}
\item {Grp. gram.:adj.}
\end{itemize}
\begin{itemize}
\item {Proveniência:(Lat. \textunderscore impellens\textunderscore )}
\end{itemize}
Que impele.
\section{Impelir}
\begin{itemize}
\item {Grp. gram.:v. t.}
\end{itemize}
\begin{itemize}
\item {Proveniência:(Lat. \textunderscore impellere\textunderscore )}
\end{itemize}
Dirigir com fôrça, empurrar.
Constranger; estimular: \textunderscore impelir ao crime\textunderscore .
\section{Impellente}
\begin{itemize}
\item {Grp. gram.:adj.}
\end{itemize}
\begin{itemize}
\item {Proveniência:(Lat. \textunderscore impellens\textunderscore )}
\end{itemize}
Que impelle.
\section{Impellir}
\begin{itemize}
\item {Grp. gram.:v. t.}
\end{itemize}
\begin{itemize}
\item {Proveniência:(Lat. \textunderscore impellere\textunderscore )}
\end{itemize}
Dirigir com fôrça, empurrar.
Constranger; estimular: \textunderscore impellir ao crime\textunderscore .
\section{Impendente}
\begin{itemize}
\item {Grp. gram.:adj.}
\end{itemize}
\begin{itemize}
\item {Proveniência:(Lat. \textunderscore impendens\textunderscore )}
\end{itemize}
Que está pendente sôbre alguém ou sôbre alguma coisa.
Imminente.
\section{Impender}
\begin{itemize}
\item {Grp. gram.:v. i.}
\end{itemize}
\begin{itemize}
\item {Proveniência:(Lat. \textunderscore impendere\textunderscore )}
\end{itemize}
Estar impendente, prestes a caír ou a acontecer.
Sêr obrigação.
Competir; cumprir: \textunderscore impende-me falar verdade\textunderscore .
\section{Impene}
\begin{itemize}
\item {Grp. gram.:adj.}
\end{itemize}
\begin{itemize}
\item {Proveniência:(De \textunderscore im...\textunderscore  + \textunderscore pennatus\textunderscore )}
\end{itemize}
Diz-se da ave, a que faltam as perfeitas rêmiges.
\section{Impenetrabilidade}
\begin{itemize}
\item {Grp. gram.:f.}
\end{itemize}
\begin{itemize}
\item {Proveniência:(Do lat. \textunderscore impenetrabilis\textunderscore )}
\end{itemize}
Qualidade daquelle ou daquillo que é impenetrável.
\section{Impenetrado}
\begin{itemize}
\item {Grp. gram.:adj.}
\end{itemize}
\begin{itemize}
\item {Proveniência:(De \textunderscore im...\textunderscore  + \textunderscore penetrado\textunderscore )}
\end{itemize}
Que não foi penetrado; em que se não entrou: \textunderscore florestas impenetradas\textunderscore .
Nunca navegado, (falando-se do mar).
\section{Impenetrável}
\begin{itemize}
\item {Grp. gram.:adj.}
\end{itemize}
\begin{itemize}
\item {Proveniência:(Lat. \textunderscore impenetrabilis\textunderscore )}
\end{itemize}
Que se não póde penetrar; incomprehensível: \textunderscore segrêdo impenetrável\textunderscore .
Inexplicável.
Que não expõe o que sente ou pensa.
\section{Impenetravelmente}
\begin{itemize}
\item {Grp. gram.:adv.}
\end{itemize}
De modo impenetrável.
\section{Impenitência}
\begin{itemize}
\item {Grp. gram.:f.}
\end{itemize}
\begin{itemize}
\item {Proveniência:(Lat. \textunderscore impoenitentia\textunderscore )}
\end{itemize}
Falta de penitência ou de arrependimento.
Qualidade ou estado de impenitente.
\section{Impenitente}
\begin{itemize}
\item {Grp. gram.:adj.}
\end{itemize}
\begin{itemize}
\item {Proveniência:(Lat. \textunderscore impoenitens\textunderscore )}
\end{itemize}
Que não é penitente; que não se arrepende.
Que é contumaz no êrro ou no crime.
\section{Impenne}
\begin{itemize}
\item {Grp. gram.:adj.}
\end{itemize}
\begin{itemize}
\item {Proveniência:(De \textunderscore im...\textunderscore  + \textunderscore pennatus\textunderscore )}
\end{itemize}
Diz-se da ave, a que faltam as perfeitas rêmiges.
\section{Impensadamente}
\begin{itemize}
\item {Grp. gram.:adv.}
\end{itemize}
De modo impensado.
Precipitadamente; sem reflexão.
\section{Impensado}
\begin{itemize}
\item {Grp. gram.:adj.}
\end{itemize}
\begin{itemize}
\item {Proveniência:(De \textunderscore im...\textunderscore  + \textunderscore pensado\textunderscore )}
\end{itemize}
Que não é pensado; em que não há cuidado ou cálculo; imprevisto.
\section{Impensável}
\begin{itemize}
\item {Grp. gram.:adj.}
\end{itemize}
\begin{itemize}
\item {Proveniência:(De \textunderscore im...\textunderscore  + \textunderscore pensar\textunderscore )}
\end{itemize}
Que se não póde pensar ou suppor.
\section{Imperador}
\begin{itemize}
\item {Grp. gram.:m.}
\end{itemize}
\begin{itemize}
\item {Utilização:Açor. do Pico}
\end{itemize}
\begin{itemize}
\item {Proveniência:(Lat. \textunderscore imperator\textunderscore )}
\end{itemize}
Aquelle que impera.
Soberano de um império.
Peixe, da fam. dos pércidas.
O mesmo que \textunderscore mordómo\textunderscore  (de festa de igreja).
\section{Imperadora}
\begin{itemize}
\item {Grp. gram.:f.}
\end{itemize}
\begin{itemize}
\item {Utilização:T. us. por clássicos}
\end{itemize}
O mesmo que \textunderscore imperatriz\textunderscore .
\section{Imperante}
\begin{itemize}
\item {Grp. gram.:adj.}
\end{itemize}
\begin{itemize}
\item {Grp. gram.:M.}
\end{itemize}
\begin{itemize}
\item {Proveniência:(Lat. \textunderscore imperans\textunderscore )}
\end{itemize}
Que impera.
O primeiro magistrado de uma monarchia.
Soberano.
\section{Imperar}
\begin{itemize}
\item {Grp. gram.:v. t.}
\end{itemize}
\begin{itemize}
\item {Grp. gram.:V. i.}
\end{itemize}
\begin{itemize}
\item {Proveniência:(Do lat. \textunderscore imperare\textunderscore )}
\end{itemize}
Mandar.
Dar ordens a.
Governar com autoridade suprema.
Exercer o poder supremo.
Dominar; predominar: \textunderscore naquellas reuniões impera a desordem\textunderscore .
\section{Imperativamente}
\begin{itemize}
\item {Grp. gram.:adv.}
\end{itemize}
De modo imperativo; autoritariamente; com arrogância.
\section{Imperativo}
\begin{itemize}
\item {Grp. gram.:adj.}
\end{itemize}
\begin{itemize}
\item {Grp. gram.:M.}
\end{itemize}
\begin{itemize}
\item {Utilização:Gram.}
\end{itemize}
\begin{itemize}
\item {Proveniência:(Lat. \textunderscore imperativus\textunderscore )}
\end{itemize}
Que manda com autoridade.
Que ordena; que governa.
Arrogante.
Despótico.
Modo dos verbos, em que se ordena, exhorta ou pede.
\section{Imperátor}
\begin{itemize}
\item {Grp. gram.:m.}
\end{itemize}
\begin{itemize}
\item {Proveniência:(T. lat.)}
\end{itemize}
Titulo, que o senado romano e ás vezes os próprios soldados conferiam ao chefe que ganhou uma victória, destruindo mais de déz mil inimigos.
\section{Imperatória}
\begin{itemize}
\item {Grp. gram.:f.}
\end{itemize}
Planta umbellífera, (\textunderscore imperatorium ostruthium\textunderscore ).
\section{Imperatoriamente}
\begin{itemize}
\item {Grp. gram.:adv.}
\end{itemize}
De modo imperatório.
\section{Imperatório}
\begin{itemize}
\item {Grp. gram.:adj.}
\end{itemize}
\begin{itemize}
\item {Proveniência:(Lat. \textunderscore imperatorius\textunderscore )}
\end{itemize}
Relativo a imperador, imperial.
Imperativo; terminante.
\section{Imperatriz}
\begin{itemize}
\item {Grp. gram.:f.}
\end{itemize}
\begin{itemize}
\item {Grp. gram.:Adj.}
\end{itemize}
\begin{itemize}
\item {Proveniência:(Lat. \textunderscore imperatrix\textunderscore )}
\end{itemize}
Mulher, que governa um império.
Espôsa de imperador.
Dominadora.
\section{Impercebivel}
\begin{itemize}
\item {Grp. gram.:adj.}
\end{itemize}
\begin{itemize}
\item {Utilização:P. us.}
\end{itemize}
\begin{itemize}
\item {Proveniência:(De \textunderscore im...\textunderscore  + \textunderscore percebivel\textunderscore )}
\end{itemize}
Que se não póde perceber; imperceptível.
\section{Imperceptibilidade}
\begin{itemize}
\item {Grp. gram.:f.}
\end{itemize}
Qualidade daquillo que é imperceptível.
\section{Imperceptível}
\begin{itemize}
\item {Grp. gram.:adj.}
\end{itemize}
\begin{itemize}
\item {Utilização:Fig.}
\end{itemize}
\begin{itemize}
\item {Proveniência:(De \textunderscore im...\textunderscore  + \textunderscore perceptível\textunderscore )}
\end{itemize}
Que não é perceptível; que se não póde perceber.
Que se não avista bem; que mal se distingue.
Insignificante; pequenino.
\section{Imperceptivelmente}
\begin{itemize}
\item {Grp. gram.:adv.}
\end{itemize}
De modo imperceptível.
\section{Imperdível}
\begin{itemize}
\item {Grp. gram.:adj.}
\end{itemize}
\begin{itemize}
\item {Proveniência:(De \textunderscore im...\textunderscore  + \textunderscore perdível\textunderscore )}
\end{itemize}
Que se não póde perder; em que não póde haver prejuizo.
\section{Imperdoável}
\begin{itemize}
\item {Grp. gram.:adj.}
\end{itemize}
\begin{itemize}
\item {Proveniência:(De \textunderscore im...\textunderscore  + \textunderscore perdoável\textunderscore )}
\end{itemize}
Que se não póde perdoar.
Que não merece perdão; condemnável: \textunderscore descuidos imperdoáveis\textunderscore .
\section{Imperecedoiro}
\begin{itemize}
\item {Grp. gram.:adj.}
\end{itemize}
O mesmo que \textunderscore imperecível\textunderscore .
\section{Imperecedor}
\begin{itemize}
\item {Grp. gram.:adj.}
\end{itemize}
O mesmo que \textunderscore imperecível\textunderscore . Cf. Latino, \textunderscore Elog.\textunderscore , 245.
\section{Imperecedouro}
\begin{itemize}
\item {Grp. gram.:adj.}
\end{itemize}
O mesmo que \textunderscore imperecível\textunderscore .
\section{Imperecível}
\begin{itemize}
\item {Grp. gram.:adj.}
\end{itemize}
\begin{itemize}
\item {Proveniência:(De \textunderscore im...\textunderscore  + \textunderscore perecível\textunderscore )}
\end{itemize}
O mesmo que \textunderscore immorredoiro\textunderscore .
\section{Imperfectibilidade}
\begin{itemize}
\item {Grp. gram.:f.}
\end{itemize}
Qualidade daquillo que é imperfectível.
\section{Imperfectível}
\begin{itemize}
\item {Grp. gram.:adj.}
\end{itemize}
\begin{itemize}
\item {Proveniência:(De \textunderscore im...\textunderscore  + \textunderscore perfectível\textunderscore )}
\end{itemize}
Que se não póde aperfeiçoar.
\section{Imperfeição}
\begin{itemize}
\item {Grp. gram.:f.}
\end{itemize}
\begin{itemize}
\item {Proveniência:(Lat. \textunderscore imperfectio\textunderscore )}
\end{itemize}
Qualidade daquelle ou daquillo que é imperfeito.
Falta de perfeição; mancha; pequeno defeito.
\section{Imperfeiçoar}
\begin{itemize}
\item {Grp. gram.:v. t.}
\end{itemize}
\begin{itemize}
\item {Proveniência:(Do lat. \textunderscore imperfectio\textunderscore )}
\end{itemize}
Tornar imperfeito.
\section{Imperfeitamente}
\begin{itemize}
\item {Grp. gram.:adv.}
\end{itemize}
De modo imperfeito; com imperfeição; incompletamente.
\section{Imperfeito}
\begin{itemize}
\item {Grp. gram.:adj.}
\end{itemize}
\begin{itemize}
\item {Utilização:Gram.}
\end{itemize}
\begin{itemize}
\item {Proveniência:(Lat. \textunderscore imperfectus\textunderscore )}
\end{itemize}
Que não é perfeito; que tem defeitos.
Que se não concluiu; incompleto.
Diz-se dos tempos dos verbos, em que se exprime estado ou acção incompleta ou não realizada.
\section{Imperfuração}
\begin{itemize}
\item {Grp. gram.:f.}
\end{itemize}
\begin{itemize}
\item {Proveniência:(Lat. \textunderscore imperforatio\textunderscore )}
\end{itemize}
Occlusão de um orifício que, no corpo humano, devia naturalmente têr communicação com o exterior.
\section{Imperfurado}
\begin{itemize}
\item {Grp. gram.:adj.}
\end{itemize}
\begin{itemize}
\item {Proveniência:(De \textunderscore im...\textunderscore  + \textunderscore perfurado\textunderscore )}
\end{itemize}
Em que há imperfuração.
\section{Imperial}
\begin{itemize}
\item {Grp. gram.:adj.}
\end{itemize}
\begin{itemize}
\item {Utilização:Fam.}
\end{itemize}
\begin{itemize}
\item {Grp. gram.:F.}
\end{itemize}
\begin{itemize}
\item {Utilização:Constr.}
\end{itemize}
\begin{itemize}
\item {Proveniência:(Lat. \textunderscore imperialis\textunderscore )}
\end{itemize}
Relativo a império ou a imperador: \textunderscore corôa imperial\textunderscore .
Autoritário.
Imperioso.
Arrogante.
Espécie de dança ou quadrilha.
Lugar para passageiros ou bagagens, na parte antero-superior de uma carruagem.
Jôgo de cartas, entre dois ou três parceiros, em que ganha quem reúna quatro ou cinco maiores trunfos.
Diz-se de uma espécie de abóbada, composta de quatro porções de abóbada cylíndrica e de uma parte plana.
\section{Imperialismo}
\begin{itemize}
\item {Grp. gram.:m.}
\end{itemize}
\begin{itemize}
\item {Proveniência:(De \textunderscore imperial\textunderscore )}
\end{itemize}
Fórma de govêrno, em que a nação é um império.
Opinião favorável a êste regime.
\section{Imperialista}
\begin{itemize}
\item {Grp. gram.:adj.}
\end{itemize}
\begin{itemize}
\item {Grp. gram.:M.}
\end{itemize}
\begin{itemize}
\item {Proveniência:(De \textunderscore imperial\textunderscore )}
\end{itemize}
Relativo ao imperialismo.
Partidário do imperialismo.
\section{Imperialmente}
\begin{itemize}
\item {Grp. gram.:adv.}
\end{itemize}
De modo imperial.
\section{Imperiante}
\begin{itemize}
\item {Grp. gram.:adj.}
\end{itemize}
\begin{itemize}
\item {Utilização:Neol.}
\end{itemize}
\begin{itemize}
\item {Proveniência:(De \textunderscore império\textunderscore )}
\end{itemize}
Altivo, orgulhoso. Cf. Camillo, \textunderscore Livro Negro\textunderscore , 288.
\section{Imperícia}
\begin{itemize}
\item {Grp. gram.:f.}
\end{itemize}
\begin{itemize}
\item {Proveniência:(Lat. \textunderscore imperitia\textunderscore )}
\end{itemize}
Qualidade de quem é imperito.
Falta de experiência ou de conhecimentos práticos.
\section{Império}
\begin{itemize}
\item {Grp. gram.:m.}
\end{itemize}
\begin{itemize}
\item {Utilização:Mad}
\end{itemize}
\begin{itemize}
\item {Utilização:Açor}
\end{itemize}
\begin{itemize}
\item {Proveniência:(Lat. \textunderscore imperium\textunderscore )}
\end{itemize}
Preceito, ordem, autoridade, que emana de superior para inferior.
Predomínio.
Arrogância.
Dominação.
Poder.
Esphera de poder.
Monarchia, geralmente importante, cujo chefe tem o título de imperador ou imperatriz.
Nação, politicamente organizada, sob qualquer fórma de govêrno.
O mesmo que \textunderscore arraial\textunderscore .
Festividade do Espírito-Santo.
Conjunto dos ornamentos em certas festividades.
\section{Imperiosamente}
\begin{itemize}
\item {Grp. gram.:adv.}
\end{itemize}
De modo imperioso.
Autoritariamente.
\section{Imperiosidade}
\begin{itemize}
\item {Grp. gram.:f.}
\end{itemize}
Qualidade de imperioso.
\section{Imperioso}
\begin{itemize}
\item {Grp. gram.:adj.}
\end{itemize}
\begin{itemize}
\item {Proveniência:(Lat. \textunderscore imperiosus\textunderscore )}
\end{itemize}
Que ordena com império.
Arrogante; soberbo.
Que tem grande influência.
Impreterível.
Que se impõe forçosamente: \textunderscore motivos imperiosos\textunderscore .
\section{Imperitamente}
\begin{itemize}
\item {Grp. gram.:adv.}
\end{itemize}
\begin{itemize}
\item {Proveniência:(De \textunderscore imperito\textunderscore )}
\end{itemize}
Com imperícia.
\section{Imperito}
\begin{itemize}
\item {Grp. gram.:adj.}
\end{itemize}
\begin{itemize}
\item {Proveniência:(Lat. \textunderscore imperitus\textunderscore )}
\end{itemize}
Que não é perito; em que não há perícia.
Que não tem experiência.
Que não sabe; ignorante.
Que não trabalha com perfeição.
\section{Impermanência}
\begin{itemize}
\item {Grp. gram.:f.}
\end{itemize}
Qualidade de impermanente.
\section{Impermanente}
\begin{itemize}
\item {Grp. gram.:adj.}
\end{itemize}
\begin{itemize}
\item {Proveniência:(De \textunderscore im...\textunderscore  + \textunderscore permanente\textunderscore )}
\end{itemize}
Que não é permanente; que é instável; inconstante.
\section{Impermanentemente}
\begin{itemize}
\item {Grp. gram.:adv.}
\end{itemize}
De modo impermanente.
\section{Impermeabilidade}
\begin{itemize}
\item {Grp. gram.:f.}
\end{itemize}
Qualidade de impermeável.
\section{Impermeabilizar}
\begin{itemize}
\item {Grp. gram.:v. t.}
\end{itemize}
Tornar impermeável.
\section{Impermear}
\begin{itemize}
\item {Grp. gram.:v. t.}
\end{itemize}
O mesmo que \textunderscore impermeabilizar\textunderscore .
\section{Impermeável}
\begin{itemize}
\item {Grp. gram.:adj.}
\end{itemize}
\begin{itemize}
\item {Proveniência:(Lat. \textunderscore impermeabilis\textunderscore )}
\end{itemize}
Que não é permeável, que se não deixa atravessar por um flúido.
Que não deixa passar a água ou a humidade: \textunderscore capa impermeável\textunderscore .
\section{Impermeavelmente}
\begin{itemize}
\item {Grp. gram.:adv.}
\end{itemize}
De modo impermeavel.
\section{Impermisto}
\begin{itemize}
\item {Grp. gram.:adj.}
\end{itemize}
\begin{itemize}
\item {Proveniência:(Lat. \textunderscore impermixtus\textunderscore )}
\end{itemize}
Que não é misturado.
\section{Impermixto}
\begin{itemize}
\item {Grp. gram.:adj.}
\end{itemize}
\begin{itemize}
\item {Proveniência:(Lat. \textunderscore impermixtus\textunderscore )}
\end{itemize}
Que não é misturado.
\section{Impermutabilidade}
\begin{itemize}
\item {Grp. gram.:f.}
\end{itemize}
Qualidade de impermutável.
\section{Impermutável}
\begin{itemize}
\item {Grp. gram.:adj.}
\end{itemize}
\begin{itemize}
\item {Proveniência:(Lat. \textunderscore impermutabilis\textunderscore )}
\end{itemize}
Que se não póde permutar.
\section{Imperscrutável}
\begin{itemize}
\item {Grp. gram.:adj.}
\end{itemize}
\begin{itemize}
\item {Proveniência:(Lat. \textunderscore imperscrutabilis\textunderscore )}
\end{itemize}
Que não é perscrutável; que se não póde pesquisar ou examinar.
\section{Impersistente}
\begin{itemize}
\item {Grp. gram.:adj.}
\end{itemize}
\begin{itemize}
\item {Proveniência:(De \textunderscore im...\textunderscore  + \textunderscore persistente\textunderscore )}
\end{itemize}
Que não é persistente; que não é immudável; inconstante.
\section{Impersonalidade}
\begin{itemize}
\item {Grp. gram.:f.}
\end{itemize}
Qualidade de impessoal.
(B. lat. \textunderscore impersonalitas\textunderscore )
\section{Imperterritamente}
\begin{itemize}
\item {Grp. gram.:adv.}
\end{itemize}
\begin{itemize}
\item {Proveniência:(De \textunderscore impertérrito\textunderscore )}
\end{itemize}
Sem medo.
Com intrepidez.
\section{Impertérrito}
\begin{itemize}
\item {Grp. gram.:adj.}
\end{itemize}
\begin{itemize}
\item {Proveniência:(Lat. \textunderscore imperterritus\textunderscore )}
\end{itemize}
Que se não aterra com coisa nenhuma; intrépido.
\section{Impertinência}
\begin{itemize}
\item {Grp. gram.:f.}
\end{itemize}
Qualidade ou estado de impertinente.
Inopportunidade.
Coisa que incommóda ou molesta.
\section{Impertinenciar}
\begin{itemize}
\item {Grp. gram.:v. t.}
\end{itemize}
Tratar com impertinência. Cf. Arn. Gama, \textunderscore Motim\textunderscore , 126.
\section{Impertinente}
\begin{itemize}
\item {Grp. gram.:adj.}
\end{itemize}
\begin{itemize}
\item {Grp. gram.:M.}
\end{itemize}
\begin{itemize}
\item {Proveniência:(Lat. \textunderscore impertinens\textunderscore )}
\end{itemize}
Estranho ao assumpto.
Que não vem a propósito: \textunderscore objecção impertinente\textunderscore .
Inopportuno.
Importuno.
Aborrecido.
Rabujento.
Incômmodo, molesto.
Pessôa importuna, molesta.
\section{Impertinentemente}
\begin{itemize}
\item {Grp. gram.:adv.}
\end{itemize}
De modo impertinente.
\section{Imperturbabilidade}
\begin{itemize}
\item {Grp. gram.:f.}
\end{itemize}
\begin{itemize}
\item {Proveniência:(Do lat. \textunderscore impertubabilis\textunderscore )}
\end{itemize}
Qualidade de imperturbável.
\section{Imperturbado}
\begin{itemize}
\item {Grp. gram.:adj.}
\end{itemize}
\begin{itemize}
\item {Proveniência:(Lat. \textunderscore imperturbatus\textunderscore )}
\end{itemize}
Que se não perturba; em que não há perturbação.
\section{Imperturbável}
\begin{itemize}
\item {Grp. gram.:adj.}
\end{itemize}
\begin{itemize}
\item {Proveniência:(Lat. \textunderscore imperturbabilis\textunderscore )}
\end{itemize}
Que se não perturba.
Impassível.
Magnânimo; corajoso.
\section{Imperturbavelmente}
\begin{itemize}
\item {Grp. gram.:adv.}
\end{itemize}
De modo imperturbável.
\section{Impérvio}
\begin{itemize}
\item {Grp. gram.:adj.}
\end{itemize}
\begin{itemize}
\item {Grp. gram.:M.}
\end{itemize}
\begin{itemize}
\item {Proveniência:(Lat. \textunderscore impervius\textunderscore )}
\end{itemize}
Que não deixa transitar; intransitável.
Impenetrável: \textunderscore selvas impérvias\textunderscore .
Inaccessível: \textunderscore montanhas impérvias\textunderscore .
Lugar intransitável ou em que não há caminho.
\section{Impessoal}
\begin{itemize}
\item {Grp. gram.:adj.}
\end{itemize}
\begin{itemize}
\item {Utilização:Gram.}
\end{itemize}
\begin{itemize}
\item {Proveniência:(Lat. \textunderscore impersonalis\textunderscore )}
\end{itemize}
Que não é pessoal; que se não refere a pessôa ou pessôas: \textunderscore referências impessoaes\textunderscore .
Que não existe como pessôa.
Diz-se dos verbos, em cujos tempos se não designam todas as pessôas.
Segundo alguns grammáticos, diz-se do modo infinitivo, em que não há flexões para designar pessôas.
\section{Impessoalidade}
\begin{itemize}
\item {Grp. gram.:f.}
\end{itemize}
Qualidade de impessoal.
\section{Impessoalmente}
\begin{itemize}
\item {Grp. gram.:adv.}
\end{itemize}
De modo impessoal.
\section{Impetar}
\begin{itemize}
\item {Grp. gram.:v. t.}
\end{itemize}
\begin{itemize}
\item {Proveniência:(De \textunderscore ímpeto\textunderscore )}
\end{itemize}
Arremessar.
Dar impetuosamente. Cf. Frei Fortun., \textunderscore Inéd.\textunderscore , I, 308; Viterbo, \textunderscore Elucid.\textunderscore 
\section{Impeticar}
\begin{itemize}
\item {Grp. gram.:v. i.}
\end{itemize}
Contender, implicar. Cf. Camillo, \textunderscore Narcót.\textunderscore , I, 235; \textunderscore Brasileira\textunderscore , 24.
(Freq. de \textunderscore impetar\textunderscore )
\section{Impetiginoso}
\begin{itemize}
\item {Grp. gram.:adj.}
\end{itemize}
\begin{itemize}
\item {Proveniência:(Lat. \textunderscore impetiginosus\textunderscore )}
\end{itemize}
Relativo a impetigo; que tem a natureza do impetigo.
\section{Impetigo}
\begin{itemize}
\item {Grp. gram.:m.}
\end{itemize}
\begin{itemize}
\item {Proveniência:(Lat. \textunderscore impetigo\textunderscore )}
\end{itemize}
O mesmo que \textunderscore impigem\textunderscore .
Sarna.
\section{Ímpeto}
\begin{itemize}
\item {Grp. gram.:m.}
\end{itemize}
\begin{itemize}
\item {Utilização:Fig.}
\end{itemize}
\begin{itemize}
\item {Proveniência:(Lat. \textunderscore impetus\textunderscore )}
\end{itemize}
Movimento repentino.
Impulso violento.
Precipitação, arrebatamento.
Agitação de espírito; violência de sentimentos.
\section{Impetra}
\begin{itemize}
\item {Grp. gram.:f.}
\end{itemize}
\begin{itemize}
\item {Proveniência:(De \textunderscore impetrar\textunderscore )}
\end{itemize}
Rôgo; súpplica.
Consecução de um benefício ecclesiástico, concedido pelo Papa.
\section{Impetrabilidade}
\begin{itemize}
\item {Grp. gram.:f.}
\end{itemize}
\begin{itemize}
\item {Proveniência:(Do lat. \textunderscore impetrabilis\textunderscore )}
\end{itemize}
Qualidade do que é impetrável.
\section{Impetração}
\begin{itemize}
\item {Grp. gram.:f.}
\end{itemize}
\begin{itemize}
\item {Proveniência:(Lat. \textunderscore impetratio\textunderscore )}
\end{itemize}
Acto ou effeito de impetrar.
\section{Impetrante}
\begin{itemize}
\item {Grp. gram.:m. ,  f.  e  adj.}
\end{itemize}
\begin{itemize}
\item {Proveniência:(Lat. \textunderscore impetrans\textunderscore )}
\end{itemize}
Pessôa que impetra.
\section{Impetrar}
\begin{itemize}
\item {Grp. gram.:v. t.}
\end{itemize}
\begin{itemize}
\item {Proveniência:(Lat. \textunderscore impetrare\textunderscore )}
\end{itemize}
Rogar.
Requerer.
Supplicar.
Obter por meio de súpplicas.
\section{Impetrativo}
\begin{itemize}
\item {Grp. gram.:adj.}
\end{itemize}
\begin{itemize}
\item {Proveniência:(Lat. \textunderscore impetrativus\textunderscore )}
\end{itemize}
Próprio para impetrar.
\section{Impetratório}
\begin{itemize}
\item {Grp. gram.:adj.}
\end{itemize}
O mesmo que \textunderscore impetrativo\textunderscore .
\section{Impetrável}
\begin{itemize}
\item {Grp. gram.:adj.}
\end{itemize}
\begin{itemize}
\item {Proveniência:(Lat. \textunderscore impetrabilis\textunderscore )}
\end{itemize}
Que se póde impetrar.
\section{Impetravelmente}
\begin{itemize}
\item {Grp. gram.:adv.}
\end{itemize}
De modo impetrável.
\section{Impetuosamente}
\begin{itemize}
\item {Grp. gram.:adv.}
\end{itemize}
De modo impetuoso.
\section{Impetuosidade}
\begin{itemize}
\item {Grp. gram.:f.}
\end{itemize}
Qualidade ou estado de impetuoso.
\section{Impetuoso}
\begin{itemize}
\item {Grp. gram.:adj.}
\end{itemize}
\begin{itemize}
\item {Proveniência:(Lat. \textunderscore impetuosus\textunderscore )}
\end{itemize}
Que tem ímpeto.
Que se move com ímpeto: \textunderscore corrente impetuosa\textunderscore .
Arrebatado.
Agitado.
Fogoso; irritado: \textunderscore gênio impetuoso\textunderscore .
\section{Impiamente}
\begin{itemize}
\item {Grp. gram.:adv.}
\end{itemize}
De modo ímpio.
\section{Impidoso}
\begin{itemize}
\item {Grp. gram.:adj.}
\end{itemize}
\begin{itemize}
\item {Utilização:Ant.}
\end{itemize}
\begin{itemize}
\item {Proveniência:(De \textunderscore ímpedir\textunderscore , na hypóthese do pres. do indic. \textunderscore impido\textunderscore )}
\end{itemize}
Que impede; que tem obstáculos.
Agastadiço. Cf. Filinto, VII, 311.
\section{Impiedade}
\begin{itemize}
\item {Grp. gram.:f.}
\end{itemize}
\begin{itemize}
\item {Utilização:Fig.}
\end{itemize}
\begin{itemize}
\item {Proveniência:(Lat. \textunderscore impietas\textunderscore )}
\end{itemize}
Qualidade de ímpio; falta de piedade.
Acto ou expressão ímpia.
Crueldade.
\section{Impiedosamente}
\begin{itemize}
\item {Grp. gram.:adv.}
\end{itemize}
De modo impiedoso.
Deshumanamente.
\section{Impiedoso}
\begin{itemize}
\item {Grp. gram.:adj.}
\end{itemize}
\begin{itemize}
\item {Proveniência:(De \textunderscore im...\textunderscore  + \textunderscore piedoso\textunderscore )}
\end{itemize}
Que não tem piedade: \textunderscore homem impiedoso\textunderscore .
Em que não há piedade: \textunderscore procedimento impiedoso\textunderscore .
Insensível; deshumano.
\section{Impigem}
\begin{itemize}
\item {Grp. gram.:f.}
\end{itemize}
\begin{itemize}
\item {Proveniência:(Do lat. \textunderscore impetigo\textunderscore )}
\end{itemize}
Erupção cutânea, caracterizada por crostas ou escamas amareladas ou gretadas.
\section{Impingidela}
\begin{itemize}
\item {Grp. gram.:f.}
\end{itemize}
\begin{itemize}
\item {Utilização:Fam.}
\end{itemize}
Acto ou effeito de impingir.
\section{Impingir}
\begin{itemize}
\item {Grp. gram.:v. t.}
\end{itemize}
\begin{itemize}
\item {Proveniência:(Lat. \textunderscore impingere\textunderscore )}
\end{itemize}
Dar com fôrça.
Dar á fôrça, ou sem vontade de quem recebe.
Pespegar.
Obrigar ou constranger alguém a ouvir: \textunderscore impingir conselhos\textunderscore .
Vender por maior preço que o razoável.
\section{Impinguar}
\begin{itemize}
\item {Grp. gram.:v. i.}
\end{itemize}
\begin{itemize}
\item {Utilização:Ant.}
\end{itemize}
\begin{itemize}
\item {Proveniência:(Lat. \textunderscore impinguare\textunderscore )}
\end{itemize}
Tornar-se pingue, engordar.
\section{Ímpio}
\begin{itemize}
\item {Grp. gram.:adj.}
\end{itemize}
\begin{itemize}
\item {Grp. gram.:M.}
\end{itemize}
\begin{itemize}
\item {Proveniência:(Lat. \textunderscore impius\textunderscore )}
\end{itemize}
Que não é pio, que não tem piedade.
Que não tem religião; que não respeita as coisas sagradas: \textunderscore homem ímpio\textunderscore .
Contrário á religião: \textunderscore palavras ímpias\textunderscore .
Deshumano, cruel.
Homem sem piedade; hereje; atheu.
\section{Implacabilidade}
\begin{itemize}
\item {Grp. gram.:f.}
\end{itemize}
Qualidade de implacável.
\section{Implacável}
\begin{itemize}
\item {Grp. gram.:adj.}
\end{itemize}
\begin{itemize}
\item {Proveniência:(Lat. \textunderscore implacabilis\textunderscore )}
\end{itemize}
Que se não póde applacar.
Inexorável; insensível.
Que não perdôa.
\section{Implacavelmente}
\begin{itemize}
\item {Grp. gram.:adv.}
\end{itemize}
De modo implacável.
\section{Implacidez}
\begin{itemize}
\item {Grp. gram.:f.}
\end{itemize}
\begin{itemize}
\item {Proveniência:(De \textunderscore im...\textunderscore  + \textunderscore placidez\textunderscore )}
\end{itemize}
Falta de placidez.
\section{Implantação}
\begin{itemize}
\item {Grp. gram.:f.}
\end{itemize}
Acto de implantar.
\section{Implantar}
\begin{itemize}
\item {Grp. gram.:v. t.}
\end{itemize}
\begin{itemize}
\item {Utilização:Fig.}
\end{itemize}
\begin{itemize}
\item {Proveniência:(De \textunderscore im...\textunderscore  + \textunderscore plantar\textunderscore )}
\end{itemize}
Inserir.
Fixar.
Arraigar.
Inaugurar; estabelecer: \textunderscore a revolução implantou a repùblica\textunderscore .
\section{Implante}
\begin{itemize}
\item {Grp. gram.:m.}
\end{itemize}
O mesmo que \textunderscore implantação\textunderscore .
\section{Implemento}
\begin{itemize}
\item {Grp. gram.:m.}
\end{itemize}
\begin{itemize}
\item {Proveniência:(Lat. \textunderscore implementum\textunderscore )}
\end{itemize}
Aquillo que completa ou perfaz alguma coisa.
Petrechos.
Aquillo que serve para cumprir ou executar.
\section{Implexo}
\begin{itemize}
\item {Grp. gram.:adj.}
\end{itemize}
\begin{itemize}
\item {Proveniência:(Lat. \textunderscore implexus\textunderscore )}
\end{itemize}
Envolvido.
Entretecido; complicado.
\section{Implicação}
\begin{itemize}
\item {Grp. gram.:f.}
\end{itemize}
\begin{itemize}
\item {Proveniência:(Lat. \textunderscore implicatio\textunderscore )}
\end{itemize}
Acto ou effeito de implicar.
\section{Implicador}
\begin{itemize}
\item {Grp. gram.:m.  e  adj.}
\end{itemize}
O que implica.
\section{Implicância}
\begin{itemize}
\item {Grp. gram.:f.}
\end{itemize}
O mesmo que \textunderscore implicação\textunderscore .
Repugnância, má vontade. Cf. Camillo, \textunderscore Caveira\textunderscore , 228.
\section{Implicante}
\begin{itemize}
\item {Grp. gram.:m.  e  adj.}
\end{itemize}
\begin{itemize}
\item {Proveniência:(Lat. \textunderscore implicans\textunderscore )}
\end{itemize}
O mesmo que \textunderscore implicador\textunderscore .
\section{Implicar}
\begin{itemize}
\item {Grp. gram.:v. t.}
\end{itemize}
\begin{itemize}
\item {Grp. gram.:V. i.  e  p.}
\end{itemize}
\begin{itemize}
\item {Proveniência:(Lat. \textunderscore implicare\textunderscore )}
\end{itemize}
Impedir.
Enredar.
Dar a entender, fazer suppor: \textunderscore êsse procedimento implica toleima\textunderscore .
Envolver: \textunderscore as suas palavras implicam injúria\textunderscore .
Tornar indispensável.
Têr como consequência: \textunderscore tal luxo implica grandes despesas\textunderscore .
Tornar contradictório.
Comprometer: \textunderscore ficou implicado no crime\textunderscore .
Contender; armar desordem.
Sêr incompatível; não se ligar bem; sêr inconsequente ou contradictório: \textunderscore a estroinice implica com a tua avançada idade\textunderscore .
\section{Implicativo}
\begin{itemize}
\item {Grp. gram.:adj.}
\end{itemize}
Que implica; que produz implicação.
\section{Implicatório}
\begin{itemize}
\item {Grp. gram.:adj.}
\end{itemize}
O mesmo que \textunderscore implicativo\textunderscore .
\section{Implicitamente}
\begin{itemize}
\item {Grp. gram.:adv.}
\end{itemize}
De modo implícito; tacitamente.
\section{Implícito}
\begin{itemize}
\item {Grp. gram.:adj.}
\end{itemize}
\begin{itemize}
\item {Proveniência:(Lat. \textunderscore implicitus\textunderscore )}
\end{itemize}
Que está envolvido ou continuado, mas não expresso claramente; subentendido.
\section{Imploração}
\begin{itemize}
\item {Grp. gram.:f.}
\end{itemize}
\begin{itemize}
\item {Proveniência:(Lat. \textunderscore imploratio\textunderscore )}
\end{itemize}
Acto de implorar.
Súpplica.
\section{Implorador}
\begin{itemize}
\item {Grp. gram.:m.  e  adj.}
\end{itemize}
\begin{itemize}
\item {Proveniência:(Lat. \textunderscore implorans\textunderscore )}
\end{itemize}
Pessôa que implora.
\section{Implorante}
\begin{itemize}
\item {Grp. gram.:m.  e  adj.}
\end{itemize}
\begin{itemize}
\item {Proveniência:(Lat. \textunderscore implorans\textunderscore )}
\end{itemize}
Pessôa que implora.
\section{Implorar}
\begin{itemize}
\item {Grp. gram.:v. t.}
\end{itemize}
\begin{itemize}
\item {Proveniência:(Lat. \textunderscore implorare\textunderscore )}
\end{itemize}
Chamar em auxílio, chorando.
Supplicar; pedir encarecidamente, ou com humildade: \textunderscore implorar piedade\textunderscore .
\section{Implorar}
\begin{itemize}
\item {Grp. gram.:v. t.}
\end{itemize}
\begin{itemize}
\item {Utilização:T. da Bairrada}
\end{itemize}
Gabar muito, exaltar.
(Provavelmente, corr. de \textunderscore empoleirar\textunderscore )
\section{Implorativamente}
\begin{itemize}
\item {Grp. gram.:adv.}
\end{itemize}
De modo implorativo; á maneira de quem implora.
\section{Implorativo}
\begin{itemize}
\item {Grp. gram.:adj.}
\end{itemize}
\begin{itemize}
\item {Proveniência:(De \textunderscore implorar\textunderscore )}
\end{itemize}
Que envolve ou revela imploração ou súpplica.
Que tem o modo de quem implora.
\section{Implorável}
\begin{itemize}
\item {Grp. gram.:adj.}
\end{itemize}
\begin{itemize}
\item {Proveniência:(Lat. \textunderscore implorabilis\textunderscore )}
\end{itemize}
Que se póde implorar.
\section{Implume}
\begin{itemize}
\item {Grp. gram.:adj.}
\end{itemize}
\begin{itemize}
\item {Proveniência:(Lat. \textunderscore implumis\textunderscore )}
\end{itemize}
Que ainda não tem pennas formadas.
Que por condição própria não tem pennas.
\section{Implúvia}
\begin{itemize}
\item {Grp. gram.:f.}
\end{itemize}
\begin{itemize}
\item {Proveniência:(Lat. \textunderscore impluvia\textunderscore )}
\end{itemize}
Parece que era, entre os Romanos, vestimenta sacerdotal para o tempo de chuva. Cf. \textunderscore Suppl. ao Dicc. Port.\textunderscore  Cf. Freund, \textunderscore Grand Diction. de la lang. lat.\textunderscore 
\section{Implúvio}
\begin{itemize}
\item {Grp. gram.:m.}
\end{itemize}
\begin{itemize}
\item {Proveniência:(Lat. \textunderscore impluvium\textunderscore )}
\end{itemize}
Pátio descoberto, em meio das casas, para o qual escorre a chuva dos telhados.
\section{Impo}
\begin{itemize}
\item {Grp. gram.:m.}
\end{itemize}
\begin{itemize}
\item {Utilização:Prov.}
\end{itemize}
\begin{itemize}
\item {Utilização:alg.}
\end{itemize}
\begin{itemize}
\item {Utilização:Prov.}
\end{itemize}
\begin{itemize}
\item {Utilização:trasm.}
\end{itemize}
Acto ou effeito de impar.
Soluço, com que ficam as crianças, depois de chorar.
(Cp. cast. \textunderscore hipo\textunderscore )
\section{Impoético}
\begin{itemize}
\item {Grp. gram.:adj.}
\end{itemize}
O mesmo que \textunderscore antipoético\textunderscore . Cf. Castilho, \textunderscore Métam.\textunderscore , 312.
\section{Impol}
\begin{itemize}
\item {Grp. gram.:m.}
\end{itemize}
Árvore da Índia portuguesa.
\section{Impolarizável}
\begin{itemize}
\item {Grp. gram.:adj.}
\end{itemize}
\begin{itemize}
\item {Proveniência:(De \textunderscore im...\textunderscore  + \textunderscore polarizável\textunderscore )}
\end{itemize}
Que se não póde polarizar.
\section{Impolidamente}
\begin{itemize}
\item {Grp. gram.:adv.}
\end{itemize}
De modo impolido.
\section{Impolidez}
\begin{itemize}
\item {Grp. gram.:f.}
\end{itemize}
Qualidade do que é impolido.
Indelicadeza.
\section{Impolido}
\begin{itemize}
\item {Grp. gram.:adj.}
\end{itemize}
\begin{itemize}
\item {Proveniência:(Lat. \textunderscore impolitus\textunderscore )}
\end{itemize}
Que não é polido; indelicado; grosseiro; descortês.
\section{Impolítica}
\begin{itemize}
\item {Grp. gram.:f.}
\end{itemize}
\begin{itemize}
\item {Proveniência:(De \textunderscore impolítico\textunderscore )}
\end{itemize}
Falta de política ou de cortesia.
\section{Impoliticamente}
\begin{itemize}
\item {Grp. gram.:adv.}
\end{itemize}
De modo impolítico.
Impolidamente.
\section{Impolítico}
\begin{itemize}
\item {Grp. gram.:adj.}
\end{itemize}
\begin{itemize}
\item {Utilização:Fig.}
\end{itemize}
\begin{itemize}
\item {Proveniência:(De \textunderscore im...\textunderscore  + \textunderscore político\textunderscore )}
\end{itemize}
Que não é político; contrário á bôa política.
Incivil; descortês.
\section{Impolluível}
\begin{itemize}
\item {Grp. gram.:adj.}
\end{itemize}
\begin{itemize}
\item {Proveniência:(De \textunderscore im...\textunderscore  + \textunderscore polluível\textunderscore )}
\end{itemize}
Que não é susceptível de se polluír; immaculável.
\section{Impolluto}
\begin{itemize}
\item {Grp. gram.:adj.}
\end{itemize}
\begin{itemize}
\item {Proveniência:(Lat. \textunderscore impollutus\textunderscore )}
\end{itemize}
Que não é polluído; immaculado; virtuoso: \textunderscore vida impolluta\textunderscore .
\section{Impolto}
\begin{itemize}
\item {Grp. gram.:m.}
\end{itemize}
\begin{itemize}
\item {Utilização:Prov.}
\end{itemize}
\begin{itemize}
\item {Utilização:trasm.}
\end{itemize}
Cada uma das peças de madeira, que se metem entre o mile e as cambas, se estas não são sufficientes para completar o círculo.
\section{Impoluível}
\begin{itemize}
\item {Grp. gram.:adj.}
\end{itemize}
\begin{itemize}
\item {Proveniência:(De \textunderscore im...\textunderscore  + \textunderscore poluível\textunderscore )}
\end{itemize}
Que não é susceptível de se poluír; imaculável.
\section{Impoluto}
\begin{itemize}
\item {Grp. gram.:adj.}
\end{itemize}
\begin{itemize}
\item {Proveniência:(Lat. \textunderscore impollutus\textunderscore )}
\end{itemize}
Que não é poluído; imaculado; virtuoso: \textunderscore vida impoluta\textunderscore .
\section{Imponderabilidade}
\begin{itemize}
\item {Grp. gram.:f.}
\end{itemize}
Qualidade de imponderável.
\section{Imponderado}
\begin{itemize}
\item {Grp. gram.:adj.}
\end{itemize}
\begin{itemize}
\item {Proveniência:(De \textunderscore im...\textunderscore  + \textunderscore ponderado\textunderscore )}
\end{itemize}
Que não tem ponderação.
Que revela falta de consideração; feito sem reflexão: \textunderscore acção imponderada\textunderscore .
\section{Imponderável}
\begin{itemize}
\item {Grp. gram.:adj.}
\end{itemize}
\begin{itemize}
\item {Utilização:Fig.}
\end{itemize}
\begin{itemize}
\item {Grp. gram.:M. pl.}
\end{itemize}
\begin{itemize}
\item {Proveniência:(De \textunderscore im...\textunderscore  + \textunderscore ponderável\textunderscore )}
\end{itemize}
Que se não póde pesar.
Que não tem pêso apreciável.
Que se não póde avaliar.
Que não merece ponderação.
Fluidos, cuja materialidade não póde sêr revelada pelos instrumentos conhecidos.
\section{Imponderavelmente}
\begin{itemize}
\item {Grp. gram.:adv.}
\end{itemize}
De modo imponderável.
\section{Imponência}
\begin{itemize}
\item {Grp. gram.:f.}
\end{itemize}
\begin{itemize}
\item {Utilização:Neol.}
\end{itemize}
Qualidade de imponente.
\section{Imponente}
\begin{itemize}
\item {Grp. gram.:adj.}
\end{itemize}
\begin{itemize}
\item {Utilização:Neol.}
\end{itemize}
\begin{itemize}
\item {Proveniência:(Lat. \textunderscore imponens\textunderscore )}
\end{itemize}
Que impõe a sua importância.
Altivo; magnificente; grandioso: \textunderscore uma festa imponente\textunderscore .
\section{Impontar}
\begin{itemize}
\item {Grp. gram.:v. t.}
\end{itemize}
\begin{itemize}
\item {Utilização:Pop.}
\end{itemize}
Fazer sair (alguém): \textunderscore impontei a minha criada\textunderscore .
Ordenar a (alguém) que vá para alguma parte. Cf. Rui Barb., \textunderscore Réplica\textunderscore , 157.
\section{Imponteiro}
\begin{itemize}
\item {Grp. gram.:m.}
\end{itemize}
\begin{itemize}
\item {Utilização:T. de Moçambique}
\end{itemize}
O mesmo que \textunderscore imbondeiro\textunderscore .
\section{Impontual}
\begin{itemize}
\item {Grp. gram.:adj.}
\end{itemize}
\begin{itemize}
\item {Proveniência:(De \textunderscore im...\textunderscore  + \textunderscore pontual\textunderscore )}
\end{itemize}
Que não é pontual; que não cumpre aquillo a que se obrigou.
\section{Impontualidade}
\begin{itemize}
\item {Grp. gram.:f.}
\end{itemize}
Qualidade de quem é impontual.
\section{Impopular}
\begin{itemize}
\item {Grp. gram.:adj.}
\end{itemize}
\begin{itemize}
\item {Proveniência:(De \textunderscore im...\textunderscore  + \textunderscore popular\textunderscore )}
\end{itemize}
Que não é popular.
\section{Impopularidade}
\begin{itemize}
\item {Grp. gram.:f.}
\end{itemize}
Qualidade de impopular.
\section{Impor}
\begin{itemize}
\item {Grp. gram.:v. t.}
\end{itemize}
\begin{itemize}
\item {Grp. gram.:V. i.}
\end{itemize}
\begin{itemize}
\item {Grp. gram.:V. p.}
\end{itemize}
\begin{itemize}
\item {Proveniência:(Lat. \textunderscore imponere\textunderscore )}
\end{itemize}
Pôr em.
Sobrepor.
Fixar.
Estabelecer.
Obrigar a.
Infligir: \textunderscore impor castigo\textunderscore .
Imputar: \textunderscore impor responsabilidades\textunderscore .
Inculcar.
Fazer retirar, despedir: \textunderscore impor um hóspede\textunderscore .
Enganar com bons modos.
Illudir.
Fingir qualidades que não tem.
Arrogar-se qualidades que não possue.
Obrigar os outros a sêr bem acceito por êlles, respeitado, etc.
\section{Imporém}
\begin{itemize}
\item {Grp. gram.:m.}
\end{itemize}
\begin{itemize}
\item {Utilização:Prov.}
\end{itemize}
\begin{itemize}
\item {Utilização:trasm.}
\end{itemize}
Obstáculo, estôrvo, óbice.
Pessôa fraquinha, magrizela.
\section{Importação}
\begin{itemize}
\item {Grp. gram.:f.}
\end{itemize}
Acto ou effeito de importar.
Aquillo que se importou.
Introducção de mercadorias num país, procedentes de outro: \textunderscore a importação, êste anno, foi escassa\textunderscore .
Introducção de usos, de uma epidemia, etc., trazidos de país estranho.
\section{Importador}
\begin{itemize}
\item {Grp. gram.:adj.}
\end{itemize}
\begin{itemize}
\item {Grp. gram.:M.}
\end{itemize}
Que importa.
Aquelle que importa ou traz de fóra.
\section{Importância}
\begin{itemize}
\item {Grp. gram.:f.}
\end{itemize}
Qualidade de importante.
Grande valor.
Qualquer quantia.
Autoridade; influência: \textunderscore homem de importância\textunderscore .
Conceito elevado ou lisonjeiro.
(B. lat. \textunderscore importantia\textunderscore )
\section{Importante}
\begin{itemize}
\item {Grp. gram.:adj.}
\end{itemize}
\begin{itemize}
\item {Grp. gram.:M.}
\end{itemize}
\begin{itemize}
\item {Proveniência:(Lat. \textunderscore importans\textunderscore )}
\end{itemize}
Que importa.
Que se impõe, que merece consideração: \textunderscore uma obra importante\textunderscore .
Indispensável.
Essencial: \textunderscore o que é importante é que não justificas o teu acto\textunderscore .
Aquillo que é essencial ou que mais interessa.
\section{Importantemente}
\begin{itemize}
\item {Grp. gram.:adv.}
\end{itemize}
De modo importante.
\section{Importar}
\begin{itemize}
\item {Grp. gram.:v. t.}
\end{itemize}
\begin{itemize}
\item {Utilização:Fig.}
\end{itemize}
\begin{itemize}
\item {Grp. gram.:V. i.}
\end{itemize}
\begin{itemize}
\item {Proveniência:(Lat. \textunderscore importare\textunderscore )}
\end{itemize}
Trazer de fóra.
Fazer vir de país estranho.
Introduzir.
Têr como resultado, produzir: \textunderscore o mentir importa descrédito\textunderscore .
Têr importância.
Attingir certo custo ou preço: \textunderscore êste fato importou em déz libras\textunderscore .
Valer.
Sêr necessário; convir: \textunderscore importa olhar para o futuro\textunderscore .
\section{Importável}
\begin{itemize}
\item {Grp. gram.:adj.}
\end{itemize}
\begin{itemize}
\item {Proveniência:(De \textunderscore importar\textunderscore )}
\end{itemize}
Que póde sêr importado ou introduzido.
\section{Importe}
\begin{itemize}
\item {Grp. gram.:m.}
\end{itemize}
\begin{itemize}
\item {Proveniência:(De \textunderscore importar\textunderscore )}
\end{itemize}
Custo, importância total.
\section{Importunação}
\begin{itemize}
\item {Grp. gram.:f.}
\end{itemize}
Acto de importunar; impertinência.
\section{Importunador}
\begin{itemize}
\item {Grp. gram.:adj.}
\end{itemize}
\begin{itemize}
\item {Grp. gram.:M.}
\end{itemize}
Que importuna.
Aquelle que importuna.
\section{Importunamente}
\begin{itemize}
\item {Grp. gram.:adv.}
\end{itemize}
De modo importuno.
\section{Importunância}
\begin{itemize}
\item {Grp. gram.:f.}
\end{itemize}
\begin{itemize}
\item {Utilização:Bras. do N}
\end{itemize}
O mesmo que \textunderscore importunação\textunderscore .
\section{Importunar}
\begin{itemize}
\item {Grp. gram.:v. t.}
\end{itemize}
\begin{itemize}
\item {Proveniência:(De \textunderscore importuno\textunderscore )}
\end{itemize}
Incommodar com instâncias ou súpplicas repetidas.
Enfadar.
Sêr molesto a.
Causar transtôrno a.
\section{Importunidade}
\begin{itemize}
\item {Grp. gram.:f.}
\end{itemize}
\begin{itemize}
\item {Proveniência:(Lat. \textunderscore importunitas\textunderscore )}
\end{itemize}
Qualidade de importuno.
Acto importuno.
\section{Importuno}
\begin{itemize}
\item {Grp. gram.:adj.}
\end{itemize}
\begin{itemize}
\item {Proveniência:(Lat. \textunderscore importunus\textunderscore )}
\end{itemize}
Que importuna.
Incômmodo.
Molesto; maçador; insupportável.
\section{Imposição}
\begin{itemize}
\item {Grp. gram.:f.}
\end{itemize}
\begin{itemize}
\item {Proveniência:(Lat. \textunderscore impositio\textunderscore )}
\end{itemize}
Acto ou effeito de impor: \textunderscore imposição de sellos nas portas\textunderscore .
Acto de obrigar: \textunderscore imposição de encargos\textunderscore .
Acto de infligir: \textunderscore imposição de penas\textunderscore .
Collação de insígnias.
Acto de deferir.
\section{Impossança}
\begin{itemize}
\item {Grp. gram.:f.}
\end{itemize}
Falta de possança.
Impotência.
\section{Impossibilidade}
\begin{itemize}
\item {Grp. gram.:f.}
\end{itemize}
\begin{itemize}
\item {Proveniência:(Lat. \textunderscore impossibilitas\textunderscore )}
\end{itemize}
Qualidade de impossível.
\section{Impossibilitar}
\begin{itemize}
\item {Grp. gram.:v. t.}
\end{itemize}
\begin{itemize}
\item {Proveniência:(Do lat. \textunderscore impossibilis\textunderscore )}
\end{itemize}
Tornar impossível.
Mostrar como impossível.
Fazer perder as fôrças ou a aptidão de: \textunderscore a cegueira impossibilitou-o de lêr\textunderscore .
\section{Impossível}
\begin{itemize}
\item {Grp. gram.:adj.}
\end{itemize}
\begin{itemize}
\item {Grp. gram.:M.}
\end{itemize}
\begin{itemize}
\item {Proveniência:(Lat. \textunderscore impossibilis\textunderscore )}
\end{itemize}
Que não é possível; muito diffícil.
Incrível.
Extravagante.
Insupportável: \textunderscore um maçador impossível\textunderscore .
Aquillo que não é possível ou que é muito diffícil: \textunderscore tentarei o impossível para lhe agradar\textunderscore .
\section{Imposta}
\begin{itemize}
\item {Grp. gram.:f.}
\end{itemize}
\begin{itemize}
\item {Utilização:Des.}
\end{itemize}
\begin{itemize}
\item {Proveniência:(Do lat. \textunderscore impositus\textunderscore )}
\end{itemize}
Cornija, que serve de base a um arco.
O mesmo que \textunderscore encosta\textunderscore :«\textunderscore pôs-se de vigia em huma emposta de terra\textunderscore ». Filinto, \textunderscore D. Man.\textunderscore , I, 217.
\section{Imposto}
\begin{itemize}
\item {Grp. gram.:adj.}
\end{itemize}
\begin{itemize}
\item {Grp. gram.:M.}
\end{itemize}
Pôsto sôbre.
Imputado.
Contribuição, tributo: \textunderscore pagar os impostos\textunderscore .
\section{Impostor}
\begin{itemize}
\item {Grp. gram.:m.  e  adj.}
\end{itemize}
\begin{itemize}
\item {Proveniência:(Lat. \textunderscore impostor\textunderscore )}
\end{itemize}
O que tem impostura.
\section{Impostoraça}
\begin{itemize}
\item {Grp. gram.:f.}
\end{itemize}
Mulher muito impostora. Cf. Camillo, \textunderscore Volcões\textunderscore , 102.
\section{Impostura}
\begin{itemize}
\item {Grp. gram.:f.}
\end{itemize}
\begin{itemize}
\item {Proveniência:(Lat. \textunderscore impostura\textunderscore )}
\end{itemize}
Artifício para enganar; hypocrisia; embuste.
Presumpção; vaidade.
Trapo, que se prende ao anzol, para chamar os peixes, como se fôsse isca.
\section{Imposturar}
\begin{itemize}
\item {Grp. gram.:v. i.}
\end{itemize}
Têr impostura.
\section{Imposturia}
\begin{itemize}
\item {Grp. gram.:f.}
\end{itemize}
\begin{itemize}
\item {Utilização:Bras}
\end{itemize}
\begin{itemize}
\item {Utilização:fam.}
\end{itemize}
O mesmo que \textunderscore imposturice\textunderscore .
\section{Imposturice}
\begin{itemize}
\item {Grp. gram.:f.}
\end{itemize}
Modos ou acção de impostor.
Impostura:«\textunderscore naquella gritadeira havia muita imposturice...\textunderscore »Camillo, \textunderscore Volcões\textunderscore , 102.
\section{Impotabilidade}
\begin{itemize}
\item {Grp. gram.:f.}
\end{itemize}
Qualidade de impotável.
\section{Impotável}
\begin{itemize}
\item {Grp. gram.:adj.}
\end{itemize}
Que não é potável; que não serve para se beber: \textunderscore água impotável\textunderscore .
(De\textunderscore im...\textunderscore  + \textunderscore potável\textunderscore )
\section{Impotência}
\begin{itemize}
\item {Grp. gram.:f.}
\end{itemize}
Qualidade ou modo de impotente.
\section{Impotente}
\begin{itemize}
\item {Grp. gram.:adj.}
\end{itemize}
\begin{itemize}
\item {Grp. gram.:M.}
\end{itemize}
\begin{itemize}
\item {Proveniência:(Lat. \textunderscore impotens\textunderscore )}
\end{itemize}
Que não póde, que não tem poder.
Fraco.
Insufficiente.
Que tem incapacidade genesíaca.
Aquelle que é impotente.
\section{Impotentemente}
\begin{itemize}
\item {Grp. gram.:adv.}
\end{itemize}
\begin{itemize}
\item {Proveniência:(De \textunderscore impotente\textunderscore )}
\end{itemize}
Com fraqueza, sem fôrça.
\section{Impraticabilidade}
\begin{itemize}
\item {Grp. gram.:f.}
\end{itemize}
Qualidade de impraticável.
\section{Impraticado}
\begin{itemize}
\item {Grp. gram.:adj.}
\end{itemize}
\begin{itemize}
\item {Proveniência:(De \textunderscore im...\textunderscore  + \textunderscore praticado\textunderscore )}
\end{itemize}
Que se não pratica; que não está em uso.
\section{Impraticável}
\begin{itemize}
\item {Grp. gram.:adj.}
\end{itemize}
\begin{itemize}
\item {Proveniência:(De \textunderscore im...\textunderscore  + \textunderscore praticável\textunderscore )}
\end{itemize}
Que não é praticável.
Impossível; inexequível.
\section{Impraticavelmente}
\begin{itemize}
\item {Grp. gram.:adv.}
\end{itemize}
De modo impraticável.
\section{Imprecação}
\begin{itemize}
\item {Grp. gram.:f.}
\end{itemize}
\begin{itemize}
\item {Proveniência:(Lat. \textunderscore imprecatio\textunderscore )}
\end{itemize}
Acto de imprecar.
Maldicção, praga.
\section{Imprecar}
\begin{itemize}
\item {Grp. gram.:v. t.}
\end{itemize}
\begin{itemize}
\item {Utilização:Des.}
\end{itemize}
\begin{itemize}
\item {Grp. gram.:V. i.}
\end{itemize}
\begin{itemize}
\item {Proveniência:(Lat. \textunderscore imprecari\textunderscore )}
\end{itemize}
Pedir contra ou a favor de alguém: \textunderscore imprecar deferimento\textunderscore .
Supplicar.
Pedir instantemente.
Rogar pragas a alguém.
Dizer pragas.
\section{Imprecatado}
\begin{itemize}
\item {Grp. gram.:adj.}
\end{itemize}
\begin{itemize}
\item {Proveniência:(De \textunderscore im...\textunderscore  + \textunderscore precatado\textunderscore )}
\end{itemize}
Que não está precatado.
\section{Imprecativo}
\begin{itemize}
\item {Grp. gram.:adj.}
\end{itemize}
\begin{itemize}
\item {Proveniência:(De \textunderscore imprecar\textunderscore )}
\end{itemize}
Que envolve imprecações.
\section{Imprecatório}
\begin{itemize}
\item {Grp. gram.:adj.}
\end{itemize}
\begin{itemize}
\item {Proveniência:(De \textunderscore imprecar\textunderscore )}
\end{itemize}
Semelhante a uma imprecação.
\section{Imprecaução}
\begin{itemize}
\item {Grp. gram.:f.}
\end{itemize}
\begin{itemize}
\item {Proveniência:(De \textunderscore im...\textunderscore  + \textunderscore precaução\textunderscore )}
\end{itemize}
Falta de precaução.
\section{Imprecisão}
\begin{itemize}
\item {Grp. gram.:f.}
\end{itemize}
\begin{itemize}
\item {Proveniência:(De \textunderscore im\textunderscore  + \textunderscore precisão\textunderscore )}
\end{itemize}
Falta de precisão, de exactidão.
\section{Impreciso}
\begin{itemize}
\item {Grp. gram.:adj.}
\end{itemize}
\begin{itemize}
\item {Utilização:Neol.}
\end{itemize}
\begin{itemize}
\item {Proveniência:(De \textunderscore im...\textunderscore  + \textunderscore preciso\textunderscore )}
\end{itemize}
Não preciso, não determinado; inexacto.
\section{Impreenchível}
\begin{itemize}
\item {Grp. gram.:adj.}
\end{itemize}
\begin{itemize}
\item {Proveniência:(De \textunderscore im\textunderscore  + \textunderscore preencher\textunderscore )}
\end{itemize}
Que se não póde preencher:«\textunderscore tinham deixado dois vácuos impreenchíveis na phalange\textunderscore ». Camillo, \textunderscore Brasileira\textunderscore , 216.
\section{Impregnação}
\begin{itemize}
\item {Grp. gram.:f.}
\end{itemize}
Acto ou effeito de impregnar.
\section{Impregnar}
\begin{itemize}
\item {Grp. gram.:v. t.}
\end{itemize}
\begin{itemize}
\item {Proveniência:(Lat. des. \textunderscore impraegnare\textunderscore )}
\end{itemize}
Fecundar.
Embeber: \textunderscore impregnar de vinagre um trapo\textunderscore .
Encher: \textunderscore impregnar de fumo a atmosphera\textunderscore .
\section{Impregnável}
\begin{itemize}
\item {Grp. gram.:adj.}
\end{itemize}
Que se póde impregnar.
\section{Impremeditação}
\begin{itemize}
\item {Grp. gram.:f.}
\end{itemize}
\begin{itemize}
\item {Proveniência:(De \textunderscore im...\textunderscore  + \textunderscore premeditação\textunderscore )}
\end{itemize}
Falta de premeditação.
\section{Impremeditadamente}
\begin{itemize}
\item {Grp. gram.:adv.}
\end{itemize}
De modo impremeditado; sem premeditação.
\section{Impremeditado}
\begin{itemize}
\item {Grp. gram.:adj.}
\end{itemize}
\begin{itemize}
\item {Proveniência:(De \textunderscore im...\textunderscore  + \textunderscore premeditado\textunderscore )}
\end{itemize}
Em que não há premeditação; impensado.
Instinctivo.
\section{Imprensa}
\begin{itemize}
\item {Grp. gram.:f.}
\end{itemize}
\begin{itemize}
\item {Utilização:Fig.}
\end{itemize}
\begin{itemize}
\item {Utilização:Prov.}
\end{itemize}
\begin{itemize}
\item {Utilização:dur.}
\end{itemize}
\begin{itemize}
\item {Proveniência:(Do lat. \textunderscore impressus\textunderscore )}
\end{itemize}
Apparelho, com que se imprime ou estampa.
Typographia.
O mesmo que \textunderscore prensa\textunderscore .
Arte de imprimir: \textunderscore a invenção da imprensa\textunderscore .
Conjunto de escritores ou jornalistas: \textunderscore a imprensa deve nobilitar-se\textunderscore .
Conjunto de jornaes: \textunderscore não vi essa notícia na imprensa\textunderscore .
Máquina, para espremer as fezes do vinho.
\section{Imprensador}
\begin{itemize}
\item {Grp. gram.:m.  e  adj.}
\end{itemize}
Aquelle que imprensa.
\section{Imprensadura}
\begin{itemize}
\item {Grp. gram.:f.}
\end{itemize}
O mesmo que \textunderscore imprensagem\textunderscore .
\section{Imprensagem}
\begin{itemize}
\item {Grp. gram.:f.}
\end{itemize}
Acto ou effeito de imprensar.
\section{Imprensar}
\begin{itemize}
\item {Grp. gram.:v. t.}
\end{itemize}
\begin{itemize}
\item {Proveniência:(De \textunderscore imprensa\textunderscore )}
\end{itemize}
Apertar no prelo.
Imprimir.
Apertar muito.
\section{Imprensor}
\begin{itemize}
\item {Grp. gram.:m.  e  adj.}
\end{itemize}
\begin{itemize}
\item {Utilização:Des.}
\end{itemize}
O mesmo que \textunderscore imprensador\textunderscore . (Colhido num testamento de 1691)
\section{Imprenta}
\begin{itemize}
\item {Grp. gram.:f.}
\end{itemize}
\begin{itemize}
\item {Utilização:Ant.}
\end{itemize}
O mesmo que \textunderscore impressão\textunderscore .
(Cp. cast. \textunderscore imprenta\textunderscore )
\section{Impresciência}
\begin{itemize}
\item {Grp. gram.:f.}
\end{itemize}
\begin{itemize}
\item {Proveniência:(De \textunderscore im...\textunderscore  + \textunderscore presciência\textunderscore )}
\end{itemize}
Falta de presciência.
\section{Imprescindível}
\begin{itemize}
\item {Grp. gram.:adj.}
\end{itemize}
\begin{itemize}
\item {Proveniência:(De \textunderscore im...\textunderscore  + \textunderscore prescindível\textunderscore )}
\end{itemize}
De que se não póde prescindir.
\section{Imprescriptibilidade}
\begin{itemize}
\item {Grp. gram.:f.}
\end{itemize}
Qualidade de imprescriptível.
\section{Imprescriptível}
\begin{itemize}
\item {Grp. gram.:adj.}
\end{itemize}
\begin{itemize}
\item {Proveniência:(De \textunderscore im...\textunderscore  + \textunderscore prescriptível\textunderscore )}
\end{itemize}
Que não prescreve ou não póde prescrever.
\section{Impressão}
\begin{itemize}
\item {Grp. gram.:f.}
\end{itemize}
\begin{itemize}
\item {Proveniência:(Lat. \textunderscore impressio\textunderscore )}
\end{itemize}
Acto ou effeito de imprimir.
O mesmo que \textunderscore edição\textunderscore :«\textunderscore exhausta logo no primeiro anno a primeira impressão da «Primavera»...\textunderscore »Castilho, \textunderscore Primavera\textunderscore , 7.
Acto ou effeito de embater.
Vestígio dêste acto.
Vestígio de uma acção exterior sôbre um objecto.
Sensação; effeito de uma causa moral no espírito.
Abalo: \textunderscore fez grande impressão aquella catástrophe\textunderscore .
Effeito pathológico, produzido no organismo por um agente morbífico.
Sentimento, despertado em alguém por um facto extranho: \textunderscore faz impressão aquelle feitio de cara\textunderscore .
\section{Impressibilidade}
\begin{itemize}
\item {Grp. gram.:f.}
\end{itemize}
\begin{itemize}
\item {Proveniência:(De \textunderscore impressível\textunderscore )}
\end{itemize}
Propriedade, que a matéria viva tem, segundo Bouchut, de sentir, sem têr órgãos de sensibilidade.
\section{Impressionabilidade}
\begin{itemize}
\item {Grp. gram.:f.}
\end{itemize}
Qualidade de impressionável.
\section{Impressionador}
\begin{itemize}
\item {Grp. gram.:adj.}
\end{itemize}
O mesmo que \textunderscore impressionante\textunderscore .
\section{Impressionante}
\begin{itemize}
\item {Grp. gram.:adj.}
\end{itemize}
Que impressiona.
Commovente.
\section{Impressionar}
\begin{itemize}
\item {Grp. gram.:v. t.}
\end{itemize}
\begin{itemize}
\item {Proveniência:(Do lat. \textunderscore impressio\textunderscore )}
\end{itemize}
Causar impressão material ou moral em: \textunderscore impressiona-me tão grande desgraça\textunderscore .
\section{Impressionável}
\begin{itemize}
\item {Grp. gram.:adj.}
\end{itemize}
\begin{itemize}
\item {Proveniência:(De \textunderscore impressionar\textunderscore )}
\end{itemize}
Que póde receber impressões.
Que se impressiona facilmente.
\section{Impressionismo}
\begin{itemize}
\item {Grp. gram.:m.}
\end{itemize}
\begin{itemize}
\item {Utilização:Neol.}
\end{itemize}
\begin{itemize}
\item {Proveniência:(Do lat. \textunderscore impressio\textunderscore , \textunderscore impressionis\textunderscore )}
\end{itemize}
O mesmo que \textunderscore impressionabilidade\textunderscore .
Systema ou qualidade esthética dos que se preoccupam especialmente de communicar pela arte as impressões que receberam dos factos ou da natureza.
\section{Impressionista}
\begin{itemize}
\item {Grp. gram.:adj.}
\end{itemize}
\begin{itemize}
\item {Utilização:Neol.}
\end{itemize}
O mesmo que \textunderscore impressionável\textunderscore .
Relativo ao impressionismo.
Que cultiva o impressionismo.
(Cp. \textunderscore impressionismo\textunderscore )
\section{Impressível}
\begin{itemize}
\item {Grp. gram.:adj.}
\end{itemize}
\begin{itemize}
\item {Proveniência:(Do lat. \textunderscore impressus\textunderscore )}
\end{itemize}
(V.impressionável)
\section{Impressivo}
\begin{itemize}
\item {Grp. gram.:adj.}
\end{itemize}
\begin{itemize}
\item {Proveniência:(Do lat. \textunderscore impressus\textunderscore )}
\end{itemize}
Que imprime.
Que tem influência moral; que deixa impressões.
\section{Impresso}
\begin{itemize}
\item {Grp. gram.:m.}
\end{itemize}
\begin{itemize}
\item {Grp. gram.:Adj.}
\end{itemize}
\begin{itemize}
\item {Proveniência:(Lat. \textunderscore impressus\textunderscore )}
\end{itemize}
Obra impressa.
Opusculo ou folheto impresso.
Que se imprimiu: \textunderscore livros impressos\textunderscore .
\section{Impressor}
\begin{itemize}
\item {Grp. gram.:m.  e  adj.}
\end{itemize}
\begin{itemize}
\item {Proveniência:(Do lat. \textunderscore impressus\textunderscore )}
\end{itemize}
Aquelle que imprime ou trabalha com o prelo.
\section{Impressório}
\begin{itemize}
\item {Grp. gram.:m.}
\end{itemize}
\begin{itemize}
\item {Utilização:Phot.}
\end{itemize}
Caixilho, para a impressão dos positivos.
(Cp. \textunderscore impressor\textunderscore )
\section{Imprestável}
\begin{itemize}
\item {Grp. gram.:adj.}
\end{itemize}
\begin{itemize}
\item {Proveniência:(De \textunderscore im...\textunderscore  + \textunderscore prestável\textunderscore )}
\end{itemize}
Que não presta; inútil.
\section{Impretendente}
\begin{itemize}
\item {Grp. gram.:adj.}
\end{itemize}
\begin{itemize}
\item {Proveniência:(De \textunderscore im...\textunderscore  + \textunderscore pretendente\textunderscore )}
\end{itemize}
Que não é pretendente.
\section{Impreterível}
\begin{itemize}
\item {Grp. gram.:adj.}
\end{itemize}
\begin{itemize}
\item {Proveniência:(De \textunderscore im...\textunderscore  + \textunderscore preterível\textunderscore )}
\end{itemize}
Que não é preterível; que não póde deixar de se fazer: \textunderscore tarefas impreteríveis\textunderscore .
Indispensável.
\section{Impreterivelmente}
\begin{itemize}
\item {Grp. gram.:adv.}
\end{itemize}
De modo impreterível.
Necessariamente.
\section{Imprevidência}
\begin{itemize}
\item {Grp. gram.:f.}
\end{itemize}
\begin{itemize}
\item {Proveniência:(De \textunderscore im...\textunderscore  + \textunderscore previdência\textunderscore )}
\end{itemize}
Falta de previdência.
\section{Imprevidente}
\begin{itemize}
\item {Grp. gram.:adj.}
\end{itemize}
\begin{itemize}
\item {Proveniência:(De \textunderscore im...\textunderscore  + \textunderscore previdente\textunderscore )}
\end{itemize}
Que não é previdente.
Incauto.
\section{Imprevidentemente}
\begin{itemize}
\item {Grp. gram.:adv.}
\end{itemize}
De modo imprevidente.
Sem previdência; sem cautela.
\section{Imprevisão}
\begin{itemize}
\item {Grp. gram.:f.}
\end{itemize}
\begin{itemize}
\item {Proveniência:(De \textunderscore im...\textunderscore  + \textunderscore previsão\textunderscore )}
\end{itemize}
Falta de previsão; desmazêlo; negligência.
\section{Imprevistamente}
\begin{itemize}
\item {Grp. gram.:adv.}
\end{itemize}
De modo imprevisto.
De súbito; inopinadamente.
\section{Imprevisto}
\begin{itemize}
\item {Grp. gram.:adj.}
\end{itemize}
\begin{itemize}
\item {Proveniência:(De \textunderscore im...\textunderscore  + \textunderscore previsto\textunderscore )}
\end{itemize}
Que não é previsto; inopinado.
Que surprehende.
\section{Imprial}
\begin{itemize}
\item {Grp. gram.:adj.}
\end{itemize}
\begin{itemize}
\item {Utilização:Bras. do N}
\end{itemize}
Muito semelhante; muito parecido com outro.
\section{Imprimação}
\begin{itemize}
\item {Grp. gram.:f.}
\end{itemize}
O mesmo que \textunderscore imprimadura\textunderscore .
\section{Imprimadura}
\begin{itemize}
\item {Grp. gram.:f.}
\end{itemize}
Acto ou effeito de imprimar.
\section{Imprimar}
\begin{itemize}
\item {Grp. gram.:v. t.}
\end{itemize}
Dar a primeira demão em (tela, lâmina, etc.).
(Cast. \textunderscore imprimar\textunderscore . Cp. \textunderscore primário\textunderscore )
\section{Imprimeiramente}
\begin{itemize}
\item {Grp. gram.:adv.}
\end{itemize}
\begin{itemize}
\item {Utilização:Ant.}
\end{itemize}
Em primeiro lugar; antes de tudo.--É do século XIII.
(Cp. lat. \textunderscore imprimis\textunderscore )
\section{Imprimidor}
\begin{itemize}
\item {Grp. gram.:m.}
\end{itemize}
\begin{itemize}
\item {Utilização:Des.}
\end{itemize}
Aquelle que imprime; impressor.
Utensílio de artilharia, para dar fórma ás espoletas de papel vazadas. Cf. Leoni, \textunderscore Diccion. de Artilh.\textunderscore , inédito.
\section{Imprimir}
\begin{itemize}
\item {Grp. gram.:v. t.}
\end{itemize}
\begin{itemize}
\item {Grp. gram.:V. p.}
\end{itemize}
\begin{itemize}
\item {Proveniência:(Lat. \textunderscore imprimere\textunderscore )}
\end{itemize}
Fixar por meio de pressão.
Pôr marca em.
Abrir traços ou figura em.
Imprensar.
Estampar.
Marcar.
Gravar.
Embuír.
Incutir; infundir; transmittir: \textunderscore imprimir bons sentimentos\textunderscore .
Reproduzir pela imprensa; publicar: \textunderscore imprimir um livro\textunderscore .
Produzir com permanência.
Despertar (ideia ou sentimento).
Fixar-se por meio de pressão.
Penetrar ou invadir alguém ou alguma coisa: \textunderscore casos que se imprimem na memória\textunderscore .
\section{Imprir}
\begin{itemize}
\item {Grp. gram.:v. t.}
\end{itemize}
\begin{itemize}
\item {Utilização:Ant.}
\end{itemize}
\begin{itemize}
\item {Utilização:T. de Pare -de-Coira}
\end{itemize}
\begin{itemize}
\item {Utilização:des.}
\end{itemize}
\begin{itemize}
\item {Proveniência:(Do lat. \textunderscore implere\textunderscore )}
\end{itemize}
O mesmo que \textunderscore encher\textunderscore :«\textunderscore o rouço da Cava imprio de tal sanha...\textunderscore »\textunderscore Poema da Cava\textunderscore .
Economizar; capitalizar.
\section{Improbabilidade}
\begin{itemize}
\item {Grp. gram.:f.}
\end{itemize}
Qualidade daquillo que é improvável.
\section{Improbar}
\begin{itemize}
\item {Grp. gram.:v. t.}
\end{itemize}
\begin{itemize}
\item {Proveniência:(Lat. \textunderscore improbare\textunderscore )}
\end{itemize}
Desapprovar, condemnar. Cf. Castilho, \textunderscore Geòrg.\textunderscore , 103.
\section{Improbidade}
\begin{itemize}
\item {Grp. gram.:f.}
\end{itemize}
\begin{itemize}
\item {Proveniência:(Lat. \textunderscore improbitas\textunderscore )}
\end{itemize}
Falta de probidade.
Má índole; mau carácter.
Maldade; perversidade.
\section{Ímprobo}
\begin{itemize}
\item {Grp. gram.:adj.}
\end{itemize}
\begin{itemize}
\item {Utilização:Des.}
\end{itemize}
\begin{itemize}
\item {Proveniência:(Lat. \textunderscore improbus\textunderscore )}
\end{itemize}
Que é de má qualidade; que excede as justas proporções.
Diffícil; árduo; fatigante: \textunderscore trabalhos ímprobos\textunderscore .
Que não tem probidade; que não é honrado.
\section{Improcedência}
\begin{itemize}
\item {Grp. gram.:f.}
\end{itemize}
Qualidade de improcedente.
\section{Improcedente}
\begin{itemize}
\item {Grp. gram.:adj}
\end{itemize}
\begin{itemize}
\item {Proveniência:(De \textunderscore im...\textunderscore  + \textunderscore procedente\textunderscore )}
\end{itemize}
Que não é procedente; que se não justifica.
Illógico, incoherente: \textunderscore argumentos improcedentes\textunderscore .
\section{Improcedentemente}
\begin{itemize}
\item {Grp. gram.:adv.}
\end{itemize}
De modo improcedente.
\section{Improdução}
\begin{itemize}
\item {Grp. gram.:f.}
\end{itemize}
\begin{itemize}
\item {Proveniência:(De \textunderscore im...\textunderscore  + \textunderscore produção\textunderscore )}
\end{itemize}
Falta de produção.
Esterilidade. Cf. Júl. Dinís, \textunderscore Serões\textunderscore , 153.
\section{Improducção}
\begin{itemize}
\item {Grp. gram.:f.}
\end{itemize}
\begin{itemize}
\item {Proveniência:(De \textunderscore im...\textunderscore  + \textunderscore producção\textunderscore )}
\end{itemize}
Falta de producção.
Esterilidade. Cf. Júl. Dinís, \textunderscore Serões\textunderscore , 153.
\section{Improducente}
\begin{itemize}
\item {Grp. gram.:adj.}
\end{itemize}
\begin{itemize}
\item {Proveniência:(De \textunderscore im...\textunderscore  + \textunderscore producente\textunderscore )}
\end{itemize}
Que não produz; estéril.
\section{Improductivamente}
\begin{itemize}
\item {Grp. gram.:adv.}
\end{itemize}
De modo improductivo.
\section{Improductível}
\begin{itemize}
\item {Grp. gram.:adj.}
\end{itemize}
\begin{itemize}
\item {Proveniência:(De \textunderscore im...\textunderscore . + \textunderscore productível\textunderscore )}
\end{itemize}
Que não é productível.
\section{Improductividade}
\begin{itemize}
\item {Grp. gram.:f.}
\end{itemize}
Qualidade de improductivo.
\section{Improductivo}
\begin{itemize}
\item {Grp. gram.:adj.}
\end{itemize}
\begin{itemize}
\item {Proveniência:(De \textunderscore im...\textunderscore  + \textunderscore productivo\textunderscore )}
\end{itemize}
Que não é productivo; que não é fecundo.
Estéril.
Que não dá resultado; frustrado, vão: \textunderscore trabalho improductivo\textunderscore .
\section{Improdutivamente}
\begin{itemize}
\item {Grp. gram.:adv.}
\end{itemize}
De modo improdutivo.
\section{Improdutível}
\begin{itemize}
\item {Grp. gram.:adj.}
\end{itemize}
\begin{itemize}
\item {Proveniência:(De \textunderscore im...\textunderscore . + \textunderscore produtível\textunderscore )}
\end{itemize}
Que não é produtível.
\section{Improdutividade}
\begin{itemize}
\item {Grp. gram.:f.}
\end{itemize}
Qualidade de improdutivo.
\section{Improdutivo}
\begin{itemize}
\item {Grp. gram.:adj.}
\end{itemize}
\begin{itemize}
\item {Proveniência:(De \textunderscore im...\textunderscore  + \textunderscore produtivo\textunderscore )}
\end{itemize}
Que não é produtivo; que não é fecundo.
Estéril.
Que não dá resultado; frustrado, vão: \textunderscore trabalho improdutivo\textunderscore .
\section{Improência}
\begin{itemize}
\item {Grp. gram.:f.}
\end{itemize}
\begin{itemize}
\item {Proveniência:(De \textunderscore prôa\textunderscore )}
\end{itemize}
Prôa, prosápia:«\textunderscore bom traje\textunderscore , \textunderscore ar sério, improência.\textunderscore »Castilho, \textunderscore Méd. á Fôrça\textunderscore , 144.
\section{Improferível}
\begin{itemize}
\item {Grp. gram.:adj.}
\end{itemize}
\begin{itemize}
\item {Utilização:Gram.}
\end{itemize}
\begin{itemize}
\item {Proveniência:(De \textunderscore im...\textunderscore  + \textunderscore proferível\textunderscore )}
\end{itemize}
Que se não profere.
Diz-se das consoantes explosivas ou momentâneas, porque não sôam sem vogal: \textunderscore b\textunderscore , \textunderscore p\textunderscore , \textunderscore t\textunderscore , etc.
\section{Improficiência}
\begin{itemize}
\item {Grp. gram.:f.}
\end{itemize}
Qualidade de improficiente.
\section{Improficiente}
\begin{itemize}
\item {Grp. gram.:adj.}
\end{itemize}
\begin{itemize}
\item {Proveniência:(De \textunderscore im...\textunderscore  + \textunderscore proficiente\textunderscore )}
\end{itemize}
Que não é proficiente.
Improfícuo.
Que não trabalha bem.
\section{Improficuidade}
\begin{itemize}
\item {fónica:cu-i}
\end{itemize}
\begin{itemize}
\item {Grp. gram.:f.}
\end{itemize}
Estado ou qualidade de improfícuo.
\section{Improfícuo}
\begin{itemize}
\item {Grp. gram.:adj.}
\end{itemize}
\begin{itemize}
\item {Proveniência:(De \textunderscore im...\textunderscore  + \textunderscore profícuo\textunderscore )}
\end{itemize}
Que não é profícuo.
Que não dá proveito; que não produz o resultado desejado; baldado, inútil: \textunderscore esforços improfícuos\textunderscore .
\section{Improfundável}
\begin{itemize}
\item {Grp. gram.:adj.}
\end{itemize}
\begin{itemize}
\item {Proveniência:(De \textunderscore im...\textunderscore  + \textunderscore profundável\textunderscore )}
\end{itemize}
Que se não póde profundar.
\section{Improgressivo}
\begin{itemize}
\item {Grp. gram.:adj.}
\end{itemize}
\begin{itemize}
\item {Proveniência:(De \textunderscore im...\textunderscore  + \textunderscore progressivo\textunderscore )}
\end{itemize}
Que não é progressivo; que se não desenvolve, que não progride.
\section{Improlífico}
\begin{itemize}
\item {Grp. gram.:adj.}
\end{itemize}
\begin{itemize}
\item {Proveniência:(De \textunderscore im...\textunderscore  + \textunderscore prolífico\textunderscore )}
\end{itemize}
Que não é prolífico; que não dá prole; que não produz; estéril.
Improfícuo.
\section{Impromptar}
\begin{itemize}
\item {Grp. gram.:v. t.}
\end{itemize}
\begin{itemize}
\item {Utilização:Des.}
\end{itemize}
\begin{itemize}
\item {Proveniência:(Do lat. \textunderscore in\textunderscore  + \textunderscore promptus\textunderscore )}
\end{itemize}
Executar com brevidade (uma obra de arte). Cf. F. Assis Rodrigues, \textunderscore Diccion. Téchn. e Hist.\textunderscore 
\section{Improntar}
\begin{itemize}
\item {Grp. gram.:v. t.}
\end{itemize}
\begin{itemize}
\item {Utilização:Des.}
\end{itemize}
\begin{itemize}
\item {Proveniência:(Do lat. \textunderscore in\textunderscore  + \textunderscore promptus\textunderscore )}
\end{itemize}
Executar com brevidade (uma obra de arte). Cf. F. Assis Rodrigues, \textunderscore Diccion. Téchn. e Hist.\textunderscore 
\section{Improperar}
\begin{itemize}
\item {Grp. gram.:v. t.}
\end{itemize}
\begin{itemize}
\item {Proveniência:(Lat. \textunderscore improperare\textunderscore )}
\end{itemize}
Dirigir impropérios a; injuriar; afrontar.
Censurar. Cf. Camillo, \textunderscore Brasileira\textunderscore , 225.
\section{Impropério}
\begin{itemize}
\item {Grp. gram.:m.}
\end{itemize}
\begin{itemize}
\item {Grp. gram.:Pl.}
\end{itemize}
\begin{itemize}
\item {Proveniência:(Lat. \textunderscore improperium\textunderscore )}
\end{itemize}
Censura ultrajante; ultraje; oppróbrio; vitupério.
Acto infamante.
Série de cânticos religiosos, que se executam nas igrejas cathólicas, em sexta-feira da Semana Santa, durante a ceremónia da adoração da cruz.
\section{Improporção}
\begin{itemize}
\item {Grp. gram.:f.}
\end{itemize}
O mesmo que \textunderscore desproporção\textunderscore .
\section{Improporcional}
\begin{itemize}
\item {Grp. gram.:adj.}
\end{itemize}
\begin{itemize}
\item {Proveniência:(De \textunderscore im...\textunderscore  + \textunderscore proporcional\textunderscore )}
\end{itemize}
Que não é proporcional.
Desproporcionado.
\section{Improporcionalidade}
\begin{itemize}
\item {Grp. gram.:f.}
\end{itemize}
Qualidade de improporcional.
\section{Improporcionalmente}
\begin{itemize}
\item {Grp. gram.:adv.}
\end{itemize}
De modo improporcional.
\section{Improporcionar}
\begin{itemize}
\item {Grp. gram.:v. t.}
\end{itemize}
O mesmo que \textunderscore desproporcionar\textunderscore .
\section{Improporcionável}
\begin{itemize}
\item {Grp. gram.:adj.}
\end{itemize}
Que se não proporciona; que não está em proporção.
\section{Impropriamente}
\begin{itemize}
\item {Grp. gram.:adv.}
\end{itemize}
De modo impróprio; inconvenientemente.
Sem propriedade: \textunderscore escrever impropriamente\textunderscore .
\section{Impropriar}
\begin{itemize}
\item {Grp. gram.:v. t.}
\end{itemize}
Tornar impróprio.
\section{Impropriedade}
\begin{itemize}
\item {Grp. gram.:f.}
\end{itemize}
\begin{itemize}
\item {Proveniência:(Lat. \textunderscore improprietas\textunderscore )}
\end{itemize}
Qualidade de impróprio.
Falta de propriedade.
\section{Impróprio}
\begin{itemize}
\item {Grp. gram.:adj.}
\end{itemize}
\begin{itemize}
\item {Proveniência:(Lat. \textunderscore improprius\textunderscore )}
\end{itemize}
Que não é próprio.
Inconveniente.
Que não é adequado.
Inexacto.
Que escandaliza.
Indecoroso.
Opposto ao costume geral.
Mal visto.
Que não é opportuno.
\section{Improrogabilidade}
\begin{itemize}
\item {fónica:prorro}
\end{itemize}
\begin{itemize}
\item {Grp. gram.:f.}
\end{itemize}
Qualidade de improrogável.
\section{Improrogável}
\begin{itemize}
\item {fónica:prorro}
\end{itemize}
\begin{itemize}
\item {Grp. gram.:adj.}
\end{itemize}
\begin{itemize}
\item {Proveniência:(De \textunderscore im...\textunderscore  + \textunderscore prorogável\textunderscore )}
\end{itemize}
Que não é prorogável.
\section{Improrrogabilidade}
\begin{itemize}
\item {Grp. gram.:f.}
\end{itemize}
Qualidade de improrrogável.
\section{Improrrogável}
\begin{itemize}
\item {Grp. gram.:adj.}
\end{itemize}
\begin{itemize}
\item {Proveniência:(De \textunderscore im...\textunderscore  + \textunderscore prorrogável\textunderscore )}
\end{itemize}
Que não é prorrogável.
\section{Impróspero}
\begin{itemize}
\item {Grp. gram.:adj.}
\end{itemize}
\begin{itemize}
\item {Proveniência:(Lat. \textunderscore improsper\textunderscore )}
\end{itemize}
Que não é próspero; nefasto; agoirento.
\section{Improtrahível}
\begin{itemize}
\item {Grp. gram.:adj.}
\end{itemize}
\begin{itemize}
\item {Proveniência:(De \textunderscore im...\textunderscore  + \textunderscore protrahível\textunderscore )}
\end{itemize}
Que se não póde protrahir. Cf. Arn. Gama, \textunderscore Segr. do Abb.\textunderscore , 264.
\section{Improtraível}
\begin{itemize}
\item {Grp. gram.:adj.}
\end{itemize}
\begin{itemize}
\item {Proveniência:(De \textunderscore im...\textunderscore  + \textunderscore protraível\textunderscore )}
\end{itemize}
Que se não póde protrair. Cf. Arn. Gama, \textunderscore Segr. do Abb.\textunderscore , 264.
\section{Improvação}
\begin{itemize}
\item {Grp. gram.:f.}
\end{itemize}
\begin{itemize}
\item {Proveniência:(Lat. \textunderscore improbatio\textunderscore )}
\end{itemize}
Acto de improvar.
\section{Improvador}
\begin{itemize}
\item {Grp. gram.:m.  e  adj.}
\end{itemize}
\begin{itemize}
\item {Proveniência:(Lat. \textunderscore improbator\textunderscore )}
\end{itemize}
O que improva.
\section{Improvar}
\begin{itemize}
\item {Grp. gram.:v. t.}
\end{itemize}
\begin{itemize}
\item {Proveniência:(Lat. \textunderscore improbare\textunderscore )}
\end{itemize}
O mesmo que \textunderscore desapprovar\textunderscore .
\section{Improvável}
\begin{itemize}
\item {Grp. gram.:adj.}
\end{itemize}
\begin{itemize}
\item {Proveniência:(Lat. \textunderscore improbabilis\textunderscore )}
\end{itemize}
Que não é provável.
\section{Improvidamente}
\begin{itemize}
\item {Grp. gram.:adv.}
\end{itemize}
De modo impróvido.
\section{Improvidência}
\begin{itemize}
\item {Grp. gram.:f.}
\end{itemize}
\begin{itemize}
\item {Proveniência:(Lat. \textunderscore improvidentia\textunderscore )}
\end{itemize}
Qualidade de improvidente.
\section{Improvidente}
\begin{itemize}
\item {Grp. gram.:adj.}
\end{itemize}
Que não é providente; desacautelado; negligente; que não sabe governar-se.
(Cp. \textunderscore improvidência\textunderscore )
\section{Impróvido}
\begin{itemize}
\item {Grp. gram.:adj.}
\end{itemize}
\begin{itemize}
\item {Proveniência:(Lat. \textunderscore improvidus\textunderscore )}
\end{itemize}
O mesmo que \textunderscore improvidente\textunderscore .
\section{Improvisação}
\begin{itemize}
\item {Grp. gram.:f.}
\end{itemize}
Acto ou effeito de improvisar.
\section{Improvisador}
\begin{itemize}
\item {Grp. gram.:m.  e  adj.}
\end{itemize}
O que improvisa.
\section{Improvisamente}
\begin{itemize}
\item {Grp. gram.:adv.}
\end{itemize}
De improviso; subitamente.
\section{Improvisar}
\begin{itemize}
\item {Grp. gram.:v. t.}
\end{itemize}
\begin{itemize}
\item {Grp. gram.:V. i.}
\end{itemize}
Inventar, preparar ou fazer de improviso, de repente: \textunderscore improvisar versos\textunderscore .
Arranjar á pressa: \textunderscore improvisar uma barraca\textunderscore .
Falsear; citar falsamente: \textunderscore improvisar phrases clássicas\textunderscore .
Mentir.
\section{Improvisata}
\begin{itemize}
\item {Grp. gram.:f.}
\end{itemize}
\begin{itemize}
\item {Utilização:Pop.}
\end{itemize}
\begin{itemize}
\item {Proveniência:(It. \textunderscore improvisata\textunderscore )}
\end{itemize}
Qualquer improviso.
\section{Improviso}
\begin{itemize}
\item {Grp. gram.:adj.}
\end{itemize}
\begin{itemize}
\item {Grp. gram.:M.}
\end{itemize}
\begin{itemize}
\item {Grp. gram.:Adv.}
\end{itemize}
\begin{itemize}
\item {Utilização:Ant.}
\end{itemize}
\begin{itemize}
\item {Proveniência:(Lat. \textunderscore improvisus\textunderscore )}
\end{itemize}
Improvisado; repentino, súbito: \textunderscore uma notícia improvisa\textunderscore .
Discurso, poesia ou trecho musical, feito de repente, sem premeditação ou preparo.
O mesmo que \textunderscore depressa\textunderscore . Cf. G. Vicente, \textunderscore Auto da Índia\textunderscore .
\section{Imprudência}
\begin{itemize}
\item {Grp. gram.:f.}
\end{itemize}
\begin{itemize}
\item {Proveniência:(Lat. \textunderscore imprudentia\textunderscore )}
\end{itemize}
Qualidade de imprudente.
Falta de prudência.
Acto ou dito imprudente.
\section{Imprudente}
\begin{itemize}
\item {Grp. gram.:adj.}
\end{itemize}
\begin{itemize}
\item {Grp. gram.:M.  e  f.}
\end{itemize}
\begin{itemize}
\item {Proveniência:(Lat. \textunderscore imprudens\textunderscore )}
\end{itemize}
Que não é prudente.
Pessôa, que não tem prudência.
\section{Imprudentemente}
\begin{itemize}
\item {Grp. gram.:adv.}
\end{itemize}
De modo imprudente.
\section{Impuberdade}
\begin{itemize}
\item {Grp. gram.:f.}
\end{itemize}
\begin{itemize}
\item {Proveniência:(De \textunderscore im...\textunderscore  + \textunderscore puberdade\textunderscore )}
\end{itemize}
Estado ou idade de pessôa impúbere.
\section{Impúbere}
\begin{itemize}
\item {Grp. gram.:adj.}
\end{itemize}
\begin{itemize}
\item {Grp. gram.:M. ,  f.  e  adj.}
\end{itemize}
\begin{itemize}
\item {Proveniência:(Do lat. \textunderscore impubis\textunderscore , \textunderscore impuberis\textunderscore )}
\end{itemize}
Que ainda não é púbere: \textunderscore rapaz impúbere\textunderscore .
Pessôa, que ainda não chegou á puberdade.
\section{Impubescência}
\begin{itemize}
\item {Grp. gram.:f.}
\end{itemize}
\begin{itemize}
\item {Proveniência:(Lat. \textunderscore impubescens\textunderscore )}
\end{itemize}
O mesmo que \textunderscore impuberdade\textunderscore .
O comêço da puberdade.
\section{Impubescente}
\begin{itemize}
\item {Grp. gram.:m. ,  f.  e  adj.}
\end{itemize}
\begin{itemize}
\item {Proveniência:(Lat. \textunderscore impubescens\textunderscore )}
\end{itemize}
O mesmo que \textunderscore impúbere\textunderscore .
\section{Impudência}
\begin{itemize}
\item {Grp. gram.:f.}
\end{itemize}
\begin{itemize}
\item {Proveniência:(Lat. \textunderscore impudentia\textunderscore )}
\end{itemize}
Falta de pudor; acto ou dito impudente.
\section{Impudente}
\begin{itemize}
\item {Grp. gram.:adj.}
\end{itemize}
\begin{itemize}
\item {Proveniência:(Lat. \textunderscore impudens\textunderscore )}
\end{itemize}
Que não tem pudor; desavergonhado; descarado.
\section{Impudentemente}
\begin{itemize}
\item {Grp. gram.:adv.}
\end{itemize}
De modo impudente.
\section{Impudicamente}
\begin{itemize}
\item {Grp. gram.:adv.}
\end{itemize}
De modo impudico.
\section{Impudicícia}
\begin{itemize}
\item {Grp. gram.:f.}
\end{itemize}
\begin{itemize}
\item {Proveniência:(Lat. \textunderscore impudicitia\textunderscore )}
\end{itemize}
Falta de pudicícia; acto ou expressão impudica.
\section{Impudico}
\begin{itemize}
\item {Grp. gram.:adj.}
\end{itemize}
\begin{itemize}
\item {Proveniência:(Lat. \textunderscore impudicus\textunderscore )}
\end{itemize}
Que não tem pudor; lascivo; luxurioso; desenvolto.
\section{Impudor}
\begin{itemize}
\item {Grp. gram.:m.}
\end{itemize}
\begin{itemize}
\item {Proveniência:(De \textunderscore im...\textunderscore  + \textunderscore pudor\textunderscore )}
\end{itemize}
O mesmo que \textunderscore impudência\textunderscore ; descaro; cynismo.
\section{Impugnabilidade}
\begin{itemize}
\item {Grp. gram.:f.}
\end{itemize}
Qualidade de impugnável.
\section{Impugnação}
\begin{itemize}
\item {Grp. gram.:f.}
\end{itemize}
\begin{itemize}
\item {Proveniência:(Lat. \textunderscore impugnatio\textunderscore )}
\end{itemize}
Acto ou effeito de impugnar.
\section{Impugnador}
\begin{itemize}
\item {Grp. gram.:m.  e  adj.}
\end{itemize}
\begin{itemize}
\item {Proveniência:(Lat. \textunderscore impugnator\textunderscore )}
\end{itemize}
O que impugna.
\section{Impugnância}
\begin{itemize}
\item {Grp. gram.:f.}
\end{itemize}
O mesmo que \textunderscore impugnação\textunderscore . Cf. Rui Barb., \textunderscore Réplica\textunderscore , 157.
\section{Impugnar}
\begin{itemize}
\item {Grp. gram.:v. t.}
\end{itemize}
\begin{itemize}
\item {Proveniência:(Lat. \textunderscore impugnare\textunderscore )}
\end{itemize}
Pugnar contra.
Contestar; contrariar.
Refutar: \textunderscore impugnar affirmações\textunderscore .
Fazer opposição a.
\section{Impugnativo}
\begin{itemize}
\item {Grp. gram.:adj.}
\end{itemize}
Que impugna.
\section{Impugnável}
\begin{itemize}
\item {Grp. gram.:adj.}
\end{itemize}
\begin{itemize}
\item {Proveniência:(De \textunderscore impugnar\textunderscore )}
\end{itemize}
Que póde ou deve sêr impugnado.
\section{Impulsão}
\begin{itemize}
\item {Grp. gram.:m.}
\end{itemize}
\begin{itemize}
\item {Proveniência:(Lat. \textunderscore impulsio\textunderscore )}
\end{itemize}
(V.impulso)
\section{Impulsar}
\begin{itemize}
\item {Grp. gram.:v. t.}
\end{itemize}
\begin{itemize}
\item {Proveniência:(Lat. \textunderscore impulsare\textunderscore )}
\end{itemize}
O mesmo que \textunderscore impellir\textunderscore ; dar impulso a: \textunderscore impulsar melhoramentos públicos\textunderscore .
\section{Impulsionamento}
\begin{itemize}
\item {Grp. gram.:m.}
\end{itemize}
\begin{itemize}
\item {Proveniência:(De \textunderscore impulsionar\textunderscore )}
\end{itemize}
O mesmo que \textunderscore impulso\textunderscore .
\section{Impulsionar}
\begin{itemize}
\item {Grp. gram.:v. t.}
\end{itemize}
\begin{itemize}
\item {Proveniência:(Do lat. \textunderscore impulsio\textunderscore )}
\end{itemize}
Dar impulsão a; estimular; incitar.
Dar impulso moral a.
\section{Impulsivismo}
\begin{itemize}
\item {Grp. gram.:m.}
\end{itemize}
\begin{itemize}
\item {Utilização:bras}
\end{itemize}
\begin{itemize}
\item {Utilização:Neol.}
\end{itemize}
Estado de impulsivo.
Tendência para arrebatamentos ou ímpetos.
\section{Impulsivo}
\begin{itemize}
\item {Grp. gram.:adj.}
\end{itemize}
\begin{itemize}
\item {Proveniência:(Do lat. \textunderscore impulsus\textunderscore )}
\end{itemize}
Que impulsa, que impulsiona: \textunderscore temperamento impulsivo\textunderscore .
Que facilmente se excita ou se enfurece: \textunderscore homem impulsivo\textunderscore .
\section{Impulso}
\begin{itemize}
\item {Grp. gram.:m.}
\end{itemize}
\begin{itemize}
\item {Proveniência:(Lat. \textunderscore impulsus\textunderscore )}
\end{itemize}
Acto de impellir; estímulo.
Ímpeto.
Esfôrço.
\section{Impulsor}
\begin{itemize}
\item {Grp. gram.:adj.}
\end{itemize}
\begin{itemize}
\item {Grp. gram.:M.}
\end{itemize}
\begin{itemize}
\item {Proveniência:(Lat. \textunderscore impulsor\textunderscore )}
\end{itemize}
Que impelle ou que impulsa.
Que incita ou estimúla.
Aquelle ou aquillo que impelle ou estimúla.
\section{Impune}
\begin{itemize}
\item {Grp. gram.:adj.}
\end{itemize}
\begin{itemize}
\item {Proveniência:(Lat. \textunderscore impunis\textunderscore )}
\end{itemize}
Que não é ou não foi punido.
Que ficou sem castigo: \textunderscore um delicto impune\textunderscore .
Que não foi reprimido.
\section{Impunemente}
\begin{itemize}
\item {Grp. gram.:adv.}
\end{itemize}
De modo impune; sem punição.
Sem repressão.
\section{Impunidade}
\begin{itemize}
\item {Grp. gram.:f.}
\end{itemize}
\begin{itemize}
\item {Proveniência:(Lat. \textunderscore impunitas\textunderscore )}
\end{itemize}
Falta de castigo devido.
Estado de impune.
\section{Impunido}
\begin{itemize}
\item {Grp. gram.:adj.}
\end{itemize}
\begin{itemize}
\item {Proveniência:(Lat. \textunderscore impunitus\textunderscore )}
\end{itemize}
O mesmo que \textunderscore impune\textunderscore .
\section{Impunível}
\begin{itemize}
\item {Grp. gram.:adj.}
\end{itemize}
\begin{itemize}
\item {Proveniência:(De \textunderscore im...\textunderscore  + \textunderscore punível\textunderscore )}
\end{itemize}
Que não póde ou não deve sêr punido.
\section{Impuramente}
\begin{itemize}
\item {Grp. gram.:adv.}
\end{itemize}
De modo impuro.
Com impureza.
\section{Impureza}
\begin{itemize}
\item {Grp. gram.:f.}
\end{itemize}
\begin{itemize}
\item {Proveniência:(Lat. \textunderscore impuritia\textunderscore )}
\end{itemize}
Qualidade daquelle ou daquillo que é impuro ou que está misturado com outras substâncias.
Falta de pureza.
Coisa impura.
Impudor.
Impiedade.
\section{Impuridade}
\begin{itemize}
\item {Grp. gram.:f.}
\end{itemize}
\begin{itemize}
\item {Proveniência:(Lat. \textunderscore impuritas\textunderscore )}
\end{itemize}
O mesmo que \textunderscore impureza\textunderscore .
\section{Impurificar}
\begin{itemize}
\item {Grp. gram.:v. t.}
\end{itemize}
\begin{itemize}
\item {Proveniência:(De \textunderscore im...\textunderscore  + \textunderscore purificar\textunderscore )}
\end{itemize}
Tornar impuro.
\section{Impuro}
\begin{itemize}
\item {Grp. gram.:adj.}
\end{itemize}
\begin{itemize}
\item {Proveniência:(Lat. \textunderscore impurus\textunderscore )}
\end{itemize}
Que não é puro.
Immundo; sórdido.
Contaminado.
Indecente.
Lúbrico, sensual.
Impróprio, não adequado.
\section{Imputa}
\begin{itemize}
\item {Grp. gram.:f.}
\end{itemize}
O mesmo que \textunderscore imputação\textunderscore . Cf. Filinto, VII, 29.
\section{Imputabilidade}
\begin{itemize}
\item {Grp. gram.:f.}
\end{itemize}
Qualidade daquillo que é imputável.
\section{Imputação}
\begin{itemize}
\item {Grp. gram.:f.}
\end{itemize}
\begin{itemize}
\item {Proveniência:(Lat. \textunderscore imputatio\textunderscore )}
\end{itemize}
Acto ou effeito de imputar.
Responsabilidade.
Declaração de culpabilidade; inculpação.
Aquillo que se imputa.
\section{Imputador}
\begin{itemize}
\item {Grp. gram.:m.  e  adj.}
\end{itemize}
\begin{itemize}
\item {Proveniência:(Lat. \textunderscore imputator\textunderscore )}
\end{itemize}
O que imputa.
\section{Imputar}
\begin{itemize}
\item {Grp. gram.:v. t.}
\end{itemize}
\begin{itemize}
\item {Utilização:P. us.}
\end{itemize}
\begin{itemize}
\item {Proveniência:(Lat. \textunderscore imputare\textunderscore )}
\end{itemize}
Attribuír a alguém a responsabilidade de.
Classificar de êrro ou crime.
\section{Imputável}
\begin{itemize}
\item {Grp. gram.:adj.}
\end{itemize}
Que se póde imputar.
\section{Imputavelmente}
\begin{itemize}
\item {Grp. gram.:adv.}
\end{itemize}
De modo imputável.
\section{Imputrefacção}
\begin{itemize}
\item {Grp. gram.:f.}
\end{itemize}
\begin{itemize}
\item {Proveniência:(De \textunderscore im...\textunderscore  + \textunderscore putrefacção\textunderscore )}
\end{itemize}
Ausência de putrefacção. Cf. Camillo, \textunderscore Narcót.\textunderscore , I, 47.
\section{Imputrescibilidade}
\begin{itemize}
\item {Grp. gram.:f.}
\end{itemize}
Qualidade de imputrescível.
\section{Imputrescível}
\begin{itemize}
\item {Grp. gram.:adj.}
\end{itemize}
\begin{itemize}
\item {Proveniência:(De \textunderscore im...\textunderscore  + \textunderscore putrescível\textunderscore )}
\end{itemize}
Que não é susceptível de apodrecer.
\section{Imudável}
\begin{itemize}
\item {Grp. gram.:adj.}
\end{itemize}
\begin{itemize}
\item {Proveniência:(De \textunderscore im...\textunderscore  + \textunderscore mudável\textunderscore )}
\end{itemize}
Que não é mudável; imutável; imóvel; permanente; constante.
\section{Imundice}
\begin{itemize}
\item {Grp. gram.:f.}
\end{itemize}
\begin{itemize}
\item {Utilização:Bras. de Minas}
\end{itemize}
Grande porção: \textunderscore uma imundice de dinheiro\textunderscore .
(Por \textunderscore inundice\textunderscore , de \textunderscore inundar\textunderscore ?)
\section{Imundícia}
\begin{itemize}
\item {Grp. gram.:f.}
\end{itemize}
\begin{itemize}
\item {Proveniência:(Lat. \textunderscore immunditia\textunderscore )}
\end{itemize}
Falta de asseio; sujidade; lixo; impureza.
\section{Imundície}
\begin{itemize}
\item {Grp. gram.:f.}
\end{itemize}
\begin{itemize}
\item {Utilização:Bras}
\end{itemize}
\begin{itemize}
\item {Proveniência:(Lat. \textunderscore immundities\textunderscore )}
\end{itemize}
O mesmo que \textunderscore imundícia\textunderscore .
Caça miúda de pêlo.
\section{Imundo}
\begin{itemize}
\item {Grp. gram.:adj.}
\end{itemize}
\begin{itemize}
\item {Proveniência:(Lat. \textunderscore immundus\textunderscore )}
\end{itemize}
Que não é limpo; sujo.
Sórdido.
Imoral; obsceno.
\section{Imundo}
\begin{itemize}
\item {Grp. gram.:adj.}
\end{itemize}
\begin{itemize}
\item {Utilização:Prov.}
\end{itemize}
\begin{itemize}
\item {Utilização:trasm.}
\end{itemize}
\begin{itemize}
\item {Proveniência:(Do lat. \textunderscore in...\textunderscore  + \textunderscore mundus\textunderscore , adj.? Ou por \textunderscore imuto\textunderscore , de \textunderscore imutar\textunderscore ?)}
\end{itemize}
Absorto, alheado, estranho ao mundo.
\section{Imune}
\begin{itemize}
\item {Grp. gram.:adj.}
\end{itemize}
\begin{itemize}
\item {Proveniência:(Lat. \textunderscore immunis\textunderscore )}
\end{itemize}
Que tem imunidade; isento; livre.
\section{Imunidade}
\begin{itemize}
\item {Grp. gram.:f.}
\end{itemize}
\begin{itemize}
\item {Proveniência:(Lat. \textunderscore immunitas\textunderscore )}
\end{itemize}
Isenção de algum encargo.
Privilégio.
Predisposição orgânica, pela qual alguns indivíduos estão isentos de moléstias que atacam outros, colocados em meio idêntico.
\section{Imunização}
\begin{itemize}
\item {Grp. gram.:f.}
\end{itemize}
Acto de imunizar.
\section{Imunizar}
\begin{itemize}
\item {Grp. gram.:v. t.}
\end{itemize}
Tornar imune.
\section{Imutabilidade}
\begin{itemize}
\item {Grp. gram.:f.}
\end{itemize}
\begin{itemize}
\item {Proveniência:(Lat. \textunderscore immutabilitas\textunderscore )}
\end{itemize}
Qualidade daquilo que é imutável.
\section{Imutação}
\begin{itemize}
\item {Grp. gram.:f.}
\end{itemize}
\begin{itemize}
\item {Proveniência:(Lat. \textunderscore immutatio\textunderscore )}
\end{itemize}
Acto de imutar.
\section{Imutar}
\begin{itemize}
\item {Grp. gram.:v. t.}
\end{itemize}
\begin{itemize}
\item {Utilização:Des.}
\end{itemize}
\begin{itemize}
\item {Proveniência:(Lat. \textunderscore immutare\textunderscore )}
\end{itemize}
Transmudar, mudar completamente.
\section{Imutável}
\begin{itemize}
\item {Grp. gram.:adj.}
\end{itemize}
\begin{itemize}
\item {Proveniência:(Lat. \textunderscore immutabilis\textunderscore )}
\end{itemize}
O mesmo que \textunderscore imudável\textunderscore .
\section{In}
\begin{itemize}
\item {Grp. gram.:adv.}
\end{itemize}
\begin{itemize}
\item {Utilização:Ant.}
\end{itemize}
O mesmo que \textunderscore inda\textunderscore :«\textunderscore in que dure um mês inteiro\textunderscore ». Simão Mach., f. 16.
(Apócope de \textunderscore inda\textunderscore )
\section{In...}
\begin{itemize}
\item {Grp. gram.:pref.}
\end{itemize}
\begin{itemize}
\item {Proveniência:(Lat. \textunderscore in\textunderscore )}
\end{itemize}
(designativo de \textunderscore privação\textunderscore , \textunderscore negação\textunderscore , \textunderscore logar\textunderscore )
\section{Inabalável}
\begin{itemize}
\item {Grp. gram.:adj.}
\end{itemize}
\begin{itemize}
\item {Utilização:Fig.}
\end{itemize}
\begin{itemize}
\item {Proveniência:(De \textunderscore in...\textunderscore  + \textunderscore abalável\textunderscore )}
\end{itemize}
Que não póde sêr abalado.
Fixo; constante: \textunderscore fé inabalável\textunderscore .
Ínquebrantável: \textunderscore resolução inabalável\textunderscore .
Magnânimo; intrépido.
\section{Inabalavelmente}
\begin{itemize}
\item {Grp. gram.:adv.}
\end{itemize}
De modo inabalável.
\section{Inabdicável}
\begin{itemize}
\item {Grp. gram.:adj.}
\end{itemize}
\begin{itemize}
\item {Proveniência:(De \textunderscore in...\textunderscore  + \textunderscore abdicável\textunderscore )}
\end{itemize}
Que se não póde abdicar.
\section{Inabordável}
\begin{itemize}
\item {Grp. gram.:adj.}
\end{itemize}
\begin{itemize}
\item {Proveniência:(De \textunderscore in...\textunderscore  + \textunderscore abordável\textunderscore )}
\end{itemize}
Que não é abordável.
\section{Inacabável}
\begin{itemize}
\item {Grp. gram.:adj.}
\end{itemize}
\begin{itemize}
\item {Proveniência:(De \textunderscore in...\textunderscore  + \textunderscore acabar\textunderscore )}
\end{itemize}
Que se não póde terminar; interminável, infindo.
\section{Inacção}
\begin{itemize}
\item {Grp. gram.:f.}
\end{itemize}
\begin{itemize}
\item {Proveniência:(De \textunderscore in...\textunderscore  + \textunderscore acção\textunderscore )}
\end{itemize}
Falta de acção.
Inércia; indecisão; froixidão de carácter.
\section{Inacceitável}
\begin{itemize}
\item {Grp. gram.:adj.}
\end{itemize}
\begin{itemize}
\item {Proveniência:(De \textunderscore in...\textunderscore  + \textunderscore acceitável\textunderscore )}
\end{itemize}
Que não é acceitável.
Que se não póde acceitar ou admittir; inadmissível; intolerável.
\section{Inaccessibilidade}
\begin{itemize}
\item {Grp. gram.:f.}
\end{itemize}
Qualidade do que é inaccessível.
\section{Inaccessível}
\begin{itemize}
\item {Grp. gram.:adj.}
\end{itemize}
\begin{itemize}
\item {Proveniência:(Lat. \textunderscore inaccessibilis\textunderscore )}
\end{itemize}
Que não é accessível; a que se não póde chegar: \textunderscore uma felicidade inaccessível\textunderscore .
Em que se não póde entrar.
Que não é tratável, insociável: \textunderscore criatura inaccessível\textunderscore .
Incomprehensivel.
Que não é impressionável.
\section{Inaccessivelmente}
\begin{itemize}
\item {Grp. gram.:adv.}
\end{itemize}
De modo inaccessível.
\section{Inaccesso}
\begin{itemize}
\item {Grp. gram.:adj.}
\end{itemize}
\begin{itemize}
\item {Utilização:Poét.}
\end{itemize}
\begin{itemize}
\item {Proveniência:(Lat. \textunderscore inaccessus\textunderscore )}
\end{itemize}
O mesmo que \textunderscore inaccessível\textunderscore .
\section{Inaccusável}
\begin{itemize}
\item {Grp. gram.:adj.}
\end{itemize}
\begin{itemize}
\item {Proveniência:(Lat. \textunderscore inaccusabilis\textunderscore )}
\end{itemize}
Que não é accusável; que se não deve accusar.
\section{Inaceitável}
\begin{itemize}
\item {Grp. gram.:adj.}
\end{itemize}
\begin{itemize}
\item {Proveniência:(De \textunderscore in...\textunderscore  + \textunderscore aceitável\textunderscore )}
\end{itemize}
Que não é aceitável.
Que se não póde aceitar ou admittir; inadmissível; intolerável.
\section{Inacessibilidade}
\begin{itemize}
\item {Grp. gram.:f.}
\end{itemize}
Qualidade do que é inacessível.
\section{Inacessível}
\begin{itemize}
\item {Grp. gram.:adj.}
\end{itemize}
\begin{itemize}
\item {Proveniência:(Lat. \textunderscore inaccessibilis\textunderscore )}
\end{itemize}
Que não é acessível; a que se não póde chegar: \textunderscore uma felicidade inacessível\textunderscore .
Em que se não póde entrar.
Que não é tratável, insociável: \textunderscore criatura inacessível\textunderscore .
Incompreensivel.
Que não é impressionável.
\section{Inacessivelmente}
\begin{itemize}
\item {Grp. gram.:adv.}
\end{itemize}
De modo inacessível.
\section{Inacesso}
\begin{itemize}
\item {Grp. gram.:adj.}
\end{itemize}
\begin{itemize}
\item {Utilização:Poét.}
\end{itemize}
\begin{itemize}
\item {Proveniência:(Lat. \textunderscore inaccessus\textunderscore )}
\end{itemize}
O mesmo que \textunderscore inacessível\textunderscore .
\section{Inácia}
\begin{itemize}
\item {Grp. gram.:f.}
\end{itemize}
O mesmo que \textunderscore inaciana\textunderscore .
\section{Inaciana}
\begin{itemize}
\item {Grp. gram.:f.}
\end{itemize}
\begin{itemize}
\item {Proveniência:(De \textunderscore Ignácio\textunderscore , ou \textunderscore Inácio\textunderscore , n. p.)}
\end{itemize}
Árvore loganiácea, (\textunderscore ignacia amara\textunderscore ), que produz a chamada \textunderscore fava de Santo-Inácio\textunderscore .
\section{Inaciano}
\begin{itemize}
\item {Grp. gram.:m.}
\end{itemize}
\begin{itemize}
\item {Utilização:deprec.}
\end{itemize}
\begin{itemize}
\item {Utilização:Ant.}
\end{itemize}
\begin{itemize}
\item {Proveniência:(De \textunderscore Ignácio\textunderscore , ou \textunderscore Inácio\textunderscore , n. p.)}
\end{itemize}
O mesmo que \textunderscore jesuíta\textunderscore .
\section{Inaclimável}
\begin{itemize}
\item {Grp. gram.:adj.}
\end{itemize}
\begin{itemize}
\item {Proveniência:(De \textunderscore in...\textunderscore  + \textunderscore aclimar\textunderscore )}
\end{itemize}
Que se não póde aclimar.
\section{Inacommodável}
\begin{itemize}
\item {Grp. gram.:adj.}
\end{itemize}
\begin{itemize}
\item {Proveniência:(De \textunderscore in...\textunderscore  + \textunderscore acommodável\textunderscore )}
\end{itemize}
Que se não póde acommodar. Cf. Th. Ribeiro, \textunderscore Jornadas\textunderscore , II, 80.
\section{Inacomodável}
\begin{itemize}
\item {Grp. gram.:adj.}
\end{itemize}
\begin{itemize}
\item {Proveniência:(De \textunderscore in...\textunderscore  + \textunderscore acomodável\textunderscore )}
\end{itemize}
Que se não póde acomodar. Cf. Th. Ribeiro, \textunderscore Jornadas\textunderscore , II, 80.
\section{Inacreditável}
\begin{itemize}
\item {Grp. gram.:adj.}
\end{itemize}
\begin{itemize}
\item {Proveniência:(De \textunderscore in...\textunderscore  + \textunderscore acreditável\textunderscore )}
\end{itemize}
Que não é acreditável; incrível.
\section{Inacreditavelmente}
\begin{itemize}
\item {Grp. gram.:adv.}
\end{itemize}
De modo inacreditável.
\section{Inactivamente}
\begin{itemize}
\item {Grp. gram.:adv.}
\end{itemize}
De modo inactivo.
Sem acção.
\section{Inactividade}
\begin{itemize}
\item {Grp. gram.:f.}
\end{itemize}
Qualidade de inactivo.
Inércia.
Situação de alguns funccionários, retirados do serviço activo por disposição superior, ou em virtude de diploma official.
\section{Inactivo}
\begin{itemize}
\item {Grp. gram.:adj.}
\end{itemize}
\begin{itemize}
\item {Proveniência:(De \textunderscore in...\textunderscore  + \textunderscore activo\textunderscore )}
\end{itemize}
Que não está em exercicio.
Inerte.
Que não exerce funcções; aposentado ou reformado, (falando-se de empregados ou funccionários).
Paralysado.
\section{Inacusável}
\begin{itemize}
\item {Grp. gram.:adj.}
\end{itemize}
\begin{itemize}
\item {Proveniência:(Lat. \textunderscore inaccusabilis\textunderscore )}
\end{itemize}
Que não é acusável; que se não deve acusar.
\section{Inadaptação}
\begin{itemize}
\item {Grp. gram.:f.}
\end{itemize}
\begin{itemize}
\item {Proveniência:(De \textunderscore in...\textunderscore  + \textunderscore adaptação\textunderscore )}
\end{itemize}
Falta de adaptação.
\section{Inadaptável}
\begin{itemize}
\item {Grp. gram.:adj.}
\end{itemize}
\begin{itemize}
\item {Proveniência:(De \textunderscore in...\textunderscore  + \textunderscore adaptável\textunderscore )}
\end{itemize}
Que se não póde adaptar.
\section{Inadequado}
\begin{itemize}
\item {Grp. gram.:adj.}
\end{itemize}
\begin{itemize}
\item {Proveniência:(De \textunderscore in...\textunderscore  + \textunderscore adequado\textunderscore )}
\end{itemize}
Que não é adequado; impróprio; inconveniente.
\section{Inaderente}
\begin{itemize}
\item {Grp. gram.:adj.}
\end{itemize}
\begin{itemize}
\item {Utilização:Bot.}
\end{itemize}
\begin{itemize}
\item {Proveniência:(De \textunderscore in...\textunderscore  + \textunderscore aderente\textunderscore )}
\end{itemize}
Que não adere.
Diz-se dos órgãos livres ou que não aderem reciprocamente.
\section{Inadherente}
\begin{itemize}
\item {Grp. gram.:adj.}
\end{itemize}
\begin{itemize}
\item {Utilização:Bot.}
\end{itemize}
\begin{itemize}
\item {Proveniência:(De \textunderscore in...\textunderscore  + \textunderscore adherente\textunderscore )}
\end{itemize}
Que não adhere.
Diz-se dos órgãos livres ou que não adherem reciprocamente.
\section{Inadiável}
\begin{itemize}
\item {Grp. gram.:adj.}
\end{itemize}
\begin{itemize}
\item {Proveniência:(De \textunderscore in...\textunderscore  + \textunderscore adiável\textunderscore )}
\end{itemize}
Que se não póde adiar; impreterível.
\section{Inadimplemento}
\begin{itemize}
\item {Grp. gram.:m.}
\end{itemize}
\begin{itemize}
\item {Proveniência:(Do lat. \textunderscore in...\textunderscore  + \textunderscore adimplere\textunderscore )}
\end{itemize}
Falta de cumprimento de um contrato ou das suas condições.
\section{Inadmissão}
\begin{itemize}
\item {Grp. gram.:f.}
\end{itemize}
\begin{itemize}
\item {Proveniência:(De \textunderscore in...\textunderscore  + \textunderscore admissão\textunderscore )}
\end{itemize}
Acto ou effeito de não admittir.
\section{Inadmissibilidade}
\begin{itemize}
\item {Grp. gram.:f.}
\end{itemize}
Qualidade de inadmissível.
\section{Inadmissível}
\begin{itemize}
\item {Grp. gram.:adj.}
\end{itemize}
\begin{itemize}
\item {Proveniência:(De \textunderscore in...\textunderscore  + \textunderscore admissível\textunderscore )}
\end{itemize}
Que não é admissível; que se não póde acceitar ou receber.
\section{Inadmissivelmente}
\begin{itemize}
\item {Grp. gram.:adv.}
\end{itemize}
De modo inadmissível.
\section{Inadquirível}
\begin{itemize}
\item {Grp. gram.:adj.}
\end{itemize}
\begin{itemize}
\item {Proveniência:(De \textunderscore in...\textunderscore  + \textunderscore adquirível\textunderscore )}
\end{itemize}
Que não é adquirível.
\section{Inadvertência}
\begin{itemize}
\item {Grp. gram.:f.}
\end{itemize}
\begin{itemize}
\item {Proveniência:(De \textunderscore in...\textunderscore  + \textunderscore advertência\textunderscore )}
\end{itemize}
Falta de advertência; descuido; irreflexão; distracção.
\section{Inadvertidamente}
\begin{itemize}
\item {Grp. gram.:adv.}
\end{itemize}
De modo inadvertido.
Irreflectidamente, impensadamente.
\section{Inadvertido}
\begin{itemize}
\item {Grp. gram.:adj.}
\end{itemize}
\begin{itemize}
\item {Proveniência:(De \textunderscore in...\textunderscore  + \textunderscore advertido\textunderscore )}
\end{itemize}
Feito sem reflexão.
\section{Inafável}
\begin{itemize}
\item {Grp. gram.:adj.}
\end{itemize}
\begin{itemize}
\item {Proveniência:(Lat. \textunderscore inaffabilis\textunderscore )}
\end{itemize}
Que não é afável.
\section{Inaffável}
\begin{itemize}
\item {Grp. gram.:adj.}
\end{itemize}
\begin{itemize}
\item {Proveniência:(Lat. \textunderscore inaffabilis\textunderscore )}
\end{itemize}
Que não é affável.
\section{Inafiançável}
\begin{itemize}
\item {Grp. gram.:adj.}
\end{itemize}
\begin{itemize}
\item {Proveniência:(De \textunderscore in...\textunderscore  + \textunderscore afiançável\textunderscore )}
\end{itemize}
Que se não póde afiançar.
\section{Inagitável}
\begin{itemize}
\item {Grp. gram.:adj.}
\end{itemize}
\begin{itemize}
\item {Proveniência:(Lat. \textunderscore inagitabilis\textunderscore )}
\end{itemize}
Que se não póde agitar.
\section{Inágua}
\begin{itemize}
\item {Grp. gram.:f.}
\end{itemize}
\begin{itemize}
\item {Utilização:Prov.}
\end{itemize}
\begin{itemize}
\item {Utilização:trasm.}
\end{itemize}
O mesmo que \textunderscore anágua\textunderscore .
\section{Inajá}
\begin{itemize}
\item {Grp. gram.:m.}
\end{itemize}
\begin{itemize}
\item {Utilização:Bras}
\end{itemize}
Gênero de plantas brasileiras, de fibras têxteis.
\section{Inajá-guaçu-ibá}
\begin{itemize}
\item {Grp. gram.:m.}
\end{itemize}
Planta palmácea do Brasil, (\textunderscore coccus nucífera\textunderscore ).
\section{Inajarana}
\begin{itemize}
\item {Grp. gram.:f.}
\end{itemize}
Planta medicinal do Amazonas.
\section{Inalado}
\begin{itemize}
\item {Grp. gram.:adj.}
\end{itemize}
\begin{itemize}
\item {Utilização:Zool.}
\end{itemize}
\begin{itemize}
\item {Proveniência:(De \textunderscore in...\textunderscore  + \textunderscore alado\textunderscore )}
\end{itemize}
Que não tem asas.
\section{Inalbuminado}
\begin{itemize}
\item {Grp. gram.:adj.}
\end{itemize}
\begin{itemize}
\item {Proveniência:(De \textunderscore in...\textunderscore  + \textunderscore albuminado\textunderscore )}
\end{itemize}
Que não tem albumina.
\section{Inalcançavel}
\begin{itemize}
\item {Grp. gram.:adj.}
\end{itemize}
\begin{itemize}
\item {Proveniência:(De \textunderscore in...\textunderscore  + \textunderscore alcançável\textunderscore )}
\end{itemize}
Que se não póde alcançar.
\section{Inalheabilidade}
\begin{itemize}
\item {Grp. gram.:f.}
\end{itemize}
Qualidade daquillo que é inalheável.
\section{Inalheável}
\begin{itemize}
\item {Grp. gram.:adj.}
\end{itemize}
O mesmo que \textunderscore inalienável\textunderscore . Cf. F. Borges, \textunderscore Diccion. Jur.\textunderscore 
\section{Inaliável}
\begin{itemize}
\item {Grp. gram.:adj.}
\end{itemize}
\begin{itemize}
\item {Proveniência:(De \textunderscore in...\textunderscore  + \textunderscore aliável\textunderscore )}
\end{itemize}
Que se não póde aliar.
\section{Inalienabilidade}
\begin{itemize}
\item {Grp. gram.:f.}
\end{itemize}
Qualidade do que é inalienável.
\section{Inalienação}
\begin{itemize}
\item {Grp. gram.:f.}
\end{itemize}
\begin{itemize}
\item {Proveniência:(De \textunderscore in...\textunderscore  + \textunderscore alienação\textunderscore )}
\end{itemize}
Estado daquillo que se não alienou.
\section{Inalienado}
\begin{itemize}
\item {Grp. gram.:adj.}
\end{itemize}
\begin{itemize}
\item {Proveniência:(De \textunderscore in...\textunderscore  + \textunderscore alienado\textunderscore )}
\end{itemize}
Que se não alienou.
\section{Inalienável}
\begin{itemize}
\item {Grp. gram.:adj.}
\end{itemize}
Que se não póde alienar.
\section{Inalienavelmente}
\begin{itemize}
\item {Grp. gram.:adv.}
\end{itemize}
De modo inalienável.
\section{Inalliável}
\begin{itemize}
\item {Grp. gram.:adj.}
\end{itemize}
\begin{itemize}
\item {Proveniência:(De \textunderscore in...\textunderscore  + \textunderscore alliável\textunderscore )}
\end{itemize}
Que se não póde alliar.
\section{Inalpino}
\begin{itemize}
\item {Grp. gram.:adj.}
\end{itemize}
\begin{itemize}
\item {Utilização:Des.}
\end{itemize}
\begin{itemize}
\item {Proveniência:(Lat. \textunderscore inalpinus\textunderscore )}
\end{itemize}
Que vive nos Alpes.
\section{Inalterabilidade}
\begin{itemize}
\item {Grp. gram.:f.}
\end{itemize}
Qualidade do que é inalterável.
\section{Inalteradamente}
\begin{itemize}
\item {Grp. gram.:adv.}
\end{itemize}
De modo inalterado.
Da mesma fórma.
Constantemente.
\section{Inalterado}
\begin{itemize}
\item {Grp. gram.:adj.}
\end{itemize}
\begin{itemize}
\item {Proveniência:(De \textunderscore in...\textunderscore  + \textunderscore alterado\textunderscore )}
\end{itemize}
Que não é alterado; que se não altera; que não soffreu modificação.
\section{Inalterável}
\begin{itemize}
\item {Grp. gram.:adj.}
\end{itemize}
\begin{itemize}
\item {Proveniência:(De \textunderscore in...\textunderscore  + \textunderscore alterável\textunderscore )}
\end{itemize}
Que não é alterável; impertubável; impassível; sereno.
\section{Inalteravelmente}
\begin{itemize}
\item {Grp. gram.:adv.}
\end{itemize}
De modo inalterável.
\section{Inama}
\begin{itemize}
\item {Grp. gram.:f.}
\end{itemize}
Terra, usufruída pelos officiaes das aldeias, nas communidades indianas.
\section{Inamável}
\begin{itemize}
\item {Grp. gram.:adj.}
\end{itemize}
\begin{itemize}
\item {Proveniência:(Lat. \textunderscore inamabilis\textunderscore )}
\end{itemize}
Que não é amável; descortês, indelicado.
\section{Inambu}
\begin{itemize}
\item {Grp. gram.:f.}
\end{itemize}
\begin{itemize}
\item {Utilização:Bras}
\end{itemize}
Nome de várias espécies de aves perdíceas.
(Do tupi)
\section{Inambulação}
\begin{itemize}
\item {Grp. gram.:f.}
\end{itemize}
\begin{itemize}
\item {Proveniência:(Lat. \textunderscore inambulatio\textunderscore )}
\end{itemize}
Acto de passear, de andar de um lado para o outro.
\section{Inambu-uaçu}
\begin{itemize}
\item {Grp. gram.:m.}
\end{itemize}
Espécie de inambu.
\section{Inameno}
\begin{itemize}
\item {Grp. gram.:adj.}
\end{itemize}
\begin{itemize}
\item {Proveniência:(De \textunderscore in...\textunderscore  + \textunderscore ameno\textunderscore )}
\end{itemize}
Que não é ameno.
Desagradável.
\section{Inamissibilidade}
\begin{itemize}
\item {Grp. gram.:f.}
\end{itemize}
Qualidade de inamissível.
\section{Inamissível}
\begin{itemize}
\item {Grp. gram.:adj.}
\end{itemize}
\begin{itemize}
\item {Proveniência:(Lat. \textunderscore inamissibilis\textunderscore )}
\end{itemize}
Que se não perde; que não é sujeito a perder-se.
\section{Inamissivelmente}
\begin{itemize}
\item {Grp. gram.:adv.}
\end{itemize}
\begin{itemize}
\item {Proveniência:(De \textunderscore inamissível\textunderscore )}
\end{itemize}
Sem perigo de se perder. Cf. Bernárdez, \textunderscore Luz e Calor\textunderscore , 297.
\section{Inamolgável}
\begin{itemize}
\item {Grp. gram.:adj.}
\end{itemize}
\begin{itemize}
\item {Proveniência:(De \textunderscore in...\textunderscore  + \textunderscore amolgável\textunderscore )}
\end{itemize}
Que não é amolgável.
\section{Inamovibilidade}
\begin{itemize}
\item {Grp. gram.:f.}
\end{itemize}
Qualidade de inamovível.
\section{Inamovível}
\begin{itemize}
\item {Grp. gram.:adj.}
\end{itemize}
\begin{itemize}
\item {Proveniência:(De \textunderscore in...\textunderscore  + \textunderscore amovível\textunderscore )}
\end{itemize}
Que não é amovível; que se não póde deslocar.
\section{Inanalisável}
\begin{itemize}
\item {Grp. gram.:adj.}
\end{itemize}
\begin{itemize}
\item {Proveniência:(De \textunderscore in...\textunderscore  + \textunderscore analisável\textunderscore )}
\end{itemize}
Que se não póde analisar.
\section{Inanalysável}
\begin{itemize}
\item {Grp. gram.:adj.}
\end{itemize}
\begin{itemize}
\item {Proveniência:(De \textunderscore in...\textunderscore  + \textunderscore analysável\textunderscore )}
\end{itemize}
Que se não póde analysar.
\section{Inane}
\begin{itemize}
\item {Grp. gram.:adj.}
\end{itemize}
\begin{itemize}
\item {Proveniência:(Lat. \textunderscore inanis\textunderscore )}
\end{itemize}
Vazio; oco.
Fútil.
\section{Inânias}
\begin{itemize}
\item {Grp. gram.:f. pl.}
\end{itemize}
\begin{itemize}
\item {Proveniência:(Lat. \textunderscore inaniae\textunderscore )}
\end{itemize}
Bagatelas.
\section{Inanição}
\begin{itemize}
\item {Grp. gram.:f.}
\end{itemize}
\begin{itemize}
\item {Proveniência:(Lat. \textunderscore inanitio\textunderscore )}
\end{itemize}
Qualidade de inane.
Enfraquecimento por falta de alimentação.
\section{Inanidade}
\begin{itemize}
\item {Grp. gram.:f.}
\end{itemize}
\begin{itemize}
\item {Utilização:Fig.}
\end{itemize}
\begin{itemize}
\item {Proveniência:(Lat. \textunderscore inanitas\textunderscore )}
\end{itemize}
Qualidade daquillo que é inane.
Vaidade; futilidade: \textunderscore a inanidade das coisas terrenas\textunderscore .
\section{Inanimado}
\begin{itemize}
\item {Grp. gram.:adj.}
\end{itemize}
\begin{itemize}
\item {Proveniência:(Lat. \textunderscore inanimatus\textunderscore )}
\end{itemize}
Que não está animado.
Que está sem sentidos.
Que não tem alma.
Que está sem movimento.
Que não tem vivacidade.
\section{Inânime}
\begin{itemize}
\item {Grp. gram.:adj.}
\end{itemize}
\begin{itemize}
\item {Proveniência:(Lat. \textunderscore inanimis\textunderscore )}
\end{itemize}
O mesmo que \textunderscore inanimado\textunderscore .
\section{Inanir}
\begin{itemize}
\item {Grp. gram.:v. t.}
\end{itemize}
\begin{itemize}
\item {Utilização:Des.}
\end{itemize}
\begin{itemize}
\item {Proveniência:(Lat. \textunderscore inanire\textunderscore )}
\end{itemize}
Tornar inane; debilitar.
\section{Inantéreo}
\begin{itemize}
\item {Grp. gram.:adj.}
\end{itemize}
\begin{itemize}
\item {Utilização:Bot.}
\end{itemize}
\begin{itemize}
\item {Proveniência:(De \textunderscore in...\textunderscore  + \textunderscore antera\textunderscore )}
\end{itemize}
Que não tem anteras.
\section{Inanthéreo}
\begin{itemize}
\item {Grp. gram.:adj.}
\end{itemize}
\begin{itemize}
\item {Utilização:Bot.}
\end{itemize}
\begin{itemize}
\item {Proveniência:(De \textunderscore in...\textunderscore  + \textunderscore anthera\textunderscore )}
\end{itemize}
Que não tem antheras.
\section{Inaparente}
\begin{itemize}
\item {Grp. gram.:adj.}
\end{itemize}
\begin{itemize}
\item {Proveniência:(De \textunderscore in...\textunderscore  + \textunderscore aparente\textunderscore )}
\end{itemize}
Que não é aparente.
\section{Inapelabilidade}
\begin{itemize}
\item {Grp. gram.:f.}
\end{itemize}
Qualidade de inapelável.
\section{Inapelável}
\begin{itemize}
\item {Grp. gram.:adj.}
\end{itemize}
\begin{itemize}
\item {Proveniência:(De \textunderscore in...\textunderscore  + \textunderscore apelável\textunderscore )}
\end{itemize}
De que se não póde recorrer ou apelar.
\section{Inapendiculado}
\begin{itemize}
\item {Grp. gram.:adj.}
\end{itemize}
\begin{itemize}
\item {Utilização:Bot.}
\end{itemize}
\begin{itemize}
\item {Proveniência:(De \textunderscore in...\textunderscore  + \textunderscore apendiculado\textunderscore )}
\end{itemize}
Que não tem apêndices.
\section{Inaperto}
\begin{itemize}
\item {Grp. gram.:adj.}
\end{itemize}
\begin{itemize}
\item {Utilização:Bot.}
\end{itemize}
\begin{itemize}
\item {Proveniência:(Lat. \textunderscore inapertus\textunderscore )}
\end{itemize}
Não aberto.
Oco.
Que não tem fenda.
\section{Inapetência}
\begin{itemize}
\item {Grp. gram.:f.}
\end{itemize}
\begin{itemize}
\item {Proveniência:(De \textunderscore in...\textunderscore  + \textunderscore apetência\textunderscore )}
\end{itemize}
Falta de apetite.
\section{Inaplicabilidade}
\begin{itemize}
\item {Grp. gram.:f.}
\end{itemize}
Qualidade ou estado do que é inaplicável.
\section{Inaplicado}
\begin{itemize}
\item {Grp. gram.:adj.}
\end{itemize}
\begin{itemize}
\item {Proveniência:(De \textunderscore in...\textunderscore  + \textunderscore aplicado\textunderscore )}
\end{itemize}
Que não tem ou não teve aplicação.
\section{Inaplicável}
\begin{itemize}
\item {Grp. gram.:adj.}
\end{itemize}
\begin{itemize}
\item {Proveniência:(De \textunderscore in...\textunderscore  + \textunderscore aplicável\textunderscore )}
\end{itemize}
Que não é aplicável.
\section{Inapparente}
\begin{itemize}
\item {Grp. gram.:adj.}
\end{itemize}
\begin{itemize}
\item {Proveniência:(De \textunderscore in...\textunderscore  + \textunderscore apparente\textunderscore )}
\end{itemize}
Que não é apparente.
\section{Inappellabilidade}
\begin{itemize}
\item {Grp. gram.:f.}
\end{itemize}
Qualidade de inappellável.
\section{Inappellável}
\begin{itemize}
\item {Grp. gram.:adj.}
\end{itemize}
\begin{itemize}
\item {Proveniência:(De \textunderscore in...\textunderscore  + \textunderscore appellável\textunderscore )}
\end{itemize}
De que se não póde recorrer ou appellar.
\section{Inappendiculado}
\begin{itemize}
\item {Grp. gram.:adj.}
\end{itemize}
\begin{itemize}
\item {Utilização:Bot.}
\end{itemize}
\begin{itemize}
\item {Proveniência:(De \textunderscore in...\textunderscore  + \textunderscore appendiculado\textunderscore )}
\end{itemize}
Que não tem appêndices.
\section{Inappetência}
\begin{itemize}
\item {Grp. gram.:f.}
\end{itemize}
\begin{itemize}
\item {Proveniência:(De \textunderscore in...\textunderscore  + \textunderscore appetência\textunderscore )}
\end{itemize}
Falta de appetite.
\section{Inapplicabilidade}
\begin{itemize}
\item {Grp. gram.:f.}
\end{itemize}
Qualidade ou estado do que é inapplicável.
\section{Inapplicado}
\begin{itemize}
\item {Grp. gram.:adj.}
\end{itemize}
\begin{itemize}
\item {Proveniência:(De \textunderscore in...\textunderscore  + \textunderscore applicado\textunderscore )}
\end{itemize}
Que não tem ou não teve applicação.
\section{Inapplicável}
\begin{itemize}
\item {Grp. gram.:adj.}
\end{itemize}
\begin{itemize}
\item {Proveniência:(De \textunderscore in...\textunderscore  + \textunderscore applicável\textunderscore )}
\end{itemize}
Que não é applicável.
\section{Inapprehensível}
\begin{itemize}
\item {Grp. gram.:adj.}
\end{itemize}
\begin{itemize}
\item {Proveniência:(De \textunderscore in...\textunderscore  + \textunderscore apprehensível\textunderscore )}
\end{itemize}
Que se não póde apprehender.
\section{Inapreciável}
\begin{itemize}
\item {Grp. gram.:adj.}
\end{itemize}
\begin{itemize}
\item {Proveniência:(De \textunderscore in...\textunderscore  + \textunderscore apreciável\textunderscore )}
\end{itemize}
Que não é apreciável.
\section{Inapreensível}
\begin{itemize}
\item {Grp. gram.:adj.}
\end{itemize}
\begin{itemize}
\item {Proveniência:(De \textunderscore in...\textunderscore  + \textunderscore apreensível\textunderscore )}
\end{itemize}
Que se não póde apreender.
\section{Inapteza}
\begin{itemize}
\item {Grp. gram.:f.}
\end{itemize}
\begin{itemize}
\item {Utilização:Ant.}
\end{itemize}
O mesmo que \textunderscore inaptidão\textunderscore .
\section{Inaptidão}
\begin{itemize}
\item {Grp. gram.:f.}
\end{itemize}
\begin{itemize}
\item {Proveniência:(De \textunderscore in...\textunderscore  + \textunderscore aptidão\textunderscore )}
\end{itemize}
Qualidade de inepto; incapacidade; estupidez.
\section{Inapto}
\begin{itemize}
\item {Grp. gram.:adj.}
\end{itemize}
(Fórma bárbara, em vez de \textunderscore inepto\textunderscore . Cf. G. Viana, \textunderscore Apostilas\textunderscore )
\section{Inárculo}
\begin{itemize}
\item {Grp. gram.:m.}
\end{itemize}
\begin{itemize}
\item {Proveniência:(Lat. \textunderscore inarculum\textunderscore )}
\end{itemize}
Varinha curva de romanzeira, que a raínha, nos sacrifícios antigos, levava á cabeça.
\section{Inarrecadável}
\begin{itemize}
\item {Grp. gram.:adj.}
\end{itemize}
\begin{itemize}
\item {Proveniência:(De \textunderscore in...\textunderscore  + \textunderscore arrecadar\textunderscore )}
\end{itemize}
Que se não póde arrecadar.
\section{Inarticulado}
\begin{itemize}
\item {Grp. gram.:adj.}
\end{itemize}
\begin{itemize}
\item {Utilização:Hist. Nat.}
\end{itemize}
\begin{itemize}
\item {Proveniência:(De \textunderscore in...\textunderscore  + \textunderscore articulado\textunderscore )}
\end{itemize}
Que não é articulado.
Que se articula mal.
Mal pronunciado.
Que não tem artículos ou articulações.
\section{Inarticulável}
\begin{itemize}
\item {Grp. gram.:adj.}
\end{itemize}
\begin{itemize}
\item {Proveniência:(De \textunderscore in...\textunderscore  + \textunderscore articular\textunderscore )}
\end{itemize}
Que não é articulável; que se não póde articular.
\section{Inartificial}
\begin{itemize}
\item {Grp. gram.:adj.}
\end{itemize}
\begin{itemize}
\item {Proveniência:(Lat. \textunderscore inartificialis\textunderscore )}
\end{itemize}
Que não é artificial.
\section{Inartificioso}
\begin{itemize}
\item {Grp. gram.:adj.}
\end{itemize}
\begin{itemize}
\item {Proveniência:(De \textunderscore in...\textunderscore  + \textunderscore artificioso\textunderscore )}
\end{itemize}
Que não é artificioso.
Sincero.
Espontâneo.
\section{Inartístico}
\begin{itemize}
\item {Grp. gram.:adj.}
\end{itemize}
\begin{itemize}
\item {Proveniência:(De \textunderscore in...\textunderscore  + \textunderscore artístico\textunderscore )}
\end{itemize}
Que não é artístico.
Feito sem arte.
\section{Inassiduidade}
\begin{itemize}
\item {fónica:du-i}
\end{itemize}
\begin{itemize}
\item {Grp. gram.:f.}
\end{itemize}
\begin{itemize}
\item {Proveniência:(De \textunderscore in...\textunderscore  + \textunderscore assiduidade\textunderscore )}
\end{itemize}
Falta de assiduidade.
\section{Inassimilável}
\begin{itemize}
\item {Grp. gram.:adj.}
\end{itemize}
\begin{itemize}
\item {Proveniência:(De \textunderscore in...\textunderscore  + \textunderscore assimilável\textunderscore )}
\end{itemize}
Que se não póde assimilar.
\section{Inassinável}
\begin{itemize}
\item {Grp. gram.:adj.}
\end{itemize}
\begin{itemize}
\item {Proveniência:(De \textunderscore in...\textunderscore  + \textunderscore assignável\textunderscore )}
\end{itemize}
Que não é assinável; que se não póde marcar ou assinalar.
\section{Inatacável}
\begin{itemize}
\item {Grp. gram.:adj.}
\end{itemize}
\begin{itemize}
\item {Proveniência:(De \textunderscore in...\textunderscore  + \textunderscore atacável\textunderscore )}
\end{itemize}
Que não é atacável; que se não póde atacar.
Incontestável.
\section{Inatendível}
\begin{itemize}
\item {Grp. gram.:adj.}
\end{itemize}
\begin{itemize}
\item {Proveniência:(De \textunderscore in...\textunderscore  + \textunderscore atendível\textunderscore )}
\end{itemize}
Que não póde ou não merece sêr atendido.
\section{Inatingível}
\begin{itemize}
\item {Grp. gram.:adj.}
\end{itemize}
\begin{itemize}
\item {Proveniência:(De \textunderscore in...\textunderscore  + \textunderscore atingível\textunderscore )}
\end{itemize}
Que se não atinge ou não póde sêr atingido.
\section{Inattendível}
\begin{itemize}
\item {Grp. gram.:adj.}
\end{itemize}
\begin{itemize}
\item {Proveniência:(De \textunderscore in...\textunderscore  + \textunderscore attendível\textunderscore )}
\end{itemize}
Que não póde ou não merece sêr attendido.
\section{Inattingível}
\begin{itemize}
\item {Grp. gram.:adj.}
\end{itemize}
\begin{itemize}
\item {Proveniência:(De \textunderscore in...\textunderscore  + \textunderscore attingível\textunderscore )}
\end{itemize}
Que se não attinge ou não póde sêr attingido.
\section{Inaturável}
\begin{itemize}
\item {Grp. gram.:adj.}
\end{itemize}
\begin{itemize}
\item {Proveniência:(De \textunderscore in...\textunderscore  + \textunderscore aturável\textunderscore )}
\end{itemize}
Que não é aturável; insupportável.
\section{Inaudito}
\begin{itemize}
\item {Grp. gram.:adj.}
\end{itemize}
\begin{itemize}
\item {Proveniência:(Lat. \textunderscore inauditus\textunderscore )}
\end{itemize}
Que nunca se ouviu; extraordinário; incrível: \textunderscore um desastre inaudito\textunderscore .
\section{Inaudível}
\begin{itemize}
\item {Grp. gram.:adj.}
\end{itemize}
\begin{itemize}
\item {Proveniência:(Lat. \textunderscore inaudibilis\textunderscore )}
\end{itemize}
Que se não póde ouvir.
\section{Inauferível}
\begin{itemize}
\item {Grp. gram.:adj.}
\end{itemize}
\begin{itemize}
\item {Proveniência:(De \textunderscore in...\textunderscore  + \textunderscore auferir\textunderscore )}
\end{itemize}
Que se não póde tirar; de que não póde privar alguém: \textunderscore direitos inauferíveis\textunderscore .
Inherente.
\section{Inauguração}
\begin{itemize}
\item {Grp. gram.:f.}
\end{itemize}
\begin{itemize}
\item {Proveniência:(Lat. \textunderscore inauguratio\textunderscore )}
\end{itemize}
Acto ou effeito de inaugurar.
Fundação.
Solennidade, com que se inaugura um estabelecimento, uma instituição, um edifício.
Início.
\section{Inaugurador}
\begin{itemize}
\item {Grp. gram.:m.  e  adj.}
\end{itemize}
O que inaugura.
\section{Inaugural}
\begin{itemize}
\item {Grp. gram.:adj.}
\end{itemize}
\begin{itemize}
\item {Proveniência:(De \textunderscore inaugurar\textunderscore )}
\end{itemize}
Relativo a inauguração; inicial: \textunderscore discurso inaugural\textunderscore .
\section{Inaugurar}
\begin{itemize}
\item {Grp. gram.:v. t.}
\end{itemize}
\begin{itemize}
\item {Proveniência:(Lat. \textunderscore inaugurare\textunderscore )}
\end{itemize}
Consagrar, dedicar.
Apresentar em público pela primeira vez.
Iniciar o serviço de; começar.
Estabelecer pela primeira vez: \textunderscore inaugurar uma escola\textunderscore .
\section{Inaugurativo}
\begin{itemize}
\item {Grp. gram.:adj.}
\end{itemize}
Próprio para inaugurar.
Relativo a inauguração: \textunderscore discurso inaugurativo\textunderscore .
\section{Inautenticidade}
\begin{itemize}
\item {Grp. gram.:f.}
\end{itemize}
\begin{itemize}
\item {Proveniência:(De \textunderscore in...\textunderscore  + \textunderscore autenticidade\textunderscore )}
\end{itemize}
Falta de autenticidade.
\section{Inautêntico}
\begin{itemize}
\item {Grp. gram.:adj.}
\end{itemize}
\begin{itemize}
\item {Proveniência:(De \textunderscore in...\textunderscore  + \textunderscore autêntico\textunderscore )}
\end{itemize}
Que não é autêntico.
Falso.
Fictício.
\section{Inauthenticidade}
\begin{itemize}
\item {Grp. gram.:f.}
\end{itemize}
\begin{itemize}
\item {Proveniência:(De \textunderscore in...\textunderscore  + \textunderscore authenticidade\textunderscore )}
\end{itemize}
Falta de authenticidade.
\section{Inauthêntico}
\begin{itemize}
\item {Grp. gram.:adj.}
\end{itemize}
\begin{itemize}
\item {Proveniência:(De \textunderscore in...\textunderscore  + \textunderscore authêntico\textunderscore )}
\end{itemize}
Que não é authêntico.
Falso.
Fictício.
\section{Inaveriguável}
\begin{itemize}
\item {Grp. gram.:adj.}
\end{itemize}
\begin{itemize}
\item {Proveniência:(De \textunderscore in...\textunderscore  + \textunderscore averiguável\textunderscore )}
\end{itemize}
Que se não póde averiguar.
\section{Inavistável}
\begin{itemize}
\item {Grp. gram.:adj.}
\end{itemize}
\begin{itemize}
\item {Proveniência:(De \textunderscore in...\textunderscore  + \textunderscore avistável\textunderscore )}
\end{itemize}
Que se não póde avistar.
\section{Inca}
\begin{itemize}
\item {Grp. gram.:m.}
\end{itemize}
Título dos Soberanos ou Príncipes do Peru, cuja dynastia foi destruída pela conquista dos espanhóis.
\section{Inçadoiro}
\begin{itemize}
\item {Grp. gram.:m.}
\end{itemize}
\begin{itemize}
\item {Utilização:Prov.}
\end{itemize}
\begin{itemize}
\item {Utilização:trasm.}
\end{itemize}
Correia de coiro, que prende o pírtigo á mangueira.
\section{Inçadouro}
\begin{itemize}
\item {Grp. gram.:m.}
\end{itemize}
\begin{itemize}
\item {Utilização:Prov.}
\end{itemize}
\begin{itemize}
\item {Utilização:trasm.}
\end{itemize}
Correia de couro, que prende o pírtigo á mangueira.
\section{Incalcinável}
\begin{itemize}
\item {Grp. gram.:adj.}
\end{itemize}
\begin{itemize}
\item {Proveniência:(De \textunderscore in...\textunderscore  + \textunderscore calcinável\textunderscore )}
\end{itemize}
Que não é calcinável.
\section{Incalculado}
\begin{itemize}
\item {Grp. gram.:adj.}
\end{itemize}
Que se não calculou; imprevisto.
\section{Incalculável}
\begin{itemize}
\item {Grp. gram.:adj.}
\end{itemize}
\begin{itemize}
\item {Proveniência:(De \textunderscore in...\textunderscore  + \textunderscore calculável\textunderscore )}
\end{itemize}
Que se não póde calcular; innumerável: \textunderscore multidão incalculável\textunderscore .
Incommensurável.
\section{Incalculavelmente}
\begin{itemize}
\item {Grp. gram.:adv.}
\end{itemize}
De modo incalculável.
\section{Incalumniável}
\begin{itemize}
\item {Grp. gram.:adj.}
\end{itemize}
\begin{itemize}
\item {Proveniência:(De \textunderscore in...\textunderscore  + \textunderscore calumniável\textunderscore )}
\end{itemize}
Que se não póde calumniar, que não é susceptível de sêr objecto de calúmnia.
\section{Incaluniável}
\begin{itemize}
\item {Grp. gram.:adj.}
\end{itemize}
\begin{itemize}
\item {Proveniência:(De \textunderscore in...\textunderscore  + \textunderscore caluniável\textunderscore )}
\end{itemize}
Que se não póde caluniar, que não é susceptível de sêr objecto de calúnia.
\section{Incameração}
\begin{itemize}
\item {Grp. gram.:f.}
\end{itemize}
\begin{itemize}
\item {Proveniência:(De \textunderscore incamerar\textunderscore )}
\end{itemize}
Encorporação de algum direito ou domínio nos haveres da Santa-Sé.
Passagem, para o Estado, de bens pertencentes a communidades.
\section{Incamerador}
\begin{itemize}
\item {Grp. gram.:m.}
\end{itemize}
Aquelle que é encarregado de incamerar.
\section{Incamerar}
\begin{itemize}
\item {Grp. gram.:v. t.}
\end{itemize}
\begin{itemize}
\item {Proveniência:(Do lat. \textunderscore in\textunderscore  + \textunderscore camerare\textunderscore )}
\end{itemize}
Reunir aos bens da Egreja.
\section{Incandescência}
\begin{itemize}
\item {Grp. gram.:f.}
\end{itemize}
Estado de incandescente.
\section{Incandescente}
\begin{itemize}
\item {Grp. gram.:adj.}
\end{itemize}
\begin{itemize}
\item {Utilização:Fig.}
\end{itemize}
\begin{itemize}
\item {Proveniência:(Lat. \textunderscore incandescens\textunderscore )}
\end{itemize}
Candente.
Pôsto em brasa; ardente.
Fogoso; exaltado.
\section{Incandescer}
\begin{itemize}
\item {Grp. gram.:v. t.}
\end{itemize}
\begin{itemize}
\item {Utilização:Fig.}
\end{itemize}
\begin{itemize}
\item {Grp. gram.:V. i.}
\end{itemize}
\begin{itemize}
\item {Utilização:Fig.}
\end{itemize}
\begin{itemize}
\item {Proveniência:(Lat. \textunderscore incandescere\textunderscore )}
\end{itemize}
Tornar candente; pôr em brasa.
Exaltar.
Tornar-se candente.
Exaltar-se.
\section{Incansável}
\begin{itemize}
\item {Grp. gram.:adj.}
\end{itemize}
\begin{itemize}
\item {Proveniência:(De \textunderscore in...\textunderscore  + \textunderscore cansar\textunderscore )}
\end{itemize}
Que se não cansa, que se não fatiga.
Laborioso.
Assíduo.
Activo.
\section{Incansavelmente}
\begin{itemize}
\item {Grp. gram.:adv.}
\end{itemize}
De modo incansável.
\section{Incapacidade}
\begin{itemize}
\item {Grp. gram.:f.}
\end{itemize}
\begin{itemize}
\item {Proveniência:(De \textunderscore in...\textunderscore  + \textunderscore capacidade\textunderscore )}
\end{itemize}
Falta de capacidade; inaptidão.
\section{Incapacitar}
\begin{itemize}
\item {Grp. gram.:v. t.}
\end{itemize}
\begin{itemize}
\item {Proveniência:(De \textunderscore in...\textunderscore  + \textunderscore capacitar\textunderscore )}
\end{itemize}
Tornar incapaz; tirar a aptidão a.
\section{Incapacitável}
\begin{itemize}
\item {Grp. gram.:adj.}
\end{itemize}
\begin{itemize}
\item {Proveniência:(De \textunderscore incapacitar\textunderscore )}
\end{itemize}
Que se não póde capacitar.
\section{Incapaz}
\begin{itemize}
\item {Grp. gram.:adj.}
\end{itemize}
\begin{itemize}
\item {Proveniência:(Lat. \textunderscore incapax\textunderscore )}
\end{itemize}
Que não tem capacidade.
Impossibilitado, inhábil: \textunderscore incapaz de trabalhar\textunderscore .
Indigno.
Ignorante.
\section{Incapilato}
\begin{itemize}
\item {Grp. gram.:adj.}
\end{itemize}
\begin{itemize}
\item {Utilização:Poét.}
\end{itemize}
\begin{itemize}
\item {Proveniência:(De \textunderscore in...\textunderscore  + \textunderscore capillus\textunderscore )}
\end{itemize}
Descabelado; calvo. Cf. \textunderscore Malaca Conq.\textunderscore , v. 21.
\section{Incapillato}
\begin{itemize}
\item {Grp. gram.:adj.}
\end{itemize}
\begin{itemize}
\item {Utilização:Poét.}
\end{itemize}
\begin{itemize}
\item {Proveniência:(De \textunderscore in...\textunderscore  + \textunderscore capillus\textunderscore )}
\end{itemize}
Descabellado; calvo. Cf. \textunderscore Malaca Conq.\textunderscore , v. 21.
\section{Inçar}
\begin{itemize}
\item {Grp. gram.:v. t.}
\end{itemize}
\begin{itemize}
\item {Utilização:Fig.}
\end{itemize}
Encher muito (de insectos e outros animaes).
Desenvolver-se em; encher.
Contagiar.
\section{Incaracterístico}
\begin{itemize}
\item {Grp. gram.:adj.}
\end{itemize}
\begin{itemize}
\item {Proveniência:(De \textunderscore in...\textunderscore  + \textunderscore característico\textunderscore )}
\end{itemize}
Que não é característico.
Confundivel.
\section{Incarnar}
\textunderscore v. t.\textunderscore  e \textunderscore i.\textunderscore  (e der.)
(V. \textunderscore encarnar\textunderscore ^2, etc.)
\section{Incásico}
\begin{itemize}
\item {Grp. gram.:adj.}
\end{itemize}
Relativo á dynastia dos Incas.
\section{Incasto}
\begin{itemize}
\item {Grp. gram.:adj.}
\end{itemize}
\begin{itemize}
\item {Proveniência:(Lat. \textunderscore incastus\textunderscore )}
\end{itemize}
Que não é casto; impudico.
\section{Incautamente}
\begin{itemize}
\item {Grp. gram.:adv.}
\end{itemize}
De modo incauto; sem cautela.
\section{Incauto}
\begin{itemize}
\item {Grp. gram.:adj.}
\end{itemize}
\begin{itemize}
\item {Grp. gram.:M.}
\end{itemize}
\begin{itemize}
\item {Proveniência:(Lat. \textunderscore incautus\textunderscore )}
\end{itemize}
Que não é cauto; que não tem cautela.
Imprudente.
Desprevenido.
Crédulo; ingênuo.
Aquelle que não é cauto: \textunderscore prevenir os incautos\textunderscore .
\section{Incender}
\textunderscore v. t.\textunderscore  (e der.)
(V. \textunderscore encender\textunderscore , etc)
\section{Incendiar}
\begin{itemize}
\item {Grp. gram.:v. t.}
\end{itemize}
\begin{itemize}
\item {Utilização:Fig.}
\end{itemize}
Atear incêndio em.
Abrasar.
Pôr fogo a: \textunderscore incendiar uma casa\textunderscore .
Estimular; excitar.
\section{Incendiário}
\begin{itemize}
\item {Grp. gram.:adj.}
\end{itemize}
\begin{itemize}
\item {Utilização:Fig.}
\end{itemize}
\begin{itemize}
\item {Grp. gram.:M.}
\end{itemize}
\begin{itemize}
\item {Utilização:Fig.}
\end{itemize}
\begin{itemize}
\item {Proveniência:(Lat. \textunderscore incendiarius\textunderscore )}
\end{itemize}
Que é próprio para incendiar; que communica incêndio.
Excitante.
Aquelle que causa incêndio voluntariamente.
Revolucionário.
\section{Incendiável}
\begin{itemize}
\item {Grp. gram.:adj.}
\end{itemize}
Que se póde incendiar.
\section{Incêndio}
\begin{itemize}
\item {Grp. gram.:m.}
\end{itemize}
\begin{itemize}
\item {Utilização:Fig.}
\end{itemize}
\begin{itemize}
\item {Proveniência:(Lat. \textunderscore incendium\textunderscore )}
\end{itemize}
Acto ou effeito de incender ou de abrasar.
Fogo, que lavra extensamente.
Grande ardor.
Conflagração; calamidade.
\section{Incendioso}
\begin{itemize}
\item {Grp. gram.:adj.}
\end{itemize}
Relativo a incêndio. Cf. Castilho, \textunderscore Metam.\textunderscore , 179.
\section{Incensação}
\begin{itemize}
\item {Grp. gram.:f.}
\end{itemize}
Acto ou effeito de incensar.
\section{Incensadela}
\begin{itemize}
\item {Grp. gram.:f.}
\end{itemize}
(V.incensação)
\section{Incensador}
\begin{itemize}
\item {Grp. gram.:m.  e  adj.}
\end{itemize}
O que incensa.
\section{Incensairo}
\begin{itemize}
\item {Grp. gram.:m.}
\end{itemize}
(Fórma ant. de \textunderscore insensário\textunderscore )
\section{Incensamento}
\begin{itemize}
\item {Grp. gram.:m.}
\end{itemize}
Acto de incensar.
\section{Incensar}
\begin{itemize}
\item {Grp. gram.:v. t.}
\end{itemize}
\begin{itemize}
\item {Utilização:Ext.}
\end{itemize}
\begin{itemize}
\item {Utilização:Fig.}
\end{itemize}
\begin{itemize}
\item {Grp. gram.:V. i.}
\end{itemize}
\begin{itemize}
\item {Utilização:Prov.}
\end{itemize}
\begin{itemize}
\item {Utilização:trasm.}
\end{itemize}
Defumar com incenso; queimar incenso, junto de: \textunderscore incensar os altares\textunderscore .
Defumar; perfumar.
Thurificar, adular; illudir com lisonjas.
Andar de um lado para o outro.
\section{Incensário}
\begin{itemize}
\item {Grp. gram.:m.}
\end{itemize}
\begin{itemize}
\item {Proveniência:(De \textunderscore incenso\textunderscore )}
\end{itemize}
Thuríbulo, utensílio para incensar.
\section{Incenso}
\begin{itemize}
\item {Grp. gram.:m.}
\end{itemize}
\begin{itemize}
\item {Utilização:Fig.}
\end{itemize}
\begin{itemize}
\item {Proveniência:(Lat. \textunderscore incensum\textunderscore )}
\end{itemize}
Resina aromática, que se queima nas festas de igreja, e que é extrahida de uma árvore terebinthácea.
Árvore açoreana e indiana.
Louvor exaggerado; lisonja.
Preito.
\section{Incensório}
\begin{itemize}
\item {Grp. gram.:m.}
\end{itemize}
O mesmo que \textunderscore incensário\textunderscore .
\section{Incensurável}
\begin{itemize}
\item {Grp. gram.:adj.}
\end{itemize}
\begin{itemize}
\item {Proveniência:(De \textunderscore in...\textunderscore  + \textunderscore censurável\textunderscore )}
\end{itemize}
Que não é censurável; correcto; impolluto.
\section{Incentiva}
\begin{itemize}
\item {Grp. gram.:f.}
\end{itemize}
\begin{itemize}
\item {Utilização:Ant.}
\end{itemize}
\begin{itemize}
\item {Proveniência:(De \textunderscore incentivo\textunderscore )}
\end{itemize}
Frauta que, á direita da orchestra nos theatros antigos, começava a música, seguindo-a os outros instrumentos.
\section{Incentivo}
\begin{itemize}
\item {Grp. gram.:adj.}
\end{itemize}
\begin{itemize}
\item {Grp. gram.:M.}
\end{itemize}
\begin{itemize}
\item {Proveniência:(Lat. \textunderscore incentivus\textunderscore )}
\end{itemize}
Estimulante; que excita.
Aquillo que estimula; estímulo.
\section{Incentor}
\begin{itemize}
\item {Grp. gram.:m.}
\end{itemize}
\begin{itemize}
\item {Proveniência:(Lat. \textunderscore incentor\textunderscore )}
\end{itemize}
Aquelle que incita.
\section{Inceremonioso}
\begin{itemize}
\item {Grp. gram.:adj.}
\end{itemize}
\begin{itemize}
\item {Proveniência:(De \textunderscore in...\textunderscore  + \textunderscore ceremonioso\textunderscore )}
\end{itemize}
Que não é ceremonioso; que não usa ou não gosta de ceremónias.
\section{Incerne}
\begin{itemize}
\item {Grp. gram.:adj.}
\end{itemize}
\begin{itemize}
\item {Utilização:Prov.}
\end{itemize}
Cuidadoso, zeloso.
Frenético no trabalho. (Colhido no concelho de Lamego)
\section{Incertamente}
\begin{itemize}
\item {Grp. gram.:adv.}
\end{itemize}
De modo incerto.
\section{Incertar}
\begin{itemize}
\item {Grp. gram.:v. t.}
\end{itemize}
Tornar incerto.
Têr como incerto; pôr dúvida em. Cf. Filinto, XIV, 186.
\section{Incerteza}
\begin{itemize}
\item {Grp. gram.:f.}
\end{itemize}
\begin{itemize}
\item {Proveniência:(De \textunderscore in...\textunderscore  + \textunderscore certeza\textunderscore )}
\end{itemize}
Falta de certeza.
Estado do que é incerto; hesitação, dúvida.
\section{Incerto}
\begin{itemize}
\item {Grp. gram.:adj.}
\end{itemize}
\begin{itemize}
\item {Grp. gram.:M.}
\end{itemize}
\begin{itemize}
\item {Proveniência:(De \textunderscore in...\textunderscore  + \textunderscore certo\textunderscore )}
\end{itemize}
Que não é certo; indeciso; hesitante.
Variável.
Precário; contingente.
Aquillo que não é certo.
\section{Incessante}
\begin{itemize}
\item {Grp. gram.:adj.}
\end{itemize}
\begin{itemize}
\item {Proveniência:(Lat. \textunderscore incessans\textunderscore )}
\end{itemize}
Que não cessa; contínuo.
Constante; assíduo.
\section{Incessantemente}
\begin{itemize}
\item {Grp. gram.:adv.}
\end{itemize}
De modo incessante.
\section{Incessável}
\begin{itemize}
\item {Grp. gram.:adj.}
\end{itemize}
\begin{itemize}
\item {Proveniência:(Lat. \textunderscore incessabilis\textunderscore )}
\end{itemize}
O mesmo que \textunderscore incessante\textunderscore .
\section{Incessibilidade}
\begin{itemize}
\item {Grp. gram.:f.}
\end{itemize}
Qualidade de incessível.
\section{Incessível}
\begin{itemize}
\item {Grp. gram.:adj.}
\end{itemize}
\begin{itemize}
\item {Proveniência:(De \textunderscore in...\textunderscore  + \textunderscore cessível\textunderscore )}
\end{itemize}
Que não é cessível; que se não póde ceder.
\section{Incesso}
\begin{itemize}
\item {Grp. gram.:m.}
\end{itemize}
\begin{itemize}
\item {Utilização:Des.}
\end{itemize}
\begin{itemize}
\item {Proveniência:(Lat. \textunderscore incessus\textunderscore )}
\end{itemize}
Acto de andar, de marchar.
\section{Incestamente}
\begin{itemize}
\item {Grp. gram.:adv.}
\end{itemize}
\begin{itemize}
\item {Proveniência:(Do lat. \textunderscore incestus\textunderscore )}
\end{itemize}
O mesmo que \textunderscore incestuosamente\textunderscore . Cf. Vieira, \textunderscore Sermões\textunderscore , IX, 412.
\section{Incestar}
\begin{itemize}
\item {Grp. gram.:v. t.}
\end{itemize}
\begin{itemize}
\item {Grp. gram.:V. i.}
\end{itemize}
\begin{itemize}
\item {Proveniência:(Lat. \textunderscore incestare\textunderscore )}
\end{itemize}
Deshonrar com incesto.
Commeter incesto.
\section{Incesto}
\begin{itemize}
\item {Grp. gram.:m.}
\end{itemize}
\begin{itemize}
\item {Grp. gram.:Adj.}
\end{itemize}
\begin{itemize}
\item {Proveniência:(Lat. \textunderscore incestus\textunderscore )}
\end{itemize}
União illícita, entre parentes.
Torpe; incasto.
\section{Incestuosamente}
\begin{itemize}
\item {Grp. gram.:adv.}
\end{itemize}
De modo incestuoso; por meio de incesto.
\section{Incestuoso}
\begin{itemize}
\item {Grp. gram.:adj.}
\end{itemize}
\begin{itemize}
\item {Proveniência:(Lat. \textunderscore incestuosus\textunderscore )}
\end{itemize}
Relativo a incesto.
Que praticou incesto.
Que procede de incesto.
\section{Incha}
\begin{itemize}
\item {Grp. gram.:f.}
\end{itemize}
\begin{itemize}
\item {Utilização:des.}
\end{itemize}
\begin{itemize}
\item {Utilização:Pleb.}
\end{itemize}
Aversão; desavença.
\section{Incha}
\begin{itemize}
\item {Grp. gram.:f.}
\end{itemize}
\begin{itemize}
\item {Utilização:Açor}
\end{itemize}
\begin{itemize}
\item {Proveniência:(De \textunderscore inchar\textunderscore )}
\end{itemize}
Onda grande.
\section{Inchação}
\begin{itemize}
\item {Grp. gram.:f.}
\end{itemize}
\begin{itemize}
\item {Utilização:Pop.}
\end{itemize}
\begin{itemize}
\item {Utilização:Fam.}
\end{itemize}
\begin{itemize}
\item {Proveniência:(Do lat. \textunderscore inflatio\textunderscore )}
\end{itemize}
Acto ou effeito de inchar.
Tumor; anasarca.
Vaidade; arrogância.
\section{Inchaço}
\begin{itemize}
\item {Grp. gram.:m.}
\end{itemize}
\begin{itemize}
\item {Utilização:Pop.}
\end{itemize}
(V.inchação)
\section{Inchadamente}
\begin{itemize}
\item {Grp. gram.:adv.}
\end{itemize}
\begin{itemize}
\item {Proveniência:(De \textunderscore inchado\textunderscore )}
\end{itemize}
Com inchação; arrogantemente.
\section{Inchado}
\begin{itemize}
\item {Grp. gram.:adj.}
\end{itemize}
\begin{itemize}
\item {Utilização:Fig.}
\end{itemize}
\begin{itemize}
\item {Proveniência:(De \textunderscore inchar\textunderscore )}
\end{itemize}
Entumecido, por doença, (falando-se de um indivíduo, ou de qualquer parte externa do corpo): \textunderscore um braço inchado\textunderscore .
Empolado.
Vaidoso.
\section{Inchamento}
\begin{itemize}
\item {Grp. gram.:m.}
\end{itemize}
(V.inchação)
\section{Inchar}
\begin{itemize}
\item {Grp. gram.:v. t.}
\end{itemize}
\begin{itemize}
\item {Utilização:Fig.}
\end{itemize}
\begin{itemize}
\item {Grp. gram.:V. i.  e  p.}
\end{itemize}
\begin{itemize}
\item {Utilização:Fig.}
\end{itemize}
\begin{itemize}
\item {Proveniência:(Do lat. \textunderscore inflare\textunderscore )}
\end{itemize}
Avolumar; entumecer.
Engrossar.
Envaidar.
Tornar soberbo.
Tornar emphático, empolado: \textunderscore inchar o estilo\textunderscore .
Tornar-se túmido; avolumar-se.
Ensoberbecer-se; envaidar-se.
\section{Inchario}
\begin{itemize}
\item {Grp. gram.:m.}
\end{itemize}
\begin{itemize}
\item {Utilização:Des.}
\end{itemize}
Casta de figo branco, muito doce. Cf. B. Pereira, \textunderscore Prosòdia\textunderscore , vb. \textunderscore ona\textunderscore .
(Outra forma de \textunderscore asserio\textunderscore ?)
\section{Inchinda}
\begin{itemize}
\item {Grp. gram.:f.}
\end{itemize}
O mesmo que \textunderscore transpiração\textunderscore , arbusto.
\section{Inchoação}
\begin{itemize}
\item {fónica:co}
\end{itemize}
\begin{itemize}
\item {Grp. gram.:f.}
\end{itemize}
\begin{itemize}
\item {Proveniência:(Lat. \textunderscore inchoatio\textunderscore )}
\end{itemize}
Começo.
\section{Inchoar}
\begin{itemize}
\item {fónica:co}
\end{itemize}
\begin{itemize}
\item {Grp. gram.:v. t.}
\end{itemize}
\begin{itemize}
\item {Proveniência:(Lat. \textunderscore inchoare\textunderscore )}
\end{itemize}
O mesmo que \textunderscore começar\textunderscore .
\section{Inchoativo}
\begin{itemize}
\item {fónica:co}
\end{itemize}
\begin{itemize}
\item {Grp. gram.:adj.}
\end{itemize}
\begin{itemize}
\item {Utilização:Gram.}
\end{itemize}
\begin{itemize}
\item {Proveniência:(Lat. \textunderscore inchoativus\textunderscore )}
\end{itemize}
Que começa.
Que dá comêço a.
Que exprime aumento progressivo de acção, (falando-se de verbos).
\section{Inchume}
\begin{itemize}
\item {Grp. gram.:m.}
\end{itemize}
\begin{itemize}
\item {Utilização:Bras. de Minas}
\end{itemize}
O mesmo que \textunderscore inchação\textunderscore .
\section{Incicatrizável}
\begin{itemize}
\item {Grp. gram.:adj.}
\end{itemize}
\begin{itemize}
\item {Proveniência:(De \textunderscore in...\textunderscore  + \textunderscore cicatrizável\textunderscore )}
\end{itemize}
Que não é cicatrizável.
\section{Incidência}
\begin{itemize}
\item {Grp. gram.:f.}
\end{itemize}
\begin{itemize}
\item {Utilização:Geom.}
\end{itemize}
\begin{itemize}
\item {Proveniência:(De \textunderscore incidente\textunderscore )}
\end{itemize}
Qualidade do que é incidente.
Acção de incidir.
Encontro de duas linhas ou superfícies.
\section{Incidentado}
\begin{itemize}
\item {Grp. gram.:adj.}
\end{itemize}
Cheio de incidentes.
\section{Incidental}
\begin{itemize}
\item {Grp. gram.:adj.}
\end{itemize}
Relativo a incidente.
Accidental.
\section{Incidente}
\begin{itemize}
\item {Grp. gram.:adj.}
\end{itemize}
\begin{itemize}
\item {Grp. gram.:M.}
\end{itemize}
\begin{itemize}
\item {Proveniência:(Lat. \textunderscore incidens\textunderscore )}
\end{itemize}
Que incide.
Superveniente.
Circunstância accidental; facto que sobrevém; episódio.
\section{Incidentemente}
\begin{itemize}
\item {Grp. gram.:adv.}
\end{itemize}
De modo incidente.
\section{Incidido}
\begin{itemize}
\item {Grp. gram.:adj.}
\end{itemize}
\begin{itemize}
\item {Utilização:Med.}
\end{itemize}
\begin{itemize}
\item {Utilização:Rhet.}
\end{itemize}
\begin{itemize}
\item {Proveniência:(De \textunderscore incidir\textunderscore ^1)}
\end{itemize}
Deminuido.
Cortado de incisos.
\section{Incidir}
\begin{itemize}
\item {Grp. gram.:v. t.}
\end{itemize}
\begin{itemize}
\item {Utilização:Ant.}
\end{itemize}
\begin{itemize}
\item {Proveniência:(Lat. \textunderscore incidere\textunderscore , de \textunderscore in\textunderscore  + \textunderscore caedere\textunderscore )}
\end{itemize}
Attenuar; atalhar.
\section{Incidir}
\begin{itemize}
\item {Grp. gram.:v. i.}
\end{itemize}
\begin{itemize}
\item {Proveniência:(Lat. \textunderscore incidere\textunderscore , de \textunderscore in\textunderscore  + \textunderscore cadere\textunderscore )}
\end{itemize}
Sobrevir; caír sôbre alguma coisa.
Acontecer.
\section{Incineração}
\begin{itemize}
\item {Grp. gram.:f.}
\end{itemize}
Acto ou effeito de incinerar.
\section{Incinerar}
\begin{itemize}
\item {Grp. gram.:v. t.}
\end{itemize}
\begin{itemize}
\item {Proveniência:(Do lat. \textunderscore cinis\textunderscore , \textunderscore cineris\textunderscore )}
\end{itemize}
Reduzir a cinzas: \textunderscore incinerar um cadáver\textunderscore .
\section{Incindir}
\begin{itemize}
\item {Grp. gram.:v. t.}
\end{itemize}
\begin{itemize}
\item {Utilização:Des.}
\end{itemize}
\begin{itemize}
\item {Proveniência:(Do lat. \textunderscore in\textunderscore  + \textunderscore scindere\textunderscore )}
\end{itemize}
Dividir, separar.
\section{Incipiente}
\begin{itemize}
\item {Grp. gram.:adj.}
\end{itemize}
\begin{itemize}
\item {Proveniência:(Lat. \textunderscore incipiens\textunderscore )}
\end{itemize}
Que está no princípio; que começa; principiante: \textunderscore poeta incipiente\textunderscore .
\section{Incircumcidado}
\begin{itemize}
\item {Grp. gram.:adj.}
\end{itemize}
\begin{itemize}
\item {Proveniência:(De \textunderscore in...\textunderscore  + \textunderscore circumcidado\textunderscore )}
\end{itemize}
Que não foi circumcidado.
\section{Incircumciso}
\begin{itemize}
\item {Grp. gram.:adj.}
\end{itemize}
\begin{itemize}
\item {Proveniência:(Do lat. \textunderscore incircumcisus\textunderscore )}
\end{itemize}
O mesmo que \textunderscore incircumcidado\textunderscore .
\section{Incircumscriptível}
\begin{itemize}
\item {Grp. gram.:adj.}
\end{itemize}
\begin{itemize}
\item {Proveniência:(De \textunderscore incircumscripto\textunderscore )}
\end{itemize}
Que se não póde circumscrever.
\section{Incircumscripto}
\begin{itemize}
\item {Grp. gram.:adj.}
\end{itemize}
\begin{itemize}
\item {Proveniência:(De \textunderscore in...\textunderscore  + \textunderscore circumscripto\textunderscore )}
\end{itemize}
Que não é circumscripto.
\section{Incircuncidado}
\begin{itemize}
\item {Grp. gram.:adj.}
\end{itemize}
\begin{itemize}
\item {Proveniência:(De \textunderscore in...\textunderscore  + \textunderscore circuncidado\textunderscore )}
\end{itemize}
Que não foi circuncidado.
\section{Incircunciso}
\begin{itemize}
\item {Grp. gram.:adj.}
\end{itemize}
\begin{itemize}
\item {Proveniência:(Do lat. \textunderscore incircumcisus\textunderscore )}
\end{itemize}
O mesmo que \textunderscore incircuncidado\textunderscore .
\section{Incircunscritível}
\begin{itemize}
\item {Grp. gram.:adj.}
\end{itemize}
\begin{itemize}
\item {Proveniência:(De \textunderscore incircunscripto\textunderscore )}
\end{itemize}
Que se não póde circunscrever.
\section{Incircunscrito}
\begin{itemize}
\item {Grp. gram.:adj.}
\end{itemize}
\begin{itemize}
\item {Proveniência:(De \textunderscore in...\textunderscore  + \textunderscore circunscripto\textunderscore )}
\end{itemize}
Que não é circunscripto.
\section{Incisamente}
\begin{itemize}
\item {Grp. gram.:adv.}
\end{itemize}
De modo inciso.
\section{Incisão}
\begin{itemize}
\item {Grp. gram.:f.}
\end{itemize}
\begin{itemize}
\item {Proveniência:(Lat. \textunderscore incisio\textunderscore )}
\end{itemize}
O mesmo que \textunderscore córte\textunderscore ^1.
\section{Incisar}
\begin{itemize}
\item {Grp. gram.:v. i.}
\end{itemize}
\begin{itemize}
\item {Utilização:Neol.}
\end{itemize}
\begin{itemize}
\item {Proveniência:(De \textunderscore inciso\textunderscore )}
\end{itemize}
Fazer incisão em.
\section{Incisivamente}
\begin{itemize}
\item {Grp. gram.:adv.}
\end{itemize}
De modo incisivo.
Efficazmente.
Terminantemente; com energia: \textunderscore falar incisivamente\textunderscore .
\section{Incisivo}
\begin{itemize}
\item {Grp. gram.:adj.}
\end{itemize}
\begin{itemize}
\item {Utilização:Fig.}
\end{itemize}
\begin{itemize}
\item {Grp. gram.:M.}
\end{itemize}
\begin{itemize}
\item {Proveniência:(De \textunderscore inciso\textunderscore )}
\end{itemize}
Que corta.
Efficaz.
Que penetra.
Diz-se do estilo conciso, cortante e enérgico.
Mordaz.
Cada um dos quatro dentes, situados na parte média e anterior de cada maxilla, entre os caninos ou presas.
\section{Inciso}
\begin{itemize}
\item {Grp. gram.:adj.}
\end{itemize}
\begin{itemize}
\item {Grp. gram.:M.}
\end{itemize}
\begin{itemize}
\item {Proveniência:(Lat. \textunderscore incisus\textunderscore )}
\end{itemize}
Ferido com gume de instrumento cortante.
Cortado.
Phrase, que interrompe o sentido de outra.
Cada um dos membros de uma phrase musical.
\section{Incisor}
\begin{itemize}
\item {Grp. gram.:adj.}
\end{itemize}
\begin{itemize}
\item {Grp. gram.:M.}
\end{itemize}
\begin{itemize}
\item {Proveniência:(Lat. \textunderscore incisor\textunderscore )}
\end{itemize}
O mesmo que \textunderscore incisório\textunderscore .
Aquelle ou aquillo que corta.
\section{Incisório}
\begin{itemize}
\item {Grp. gram.:adj.}
\end{itemize}
\begin{itemize}
\item {Proveniência:(Do lat. \textunderscore incisus\textunderscore )}
\end{itemize}
Que corta; que é incisivo.
\section{Incisura}
\begin{itemize}
\item {Grp. gram.:f.}
\end{itemize}
\begin{itemize}
\item {Proveniência:(Lat. \textunderscore incisura\textunderscore )}
\end{itemize}
O mesmo que \textunderscore incisão\textunderscore .
\section{Incitabilidade}
\begin{itemize}
\item {Grp. gram.:f.}
\end{itemize}
\begin{itemize}
\item {Proveniência:(Do lat. \textunderscore incitabilis\textunderscore )}
\end{itemize}
Qualidade de incitável.
\section{Incitação}
\begin{itemize}
\item {Grp. gram.:f.}
\end{itemize}
\begin{itemize}
\item {Proveniência:(Lat. \textunderscore incitatio\textunderscore )}
\end{itemize}
Acto ou effeito de incitar.
Estímulo.
Provocação.
Tonificação do organismo.
\section{Incitador}
\begin{itemize}
\item {Grp. gram.:adj.}
\end{itemize}
\begin{itemize}
\item {Grp. gram.:M.}
\end{itemize}
\begin{itemize}
\item {Proveniência:(Lat. \textunderscore incitator\textunderscore )}
\end{itemize}
Que incita.
Aquelle que incita.
\section{Incitamento}
\begin{itemize}
\item {Grp. gram.:m.}
\end{itemize}
\begin{itemize}
\item {Proveniência:(Lat. \textunderscore incitamentum\textunderscore )}
\end{itemize}
O mesmo que \textunderscore incitação\textunderscore .
\section{Incitante}
\begin{itemize}
\item {Grp. gram.:adj.}
\end{itemize}
\begin{itemize}
\item {Proveniência:(Lat. \textunderscore incitans\textunderscore )}
\end{itemize}
Que incita.
\section{Incitar}
\begin{itemize}
\item {Grp. gram.:v. t.}
\end{itemize}
\begin{itemize}
\item {Proveniência:(Lat. \textunderscore incitare\textunderscore )}
\end{itemize}
Impellir com vehemência; abalar.
Instigar.
Excitar.
Açular: \textunderscore incitar os cães\textunderscore .
Provocar.
Enraivecer.
\section{Incitativamente}
\begin{itemize}
\item {Grp. gram.:adv.}
\end{itemize}
De modo incitativo.
Com estímulo.
\section{Incitativo}
\begin{itemize}
\item {Grp. gram.:adj.}
\end{itemize}
O mesmo que \textunderscore incitante\textunderscore .
\section{Incitável}
\begin{itemize}
\item {Grp. gram.:adj.}
\end{itemize}
\begin{itemize}
\item {Proveniência:(Lat. \textunderscore incitabilis\textunderscore )}
\end{itemize}
Que póde sêr incitado; que se incita facilmente.
\section{Incitega}
\begin{itemize}
\item {Grp. gram.:f.}
\end{itemize}
\begin{itemize}
\item {Proveniência:(Lat. \textunderscore incitega\textunderscore )}
\end{itemize}
Maquineta, em que, nos banquetes romanos, se collocava a âmphora, donde se extrahia o vinho para os copos dos commensaes.
\section{Incito-motor}
\begin{itemize}
\item {Grp. gram.:adj.}
\end{itemize}
\begin{itemize}
\item {Utilização:Physiol.}
\end{itemize}
\begin{itemize}
\item {Proveniência:(De \textunderscore incitar\textunderscore  + \textunderscore motor\textunderscore )}
\end{itemize}
Que produz a contracção muscular, (falando-se do influxo dos nervos).
\section{Incito-motriz}
\begin{itemize}
\item {Grp. gram.:adj.}
\end{itemize}
(\textunderscore fem.\textunderscore  de \textunderscore incito-motor\textunderscore )
\section{Incivil}
\begin{itemize}
\item {Grp. gram.:adj.}
\end{itemize}
\begin{itemize}
\item {Proveniência:(Lat. \textunderscore incivilis\textunderscore )}
\end{itemize}
Que não é civil, que não tem civilidade; descortês.
\section{Incivilidade}
\begin{itemize}
\item {Grp. gram.:f.}
\end{itemize}
\begin{itemize}
\item {Proveniência:(Lat. \textunderscore incivilitas\textunderscore )}
\end{itemize}
Qualidade de incivil; falta de civilidade.
Acto ou expressão incivil.
\section{Incivilizado}
\begin{itemize}
\item {Grp. gram.:adj.}
\end{itemize}
\begin{itemize}
\item {Proveniência:(De \textunderscore in...\textunderscore  + \textunderscore civilizado\textunderscore )}
\end{itemize}
Que não é civilizado; inculto; selvático.
\section{Incivilizável}
\begin{itemize}
\item {Grp. gram.:adj.}
\end{itemize}
\begin{itemize}
\item {Proveniência:(De \textunderscore in...\textunderscore  + \textunderscore civilizável\textunderscore )}
\end{itemize}
Que não é civilizável.
\section{Incivilmente}
\begin{itemize}
\item {Grp. gram.:adv.}
\end{itemize}
De modo incivil; com descortesia.
\section{Incivismo}
\begin{itemize}
\item {Grp. gram.:m.}
\end{itemize}
Falta de civismo ou de patriotismo. Cf. Rui Barb., \textunderscore Réplica\textunderscore , 157.
\section{Inclassificado}
\begin{itemize}
\item {Grp. gram.:adj.}
\end{itemize}
Não classificado. Cf. Th. Ribeiro, \textunderscore Jornadas\textunderscore , I, 285.
\section{Inclassificável}
\begin{itemize}
\item {Grp. gram.:adj.}
\end{itemize}
\begin{itemize}
\item {Utilização:Ext.}
\end{itemize}
\begin{itemize}
\item {Proveniência:(De \textunderscore in...\textunderscore  + \textunderscore classificável\textunderscore )}
\end{itemize}
Que se não póde classificar; que está em confusão.
Inqualificável; digno de censura ou reprovação.
\section{Inclemência}
\begin{itemize}
\item {Grp. gram.:f.}
\end{itemize}
\begin{itemize}
\item {Proveniência:(Lat. \textunderscore inclementia\textunderscore )}
\end{itemize}
Qualidade de inclemente; falta de clemência; severidade; rigor.
\section{Inclemente}
\begin{itemize}
\item {Grp. gram.:adj.}
\end{itemize}
\begin{itemize}
\item {Proveniência:(Lat. \textunderscore inclemens\textunderscore )}
\end{itemize}
Que não é clemente.
Áspero; severo; rigoroso; desagradável: \textunderscore inverno inclemente\textunderscore .
\section{Inclementemente}
\begin{itemize}
\item {Grp. gram.:adv.}
\end{itemize}
De modo inclemente.
\section{Inclinação}
\begin{itemize}
\item {Grp. gram.:f.}
\end{itemize}
\begin{itemize}
\item {Utilização:Fig.}
\end{itemize}
\begin{itemize}
\item {Proveniência:(Lat. \textunderscore inclinatio\textunderscore )}
\end{itemize}
Acto ou effeito de inclinar.
Estado daquillo que se acha inclinado.
Tendência, propensão.
Pessôa estimada ou amada.
\section{Inclinadamente}
\begin{itemize}
\item {Grp. gram.:adv.}
\end{itemize}
De modo inclinado; com inclinação.
\section{Inclinado}
\begin{itemize}
\item {Grp. gram.:adj.}
\end{itemize}
\begin{itemize}
\item {Utilização:Fig.}
\end{itemize}
Desviado da linha vertical.
Que está pendente ou curvo.
Affeiçoado.
Tendente; disposto.
\section{Inclinar}
\begin{itemize}
\item {Grp. gram.:v. t.}
\end{itemize}
\begin{itemize}
\item {Utilização:Fig.}
\end{itemize}
\begin{itemize}
\item {Grp. gram.:V. i.}
\end{itemize}
\begin{itemize}
\item {Proveniência:(Lat. \textunderscore inclinare\textunderscore )}
\end{itemize}
Desviar da verticalidade.
Tornar oblíquo, relativamente a um plano.
Curvar, fazer pender: \textunderscore inclinar a cabeça\textunderscore .
Recostar.
Abater.
Dar tendência a.
Tornar propenso.
Dirigir, fazendo curva.
Pender.
Descair: \textunderscore já o Sol inclinava\textunderscore .
Tender, propender.
\section{Inclinável}
\begin{itemize}
\item {Grp. gram.:adj.}
\end{itemize}
\begin{itemize}
\item {Proveniência:(Lat. \textunderscore inclinabilis\textunderscore )}
\end{itemize}
Que se inclina facilmente.
\section{Inclitamente}
\begin{itemize}
\item {Grp. gram.:adv.}
\end{itemize}
\begin{itemize}
\item {Proveniência:(De \textunderscore ínclito\textunderscore )}
\end{itemize}
Com fama; de modo célebre.
\section{Ínclito}
\begin{itemize}
\item {Grp. gram.:adj.}
\end{itemize}
\begin{itemize}
\item {Proveniência:(Lat. \textunderscore inclitus\textunderscore )}
\end{itemize}
Egrégio; celebrado; illustre: \textunderscore ínclitos varões\textunderscore .
\section{Includir}
\begin{itemize}
\item {Grp. gram.:v. t.}
\end{itemize}
\begin{itemize}
\item {Utilização:Ant.}
\end{itemize}
O mesmo que \textunderscore incluir\textunderscore .
\section{Incluir}
\begin{itemize}
\item {Grp. gram.:v. t.}
\end{itemize}
\begin{itemize}
\item {Proveniência:(Lat. \textunderscore includere\textunderscore )}
\end{itemize}
Fechar, encerrar.
Envolver em.
Abranger: \textunderscore incluir alguém no número dos despeitados\textunderscore .
Inserir.
\section{Inclusa}
\begin{itemize}
\item {Grp. gram.:f.}
\end{itemize}
\begin{itemize}
\item {Utilização:Des.}
\end{itemize}
O mesmo que \textunderscore esclusa\textunderscore  ou \textunderscore comporta\textunderscore .
(Cast. \textunderscore inclusa\textunderscore )
\section{Inclusa}
\begin{itemize}
\item {Grp. gram.:f.}
\end{itemize}
Antiga moéda da Hollanda. Cp. F. Manuel, \textunderscore Apólogos\textunderscore .
\section{Inclusão}
\begin{itemize}
\item {Grp. gram.:f.}
\end{itemize}
\begin{itemize}
\item {Proveniência:(Lat. \textunderscore inclusio\textunderscore )}
\end{itemize}
Acto ou effeito de incluir.
\section{Inclusiva}
\begin{itemize}
\item {Grp. gram.:f.}
\end{itemize}
\begin{itemize}
\item {Proveniência:(De \textunderscore inclusivo\textunderscore )}
\end{itemize}
Acção de admittir no conclave um Cardeal, que chegou depois de findo o prazo.
\section{Inclusivamente}
\begin{itemize}
\item {Grp. gram.:adv.}
\end{itemize}
De modo inclusivo.
\section{Inclusivo}
\begin{itemize}
\item {Grp. gram.:adj.}
\end{itemize}
\begin{itemize}
\item {Proveniência:(De \textunderscore incluso\textunderscore )}
\end{itemize}
Que inclue; que abrange.
\section{Incluso}
\begin{itemize}
\item {Proveniência:(Lat. \textunderscore inclusus\textunderscore )}
\end{itemize}
Que se incluiu.
Abrangido, comprehendido.
\section{Inço}
\begin{itemize}
\item {Grp. gram.:m.}
\end{itemize}
\begin{itemize}
\item {Utilização:Prov.}
\end{itemize}
\begin{itemize}
\item {Utilização:alg.}
\end{itemize}
\begin{itemize}
\item {Utilização:Prov.}
\end{itemize}
\begin{itemize}
\item {Utilização:trasm.}
\end{itemize}
\begin{itemize}
\item {Proveniência:(De \textunderscore inçar\textunderscore )}
\end{itemize}
Vegetaes, que na ceifa ou em outro córte, se deixam illesos, para frutificarem e reproduzirem-se.
Restos, resquícios: \textunderscore por mais que se destruam os insectos daninhos, sempre fica inço\textunderscore .
\section{Incoação}
\begin{itemize}
\item {Grp. gram.:f.}
\end{itemize}
\begin{itemize}
\item {Proveniência:(Lat. \textunderscore inchoatio\textunderscore )}
\end{itemize}
Começo.
\section{Incoar}
\begin{itemize}
\item {Grp. gram.:v. t.}
\end{itemize}
\begin{itemize}
\item {Proveniência:(Lat. \textunderscore inchoare\textunderscore )}
\end{itemize}
O mesmo que \textunderscore começar\textunderscore .
\section{Incoativo}
\begin{itemize}
\item {Grp. gram.:adj.}
\end{itemize}
\begin{itemize}
\item {Utilização:Gram.}
\end{itemize}
\begin{itemize}
\item {Proveniência:(Lat. \textunderscore inchoativus\textunderscore )}
\end{itemize}
Que começa.
Que dá comêço a.
Que exprime aumento progressivo de acção, (falando-se de verbos).
\section{Incoadunável}
\begin{itemize}
\item {Grp. gram.:adj.}
\end{itemize}
\begin{itemize}
\item {Proveniência:(De \textunderscore in...\textunderscore  + \textunderscore coadunável\textunderscore )}
\end{itemize}
Que se não póde coadunar; inconciliável.
\section{Incoagulável}
\begin{itemize}
\item {Grp. gram.:adj.}
\end{itemize}
\begin{itemize}
\item {Proveniência:(De \textunderscore in...\textunderscore  + \textunderscore coagulável\textunderscore )}
\end{itemize}
Que não é coagulável.
\section{Incobrável}
\begin{itemize}
\item {Grp. gram.:adj.}
\end{itemize}
\begin{itemize}
\item {Proveniência:(De \textunderscore in...\textunderscore  + \textunderscore cobrável\textunderscore )}
\end{itemize}
Que não é cobrável; que se não póde cobrar.
\section{Incocção}
\begin{itemize}
\item {Grp. gram.:f.}
\end{itemize}
O mesmo que \textunderscore decocção\textunderscore .
\section{Incoctível}
\begin{itemize}
\item {Grp. gram.:adj.}
\end{itemize}
Que se não póde cozer; que se não digere.
(Cp. lat. \textunderscore incoctus\textunderscore )
\section{Incoercibilidade}
\begin{itemize}
\item {fónica:co-er}
\end{itemize}
\begin{itemize}
\item {Grp. gram.:f.}
\end{itemize}
Qualidade de incoercível.
\section{Incoercível}
\begin{itemize}
\item {fónica:co-er}
\end{itemize}
\begin{itemize}
\item {Grp. gram.:adj.}
\end{itemize}
\begin{itemize}
\item {Proveniência:(De \textunderscore in...\textunderscore  + \textunderscore coercível\textunderscore )}
\end{itemize}
Não coercível.
\section{Incoerência}
\begin{itemize}
\item {fónica:co-e}
\end{itemize}
\begin{itemize}
\item {Grp. gram.:f.}
\end{itemize}
\begin{itemize}
\item {Proveniência:(De \textunderscore in...\textunderscore  + \textunderscore coerência\textunderscore )}
\end{itemize}
Qualidade de incoerente; falta de coerência.
\section{Incoerente}
\begin{itemize}
\item {fónica:co-e}
\end{itemize}
\begin{itemize}
\item {Grp. gram.:adj.}
\end{itemize}
\begin{itemize}
\item {Proveniência:(De \textunderscore in...\textunderscore  + \textunderscore coerente\textunderscore )}
\end{itemize}
Que não é coerente; que não tem ligação ou harmonia.
Desconexo; em que não há ordem ou sistema: \textunderscore palavras incoerentes\textunderscore .
Ilógico; disparatado.
\section{Incoerentemente}
\begin{itemize}
\item {Grp. gram.:adv.}
\end{itemize}
De modo incoerente; sem coerência.
\section{Incoesão}
\begin{itemize}
\item {fónica:co-e}
\end{itemize}
\begin{itemize}
\item {Grp. gram.:f.}
\end{itemize}
\begin{itemize}
\item {Proveniência:(De \textunderscore in...\textunderscore  + \textunderscore coesão\textunderscore )}
\end{itemize}
Falta de coesão.
\section{Incogitado}
\begin{itemize}
\item {Grp. gram.:adj.}
\end{itemize}
\begin{itemize}
\item {Proveniência:(De \textunderscore in...\textunderscore  + \textunderscore cogitado\textunderscore )}
\end{itemize}
Que não é cogitado; impensado; não previsto.
\section{Incogitável}
\begin{itemize}
\item {Grp. gram.:adj.}
\end{itemize}
\begin{itemize}
\item {Proveniência:(De \textunderscore in...\textunderscore  + \textunderscore cogitável\textunderscore )}
\end{itemize}
Que não é cogitável; incalculável.
\section{Incógnita}
\begin{itemize}
\item {Grp. gram.:f.}
\end{itemize}
\begin{itemize}
\item {Proveniência:(De \textunderscore incógnito\textunderscore )}
\end{itemize}
Quantidade desconhecida, que é preciso determinar, para a solução de um problema.
Aquillo que é desconhecido e que se procura saber.
\section{Incognitamente}
\begin{itemize}
\item {Grp. gram.:adv.}
\end{itemize}
De modo incógnito; ás occultas.
Dissimuladamente.
\section{Incógnito}
\begin{itemize}
\item {Grp. gram.:adj.}
\end{itemize}
\begin{itemize}
\item {Grp. gram.:M.}
\end{itemize}
\begin{itemize}
\item {Proveniência:(Lat. \textunderscore incognitus\textunderscore )}
\end{itemize}
Que não é conhecido: \textunderscore filho de pais incógnitos\textunderscore .
Aquelle ou aquillo que é desconhecido.
\section{Incognoscível}
\begin{itemize}
\item {Grp. gram.:adj.}
\end{itemize}
\begin{itemize}
\item {Grp. gram.:M.}
\end{itemize}
\begin{itemize}
\item {Proveniência:(De \textunderscore in...\textunderscore  + \textunderscore cognoscível\textunderscore )}
\end{itemize}
Que se não póde conhecer.
Aquillo ou aquelle que se não póde conhecer.
\section{Incoherência}
\begin{itemize}
\item {Grp. gram.:f.}
\end{itemize}
\begin{itemize}
\item {Proveniência:(De \textunderscore in...\textunderscore  + \textunderscore coherência\textunderscore )}
\end{itemize}
Qualidade de incoherente; falta de coherência.
\section{Incoherente}
\begin{itemize}
\item {Grp. gram.:adj.}
\end{itemize}
\begin{itemize}
\item {Proveniência:(De \textunderscore in...\textunderscore  + \textunderscore coherente\textunderscore )}
\end{itemize}
Que não é coherente; que não tem ligação ou harmonia.
Desconnexo; em que não há ordem ou systema: \textunderscore palavras incoherentes\textunderscore .
Illógico; disparatado.
\section{Incoherentemente}
\begin{itemize}
\item {Grp. gram.:adv.}
\end{itemize}
De modo incoherente; sem coherência.
\section{Incohesão}
\begin{itemize}
\item {Grp. gram.:f.}
\end{itemize}
\begin{itemize}
\item {Proveniência:(De \textunderscore in...\textunderscore  + \textunderscore cohesão\textunderscore )}
\end{itemize}
Falta de cohesão.
\section{Incoirapato}
\begin{itemize}
\item {Grp. gram.:adj.}
\end{itemize}
\begin{itemize}
\item {Utilização:Prov.}
\end{itemize}
\begin{itemize}
\item {Utilização:beir.}
\end{itemize}
Que está despido, nu, ou que anda em coiro.
(Cp. \textunderscore coiro\textunderscore )
\section{Íncola}
\begin{itemize}
\item {Grp. gram.:m.}
\end{itemize}
\begin{itemize}
\item {Utilização:Poét.}
\end{itemize}
\begin{itemize}
\item {Proveniência:(Lat. \textunderscore incola\textunderscore )}
\end{itemize}
Morador; habitante.
\section{Incolor}
\begin{itemize}
\item {Grp. gram.:adj.}
\end{itemize}
\begin{itemize}
\item {Utilização:Fig.}
\end{itemize}
\begin{itemize}
\item {Proveniência:(Lat. \textunderscore incolor\textunderscore )}
\end{itemize}
Que não tem côr; descolorido.
Que não tem partido político: \textunderscore um jornal incolor\textunderscore .
Dúbio, indeciso.
\section{Incolorar}
\begin{itemize}
\item {Grp. gram.:v. t.}
\end{itemize}
O mesmo que \textunderscore descorar\textunderscore . Cf. Castilho, \textunderscore Geórgicas\textunderscore .
\section{Incoloro}
\begin{itemize}
\item {Grp. gram.:adj.}
\end{itemize}
O mesmo que \textunderscore incolor\textunderscore .
\section{Incólume}
\begin{itemize}
\item {Grp. gram.:adj.}
\end{itemize}
\begin{itemize}
\item {Proveniência:(Lat. \textunderscore incolumis\textunderscore )}
\end{itemize}
Intacto; illeso.
Bem conservado.
São e salvo.
Que não soffreu perigo ou damno: \textunderscore saiu incólume das chammas\textunderscore .
\section{Incolumidade}
\begin{itemize}
\item {Grp. gram.:f.}
\end{itemize}
\begin{itemize}
\item {Proveniência:(Lat. \textunderscore incolumitas\textunderscore )}
\end{itemize}
Qualidade ou estado de incólume.
\section{Incombatível}
\begin{itemize}
\item {Grp. gram.:adj.}
\end{itemize}
\begin{itemize}
\item {Proveniência:(De \textunderscore in...\textunderscore  + \textunderscore combatível\textunderscore )}
\end{itemize}
Que se não póde combater; inatacável.
\section{Incombinável}
\begin{itemize}
\item {Grp. gram.:adj.}
\end{itemize}
\begin{itemize}
\item {Proveniência:(De \textunderscore in...\textunderscore  + \textunderscore combinável\textunderscore )}
\end{itemize}
Que se não póde combinar.
\section{Incombustibilidade}
\begin{itemize}
\item {Grp. gram.:f.}
\end{itemize}
Qualidade daquillo que é incombustível.
\section{Incombustível}
\begin{itemize}
\item {Grp. gram.:adj.}
\end{itemize}
\begin{itemize}
\item {Proveniência:(De \textunderscore in...\textunderscore  + \textunderscore combustível\textunderscore )}
\end{itemize}
Que não é combustível, que não póde arder; que se não queima.
\section{Incombusto}
\begin{itemize}
\item {Grp. gram.:adj.}
\end{itemize}
\begin{itemize}
\item {Proveniência:(De \textunderscore in...\textunderscore  + \textunderscore combusto\textunderscore )}
\end{itemize}
Que não foi queimado.
\section{Incomensurabilidade}
\begin{itemize}
\item {Grp. gram.:f.}
\end{itemize}
Qualidade de incomensurável.
\section{Incomensurável}
\begin{itemize}
\item {Grp. gram.:adj.}
\end{itemize}
\begin{itemize}
\item {Proveniência:(De \textunderscore in...\textunderscore  + \textunderscore comensurável\textunderscore )}
\end{itemize}
Que não é comensurável; imenso; desmedido.
\section{Incomensuravelmente}
\begin{itemize}
\item {Grp. gram.:adv.}
\end{itemize}
De modo incomensurável.
\section{Incommensurabilidade}
\begin{itemize}
\item {Grp. gram.:f.}
\end{itemize}
Qualidade de incommensurável.
\section{Incommensurável}
\begin{itemize}
\item {Grp. gram.:adj.}
\end{itemize}
\begin{itemize}
\item {Proveniência:(De \textunderscore in...\textunderscore  + \textunderscore commensurável\textunderscore )}
\end{itemize}
Que não é commensurável; immenso; desmedido.
\section{Incommensuravelmente}
\begin{itemize}
\item {Grp. gram.:adv.}
\end{itemize}
De modo incommensurável.
\section{Incommodador}
\begin{itemize}
\item {Grp. gram.:adj.}
\end{itemize}
\begin{itemize}
\item {Grp. gram.:M.}
\end{itemize}
Que incommoda.
Aquelle que incommoda.
\section{Incommodamente}
\begin{itemize}
\item {Grp. gram.:adv.}
\end{itemize}
De modo incômmodo: \textunderscore viajar incommodamente\textunderscore .
\section{Incommodante}
\begin{itemize}
\item {Grp. gram.:adj.}
\end{itemize}
\begin{itemize}
\item {Proveniência:(Lat. \textunderscore incommodans\textunderscore )}
\end{itemize}
Que incommoda.
\section{Incommodar}
\begin{itemize}
\item {Grp. gram.:v. t.}
\end{itemize}
\begin{itemize}
\item {Proveniência:(Lat. \textunderscore incommodare\textunderscore )}
\end{itemize}
Dar incômmodo a.
Importunar.
Desgostar; molestar.
\section{Incommodativo}
\begin{itemize}
\item {Grp. gram.:adj.}
\end{itemize}
Que causa incômmodo, incommodante: \textunderscore êste rapaz é muito incommodativo\textunderscore . Cf. Camillo, \textunderscore Myst. de Lisb.\textunderscore , 113 e 121; \textunderscore Sc. da Foz\textunderscore , 49; Th. Ribeiro, \textunderscore Jornadas\textunderscore , I, 319 e 358.
\section{Incommodidade}
\begin{itemize}
\item {Grp. gram.:f.}
\end{itemize}
\begin{itemize}
\item {Proveniência:(Lat. \textunderscore incommoditas\textunderscore )}
\end{itemize}
Qualidade daquelle ou daquillo que é incômmodo.
Falta de commodidade: \textunderscore a incommodidade de um aposento\textunderscore .
\section{Incômmodo}
\begin{itemize}
\item {Grp. gram.:adj.}
\end{itemize}
\begin{itemize}
\item {Proveniência:(Lat. \textunderscore incommodus\textunderscore )}
\end{itemize}
Que não é cômmodo; que incommoda; que enfada; que embaraça; que importuna.
\section{Incômmodo}
\begin{itemize}
\item {Grp. gram.:m.}
\end{itemize}
\begin{itemize}
\item {Utilização:Fam.}
\end{itemize}
\begin{itemize}
\item {Proveniência:(Lat. \textunderscore incommodum\textunderscore )}
\end{itemize}
O mesmo que \textunderscore incommodidade\textunderscore .
Doença passageira.
Catamênio.
\section{Incommunicabilidade}
\begin{itemize}
\item {Grp. gram.:f.}
\end{itemize}
Qualidade ou estado de incommunicável.
\section{Incommunicação}
\begin{itemize}
\item {Grp. gram.:f.}
\end{itemize}
\begin{itemize}
\item {Proveniência:(De \textunderscore incommunicar\textunderscore )}
\end{itemize}
Falta de communicação.
\section{Incommunicar}
\begin{itemize}
\item {Grp. gram.:v. t.}
\end{itemize}
Interromper a communicação de. Cf. Herculano, \textunderscore Hist. de Port.\textunderscore , III, 317.
\section{Incommunicável}
\begin{itemize}
\item {Grp. gram.:adj.}
\end{itemize}
\begin{itemize}
\item {Utilização:Fig.}
\end{itemize}
\begin{itemize}
\item {Proveniência:(Lat. \textunderscore incommunicabilis\textunderscore )}
\end{itemize}
Não communicável.
Que se não deve ou que se não póde communicar.
Que não póde falar ou communicar-se com outrem: \textunderscore o preso está incommunicável\textunderscore .
Que não é ligado ou não apresenta communicação.
Que não faz parte de determinada communhão ou se conserva insulado.
Intratável, insociável; misanthropo.
\section{Incommunicavelmente}
\begin{itemize}
\item {Grp. gram.:adv.}
\end{itemize}
De modo incommunicável.
\section{Incommutabilidade}
\begin{itemize}
\item {Grp. gram.:f.}
\end{itemize}
Qualidade de incommutável.
\section{Incommutável}
\begin{itemize}
\item {Grp. gram.:adj.}
\end{itemize}
\begin{itemize}
\item {Proveniência:(Lat. \textunderscore incommutabilis\textunderscore )}
\end{itemize}
Que se não póde commutar.
\section{Incomodador}
\begin{itemize}
\item {Grp. gram.:adj.}
\end{itemize}
\begin{itemize}
\item {Grp. gram.:M.}
\end{itemize}
Que incomoda.
Aquele que incomoda.
\section{Incomodamente}
\begin{itemize}
\item {Grp. gram.:adv.}
\end{itemize}
De modo incômodo: \textunderscore viajar incomodamente\textunderscore .
\section{Incomodante}
\begin{itemize}
\item {Grp. gram.:adj.}
\end{itemize}
\begin{itemize}
\item {Proveniência:(Lat. \textunderscore incommodans\textunderscore )}
\end{itemize}
Que incomoda.
\section{Incomodar}
\begin{itemize}
\item {Grp. gram.:v. t.}
\end{itemize}
\begin{itemize}
\item {Proveniência:(Lat. \textunderscore incommodare\textunderscore )}
\end{itemize}
Dar incômodo a.
Importunar.
Desgostar; molestar.
\section{Incomodativo}
\begin{itemize}
\item {Grp. gram.:adj.}
\end{itemize}
Que causa incômodo, incomodante: \textunderscore êste rapaz é muito incomodativo\textunderscore . Cf. Camillo, \textunderscore Myst. de Lisb.\textunderscore , 113 e 121; \textunderscore Sc. da Foz\textunderscore , 49; Th. Ribeiro, \textunderscore Jornadas\textunderscore , I, 319 e 358.
\section{Incomodidade}
\begin{itemize}
\item {Grp. gram.:f.}
\end{itemize}
\begin{itemize}
\item {Proveniência:(Lat. \textunderscore incommoditas\textunderscore )}
\end{itemize}
Qualidade daquele ou daquilo que é incômodo.
Falta de comodidade: \textunderscore a incomodidade de um aposento\textunderscore .
\section{Incômodo}
\begin{itemize}
\item {Grp. gram.:adj.}
\end{itemize}
\begin{itemize}
\item {Proveniência:(Lat. \textunderscore incommodus\textunderscore )}
\end{itemize}
Que não é cômodo; que incomoda; que enfada; que embaraça; que importuna.
\section{Incômodo}
\begin{itemize}
\item {Grp. gram.:m.}
\end{itemize}
\begin{itemize}
\item {Utilização:Fam.}
\end{itemize}
\begin{itemize}
\item {Proveniência:(Lat. \textunderscore incommodum\textunderscore )}
\end{itemize}
O mesmo que \textunderscore incomodidade\textunderscore .
Doença passageira.
Catamênio.
\section{Incomparabilidade}
\begin{itemize}
\item {Grp. gram.:f.}
\end{itemize}
Qualidade daquelle ou daquillo que é incomparável.
\section{Incomparável}
\begin{itemize}
\item {Grp. gram.:adj.}
\end{itemize}
\begin{itemize}
\item {Proveniência:(Lat. \textunderscore incomparabilis\textunderscore )}
\end{itemize}
Que não é comparável.
Extraordinário; único: \textunderscore belleza incomparável\textunderscore .
\section{Incomparavelmente}
\begin{itemize}
\item {Grp. gram.:adv.}
\end{itemize}
De modo incomparável.
Sem comparação.
\section{Incompassível}
\begin{itemize}
\item {Grp. gram.:adj.}
\end{itemize}
\begin{itemize}
\item {Proveniência:(De \textunderscore in...\textunderscore  + \textunderscore compassível\textunderscore )}
\end{itemize}
Que não sente compaixão; implacável; inexorável.
\section{Incompassivo}
\begin{itemize}
\item {Grp. gram.:adj.}
\end{itemize}
\begin{itemize}
\item {Proveniência:(De \textunderscore in...\textunderscore  + \textunderscore compassivo\textunderscore )}
\end{itemize}
Que não é compassivo; incompassível.
\section{Incompatibilidade}
\begin{itemize}
\item {Grp. gram.:f.}
\end{itemize}
Qualidade daquelle ou daquillo que é incompatível.
\section{Incompatibilizar}
\begin{itemize}
\item {Grp. gram.:v. t.}
\end{itemize}
Tornar incompatível.
\section{Incompatível}
\begin{itemize}
\item {Grp. gram.:adj.}
\end{itemize}
\begin{itemize}
\item {Proveniência:(De \textunderscore in...\textunderscore  + \textunderscore compatível\textunderscore )}
\end{itemize}
Que não é compatível.
Que não póde existir juntamente com outro ou outrem: \textunderscore dois gênios incompatíveis\textunderscore .
Que não póde harmonizar-se.
Diz-se dos cargos ou funcções que não podem exercer-se cumulativamente.
Inconciliável.
Que exclue outra coisa.
\section{Incompativelmente}
\begin{itemize}
\item {Grp. gram.:adv.}
\end{itemize}
De modo incompatível.
\section{Incompensado}
\begin{itemize}
\item {Grp. gram.:adj.}
\end{itemize}
\begin{itemize}
\item {Proveniência:(De \textunderscore in...\textunderscore  + \textunderscore compensado\textunderscore )}
\end{itemize}
Que não é compensado; para que não há compensação.
\section{Incompensável}
\begin{itemize}
\item {Grp. gram.:adj.}
\end{itemize}
\begin{itemize}
\item {Proveniência:(De \textunderscore in...\textunderscore  + \textunderscore compensável\textunderscore )}
\end{itemize}
Que se não póde compensar; impagável.
\section{Incompetência}
\begin{itemize}
\item {Grp. gram.:f.}
\end{itemize}
\begin{itemize}
\item {Proveniência:(De \textunderscore in...\textunderscore  + \textunderscore competência\textunderscore )}
\end{itemize}
Qualidade de incompetente; falta de competência.
\section{Incompetente}
\begin{itemize}
\item {Grp. gram.:adj.}
\end{itemize}
\begin{itemize}
\item {Proveniência:(Lat. \textunderscore incompetens\textunderscore )}
\end{itemize}
Que não é competente.
Que não tem as condições exigidas para certos fins.
\section{Incompetentemente}
\begin{itemize}
\item {Grp. gram.:adv.}
\end{itemize}
De modo incompetente.
\section{Incomplacência}
\begin{itemize}
\item {Grp. gram.:f.}
\end{itemize}
\begin{itemize}
\item {Proveniência:(De \textunderscore in...\textunderscore  + \textunderscore complacência\textunderscore )}
\end{itemize}
Falta de complacência.
\section{Incomplacente}
\begin{itemize}
\item {Grp. gram.:adj.}
\end{itemize}
\begin{itemize}
\item {Proveniência:(De \textunderscore in...\textunderscore  + \textunderscore complacente\textunderscore )}
\end{itemize}
Que não é complacente.
Severo.
\section{Incompletamente}
\begin{itemize}
\item {Grp. gram.:adv.}
\end{itemize}
De modo incompleto; imperfeitamente.
\section{Incompleto}
\begin{itemize}
\item {Grp. gram.:adj.}
\end{itemize}
\begin{itemize}
\item {Proveniência:(Lat. \textunderscore incompletus\textunderscore )}
\end{itemize}
Que não é completo.
Truncado; em que falta alguma coisa: \textunderscore uma obra incompleta\textunderscore .
Imperfeito.
\section{Incomplexidade}
\begin{itemize}
\item {fónica:csi}
\end{itemize}
\begin{itemize}
\item {Grp. gram.:f.}
\end{itemize}
Qualidade do que é incomplexo.
\section{Incomplexivo}
\begin{itemize}
\item {fónica:csi}
\end{itemize}
\begin{itemize}
\item {Grp. gram.:adj.}
\end{itemize}
\begin{itemize}
\item {Proveniência:(Lat. \textunderscore incomplexivus\textunderscore )}
\end{itemize}
O mesmo que \textunderscore incomplexo\textunderscore .
\section{Incomplexo}
\begin{itemize}
\item {fónica:cso}
\end{itemize}
\begin{itemize}
\item {Grp. gram.:adj.}
\end{itemize}
\begin{itemize}
\item {Proveniência:(Lat. \textunderscore incomplexus\textunderscore )}
\end{itemize}
Que não é complexo; que abrange só uma coisa.
Que se observa num lance.
Simples.
Que envolve só uma ideia ou é expresso por uma só palavra.
Composto de proposições incomplexas, (falando-se do syllogismo).
\section{Incomponível}
\begin{itemize}
\item {Grp. gram.:adj.}
\end{itemize}
\begin{itemize}
\item {Proveniência:(De \textunderscore in...\textunderscore  + \textunderscore componível\textunderscore )}
\end{itemize}
Que se não póde compor ou conciliar com outro.
\section{Incomportável}
\begin{itemize}
\item {Grp. gram.:adj.}
\end{itemize}
\begin{itemize}
\item {Proveniência:(De \textunderscore in...\textunderscore  + \textunderscore comportável\textunderscore )}
\end{itemize}
Que não é comportável.
Que se não póde soffrer; intolerável; insupportável: \textunderscore trabalhos incomportáveis\textunderscore .
\section{Incomportavelmente}
\begin{itemize}
\item {Grp. gram.:adv.}
\end{itemize}
De modo incomportável.
\section{Incompossível}
\begin{itemize}
\item {Grp. gram.:adj.}
\end{itemize}
\begin{itemize}
\item {Proveniência:(De \textunderscore in...\textunderscore  + \textunderscore compossível\textunderscore )}
\end{itemize}
Incompatível; inconciliável.
\section{Incomposto}
\begin{itemize}
\item {Grp. gram.:adj.}
\end{itemize}
\begin{itemize}
\item {Utilização:P. us.}
\end{itemize}
\begin{itemize}
\item {Proveniência:(De \textunderscore in...\textunderscore  + \textunderscore composto\textunderscore )}
\end{itemize}
Que não tem composição; que é simples, incomplexo.
\section{Incompreendido}
\begin{itemize}
\item {Grp. gram.:adj.}
\end{itemize}
\begin{itemize}
\item {Proveniência:(De \textunderscore in...\textunderscore  + \textunderscore compreendido\textunderscore )}
\end{itemize}
Que não é compreendido.
Despercebido.
Que não é bem avaliado ou julgado: \textunderscore talento incompreendido\textunderscore .
\section{Incompreensão}
\begin{itemize}
\item {Grp. gram.:f.}
\end{itemize}
\begin{itemize}
\item {Proveniência:(De \textunderscore in...\textunderscore  + \textunderscore compreensão\textunderscore )}
\end{itemize}
Falta de compreensão.
\section{Incompreensibilidade}
\begin{itemize}
\item {Grp. gram.:f.}
\end{itemize}
Qualidade de incompreensível.
\section{Incompreensível}
\begin{itemize}
\item {Grp. gram.:adj.}
\end{itemize}
\begin{itemize}
\item {Grp. gram.:M.}
\end{itemize}
\begin{itemize}
\item {Proveniência:(Lat. \textunderscore incomprehensibilis\textunderscore )}
\end{itemize}
Que não póde sêr compreendido; que é muito difícil de compreender: \textunderscore teorias incompreensíveis\textunderscore .
Aquele ou aquilo que não póde ser compreendido.
\section{Incompreensivelmente}
\begin{itemize}
\item {Grp. gram.:adv.}
\end{itemize}
De modo incompreensível.
\section{Incomprehendido}
\begin{itemize}
\item {Grp. gram.:adj.}
\end{itemize}
\begin{itemize}
\item {Proveniência:(De \textunderscore in...\textunderscore  + \textunderscore comprehendido\textunderscore )}
\end{itemize}
Que não é comprehendido.
Despercebido.
Que não é bem avaliado ou julgado: \textunderscore talento incomprehendido\textunderscore .
\section{Incomprehensão}
\begin{itemize}
\item {Grp. gram.:f.}
\end{itemize}
\begin{itemize}
\item {Proveniência:(De \textunderscore in...\textunderscore  + \textunderscore comprehensão\textunderscore )}
\end{itemize}
Falta de comprehensão.
\section{Incomprehensibilidade}
\begin{itemize}
\item {Grp. gram.:f.}
\end{itemize}
Qualidade de incomprehensível.
\section{Incomprehensível}
\begin{itemize}
\item {Grp. gram.:adj.}
\end{itemize}
\begin{itemize}
\item {Grp. gram.:M.}
\end{itemize}
\begin{itemize}
\item {Proveniência:(Lat. \textunderscore incomprehensibilis\textunderscore )}
\end{itemize}
Que não póde sêr comprehendido; que é muito diffícil de comprehender: \textunderscore theorias incomprehensíveis\textunderscore .
Aquelle ou aquillo que não póde ser comprehendido.
\section{Incomprehensivelmente}
\begin{itemize}
\item {Grp. gram.:adv.}
\end{itemize}
De modo incomprehensível.
\section{Incompressibilidade}
\begin{itemize}
\item {Grp. gram.:f.}
\end{itemize}
Qualidade de incompressível.
\section{Incompressível}
\begin{itemize}
\item {Grp. gram.:adj.}
\end{itemize}
\begin{itemize}
\item {Utilização:Fig.}
\end{itemize}
\begin{itemize}
\item {Proveniência:(De \textunderscore in...\textunderscore  + \textunderscore compressível\textunderscore )}
\end{itemize}
Que se não póde comprimir.
Que se não póde reprimir: \textunderscore tentação incompressível\textunderscore .
\section{Incomprimido}
\begin{itemize}
\item {Grp. gram.:adj.}
\end{itemize}
\begin{itemize}
\item {Proveniência:(De \textunderscore in...\textunderscore  + \textunderscore comprimido\textunderscore )}
\end{itemize}
Que não é comprimido.
\section{Incompto}
\begin{itemize}
\item {Grp. gram.:adj.}
\end{itemize}
\begin{itemize}
\item {Utilização:P. us.}
\end{itemize}
\begin{itemize}
\item {Proveniência:(Lat. \textunderscore incomptus\textunderscore )}
\end{itemize}
Grosseiro; em que não há arte; que não tem adôrno.
\section{Incomputável}
\begin{itemize}
\item {Grp. gram.:adj.}
\end{itemize}
\begin{itemize}
\item {Proveniência:(De \textunderscore in...\textunderscore  + \textunderscore computável\textunderscore )}
\end{itemize}
Que se não póde computar; innumerável.
\section{Incomunicabilidade}
\begin{itemize}
\item {Grp. gram.:f.}
\end{itemize}
Qualidade ou estado de incomunicável.
\section{Incomunicação}
\begin{itemize}
\item {Grp. gram.:f.}
\end{itemize}
\begin{itemize}
\item {Proveniência:(De \textunderscore incomunicar\textunderscore )}
\end{itemize}
Falta de comunicação.
\section{Incomunicar}
\begin{itemize}
\item {Grp. gram.:v. t.}
\end{itemize}
Interromper a comunicação de. Cf. Herculano, \textunderscore Hist. de Port.\textunderscore , III, 317.
\section{Incomunicável}
\begin{itemize}
\item {Grp. gram.:adj.}
\end{itemize}
\begin{itemize}
\item {Utilização:Fig.}
\end{itemize}
\begin{itemize}
\item {Proveniência:(Lat. \textunderscore incommunicabilis\textunderscore )}
\end{itemize}
Não comunicável.
Que se não deve ou que se não póde comunicar.
Que não póde falar ou comunicar-se com outrem: \textunderscore o preso está incomunicável\textunderscore .
Que não é ligado ou não apresenta comunicação.
Que não faz parte de determinada comunhão ou se conserva insulado.
Intratável, insociável; misantropo.
\section{Incomunicavelmente}
\begin{itemize}
\item {Grp. gram.:adv.}
\end{itemize}
De modo incomunicável.
\section{Incomutabilidade}
\begin{itemize}
\item {Grp. gram.:f.}
\end{itemize}
Qualidade de incomutável.
\section{Incomutável}
\begin{itemize}
\item {Grp. gram.:adj.}
\end{itemize}
\begin{itemize}
\item {Proveniência:(Lat. \textunderscore incommutabilis\textunderscore )}
\end{itemize}
Que se não póde comutar.
\section{Inconcebível}
\begin{itemize}
\item {Grp. gram.:adj.}
\end{itemize}
\begin{itemize}
\item {Proveniência:(De \textunderscore in...\textunderscore  + \textunderscore concebível\textunderscore )}
\end{itemize}
Que se não póde conceber.
Inacreditável; extraordinário.
\section{Inconcebivelmente}
\begin{itemize}
\item {Grp. gram.:adv.}
\end{itemize}
De modo inconcebível.
\section{Inconcepto}
\begin{itemize}
\item {Grp. gram.:adj.}
\end{itemize}
\begin{itemize}
\item {Utilização:Poét.}
\end{itemize}
\begin{itemize}
\item {Proveniência:(Do lat. \textunderscore in...\textunderscore  + \textunderscore conceptus\textunderscore )}
\end{itemize}
O mesmo que \textunderscore inconcebível\textunderscore . Cf. Filinto, XV, 25.
\section{Inconcessível}
\begin{itemize}
\item {Grp. gram.:adj.}
\end{itemize}
\begin{itemize}
\item {Proveniência:(Lat. \textunderscore inconcessibilis\textunderscore )}
\end{itemize}
Que se não póde ou que se não deve conceder.
\section{Inconcesso}
\begin{itemize}
\item {Grp. gram.:adj.}
\end{itemize}
\begin{itemize}
\item {Proveniência:(Lat. \textunderscore inconcessus\textunderscore )}
\end{itemize}
Que não é concedido; que é prohibido. Cf. \textunderscore Lusíadas\textunderscore , III, 141.
\section{Inconciliabilidade}
\begin{itemize}
\item {Grp. gram.:f.}
\end{itemize}
Qualidade daquelle ou daquillo que é inconciliável.
\section{Inconciliação}
\begin{itemize}
\item {Grp. gram.:f.}
\end{itemize}
\begin{itemize}
\item {Proveniência:(De \textunderscore in...\textunderscore  + \textunderscore conciliação\textunderscore )}
\end{itemize}
Qualidade ou estado do que é inconciliável.
\section{Inconciliado}
\begin{itemize}
\item {Grp. gram.:adj.}
\end{itemize}
\begin{itemize}
\item {Proveniência:(De \textunderscore in...\textunderscore  + \textunderscore conciliado\textunderscore )}
\end{itemize}
Que não é conciliado; desharmònico; divergente.
\section{Inconciliável}
\begin{itemize}
\item {Grp. gram.:adj.}
\end{itemize}
\begin{itemize}
\item {Proveniência:(De \textunderscore in...\textunderscore  + \textunderscore conciliável\textunderscore )}
\end{itemize}
Que se não póde conciliar ou harmonizar; incompatível.
\section{Inconciliavelmente}
\begin{itemize}
\item {Grp. gram.:adv.}
\end{itemize}
De modo inconciliável.
\section{Inconcludente}
\begin{itemize}
\item {Grp. gram.:adj.}
\end{itemize}
\begin{itemize}
\item {Proveniência:(De \textunderscore in...\textunderscore  + \textunderscore concludente\textunderscore )}
\end{itemize}
Que não é concludente; illógico.
\section{Inconcordável}
\begin{itemize}
\item {Grp. gram.:adj.}
\end{itemize}
\begin{itemize}
\item {Proveniência:(De \textunderscore in...\textunderscore  + \textunderscore concordar\textunderscore )}
\end{itemize}
O mesmo que \textunderscore inconciliável\textunderscore .
\section{Inconcussamente}
\begin{itemize}
\item {Grp. gram.:adv.}
\end{itemize}
De modo inconcusso; sem contestação; incontestavelmente; fóra de dúvida.
\section{Inconcusso}
\begin{itemize}
\item {Grp. gram.:adj.}
\end{itemize}
\begin{itemize}
\item {Utilização:Fig.}
\end{itemize}
\begin{itemize}
\item {Proveniência:(Lat. \textunderscore inconcussus\textunderscore )}
\end{itemize}
Inabalável; firme.
Incontestável: \textunderscore verdades inconcussas\textunderscore .
Austero.
\section{Incondicionado}
\begin{itemize}
\item {Grp. gram.:adj.}
\end{itemize}
\begin{itemize}
\item {Proveniência:(De \textunderscore in...\textunderscore  + \textunderscore condicionado\textunderscore )}
\end{itemize}
Que não está sujeito a condições ou restricções: \textunderscore ordens incondicionadas\textunderscore .
\section{Incondicional}
\begin{itemize}
\item {Grp. gram.:adj.}
\end{itemize}
\begin{itemize}
\item {Proveniência:(De \textunderscore in...\textunderscore  + \textunderscore condicional\textunderscore )}
\end{itemize}
Que não é condicional; que não depende de condições.
\section{Incondicionalidade}
\begin{itemize}
\item {Grp. gram.:f.}
\end{itemize}
Qualidade de incondicional.
\section{Incondicionalmente}
\begin{itemize}
\item {Grp. gram.:adv.}
\end{itemize}
De modo incondicional; sem condições.
\section{Incôndito}
\begin{itemize}
\item {Grp. gram.:adj.}
\end{itemize}
\begin{itemize}
\item {Proveniência:(Lat. \textunderscore inconditus\textunderscore )}
\end{itemize}
Não organizado.
Que não tem regra.
Desordenado; confuso.
\section{Inconexamente}
\begin{itemize}
\item {fónica:csa}
\end{itemize}
\begin{itemize}
\item {Grp. gram.:adv.}
\end{itemize}
De modo inconexo.
\section{Inconexão}
\begin{itemize}
\item {fónica:csão}
\end{itemize}
\begin{itemize}
\item {Grp. gram.:f.}
\end{itemize}
\begin{itemize}
\item {Proveniência:(De \textunderscore in...\textunderscore  + \textunderscore conexão\textunderscore )}
\end{itemize}
Falta de conexão.
\section{Inconexo}
\begin{itemize}
\item {fónica:cso}
\end{itemize}
\begin{itemize}
\item {Grp. gram.:adj.}
\end{itemize}
\begin{itemize}
\item {Proveniência:(Lat. \textunderscore inconnexus\textunderscore )}
\end{itemize}
Que não tem conexão; desligado; inarmónico.
\section{Inconfessado}
\begin{itemize}
\item {Grp. gram.:adj.}
\end{itemize}
\begin{itemize}
\item {Proveniência:(De \textunderscore in...\textunderscore  + \textunderscore confessado\textunderscore )}
\end{itemize}
Que se não confessou, que se occultou ou que se dissimulou: \textunderscore culpas inconfessadas\textunderscore .
\section{Inconfessável}
\begin{itemize}
\item {Grp. gram.:adj.}
\end{itemize}
\begin{itemize}
\item {Proveniência:(De \textunderscore in...\textunderscore  + \textunderscore confessável\textunderscore )}
\end{itemize}
Que se não póde ou que se não deve confessar.
\section{Inconfesso}
\begin{itemize}
\item {Grp. gram.:adj.}
\end{itemize}
\begin{itemize}
\item {Proveniência:(De \textunderscore in...\textunderscore  + \textunderscore confesso\textunderscore )}
\end{itemize}
Que não é confesso; que não confessou.
\section{Inconfidência}
\begin{itemize}
\item {Grp. gram.:f.}
\end{itemize}
\begin{itemize}
\item {Proveniência:(De \textunderscore in...\textunderscore  + \textunderscore confidência\textunderscore )}
\end{itemize}
Falta de lealdade; abuso de confiança; infidelidade: \textunderscore praticar inconfidências\textunderscore .
\section{Inconfidente}
\begin{itemize}
\item {Grp. gram.:adj.}
\end{itemize}
\begin{itemize}
\item {Proveniência:(De \textunderscore in...\textunderscore  + \textunderscore confidente\textunderscore )}
\end{itemize}
Infiel; que revela segredos de outrem.
\section{Inconfortável}
\begin{itemize}
\item {Grp. gram.:adj.}
\end{itemize}
\begin{itemize}
\item {Proveniência:(De \textunderscore in...\textunderscore  + \textunderscore confortável\textunderscore )}
\end{itemize}
Que não é confortável.
Desabrigado.
\section{Inconfundível}
\begin{itemize}
\item {Grp. gram.:adj.}
\end{itemize}
\begin{itemize}
\item {Proveniência:(De \textunderscore in...\textunderscore  + \textunderscore confundível\textunderscore )}
\end{itemize}
Que se não póde confundir; distinto.
\section{Incongelado}
\begin{itemize}
\item {Grp. gram.:adj.}
\end{itemize}
\begin{itemize}
\item {Proveniência:(De \textunderscore in...\textunderscore  + \textunderscore congelado\textunderscore )}
\end{itemize}
Que se não congelou.
\section{Incongelável}
\begin{itemize}
\item {Grp. gram.:adj.}
\end{itemize}
\begin{itemize}
\item {Proveniência:(Lat. \textunderscore incongelabilis\textunderscore )}
\end{itemize}
Que não é congelável.
\section{Incongruamente}
\begin{itemize}
\item {Grp. gram.:adv.}
\end{itemize}
De modo incôngruo; inconvenientemente.
\section{Incongruência}
\begin{itemize}
\item {Grp. gram.:f.}
\end{itemize}
\begin{itemize}
\item {Proveniência:(Lat. \textunderscore incongruentia\textunderscore )}
\end{itemize}
Qualidade de incongruente; falta de congruência.
\section{Incongruente}
\begin{itemize}
\item {Grp. gram.:adj.}
\end{itemize}
\begin{itemize}
\item {Proveniência:(Lat. \textunderscore incongruens\textunderscore )}
\end{itemize}
Impróprio; inconveniente; incompativel.
\section{Incongruidade}
\begin{itemize}
\item {fónica:gru-i}
\end{itemize}
\begin{itemize}
\item {Grp. gram.:f.}
\end{itemize}
\begin{itemize}
\item {Proveniência:(Lat. \textunderscore incongruitas\textunderscore )}
\end{itemize}
Qualidade de incôngruo.
\section{Incôngruo}
\begin{itemize}
\item {Grp. gram.:adj.}
\end{itemize}
\begin{itemize}
\item {Proveniência:(Lat. \textunderscore incongruus\textunderscore )}
\end{itemize}
O mesmo que \textunderscore incongruente\textunderscore .
\section{Inconhecível}
\begin{itemize}
\item {Grp. gram.:adj.}
\end{itemize}
\begin{itemize}
\item {Proveniência:(De \textunderscore in...\textunderscore  + \textunderscore conhecível\textunderscore )}
\end{itemize}
Que se não póde conhecer; desconhecível.
\section{Inconho}
\begin{itemize}
\item {Grp. gram.:adj.}
\end{itemize}
Diz-se do fruto, que está naturalmente unido a outro.
(Cp. \textunderscore cónho\textunderscore )
\section{Inconivente}
\begin{itemize}
\item {Grp. gram.:adj.}
\end{itemize}
\begin{itemize}
\item {Proveniência:(Lat. \textunderscore inconnivens\textunderscore )}
\end{itemize}
Que não é connivente.
\section{Inconjugável}
\begin{itemize}
\item {Grp. gram.:adj.}
\end{itemize}
\begin{itemize}
\item {Proveniência:(De \textunderscore in...\textunderscore  + \textunderscore conjugável\textunderscore )}
\end{itemize}
Que se não póde conjugar.
\section{Inconnexamente}
\begin{itemize}
\item {fónica:csa}
\end{itemize}
\begin{itemize}
\item {Grp. gram.:adv.}
\end{itemize}
De modo inconnexo.
\section{Inconnexão}
\begin{itemize}
\item {fónica:csão}
\end{itemize}
\begin{itemize}
\item {Grp. gram.:f.}
\end{itemize}
\begin{itemize}
\item {Proveniência:(De \textunderscore in...\textunderscore  + \textunderscore connexão\textunderscore )}
\end{itemize}
Falta de connexão.
\section{Inconnexo}
\begin{itemize}
\item {fónica:cso}
\end{itemize}
\begin{itemize}
\item {Grp. gram.:adj.}
\end{itemize}
\begin{itemize}
\item {Proveniência:(Lat. \textunderscore inconnexus\textunderscore )}
\end{itemize}
Que não tem connexão; desligado; inharmónico.
\section{Inconnivente}
\begin{itemize}
\item {Grp. gram.:adj.}
\end{itemize}
\begin{itemize}
\item {Proveniência:(Lat. \textunderscore inconnivens\textunderscore )}
\end{itemize}
Que não é connivente.
\section{Inconquistabilidade}
\begin{itemize}
\item {Grp. gram.:f.}
\end{itemize}
Qualidade do que é inconquistável.
\section{Inconquistado}
\begin{itemize}
\item {Grp. gram.:adj.}
\end{itemize}
\begin{itemize}
\item {Utilização:Fig.}
\end{itemize}
\begin{itemize}
\item {Proveniência:(De \textunderscore in...\textunderscore  + \textunderscore conquistado\textunderscore )}
\end{itemize}
Que não foi conquistado.
Insubmísso.
\section{Inconquistável}
\begin{itemize}
\item {Grp. gram.:adj.}
\end{itemize}
\begin{itemize}
\item {Proveniência:(De \textunderscore in...\textunderscore  + \textunderscore conquistável\textunderscore )}
\end{itemize}
Que não é conquistável.
\section{Inconsciamente}
\begin{itemize}
\item {Grp. gram.:adv.}
\end{itemize}
De modo incônscio, inconscientemente.
\section{Inconsciência}
\begin{itemize}
\item {Grp. gram.:f.}
\end{itemize}
\begin{itemize}
\item {Utilização:Fig.}
\end{itemize}
\begin{itemize}
\item {Proveniência:(De \textunderscore in...\textunderscore  + \textunderscore consciência\textunderscore )}
\end{itemize}
Qualidade de inconsciente.
Facto, opposto aos ditames da consciência.
Falta de generosidade; deshumanidade; barbaridade.
\section{Inconscienciosamente}
\begin{itemize}
\item {Grp. gram.:adv.}
\end{itemize}
De modo inconsciencioso; sem consciência.
\section{Inconsciencioso}
\begin{itemize}
\item {Grp. gram.:adj.}
\end{itemize}
\begin{itemize}
\item {Proveniência:(De \textunderscore in...\textunderscore  + \textunderscore consciencioso\textunderscore )}
\end{itemize}
Que não é consciencioso; que não tem consciência.
\section{Inconsciente}
\begin{itemize}
\item {Grp. gram.:adj.}
\end{itemize}
\begin{itemize}
\item {Grp. gram.:M.}
\end{itemize}
\begin{itemize}
\item {Proveniência:(De \textunderscore in...\textunderscore  + \textunderscore consciente\textunderscore )}
\end{itemize}
Que não é consciente.
Em que não há consciência.
Praticado sem consciência ou sem reconhecimento do alcance moral daquillo que se praticou: \textunderscore êrro inconsciente\textunderscore .
Que procede sem consciência ou sem conhecimento claro do que faz.
Aquelle que procede sem consciência do que faz: \textunderscore é um inconsciente\textunderscore .
\section{Inconscientemente}
\begin{itemize}
\item {Grp. gram.:adv.}
\end{itemize}
De modo inconsciente.
\section{Incônscio}
\begin{itemize}
\item {Grp. gram.:adj.}
\end{itemize}
O mesmo que \textunderscore inconsciente\textunderscore :«\textunderscore ungir um moribunno incônscio\textunderscore ». Camillo, \textunderscore Mulher Fatal\textunderscore , 211.
\section{Inconsequência}
\begin{itemize}
\item {fónica:cu-en}
\end{itemize}
\begin{itemize}
\item {Grp. gram.:f.}
\end{itemize}
\begin{itemize}
\item {Proveniência:(Lat. \textunderscore inconsequentia\textunderscore )}
\end{itemize}
Falta de consequência.
Incongruência.
Inconnexão; contradicção.
Illação, que se não contém nas premissas.
\section{Inconsequente}
\begin{itemize}
\item {fónica:cu-en}
\end{itemize}
\begin{itemize}
\item {Grp. gram.:adj.}
\end{itemize}
\begin{itemize}
\item {Proveniência:(Lat. \textunderscore inconsequens\textunderscore )}
\end{itemize}
Em que há inconsequência.
\section{Inconsequentemente}
\begin{itemize}
\item {Grp. gram.:adv.}
\end{itemize}
De modo inconsequente.
\section{Inconsideração}
\begin{itemize}
\item {Grp. gram.:f.}
\end{itemize}
\begin{itemize}
\item {Utilização:Fig.}
\end{itemize}
\begin{itemize}
\item {Proveniência:(Lat. \textunderscore inconsideratio\textunderscore )}
\end{itemize}
Falta de consideração.
Precipitação, leviandade.
\section{Inconsideradamente}
\begin{itemize}
\item {Grp. gram.:adv.}
\end{itemize}
De modo inconsiderado.
Precipitadamente; levianamente; sem reflexão.
\section{Inconsiderado}
\begin{itemize}
\item {Grp. gram.:adj.}
\end{itemize}
\begin{itemize}
\item {Proveniência:(Lat. \textunderscore inconsideratus\textunderscore )}
\end{itemize}
Que não considera.
Imprudente; temerário.
Impensado; irreflectido.
\section{Inconsiderância}
\begin{itemize}
\item {Grp. gram.:f.}
\end{itemize}
\begin{itemize}
\item {Proveniência:(Lat. \textunderscore inconsiderantia\textunderscore )}
\end{itemize}
O mesmo que \textunderscore inconsideração\textunderscore .
\section{Inconsistência}
\begin{itemize}
\item {Grp. gram.:f.}
\end{itemize}
Qualidade de inconsistente; falta de consistência.
\section{Inconsistente}
\begin{itemize}
\item {Grp. gram.:adj.}
\end{itemize}
\begin{itemize}
\item {Proveniência:(De \textunderscore in...\textunderscore  + \textunderscore consistente\textunderscore )}
\end{itemize}
Que não é consistente.
Incerto.
Inconstante; incongruente.
\section{Inconsolabilidade}
\begin{itemize}
\item {Grp. gram.:f.}
\end{itemize}
Qualidade de inconsolável.
\section{Inconsolabilíssimo}
\begin{itemize}
\item {Grp. gram.:adj.}
\end{itemize}
Muito inconsolável.
\section{Inconsolado}
\begin{itemize}
\item {Grp. gram.:adj.}
\end{itemize}
\begin{itemize}
\item {Proveniência:(De \textunderscore in...\textunderscore  + \textunderscore consolado\textunderscore )}
\end{itemize}
Que não tem consolação.
\section{Inconsolativo}
\begin{itemize}
\item {Grp. gram.:adj.}
\end{itemize}
Que não consola; desconsolativo. Cf. Camillo, \textunderscore Estrell. Propícias\textunderscore , 186 e 194.
\section{Inconsolável}
\begin{itemize}
\item {Grp. gram.:adj.}
\end{itemize}
\begin{itemize}
\item {Proveniência:(Lat. \textunderscore inconsolabilis\textunderscore )}
\end{itemize}
Que não é consolável.
Que não póde sêr consolado.
\section{Inconsolavelmente}
\begin{itemize}
\item {Grp. gram.:adv.}
\end{itemize}
De modo inconsolável.
\section{Inconsonância}
\begin{itemize}
\item {Grp. gram.:f.}
\end{itemize}
\begin{itemize}
\item {Proveniência:(De \textunderscore in...\textunderscore  + \textunderscore consonância\textunderscore )}
\end{itemize}
Falta de consonância.
\section{Inconsonante}
\begin{itemize}
\item {Grp. gram.:adj.}
\end{itemize}
\begin{itemize}
\item {Proveniência:(Lat. \textunderscore inconsonans\textunderscore )}
\end{itemize}
Que não tem consonância.
\section{Inconstância}
\begin{itemize}
\item {Grp. gram.:f.}
\end{itemize}
\begin{itemize}
\item {Proveniência:(Lat. \textunderscore inconstantia\textunderscore )}
\end{itemize}
Falta de constância; instabilidade; leviandade; infidelidade.
\section{Inconstante}
\begin{itemize}
\item {Grp. gram.:adj.}
\end{itemize}
\begin{itemize}
\item {Proveniência:(Lat. \textunderscore inconstans\textunderscore )}
\end{itemize}
Que não é constante; variável; mudável; incerto.
Versátil; inconsistente.
Infiel.
\section{Inconstantemente}
\begin{itemize}
\item {Grp. gram.:adv.}
\end{itemize}
De modo inconstante.
\section{Inconstelado}
\begin{itemize}
\item {Grp. gram.:adj.}
\end{itemize}
Que não tem estrêlas; escuro, (falando-se da noite):«\textunderscore atravès da noite inconstelada...\textunderscore »Duarte de Almeida, \textunderscore Estâncias ao Inf.\textunderscore 
\section{Inconstellado}
\begin{itemize}
\item {Grp. gram.:adj.}
\end{itemize}
Que não tem estrêllas; escuro, (falando-se da noite):«\textunderscore atravès da noite inconstellada...\textunderscore »Duarte de Almeida, \textunderscore Estâncias ao Inf.\textunderscore 
\section{Inconstitucional}
\begin{itemize}
\item {Grp. gram.:adj.}
\end{itemize}
\begin{itemize}
\item {Proveniência:(De \textunderscore in...\textunderscore  + \textunderscore constitucional\textunderscore )}
\end{itemize}
Que não é constitucional, ou que é opposto á Constituição do Estado.
\section{Inconstitucionalidade}
\begin{itemize}
\item {Grp. gram.:f.}
\end{itemize}
Qualidade de inconstitucional.
\section{Inconstitucionalmente}
\begin{itemize}
\item {Grp. gram.:adv.}
\end{itemize}
De modo inconstitucional.
\section{Inconstructo}
\begin{itemize}
\item {Grp. gram.:adj.}
\end{itemize}
Construido sobre. Cf. Filinto, XXII, 13.
\section{Inconsulto}
\begin{itemize}
\item {Grp. gram.:adj.}
\end{itemize}
\begin{itemize}
\item {Proveniência:(Lat. \textunderscore inconsultus\textunderscore )}
\end{itemize}
Que não foi consultado.
Irreflectido.
\section{Inconsumível}
\begin{itemize}
\item {Grp. gram.:adj.}
\end{itemize}
\begin{itemize}
\item {Proveniência:(De \textunderscore in...\textunderscore  + \textunderscore consumível\textunderscore )}
\end{itemize}
Que não é consumível; que se não consome.
\section{Inconsumptível}
\begin{itemize}
\item {Grp. gram.:adj.}
\end{itemize}
\begin{itemize}
\item {Proveniência:(De \textunderscore inconsumpto\textunderscore )}
\end{itemize}
Que se não póde consumir.
\section{Inconsumpto}
\begin{itemize}
\item {Grp. gram.:adj.}
\end{itemize}
\begin{itemize}
\item {Proveniência:(Lat. \textunderscore inconsumptus\textunderscore )}
\end{itemize}
Que não é consumido; que não foi destruído.
\section{Inconsútil}
\begin{itemize}
\item {Grp. gram.:adj.}
\end{itemize}
\begin{itemize}
\item {Utilização:Fig.}
\end{itemize}
\begin{itemize}
\item {Proveniência:(Lat. \textunderscore inconsutilis\textunderscore )}
\end{itemize}
Que não tem costuras: \textunderscore túnica inconsútil\textunderscore .
Inteiriço.
\section{Incontaminado}
\begin{itemize}
\item {Grp. gram.:adj.}
\end{itemize}
\begin{itemize}
\item {Proveniência:(Lat. \textunderscore incontaminatus\textunderscore )}
\end{itemize}
Que não é contaminado; indemne.
\section{Incontável}
\begin{itemize}
\item {Grp. gram.:adj.}
\end{itemize}
\begin{itemize}
\item {Proveniência:(De \textunderscore in...\textunderscore  + \textunderscore contável\textunderscore )}
\end{itemize}
Que se não póde contar.
Innumerável.
\section{Incontentável}
\begin{itemize}
\item {Grp. gram.:adj.}
\end{itemize}
\begin{itemize}
\item {Proveniência:(De \textunderscore in...\textunderscore  + \textunderscore contentável\textunderscore )}
\end{itemize}
Que se não póde contentar.
\section{Incontestabilidade}
\begin{itemize}
\item {Grp. gram.:f.}
\end{itemize}
Qualidade daquillo que é incontestável.
\section{Incontestado}
\begin{itemize}
\item {Grp. gram.:adj.}
\end{itemize}
\begin{itemize}
\item {Proveniência:(De \textunderscore in...\textunderscore  + \textunderscore contestado\textunderscore )}
\end{itemize}
Que não é contestado; inconcusso.
\section{Incontestável}
\begin{itemize}
\item {Grp. gram.:adj.}
\end{itemize}
\begin{itemize}
\item {Proveniência:(De \textunderscore in...\textunderscore  + \textunderscore contestável\textunderscore )}
\end{itemize}
Que se não póde contestar; inconcusso; indiscutível: \textunderscore affirmações incontestáveis\textunderscore .
\section{Incontestavelmente}
\begin{itemize}
\item {Grp. gram.:adv.}
\end{itemize}
De modo incontestável.
\section{Incontinência}
\begin{itemize}
\item {Grp. gram.:f.}
\end{itemize}
\begin{itemize}
\item {Proveniência:(Lat. \textunderscore incontinentia\textunderscore )}
\end{itemize}
Qualidade de incontinente; falta de continência.
Difficuldade em reter: \textunderscore incontinência de urinas\textunderscore .
\section{Incontinente}
\begin{itemize}
\item {Grp. gram.:adj.}
\end{itemize}
\begin{itemize}
\item {Grp. gram.:M.  e  f.}
\end{itemize}
\begin{itemize}
\item {Proveniência:(Lat. \textunderscore incontinens\textunderscore )}
\end{itemize}
Immoderado; que não tem continência; sensual.
Pessôa immoderada em sensualidade.
\section{Incontinentemente}
\begin{itemize}
\item {Grp. gram.:adv.}
\end{itemize}
De modo incontinente, com incontinência.
Immediatamente.
\section{Incontingência}
\begin{itemize}
\item {Grp. gram.:f.}
\end{itemize}
Qualidade de incontingente.
\section{Incontingente}
\begin{itemize}
\item {Grp. gram.:adj.}
\end{itemize}
\begin{itemize}
\item {Proveniência:(De \textunderscore in...\textunderscore  + \textunderscore contingente\textunderscore )}
\end{itemize}
Que não é contingente.
\section{Incontinuidade}
\begin{itemize}
\item {fónica:nu-i}
\end{itemize}
\begin{itemize}
\item {Grp. gram.:f.}
\end{itemize}
\begin{itemize}
\item {Proveniência:(De \textunderscore in...\textunderscore  + \textunderscore continuidade\textunderscore )}
\end{itemize}
Falta de continuidade.
Interrupção.
\section{Incontínuo}
\begin{itemize}
\item {Grp. gram.:adj.}
\end{itemize}
\begin{itemize}
\item {Proveniência:(De \textunderscore in...\textunderscore  + \textunderscore contínuo\textunderscore )}
\end{itemize}
Que não é contínuo.
\section{Incontrariável}
\begin{itemize}
\item {Grp. gram.:adj.}
\end{itemize}
Que se não póde contrariar. Cf. Arn. Gama, \textunderscore Motim\textunderscore , 411 e 437.
\section{Incontrastado}
\begin{itemize}
\item {Grp. gram.:adj.}
\end{itemize}
\begin{itemize}
\item {Proveniência:(De \textunderscore in...\textunderscore  + \textunderscore contrastar\textunderscore )}
\end{itemize}
Que não tem opposição; que não é contestado.
\section{Incontrastável}
\begin{itemize}
\item {Grp. gram.:adj.}
\end{itemize}
\begin{itemize}
\item {Proveniência:(De \textunderscore in...\textunderscore  + \textunderscore contrastável\textunderscore )}
\end{itemize}
Que não é contrastável; irrespondível; irrevogável.
\section{Incontrastavelmente}
\begin{itemize}
\item {Grp. gram.:adv.}
\end{itemize}
De modo incontrastável.
\section{Incontrito}
\begin{itemize}
\item {Grp. gram.:adj.}
\end{itemize}
\begin{itemize}
\item {Proveniência:(De \textunderscore in...\textunderscore  + \textunderscore contrito\textunderscore )}
\end{itemize}
Não contrito; impenitente. Cf. Camillo, \textunderscore Noites de Insómn.\textunderscore , IX, 16.
\section{Incontroverso}
\begin{itemize}
\item {Grp. gram.:adj.}
\end{itemize}
\begin{itemize}
\item {Proveniência:(Lat. \textunderscore incontroversus\textunderscore )}
\end{itemize}
O mesmo que \textunderscore incontestável\textunderscore ; certíssimo; inconcusso.
\section{Incontrovertido}
\begin{itemize}
\item {Grp. gram.:adj.}
\end{itemize}
\begin{itemize}
\item {Proveniência:(De \textunderscore in...\textunderscore  + \textunderscore controverter\textunderscore )}
\end{itemize}
O mesmo que \textunderscore incontroverso\textunderscore .
\section{Incontrovertível}
\begin{itemize}
\item {Grp. gram.:adj.}
\end{itemize}
\begin{itemize}
\item {Proveniência:(De \textunderscore in...\textunderscore  + \textunderscore controvertível\textunderscore )}
\end{itemize}
Que não é controvertível; que se não póde controverter.
\section{Inconveniência}
\begin{itemize}
\item {Grp. gram.:f.}
\end{itemize}
\begin{itemize}
\item {Proveniência:(Lat. \textunderscore inconvenientia\textunderscore )}
\end{itemize}
Qualidade ou estado do que é inconveniente.
Incapacidade.
Palavra ou facto inconveniente, grosseiro.
\section{Inconveniente}
\begin{itemize}
\item {Grp. gram.:adj.}
\end{itemize}
\begin{itemize}
\item {Grp. gram.:M.}
\end{itemize}
\begin{itemize}
\item {Proveniência:(Lat. \textunderscore inconveniens\textunderscore )}
\end{itemize}
Que não é conveniente; impróprio, inopportuno.
Contradictório.
Indecente.
Incômmodo; estôrvo; desvantagem: \textunderscore isso tem inconvenientes\textunderscore .
\section{Inconvenientemente}
\begin{itemize}
\item {Grp. gram.:adv.}
\end{itemize}
De modo inconveniente.
\section{Inconversável}
\begin{itemize}
\item {Grp. gram.:adj.}
\end{itemize}
\begin{itemize}
\item {Proveniência:(De \textunderscore in...\textunderscore  + \textunderscore conversável\textunderscore )}
\end{itemize}
Que, não é conversável; intratável.
\section{Inconversível}
\begin{itemize}
\item {Grp. gram.:adj.}
\end{itemize}
O mesmo que \textunderscore inconvertível\textunderscore .
\section{Inconvertível}
\begin{itemize}
\item {Grp. gram.:adj.}
\end{itemize}
\begin{itemize}
\item {Proveniência:(Lat. \textunderscore inconvertibilis\textunderscore )}
\end{itemize}
Que não é convertível; que se não póde converter.
\section{Inconvicto}
\begin{itemize}
\item {Grp. gram.:adj.}
\end{itemize}
\begin{itemize}
\item {Proveniência:(De \textunderscore in...\textunderscore  + \textunderscore convicto\textunderscore )}
\end{itemize}
Que não está convicto.
\section{Incoordenação}
\begin{itemize}
\item {Grp. gram.:f.}
\end{itemize}
\begin{itemize}
\item {Proveniência:(De \textunderscore in...\textunderscore  + \textunderscore coordenação\textunderscore )}
\end{itemize}
Falta de coordenação.
\section{Incórdio}
\begin{itemize}
\item {Grp. gram.:m.}
\end{itemize}
\begin{itemize}
\item {Utilização:Ant.}
\end{itemize}
Bubão na virilha.
\section{Incorporal}
\begin{itemize}
\item {Grp. gram.:adj.}
\end{itemize}
\begin{itemize}
\item {Proveniência:(Lat. \textunderscore incorporalis\textunderscore )}
\end{itemize}
O mesmo que \textunderscore incorpóreo\textunderscore .
\section{Incorporalidade}
\begin{itemize}
\item {Grp. gram.:f.}
\end{itemize}
Qualidade de incorporal.
\section{Incorporar}
\textunderscore v. t.\textunderscore  (e der.)
(V. \textunderscore encorporar\textunderscore , etc.)
\section{Incorporeidade}
\begin{itemize}
\item {Grp. gram.:f.}
\end{itemize}
Qualidade daquillo que é incorpóreo.
\section{Incorpóreo}
\begin{itemize}
\item {Grp. gram.:adj.}
\end{itemize}
\begin{itemize}
\item {Proveniência:(Lat. \textunderscore incorporeus\textunderscore )}
\end{itemize}
Que não é corpóreo; immaterial; impalpável.
\section{Incorrecção}
\begin{itemize}
\item {Grp. gram.:f.}
\end{itemize}
\begin{itemize}
\item {Proveniência:(De \textunderscore in...\textunderscore  + \textunderscore correcção\textunderscore )}
\end{itemize}
Falta de correcção; qualidade de incorrecto: \textunderscore incorrecção grammatical\textunderscore .
\section{Incorrectamente}
\begin{itemize}
\item {Grp. gram.:adv.}
\end{itemize}
De modo incorrecto; sem correcção.
\section{Incorrecto}
\begin{itemize}
\item {Grp. gram.:adj.}
\end{itemize}
\begin{itemize}
\item {Proveniência:(Lat. \textunderscore incorrectus\textunderscore )}
\end{itemize}
Que não é correcto; em que não há correcção: \textunderscore phrase incorrecta\textunderscore .
\section{Incorrer}
\begin{itemize}
\item {Grp. gram.:v. i.}
\end{itemize}
\begin{itemize}
\item {Grp. gram.:V. t.}
\end{itemize}
\begin{itemize}
\item {Utilização:Des.}
\end{itemize}
\begin{itemize}
\item {Proveniência:(Lat. \textunderscore incurrere\textunderscore )}
\end{itemize}
Caír.
Comprometer-se; ficar implicado.
Chamar sôbre si desagrado, castigo, etc.: \textunderscore incorrer na desapprovação pública\textunderscore .
Expor-se a, attrahir sôbre si:«\textunderscore ...incorressem eterno aborrecimento dos homens.\textunderscore »Filinto, \textunderscore D. Man.\textunderscore , I, 30.«\textunderscore ...peccado que incorremos em nosso primeiro pay.\textunderscore »\textunderscore Luz e Calor\textunderscore , 245.
\section{Incorrigibilidade}
\begin{itemize}
\item {Grp. gram.:f.}
\end{itemize}
Qualidade de incorrigível.
\section{Incorrigível}
\begin{itemize}
\item {Grp. gram.:adj.}
\end{itemize}
\begin{itemize}
\item {Proveniência:(De \textunderscore in...\textunderscore  + \textunderscore corrigível\textunderscore )}
\end{itemize}
Que não é corrigível; que não é susceptível de emenda.
Que reincide amiúde numa falta ou crime.
\section{Incorrigivelmente}
\begin{itemize}
\item {Grp. gram.:adv.}
\end{itemize}
De modo incorrigível.
\section{Incorrimento}
\begin{itemize}
\item {Grp. gram.:m.}
\end{itemize}
\begin{itemize}
\item {Utilização:Ant.}
\end{itemize}
Acto de incorrer.
Incursão; ataque.
\section{Incorrução}
\begin{itemize}
\item {Grp. gram.:f.}
\end{itemize}
\begin{itemize}
\item {Proveniência:(Lat. \textunderscore incorruptio\textunderscore )}
\end{itemize}
Qualidade ou estado de incorrutível.
\section{Incorrupção}
\begin{itemize}
\item {Grp. gram.:f.}
\end{itemize}
\begin{itemize}
\item {Proveniência:(Lat. \textunderscore incorruptio\textunderscore )}
\end{itemize}
Qualidade ou estado de incorruptível.
\section{Incorruptamente}
\begin{itemize}
\item {Grp. gram.:adv.}
\end{itemize}
De modo incorrupto.
\section{Incorruptibilidade}
\begin{itemize}
\item {Grp. gram.:f.}
\end{itemize}
\begin{itemize}
\item {Proveniência:(Lat. \textunderscore incorruptibilitas\textunderscore )}
\end{itemize}
Qualidade de quem é incorruptível.
Integridade; austeridade.
\section{Incorruptível}
\begin{itemize}
\item {Grp. gram.:adj.}
\end{itemize}
\begin{itemize}
\item {Proveniência:(Lat. \textunderscore incorruptibilis\textunderscore )}
\end{itemize}
Que não é corruptível; inalterável.
Que se não deteriora.
Que se não deixa subornar.
Integro, recto, justiceiro: \textunderscore juiz incorruptível\textunderscore .
\section{Incorruptivelmente}
\begin{itemize}
\item {Grp. gram.:adv.}
\end{itemize}
De modo incorruptível.
\section{Incorruptivo}
\begin{itemize}
\item {Grp. gram.:adj.}
\end{itemize}
\begin{itemize}
\item {Proveniência:(Do lat. \textunderscore incorruptivus\textunderscore )}
\end{itemize}
O mesmo que \textunderscore incorruptível\textunderscore .
\section{Incorrupto}
\begin{itemize}
\item {Grp. gram.:adj.}
\end{itemize}
\begin{itemize}
\item {Proveniência:(Lat. \textunderscore incorruptus\textunderscore )}
\end{itemize}
Que se não corrompeu, incorruptível.
\section{Incorrutamente}
\begin{itemize}
\item {Grp. gram.:adv.}
\end{itemize}
De modo incorruto.
\section{Incorrutibilidade}
\begin{itemize}
\item {Grp. gram.:f.}
\end{itemize}
\begin{itemize}
\item {Proveniência:(Lat. \textunderscore incorruptibilitas\textunderscore )}
\end{itemize}
Qualidade de quem é incorrutível.
Integridade; austeridade.
\section{Incrassante}
\begin{itemize}
\item {Grp. gram.:adj.}
\end{itemize}
Que incrassa.
\section{Incrassar}
\begin{itemize}
\item {Grp. gram.:v. t.}
\end{itemize}
\begin{itemize}
\item {Proveniência:(Lat. \textunderscore incrassare\textunderscore )}
\end{itemize}
Tornar crasso, gordo.
Engrossar.
\section{Incredibilidade}
\begin{itemize}
\item {Grp. gram.:f.}
\end{itemize}
\begin{itemize}
\item {Proveniência:(Lat. \textunderscore incredibilitas\textunderscore )}
\end{itemize}
Qualidade do que incrível.
\section{Incredível}
\begin{itemize}
\item {Grp. gram.:adj.}
\end{itemize}
\begin{itemize}
\item {Proveniência:(Lat. \textunderscore incredibilis\textunderscore )}
\end{itemize}
O mesmo que \textunderscore incrível\textunderscore . Cf. \textunderscore Ethiòpia Or.\textunderscore , I. 27.
\section{Incredulamente}
\begin{itemize}
\item {Grp. gram.:adv.}
\end{itemize}
De modo incrédulo.
\section{Incredulidade}
\begin{itemize}
\item {Grp. gram.:f.}
\end{itemize}
\begin{itemize}
\item {Proveniência:(Lat. \textunderscore incredulitas\textunderscore )}
\end{itemize}
Qualidade de quem é incrédulo; falta de credulidade; falta de fé.
Irreligião; atheísmo.
\section{Incrédulo}
\begin{itemize}
\item {Grp. gram.:adj.}
\end{itemize}
\begin{itemize}
\item {Grp. gram.:M.}
\end{itemize}
\begin{itemize}
\item {Proveniência:(Lat. \textunderscore incredulus\textunderscore )}
\end{itemize}
Que não crê, que não acredita; ímpio.
Homem, que não é crédulo.
Atheu.
\section{Increível}
\begin{itemize}
\item {Grp. gram.:adj.}
\end{itemize}
\begin{itemize}
\item {Utilização:Des.}
\end{itemize}
O mesmo que \textunderscore incrível\textunderscore . Cf. Vieira, I, 669 e 670.
\section{Incrementar}
\begin{itemize}
\item {Grp. gram.:v. t.}
\end{itemize}
\begin{itemize}
\item {Utilização:Neol.}
\end{itemize}
Dar incremento a.
\section{Incremento}
\begin{itemize}
\item {Grp. gram.:m.}
\end{itemize}
\begin{itemize}
\item {Proveniência:(Lat. \textunderscore incrementum\textunderscore )}
\end{itemize}
Desenvolvimento; aumento.
Acto de crescer, de aumentar: \textunderscore o incremento da população\textunderscore .
\section{Increnque}
\begin{itemize}
\item {Grp. gram.:adj.}
\end{itemize}
\begin{itemize}
\item {Utilização:Bras. de Goiás}
\end{itemize}
Ruim.
\section{Increpação}
\begin{itemize}
\item {Grp. gram.:f.}
\end{itemize}
\begin{itemize}
\item {Proveniência:(Lat. \textunderscore increpatio\textunderscore )}
\end{itemize}
Acto ou effeito de increpar.
\section{Increpador}
\begin{itemize}
\item {Grp. gram.:m.  e  adj.}
\end{itemize}
\begin{itemize}
\item {Proveniência:(Lat. \textunderscore increpator\textunderscore )}
\end{itemize}
O que increpa.
\section{Increpamento}
\begin{itemize}
\item {Grp. gram.:m.}
\end{itemize}
O mesmo que \textunderscore increpação\textunderscore .
\section{Increpante}
\begin{itemize}
\item {Grp. gram.:adj.}
\end{itemize}
\begin{itemize}
\item {Proveniência:(Lat. \textunderscore increpans\textunderscore )}
\end{itemize}
Que increpa.
\section{Increpar}
\begin{itemize}
\item {Grp. gram.:v. t.}
\end{itemize}
\begin{itemize}
\item {Proveniência:(Lat. \textunderscore increpare\textunderscore )}
\end{itemize}
Reprehender severamente; accusar; censurar.
\section{Incréu}
\begin{itemize}
\item {Grp. gram.:m.}
\end{itemize}
\begin{itemize}
\item {Utilização:Ant.}
\end{itemize}
\begin{itemize}
\item {Grp. gram.:Pl.}
\end{itemize}
O mesmo que \textunderscore incrédulo\textunderscore .
Os infiéis, os que não são christãos. Cf. G. Vicente.
\section{Incriado}
\begin{itemize}
\item {Grp. gram.:adj.}
\end{itemize}
\begin{itemize}
\item {Grp. gram.:M.}
\end{itemize}
\begin{itemize}
\item {Utilização:Restrict.}
\end{itemize}
\begin{itemize}
\item {Proveniência:(Lat. \textunderscore increatus\textunderscore )}
\end{itemize}
Que não foi criado.
Aquillo que não teve princípio.
Deus.
\section{Incrible}
\begin{itemize}
\item {Grp. gram.:adj.}
\end{itemize}
\begin{itemize}
\item {Utilização:pop.}
\end{itemize}
\begin{itemize}
\item {Utilização:Ant.}
\end{itemize}
O mesmo que \textunderscore incrível\textunderscore .
\section{Incriminação}
\begin{itemize}
\item {Grp. gram.:f.}
\end{itemize}
\begin{itemize}
\item {Proveniência:(Lat. \textunderscore incriminatio\textunderscore )}
\end{itemize}
Acto ou effeito de incriminar.
\section{Incriminar}
\begin{itemize}
\item {Grp. gram.:v. t.}
\end{itemize}
\begin{itemize}
\item {Proveniência:(Do lat. \textunderscore in\textunderscore  + \textunderscore criminare\textunderscore )}
\end{itemize}
Attribuir um crime a.
Accusar.
Considerar como crime.
\section{Incrinar}
\textunderscore v. t.\textunderscore  (e der.)
(Fórma antiga de \textunderscore inclinar\textunderscore , etc. Cf. \textunderscore Eufrosina\textunderscore , 106; G. Vicente, etc)
\section{Incristalizável}
\begin{itemize}
\item {Grp. gram.:adj.}
\end{itemize}
\begin{itemize}
\item {Proveniência:(De \textunderscore in...\textunderscore  + \textunderscore cristalizável\textunderscore )}
\end{itemize}
Que se não póde cristalizar.
\section{Incriticável}
\begin{itemize}
\item {Grp. gram.:adj.}
\end{itemize}
\begin{itemize}
\item {Proveniência:(De \textunderscore in...\textunderscore  + \textunderscore criticável\textunderscore )}
\end{itemize}
Que não é criticável; superior á crítica.
\section{Incrível}
\begin{itemize}
\item {Grp. gram.:adj.}
\end{itemize}
\begin{itemize}
\item {Proveniência:(De \textunderscore in...\textunderscore  + \textunderscore crivel\textunderscore )}
\end{itemize}
Que se não póde acreditar.
Extraordinário; inexplicável.
\section{Incrivelmente}
\begin{itemize}
\item {Grp. gram.:adv.}
\end{itemize}
De modo incrivel.
\section{Incruentamente}
\begin{itemize}
\item {Grp. gram.:adv.}
\end{itemize}
De modo incruento; sem derramamento de sangue.
\section{Incruento}
\begin{itemize}
\item {Grp. gram.:adj.}
\end{itemize}
\begin{itemize}
\item {Proveniência:(Lat. \textunderscore incruentus\textunderscore )}
\end{itemize}
Que não é ensanguentado; em que não houve derramamento de sangue; que não custou sangue: \textunderscore as batalhas incruentas da sciência\textunderscore .
\section{Incrustação}
\begin{itemize}
\item {Grp. gram.:f.}
\end{itemize}
\begin{itemize}
\item {Proveniência:(Lat. \textunderscore incrustatio\textunderscore )}
\end{itemize}
Acto ou effeito de incrustar.
\section{Incrustador}
\begin{itemize}
\item {Grp. gram.:adj.}
\end{itemize}
\begin{itemize}
\item {Grp. gram.:M.}
\end{itemize}
Que encrusta; encrustante.
Aquelle que incrusta.
Aquelle que faz incrustações ou embutidos.
\section{Incrustante}
\begin{itemize}
\item {Grp. gram.:adj.}
\end{itemize}
\begin{itemize}
\item {Proveniência:(Lat. \textunderscore incrustans\textunderscore )}
\end{itemize}
Que incrusta.
\section{Incrustar}
\begin{itemize}
\item {Grp. gram.:v. t.}
\end{itemize}
\begin{itemize}
\item {Proveniência:(Lat. \textunderscore incrustare\textunderscore )}
\end{itemize}
Cobrir de crosta.
Sobrepor uma camada a.
Embutir; tauxiar; gravar; inserir.
\section{Incrystallizável}
\begin{itemize}
\item {Grp. gram.:adj.}
\end{itemize}
\begin{itemize}
\item {Proveniência:(De \textunderscore in...\textunderscore  + \textunderscore crystallizável\textunderscore )}
\end{itemize}
Que se não póde crystallizar.
\section{Incuaia}
\begin{itemize}
\item {Grp. gram.:m.}
\end{itemize}
Festa annual dos Vátuas, que dura oito dias e consta especialmente de danças e cantos guerreiros.
\section{Incubação}
\begin{itemize}
\item {Grp. gram.:f.}
\end{itemize}
\begin{itemize}
\item {Utilização:Fig.}
\end{itemize}
\begin{itemize}
\item {Proveniência:(Lat. \textunderscore incubatio\textunderscore )}
\end{itemize}
Acto ou effeito de incubar.
Chôco.
Preparação.
Premeditação.
Espaço entre a acquisição de uma doença e a sua manifestação.
\section{Incubador}
\begin{itemize}
\item {Grp. gram.:adj.}
\end{itemize}
\begin{itemize}
\item {Grp. gram.:M.}
\end{itemize}
\begin{itemize}
\item {Proveniência:(Lat. \textunderscore incubator\textunderscore )}
\end{itemize}
Que serve para incubar ou chocar ovos.
Apparelho para incubar.
\section{Incubadora}
\begin{itemize}
\item {Grp. gram.:f.}
\end{itemize}
Apparelho, para incubação artificial de gallináceos; o mesmo que \textunderscore incubador\textunderscore .
\section{Incubar}
\begin{itemize}
\item {Grp. gram.:v. t.}
\end{itemize}
\begin{itemize}
\item {Utilização:Fig.}
\end{itemize}
\begin{itemize}
\item {Grp. gram.:V. i.}
\end{itemize}
\begin{itemize}
\item {Proveniência:(Lat. \textunderscore incubare\textunderscore )}
\end{itemize}
Chocar (ovos), natural ou artificialmente.
Predispor.
Planear.
Projectar; preparar.
Chocar ovos.
\section{Íncubo}
\begin{itemize}
\item {Grp. gram.:adj.}
\end{itemize}
\begin{itemize}
\item {Grp. gram.:M.}
\end{itemize}
\begin{itemize}
\item {Proveniência:(Lat. \textunderscore incubus\textunderscore )}
\end{itemize}
Que se deita sôbre alguma coisa.
Dizia-se principalmente dos demónios, a que se attribuíam pesadelos e más acções praticadas pelos homens.
Pesadelo.
Demónio, a que se attribuíam os pesadelos.
\section{Incuça}
\begin{itemize}
\item {Grp. gram.:f.}
\end{itemize}
Planta africana.
\section{Íncude}
\begin{itemize}
\item {Grp. gram.:f.}
\end{itemize}
\begin{itemize}
\item {Utilização:Poét.}
\end{itemize}
\begin{itemize}
\item {Proveniência:(Lat. \textunderscore incus\textunderscore , \textunderscore incudis\textunderscore )}
\end{itemize}
O mesmo que \textunderscore bigorna\textunderscore .
\section{Incúdico}
\begin{itemize}
\item {Grp. gram.:adj.}
\end{itemize}
\begin{itemize}
\item {Utilização:Anat.}
\end{itemize}
\begin{itemize}
\item {Proveniência:(Do lat. \textunderscore incus\textunderscore , \textunderscore incudis\textunderscore )}
\end{itemize}
Diz-se da articulação com o osso, que, no ouvido interno, se chama bigorna.
\section{Inculca}
\begin{itemize}
\item {Grp. gram.:f.}
\end{itemize}
\begin{itemize}
\item {Utilização:Fig.}
\end{itemize}
\begin{itemize}
\item {Grp. gram.:M.}
\end{itemize}
Acto ou effeito de inculcar.
Pessôa que inculca.
Suggestão.
Inculcador.
Busca, pesquisa:«\textunderscore ...á nossa inculca...\textunderscore »R. Lobo, \textunderscore Côrte na Ald.\textunderscore , II, 39.
\section{Inculcadeira}
\begin{itemize}
\item {Grp. gram.:f.  e  adj.}
\end{itemize}
\begin{itemize}
\item {Proveniência:(De \textunderscore inculcar\textunderscore )}
\end{itemize}
Mulher, que inculca, que dá informações.
Alcoviteira.
\section{Inculcador}
\begin{itemize}
\item {Grp. gram.:m.  e  adj.}
\end{itemize}
\begin{itemize}
\item {Proveniência:(Lat. \textunderscore inculcator\textunderscore )}
\end{itemize}
O que inculca.
\section{Inculcar}
\begin{itemize}
\item {Grp. gram.:v. t.}
\end{itemize}
\begin{itemize}
\item {Grp. gram.:V. p.}
\end{itemize}
\begin{itemize}
\item {Proveniência:(Lat. \textunderscore inculcare\textunderscore )}
\end{itemize}
Informar á cêrca de: \textunderscore inculcar criadas\textunderscore .
Recommendar.
Aconselhar.
Preconizar.
Encarecer.
Revelar; suggerir.
Impor ou insinuar a importância própria ou o valor próprio.
Fazer por dar na vista; fazer reclamo a si próprio.
\section{Inculpabilidade}
\begin{itemize}
\item {Grp. gram.:f.}
\end{itemize}
\begin{itemize}
\item {Proveniência:(De \textunderscore in...\textunderscore  + \textunderscore culpabilidade\textunderscore )}
\end{itemize}
Falta de culpabilidade.
\section{Inculpação}
\begin{itemize}
\item {Grp. gram.:f.}
\end{itemize}
\begin{itemize}
\item {Proveniência:(Lat. \textunderscore inculpatio\textunderscore )}
\end{itemize}
Acto ou effeito de inculpar.
Estado de quem é inculpado.
\section{Inculpadamente}
\begin{itemize}
\item {Grp. gram.:adv.}
\end{itemize}
\begin{itemize}
\item {Proveniência:(De \textunderscore inculpado\textunderscore )}
\end{itemize}
Sem culpa.
\section{Inculpado}
\begin{itemize}
\item {Grp. gram.:adj.}
\end{itemize}
\begin{itemize}
\item {Proveniência:(Lat. \textunderscore inculpatus\textunderscore )}
\end{itemize}
Que não tem culpa; que está innocente.
\section{Inculpar}
\begin{itemize}
\item {Grp. gram.:v. t.}
\end{itemize}
\begin{itemize}
\item {Proveniência:(Lat. \textunderscore inculpare\textunderscore )}
\end{itemize}
Attribuir culpas a; accusar; incriminar.
\section{Inculpável}
\begin{itemize}
\item {Grp. gram.:adj.}
\end{itemize}
\begin{itemize}
\item {Proveniência:(Lat. \textunderscore inculpabilis\textunderscore )}
\end{itemize}
O mesmo que \textunderscore inculpado\textunderscore .
\section{Inculpavelmente}
\begin{itemize}
\item {Grp. gram.:adv.}
\end{itemize}
De modo inculpável.
\section{Inculposamente}
\begin{itemize}
\item {Grp. gram.:adv.}
\end{itemize}
De modo inculposo; sem culpa; innocentemente.
\section{Inculposo}
\begin{itemize}
\item {Grp. gram.:adj.}
\end{itemize}
\begin{itemize}
\item {Proveniência:(De \textunderscore in...\textunderscore  + \textunderscore culposo\textunderscore )}
\end{itemize}
Em que não há culpa.
\section{Incultivável}
\begin{itemize}
\item {Grp. gram.:adj.}
\end{itemize}
\begin{itemize}
\item {Proveniência:(De \textunderscore in...\textunderscore  + \textunderscore cultivável\textunderscore )}
\end{itemize}
Que não é cultivável; improductivo.
\section{Inculto}
\begin{itemize}
\item {Grp. gram.:adj.}
\end{itemize}
\begin{itemize}
\item {Proveniência:(Lat. \textunderscore incultus\textunderscore )}
\end{itemize}
Que não é culto, que não é cultivado.
Árido, agreste.
Rude.
Desataviado, singelo.
\section{Incumbência}
\begin{itemize}
\item {Grp. gram.:f.}
\end{itemize}
Acto ou effeito de incumbir; negócio que se incumbe.
\section{Incumbente}
\begin{itemize}
\item {Grp. gram.:adj.}
\end{itemize}
\begin{itemize}
\item {Proveniência:(Lat. \textunderscore incumbens\textunderscore )}
\end{itemize}
Inclinado para a terra.
\section{Incumbir}
\begin{itemize}
\item {Grp. gram.:v. t.}
\end{itemize}
\begin{itemize}
\item {Grp. gram.:V. i.}
\end{itemize}
\begin{itemize}
\item {Proveniência:(Lat. \textunderscore incumbere\textunderscore )}
\end{itemize}
Encarregar de: \textunderscore incumbi-o da compra de um cavallo\textunderscore .
Confiar; encarregar: \textunderscore incumbi-lhe a compra de um cavallo\textunderscore .
Estar a cargo.
Impender; caber: \textunderscore incumbe-me estudar êsse processo\textunderscore .
\section{Incumprido}
\begin{itemize}
\item {Grp. gram.:adj.}
\end{itemize}
Não cumprido. Cf. Garrett, \textunderscore Port. na Balança\textunderscore , 247.
\section{Incunábulo}
\begin{itemize}
\item {Grp. gram.:m.}
\end{itemize}
\begin{itemize}
\item {Proveniência:(Lat. \textunderscore incunabula\textunderscore )}
\end{itemize}
Obra impressa, que data da origem da imprensa. Cf. Latino, \textunderscore Elogios\textunderscore , 368.
\section{Incunável}
\begin{itemize}
\item {Grp. gram.:m.}
\end{itemize}
\begin{itemize}
\item {Utilização:Des.}
\end{itemize}
O mesmo que \textunderscore incunábulo\textunderscore .
\section{Incurabilidade}
\begin{itemize}
\item {Grp. gram.:f.}
\end{itemize}
Qualidade de incurável.
\section{Incurável}
\begin{itemize}
\item {Grp. gram.:adj.}
\end{itemize}
\begin{itemize}
\item {Proveniência:(De \textunderscore in...\textunderscore  + \textunderscore curável\textunderscore )}
\end{itemize}
Que não é curável; que se não póde curar.
Irremediável: \textunderscore desgraças incuráveis\textunderscore .
\section{Incuravelmente}
\begin{itemize}
\item {Grp. gram.:adv.}
\end{itemize}
De modo incurável; irremediavelmente.
\section{Incúria}
\begin{itemize}
\item {Grp. gram.:f.}
\end{itemize}
\begin{itemize}
\item {Proveniência:(Lat. \textunderscore incuria\textunderscore )}
\end{itemize}
Falta de cuidado; desleixo; inércia.
\section{Incurial}
\begin{itemize}
\item {Grp. gram.:adj.}
\end{itemize}
\begin{itemize}
\item {Proveniência:(De \textunderscore in...\textunderscore  + \textunderscore curial\textunderscore )}
\end{itemize}
Que não é curial.
\section{Incurialidade}
\begin{itemize}
\item {Grp. gram.:f.}
\end{itemize}
Qualidade de incurial.
\section{Incuriosamente}
\begin{itemize}
\item {Grp. gram.:adv.}
\end{itemize}
\begin{itemize}
\item {Proveniência:(De \textunderscore in...\textunderscore  + \textunderscore curiosamente\textunderscore )}
\end{itemize}
Sem curiosidade.
\section{Incuriosidade}
\begin{itemize}
\item {Grp. gram.:f.}
\end{itemize}
\begin{itemize}
\item {Proveniência:(Lat. \textunderscore incuriositas\textunderscore )}
\end{itemize}
Qualidade de incurioso; falta de curiosidade.
\section{Incurioso}
\begin{itemize}
\item {Grp. gram.:adj.}
\end{itemize}
\begin{itemize}
\item {Proveniência:(Lat. \textunderscore incuriosus\textunderscore )}
\end{itemize}
Que não é curioso; que não tem curiosidade.
Negligente, indolente.
\section{Incursão}
\begin{itemize}
\item {Grp. gram.:f.}
\end{itemize}
\begin{itemize}
\item {Utilização:Fig.}
\end{itemize}
\begin{itemize}
\item {Proveniência:(Lat. \textunderscore incursio\textunderscore )}
\end{itemize}
Invasão militar.
Correria hostil; invasão.
Contaminação.
\section{Incurso}
\begin{itemize}
\item {Grp. gram.:m.}
\end{itemize}
\begin{itemize}
\item {Grp. gram.:Adj.}
\end{itemize}
\begin{itemize}
\item {Proveniência:(Lat. \textunderscore incursus\textunderscore )}
\end{itemize}
Acto de incorrer; incursão.
Invasão. Cf. Latino, \textunderscore Humboldt\textunderscore , 102.
Que incorreu.
Que está sujeito a penalidades, a censuras, etc.
Que é abrangido por uma disposição legal: \textunderscore incurso num artigo do Código Penal\textunderscore .
\section{Incusa}
\begin{itemize}
\item {Grp. gram.:f.}
\end{itemize}
\begin{itemize}
\item {Utilização:Des.}
\end{itemize}
\begin{itemize}
\item {Proveniência:(De \textunderscore incuso\textunderscore )}
\end{itemize}
Moéda ou medalha, cunhada de um só lado, por incúria ou precipitação do fabricante.
\section{Incuso}
\begin{itemize}
\item {Grp. gram.:adj.}
\end{itemize}
\begin{itemize}
\item {Proveniência:(Lat. \textunderscore incusus\textunderscore )}
\end{itemize}
Que é cunhado só de um lado, (falando-se de moédas ou medalhas).
\section{Incutir}
\begin{itemize}
\item {Grp. gram.:v. t.}
\end{itemize}
\begin{itemize}
\item {Proveniência:(Lat. \textunderscore incutere\textunderscore )}
\end{itemize}
Insinuar; infundir no ânimo de alguém; suggerir: \textunderscore incutir ódios\textunderscore .
\section{Inda}
\begin{itemize}
\item {Grp. gram.:adv.}
\end{itemize}
\begin{itemize}
\item {Proveniência:(Lat. \textunderscore inde\textunderscore )}
\end{itemize}
O mesmo que \textunderscore ainda\textunderscore .--É fórma \textunderscore pop.\textunderscore  e \textunderscore fam.\textunderscore  O \textunderscore Elucidário\textunderscore  considera-a mais culta que \textunderscore ainda\textunderscore .
\section{Indagação}
\begin{itemize}
\item {Grp. gram.:f.}
\end{itemize}
\begin{itemize}
\item {Proveniência:(Lat. \textunderscore indagatio\textunderscore )}
\end{itemize}
Acto ou effeito de indagar.
Pesquisa; investigação; devassa.
\section{Indagador}
\begin{itemize}
\item {Grp. gram.:adj.}
\end{itemize}
\begin{itemize}
\item {Grp. gram.:M.}
\end{itemize}
Que indaga, que pesquisa, que perscruta.
Aquelle que indaga.
\section{Indagar}
\begin{itemize}
\item {Grp. gram.:v. t.}
\end{itemize}
\begin{itemize}
\item {Grp. gram.:V. i.}
\end{itemize}
\begin{itemize}
\item {Proveniência:(Lat. \textunderscore indagare\textunderscore )}
\end{itemize}
Seguir a pista de; procurar, buscar.
Pesquisar; investigar; esquadrinhar.
Proceder a investigações; fazer pesquisas.
\section{Indagatório}
\begin{itemize}
\item {Grp. gram.:adj.}
\end{itemize}
\begin{itemize}
\item {Proveniência:(De \textunderscore indagar\textunderscore )}
\end{itemize}
Que serve para indagações; que faculta a verificação de um facto.
\section{Indagável}
\begin{itemize}
\item {Grp. gram.:adj.}
\end{itemize}
\begin{itemize}
\item {Proveniência:(Lat. \textunderscore indagabilis\textunderscore )}
\end{itemize}
Que se póde indagar; perscrutável.
\section{Indaiá}
\begin{itemize}
\item {Grp. gram.:m.}
\end{itemize}
Gênero de palmeiras americanas.
\section{Indaiá-açu}
\begin{itemize}
\item {Grp. gram.:m.}
\end{itemize}
Espécie de palmeira do Brasil.
\section{Indaiá-rasteiro}
\begin{itemize}
\item {Grp. gram.:m.}
\end{itemize}
\begin{itemize}
\item {Utilização:Bras}
\end{itemize}
Espécie de palmeira.
\section{Indajá}
\begin{itemize}
\item {Grp. gram.:m.}
\end{itemize}
O mesmo que \textunderscore indaiá-açu\textunderscore .
\section{Indas}
\begin{itemize}
\item {Grp. gram.:adv.}
\end{itemize}
\begin{itemize}
\item {Utilização:Pop.}
\end{itemize}
O mesmo que \textunderscore inda\textunderscore .
\section{Inde}
\begin{itemize}
\item {Grp. gram.:adv.}
\end{itemize}
\begin{itemize}
\item {Utilização:Ant.}
\end{itemize}
O mesmo que \textunderscore inda\textunderscore .
\section{Indébito}
\begin{itemize}
\item {Grp. gram.:adj.}
\end{itemize}
\begin{itemize}
\item {Proveniência:(Lat. \textunderscore indebitus\textunderscore )}
\end{itemize}
Que não é devido.
Que se pagou sem sêr devido.
Immerecido.
\section{Indecência}
\begin{itemize}
\item {Grp. gram.:f.}
\end{itemize}
\begin{itemize}
\item {Proveniência:(Lat. \textunderscore indecentia\textunderscore )}
\end{itemize}
Qualidade de indecente.
Obscenidade; deshonestidade.
Inconveniência.
Acto ou dito indecente.
\section{Indecente}
\begin{itemize}
\item {Grp. gram.:adj.}
\end{itemize}
\begin{itemize}
\item {Proveniência:(Lat. \textunderscore indecens\textunderscore )}
\end{itemize}
Que não é decente; indecoroso; deshonesto; inconveniente.
\section{Indecentemente}
\begin{itemize}
\item {Grp. gram.:adv.}
\end{itemize}
De modo indecente; deshonestamente; indecorosamente; inconvenientemente.
\section{Indecidido}
\begin{itemize}
\item {Grp. gram.:adj.}
\end{itemize}
\begin{itemize}
\item {Proveniência:(De \textunderscore in...\textunderscore  + \textunderscore decidido\textunderscore )}
\end{itemize}
Não decidido; que não está averiguado; não resolvido: \textunderscore questão indecidida\textunderscore .
\section{Indecifrável}
\begin{itemize}
\item {Grp. gram.:adj.}
\end{itemize}
\begin{itemize}
\item {Proveniência:(De \textunderscore in...\textunderscore  + \textunderscore decifrável\textunderscore )}
\end{itemize}
Que se não póde decifrar.
\section{Indecifravelmente}
\begin{itemize}
\item {Grp. gram.:adv.}
\end{itemize}
De modo indecifrável.
\section{Indecisamente}
\begin{itemize}
\item {Grp. gram.:adv.}
\end{itemize}
De modo indeciso; com indecisão; perplexamente.
\section{Indecisão}
\begin{itemize}
\item {Grp. gram.:f.}
\end{itemize}
\begin{itemize}
\item {Proveniência:(De \textunderscore in...\textunderscore  + \textunderscore decisão\textunderscore )}
\end{itemize}
Qualidade de indeciso; hesitação; perplexidade.
\section{Indeciso}
\begin{itemize}
\item {Grp. gram.:adj.}
\end{itemize}
\begin{itemize}
\item {Proveniência:(Do lat. \textunderscore in\textunderscore  + \textunderscore decisus\textunderscore )}
\end{itemize}
Que não está decidido.
Duvidoso.
Hesitante; irresoluto.
Indeterminado; vago.
Froixo.
\section{Indeclarável}
\begin{itemize}
\item {Grp. gram.:adj.}
\end{itemize}
\begin{itemize}
\item {Proveniência:(De \textunderscore in...\textunderscore  + \textunderscore declarar\textunderscore )}
\end{itemize}
Que se não póde declarar.
\section{Indeclinabilidade}
\begin{itemize}
\item {Grp. gram.:f.}
\end{itemize}
Qualidade daquillo que é indeclinável.
\section{Indeclinável}
\begin{itemize}
\item {Grp. gram.:adj.}
\end{itemize}
\begin{itemize}
\item {Proveniência:(Lat. \textunderscore indeclinabilis\textunderscore )}
\end{itemize}
Que não é declinável: \textunderscore o latim tem nomes indeclinaveis\textunderscore .
Inevitável; irrecusável: \textunderscore um preceito indeclinável\textunderscore .
\section{Indeclinavelmente}
\begin{itemize}
\item {Grp. gram.:adv.}
\end{itemize}
De modo indeclinável.
\section{Indecomponível}
\begin{itemize}
\item {Grp. gram.:adj.}
\end{itemize}
\begin{itemize}
\item {Proveniência:(De \textunderscore in...\textunderscore  + \textunderscore decomponivel\textunderscore )}
\end{itemize}
Que se não póde decompor.
\section{Indecomposto}
\begin{itemize}
\item {Grp. gram.:adj.}
\end{itemize}
\begin{itemize}
\item {Proveniência:(De \textunderscore in...\textunderscore  + \textunderscore decomposto\textunderscore )}
\end{itemize}
Que não é decomposto.
\section{Indecoramente}
\begin{itemize}
\item {Grp. gram.:adv.}
\end{itemize}
De modo indécoro; indecorosamente.
\section{Indécoro}
\begin{itemize}
\item {Grp. gram.:adj.}
\end{itemize}
\begin{itemize}
\item {Proveniência:(Lat. \textunderscore indecorus\textunderscore )}
\end{itemize}
O mesmo que \textunderscore indecoroso\textunderscore .
\section{Indecóro}
\begin{itemize}
\item {Grp. gram.:m.}
\end{itemize}
\begin{itemize}
\item {Proveniência:(De \textunderscore in...\textunderscore  + \textunderscore decóro\textunderscore . Cp. lat. \textunderscore indecor\textunderscore )}
\end{itemize}
Falta de decôro.
Acto indecoroso.
\section{Indecôro}
\begin{itemize}
\item {Grp. gram.:m.}
\end{itemize}
\begin{itemize}
\item {Proveniência:(De \textunderscore in...\textunderscore  + \textunderscore decóro\textunderscore . Cp. lat. \textunderscore indecor\textunderscore )}
\end{itemize}
Falta de decôro.
Acto indecoroso.
\section{Indecorosamente}
\begin{itemize}
\item {Grp. gram.:adv.}
\end{itemize}
De modo indecoroso, indecentemente; indignamente.
\section{Indecoroso}
\begin{itemize}
\item {Grp. gram.:adj.}
\end{itemize}
\begin{itemize}
\item {Proveniência:(De \textunderscore in...\textunderscore  + \textunderscore decoroso\textunderscore )}
\end{itemize}
Que não é decoroso.
Indigno.
Indecente; escandaloso; obsceno.
\section{Indefectibilidade}
\begin{itemize}
\item {Grp. gram.:f.}
\end{itemize}
Qualidade de indefectível.
\section{Indefectível}
\begin{itemize}
\item {Grp. gram.:adj.}
\end{itemize}
\begin{itemize}
\item {Proveniência:(De \textunderscore in...\textunderscore  + \textunderscore defectível\textunderscore )}
\end{itemize}
Que não falha; infallível.
Que não perece; que se não destrói.
\section{Indefectivelmente}
\begin{itemize}
\item {Grp. gram.:adv.}
\end{itemize}
De modo indefectivel.
\section{Indefensável}
\begin{itemize}
\item {Grp. gram.:adj.}
\end{itemize}
\begin{itemize}
\item {Proveniência:(De \textunderscore in...\textunderscore  + \textunderscore defensável\textunderscore )}
\end{itemize}
Que não é defensável; que não tem ou não merece defesa.
\section{Indefensível}
\begin{itemize}
\item {Grp. gram.:adj.}
\end{itemize}
O mesmo que \textunderscore indefensável\textunderscore .
\section{Indefenso}
\begin{itemize}
\item {Grp. gram.:adj.}
\end{itemize}
\begin{itemize}
\item {Proveniência:(Lat. \textunderscore indefensus\textunderscore )}
\end{itemize}
Que não tem defesa, que não é defendido.
Desarmado; fraco.
\section{Indeferido}
\begin{itemize}
\item {Grp. gram.:adj.}
\end{itemize}
\begin{itemize}
\item {Proveniência:(De \textunderscore indeferir\textunderscore )}
\end{itemize}
Desattendido.
Que não teve despacho, ou que teve despacho contrário ao que se requereu: \textunderscore requerimento indeferido\textunderscore .
\section{Indeferimento}
\begin{itemize}
\item {Grp. gram.:m.}
\end{itemize}
Acto ou effeito de indeferir.
\section{Indeferir}
\begin{itemize}
\item {Grp. gram.:v. t.}
\end{itemize}
\begin{itemize}
\item {Proveniência:(De \textunderscore in...\textunderscore  + \textunderscore deferir\textunderscore )}
\end{itemize}
Despachar desfavoravelmente.
Dar despacho contrário a.
Não deferir; desattender: \textunderscore indeferir um pedido\textunderscore .
\section{Indeferível}
\begin{itemize}
\item {Grp. gram.:adj.}
\end{itemize}
\begin{itemize}
\item {Proveniência:(De \textunderscore indeferir\textunderscore )}
\end{itemize}
Que não póde ou não deve sêr deferido.
\section{Indefeso}
\begin{itemize}
\item {fónica:fê}
\end{itemize}
\begin{itemize}
\item {Grp. gram.:adj.}
\end{itemize}
O mesmo que \textunderscore indefenso\textunderscore .
\section{Indefessamente}
\begin{itemize}
\item {Grp. gram.:adv.}
\end{itemize}
De modo indefesso; incansavelmente.
\section{Indefesso}
\begin{itemize}
\item {Grp. gram.:adj.}
\end{itemize}
\begin{itemize}
\item {Proveniência:(Lat. \textunderscore indefessus\textunderscore )}
\end{itemize}
Não cansado; incansável; laborioso: \textunderscore o indefesso escritor\textunderscore .
\section{Indeficiente}
\begin{itemize}
\item {Grp. gram.:adj.}
\end{itemize}
\begin{itemize}
\item {Proveniência:(De \textunderscore in...\textunderscore  + \textunderscore deficiente\textunderscore )}
\end{itemize}
Que não é deficiente; bastante.
\section{Indeficientemente}
\begin{itemize}
\item {Grp. gram.:adv.}
\end{itemize}
De modo indeficiente.
\section{Indefinidamente}
\begin{itemize}
\item {Grp. gram.:adv.}
\end{itemize}
De modo indefinido, interminavelmente; sem termo conhecido.
\section{Indefinido}
\begin{itemize}
\item {Grp. gram.:adj.}
\end{itemize}
\begin{itemize}
\item {Grp. gram.:M.}
\end{itemize}
\begin{itemize}
\item {Proveniência:(Lat. \textunderscore indefinitus\textunderscore )}
\end{itemize}
Não definido; vago; indeterminado.
Genérico; que não tem limites ou cujos limites são desconhecidos: \textunderscore o espaço indefinido\textunderscore .
Aquillo que é indefinido.
\section{Indefinito}
\begin{itemize}
\item {Grp. gram.:adj.}
\end{itemize}
O mesmo que \textunderscore indefinido\textunderscore .
\section{Indefinível}
\begin{itemize}
\item {Grp. gram.:adj.}
\end{itemize}
\begin{itemize}
\item {Proveniência:(De \textunderscore in...\textunderscore  + \textunderscore definível\textunderscore )}
\end{itemize}
Que se não póde definir; indefinido.
\section{Indegrar}
\begin{itemize}
\item {Grp. gram.:v. i.}
\end{itemize}
\begin{itemize}
\item {Utilização:Prov.}
\end{itemize}
\begin{itemize}
\item {Utilização:beir.}
\end{itemize}
Abastecer-se de água pluvial a nascente.
(Fórma pop. de \textunderscore integrar\textunderscore )
\section{Indehiscência}
\begin{itemize}
\item {Grp. gram.:f.}
\end{itemize}
\begin{itemize}
\item {Proveniência:(De \textunderscore in\textunderscore  + \textunderscore dehiscência\textunderscore )}
\end{itemize}
Propriedade dos frutos indehiscentes.
\section{Indehiscente}
\begin{itemize}
\item {Grp. gram.:adj.}
\end{itemize}
\begin{itemize}
\item {Proveniência:(De \textunderscore in...\textunderscore  + \textunderscore dehiscente\textunderscore )}
\end{itemize}
Que se não abre naturalmente na época da maturação, (falando-se de certos frutos).
\section{Indeiscência}
\begin{itemize}
\item {fónica:de-is}
\end{itemize}
\begin{itemize}
\item {Grp. gram.:f.}
\end{itemize}
\begin{itemize}
\item {Proveniência:(De \textunderscore in\textunderscore  + \textunderscore deiscência\textunderscore )}
\end{itemize}
Propriedade dos frutos indeiscentes.
\section{Indeiscente}
\begin{itemize}
\item {fónica:de-is}
\end{itemize}
\begin{itemize}
\item {Grp. gram.:adj.}
\end{itemize}
\begin{itemize}
\item {Proveniência:(De \textunderscore in...\textunderscore  + \textunderscore deiscente\textunderscore )}
\end{itemize}
Que se não abre naturalmente na época da maturação, (falando-se de certos frutos).
\section{Indelebilidade}
\begin{itemize}
\item {Grp. gram.:f.}
\end{itemize}
Qualidade daquillo que é indelével.
\section{Indelével}
\begin{itemize}
\item {Grp. gram.:adj.}
\end{itemize}
\begin{itemize}
\item {Proveniência:(Lat. \textunderscore indelebilis\textunderscore )}
\end{itemize}
Que se não póde apagar ou destruir: \textunderscore recordação indelével\textunderscore .
Que se não dissipa.
Indestructível.
\section{Indelevelmente}
\begin{itemize}
\item {Grp. gram.:adv.}
\end{itemize}
De modo indelével.
\section{Indeliberação}
\begin{itemize}
\item {Grp. gram.:f.}
\end{itemize}
\begin{itemize}
\item {Proveniência:(De \textunderscore in...\textunderscore  + \textunderscore deliberação\textunderscore )}
\end{itemize}
Falta de deliberação; perplexidade; inércia.
\section{Indeliberadamente}
\begin{itemize}
\item {Grp. gram.:adv.}
\end{itemize}
De modo indeliberado; irreflectidamente.
\section{Indeliberado}
\begin{itemize}
\item {Grp. gram.:adj.}
\end{itemize}
\begin{itemize}
\item {Proveniência:(De \textunderscore in...\textunderscore  + \textunderscore deliberado\textunderscore )}
\end{itemize}
Indeciso; irresoluto; impensado.
\section{Indelicadamente}
\begin{itemize}
\item {Grp. gram.:adv.}
\end{itemize}
De modo indelicado.
\section{Indelicadeza}
\begin{itemize}
\item {Grp. gram.:f.}
\end{itemize}
\begin{itemize}
\item {Proveniência:(De \textunderscore in...\textunderscore  + \textunderscore delicadeza\textunderscore )}
\end{itemize}
Falta de delicadeza; acto ou dito indelicado.
\section{Indelicado}
\begin{itemize}
\item {Grp. gram.:adj.}
\end{itemize}
\begin{itemize}
\item {Proveniência:(De \textunderscore in...\textunderscore  + \textunderscore delicado\textunderscore )}
\end{itemize}
Que não é delicado; rude, grosseiro; inconveniente.
\section{Indelineável}
\begin{itemize}
\item {Grp. gram.:adj.}
\end{itemize}
\begin{itemize}
\item {Proveniência:(De \textunderscore in...\textunderscore  + \textunderscore delinear\textunderscore )}
\end{itemize}
Que se não póde delinear.
\section{Indeminuto}
\begin{itemize}
\item {Grp. gram.:adj.}
\end{itemize}
\begin{itemize}
\item {Proveniência:(De \textunderscore in...\textunderscore  + \textunderscore deminuto\textunderscore )}
\end{itemize}
Que não é deminuto.
\section{Indemne}
\begin{itemize}
\item {Grp. gram.:adj.}
\end{itemize}
\begin{itemize}
\item {Proveniência:(Lat. \textunderscore indemnis\textunderscore )}
\end{itemize}
Que não soffreu damno ou prejuizo; incólume; illeso.
\section{Indemnidade}
\begin{itemize}
\item {Grp. gram.:f.}
\end{itemize}
\begin{itemize}
\item {Proveniência:(Lat. \textunderscore indemnitas\textunderscore )}
\end{itemize}
Qualidade de indemne.
Esquecimento ou perdão de culpa ou de um acto irregular.
\section{Indemnização}
\begin{itemize}
\item {Grp. gram.:f.}
\end{itemize}
Acto ou effeito de \textunderscore indemnizar\textunderscore .
\section{Indemnizador}
\begin{itemize}
\item {Grp. gram.:adj.}
\end{itemize}
\begin{itemize}
\item {Grp. gram.:M.}
\end{itemize}
Que indemniza.
Aquelle que indemniza.
\section{Indemnizar}
\begin{itemize}
\item {Grp. gram.:v. t.}
\end{itemize}
\begin{itemize}
\item {Proveniência:(De \textunderscore indemne\textunderscore )}
\end{itemize}
Compensar; dar reparação a; resarcir.
\section{Indemnizavel}
\begin{itemize}
\item {Grp. gram.:adj.}
\end{itemize}
Que se póde indemnizar.
\section{Indemonstrado}
\begin{itemize}
\item {Grp. gram.:adj.}
\end{itemize}
\begin{itemize}
\item {Proveniência:(Lat. \textunderscore indemonstratus\textunderscore )}
\end{itemize}
Não demonstrado.
\section{Indemonstrável}
\begin{itemize}
\item {Grp. gram.:adj.}
\end{itemize}
\begin{itemize}
\item {Proveniência:(Lat. \textunderscore indemonstrabilis\textunderscore )}
\end{itemize}
Que não póde sêr demonstrado.
\section{Independência}
\begin{itemize}
\item {Grp. gram.:f.}
\end{itemize}
\begin{itemize}
\item {Utilização:Bras}
\end{itemize}
\begin{itemize}
\item {Proveniência:(De \textunderscore in...\textunderscore  + \textunderscore dependência\textunderscore )}
\end{itemize}
Qualidade de independente; falta de dependência.
Libertação.
Planta euphorbiácea, que se tomou como symbolo da independência brasileira.
\section{Independente}
\begin{itemize}
\item {Grp. gram.:adj.}
\end{itemize}
\begin{itemize}
\item {Proveniência:(De \textunderscore in...\textunderscore  + \textunderscore dependente\textunderscore )}
\end{itemize}
Que não é dependente.
Contrário á dependência ou ás ideias de oppressão: \textunderscore espírito independente\textunderscore .
Livre; que procede por seu arbítrio; que não está sujeito.
Que se governa por leis próprias: \textunderscore nação independente\textunderscore .
Adverso á tyrannia ou ao despotismo.
\section{Independentemente}
\begin{itemize}
\item {Grp. gram.:adj.}
\end{itemize}
De modo independente; com independência; altivamente.
\section{Independer}
\begin{itemize}
\item {Grp. gram.:v. t.}
\end{itemize}
\begin{itemize}
\item {Utilização:bras}
\end{itemize}
\begin{itemize}
\item {Utilização:Neol.}
\end{itemize}
\begin{itemize}
\item {Proveniência:(De \textunderscore in...\textunderscore  + \textunderscore depender\textunderscore )}
\end{itemize}
Não depender; ser independente. Cf. \textunderscore Jornal do Comm.\textunderscore , do Rio, de 16-11-904.
\section{Indesatável}
\begin{itemize}
\item {Grp. gram.:adj.}
\end{itemize}
\begin{itemize}
\item {Proveniência:(De \textunderscore in...\textunderscore  + \textunderscore desatar\textunderscore )}
\end{itemize}
Que se não póde desatar.
\section{Indescortinável}
\begin{itemize}
\item {Grp. gram.:adj.}
\end{itemize}
\begin{itemize}
\item {Proveniência:(De \textunderscore in...\textunderscore  + \textunderscore descortinável\textunderscore )}
\end{itemize}
Que se não póde descortinar. Cf. Camillo, \textunderscore Noites de Insòmn.\textunderscore , X, 87.
\section{Indescrevível}
\begin{itemize}
\item {Grp. gram.:adj.}
\end{itemize}
O mesmo que \textunderscore indescriptível\textunderscore . Cf. Arn. Gama, \textunderscore Segr. do Abb.\textunderscore , 363.
\section{Indescriptibilidade}
\begin{itemize}
\item {Grp. gram.:f.}
\end{itemize}
Qualidade de indescriptível.
\section{Indescriptível}
\begin{itemize}
\item {Grp. gram.:adj.}
\end{itemize}
\begin{itemize}
\item {Proveniência:(De \textunderscore in...\textunderscore  + \textunderscore descriptível\textunderscore )}
\end{itemize}
Que se não póde descrever.
\section{Indescriptivelmente}
\begin{itemize}
\item {Grp. gram.:adv.}
\end{itemize}
De modo indescriptível.
\section{Indescritibilidade}
\begin{itemize}
\item {Grp. gram.:f.}
\end{itemize}
Qualidade de indescritível.
\section{Indescritível}
\begin{itemize}
\item {Grp. gram.:adj.}
\end{itemize}
\begin{itemize}
\item {Proveniência:(De \textunderscore in...\textunderscore  + \textunderscore descritível\textunderscore )}
\end{itemize}
Que se não póde descrever.
\section{Indescritivelmente}
\begin{itemize}
\item {Grp. gram.:adv.}
\end{itemize}
De modo indescritível.
\section{Indesculpável}
\begin{itemize}
\item {Grp. gram.:adj.}
\end{itemize}
\begin{itemize}
\item {Proveniência:(De \textunderscore in...\textunderscore  + \textunderscore desculpável\textunderscore )}
\end{itemize}
Que não é desculpável; que não merece desculpa.
\section{Indesenvolvido}
\begin{itemize}
\item {Grp. gram.:adj.}
\end{itemize}
Não desenvolvido; enfezado. Cf. Rui Barb., \textunderscore Réplica\textunderscore , 157.
\section{Indesinentemente}
\begin{itemize}
\item {Grp. gram.:adv.}
\end{itemize}
Sem cessar; constantemente.
(Cp. lat. \textunderscore indesinenter\textunderscore )
\section{Indesthronável}
\begin{itemize}
\item {Grp. gram.:adj.}
\end{itemize}
\begin{itemize}
\item {Proveniência:(De \textunderscore in...\textunderscore  + \textunderscore desthronar\textunderscore )}
\end{itemize}
Que se não póde desthronar.
\section{Indesthronizável}
\begin{itemize}
\item {Grp. gram.:adj.}
\end{itemize}
O mesmo que \textunderscore indesthronável\textunderscore .
\section{Indestreza}
\begin{itemize}
\item {Grp. gram.:f.}
\end{itemize}
\begin{itemize}
\item {Proveniência:(De \textunderscore in...\textunderscore  + \textunderscore destreza\textunderscore )}
\end{itemize}
Falta de destreza; ineptidão. Cf. Filinto, XX, 121 e 291.
\section{Indestrinçável}
\begin{itemize}
\item {Grp. gram.:adj.}
\end{itemize}
\begin{itemize}
\item {Proveniência:(De \textunderscore in...\textunderscore  + \textunderscore destrinçável\textunderscore )}
\end{itemize}
Que se não póde destrinçar.
\section{Indestronável}
\begin{itemize}
\item {Grp. gram.:adj.}
\end{itemize}
\begin{itemize}
\item {Proveniência:(De \textunderscore in...\textunderscore  + \textunderscore destronar\textunderscore )}
\end{itemize}
Que se não póde destronar.
\section{Indestronizável}
\begin{itemize}
\item {Grp. gram.:adj.}
\end{itemize}
O mesmo que \textunderscore indestronável\textunderscore .
\section{Indestructibilidade}
\begin{itemize}
\item {Grp. gram.:f.}
\end{itemize}
Qualidade daquillo que é indestructível.
\section{Indestructível}
\begin{itemize}
\item {Grp. gram.:adj.}
\end{itemize}
\begin{itemize}
\item {Proveniência:(De \textunderscore in...\textunderscore  + \textunderscore destructível\textunderscore )}
\end{itemize}
Que se não póde destruir.
Inalterável; firme.
\section{Indestructivelmente}
\begin{itemize}
\item {Grp. gram.:adv.}
\end{itemize}
De modo indestructível.
\section{Indestructo}
\begin{itemize}
\item {Grp. gram.:adj.}
\end{itemize}
\begin{itemize}
\item {Proveniência:(Do lat. \textunderscore in\textunderscore  + \textunderscore destructus\textunderscore )}
\end{itemize}
Não destruido. Cf. Filinto, XVI, 318.
\section{Indestrutibilidade}
\begin{itemize}
\item {Grp. gram.:f.}
\end{itemize}
Qualidade daquilo que é indestrutível.
\section{Indestrutível}
\begin{itemize}
\item {Grp. gram.:adj.}
\end{itemize}
\begin{itemize}
\item {Proveniência:(De \textunderscore in...\textunderscore  + \textunderscore destrutível\textunderscore )}
\end{itemize}
Que se não póde destruir.
Inalterável; firme.
\section{Indestrutivelmente}
\begin{itemize}
\item {Grp. gram.:adv.}
\end{itemize}
De modo indestrutível.
\section{Indestruto}
\begin{itemize}
\item {Grp. gram.:adj.}
\end{itemize}
\begin{itemize}
\item {Proveniência:(Do lat. \textunderscore in\textunderscore  + \textunderscore destructus\textunderscore )}
\end{itemize}
Não destruido. Cf. Filinto, XVI, 318.
\section{Indeterminabilidade}
\begin{itemize}
\item {Grp. gram.:f.}
\end{itemize}
Qualidade daquillo que é indeterminável.
\section{Indeterminação}
\begin{itemize}
\item {Grp. gram.:f.}
\end{itemize}
\begin{itemize}
\item {Proveniência:(De \textunderscore in...\textunderscore  + \textunderscore determinação\textunderscore )}
\end{itemize}
Falta de determinação.
Indecisão; perplexidade.
Qualidade daquillo que é indeterminado.
\section{Indeterminadamente}
\begin{itemize}
\item {Grp. gram.:adv.}
\end{itemize}
De modo indeterminado; vagamente.
Sem termo.
\section{Indeterminado}
\begin{itemize}
\item {Grp. gram.:adj.}
\end{itemize}
\begin{itemize}
\item {Grp. gram.:M.}
\end{itemize}
\begin{itemize}
\item {Proveniência:(Lat. \textunderscore indeterminatus\textunderscore )}
\end{itemize}
Que não é determinado.
Indeciso; indefinido.
Perplexo.
Aquillo que é vago, indeciso, indeterminado.
\section{Indeterminar}
\begin{itemize}
\item {Grp. gram.:v. t.}
\end{itemize}
\begin{itemize}
\item {Proveniência:(Lat. \textunderscore indeterminare\textunderscore )}
\end{itemize}
Tornar indeterminado.
\section{Indeterminável}
\begin{itemize}
\item {Grp. gram.:adj.}
\end{itemize}
\begin{itemize}
\item {Proveniência:(De \textunderscore in...\textunderscore  + \textunderscore determinável\textunderscore )}
\end{itemize}
Que se não póde determinar.
Indeterminado, indeciso.
\section{Indevidamente}
\begin{itemize}
\item {Grp. gram.:adv.}
\end{itemize}
De modo indevido; inconvenientemente.
\section{Indevido}
\begin{itemize}
\item {Grp. gram.:adj.}
\end{itemize}
\begin{itemize}
\item {Proveniência:(De \textunderscore in...\textunderscore  + \textunderscore devido\textunderscore )}
\end{itemize}
Que não é devido.
Que não é próprio; não merecido: \textunderscore castigo indevido\textunderscore .
Inconveniente.
\section{Indevoção}
\begin{itemize}
\item {Grp. gram.:f.}
\end{itemize}
\begin{itemize}
\item {Proveniência:(Lat. \textunderscore indevotio\textunderscore )}
\end{itemize}
Falta de devoção; impiedade.
\section{Indevotamente}
\begin{itemize}
\item {Grp. gram.:adv.}
\end{itemize}
De modo indevoto; sem devoção.
\section{Indevoto}
\begin{itemize}
\item {Grp. gram.:adj.}
\end{itemize}
\begin{itemize}
\item {Proveniência:(Lat. \textunderscore indevotus\textunderscore )}
\end{itemize}
Que não tem devoção; irreligioso.
Que mostra falta de devoção.
\section{Índex}
\begin{itemize}
\item {Grp. gram.:m.}
\end{itemize}
\begin{itemize}
\item {Grp. gram.:Adj.}
\end{itemize}
\begin{itemize}
\item {Proveniência:(Lat. \textunderscore index\textunderscore )}
\end{itemize}
O mesmo que \textunderscore índice\textunderscore .
Diz-se do dedo, também chamado indicador e que está collocado entre o dedo médio e o pollegar.
\section{Indexteridade}
\begin{itemize}
\item {Grp. gram.:f.}
\end{itemize}
O mesmo que \textunderscore indestreza\textunderscore . Cf. Filinto, XX, 266.
\section{Indez}
\begin{itemize}
\item {Grp. gram.:m.}
\end{itemize}
O mesmo que \textunderscore êndes\textunderscore .
\section{Indi}
\begin{itemize}
\item {Grp. gram.:m.}
\end{itemize}
Uma das línguas dos Indus.
\section{Indiana}
\begin{itemize}
\item {Grp. gram.:f.}
\end{itemize}
\begin{itemize}
\item {Utilização:Bras}
\end{itemize}
Espécie de bananeira.
\section{Indianamente}
\begin{itemize}
\item {Grp. gram.:adv.}
\end{itemize}
\begin{itemize}
\item {Proveniência:(De \textunderscore indiano\textunderscore )}
\end{itemize}
Á maneira dos Índios.
\section{Indianista}
\begin{itemize}
\item {Grp. gram.:m.}
\end{itemize}
\begin{itemize}
\item {Proveniência:(De \textunderscore indiano\textunderscore )}
\end{itemize}
Indivíduo, perito em coisas da Índia.
Orientalista; sanscritista.
\section{Indianita}
\begin{itemize}
\item {Grp. gram.:f.}
\end{itemize}
\begin{itemize}
\item {Proveniência:(De \textunderscore indiano\textunderscore )}
\end{itemize}
Substância mineral, branca ou esbranquiçada ou rosada, que se encontra na Índia.
\section{Indianizar}
\begin{itemize}
\item {Grp. gram.:v. t.}
\end{itemize}
\begin{itemize}
\item {Proveniência:(De \textunderscore indiano\textunderscore )}
\end{itemize}
Dar feição indiana a.
\section{Indiano}
\begin{itemize}
\item {Grp. gram.:adj.}
\end{itemize}
\begin{itemize}
\item {Grp. gram.:M.}
\end{itemize}
\begin{itemize}
\item {Proveniência:(Lat. \textunderscore indianus\textunderscore )}
\end{itemize}
Relativo á Índia.
Aquêlle que é natural da Índia.
\section{Indianólogo}
\begin{itemize}
\item {Grp. gram.:m.}
\end{itemize}
\begin{itemize}
\item {Proveniência:(De \textunderscore indiano\textunderscore  + gr. \textunderscore logos\textunderscore )}
\end{itemize}
O mesmo que \textunderscore indianista\textunderscore .
\section{Indiaru}
\begin{itemize}
\item {Grp. gram.:m.}
\end{itemize}
Sacerdote dos Parses.
\section{Indiático}
\begin{itemize}
\item {Grp. gram.:adj.}
\end{itemize}
Relativo á Índia; indiano. Cf. Camillo, \textunderscore Suicida\textunderscore , 33.
\section{Indicação}
\begin{itemize}
\item {Grp. gram.:f.}
\end{itemize}
\begin{itemize}
\item {Proveniência:(Lat. \textunderscore indicatio\textunderscore )}
\end{itemize}
Acto ou effeito de indicar.
Aquillo que indica.
\section{Indicador}
\begin{itemize}
\item {Grp. gram.:adj.}
\end{itemize}
\begin{itemize}
\item {Grp. gram.:M.}
\end{itemize}
\begin{itemize}
\item {Proveniência:(Lat. \textunderscore indicator\textunderscore )}
\end{itemize}
Que indica.
O mesmo que \textunderscore índex\textunderscore , (falando-se de um dedo).
Ponteiro, que vai indicando no mostrador telegráphico, letra por letra, os despachos que se transmittem.
Nome de outros apparelhos, que indicam a tensão dos vapores nas máquinas, o trabalho effectuado, etc.
Espécie de cuco africano.
Livro, caderno, folheto ou periódico, que dá indicações úteis.
\section{Indicante}
\begin{itemize}
\item {Grp. gram.:adj.}
\end{itemize}
\begin{itemize}
\item {Proveniência:(Lat. \textunderscore indicans\textunderscore )}
\end{itemize}
Que indica.
Que dá indício.
\section{Indicar}
\begin{itemize}
\item {Grp. gram.:v. t.}
\end{itemize}
\begin{itemize}
\item {Proveniência:(Lat. \textunderscore indicare\textunderscore )}
\end{itemize}
Mostrar com o dedo ou com um sinal qualquer.
Mostrar; revelar.
Determinar.
Mencionar.
Aconselhar: \textunderscore indicar o melhor meio\textunderscore .
Esboçar levemente: \textunderscore indicar um plano\textunderscore .
\section{Indicativo}
\begin{itemize}
\item {Grp. gram.:adj.}
\end{itemize}
\begin{itemize}
\item {Utilização:Gram.}
\end{itemize}
\begin{itemize}
\item {Grp. gram.:M.}
\end{itemize}
\begin{itemize}
\item {Utilização:Gram.}
\end{itemize}
\begin{itemize}
\item {Proveniência:(Lat. \textunderscore indicativus\textunderscore )}
\end{itemize}
Que indica ou serve para indicar.
Diz-se do modo verbal, que exprime a acção, de maneira positiva e absoluta.
Indicação; sinal.
Primeiro modo da conjugação verbal, no qual se exprime uma acção como real e certa ou positiva.
\section{Indicção}
\begin{itemize}
\item {Grp. gram.:f.}
\end{itemize}
\begin{itemize}
\item {Proveniência:(Lat. \textunderscore indictio\textunderscore )}
\end{itemize}
Período de quinze annos.
Convocação de assembleia ecclesiástica, para dia certo.
Prescripção, preceito.
\section{Índice}
\begin{itemize}
\item {Grp. gram.:m.}
\end{itemize}
\begin{itemize}
\item {Proveniência:(Do lat. \textunderscore index\textunderscore . Cp. \textunderscore índex\textunderscore )}
\end{itemize}
Tabella.
Lista dos capítulos, secções, parágraphos, etc., de um livro ou opúsculo, com indicação da página respectiva a cada uma dessas partes da obra.
Catálogo.
Relação alphabética.
Ponteiro.
Indicador.
Dedo indicador.
\section{Indiciado}
\begin{itemize}
\item {Grp. gram.:m.}
\end{itemize}
\begin{itemize}
\item {Proveniência:(De \textunderscore indiciar\textunderscore )}
\end{itemize}
Indivíduo que, num processo criminal, é considerado ou declarado criminoso, para sêr pronunciado e julgado.
\section{Indiciador}
\begin{itemize}
\item {Grp. gram.:adj.}
\end{itemize}
\begin{itemize}
\item {Grp. gram.:M.}
\end{itemize}
Que indicía, que dá indícios.
Aquelle que dá indícios, aquelle que accusa.
\section{Indiciar}
\begin{itemize}
\item {Grp. gram.:v. t.}
\end{itemize}
\begin{itemize}
\item {Proveniência:(De \textunderscore indicio\textunderscore )}
\end{itemize}
Dar indícios de.
Entremostrar.
Denunciar.
Declarar, em processo criminal, que há indícios de que a responsabilidade de um crime pesa sôbre alguém que deve sêr pronunciado.
\section{Indiciário}
\begin{itemize}
\item {Grp. gram.:adj.}
\end{itemize}
Relativo a indício; que envolve indício.
\section{Indícias}
\begin{itemize}
\item {Grp. gram.:f. pl.}
\end{itemize}
\begin{itemize}
\item {Proveniência:(De \textunderscore indício\textunderscore )}
\end{itemize}
Tributo antigo, a que eram obrigados os assassinos e malfeitores.
\section{Indiciativo}
\begin{itemize}
\item {Grp. gram.:adj.}
\end{itemize}
Que indicia ou dá indícios. Cf. Camillo, \textunderscore Críticos do Canc.\textunderscore , p. VI.
\section{Indício}
\begin{itemize}
\item {Grp. gram.:m.}
\end{itemize}
\begin{itemize}
\item {Proveniência:(Lat. \textunderscore indicium\textunderscore )}
\end{itemize}
O mesmo que \textunderscore indicação\textunderscore .
Vestígio.
Sinal ou facto, que deixa entrever alguma coisa, sem a descobrir completamente, mas constituindo princípio de prova.
\section{Indicioso}
\begin{itemize}
\item {Grp. gram.:adj.}
\end{itemize}
Que contém indícios; em que há indícios. Cf. Filinto, XVII, 203.
\section{Indicível}
\begin{itemize}
\item {Grp. gram.:adj.}
\end{itemize}
\begin{itemize}
\item {Utilização:P. us.}
\end{itemize}
\begin{itemize}
\item {Proveniência:(Do lat. \textunderscore in...\textunderscore  + \textunderscore dicere\textunderscore )}
\end{itemize}
O mesmo que \textunderscore indizível\textunderscore . Cf. Júl. Ribeiro, \textunderscore Carne\textunderscore .
\section{Índico}
\begin{itemize}
\item {Grp. gram.:adj.}
\end{itemize}
\begin{itemize}
\item {Proveniência:(Lat. \textunderscore indicus\textunderscore )}
\end{itemize}
O mesmo que \textunderscore indiano\textunderscore .
\section{Indictado}
\begin{itemize}
\item {Grp. gram.:adj.}
\end{itemize}
\begin{itemize}
\item {Utilização:Des.}
\end{itemize}
\begin{itemize}
\item {Proveniência:(Do lat. \textunderscore indictus\textunderscore )}
\end{itemize}
Notificado.
Annunciado publicamente:«\textunderscore ...que a proseguisse o concílio indictado...\textunderscore »Filinto, \textunderscore D. Man.\textunderscore , III, 58.
\section{Indículo}
\begin{itemize}
\item {Grp. gram.:m.}
\end{itemize}
\begin{itemize}
\item {Proveniência:(Lat. \textunderscore indiculus\textunderscore )}
\end{itemize}
Pequeno índice.
Pequeno apontamento; resenha; indicação summária.
\section{Indiferença}
\begin{itemize}
\item {Grp. gram.:f.}
\end{itemize}
\begin{itemize}
\item {Utilização:Phýs.}
\end{itemize}
\begin{itemize}
\item {Proveniência:(Lat. \textunderscore indifferentia\textunderscore )}
\end{itemize}
Qualidade daquele ou daquilo que é indiferente.
Desinteresse; desprendimento.
Negligência.
Insensibilidade moral; apatia; inconsciência mórbida.
Inércia dos corpos.
\section{Indiferente}
\begin{itemize}
\item {Grp. gram.:adj.}
\end{itemize}
\begin{itemize}
\item {Utilização:Chím.}
\end{itemize}
\begin{itemize}
\item {Utilização:Phýs.}
\end{itemize}
\begin{itemize}
\item {Grp. gram.:M.}
\end{itemize}
\begin{itemize}
\item {Proveniência:(Lat. \textunderscore indifferens\textunderscore )}
\end{itemize}
Que não apresenta motivo de preferência.
Que não é bom nem mau.
Que não tem cuidado ou diligência em alguma coisa.
Apático.
Que não tem amizade nem ódio a alguém.
Que antipatiza.
Que não tem tendência para se combinar com outro, (falando-se dos corpos).
Que não tem afinidade alguma, relativamente a outros corpos.
Que fórma igual juízo, á cêrca de todas as religiões, de todos os sistemas políticos, etc.
Aquele que não tem amizade a outro ou que apenas o conhece.
Aquele que quebrou relações de amizade com outrem.
Aquele que se desinteressa de qualquer religião ou de qualquer sistema político.
\section{Indiferentemente}
\begin{itemize}
\item {Grp. gram.:adv.}
\end{itemize}
De modo indiferente.
Sem preferência.
\section{Indiferentismo}
\begin{itemize}
\item {Grp. gram.:m.}
\end{itemize}
\begin{itemize}
\item {Proveniência:(De \textunderscore indiferente\textunderscore )}
\end{itemize}
Sistema dos que são indiferentes em religião, política, filosofia, etc.
\section{Indiferentista}
\begin{itemize}
\item {Grp. gram.:m.  e  adj.}
\end{itemize}
\begin{itemize}
\item {Proveniência:(De \textunderscore indiferente\textunderscore )}
\end{itemize}
Sectário do indiferentismo.
\section{Indifferença}
\begin{itemize}
\item {Grp. gram.:f.}
\end{itemize}
\begin{itemize}
\item {Utilização:Phýs.}
\end{itemize}
\begin{itemize}
\item {Proveniência:(Lat. \textunderscore indifferentia\textunderscore )}
\end{itemize}
Qualidade daquelle ou daquillo que é indifferente.
Desinteresse; desprendimento.
Negligência.
Insensibilidade moral; apathia; inconsciência mórbida.
Inércia dos corpos.
\section{Indifferente}
\begin{itemize}
\item {Grp. gram.:adj.}
\end{itemize}
\begin{itemize}
\item {Utilização:Chím.}
\end{itemize}
\begin{itemize}
\item {Utilização:Phýs.}
\end{itemize}
\begin{itemize}
\item {Grp. gram.:M.}
\end{itemize}
\begin{itemize}
\item {Proveniência:(Lat. \textunderscore indifferens\textunderscore )}
\end{itemize}
Que não apresenta motivo de preferência.
Que não é bom nem mau.
Que não tem cuidado ou diligência em alguma coisa.
Apáthico.
Que não tem amizade nem ódio a alguém.
Que anthipatiza.
Que não tem tendência para se combinar com outro, (falando-se dos corpos)
Que não tem affinidade alguma, relativamente a outros corpos.
Que fórma igual juízo, á cêrca de todas as religiões, de todos os systemas políticos, etc.
Aquelle que não tem amizade a outro ou que apenas o conhece.
Aquelle que quebrou relações de amizade com outrem.
Aquelle que se desinteressa de qualquer religião ou de qualquer systema politico.
\section{Indifferentemente}
\begin{itemize}
\item {Grp. gram.:adv.}
\end{itemize}
De modo indifferente.
Sem preferência.
\section{Indifferentismo}
\begin{itemize}
\item {Grp. gram.:m.}
\end{itemize}
\begin{itemize}
\item {Proveniência:(De \textunderscore indifferente\textunderscore )}
\end{itemize}
Systema dos que são indifferentes em religião, política, philosophia, etc.
\section{Indifferentista}
\begin{itemize}
\item {Grp. gram.:m.  e  adj.}
\end{itemize}
\begin{itemize}
\item {Proveniência:(De \textunderscore indifferente\textunderscore )}
\end{itemize}
Sectário do indifferentismo.
\section{Indiffusível}
\begin{itemize}
\item {Grp. gram.:adj.}
\end{itemize}
\begin{itemize}
\item {Proveniência:(De \textunderscore in...\textunderscore  + \textunderscore diffusivel\textunderscore )}
\end{itemize}
Que não é diffusivel.
\section{Indifusível}
\begin{itemize}
\item {Grp. gram.:adj.}
\end{itemize}
\begin{itemize}
\item {Proveniência:(De \textunderscore in...\textunderscore  + \textunderscore difusivel\textunderscore )}
\end{itemize}
Que não é difusivel.
\section{Indígena}
\begin{itemize}
\item {Grp. gram.:m.}
\end{itemize}
\begin{itemize}
\item {Grp. gram.:Adj.}
\end{itemize}
\begin{itemize}
\item {Proveniência:(Lat. \textunderscore indigena\textunderscore )}
\end{itemize}
Aquelle que nasceu no lugar ou país em que habita.
Originário ou próprio de um país ou de uma localidade.
\section{Indigenato}
\begin{itemize}
\item {Grp. gram.:m.}
\end{itemize}
Qualidade de indígena.
\section{Indigência}
\begin{itemize}
\item {Grp. gram.:f.}
\end{itemize}
\begin{itemize}
\item {Utilização:Fig.}
\end{itemize}
\begin{itemize}
\item {Proveniência:(Lat. \textunderscore indigentia\textunderscore )}
\end{itemize}
Falta do que é indispensável á vida.
Pobreza extrema.
Miséria.
Os indigentes.
Privação.
\section{Indígeno}
\begin{itemize}
\item {Grp. gram.:adj.}
\end{itemize}
\begin{itemize}
\item {Utilização:Des.}
\end{itemize}
\begin{itemize}
\item {Proveniência:(Lat. \textunderscore indigenus\textunderscore . Cp. \textunderscore indígena\textunderscore )}
\end{itemize}
O mesmo que \textunderscore indígena\textunderscore .
\section{Indigente}
\begin{itemize}
\item {Grp. gram.:m.  e  adj.}
\end{itemize}
\begin{itemize}
\item {Proveniência:(Lat. \textunderscore indigens\textunderscore )}
\end{itemize}
Aquelle que vive na indigência.
\section{Indigentemente}
\begin{itemize}
\item {Grp. gram.:adv.}
\end{itemize}
\begin{itemize}
\item {Proveniência:(De \textunderscore indigente\textunderscore )}
\end{itemize}
Com indigência; miseravelmente.
\section{Indigerível}
\begin{itemize}
\item {Grp. gram.:adj.}
\end{itemize}
\begin{itemize}
\item {Proveniência:(De \textunderscore in...\textunderscore  + \textunderscore digerível\textunderscore )}
\end{itemize}
Que não é digerível; que se não digere.
\section{Indigestão}
\begin{itemize}
\item {Grp. gram.:f.}
\end{itemize}
\begin{itemize}
\item {Utilização:Fam.}
\end{itemize}
\begin{itemize}
\item {Proveniência:(Lat. \textunderscore indigestio\textunderscore )}
\end{itemize}
Perturbação do estômago, proveniente da má digestão dos alimentos.
Acto de fartar-se: \textunderscore uma indigestão de theatro\textunderscore .
Grande quantidade.
\section{Indigestibilidade}
\begin{itemize}
\item {Grp. gram.:f.}
\end{itemize}
Qualidade de indigestível. Cf. Júl. Dinis, \textunderscore Morgadinha\textunderscore , 181.
\section{Indigestível}
\begin{itemize}
\item {Grp. gram.:adj.}
\end{itemize}
\begin{itemize}
\item {Proveniência:(De \textunderscore in...\textunderscore  + \textunderscore digestivel\textunderscore )}
\end{itemize}
O mesmo que \textunderscore indigerível\textunderscore .
\section{Indigesto}
\begin{itemize}
\item {Grp. gram.:adj.}
\end{itemize}
\begin{itemize}
\item {Utilização:Fig.}
\end{itemize}
\begin{itemize}
\item {Proveniência:(Lat. \textunderscore índigestus\textunderscore )}
\end{itemize}
Que se não digeriu.
Que se digere difficilmente ou que produz indigestão: \textunderscore o pepino é indigesto\textunderscore .
Desordenado; sem nexo; enfadonho; desagradável: \textunderscore livro indigesto\textunderscore .
\section{Indígete}
\begin{itemize}
\item {Grp. gram.:m.}
\end{itemize}
\begin{itemize}
\item {Utilização:Fig.}
\end{itemize}
\begin{itemize}
\item {Proveniência:(Lat. \textunderscore indiges\textunderscore )}
\end{itemize}
Homem divinizado.
Herói.
\section{Indigitamento}
\begin{itemize}
\item {Grp. gram.:m.}
\end{itemize}
\begin{itemize}
\item {Proveniência:(Lat. \textunderscore indigitamentum\textunderscore )}
\end{itemize}
Acto de indigitar.
\section{Indigitar}
\begin{itemize}
\item {Grp. gram.:v. t.}
\end{itemize}
\begin{itemize}
\item {Proveniência:(Lat. \textunderscore indigitare\textunderscore )}
\end{itemize}
Apontar com o dedo; mostrar, indicar; designar.
\section{Indignação}
\begin{itemize}
\item {Grp. gram.:f.}
\end{itemize}
\begin{itemize}
\item {Proveniência:(Lat. \textunderscore indignatio\textunderscore )}
\end{itemize}
Estado de quem indigna ou de quem se indigna.
Repulsão; aversão.
\section{Indignadamente}
\begin{itemize}
\item {Grp. gram.:adv.}
\end{itemize}
\begin{itemize}
\item {Proveniência:(De \textunderscore indignado\textunderscore )}
\end{itemize}
Com indignação.
\section{Indignado}
\begin{itemize}
\item {Grp. gram.:adj.}
\end{itemize}
\begin{itemize}
\item {Proveniência:(De \textunderscore indignar\textunderscore )}
\end{itemize}
Que mostra ou tem indignação; irado.
\section{Indignamente}
\begin{itemize}
\item {Grp. gram.:adv.}
\end{itemize}
De modo indigno.
\section{Indignar}
\begin{itemize}
\item {Grp. gram.:v. t.}
\end{itemize}
\begin{itemize}
\item {Grp. gram.:V. p.}
\end{itemize}
\begin{itemize}
\item {Proveniência:(Lat. \textunderscore indignari\textunderscore )}
\end{itemize}
Encher de cólera ou inspirar desprêzo, por sêr coisa ou pessôa indigna.
Indispor gravemente.
Sentir desprêzo ou cólera produzida por coisa ou pessôa indigna.
Revoltar-se.
Não se dignar, dedignar-se.
\section{Indignativo}
\begin{itemize}
\item {Grp. gram.:adj.}
\end{itemize}
\begin{itemize}
\item {Proveniência:(Lat. \textunderscore indignativus\textunderscore )}
\end{itemize}
Que mostra indignação.
Irascível.
\section{Indignatório}
\begin{itemize}
\item {Grp. gram.:m.}
\end{itemize}
\begin{itemize}
\item {Utilização:Anat.}
\end{itemize}
\begin{itemize}
\item {Utilização:ant.}
\end{itemize}
\begin{itemize}
\item {Proveniência:(Lat. \textunderscore indignator\textunderscore )}
\end{itemize}
Músculo abductor do ôlho.
\section{Indignidade}
\begin{itemize}
\item {Grp. gram.:f.}
\end{itemize}
\begin{itemize}
\item {Proveniência:(Lat. \textunderscore indignitas\textunderscore )}
\end{itemize}
Falta de dignidade.
Afronta.
Acção indigna; qualidade de indigno.
\section{Indigno}
\begin{itemize}
\item {Grp. gram.:adj.}
\end{itemize}
\begin{itemize}
\item {Proveniência:(Lat. \textunderscore indignus\textunderscore )}
\end{itemize}
Que não é digno; que não merece.
Que não é próprio.
Inconveniente.
Incapaz.
Desprezível; vil, torpe: \textunderscore acções indignas\textunderscore .
\section{Indignoso}
\begin{itemize}
\item {Grp. gram.:adj.}
\end{itemize}
\begin{itemize}
\item {Utilização:Des.}
\end{itemize}
\begin{itemize}
\item {Proveniência:(De \textunderscore indignar\textunderscore )}
\end{itemize}
Que causa indignação:«\textunderscore ...espectáculo... desagradável e indignoso para os Portugueses...\textunderscore »Filinto, \textunderscore D. Man.\textunderscore , II, 38.
\section{Índigo}
\begin{itemize}
\item {Grp. gram.:m.}
\end{itemize}
\begin{itemize}
\item {Proveniência:(Do lat. \textunderscore indicus\textunderscore )}
\end{itemize}
Substância còrante, que serve para tingir de azul, e que se extrai do indigueiro; anil.
Árvore solânea do Brasil.
\section{Indigófera}
\begin{itemize}
\item {Grp. gram.:f.}
\end{itemize}
\begin{itemize}
\item {Proveniência:(De \textunderscore indigo\textunderscore  + lat. \textunderscore ferre\textunderscore )}
\end{itemize}
Designação scientífica da anileira.
\section{Indigotato}
\begin{itemize}
\item {Grp. gram.:m.}
\end{itemize}
\begin{itemize}
\item {Proveniência:(De \textunderscore indigo\textunderscore )}
\end{itemize}
Combinação do ácido indigótico com uma base.
\section{Indigoteiro}
\begin{itemize}
\item {Grp. gram.:m.}
\end{itemize}
\begin{itemize}
\item {Proveniência:(Do fr. \textunderscore indigotier\textunderscore )}
\end{itemize}
O mesmo que \textunderscore indigueiro\textunderscore .
\section{Indigótico}
\begin{itemize}
\item {Grp. gram.:adj.}
\end{itemize}
\begin{itemize}
\item {Proveniência:(De \textunderscore índigo\textunderscore )}
\end{itemize}
Diz-se de um ácido, produzido pela acção do ácido nítrico sôbre o anil.
\section{Indigotina}
\begin{itemize}
\item {Grp. gram.:f.}
\end{itemize}
\begin{itemize}
\item {Proveniência:(De \textunderscore índigo\textunderscore )}
\end{itemize}
Substância sólida, volátil, insípida, inodora, crystallizável, de côr azul acobreada.
\section{Indigueiro}
\begin{itemize}
\item {Grp. gram.:m.}
\end{itemize}
\begin{itemize}
\item {Proveniência:(De \textunderscore índigo\textunderscore )}
\end{itemize}
Planta leguminosa, de que se extrai o anil; anileira.
\section{Indilgadeira}
\begin{itemize}
\item {Grp. gram.:f.}
\end{itemize}
\begin{itemize}
\item {Utilização:T. de Escalhão}
\end{itemize}
\begin{itemize}
\item {Proveniência:(De \textunderscore indilgar\textunderscore )}
\end{itemize}
Mulher activa, diligente.
\section{Indilgar}
\begin{itemize}
\item {Grp. gram.:v. i.}
\end{itemize}
\begin{itemize}
\item {Utilização:T. de Escalhão}
\end{itemize}
Trabalhar com actividade e intelligência.
\section{Indiligência}
\begin{itemize}
\item {Grp. gram.:f.}
\end{itemize}
\begin{itemize}
\item {Proveniência:(Lat. \textunderscore indiligentia\textunderscore )}
\end{itemize}
Falta de diligência.
Froixidão; inércia.
\section{Indiligente}
\begin{itemize}
\item {Grp. gram.:adj.}
\end{itemize}
\begin{itemize}
\item {Proveniência:(Lat. \textunderscore indiligens\textunderscore )}
\end{itemize}
Que não é diligente; negligente; desleixado.
\section{Indiligentemente}
\begin{itemize}
\item {Grp. gram.:adv.}
\end{itemize}
De modo indiligente.
\section{Indino}
\textunderscore adj.\textunderscore  (e der.)
(Fórma ant. de \textunderscore indigno\textunderscore  etc.)
\section{Índio}
\begin{itemize}
\item {Grp. gram.:m.  e  adj.}
\end{itemize}
\begin{itemize}
\item {Grp. gram.:M.}
\end{itemize}
\begin{itemize}
\item {Proveniência:(De \textunderscore Índia\textunderscore , n. p.)}
\end{itemize}
O mesmo que \textunderscore indiano\textunderscore .
Antiga moéda de prata, do tempo de D. Manuel, cunhada em memória do descobrimento da Índia.
Novo metal, encontrado no sulfureto de zinco natural.
\section{Índio}
\begin{itemize}
\item {Grp. gram.:m.}
\end{itemize}
\begin{itemize}
\item {Utilização:Ant.}
\end{itemize}
Barco indiano?:«\textunderscore ...deyxô o assy estar tamtos dias que os vasos se arearam, e apodreceram os índios, e a emvasadura alargou...\textunderscore »\textunderscore Cartas\textunderscore  de Aff. de Albuq., 120.
\section{Índios}
\begin{itemize}
\item {Grp. gram.:m. pl.}
\end{itemize}
Nome que, propriamente, designa os habitantes da Índia, também designados hoje por \textunderscore Indus\textunderscore , e que se estende aos habitantes da América, por supporem os descobridores do Novo-Mundo que, ao descobri-lo, tinham chegado á Índia pelo Occidente.
\section{Indirectamente}
\begin{itemize}
\item {Grp. gram.:adv.}
\end{itemize}
De modo indirecto; dissimuladamente.
\section{Indirecto}
\begin{itemize}
\item {Grp. gram.:adj.}
\end{itemize}
\begin{itemize}
\item {Utilização:Gram.}
\end{itemize}
\begin{itemize}
\item {Proveniência:(Lat. \textunderscore indirectus\textunderscore )}
\end{itemize}
Que não é directo; que não segue o meio ou caminho mais curto.
Oblíquo.
Que não é franco.
Disfarçado; simulado.
Que segue rodeios.
Collateral.
Diz-se dos complementos verbais precedidos de preposição.
\section{Indireitamente}
\begin{itemize}
\item {Grp. gram.:adv.}
\end{itemize}
\begin{itemize}
\item {Utilização:Ant.}
\end{itemize}
\begin{itemize}
\item {Proveniência:(De \textunderscore in...\textunderscore  + \textunderscore direito\textunderscore )}
\end{itemize}
Sem direito; com injustiça; por caminhos tortos.
\section{Indirigível}
\begin{itemize}
\item {Grp. gram.:adj.}
\end{itemize}
\begin{itemize}
\item {Proveniência:(De \textunderscore in...\textunderscore  + \textunderscore dirigivel\textunderscore )}
\end{itemize}
Que não é dirigível; que se não deixa dirigir ou governar.
\section{Indisceptável}
\begin{itemize}
\item {Grp. gram.:adj.}
\end{itemize}
\begin{itemize}
\item {Utilização:Des.}
\end{itemize}
Incontestável, indiscutível.
(Cp. lat. \textunderscore disceptare\textunderscore )
\section{Indiscernível}
\begin{itemize}
\item {Grp. gram.:adj.}
\end{itemize}
\begin{itemize}
\item {Proveniência:(De \textunderscore in...\textunderscore  + \textunderscore discernir\textunderscore )}
\end{itemize}
Que se não póde discernir.
\section{Indisciplina}
\begin{itemize}
\item {Grp. gram.:f.}
\end{itemize}
\begin{itemize}
\item {Proveniência:(Lat. \textunderscore indisciplina\textunderscore )}
\end{itemize}
Falta de disciplina; desobediência; rebellião.
\section{Indisciplinação}
\begin{itemize}
\item {Grp. gram.:f.}
\end{itemize}
\begin{itemize}
\item {Proveniência:(Lat. \textunderscore indisciplinatio\textunderscore )}
\end{itemize}
Acto de indisciplinar.
Indisciplina.
\section{Indisciplinabilidade}
\begin{itemize}
\item {Grp. gram.:f.}
\end{itemize}
Qualidade de indisciplinável.
\section{Indisciplinadamente}
\begin{itemize}
\item {Grp. gram.:adv.}
\end{itemize}
\begin{itemize}
\item {Proveniência:(De \textunderscore indisciplinar\textunderscore )}
\end{itemize}
Sem disciplina.
\section{Indisciplinar}
\begin{itemize}
\item {Grp. gram.:v. t.}
\end{itemize}
\begin{itemize}
\item {Proveniência:(Lat. \textunderscore indisciplinare\textunderscore )}
\end{itemize}
Promover a indisciplina de.
Revoltar.
Desmoralizar.
\section{Indisciplinável}
\begin{itemize}
\item {Grp. gram.:adj.}
\end{itemize}
\begin{itemize}
\item {Proveniência:(De \textunderscore in...\textunderscore  + \textunderscore disciplinável\textunderscore )}
\end{itemize}
Que não é disciplinável; que se não póde disciplinar.
Insubmisso.
\section{Indisciplinoso}
\begin{itemize}
\item {Grp. gram.:adj.}
\end{itemize}
\begin{itemize}
\item {Utilização:Des.}
\end{itemize}
\begin{itemize}
\item {Proveniência:(De \textunderscore indisciplina\textunderscore )}
\end{itemize}
Contrário á disciplina.
Em que não há disciplina.
\section{Indiscretamente}
\begin{itemize}
\item {Grp. gram.:adv.}
\end{itemize}
De modo indiscreto.
Levianamente.
\section{Indiscreto}
\begin{itemize}
\item {Grp. gram.:adj.}
\end{itemize}
\begin{itemize}
\item {Grp. gram.:M.}
\end{itemize}
\begin{itemize}
\item {Proveniência:(Lat. \textunderscore indiscretus\textunderscore )}
\end{itemize}
Que não é discreto.
Leviano.
Tagarela.
Inconfidente.
Aquelle que não tem discrição.
\section{Indiscrição}
\begin{itemize}
\item {Grp. gram.:f.}
\end{itemize}
\begin{itemize}
\item {Proveniência:(Lat. \textunderscore indiscretio\textunderscore )}
\end{itemize}
Qualidade daquelle ou daquillo que é indiscreto.
Falta de discrição; acto ou dito indiscreto.
\section{Indiscriminadamente}
\begin{itemize}
\item {Grp. gram.:adv.}
\end{itemize}
De modo indiscriminado; confusamente; a granel.
\section{Indiscriminado}
\begin{itemize}
\item {Grp. gram.:adj.}
\end{itemize}
\begin{itemize}
\item {Proveniência:(De \textunderscore in...\textunderscore  + \textunderscore discriminado\textunderscore )}
\end{itemize}
Não discriminado; misturado; confuso.
\section{Indiscriminável}
\begin{itemize}
\item {Grp. gram.:adj.}
\end{itemize}
\begin{itemize}
\item {Proveniência:(Lat. \textunderscore indiscriminabilis\textunderscore )}
\end{itemize}
Que se não póde discriminar.
\section{Indiscutibilidade}
\begin{itemize}
\item {Grp. gram.:f.}
\end{itemize}
Qualidade de indiscutível.
\section{Indiscutido}
\begin{itemize}
\item {Grp. gram.:adj.}
\end{itemize}
\begin{itemize}
\item {Proveniência:(De \textunderscore in...\textunderscore  + \textunderscore discutir\textunderscore )}
\end{itemize}
Não discutido; inconcusso.
\section{Indiscutível}
\begin{itemize}
\item {Grp. gram.:adj.}
\end{itemize}
\begin{itemize}
\item {Proveniência:(De \textunderscore in...\textunderscore  + \textunderscore discutível\textunderscore )}
\end{itemize}
Que não é discutível; indubitável.
Que não merece discussão.
\section{Indiscutivelmente}
\begin{itemize}
\item {Grp. gram.:adv.}
\end{itemize}
De modo indiscutível.
\section{Indiserto}
\begin{itemize}
\item {Grp. gram.:adj.}
\end{itemize}
\begin{itemize}
\item {Proveniência:(De \textunderscore in...\textunderscore  + \textunderscore diserto\textunderscore )}
\end{itemize}
Não diserto; pouco claro.
Pouco instructivo.
\section{Indisina}
\begin{itemize}
\item {Grp. gram.:f.}
\end{itemize}
Substância, extrahida da anilina.
(Cp. \textunderscore índigo\textunderscore )
\section{Indispensabilidade}
\begin{itemize}
\item {Grp. gram.:f.}
\end{itemize}
Qualidade de indispensável.
\section{Indispensável}
\begin{itemize}
\item {Grp. gram.:adj.}
\end{itemize}
\begin{itemize}
\item {Grp. gram.:M.}
\end{itemize}
\begin{itemize}
\item {Proveniência:(De \textunderscore in...\textunderscore  + \textunderscore dispensável\textunderscore )}
\end{itemize}
Que não é dispensável; de que não póde haver dispensa.
Infallível.
Que é absolutamente necessário.
Aquillo que é essencial.
Pequena mala ou bolsa, em que as senhoras levam dinheiro, lenço ou outros pequenos objectos, que se reputam indispensáveis.
\section{Indispensavelmente}
\begin{itemize}
\item {Grp. gram.:adv.}
\end{itemize}
De modo indispensável.
\section{Indisponibilidade}
\begin{itemize}
\item {Grp. gram.:f.}
\end{itemize}
Estado de indisponível.
\section{Indisponível}
\begin{itemize}
\item {Grp. gram.:adj.}
\end{itemize}
\begin{itemize}
\item {Proveniência:(De \textunderscore in...\textunderscore  + \textunderscore disponível\textunderscore )}
\end{itemize}
Que não é disponível; inalienável.
\section{Indispor}
\begin{itemize}
\item {Grp. gram.:v. t.}
\end{itemize}
\begin{itemize}
\item {Proveniência:(De \textunderscore in...\textunderscore  + \textunderscore dispor\textunderscore )}
\end{itemize}
Alterar a disposição de.
Perturbar ligeiramente o organismo de.
Produzir incômmodo em.
Zangar; indignar.
Malquistar; inimizar: \textunderscore a mulher indispô-lo com os irmãos\textunderscore .
\section{Indisposição}
\begin{itemize}
\item {Grp. gram.:f.}
\end{itemize}
\begin{itemize}
\item {Proveniência:(De \textunderscore in...\textunderscore  + \textunderscore disposição\textunderscore )}
\end{itemize}
Falta de disposição; desorganização.
Incommodo de saúde.
Desavença; inimizade.
\section{Indisposto}
\begin{itemize}
\item {Grp. gram.:adj.}
\end{itemize}
\begin{itemize}
\item {Proveniência:(Lat. \textunderscore indispositus\textunderscore )}
\end{itemize}
Desavindo.
Incommodado; adoentado.
\section{Indisputabilidade}
\begin{itemize}
\item {Grp. gram.:f.}
\end{itemize}
Qualidade daquillo que é indisputável.
\section{Indisputado}
\begin{itemize}
\item {Grp. gram.:adj.}
\end{itemize}
\begin{itemize}
\item {Proveniência:(De \textunderscore in\textunderscore  + \textunderscore disputado\textunderscore )}
\end{itemize}
Que não é disputado; inconcusso.
\section{Indisputável}
\begin{itemize}
\item {Grp. gram.:adj.}
\end{itemize}
\begin{itemize}
\item {Proveniência:(Lat. \textunderscore indisputabilis\textunderscore )}
\end{itemize}
Que não é disputável; que não deve sêr objecto de questão.
Inquestionável; incontestável.
\section{Indisputavelmente}
\begin{itemize}
\item {Grp. gram.:adv.}
\end{itemize}
De modo indisputável.
\section{Indissimulável}
\begin{itemize}
\item {Grp. gram.:adj.}
\end{itemize}
\begin{itemize}
\item {Proveniência:(Lat. \textunderscore indissimulabilis\textunderscore )}
\end{itemize}
Que se não póde dissimular. Cf. Camillo, \textunderscore Mulher Fatal\textunderscore , 79.
\section{Indissolubilidade}
\begin{itemize}
\item {Grp. gram.:f.}
\end{itemize}
Qualidade daquillo que é indissolúvel.
\section{Indissolução}
\begin{itemize}
\item {Grp. gram.:f.}
\end{itemize}
\begin{itemize}
\item {Proveniência:(De \textunderscore in...\textunderscore  + \textunderscore dissolução\textunderscore )}
\end{itemize}
Estado daquillo que não é dissolvido.
\section{Indissolúvel}
\begin{itemize}
\item {Grp. gram.:adj.}
\end{itemize}
\begin{itemize}
\item {Proveniência:(Lat. \textunderscore indissolubilis\textunderscore )}
\end{itemize}
Que não é dissolúvel.
\section{Indissoluvelmente}
\begin{itemize}
\item {Grp. gram.:adv.}
\end{itemize}
De modo indissolúvel.
\section{Indistinção}
\begin{itemize}
\item {Grp. gram.:f.}
\end{itemize}
\begin{itemize}
\item {Proveniência:(De \textunderscore in...\textunderscore  + \textunderscore distinção\textunderscore )}
\end{itemize}
Qualidade do que é indistinto; confusão.
\section{Indistincção}
\begin{itemize}
\item {Grp. gram.:f.}
\end{itemize}
\begin{itemize}
\item {Proveniência:(De \textunderscore in...\textunderscore  + \textunderscore distincção\textunderscore )}
\end{itemize}
Qualidade do que é indistinto; confusão.
\section{Indistincto}
\begin{itemize}
\item {Grp. gram.:adj.}
\end{itemize}
\begin{itemize}
\item {Proveniência:(Lat. \textunderscore indistinctus\textunderscore )}
\end{itemize}
Que se não distingue; indeciso; mal definido; vago; confuso.
\section{Indistinguível}
\begin{itemize}
\item {Grp. gram.:adj.}
\end{itemize}
\begin{itemize}
\item {Proveniência:(De \textunderscore in...\textunderscore  + \textunderscore distinguível\textunderscore )}
\end{itemize}
Que se não póde distinguir; que se não avista bem.
Misturado.
\section{Indistintamente}
\begin{itemize}
\item {Grp. gram.:adv.}
\end{itemize}
De modo indistinto; vagamente; em confusão.
\section{Indistinto}
\begin{itemize}
\item {Grp. gram.:adj.}
\end{itemize}
\begin{itemize}
\item {Proveniência:(Lat. \textunderscore indistinctus\textunderscore )}
\end{itemize}
Que se não distingue; indeciso; mal definido; vago; confuso.
\section{Inditoso}
\begin{itemize}
\item {Grp. gram.:adj.}
\end{itemize}
\begin{itemize}
\item {Proveniência:(De \textunderscore in...\textunderscore  + \textunderscore ditoso\textunderscore )}
\end{itemize}
O mesmo que \textunderscore desditoso\textunderscore .
\section{Indivídua}
\begin{itemize}
\item {Grp. gram.:f.}
\end{itemize}
\begin{itemize}
\item {Utilização:Prov.}
\end{itemize}
\begin{itemize}
\item {Utilização:alg.}
\end{itemize}
\begin{itemize}
\item {Utilização:beir.}
\end{itemize}
\begin{itemize}
\item {Proveniência:(De \textunderscore individuo\textunderscore )}
\end{itemize}
Qualquer mulher.
Mulher, de que se não diz o nome: \textunderscore encontrei hoje uma indivídua, que eu já não via há muito...\textunderscore 
\section{Individuação}
\begin{itemize}
\item {Grp. gram.:f.}
\end{itemize}
Acto ou effeito de individuar.
\section{Individuador}
\begin{itemize}
\item {Grp. gram.:m.  e  adj.}
\end{itemize}
O que individua.
\section{Individual}
\begin{itemize}
\item {Grp. gram.:adj.}
\end{itemize}
Relativo a indivíduo, ou a indivíduos; especial.
\section{Individualidade}
\begin{itemize}
\item {Grp. gram.:f.}
\end{itemize}
\begin{itemize}
\item {Proveniência:(De \textunderscore individual\textunderscore )}
\end{itemize}
Conjunto das qualidades que caracterizam um indivíduo.
Originalidade; personalidade.
\section{Individualismo}
\begin{itemize}
\item {Grp. gram.:m.}
\end{itemize}
\begin{itemize}
\item {Proveniência:(De \textunderscore individual\textunderscore )}
\end{itemize}
Insulamento systemático.
Existência individual.
Theoria, que sustenta a preferência do direito individual ao collectivo.
\section{Individualista}
\begin{itemize}
\item {Grp. gram.:adj.}
\end{itemize}
\begin{itemize}
\item {Grp. gram.:M.}
\end{itemize}
\begin{itemize}
\item {Proveniência:(De \textunderscore individual\textunderscore )}
\end{itemize}
Relativo ao individualismo.
Sectário do individualismo.
\section{Individualização}
\begin{itemize}
\item {Grp. gram.:f.}
\end{itemize}
Acto ou effeito de individualizar.
\section{Individualizar}
\begin{itemize}
\item {Grp. gram.:v. t.}
\end{itemize}
\begin{itemize}
\item {Proveniência:(De \textunderscore individual\textunderscore )}
\end{itemize}
Tornar individual.
Especializar.
\section{Individualmente}
\begin{itemize}
\item {Grp. gram.:adv.}
\end{itemize}
De modo individual.
\section{Individuante}
\begin{itemize}
\item {Grp. gram.:adj.}
\end{itemize}
Que individua.
\section{Individuar}
\begin{itemize}
\item {Grp. gram.:v. t.}
\end{itemize}
\begin{itemize}
\item {Proveniência:(De \textunderscore indivíduo\textunderscore )}
\end{itemize}
Narrar minuciosamente; especificar; individualizar.
\section{Indivíduo}
\begin{itemize}
\item {Grp. gram.:adj.}
\end{itemize}
\begin{itemize}
\item {Grp. gram.:M.}
\end{itemize}
\begin{itemize}
\item {Utilização:Fam.}
\end{itemize}
\begin{itemize}
\item {Utilização:açor}
\end{itemize}
\begin{itemize}
\item {Utilização:Bras}
\end{itemize}
\begin{itemize}
\item {Utilização:deprec.}
\end{itemize}
\begin{itemize}
\item {Proveniência:(Lat. \textunderscore individuus\textunderscore )}
\end{itemize}
Que se não divide; indiviso.
Qualquer corpo, que constitue um todo distinto, em relação á espécie de que faz parte.
Exemplar de uma espécie qualquer, orgânica ou inorgânica.
Pessôa: \textunderscore há indivíduos muito petulantes\textunderscore !
Homem determinado; sujeito.
Homem reles, desprezível, pandilha.
\section{Indivisamente}
\begin{itemize}
\item {Grp. gram.:adv.}
\end{itemize}
De modo indiviso.
\section{Indivisão}
\begin{itemize}
\item {Grp. gram.:f.}
\end{itemize}
\begin{itemize}
\item {Proveniência:(Lat. \textunderscore indivisio\textunderscore )}
\end{itemize}
Qualidade daquillo que é indiviso; falta de divisão.
\section{Indivisibilidade}
\begin{itemize}
\item {Grp. gram.:f.}
\end{itemize}
Qualidade daquillo que é indivisível.
\section{Indivisível}
\begin{itemize}
\item {Grp. gram.:adj.}
\end{itemize}
\begin{itemize}
\item {Grp. gram.:M.}
\end{itemize}
\begin{itemize}
\item {Proveniência:(Lat. \textunderscore indivisibilis\textunderscore )}
\end{itemize}
Que não é divisível; que se não póde dividir.
Partícula tenuíssima; átomo.
\section{Indivisivelmente}
\begin{itemize}
\item {Grp. gram.:adv.}
\end{itemize}
De modo indivisível.
\section{Indivisivo}
\begin{itemize}
\item {Grp. gram.:adj.}
\end{itemize}
\begin{itemize}
\item {Proveniência:(De \textunderscore in...\textunderscore  + \textunderscore divisivo\textunderscore )}
\end{itemize}
O mesmo que \textunderscore indivisível\textunderscore .
\section{Indiviso}
\begin{itemize}
\item {Grp. gram.:adj.}
\end{itemize}
\begin{itemize}
\item {Proveniência:(Lat. \textunderscore indivisus\textunderscore )}
\end{itemize}
Não dividido.
Que pertence cumulativamente a vários indivíduos.
Que possue bens indivisos.
\section{Indizível}
\begin{itemize}
\item {Grp. gram.:adj.}
\end{itemize}
\begin{itemize}
\item {Grp. gram.:M.}
\end{itemize}
\begin{itemize}
\item {Utilização:Pop.}
\end{itemize}
\begin{itemize}
\item {Proveniência:(De \textunderscore in...\textunderscore  + \textunderscore dizível\textunderscore )}
\end{itemize}
Que se não póde dizer; inexplicável.
Ineffável.
Extraordinário.
Coisa muito pequenina; pequenina porção: \textunderscore deitou nas couves um indizível de azeite\textunderscore .
\section{Indizivelmente}
\begin{itemize}
\item {Grp. gram.:adv.}
\end{itemize}
De modo indizível.
\section{Indo...}
Elemento, que entra em palavras compostas, com a significação de \textunderscore relativo á Índia\textunderscore  ou \textunderscore aos Indios\textunderscore .
\section{Indo-africano}
\begin{itemize}
\item {Grp. gram.:adj.}
\end{itemize}
Relativo a Índios e Africanos.
\section{Indo-árabe}
\begin{itemize}
\item {Grp. gram.:adj.}
\end{itemize}
Relativo a Índios e Árabes.
\section{Indobrável}
\begin{itemize}
\item {Grp. gram.:adj.}
\end{itemize}
\begin{itemize}
\item {Proveniência:(De \textunderscore in...\textunderscore  + \textunderscore dobrável\textunderscore )}
\end{itemize}
Que se não verga.
Que se não submete:«\textunderscore ...indobrável a toda a fôrça humana\textunderscore ». Ortigão, \textunderscore Hollanda\textunderscore , 16.
\section{Indo-britânnico}
\begin{itemize}
\item {Grp. gram.:adj.}
\end{itemize}
Relativo á Índia inglesa.
\section{Indo-céltico}
\begin{itemize}
\item {Grp. gram.:adj.}
\end{itemize}
O mesmo que \textunderscore indo-europeu\textunderscore .
\section{Indo-chim}
\begin{itemize}
\item {Grp. gram.:adj.}
\end{itemize}
Relativo a Índios e Chineses.
Relativo á Indo-China.
\section{Indo-china}
\begin{itemize}
\item {Grp. gram.:adj.}
\end{itemize}
Relativo a Índios e Chineses.
Relativo á Indo-China.
\section{Indo-chinês}
\begin{itemize}
\item {Grp. gram.:adj.}
\end{itemize}
Relativo a Índios e Chineses.
Relativo á Indo-China.
\section{Indocibilidade}
\begin{itemize}
\item {Grp. gram.:f.}
\end{itemize}
Qualidade de indocível.
\section{Indócil}
\begin{itemize}
\item {Grp. gram.:adj.}
\end{itemize}
\begin{itemize}
\item {Proveniência:(Lat. \textunderscore indocilis\textunderscore )}
\end{itemize}
Que não é dócil; indomável; incorrigível; insubmisso.
\section{Indocilidade}
\begin{itemize}
\item {Grp. gram.:f.}
\end{itemize}
\begin{itemize}
\item {Proveniência:(Lat. \textunderscore indocilitas\textunderscore )}
\end{itemize}
Qualidade de indócil.
\section{Indocilizar}
\begin{itemize}
\item {Grp. gram.:v. t.}
\end{itemize}
Tornar indócil.
\section{Indocilmente}
\begin{itemize}
\item {Grp. gram.:adv.}
\end{itemize}
\begin{itemize}
\item {Proveniência:(De \textunderscore indócil\textunderscore )}
\end{itemize}
Com indocilidade.
\section{Indocível}
\begin{itemize}
\item {Grp. gram.:adj.}
\end{itemize}
\begin{itemize}
\item {Proveniência:(Do lat. \textunderscore in\textunderscore  + \textunderscore docibilis\textunderscore )}
\end{itemize}
Incapaz de receber ensino.
Estúpido.
\section{Indocumentado}
\begin{itemize}
\item {Grp. gram.:adj.}
\end{itemize}
\begin{itemize}
\item {Proveniência:(De \textunderscore in...\textunderscore  + \textunderscore documentar\textunderscore )}
\end{itemize}
Desacompanhado de documentos.
\section{Indo-europeu}
\begin{itemize}
\item {Grp. gram.:adj.}
\end{itemize}
Relativo á Índia e á Europa.
Diz-se especialmente das línguas orientaes e occidentaes, que, pelos seus radicaes e pela sua grammática, revelam origens communs ou aproximadas.
\section{Indo-germânico}
\begin{itemize}
\item {Grp. gram.:adj.}
\end{itemize}
Que se estende desde a Índia á Germânia.
Chamaram-se indo-germânicas as línguas indo-europeias, quando ainda se não tinha reconhecido que o céltico pertencia á mesma origem.
\section{Indo-hellênico}
\begin{itemize}
\item {Grp. gram.:adj.}
\end{itemize}
Relativo á Índia e á Grécia.
\section{Índole}
\begin{itemize}
\item {Grp. gram.:f.}
\end{itemize}
\begin{itemize}
\item {Proveniência:(Lat. \textunderscore indoles\textunderscore )}
\end{itemize}
Propensão innata; tendência especial; carácter; temperamento.
\section{Indolência}
\begin{itemize}
\item {Grp. gram.:f.}
\end{itemize}
\begin{itemize}
\item {Proveniência:(Lat. \textunderscore indolentia\textunderscore )}
\end{itemize}
Qualidade de indolente.
Insensibilidade phýsica; insensibilidade moral; impassibilidade.
Negligência; preguiça; desmazêlo.
\section{Indolente}
\begin{itemize}
\item {Grp. gram.:adj.}
\end{itemize}
\begin{itemize}
\item {Utilização:Fig.}
\end{itemize}
\begin{itemize}
\item {Proveniência:(Lat. \textunderscore indolens\textunderscore )}
\end{itemize}
Que é insensível á dôr.
Que não faz doer.
Negligente; apáthico; sem actividade; ocioso.
\section{Indolentemente}
\begin{itemize}
\item {Grp. gram.:adv.}
\end{itemize}
De modo indolente.
\section{Indomado}
\begin{itemize}
\item {Grp. gram.:adj.}
\end{itemize}
\begin{itemize}
\item {Proveniência:(De \textunderscore in...\textunderscore  + \textunderscore domado\textunderscore )}
\end{itemize}
Que não é domado; insubmisso; não domesticado.
\section{Indomável}
\begin{itemize}
\item {Grp. gram.:adj.}
\end{itemize}
\begin{itemize}
\item {Proveniência:(Lat. \textunderscore indomabilis\textunderscore )}
\end{itemize}
Que não é domável.
Implacável; inflexível.
Invencível.
\section{Indomavelmente}
\begin{itemize}
\item {Grp. gram.:adv.}
\end{itemize}
De modo indomável.
\section{Indomesticado}
\begin{itemize}
\item {Grp. gram.:adj.}
\end{itemize}
\begin{itemize}
\item {Proveniência:(De \textunderscore in...\textunderscore  + \textunderscore domesticado\textunderscore )}
\end{itemize}
Não domesticado; bravio.
\section{Indomesticável}
\begin{itemize}
\item {Grp. gram.:adj.}
\end{itemize}
\begin{itemize}
\item {Proveniência:(De \textunderscore in...\textunderscore  + \textunderscore domesticável\textunderscore )}
\end{itemize}
Que não é domesticável.
\section{Indoméstico}
\begin{itemize}
\item {Grp. gram.:adj.}
\end{itemize}
O mesmo que \textunderscore indomesticável\textunderscore ; bravio; selvagem.
\section{Indominável}
\begin{itemize}
\item {Grp. gram.:adj.}
\end{itemize}
\begin{itemize}
\item {Proveniência:(De \textunderscore in...\textunderscore  + \textunderscore dominável\textunderscore )}
\end{itemize}
Que se não póde dominar.
Indomável. Cf. Eça, \textunderscore P. Basílio\textunderscore , 365.
\section{Indómito}
\begin{itemize}
\item {Grp. gram.:adj.}
\end{itemize}
\begin{itemize}
\item {Utilização:Fig.}
\end{itemize}
\begin{itemize}
\item {Proveniência:(Lat. \textunderscore indomitus\textunderscore )}
\end{itemize}
Indomado; não vencido.
Arrogante.
\section{Indona}
\begin{itemize}
\item {Grp. gram.:m.}
\end{itemize}
Orifício, que algumas tríbos africanas fazem no lábio superior, para segurar e usar uma rodela.
\section{Indo-persa}
\begin{itemize}
\item {Grp. gram.:adj.}
\end{itemize}
Relativo a Índios e Persas.
\section{Indo-português}
\begin{itemize}
\item {Grp. gram.:adj.}
\end{itemize}
\begin{itemize}
\item {Grp. gram.:M.}
\end{itemize}
Relativo a Portugal e á Índia.
Relativo á Índia portuguesa.
Língua crioula, na Índia portuguesa.
\section{Indo-russo}
\begin{itemize}
\item {Grp. gram.:adj.}
\end{itemize}
Relativo a Índios e Russos.
\section{Indostânico}
\begin{itemize}
\item {Grp. gram.:adj.}
\end{itemize}
Relativo ao Indostão.
\section{Indostano}
\begin{itemize}
\item {Grp. gram.:m.}
\end{itemize}
A língua moderna dos indus.
\section{Indoutamente}
\begin{itemize}
\item {Grp. gram.:adv.}
\end{itemize}
\begin{itemize}
\item {Proveniência:(De \textunderscore indouto\textunderscore )}
\end{itemize}
Com ignorância.
\section{Indouto}
\begin{itemize}
\item {Grp. gram.:adj.}
\end{itemize}
\begin{itemize}
\item {Proveniência:(Lat. \textunderscore indoctus\textunderscore )}
\end{itemize}
Que não é douto; que tem pouca instrucção ou nenhuma.
Inepto.
\section{Indri}
\begin{itemize}
\item {Grp. gram.:m.}
\end{itemize}
Grande lêmur de Madagáscar.
\section{Indu}
\begin{itemize}
\item {Grp. gram.:adj.}
\end{itemize}
\begin{itemize}
\item {Grp. gram.:M.  e  f.}
\end{itemize}
Relativo ao Indostão ou aos seus habitantes; indiano.
Pessôa, que é indígena da Índia e gentia: \textunderscore há índios christãos, moiros e indus\textunderscore .
\section{Indua}
\begin{itemize}
\item {Grp. gram.:f.  e  adj.}
\end{itemize}
(fem. de \textunderscore indu\textunderscore )
\section{Indua}
\begin{itemize}
\item {Grp. gram.:f.}
\end{itemize}
Planta africana.
Bebida venenosa, feita com a casca dessa planta.
\section{Indubitado}
\begin{itemize}
\item {Grp. gram.:adj.}
\end{itemize}
\begin{itemize}
\item {Proveniência:(Lat. \textunderscore indubitatus\textunderscore )}
\end{itemize}
Incontestável; sôbre que não há dúvida.
\section{Indubitável}
\begin{itemize}
\item {Grp. gram.:adj.}
\end{itemize}
\begin{itemize}
\item {Proveniência:(Lat. \textunderscore indubitabilis\textunderscore )}
\end{itemize}
Que não offerece dúvida; incontestado, certo.
\section{Indubitavelmente}
\begin{itemize}
\item {Grp. gram.:adv.}
\end{itemize}
De modo indubitável; incontestavelmente; com certeza.
\section{Indução}
\begin{itemize}
\item {Grp. gram.:f.}
\end{itemize}
\begin{itemize}
\item {Proveniência:(Lat. \textunderscore inductio\textunderscore )}
\end{itemize}
Acto ou efeito de induzir.
Sugestão.
Raciocínio, em que, de factos particulares, se tira uma conclusão genérica.
Conclusão.
Acto de estabelecer uma corrente eléctrica, produzindo corrente inversa num circuito próximo.
Acto de interromper uma corrente eléctrica, produzindo corrente semelhante num circuito próximo.
\section{Induças}
\begin{itemize}
\item {Grp. gram.:f. pl.}
\end{itemize}
\begin{itemize}
\item {Utilização:Des.}
\end{itemize}
(V.indúcias)
\section{Inducção}
\begin{itemize}
\item {Grp. gram.:f.}
\end{itemize}
\begin{itemize}
\item {Proveniência:(Lat. \textunderscore inductio\textunderscore )}
\end{itemize}
Acto ou effeito de induzir.
Suggestão.
Raciocínio, em que, de factos particulares, se tira uma conclusão genérica.
Conclusão.
Acto de estabelecer uma corrente eléctrica, produzindo corrente inversa num circuito próximo.
Acto de interromper uma corrente eléctrica, produzindo corrente semelhante num circuito próximo.
\section{Indúcias}
\begin{itemize}
\item {Grp. gram.:f. pl.}
\end{itemize}
\begin{itemize}
\item {Utilização:Jur.}
\end{itemize}
\begin{itemize}
\item {Proveniência:(Lat. \textunderscore indutiae\textunderscore )}
\end{itemize}
Tréguas.
Dilação.
Esperas, que os credores de um negociante concedem por concordata ao seu devedor, além do vencimento dos respectivos créditos. Cf. F. Borges, \textunderscore Diccion. Jur.\textunderscore 
\section{Indúctil}
\begin{itemize}
\item {Grp. gram.:adj.}
\end{itemize}
\begin{itemize}
\item {Proveniência:(De \textunderscore in...\textunderscore  + \textunderscore dúctil\textunderscore )}
\end{itemize}
Que não é dúctil.
\section{Inductilidade}
\begin{itemize}
\item {Grp. gram.:f.}
\end{itemize}
\begin{itemize}
\item {Proveniência:(De \textunderscore in...\textunderscore  + \textunderscore ductilidade\textunderscore )}
\end{itemize}
Falta de ductilidade.
\section{Inductivo}
\begin{itemize}
\item {Grp. gram.:adj.}
\end{itemize}
\begin{itemize}
\item {Proveniência:(Lat. \textunderscore inductivus\textunderscore )}
\end{itemize}
Que induz; que procede por inducção.
\section{Inductor}
\begin{itemize}
\item {Grp. gram.:adj.}
\end{itemize}
\begin{itemize}
\item {Utilização:Phýs.}
\end{itemize}
\begin{itemize}
\item {Grp. gram.:M.}
\end{itemize}
\begin{itemize}
\item {Utilização:Phýs.}
\end{itemize}
\begin{itemize}
\item {Utilização:Anat.}
\end{itemize}
\begin{itemize}
\item {Proveniência:(Lat. \textunderscore inductor\textunderscore )}
\end{itemize}
Que induz; que instiga; que suggere.
Que produz a inducção.
Aquelle que induz.
Circuito, que produz a inducção.
Músculo, que se contrai e sôbre o qual assenta o nervo motor de outro músculo.
\section{Induísmo}
\begin{itemize}
\item {Grp. gram.:m.}
\end{itemize}
Organização religiosa e social, que fundiu num só corpo ou systema as antigas religiões e sociedades indianas.
O mesmo que \textunderscore neo-brahmanismo\textunderscore . Cf. Camillo, \textunderscore Corja\textunderscore , 89.
\section{Indulgência}
\begin{itemize}
\item {Grp. gram.:f.}
\end{itemize}
\begin{itemize}
\item {Proveniência:(Lat. \textunderscore indulgentia\textunderscore )}
\end{itemize}
Qualidade de quem é indulgente; clemência.
Perdão; remissão das penas relativas aos peccados.
\section{Indulgenciar}
\begin{itemize}
\item {Grp. gram.:v. t.}
\end{itemize}
\begin{itemize}
\item {Proveniência:(De \textunderscore indulgência\textunderscore )}
\end{itemize}
Tratar indulgentemente; perdoar.
\section{Indulgente}
\begin{itemize}
\item {Grp. gram.:f.}
\end{itemize}
\begin{itemize}
\item {Proveniência:(Lat. \textunderscore indulgens\textunderscore )}
\end{itemize}
Que tem disposição para desculpar ou perdoar.
Clemente; benígno, tolerante; que perdôa facilmente.
\section{Indulgentemente}
\begin{itemize}
\item {Grp. gram.:adv.}
\end{itemize}
De modo indulgente.
\section{Indulina}
\begin{itemize}
\item {Grp. gram.:f.}
\end{itemize}
\begin{itemize}
\item {Utilização:Chím.}
\end{itemize}
\begin{itemize}
\item {Proveniência:(De \textunderscore indigo\textunderscore  + \textunderscore anilina\textunderscore )}
\end{itemize}
Matéria còrante, que se obtém do chlorhydrato de anilina com nitritos.
\section{Indultado}
\begin{itemize}
\item {Grp. gram.:m.}
\end{itemize}
\begin{itemize}
\item {Proveniência:(De \textunderscore indultar\textunderscore )}
\end{itemize}
Indivíduo, que teve indulto ou perdão de uma pena, ou a quem se attenuou a pena ou castigo.
\section{Indultar}
\begin{itemize}
\item {Grp. gram.:v. t.}
\end{itemize}
Dar indulto a; perdoar.
Attenuar a pena, que foi imposta a.
\section{Indútil}
\begin{itemize}
\item {Grp. gram.:adj.}
\end{itemize}
\begin{itemize}
\item {Proveniência:(De \textunderscore in...\textunderscore  + \textunderscore dútil\textunderscore )}
\end{itemize}
Que não é dútil.
\section{Indutilidade}
\begin{itemize}
\item {Grp. gram.:f.}
\end{itemize}
\begin{itemize}
\item {Proveniência:(De \textunderscore in...\textunderscore  + \textunderscore dutilidade\textunderscore )}
\end{itemize}
Falta de dutilidade.
\section{Indutivo}
\begin{itemize}
\item {Grp. gram.:adj.}
\end{itemize}
\begin{itemize}
\item {Proveniência:(Lat. \textunderscore inductivus\textunderscore )}
\end{itemize}
Que induz; que procede por indução.
\section{Indutor}
\begin{itemize}
\item {Grp. gram.:adj.}
\end{itemize}
\begin{itemize}
\item {Utilização:Phýs.}
\end{itemize}
\begin{itemize}
\item {Grp. gram.:M.}
\end{itemize}
\begin{itemize}
\item {Utilização:Phýs.}
\end{itemize}
\begin{itemize}
\item {Utilização:Anat.}
\end{itemize}
\begin{itemize}
\item {Proveniência:(Lat. \textunderscore inductor\textunderscore )}
\end{itemize}
Que induz; que instiga; que sugere.
Que produz a indução.
Aquele que induz.
Circuito, que produz a indução.
Músculo, que se contrai e sôbre o qual assenta o nervo motor de outro músculo.
\section{Indultário}
\begin{itemize}
\item {Grp. gram.:adj.}
\end{itemize}
Que goza do indulto.
\section{Indulto}
\begin{itemize}
\item {Grp. gram.:m.}
\end{itemize}
\begin{itemize}
\item {Proveniência:(Lat. \textunderscore indultum\textunderscore )}
\end{itemize}
Reducção ou commutação de pena; perdão.
Privilégio.
Concessão de uma graça.
\section{Indumentaria}
\begin{itemize}
\item {Grp. gram.:f.}
\end{itemize}
\begin{itemize}
\item {Proveniência:(De \textunderscore indumento\textunderscore )}
\end{itemize}
Arte do vestuário.
História do vestuário.
Systema do vestuário, em relação a certas épocas ou povos.
\section{Indumentário}
\begin{itemize}
\item {Grp. gram.:adj.}
\end{itemize}
\begin{itemize}
\item {Proveniência:(De \textunderscore indumento\textunderscore )}
\end{itemize}
Relativo a vestuário.
\section{Indumento}
\begin{itemize}
\item {Grp. gram.:m.}
\end{itemize}
\begin{itemize}
\item {Proveniência:(Lat. \textunderscore indumentum\textunderscore )}
\end{itemize}
Vestuário.
Revestimento.
Epiderme vegetal.
\section{Induna}
\begin{itemize}
\item {Grp. gram.:m.}
\end{itemize}
\begin{itemize}
\item {Utilização:T. da África port}
\end{itemize}
Cada um dos homens principaes do séquito do régulo.
Espécie de ministro.
\section{Induração}
\begin{itemize}
\item {Grp. gram.:f.}
\end{itemize}
\begin{itemize}
\item {Utilização:Fig.}
\end{itemize}
\begin{itemize}
\item {Proveniência:(Lat. \textunderscore induratio\textunderscore )}
\end{itemize}
Acto de endurecer; endurecimento de tecidos orgânicos.
Contumácia no mal.
\section{Indurado}
\begin{itemize}
\item {Grp. gram.:adj.}
\end{itemize}
\begin{itemize}
\item {Utilização:Fig.}
\end{itemize}
\begin{itemize}
\item {Proveniência:(Lat. \textunderscore induratus\textunderscore )}
\end{itemize}
O mesmo que [[endurecido|endurecer]].
Contumaz.
\section{Indúsia}
\begin{itemize}
\item {Grp. gram.:f.}
\end{itemize}
\begin{itemize}
\item {Utilização:Bot.}
\end{itemize}
Órgão, que envolve os esporos das cryptogâmicas.
(Cp. \textunderscore indúsio\textunderscore )
\section{Indúsio}
\begin{itemize}
\item {Grp. gram.:m.}
\end{itemize}
\begin{itemize}
\item {Proveniência:(Lat. \textunderscore indusium\textunderscore )}
\end{itemize}
Túnica, que as damas romanas usavam por baixo dos vestidos e lhes servia de camisa.
\section{Industre}
\begin{itemize}
\item {Grp. gram.:adj.}
\end{itemize}
\begin{itemize}
\item {Utilização:Des.}
\end{itemize}
Produzido pela indústria.
\section{Indústria}
\begin{itemize}
\item {Grp. gram.:f.}
\end{itemize}
\begin{itemize}
\item {Utilização:Fig.}
\end{itemize}
\begin{itemize}
\item {Grp. gram.:Loc. adv.}
\end{itemize}
\begin{itemize}
\item {Proveniência:(Lat. \textunderscore industria\textunderscore )}
\end{itemize}
Aptidão, destreza, com que se executa um trabalho manual.
Habilidade, para fazer alguma coisa.
Arte, offício.
Profissão mecânica ou mercantil.
Conjunto dos trabalhos, de que deriva a producção das riquezas.
Conjunto das artes industriaes, exceptuando a agricultura.
Invenção; engenho; astúcia.
\textunderscore De indústria\textunderscore , propositadamente, adrede:«\textunderscore muttos de indústria se cohibem...\textunderscore »Camillo, \textunderscore Filha do Reg.\textunderscore 
\section{Industriador}
\begin{itemize}
\item {Grp. gram.:adj.}
\end{itemize}
\begin{itemize}
\item {Grp. gram.:M.}
\end{itemize}
Que industría.
Aquelle que industría.
\section{Industrial}
\begin{itemize}
\item {Grp. gram.:adj.}
\end{itemize}
\begin{itemize}
\item {Grp. gram.:M.}
\end{itemize}
Relativo á industría: \textunderscore associação industrial\textunderscore .
Pessôa, que exerce uma indústria: \textunderscore os industriaes formaram uma associação\textunderscore .
\section{Industrialismo}
\begin{itemize}
\item {Grp. gram.:m.}
\end{itemize}
\begin{itemize}
\item {Proveniência:(De \textunderscore industrial\textunderscore )}
\end{itemize}
Predomínio da indústria sôbre as outras espheras da actividade humana.
Systema, em que se considera a indústria como principal fim da sociedade.
Gôsto pela indústria.
\section{Industrialista}
\begin{itemize}
\item {Grp. gram.:adj.}
\end{itemize}
\begin{itemize}
\item {Grp. gram.:M.}
\end{itemize}
\begin{itemize}
\item {Proveniência:(De \textunderscore industrial\textunderscore )}
\end{itemize}
Relativo ao industrialismo.
Partidário do industrialismo.
\section{Industrialização}
\begin{itemize}
\item {Grp. gram.:f.}
\end{itemize}
Acto ou effeito de industrializar.
\section{Industrializar}
\begin{itemize}
\item {Grp. gram.:v. t.}
\end{itemize}
Tornar industrial; dar carácter industrial a. Cf. Ans. de Andrade, \textunderscore A Terra\textunderscore .
\section{Industrialmente}
\begin{itemize}
\item {Grp. gram.:adv.}
\end{itemize}
De modo industrial; sob o ponto de vista industrial.
\section{Industriar}
\begin{itemize}
\item {Grp. gram.:v. t.}
\end{itemize}
\begin{itemize}
\item {Proveniência:(De \textunderscore indústria\textunderscore )}
\end{itemize}
Tornar hábil, tornar destro.
Ensinar.
Dispor os meios de obter.
Tornar lucrativo ou rendoso, por meio da indústria.
\section{Indústrio}
\begin{itemize}
\item {Grp. gram.:adj.}
\end{itemize}
\begin{itemize}
\item {Utilização:Ant.}
\end{itemize}
\begin{itemize}
\item {Proveniência:(Lat. \textunderscore industrius\textunderscore )}
\end{itemize}
O mesmo que \textunderscore industrioso\textunderscore .
\section{Industriosamente}
\begin{itemize}
\item {Grp. gram.:adv.}
\end{itemize}
De modo industrioso, habilmente; astutamente.
\section{Industrioso}
\begin{itemize}
\item {Grp. gram.:adj.}
\end{itemize}
\begin{itemize}
\item {Proveniência:(Lat. \textunderscore industriosus\textunderscore )}
\end{itemize}
Que exerce indústria; laborioso.
Executado com arte.
Habil; astuto.
\section{Indutar}
\begin{itemize}
\item {Grp. gram.:v. t.}
\end{itemize}
\begin{itemize}
\item {Proveniência:(De \textunderscore induto\textunderscore )}
\end{itemize}
Cobrir, revestir.
Guarnecer.
\section{Induto}
\begin{itemize}
\item {Grp. gram.:m.}
\end{itemize}
\begin{itemize}
\item {Proveniência:(Lat. \textunderscore indutus\textunderscore )}
\end{itemize}
O mesmo que \textunderscore indumento\textunderscore .
\section{Indúvia}
\begin{itemize}
\item {Grp. gram.:f.}
\end{itemize}
\begin{itemize}
\item {Proveniência:(Lat. \textunderscore induviae\textunderscore )}
\end{itemize}
O mesmo que \textunderscore indumento\textunderscore .
Parte do invólucro floral, que se conserva depois de desabrochar a flôr e acompanha o desenvolvimento do fruto.
\section{Induviado}
\begin{itemize}
\item {Grp. gram.:adj.}
\end{itemize}
\begin{itemize}
\item {Utilização:Bot.}
\end{itemize}
Que mantém as indúvias.
\section{Induvial}
\begin{itemize}
\item {Grp. gram.:adj.}
\end{itemize}
Relativo a indúvias.
\section{Induzido}
\begin{itemize}
\item {Grp. gram.:m.}
\end{itemize}
\begin{itemize}
\item {Utilização:Phýs.}
\end{itemize}
\begin{itemize}
\item {Proveniência:(De \textunderscore induzir\textunderscore )}
\end{itemize}
Parte da bobina, por onde passa a corrente eléctrica.
\section{Induzidor}
\begin{itemize}
\item {Grp. gram.:m.  e  adj.}
\end{itemize}
O que induz.
\section{Induzimento}
\begin{itemize}
\item {Grp. gram.:m.}
\end{itemize}
Acto ou effeito de induzir.
\section{Induzir}
\begin{itemize}
\item {Grp. gram.:v. t.}
\end{itemize}
\begin{itemize}
\item {Proveniência:(Lat. \textunderscore inducere\textunderscore )}
\end{itemize}
Instigar á prática de alguma coisa.
Suggerir alguma coisa a.
Persuadir ardilosamente.
Fazer errar.
Incutir.
Deduzir, inferir.
\section{Inebriar}
\textunderscore v. t.\textunderscore  (e der.)
(V. \textunderscore enebriar\textunderscore , etc.)
\section{Inédia}
\begin{itemize}
\item {Grp. gram.:f.}
\end{itemize}
\begin{itemize}
\item {Proveniência:(Lat. \textunderscore inedia\textunderscore )}
\end{itemize}
Abstinência de alimento.
Espaço de tempo, em que há abstinência de alimento.
\section{Inédito}
\begin{itemize}
\item {Grp. gram.:adj.}
\end{itemize}
\begin{itemize}
\item {Utilização:Fig.}
\end{itemize}
\begin{itemize}
\item {Grp. gram.:M.}
\end{itemize}
\begin{itemize}
\item {Proveniência:(Lat. \textunderscore ineditus\textunderscore )}
\end{itemize}
Que não foi publicado ou promulgado.
Que ainda não foi impresso: \textunderscore obra inédita\textunderscore .
Nunca visto; original.
Obra, que ainda não foi publicada ou impressa.
\section{Inefabilidade}
\begin{itemize}
\item {Grp. gram.:f.}
\end{itemize}
\begin{itemize}
\item {Proveniência:(Lat. \textunderscore ineffabilitas\textunderscore )}
\end{itemize}
Qualidade de inefável.
\section{Inefável}
\begin{itemize}
\item {Grp. gram.:adj.}
\end{itemize}
\begin{itemize}
\item {Utilização:Fig.}
\end{itemize}
\begin{itemize}
\item {Proveniência:(Lat. \textunderscore ineffabilis\textunderscore )}
\end{itemize}
Que se não póde exprimir falando; indizível.
Encantador.
\section{Inefavelmente}
\begin{itemize}
\item {Grp. gram.:adv.}
\end{itemize}
De modo inefável.
\section{Ineffabilidade}
\begin{itemize}
\item {Grp. gram.:f.}
\end{itemize}
\begin{itemize}
\item {Proveniência:(Lat. \textunderscore ineffabilitas\textunderscore )}
\end{itemize}
Qualidade de ineffável.
\section{Ineffável}
\begin{itemize}
\item {Grp. gram.:adj.}
\end{itemize}
\begin{itemize}
\item {Utilização:Fig.}
\end{itemize}
\begin{itemize}
\item {Proveniência:(Lat. \textunderscore ineffabilis\textunderscore )}
\end{itemize}
Que se não póde exprimir falando; indizível.
Encantador.
\section{Ineffavelmente}
\begin{itemize}
\item {Grp. gram.:adv.}
\end{itemize}
De modo ineffável.
\section{Inefficácia}
\begin{itemize}
\item {Grp. gram.:f.}
\end{itemize}
\begin{itemize}
\item {Proveniência:(Lat. \textunderscore inefficacia\textunderscore )}
\end{itemize}
Qualidade daquillo que é inefficaz; falta de efficácia.
\section{Inefficaz}
\begin{itemize}
\item {Grp. gram.:adj.}
\end{itemize}
\begin{itemize}
\item {Proveniência:(Lat. \textunderscore inefficax\textunderscore )}
\end{itemize}
Que não é efficaz; que não dá resultado; que é inútil.
Inconveniente.
\section{Inefficazmente}
\begin{itemize}
\item {Grp. gram.:adv.}
\end{itemize}
De modo inefficaz.
\section{Ineficácia}
\begin{itemize}
\item {Grp. gram.:f.}
\end{itemize}
\begin{itemize}
\item {Proveniência:(Lat. \textunderscore inefficacia\textunderscore )}
\end{itemize}
Qualidade daquilo que é ineficaz; falta de eficácia.
\section{Ineficaz}
\begin{itemize}
\item {Grp. gram.:adj.}
\end{itemize}
\begin{itemize}
\item {Proveniência:(Lat. \textunderscore inefficax\textunderscore )}
\end{itemize}
Que não é eficaz; que não dá resultado; que é inútil.
Inconveniente.
\section{Ineficazmente}
\begin{itemize}
\item {Grp. gram.:adv.}
\end{itemize}
De modo ineficaz.
\section{Ineixa}
\begin{itemize}
\item {Grp. gram.:f.}
\end{itemize}
Planta crucífera, (\textunderscore sinapis incana\textunderscore , Lin.).
\section{Inelegância}
\begin{itemize}
\item {Grp. gram.:f.}
\end{itemize}
\begin{itemize}
\item {Proveniência:(Lat. \textunderscore inelegante\textunderscore )}
\end{itemize}
Qualidade daquelle ou daquillo que é inelegante; falta de elegância.
\section{Inelegante}
\begin{itemize}
\item {Grp. gram.:adj.}
\end{itemize}
\begin{itemize}
\item {Proveniência:(Lat. \textunderscore inelegans\textunderscore )}
\end{itemize}
Que não é elegante; que não tem distincção.
\section{Inelegibilidade}
\begin{itemize}
\item {Grp. gram.:f.}
\end{itemize}
Qualidade de inelegível.
\section{Inelegível}
\begin{itemize}
\item {Grp. gram.:adj.}
\end{itemize}
\begin{itemize}
\item {Proveniência:(De \textunderscore in...\textunderscore  + \textunderscore elegível\textunderscore )}
\end{itemize}
Que não é elegível.
\section{Inelidível}
\begin{itemize}
\item {Grp. gram.:adj.}
\end{itemize}
\begin{itemize}
\item {Proveniência:(De \textunderscore in...\textunderscore  + \textunderscore elidível\textunderscore )}
\end{itemize}
Que se não póde elidir; incontestável.
\section{Ineloquente}
\begin{itemize}
\item {fónica:cu-en}
\end{itemize}
\begin{itemize}
\item {Grp. gram.:adj.}
\end{itemize}
Não eloquente. Cf. Garrett, \textunderscore Catão\textunderscore , 100.
\section{Ineluctável}
\begin{itemize}
\item {Grp. gram.:adj.}
\end{itemize}
\begin{itemize}
\item {Proveniência:(Lat. \textunderscore ineluctabilis\textunderscore )}
\end{itemize}
Que se não póde evitar; com que se luta debalde.
Indiscutível.
\section{Ineluctavelmente}
\begin{itemize}
\item {Grp. gram.:adv.}
\end{itemize}
De modo inelutável.
\section{Inelutável}
\begin{itemize}
\item {Grp. gram.:adj.}
\end{itemize}
\begin{itemize}
\item {Proveniência:(Lat. \textunderscore ineluctabilis\textunderscore )}
\end{itemize}
Que se não póde evitar; com que se luta debalde.
Indiscutível.
\section{Inelutavelmente}
\begin{itemize}
\item {Grp. gram.:adv.}
\end{itemize}
De modo inelutável.
\section{Inembrionado}
\begin{itemize}
\item {Grp. gram.:adj.}
\end{itemize}
\begin{itemize}
\item {Proveniência:(De \textunderscore in...\textunderscore  + \textunderscore embrionado\textunderscore )}
\end{itemize}
Que não tem embrião.
\section{Inembryonado}
\begin{itemize}
\item {Grp. gram.:adj.}
\end{itemize}
\begin{itemize}
\item {Proveniência:(De \textunderscore in...\textunderscore  + \textunderscore embryonado\textunderscore )}
\end{itemize}
Que não tem embryão.
\section{Inenarrável}
\begin{itemize}
\item {Grp. gram.:adj.}
\end{itemize}
\begin{itemize}
\item {Proveniência:(Lat. \textunderscore inenarrabilis\textunderscore )}
\end{itemize}
O mesmo ou melhor que \textunderscore innarrável\textunderscore . Cf. Bernárdez, \textunderscore Luz e Calor\textunderscore , 527.
\section{Inépcia}
\begin{itemize}
\item {Grp. gram.:f.}
\end{itemize}
\begin{itemize}
\item {Proveniência:(Lat. \textunderscore ineptiae\textunderscore )}
\end{itemize}
Falta de aptidão; idiotismo; escassez de intelligência.
Acto ou dito absurdo.
\section{Ineptamente}
\begin{itemize}
\item {Grp. gram.:adv.}
\end{itemize}
De modo inepto.
\section{Ineptidão}
\begin{itemize}
\item {Grp. gram.:f.}
\end{itemize}
\begin{itemize}
\item {Proveniência:(Lat. \textunderscore ineptitudo\textunderscore )}
\end{itemize}
O mesmo que \textunderscore inépcia\textunderscore .
\section{Inepto}
\begin{itemize}
\item {Grp. gram.:adj.}
\end{itemize}
\begin{itemize}
\item {Proveniência:(Lat. \textunderscore ineptus\textunderscore )}
\end{itemize}
Que não é apto; que não tem habilidade.
Que revela toleima ou absurdo; que não é intelligente.
\section{Inequalidade}
\begin{itemize}
\item {Grp. gram.:f.}
\end{itemize}
\begin{itemize}
\item {Utilização:Des.}
\end{itemize}
\begin{itemize}
\item {Proveniência:(Do lat. \textunderscore inaequabilitas\textunderscore )}
\end{itemize}
Desigualdade.
Dissemelhança.
\section{Inequiângulo}
\begin{itemize}
\item {fónica:cu-i}
\end{itemize}
\begin{itemize}
\item {Grp. gram.:adj.}
\end{itemize}
\begin{itemize}
\item {Utilização:Geom.}
\end{itemize}
\begin{itemize}
\item {Proveniência:(De \textunderscore in...\textunderscore  + \textunderscore equiângulo\textunderscore )}
\end{itemize}
Cujos ângulos não são iguaes entre si.
\section{Inequilateral}
\begin{itemize}
\item {fónica:cu-i}
\end{itemize}
\begin{itemize}
\item {Grp. gram.:adj.}
\end{itemize}
\begin{itemize}
\item {Proveniência:(De \textunderscore in...\textunderscore  + \textunderscore equilateral\textunderscore )}
\end{itemize}
Que não é equilateral.
\section{Inequivalve}
\begin{itemize}
\item {fónica:cu-i}
\end{itemize}
\begin{itemize}
\item {Grp. gram.:adj.}
\end{itemize}
\begin{itemize}
\item {Proveniência:(De \textunderscore in...\textunderscore  + \textunderscore equivalve\textunderscore )}
\end{itemize}
Que não tem valvas iguaes.
\section{Inequívoco}
\begin{itemize}
\item {Grp. gram.:adj.}
\end{itemize}
\begin{itemize}
\item {Proveniência:(De \textunderscore in...\textunderscore  + \textunderscore equivoco\textunderscore )}
\end{itemize}
Que não é duvidoso.
Em que não há equívoco; evidente.
\section{Inércia}
\begin{itemize}
\item {Grp. gram.:f.}
\end{itemize}
\begin{itemize}
\item {Utilização:Des.}
\end{itemize}
\begin{itemize}
\item {Proveniência:(Lat. \textunderscore inertia\textunderscore )}
\end{itemize}
Inacção.
Preguiça; indolência.
Torpor.
Propriedade dos corpos, que não pódem modificar por si o seu estado de repoiso ou movimento.
Ignorância de qualquer arte.
Incapacidade.
\section{Inerciar}
\begin{itemize}
\item {Grp. gram.:v. t.}
\end{itemize}
Communicar inércia a; tornar inerte.
\section{Inerme}
\begin{itemize}
\item {Grp. gram.:adj.}
\end{itemize}
\begin{itemize}
\item {Proveniência:(Lat. \textunderscore inermis\textunderscore )}
\end{itemize}
Que não está armado; que não tem meios de defesa.
\section{Inerrância}
\begin{itemize}
\item {Grp. gram.:f.}
\end{itemize}
Qualidade de inerrante.
\section{Inerrante}
\begin{itemize}
\item {Grp. gram.:adj.}
\end{itemize}
\begin{itemize}
\item {Proveniência:(Lat. \textunderscore inerrans\textunderscore )}
\end{itemize}
Que não póde errar.
Que é fixo, não errante.
\section{Inerte}
\begin{itemize}
\item {Grp. gram.:adj.}
\end{itemize}
\begin{itemize}
\item {Proveniência:(Lat. \textunderscore iners\textunderscore )}
\end{itemize}
Que tem inércia; que não é dotado de actividade.
Que produz inércia.
\section{Inescado}
\begin{itemize}
\item {Grp. gram.:adj.}
\end{itemize}
\begin{itemize}
\item {Utilização:Ant.}
\end{itemize}
Engodado, seduzido.
(Cp. lat. \textunderscore esca\textunderscore )
\section{Inescrito}
\begin{itemize}
\item {Grp. gram.:adj.}
\end{itemize}
Não escrito:«\textunderscore ...sentimentos inescritos\textunderscore ». J. Ribeiro, \textunderscore Estética\textunderscore .
\section{Inescrutabilidade}
\begin{itemize}
\item {Grp. gram.:f.}
\end{itemize}
Qualidade daquillo que é inescrutável.
\section{Inescrutável}
\begin{itemize}
\item {Grp. gram.:adj.}
\end{itemize}
\begin{itemize}
\item {Proveniência:(De \textunderscore in...\textunderscore  + \textunderscore escrutar\textunderscore )}
\end{itemize}
Que se não póde escrutar.
\section{Inescurecível}
\begin{itemize}
\item {Grp. gram.:adj.}
\end{itemize}
\begin{itemize}
\item {Proveniência:(De \textunderscore in...\textunderscore  + \textunderscore escurecível\textunderscore )}
\end{itemize}
Que não póde sêr escurecido ou deslembrado.
Digno de memória; memorável:«\textunderscore luz inescurecível dêste século demolidor.\textunderscore »Camillo, \textunderscore Bohemia\textunderscore , 371.
\section{Inês-dorta}
\begin{itemize}
\item {Grp. gram.:m.  e  f.}
\end{itemize}
\begin{itemize}
\item {Utilização:Pop.}
\end{itemize}
Pessôa insignificante, que não merece attenção, que é excessivamente ingênua ou atoleimada. Cf. Filinto, VIII, 226.
\section{Inesgotável}
\begin{itemize}
\item {Grp. gram.:adj.}
\end{itemize}
\begin{itemize}
\item {Proveniência:(De \textunderscore in...\textunderscore  + \textunderscore esgotável\textunderscore )}
\end{itemize}
Que se não póde esgotar; abundantíssimo.
\section{Inesperadamente}
\begin{itemize}
\item {Grp. gram.:adv.}
\end{itemize}
De modo inesperado; de súbito; imprevistamente.
\section{Inesperado}
\begin{itemize}
\item {Grp. gram.:adj.}
\end{itemize}
\begin{itemize}
\item {Proveniência:(De \textunderscore in...\textunderscore  + \textunderscore esperado\textunderscore )}
\end{itemize}
Não esperado; imprevisto; repentino.
\section{Inestancável}
\begin{itemize}
\item {Grp. gram.:adj.}
\end{itemize}
\begin{itemize}
\item {Proveniência:(De \textunderscore in...\textunderscore  + \textunderscore estancável\textunderscore )}
\end{itemize}
Que se não póde estancar.
\section{Inestendível}
\begin{itemize}
\item {Grp. gram.:adj.}
\end{itemize}
\begin{itemize}
\item {Proveniência:(De \textunderscore in...\textunderscore  + \textunderscore estender\textunderscore )}
\end{itemize}
Que se não póde estender.
\section{Inesthético}
\begin{itemize}
\item {Grp. gram.:adj.}
\end{itemize}
\begin{itemize}
\item {Proveniência:(De \textunderscore in...\textunderscore  + \textunderscore esthético\textunderscore )}
\end{itemize}
Contrário á esthética, á árte, ao bom gôsto.
\section{Inestimável}
\begin{itemize}
\item {Grp. gram.:adj.}
\end{itemize}
\begin{itemize}
\item {Proveniência:(Lat. \textunderscore inaestimabilis\textunderscore )}
\end{itemize}
Que se não póde avaliar; inapreciável.
Que tem um valôr enorme, incalculável.
\section{Inevidência}
\begin{itemize}
\item {Grp. gram.:f.}
\end{itemize}
Falta de evidência; qualidade de inevidente.
\section{Inevidente}
\begin{itemize}
\item {Grp. gram.:adj.}
\end{itemize}
\begin{itemize}
\item {Proveniência:(Lat. \textunderscore inevidens\textunderscore )}
\end{itemize}
Que não é evidente.
\section{Inevitável}
\begin{itemize}
\item {Grp. gram.:adj.}
\end{itemize}
\begin{itemize}
\item {Proveniência:(Lat. \textunderscore inevitabilis\textunderscore )}
\end{itemize}
Que se não póde evitar; fatal.
\section{Inevitavelmente}
\begin{itemize}
\item {Grp. gram.:adv.}
\end{itemize}
De modo inevitável.
\section{Inexacção}
\begin{itemize}
\item {Grp. gram.:f.}
\end{itemize}
(V.inexactidão)
\section{Inexactamente}
\begin{itemize}
\item {Grp. gram.:adv.}
\end{itemize}
De modo inexacto; sem exactidão; erradamente.
\section{Inexactidão}
\begin{itemize}
\item {Grp. gram.:f.}
\end{itemize}
\begin{itemize}
\item {Proveniência:(De \textunderscore in...\textunderscore  + \textunderscore exactidão\textunderscore )}
\end{itemize}
Qualidade daquillo que é inexacto; falta de exactidão; coisa inexacta.
\section{Inexacto}
\begin{itemize}
\item {Grp. gram.:adj.}
\end{itemize}
\begin{itemize}
\item {Proveniência:(De \textunderscore in...\textunderscore  + \textunderscore exacto\textunderscore )}
\end{itemize}
Que não é exacto; em que há êrro.
\section{Inexaminável}
\begin{itemize}
\item {Grp. gram.:adj.}
\end{itemize}
\begin{itemize}
\item {Proveniência:(De \textunderscore in...\textunderscore  + \textunderscore examinável\textunderscore )}
\end{itemize}
Que se não póde examinar.
\section{Inexaurível}
\begin{itemize}
\item {Grp. gram.:adj.}
\end{itemize}
\begin{itemize}
\item {Proveniência:(De \textunderscore in...\textunderscore  + \textunderscore exhaurível\textunderscore )}
\end{itemize}
Que não é exaurível; inesgotável; copiosíssimo.
\section{Inexausto}
\begin{itemize}
\item {Grp. gram.:adj.}
\end{itemize}
\begin{itemize}
\item {Proveniência:(Lat. \textunderscore inexhaustus\textunderscore )}
\end{itemize}
Que não está exausto.
\section{Inexcedível}
\begin{itemize}
\item {Grp. gram.:adj.}
\end{itemize}
\begin{itemize}
\item {Proveniência:(De \textunderscore in...\textunderscore  + \textunderscore excedível\textunderscore )}
\end{itemize}
Que não póde sêr excedido.
Muito grande: \textunderscore bondade inexcedível\textunderscore .
\section{Inexcitabilidade}
\begin{itemize}
\item {Grp. gram.:f.}
\end{itemize}
Qualidade de quem é inexcitável.
\section{Inexcitável}
\begin{itemize}
\item {Grp. gram.:adj.}
\end{itemize}
\begin{itemize}
\item {Proveniência:(Lat. \textunderscore inexcitabilis\textunderscore )}
\end{itemize}
Que não é susceptível de se excitar.
Impassível; impertubável.
\section{Inexcusável}
\begin{itemize}
\item {Grp. gram.:adj.}
\end{itemize}
\begin{itemize}
\item {Proveniência:(Lat. \textunderscore inexcusabilis\textunderscore )}
\end{itemize}
Que se não dispensa; indesculpável.
\section{Inexecução}
\begin{itemize}
\item {Grp. gram.:f.}
\end{itemize}
\begin{itemize}
\item {Proveniência:(De \textunderscore in...\textunderscore  + \textunderscore execução\textunderscore )}
\end{itemize}
Falta de execução.
\section{Inexecutável}
\begin{itemize}
\item {Grp. gram.:adj.}
\end{itemize}
O mesmo que \textunderscore inexequível\textunderscore .
\section{Inexequibilidade}
\begin{itemize}
\item {fónica:cu-i}
\end{itemize}
\begin{itemize}
\item {Grp. gram.:f.}
\end{itemize}
Qualidade de inexequível.
\section{Inexequível}
\begin{itemize}
\item {fónica:cu-i}
\end{itemize}
\begin{itemize}
\item {Grp. gram.:adj.}
\end{itemize}
\begin{itemize}
\item {Proveniência:(De \textunderscore in...\textunderscore  + \textunderscore exequível\textunderscore )}
\end{itemize}
Que se não póde executar; irrealizável: \textunderscore planos inexequíveis\textunderscore .
\section{Inexhaurível}
\begin{itemize}
\item {Grp. gram.:adj.}
\end{itemize}
\begin{itemize}
\item {Proveniência:(De \textunderscore in...\textunderscore  + \textunderscore exhaurível\textunderscore )}
\end{itemize}
Que não é exhaurível; inesgotável; copiosíssimo.
\section{Inexhausto}
\begin{itemize}
\item {Grp. gram.:adj.}
\end{itemize}
\begin{itemize}
\item {Proveniência:(Lat. \textunderscore inexhaustus\textunderscore )}
\end{itemize}
Que não está exhausto.
\section{Inexigível}
\begin{itemize}
\item {Grp. gram.:adj.}
\end{itemize}
\begin{itemize}
\item {Proveniência:(De \textunderscore in...\textunderscore  + \textunderscore exigível\textunderscore )}
\end{itemize}
Que não é exigível.
\section{Inexistência}
\begin{itemize}
\item {Grp. gram.:f.}
\end{itemize}
\begin{itemize}
\item {Proveniência:(De \textunderscore in...\textunderscore  + \textunderscore existência\textunderscore )}
\end{itemize}
Falta de existência; carência.
\section{Inexistente}
\begin{itemize}
\item {Grp. gram.:adj.}
\end{itemize}
\begin{itemize}
\item {Proveniência:(Lat. \textunderscore inexistens\textunderscore )}
\end{itemize}
Que não existe.
\section{Inexorabilidade}
\begin{itemize}
\item {Grp. gram.:f.}
\end{itemize}
Qualidade de quem ou daquillo que é inexorável.
\section{Inexorado}
\begin{itemize}
\item {Grp. gram.:adj.}
\end{itemize}
\begin{itemize}
\item {Proveniência:(Lat. \textunderscore inexoratus\textunderscore )}
\end{itemize}
Que não foi exorado; a quem se não supplicou.
\section{Inexorável}
\begin{itemize}
\item {Grp. gram.:adj.}
\end{itemize}
\begin{itemize}
\item {Proveniência:(Lat. \textunderscore inexorabilis\textunderscore )}
\end{itemize}
Que não é exorável; implacável.
Austero; rígido.
\section{Inexoravelmente}
\begin{itemize}
\item {Grp. gram.:adv.}
\end{itemize}
De modo inexorável.
\section{Inexpedito}
\begin{itemize}
\item {Grp. gram.:adj.}
\end{itemize}
\begin{itemize}
\item {Proveniência:(Do lat. \textunderscore in\textunderscore  + \textunderscore expeditus\textunderscore )}
\end{itemize}
Que não é expedito; que não tem desembaraço.
\section{Inexperiência}
\begin{itemize}
\item {Grp. gram.:f.}
\end{itemize}
\begin{itemize}
\item {Proveniência:(Lat. \textunderscore inexperientia\textunderscore )}
\end{itemize}
Qualidade de quem é inexperiente; falta de experiência.
\section{Inexperiente}
\begin{itemize}
\item {Grp. gram.:adj.}
\end{itemize}
\begin{itemize}
\item {Proveniência:(Lat. \textunderscore inexperiens\textunderscore )}
\end{itemize}
Que não é experiente; ingênuo; innocente.
\section{Inexperto}
\begin{itemize}
\item {Grp. gram.:adj.}
\end{itemize}
\begin{itemize}
\item {Proveniência:(Lat. \textunderscore inexpertus\textunderscore )}
\end{itemize}
O mesmo que \textunderscore inexperiente\textunderscore .
\section{Inexpiado}
\begin{itemize}
\item {Grp. gram.:adj.}
\end{itemize}
\begin{itemize}
\item {Proveniência:(Lat. \textunderscore inexpiatus\textunderscore )}
\end{itemize}
Não expiado.
\section{Inexpiável}
\begin{itemize}
\item {Grp. gram.:adj.}
\end{itemize}
\begin{itemize}
\item {Proveniência:(Lat. \textunderscore inexpiabilis\textunderscore )}
\end{itemize}
Que não é expiável.
\section{Inexplanável}
\begin{itemize}
\item {Grp. gram.:adj.}
\end{itemize}
\begin{itemize}
\item {Utilização:Des.}
\end{itemize}
\begin{itemize}
\item {Proveniência:(Do lat. \textunderscore in\textunderscore  + \textunderscore explanabilis\textunderscore )}
\end{itemize}
O mesmo que \textunderscore inexplicável\textunderscore .
\section{Inexplicabilidade}
\begin{itemize}
\item {Grp. gram.:f.}
\end{itemize}
Qualidade daquillo que é inexplicável.
\section{Inexplicável}
\begin{itemize}
\item {Grp. gram.:adj.}
\end{itemize}
\begin{itemize}
\item {Proveniência:(Lat. \textunderscore inexplicabilis\textunderscore )}
\end{itemize}
Que não é explicável; obscuro; intrincado.
\section{Inexplicavelmente}
\begin{itemize}
\item {Grp. gram.:adv.}
\end{itemize}
De modo inexplicável.
\section{Inexplícito}
\begin{itemize}
\item {Grp. gram.:adj.}
\end{itemize}
\begin{itemize}
\item {Proveniência:(De \textunderscore in...\textunderscore  + \textunderscore explícito\textunderscore )}
\end{itemize}
Não explícito; obscuro.
\section{Inexplorado}
\begin{itemize}
\item {Grp. gram.:adj.}
\end{itemize}
\begin{itemize}
\item {Proveniência:(Lat. \textunderscore inexploratus\textunderscore )}
\end{itemize}
Não explorado.
\section{Inexplorável}
\begin{itemize}
\item {Grp. gram.:adj.}
\end{itemize}
\begin{itemize}
\item {Proveniência:(De \textunderscore in...\textunderscore  + \textunderscore explorável\textunderscore )}
\end{itemize}
Que não é explorável.
\section{Inexpressável}
\begin{itemize}
\item {Grp. gram.:adj.}
\end{itemize}
\begin{itemize}
\item {Proveniência:(De \textunderscore in...\textunderscore  + \textunderscore expressar\textunderscore )}
\end{itemize}
O mesmo que \textunderscore inexprimível\textunderscore .
\section{Inexpressividade}
\begin{itemize}
\item {Grp. gram.:f.}
\end{itemize}
Qualidade de inexpressivo.
\section{Inexpressivo}
\begin{itemize}
\item {Grp. gram.:adj.}
\end{itemize}
\begin{itemize}
\item {Proveniência:(De \textunderscore in...\textunderscore  + \textunderscore expressivo\textunderscore )}
\end{itemize}
Que não é expressivo.
\section{Inexprimível}
\begin{itemize}
\item {Grp. gram.:adj.}
\end{itemize}
\begin{itemize}
\item {Proveniência:(De \textunderscore in...\textunderscore  + \textunderscore exprimível\textunderscore )}
\end{itemize}
Que não é exprimível.
Indizível.
\section{Inexprimivelmente}
\begin{itemize}
\item {Grp. gram.:adv.}
\end{itemize}
De modo inexprimível.
\section{Inexpugnabilidade}
\begin{itemize}
\item {Grp. gram.:f.}
\end{itemize}
Qualidade daquillo que é inexpugnável.
\section{Inexpugnado}
\begin{itemize}
\item {Grp. gram.:adj.}
\end{itemize}
\begin{itemize}
\item {Proveniência:(Do lat. \textunderscore in\textunderscore  + \textunderscore expugnatus\textunderscore )}
\end{itemize}
Não vencido; invencível.
\section{Inexpugnável}
\begin{itemize}
\item {Grp. gram.:adj.}
\end{itemize}
\begin{itemize}
\item {Utilização:Fig.}
\end{itemize}
\begin{itemize}
\item {Proveniência:(Lat. \textunderscore inexpugnabilis\textunderscore )}
\end{itemize}
Que não é expugnável; inconquistável.
Invencível.
Intrépido.
\section{Inexpunhável}
\begin{itemize}
\item {Grp. gram.:adj.}
\end{itemize}
\begin{itemize}
\item {Utilização:Ant.}
\end{itemize}
O mesmo que \textunderscore inexpugnável\textunderscore . Cf. Usque, 12.
\section{Inextendível}
\begin{itemize}
\item {Grp. gram.:adj.}
\end{itemize}
(V.inestendível)
\section{Inextensibilidade}
\begin{itemize}
\item {Grp. gram.:f.}
\end{itemize}
Qualidade daquillo que é inextensível.
\section{Inextensível}
\begin{itemize}
\item {Grp. gram.:adj.}
\end{itemize}
\begin{itemize}
\item {Proveniência:(De \textunderscore in...\textunderscore  + \textunderscore extensível\textunderscore )}
\end{itemize}
Que não é extensível.
\section{Inextenso}
\begin{itemize}
\item {Grp. gram.:adj.}
\end{itemize}
\begin{itemize}
\item {Proveniência:(De \textunderscore in...\textunderscore  + \textunderscore extenso\textunderscore )}
\end{itemize}
Não estendido.
\section{Inexterminável}
\begin{itemize}
\item {Grp. gram.:adj.}
\end{itemize}
\begin{itemize}
\item {Proveniência:(De \textunderscore in...\textunderscore  + \textunderscore exterminável\textunderscore )}
\end{itemize}
Que se não póde exterminar.
\section{Inextinguibilidade}
\begin{itemize}
\item {Grp. gram.:f.}
\end{itemize}
Qualidade daquillo que é inextinguível.
\section{Inextinguível}
\begin{itemize}
\item {Grp. gram.:adj.}
\end{itemize}
\begin{itemize}
\item {Proveniência:(Lat. \textunderscore inextinguibilis\textunderscore )}
\end{itemize}
Que não é extinguível; permanente.
\section{Inextinto}
\begin{itemize}
\item {Grp. gram.:adj.}
\end{itemize}
\begin{itemize}
\item {Proveniência:(De \textunderscore in...\textunderscore  + \textunderscore extinto\textunderscore )}
\end{itemize}
Que se não extinguiu.
\section{Inextirpável}
\begin{itemize}
\item {Grp. gram.:adj.}
\end{itemize}
\begin{itemize}
\item {Proveniência:(De \textunderscore in...\textunderscore  + \textunderscore extirpável\textunderscore )}
\end{itemize}
Que se não póde extirpar.
\section{Inextricabilidade}
\begin{itemize}
\item {Grp. gram.:f.}
\end{itemize}
Qualidade daquillo que é inextricável.
\section{Inextricável}
\begin{itemize}
\item {Grp. gram.:adj.}
\end{itemize}
\begin{itemize}
\item {Proveniência:(Lat. \textunderscore inextricabilis\textunderscore )}
\end{itemize}
Que se não póde desembaraçar; enredado; emmaranhado.
\section{Inextricavelmente}
\begin{itemize}
\item {Grp. gram.:adv.}
\end{itemize}
De modo inextricável.
\section{Infaceto}
\begin{itemize}
\item {Grp. gram.:adj.}
\end{itemize}
\begin{itemize}
\item {Utilização:Des.}
\end{itemize}
\begin{itemize}
\item {Proveniência:(Do lat. \textunderscore in\textunderscore  + \textunderscore facetus\textunderscore )}
\end{itemize}
Grosseiro.
Indelicado.
\section{Infactível}
\begin{itemize}
\item {Grp. gram.:adj.}
\end{itemize}
\begin{itemize}
\item {Proveniência:(De \textunderscore in...\textunderscore  + \textunderscore factível\textunderscore )}
\end{itemize}
Que não é factível; irrealizável; inexequível.
\section{Infacundo}
\begin{itemize}
\item {Grp. gram.:adj.}
\end{itemize}
\begin{itemize}
\item {Proveniência:(De \textunderscore in...\textunderscore  + \textunderscore facundo\textunderscore )}
\end{itemize}
Não facundo; pouco eloquente.
\section{Infalibilidade}
\begin{itemize}
\item {Grp. gram.:f.}
\end{itemize}
Qualidade daquele ou daquilo que é infalível.
\section{Infalibilista}
\begin{itemize}
\item {Grp. gram.:m.  e  adj.}
\end{itemize}
\begin{itemize}
\item {Proveniência:(De \textunderscore infalível\textunderscore )}
\end{itemize}
Sectário da infalibilidade do Papa.
\section{Infalibilismo}
\begin{itemize}
\item {Grp. gram.:m.}
\end{itemize}
\begin{itemize}
\item {Proveniência:(Do lat. \textunderscore infallibilis\textunderscore )}
\end{itemize}
Doutrina católica sobre a infalibilidade do Papa, em matéria de fé e de moral, segundo as decisões do concílio do Vaticano:«\textunderscore o marianismo e o infalibilismo quási levam o cristianismo de vencida\textunderscore ». Herculano, \textunderscore Quest. Públ.\textunderscore , I, 266.
\section{Infalível}
\begin{itemize}
\item {Grp. gram.:adj.}
\end{itemize}
\begin{itemize}
\item {Proveniência:(De \textunderscore in...\textunderscore  + \textunderscore falível\textunderscore )}
\end{itemize}
Que não é falível; que se não póde enganar.
Que nunca se engana.
Que não póde deixar de acontecer; inevitável; fatal.
\section{Infalivelmente}
\begin{itemize}
\item {Grp. gram.:adv.}
\end{itemize}
De modo infalível.
Sem falta: \textunderscore aparece todos os dias infalivelmente\textunderscore .
\section{Infallibilidade}
\begin{itemize}
\item {Grp. gram.:f.}
\end{itemize}
Qualidade daquelle ou daquillo que é infallível.
\section{Infallibilista}
\begin{itemize}
\item {Grp. gram.:m.  e  adj.}
\end{itemize}
\begin{itemize}
\item {Proveniência:(De \textunderscore infallível\textunderscore )}
\end{itemize}
Sectário da infallibilidade do Papa.
\section{Infallibilismo}
\begin{itemize}
\item {Grp. gram.:m.}
\end{itemize}
\begin{itemize}
\item {Proveniência:(Do lat. \textunderscore infallibilis\textunderscore )}
\end{itemize}
Doutrina cathólica sobre a infallibilidade do Papa, em matéria de fé e de moral, segundo as decisões do concílio do Vaticano:«\textunderscore o marianismo e o infallibilismo quási levam o christianismo de vencida\textunderscore ». Herculano, \textunderscore Quest. Públ.\textunderscore , I, 266.
\section{Infallível}
\begin{itemize}
\item {Grp. gram.:adj.}
\end{itemize}
\begin{itemize}
\item {Proveniência:(De \textunderscore in...\textunderscore  + \textunderscore fallível\textunderscore )}
\end{itemize}
Que não é fallível; que se não póde enganar.
Que nunca se engana.
Que não póde deixar de acontecer; inevitável; fatal.
\section{Infallivelmente}
\begin{itemize}
\item {Grp. gram.:adv.}
\end{itemize}
De modo infallível.
Sem falta: \textunderscore apparece todos os dias infallivelmente\textunderscore .
\section{Infalsificável}
\begin{itemize}
\item {Grp. gram.:adj.}
\end{itemize}
\begin{itemize}
\item {Proveniência:(De \textunderscore in...\textunderscore  + \textunderscore falsificável\textunderscore )}
\end{itemize}
Que se não póde falsificar.
\section{Infamação}
\begin{itemize}
\item {Grp. gram.:f.}
\end{itemize}
\begin{itemize}
\item {Proveniência:(Lat. \textunderscore infamatio\textunderscore )}
\end{itemize}
Acto ou effeito de infamar.
Diffamação; descrédito.
\section{Infamador}
\begin{itemize}
\item {Grp. gram.:adj.}
\end{itemize}
\begin{itemize}
\item {Grp. gram.:M.}
\end{itemize}
Que infama.
Aquelle que infama.
\section{Infamante}
\begin{itemize}
\item {Grp. gram.:adj.}
\end{itemize}
\begin{itemize}
\item {Proveniência:(Lat. \textunderscore infamans\textunderscore )}
\end{itemize}
Que infama.
\section{Infamar}
\begin{itemize}
\item {Grp. gram.:v. t.}
\end{itemize}
\begin{itemize}
\item {Proveniência:(Lat. \textunderscore infamare\textunderscore )}
\end{itemize}
Tornar infame, ignominioso, deshonrado.
Desacreditar, polluír.
Attribuír infâmias a.
\section{Infamatório}
\begin{itemize}
\item {Grp. gram.:adj.}
\end{itemize}
O mesmo que \textunderscore infamante\textunderscore .
\section{Infame}
\begin{itemize}
\item {Grp. gram.:adj.}
\end{itemize}
\begin{itemize}
\item {Grp. gram.:M.}
\end{itemize}
\begin{itemize}
\item {Proveniência:(Lat. \textunderscore infamis\textunderscore )}
\end{itemize}
Que tem má fama.
Torpe.
Que pratíca actos abjectos; vil.
Aquelle que pratíca actos infames.
Bandalho.
\section{Infamemente}
\begin{itemize}
\item {Grp. gram.:adv.}
\end{itemize}
De modo infame.
\section{Infâmia}
\begin{itemize}
\item {Grp. gram.:f.}
\end{itemize}
\begin{itemize}
\item {Proveniência:(Lat. \textunderscore infamia\textunderscore )}
\end{itemize}
Perda de credito, de bôa fama.
Acto ou dito infame; qualidade de quem é infame.
\section{Infâmio}
\begin{itemize}
\item {Grp. gram.:m.  e  adj.}
\end{itemize}
\begin{itemize}
\item {Utilização:Ant.}
\end{itemize}
O mesmo que \textunderscore infame\textunderscore . Cf. \textunderscore Anat. Joc.\textunderscore , I, 440.
\section{Infanção}
\begin{itemize}
\item {Grp. gram.:m.}
\end{itemize}
Antigo título de nobreza, inferior ao de \textunderscore rico-homem\textunderscore .
Talvez escudeiro fidalgo, que ás vezes regia terras ou era guarda de castellos.--Há divergências sôbre o significado rigoroso da palavra. Cf. \textunderscore Elucidário\textunderscore , vb. \textunderscore infançon\textunderscore .
(B. lat. \textunderscore infancio\textunderscore )
\section{Infância}
\begin{itemize}
\item {Grp. gram.:f.}
\end{itemize}
\begin{itemize}
\item {Utilização:Fig.}
\end{itemize}
\begin{itemize}
\item {Proveniência:(Lat. \textunderscore infantia\textunderscore )}
\end{itemize}
Meninice; primeiro período da existência humana.
Primeiro período de uma instituição, arte, sociedade, etc.: \textunderscore a infância da Renascença\textunderscore .
As crianças, em geral: \textunderscore a infância é o encanto do lar\textunderscore .
\section{Infando}
\begin{itemize}
\item {Grp. gram.:adj.}
\end{itemize}
\begin{itemize}
\item {Proveniência:(Do lat. \textunderscore infandus\textunderscore )}
\end{itemize}
Indigno de se dizer; inaudito; horrível; abominável.
Cruel.
\section{Infanta}
\begin{itemize}
\item {Grp. gram.:f.}
\end{itemize}
\begin{itemize}
\item {Utilização:Ext.}
\end{itemize}
\begin{itemize}
\item {Proveniência:(De \textunderscore infante\textunderscore ^1)}
\end{itemize}
Filha de rei português ou espanhol, que não é herdeira da corôa.
Mulher de um infante.
\section{Infantádigo}
\begin{itemize}
\item {Grp. gram.:m.}
\end{itemize}
O mesmo que \textunderscore infantático\textunderscore .
\section{Infantado}
\begin{itemize}
\item {Grp. gram.:m.}
\end{itemize}
\begin{itemize}
\item {Proveniência:(De \textunderscore infante\textunderscore ^1)}
\end{itemize}
Terras ou rendas, pertencentes a um infante.
Estado de um infante.
\section{Infantal}
\begin{itemize}
\item {Grp. gram.:adj.}
\end{itemize}
\begin{itemize}
\item {Utilização:Des.}
\end{itemize}
Relativo a infante^1.
\section{Infantaria}
\begin{itemize}
\item {Grp. gram.:f.}
\end{itemize}
\begin{itemize}
\item {Utilização:Restrict.}
\end{itemize}
\begin{itemize}
\item {Proveniência:(De \textunderscore infante\textunderscore ^2)}
\end{itemize}
Tropa, que faz serviço a pé.
Parte do exército, que faz serviço a pé, exceptuando-se caçadores.
\section{Infantático}
\begin{itemize}
\item {Grp. gram.:m.}
\end{itemize}
Conjunto dos direitos e privilégios do infanção. Cf. Herculano, \textunderscore Bobo\textunderscore , no \textunderscore Panorama\textunderscore , VII, 171.
(B. lat. \textunderscore infantaticum\textunderscore )
\section{Infante}
\begin{itemize}
\item {Grp. gram.:adj.}
\end{itemize}
\begin{itemize}
\item {Grp. gram.:M.  e  f.}
\end{itemize}
\begin{itemize}
\item {Grp. gram.:M.}
\end{itemize}
\begin{itemize}
\item {Utilização:Ant.}
\end{itemize}
\begin{itemize}
\item {Proveniência:(Lat. \textunderscore infans\textunderscore )}
\end{itemize}
Relativo á infância; infantil.
Filho ou filha de rei português ou espanhol, que não são herdeiros da corôa.
O mesmo que \textunderscore criança\textunderscore .
Monge de poucos annos na Ordem de San-Bento.
\section{Infante}
\begin{itemize}
\item {Grp. gram.:m.}
\end{itemize}
\begin{itemize}
\item {Proveniência:(It. \textunderscore fante\textunderscore , soldado a pé)}
\end{itemize}
Soldado de infantaria; peão.
\section{Infanticida}
\begin{itemize}
\item {Grp. gram.:m. ,  f.  e  adj.}
\end{itemize}
\begin{itemize}
\item {Proveniência:(Lat. \textunderscore infanticida\textunderscore )}
\end{itemize}
Pessôa, que commetteu infanticídio.
\section{Infanticídio}
\begin{itemize}
\item {Grp. gram.:m.}
\end{itemize}
\begin{itemize}
\item {Proveniência:(Lat. \textunderscore infanticidium\textunderscore )}
\end{itemize}
Morte, dada voluntariamente a uma criança.
\section{Infantil}
\begin{itemize}
\item {Grp. gram.:adj.}
\end{itemize}
\begin{itemize}
\item {Utilização:Ext.}
\end{itemize}
\begin{itemize}
\item {Proveniência:(Lat. \textunderscore infantilis\textunderscore )}
\end{itemize}
Relativo a crianças; próprio de crianças: \textunderscore as graças infantis\textunderscore .
Ingênuo, innocente.
\section{Infantilidade}
\begin{itemize}
\item {Grp. gram.:f.}
\end{itemize}
Qualidade daquelle ou daquillo que é infantil.
\section{Infantilismo}
\begin{itemize}
\item {Grp. gram.:m.}
\end{itemize}
\begin{itemize}
\item {Utilização:Med.}
\end{itemize}
\begin{itemize}
\item {Proveniência:(De \textunderscore infantil\textunderscore )}
\end{itemize}
Estado anormal dos indivíduos, que, não obstante a idade, têm aspecto, sentimentos e modos de criança.
\section{Infantilizar}
\begin{itemize}
\item {Grp. gram.:v. t.}
\end{itemize}
Tornar infantil.
Dar feição infantil a:«\textunderscore e infantilizava o timbre da voz\textunderscore ». Camillo, \textunderscore Brasileira\textunderscore , 336.
\section{Infantinho}
\begin{itemize}
\item {Grp. gram.:adj.}
\end{itemize}
\begin{itemize}
\item {Proveniência:(De \textunderscore infante\textunderscore ^1)}
\end{itemize}
Que é muito pequeno, que ainda tem muito pouca idade:«\textunderscore ...grita o mais infantinho...\textunderscore »Júl. Castilho, \textunderscore Prim. Versos\textunderscore , 47.
\section{Infantino}
\begin{itemize}
\item {Grp. gram.:adj.}
\end{itemize}
\begin{itemize}
\item {Utilização:Gal}
\end{itemize}
\begin{itemize}
\item {Proveniência:(Fr. \textunderscore enfantin\textunderscore )}
\end{itemize}
Próprio de crianças; infantil:«\textunderscore Julio era infantino no rosto...\textunderscore »Garrett, \textunderscore Viagens\textunderscore , II, 169.
\section{Infantista}
\begin{itemize}
\item {Grp. gram.:m.}
\end{itemize}
Partidário de um infante. Cf. Garrett, \textunderscore Port. na Balança\textunderscore , 166.
\section{Infatigabilidade}
\begin{itemize}
\item {Grp. gram.:f.}
\end{itemize}
Qualidade de infatigável.
\section{Infatigável}
\begin{itemize}
\item {Grp. gram.:adj.}
\end{itemize}
\begin{itemize}
\item {Proveniência:(Lat. \textunderscore infatigabilis\textunderscore )}
\end{itemize}
Que se não fatiga; incansável; zeloso; desvelado.
\section{Infatigavelmente}
\begin{itemize}
\item {Grp. gram.:adv.}
\end{itemize}
De modo infatigável.
\section{Infaustamente}
\begin{itemize}
\item {Grp. gram.:adv.}
\end{itemize}
De modo infausto; desgraçadamente.
\section{Infausto}
\begin{itemize}
\item {Grp. gram.:adj.}
\end{itemize}
\begin{itemize}
\item {Proveniência:(Lat. \textunderscore infaustus\textunderscore )}
\end{itemize}
Que não é fausto; desgraçado.
Agoirento.
\section{Infavorável}
\begin{itemize}
\item {Grp. gram.:adj.}
\end{itemize}
\begin{itemize}
\item {Proveniência:(Do lat. \textunderscore in\textunderscore  + \textunderscore favorabilis\textunderscore )}
\end{itemize}
O mesmo que \textunderscore desfavorável\textunderscore .
\section{Infecção}
\begin{itemize}
\item {Grp. gram.:f.}
\end{itemize}
\begin{itemize}
\item {Proveniência:(Lat. \textunderscore infectio\textunderscore )}
\end{itemize}
Acto ou effeito de inficionar.
Corrupção; contágio.
\section{Infeccionar}
\begin{itemize}
\item {Grp. gram.:v. t.}
\end{itemize}
\begin{itemize}
\item {Proveniência:(Do lat. \textunderscore infectio\textunderscore )}
\end{itemize}
O mesmo que \textunderscore inficionar\textunderscore .
\section{Infeccioso}
\begin{itemize}
\item {Grp. gram.:adj.}
\end{itemize}
\begin{itemize}
\item {Proveniência:(Do lat. \textunderscore infectio\textunderscore )}
\end{itemize}
Que resulta de infecção: \textunderscore moléstia infecciosa\textunderscore .
Que produz infecção: \textunderscore pântano infeccioso\textunderscore .
\section{Infectante}
\begin{itemize}
\item {Grp. gram.:adj.}
\end{itemize}
Que infecta.
\section{Infectar}
\begin{itemize}
\item {Grp. gram.:v. t.}
\end{itemize}
Tornar infecto; contagiar; corromper.
\section{Infecto}
\begin{itemize}
\item {Grp. gram.:adj.}
\end{itemize}
\begin{itemize}
\item {Proveniência:(Lat. \textunderscore infectus\textunderscore )}
\end{itemize}
Corrupto.
Que lança mau cheiro.
Pestilento.
\section{Infectuoso}
\begin{itemize}
\item {Grp. gram.:adj.}
\end{itemize}
\begin{itemize}
\item {Proveniência:(De \textunderscore infecto\textunderscore )}
\end{itemize}
Que produz infecção.
\section{Infecundar}
\begin{itemize}
\item {Grp. gram.:v. t.}
\end{itemize}
\begin{itemize}
\item {Proveniência:(De \textunderscore in...\textunderscore  + \textunderscore fecundar\textunderscore )}
\end{itemize}
O mesmo que \textunderscore castrar\textunderscore , Cf. Camillo, \textunderscore Críticos do Cancion.\textunderscore , p. VIII.
\section{Infecundidade}
\begin{itemize}
\item {Grp. gram.:f.}
\end{itemize}
\begin{itemize}
\item {Proveniência:(Lat. \textunderscore infecunditas\textunderscore )}
\end{itemize}
Qualidade de infecundo; falta de fecundidade.
\section{Infecundo}
\begin{itemize}
\item {Grp. gram.:adj.}
\end{itemize}
\begin{itemize}
\item {Proveniência:(Lat. \textunderscore infecundus\textunderscore )}
\end{itemize}
Que não é fecundo; estéril; que não dá fruto; que produz pouco ou nada.
\section{Infelice}
\begin{itemize}
\item {Grp. gram.:adj.}
\end{itemize}
\begin{itemize}
\item {Utilização:Des.}
\end{itemize}
O mesmo que \textunderscore infeliz\textunderscore .
\section{Infelicidade}
\begin{itemize}
\item {Grp. gram.:f.}
\end{itemize}
\begin{itemize}
\item {Proveniência:(Lat. \textunderscore infelicitas\textunderscore )}
\end{itemize}
Qualidade ou estado de infeliz; falta do felicidade; desventura, desgraça.
\section{Infelicitação}
\begin{itemize}
\item {Grp. gram.:f.}
\end{itemize}
Acto ou effeito de infelicitar.
\section{Infelicitador}
\begin{itemize}
\item {Grp. gram.:adj.}
\end{itemize}
\begin{itemize}
\item {Grp. gram.:M.}
\end{itemize}
Que infelicita.
Aquelle que infelicita.
\section{Infelicitar}
\begin{itemize}
\item {Grp. gram.:v. t.}
\end{itemize}
\begin{itemize}
\item {Proveniência:(Lat. \textunderscore infelicitare\textunderscore )}
\end{itemize}
Tornar infeliz.
\section{Infeliz}
\begin{itemize}
\item {Grp. gram.:adj.}
\end{itemize}
\begin{itemize}
\item {Grp. gram.:M.}
\end{itemize}
\begin{itemize}
\item {Proveniência:(Lat. \textunderscore infelix\textunderscore )}
\end{itemize}
Não feliz; desafortunado; desditoso: \textunderscore um homem infeliz\textunderscore .
Desastrado.
Infausto, adverso: \textunderscore um anno infeliz\textunderscore .
Homem infeliz.
\section{Infelizmente}
\begin{itemize}
\item {Grp. gram.:adv.}
\end{itemize}
De modo infeliz; com infelicidade; desgraçadamente.
\section{Infenso}
\begin{itemize}
\item {Grp. gram.:adj.}
\end{itemize}
\begin{itemize}
\item {Proveniência:(Lat. \textunderscore infensus\textunderscore )}
\end{itemize}
Irado; inimigo. Cf. Vieira, IV, 132.
\section{Inferaxillar}
\begin{itemize}
\item {Grp. gram.:adj.}
\end{itemize}
\begin{itemize}
\item {Utilização:Bot.}
\end{itemize}
\begin{itemize}
\item {Proveniência:(De \textunderscore infero\textunderscore  + \textunderscore axillar\textunderscore )}
\end{itemize}
Diz-se dos órgãos vegetaes, que ficam por baixo das axillas.
\section{Inferência}
\begin{itemize}
\item {Grp. gram.:f.}
\end{itemize}
Acto ou effeito de inferir; consequência; conclusão.
\section{Inférias}
\begin{itemize}
\item {Grp. gram.:f. pl.}
\end{itemize}
\begin{itemize}
\item {Proveniência:(Lat. \textunderscore inferiae\textunderscore )}
\end{itemize}
Sacrifícios, que os antigos faziam aos mortos.
Libações fúnebres.
\section{Inferior}
\begin{itemize}
\item {Grp. gram.:adj.}
\end{itemize}
\begin{itemize}
\item {Grp. gram.:M.}
\end{itemize}
\begin{itemize}
\item {Proveniência:(Lat. \textunderscore inferior\textunderscore )}
\end{itemize}
Que está abaixo ou por baixo: \textunderscore um pavimento inferior\textunderscore .
Que vale menos que outro: \textunderscore Macedo é inferior a Bocage\textunderscore .
Que tem categoria subordinada á do outro: \textunderscore um empregado inferior\textunderscore .
Insignificante, de pouco valor: \textunderscore um livro inferior\textunderscore .
Que occupa o lugar mais baixo na escala zoológica, ou cuja organização é a menos complicada: \textunderscore os seres inferiores\textunderscore .
Aquelle que está abaixo de outro, em categoria, dignidade, etc.: \textunderscore os nossos inferiores\textunderscore .
\section{Inferioridade}
\begin{itemize}
\item {Grp. gram.:f.}
\end{itemize}
Qualidade ou estado daquelle ou daquillo que é inferior.
\section{Inferiormente}
\begin{itemize}
\item {Grp. gram.:adv.}
\end{itemize}
Na parte inferior; inferioridade.
De modo insignificante ou reles.
\section{Inferir}
\begin{itemize}
\item {Grp. gram.:v. t.}
\end{itemize}
\begin{itemize}
\item {Proveniência:(Do lat. \textunderscore inferre\textunderscore )}
\end{itemize}
Deduzir por meio de raciocínio; tirar como consequência, concluir.
\section{Infermentescibilidade}
\begin{itemize}
\item {Grp. gram.:f.}
\end{itemize}
Qualidade daquillo que é infermentescível.
\section{Infermentescível}
\begin{itemize}
\item {Grp. gram.:adj.}
\end{itemize}
\begin{itemize}
\item {Proveniência:(De \textunderscore in...\textunderscore  + \textunderscore fermentescível\textunderscore )}
\end{itemize}
Que não é fermentescível.
\section{Infermo}
\begin{itemize}
\item {Grp. gram.:m.  e  adj.}
\end{itemize}
(V. \textunderscore enfermo\textunderscore , etc.)
\section{Infernal}
\begin{itemize}
\item {Grp. gram.:adj.}
\end{itemize}
\begin{itemize}
\item {Utilização:Fig.}
\end{itemize}
\begin{itemize}
\item {Proveniência:(Lat. \textunderscore infernalis\textunderscore )}
\end{itemize}
Relativo ao inferno: \textunderscore as penas infernaes\textunderscore .
Terrivel.
Atroz: \textunderscore dores infernaes\textunderscore .
Furioso.
Horrendo.
Desmedido, exaggerado.
\section{Infernalidade}
\begin{itemize}
\item {Grp. gram.:f.}
\end{itemize}
Qualidade daquillo que é infernal.
\section{Infernalmente}
\begin{itemize}
\item {Grp. gram.:adv.}
\end{itemize}
De modo infernal.
\section{Infernar}
\begin{itemize}
\item {Grp. gram.:v. t.}
\end{itemize}
\begin{itemize}
\item {Utilização:Fig.}
\end{itemize}
Meter no inferno.
Atormentar; affligir; attribular; desesperar.
\section{Inferneira}
\begin{itemize}
\item {Grp. gram.:f.}
\end{itemize}
\begin{itemize}
\item {Proveniência:(De \textunderscore inferno\textunderscore )}
\end{itemize}
Barulho; confusão; tumulto; gente em tumulto.
\section{Infernizar}
\begin{itemize}
\item {Grp. gram.:v. t.}
\end{itemize}
\begin{itemize}
\item {Proveniência:(De \textunderscore inferno\textunderscore )}
\end{itemize}
O mesmo que \textunderscore infernar\textunderscore ; enfrenesiar. Cf. Th. Ribeiro, \textunderscore Jornadas\textunderscore , I, 71.
\section{Inferno}
\begin{itemize}
\item {Grp. gram.:m.}
\end{itemize}
\begin{itemize}
\item {Utilização:Fig.}
\end{itemize}
\begin{itemize}
\item {Utilização:Prov.}
\end{itemize}
\begin{itemize}
\item {Utilização:minh.}
\end{itemize}
\begin{itemize}
\item {Grp. gram.:Adj.}
\end{itemize}
\begin{itemize}
\item {Proveniência:(Lat. \textunderscore infernus\textunderscore )}
\end{itemize}
Lugar subterrâneo, em que, segundo a Mythologia, habitam as almas dos mortos.
Lugar que, segundo o Christianismo, é destinado ao supplício das almas dos condemnados.
Os demónios: \textunderscore são tentações do inferno\textunderscore .
Coisa penosa, muito desagradável: \textunderscore essa tua vida é um inferno\textunderscore .
Lugar ou vida de desordem ou confusão.
Desordem.
Tormento.
Poço, que recebe os resíduos líquidos do fabríco do azeite.
Cavouco ou lugar, onde gira o rodizio das azenhas.
Infernal. Cf. Filinto, VI, 188; XV, 14.
\section{Inferno-e-paraíso}
\begin{itemize}
\item {Grp. gram.:m.}
\end{itemize}
Espécie de jôgo popular.
\section{Ínfero}
\begin{itemize}
\item {Grp. gram.:adj.}
\end{itemize}
\begin{itemize}
\item {Grp. gram.:M.}
\end{itemize}
\begin{itemize}
\item {Proveniência:(Lat. \textunderscore inferus\textunderscore )}
\end{itemize}
O mesmo que \textunderscore inferior\textunderscore .
O mesmo que \textunderscore inferno\textunderscore .
\section{Ínfero-anterior}
\begin{itemize}
\item {Grp. gram.:adj.}
\end{itemize}
Situado abaixo e na parte anterior.
\section{Ínfero-exterior}
\begin{itemize}
\item {Grp. gram.:adj.}
\end{itemize}
Que está abaixo e na parte exterior.
\section{Ínfero-interior}
\begin{itemize}
\item {Grp. gram.:adj.}
\end{itemize}
Situado abaixo e na parte interior.
\section{Ínfero-posterior}
\begin{itemize}
\item {Grp. gram.:adj.}
\end{itemize}
Situado atrás, na parte inferior.
\section{Ínfero-súpero}
\begin{itemize}
\item {Grp. gram.:adj.}
\end{itemize}
\begin{itemize}
\item {Utilização:Bot.}
\end{itemize}
Diz-se do fruto que fica abaixo da corolla e acima do cálice.
\section{Inferovariado}
\begin{itemize}
\item {Grp. gram.:adj.}
\end{itemize}
\begin{itemize}
\item {Utilização:Bot.}
\end{itemize}
\begin{itemize}
\item {Proveniência:(De \textunderscore ínfero\textunderscore  + \textunderscore ovário\textunderscore )}
\end{itemize}
Que tem ovário ínfero ou adherente.
\section{Infértil}
\begin{itemize}
\item {Grp. gram.:adj.}
\end{itemize}
\begin{itemize}
\item {Proveniência:(Lat. \textunderscore infertilis\textunderscore )}
\end{itemize}
Que não é fértil; estéril; improductivo; que produz pouco ou nada.
\section{Infertilidade}
\begin{itemize}
\item {Grp. gram.:f.}
\end{itemize}
\begin{itemize}
\item {Proveniência:(Lat. \textunderscore infertilitas\textunderscore )}
\end{itemize}
Falta de fertilidade; qualidade de infértil.
\section{Infertilizar}
\begin{itemize}
\item {Grp. gram.:v. t.}
\end{itemize}
\begin{itemize}
\item {Proveniência:(De \textunderscore infértil\textunderscore )}
\end{itemize}
Tornar infértil, infecundo; esterilizar.
\section{Infertilizável}
\begin{itemize}
\item {Grp. gram.:adj.}
\end{itemize}
\begin{itemize}
\item {Proveniência:(De \textunderscore in...\textunderscore  + \textunderscore fertilizável\textunderscore )}
\end{itemize}
Que não é fertilizável.
\section{Infesso}
\begin{itemize}
\item {Grp. gram.:adj.}
\end{itemize}
\begin{itemize}
\item {Proveniência:(Lat. \textunderscore infessus\textunderscore )}
\end{itemize}
Incansável, o mesmo que \textunderscore indefesso\textunderscore .
\section{Infestação}
\begin{itemize}
\item {Grp. gram.:f.}
\end{itemize}
\begin{itemize}
\item {Proveniência:(Lat. \textunderscore infestatio\textunderscore )}
\end{itemize}
Acto ou effeito de infestar.
\section{Infestador}
\begin{itemize}
\item {Grp. gram.:adj.}
\end{itemize}
\begin{itemize}
\item {Grp. gram.:M.}
\end{itemize}
\begin{itemize}
\item {Proveniência:(Lat. \textunderscore infestator\textunderscore )}
\end{itemize}
Que infesta.
Aquelle que infesta.
\section{Infestante}
\begin{itemize}
\item {Grp. gram.:adj.}
\end{itemize}
\begin{itemize}
\item {Proveniência:(Lat. \textunderscore infestans\textunderscore )}
\end{itemize}
Que infesta.
\section{Infestar}
\begin{itemize}
\item {Grp. gram.:v. t.}
\end{itemize}
\begin{itemize}
\item {Proveniência:(Lat. \textunderscore infestare\textunderscore )}
\end{itemize}
Sêr infesto a; assolar; invadir, devastando.
Percorrer como corsário (os mares)
\section{Infesto}
\begin{itemize}
\item {Grp. gram.:adj.}
\end{itemize}
\begin{itemize}
\item {Utilização:Ext.}
\end{itemize}
\begin{itemize}
\item {Proveniência:(Lat. \textunderscore infestus\textunderscore )}
\end{itemize}
Molesto; adverso.
Hostil.
Pernicioso.
\section{Infibulação}
\begin{itemize}
\item {Grp. gram.:f.}
\end{itemize}
Acto ou effeito de infibular.
Ligação artificial dos lábios da vagina.
\section{Infibulador}
\begin{itemize}
\item {Grp. gram.:adj.}
\end{itemize}
\begin{itemize}
\item {Grp. gram.:M.}
\end{itemize}
Que infibula.
Aquelle que infibula.
\section{Infibular}
\begin{itemize}
\item {Grp. gram.:v. t.}
\end{itemize}
\begin{itemize}
\item {Proveniência:(Lat. \textunderscore infibulare\textunderscore )}
\end{itemize}
Ligar por meio de anel ou colchete; acolchetar.
Afivelar.
Reunir, por meio de anel, cadeado ou costura (os lábios da vagina), para impossibilitar o cóito.
Prender ou coser os órgaos genitaes de (adolescentes), em proveito da voz, e da saúde, como recommendava Celso.
\section{Inficionação}
\begin{itemize}
\item {Grp. gram.:f.}
\end{itemize}
Acto ou effeito de inficionar.
\section{Inficionador}
\begin{itemize}
\item {Grp. gram.:adj.}
\end{itemize}
\begin{itemize}
\item {Grp. gram.:M.}
\end{itemize}
Que inficiona.
Aquelle que inficiona.
\section{Inficionar}
\begin{itemize}
\item {Grp. gram.:v. t.}
\end{itemize}
\begin{itemize}
\item {Utilização:Fig.}
\end{itemize}
\begin{itemize}
\item {Proveniência:(Do rad. do lat. \textunderscore inficere\textunderscore )}
\end{itemize}
Viciar; corromper; contaminar.
Depravar, perverter.
\section{Infidelidade}
\begin{itemize}
\item {Grp. gram.:f.}
\end{itemize}
\begin{itemize}
\item {Proveniência:(Lat. \textunderscore infidelitas\textunderscore )}
\end{itemize}
Qualidade de quem é infiel; falta de fidelidade; traição.
\section{Infidelissimo}
\begin{itemize}
\item {Grp. gram.:adj.}
\end{itemize}
\begin{itemize}
\item {Proveniência:(Do lat. \textunderscore infidelis\textunderscore )}
\end{itemize}
Muito infiel.
\section{Infido}
\begin{itemize}
\item {Grp. gram.:adj.}
\end{itemize}
\begin{itemize}
\item {Utilização:Poét.}
\end{itemize}
\begin{itemize}
\item {Proveniência:(Lat. \textunderscore infidus\textunderscore )}
\end{itemize}
O mesmo que \textunderscore infiel\textunderscore .
\section{Infiel}
\begin{itemize}
\item {Grp. gram.:adj.}
\end{itemize}
\begin{itemize}
\item {Grp. gram.:M.  e  f.}
\end{itemize}
\begin{itemize}
\item {Proveniência:(Lat. \textunderscore infidelis\textunderscore )}
\end{itemize}
Não fiel.
Desleal; pérfido; que falta aos seus compromissos: \textunderscore marido infiel\textunderscore .
Traidor.
Pessôa infiel.
Pagão, gentio: \textunderscore prègar aos infiéis\textunderscore .
\section{Infieldade}
\begin{itemize}
\item {Grp. gram.:f.}
\end{itemize}
\begin{itemize}
\item {Utilização:pop.}
\end{itemize}
\begin{itemize}
\item {Utilização:Ant.}
\end{itemize}
\begin{itemize}
\item {Proveniência:(De \textunderscore infiel\textunderscore )}
\end{itemize}
O mesmo que \textunderscore infidelidade\textunderscore .
\section{Infielmente}
\begin{itemize}
\item {Grp. gram.:adv.}
\end{itemize}
De modo infiel; com deslealdade.
\section{Infiltração}
\begin{itemize}
\item {Grp. gram.:f.}
\end{itemize}
\begin{itemize}
\item {Utilização:Fig.}
\end{itemize}
Acto ou effeito de infiltrar.
Acção do liquido ou fluido que se embebe ou penetra nos interstícios dos corpos sólidos.
Derramamento anormal de um líquido nos tecidos orgânicos.
Adopção ou diffusão de ideias ou systemas.
\section{Infiltrar}
\begin{itemize}
\item {Grp. gram.:v. i.}
\end{itemize}
\begin{itemize}
\item {Utilização:Fig.}
\end{itemize}
\begin{itemize}
\item {Proveniência:(De \textunderscore filtrar\textunderscore )}
\end{itemize}
Fazer entrar ou penetrar, como por um filtro.
Introduzir a pouco e pouco; incutir; insinuar.
\section{Infiltrável}
\begin{itemize}
\item {Grp. gram.:adj.}
\end{itemize}
Que se póde infiltrar.
\section{Ínfimo}
\begin{itemize}
\item {Grp. gram.:adj.}
\end{itemize}
\begin{itemize}
\item {Proveniência:(Lat. \textunderscore infimus\textunderscore )}
\end{itemize}
Que está na parte mais baixa; que occupa o último lugar; inferior.
\section{Infinamente}
\begin{itemize}
\item {Grp. gram.:adv.}
\end{itemize}
Infinitamente. Cf. Arn. Gama, \textunderscore Segr. do Abb.\textunderscore , 52.--Êrro typogr.?
\section{Infindamente}
\begin{itemize}
\item {Grp. gram.:adv.}
\end{itemize}
De modo infindo; infinitamente; desmedidamente.
\section{Infindável}
\begin{itemize}
\item {Grp. gram.:adj.}
\end{itemize}
\begin{itemize}
\item {Proveniência:(De \textunderscore in...\textunderscore  + \textunderscore findável\textunderscore )}
\end{itemize}
Que não finda; permanente. Cf. Eça, \textunderscore P. Basílio\textunderscore , 68; \textunderscore Mandarim\textunderscore , 11, 80 e 109; \textunderscore P. Amaro\textunderscore , 271, 534 e 536.
\section{Infindavelmente}
\begin{itemize}
\item {Grp. gram.:adv.}
\end{itemize}
De modo infindável.
\section{Infindo}
\begin{itemize}
\item {Grp. gram.:adj.}
\end{itemize}
\begin{itemize}
\item {Proveniência:(De \textunderscore in...\textunderscore  + \textunderscore findo\textunderscore )}
\end{itemize}
O mesmo que \textunderscore infinito\textunderscore .
Illimitado.
Innumerável; muito numeroso.
\section{Infingir}
\begin{itemize}
\item {Grp. gram.:v. t.}
\end{itemize}
\begin{itemize}
\item {Utilização:Pop.}
\end{itemize}
O mesmo que \textunderscore fingir\textunderscore ^1.
\section{Infinidade}
\begin{itemize}
\item {Grp. gram.:f.}
\end{itemize}
\begin{itemize}
\item {Proveniência:(Lat. \textunderscore infinitas\textunderscore )}
\end{itemize}
Qualidade daquillo que é infinito; grande porção: \textunderscore uma infinidade de porcos\textunderscore .
\section{Infinitamente}
\begin{itemize}
\item {Grp. gram.:adv.}
\end{itemize}
De modo infinito; desmedidamennte; sem limite.
\section{Infinitésima}
\begin{itemize}
\item {Grp. gram.:f.}
\end{itemize}
\begin{itemize}
\item {Utilização:Mathem.}
\end{itemize}
\begin{itemize}
\item {Proveniência:(De \textunderscore infinitésimo\textunderscore )}
\end{itemize}
Parte infinitamente pequena.
\section{Infinitesimal}
\begin{itemize}
\item {Grp. gram.:adj.}
\end{itemize}
\begin{itemize}
\item {Utilização:Mathem.}
\end{itemize}
\begin{itemize}
\item {Proveniência:(De \textunderscore infinitésimo\textunderscore )}
\end{itemize}
Que tem o carácter de infinitésima.
Diz-se do cálculo differencial e do integral.
\section{Infinitésimo}
\begin{itemize}
\item {Grp. gram.:adj.}
\end{itemize}
\begin{itemize}
\item {Utilização:Mathem.}
\end{itemize}
\begin{itemize}
\item {Proveniência:(De \textunderscore infinito\textunderscore , com a desinência \textunderscore ésimo\textunderscore , comum a \textunderscore vigésimo\textunderscore , \textunderscore trigésimo\textunderscore , etc.)}
\end{itemize}
Que é infinitamente pequeno.
\section{Infinitivo}
\begin{itemize}
\item {Grp. gram.:m.  e  adj.}
\end{itemize}
\begin{itemize}
\item {Utilização:Gram.}
\end{itemize}
\begin{itemize}
\item {Proveniência:(Lat. \textunderscore infinitivus\textunderscore )}
\end{itemize}
Modo dos verbos que, exprimindo estado ou acção, não determina o número, nem, geralmente, a pessôa.
\section{Infinito}
\begin{itemize}
\item {Grp. gram.:adj.}
\end{itemize}
\begin{itemize}
\item {Grp. gram.:M.  e  adj.}
\end{itemize}
\begin{itemize}
\item {Utilização:Gram.}
\end{itemize}
\begin{itemize}
\item {Grp. gram.:Adv.}
\end{itemize}
\begin{itemize}
\item {Proveniência:(Do lat. \textunderscore infinitus\textunderscore )}
\end{itemize}
Não finito; infindo; innumerável.
O mesmo que \textunderscore infinitivo\textunderscore .
O mesmo que \textunderscore infinitamente\textunderscore :«\textunderscore pesava infinito.\textunderscore »\textunderscore Luz e Calor\textunderscore , 105.
\section{Infinitovista}
\begin{itemize}
\item {Grp. gram.:m.}
\end{itemize}
\begin{itemize}
\item {Proveniência:(Do lat. \textunderscore infinitus\textunderscore  + \textunderscore ovum\textunderscore )}
\end{itemize}
Physiologista, que considera todos os germes encaixados ou embebidos infinitamente uns aos outros, e entende que a geração é apenas o desencaixe desses germes, uns após outros.
\section{Infinto}
\begin{itemize}
\item {Grp. gram.:adj.}
\end{itemize}
Fingido?:«\textunderscore E era mais infinta...\textunderscore »\textunderscore Eufrosina\textunderscore , 85. Cf. \textunderscore Aulegrafia\textunderscore , 14.
\section{Infirmar}
\begin{itemize}
\item {Grp. gram.:v. t.}
\end{itemize}
\begin{itemize}
\item {Utilização:Fig.}
\end{itemize}
\begin{itemize}
\item {Proveniência:(Lat. \textunderscore infirmare\textunderscore )}
\end{itemize}
Tirar a firmeza ou a fôrça a.
Refutar.
Invalidar; tornar nullo.
\section{Infirmativo}
\begin{itemize}
\item {Grp. gram.:adj.}
\end{itemize}
\begin{itemize}
\item {Proveniência:(De \textunderscore infirmar\textunderscore )}
\end{itemize}
Capaz de infirmar; próprio para infirmar.
\section{Infirmidade}
\begin{itemize}
\item {Grp. gram.:f.}
\end{itemize}
\begin{itemize}
\item {Utilização:Ant.}
\end{itemize}
\begin{itemize}
\item {Proveniência:(Do lat. \textunderscore infirmus\textunderscore )}
\end{itemize}
O mesmo que \textunderscore enfermidade\textunderscore . Cf. \textunderscore Eufrosina\textunderscore , 143; \textunderscore Tenreiro\textunderscore , XXV.
\section{Infixidez}
\begin{itemize}
\item {Grp. gram.:f.}
\end{itemize}
\begin{itemize}
\item {Proveniência:(De \textunderscore in...\textunderscore  + \textunderscore fixidez\textunderscore )}
\end{itemize}
Falta de fixidez.
\section{Infixo}
\begin{itemize}
\item {Grp. gram.:m.}
\end{itemize}
\begin{itemize}
\item {Utilização:Gram.}
\end{itemize}
\begin{itemize}
\item {Proveniência:(Lat. \textunderscore infixus\textunderscore )}
\end{itemize}
Affixo no interior da palavra.
\section{Inflação}
\begin{itemize}
\item {Grp. gram.:f.}
\end{itemize}
\begin{itemize}
\item {Proveniência:(Lat. \textunderscore inflatio\textunderscore )}
\end{itemize}
Acto ou effeito de inflar.
\section{Inflamabilidade}
\begin{itemize}
\item {Grp. gram.:f.}
\end{itemize}
Qualidade daquilo que é inflamável.
\section{Inflamação}
\begin{itemize}
\item {Grp. gram.:f.}
\end{itemize}
\begin{itemize}
\item {Utilização:Med.}
\end{itemize}
\begin{itemize}
\item {Proveniência:(Lat. \textunderscore inflammatio\textunderscore )}
\end{itemize}
Acto ou efeito de inflamar.
Ardor intenso.
Rubor excessivo.
Tumefacção vermelha e dolorosa.
\section{Inflamado}
\begin{itemize}
\item {Grp. gram.:adj.}
\end{itemize}
Exaltado; irritado; esbraseado.
Inchado, com ferimento ou escoriação.
\section{Inflamador}
\begin{itemize}
\item {Grp. gram.:adj.}
\end{itemize}
\begin{itemize}
\item {Grp. gram.:M.}
\end{itemize}
Que inflama.
Aquele que inflama.
\section{Inflamar}
\begin{itemize}
\item {Grp. gram.:v. t.}
\end{itemize}
\begin{itemize}
\item {Utilização:Fig.}
\end{itemize}
\begin{itemize}
\item {Proveniência:(Lat. \textunderscore inflammare\textunderscore )}
\end{itemize}
Converter em chamas.
Acender; abrasar.
Afoguear.
Tornar vermelho.
Tornar dolorido, vermelho e inchado.
Excitar; irritar.
Abrasar de afecto.
\section{Inflamativo}
\begin{itemize}
\item {Grp. gram.:adj.}
\end{itemize}
Que inflama.
\section{Inflamatório}
\begin{itemize}
\item {Grp. gram.:adj.}
\end{itemize}
\begin{itemize}
\item {Proveniência:(De \textunderscore inflamar\textunderscore )}
\end{itemize}
Inflamativo; relativo á inflamação.
\section{Inflamável}
\begin{itemize}
\item {Grp. gram.:adj.}
\end{itemize}
Que se póde inflamar; que se inflama facilmente.
\section{Inflammabilidade}
\begin{itemize}
\item {Grp. gram.:f.}
\end{itemize}
Qualidade daquillo que é inflammável.
\section{Inflammação}
\begin{itemize}
\item {Grp. gram.:f.}
\end{itemize}
\begin{itemize}
\item {Utilização:Med.}
\end{itemize}
\begin{itemize}
\item {Proveniência:(Lat. \textunderscore inflammatio\textunderscore )}
\end{itemize}
Acto ou effeito de inflammar.
Ardor intenso.
Rubor excessivo.
Tumefacção vermelha e dolorosa.
\section{Inflammado}
\begin{itemize}
\item {Grp. gram.:adj.}
\end{itemize}
Exaltado; irritado; esbraseado.
Inchado, com ferimento ou escoriação.
\section{Inflammador}
\begin{itemize}
\item {Grp. gram.:adj.}
\end{itemize}
\begin{itemize}
\item {Grp. gram.:M.}
\end{itemize}
Que inflamma.
Aquelle que inflamma.
\section{Inflammar}
\begin{itemize}
\item {Grp. gram.:v. t.}
\end{itemize}
\begin{itemize}
\item {Utilização:Fig.}
\end{itemize}
\begin{itemize}
\item {Proveniência:(Lat. \textunderscore inflammare\textunderscore )}
\end{itemize}
Converter em chammas.
Accender; abrasar.
Afoguear.
Tornar vermelho.
Tornar dolorido, vermelho e inchado.
Excitar; irritar.
Abrasar de affecto.
\section{Inflammativo}
\begin{itemize}
\item {Grp. gram.:adj.}
\end{itemize}
Que inflamma.
\section{Inflammatório}
\begin{itemize}
\item {Grp. gram.:adj.}
\end{itemize}
\begin{itemize}
\item {Proveniência:(De \textunderscore inflammar\textunderscore )}
\end{itemize}
Inflammativo; relativo á inflammação.
\section{Inflammável}
\begin{itemize}
\item {Grp. gram.:adj.}
\end{itemize}
Que se póde inflammar; que se inflamma facilmente.
\section{Inflar}
\begin{itemize}
\item {Grp. gram.:v. t.}
\end{itemize}
\begin{itemize}
\item {Utilização:Fig.}
\end{itemize}
\begin{itemize}
\item {Proveniência:(Lat. \textunderscore inflare\textunderscore )}
\end{itemize}
Encher de vento.
Entumecer; enfunar: \textunderscore o vento inflava as velas\textunderscore .
Tornar vaidoso; encher de soberba.
\section{Inflativo}
\begin{itemize}
\item {Grp. gram.:adj.}
\end{itemize}
Que tem a propriedade de inflar ou inchar.
\section{Inflatório}
\begin{itemize}
\item {Grp. gram.:adj.}
\end{itemize}
\begin{itemize}
\item {Proveniência:(De \textunderscore inflar\textunderscore )}
\end{itemize}
Que produz inflação.
\section{Inflectir}
\begin{itemize}
\item {Grp. gram.:v. t.}
\end{itemize}
\begin{itemize}
\item {Utilização:Neol.}
\end{itemize}
\begin{itemize}
\item {Utilização:Gram.}
\end{itemize}
\begin{itemize}
\item {Proveniência:(Lat. \textunderscore inflectere\textunderscore )}
\end{itemize}
Dobrar, curvar.
Desviar, inclinar.
Modificar (a voz).
Variar a terminação de.
\section{Inflexão}
\begin{itemize}
\item {fónica:csão}
\end{itemize}
\begin{itemize}
\item {Grp. gram.:f.}
\end{itemize}
\begin{itemize}
\item {Proveniência:(Lat. \textunderscore inflexio\textunderscore )}
\end{itemize}
Acto ou effeito de curvar.
Curvatura.
Inclinação de uma linha.
Diffracção.
Desvio.
Modulação na voz.
Mudança de tom.
Flexão grammatical, variação das desinências dos vocábulos.
\section{Inflexibilidade}
\begin{itemize}
\item {fónica:csi}
\end{itemize}
\begin{itemize}
\item {Grp. gram.:f.}
\end{itemize}
Qualidade daquelle ou daquillo que é inflexível.
\section{Inflexível}
\begin{itemize}
\item {fónica:csi}
\end{itemize}
\begin{itemize}
\item {Grp. gram.:adj.}
\end{itemize}
\begin{itemize}
\item {Utilização:Fig.}
\end{itemize}
\begin{itemize}
\item {Proveniência:(Lat. \textunderscore inflexibilis\textunderscore )}
\end{itemize}
Que não é flexível.
Inexorável; implacável: \textunderscore julgador inflexível\textunderscore .
Impassível; indifferente.
\section{Inflexivelmente}
\begin{itemize}
\item {fónica:csi}
\end{itemize}
\begin{itemize}
\item {Grp. gram.:adv.}
\end{itemize}
De modo inflexível.
\section{Inflexivo}
\begin{itemize}
\item {fónica:csi}
\end{itemize}
\begin{itemize}
\item {Grp. gram.:adj.}
\end{itemize}
\begin{itemize}
\item {Proveniência:(De \textunderscore inflexo\textunderscore )}
\end{itemize}
Que não tem flexões grammaticaes.
\section{Inflexo}
\begin{itemize}
\item {Grp. gram.:adj.}
\end{itemize}
\begin{itemize}
\item {Proveniência:(Lat. \textunderscore inflexus\textunderscore )}
\end{itemize}
Que se inflectiu.
Inclinado.
\section{Inflicção}
\begin{itemize}
\item {Grp. gram.:f.}
\end{itemize}
\begin{itemize}
\item {Proveniência:(Lat. \textunderscore inflictio\textunderscore )}
\end{itemize}
Acto ou effeito de infligir.
\section{Infligidor}
\begin{itemize}
\item {Grp. gram.:adj.}
\end{itemize}
Que inflige.
\section{Infligir}
\begin{itemize}
\item {Grp. gram.:v. t.}
\end{itemize}
\begin{itemize}
\item {Proveniência:(Lat. \textunderscore infligere\textunderscore )}
\end{itemize}
Impor ou applicar (pena, castigo, reprehensão, etc.).
\section{Inflorescência}
\begin{itemize}
\item {Grp. gram.:f.}
\end{itemize}
\begin{itemize}
\item {Utilização:Bot.}
\end{itemize}
\begin{itemize}
\item {Proveniência:(Do lat. \textunderscore inflorescens\textunderscore )}
\end{itemize}
Conjunto das flôres de uma planta.
Conjunto dos órgãos e operações, que preparam ou realizam o desenvolvimento das flôres.
\section{Inflorescente}
\begin{itemize}
\item {Grp. gram.:adj.}
\end{itemize}
\begin{itemize}
\item {Proveniência:(Lat. \textunderscore inflorescens\textunderscore )}
\end{itemize}
Relativo á inflorescência.
\section{Influência}
\begin{itemize}
\item {Grp. gram.:f.}
\end{itemize}
\begin{itemize}
\item {Proveniência:(Lat. \textunderscore influentia\textunderscore )}
\end{itemize}
Acto ou effeito de influir.
Poder ou acção, que alguém exerce sôbre outrem ou sôbre certos factos ou negócios.
Influxo.
Preponderância; prestígio; autoridade moral.
Crédito.
Empenho, enthusiasmo.
Acção, que as substâncias electrizadas exercem noutras, que existam no seu estado natural.
Doença epidêmica, de symptomas análogos aos da gripe e também conhecida por \textunderscore peste russa\textunderscore . Cf. Freund, vb. \textunderscore influentia\textunderscore .
\section{Influenciação}
\begin{itemize}
\item {Grp. gram.:f.}
\end{itemize}
Acto ou effeito de influenciar.
\section{Influenciar}
\begin{itemize}
\item {Grp. gram.:v. t.}
\end{itemize}
Ter ou exercer influência em.
\section{Influente}
\begin{itemize}
\item {Grp. gram.:m.  e  adj.}
\end{itemize}
\begin{itemize}
\item {Proveniência:(Lat. \textunderscore influens\textunderscore )}
\end{itemize}
O que influe ou exerce influência: \textunderscore os influentes de um circulo eleitoral\textunderscore .
\section{Influição}
\begin{itemize}
\item {fónica:flu-i}
\end{itemize}
\begin{itemize}
\item {Grp. gram.:f.}
\end{itemize}
Acto ou effeito de influir; influência.
\section{Influidor}
\begin{itemize}
\item {fónica:flu-i}
\end{itemize}
\begin{itemize}
\item {Grp. gram.:m.  e  adj.}
\end{itemize}
O que influe.
\section{Influir}
\begin{itemize}
\item {Grp. gram.:v. t.}
\end{itemize}
\begin{itemize}
\item {Utilização:Fig.}
\end{itemize}
\begin{itemize}
\item {Grp. gram.:V. i.}
\end{itemize}
\begin{itemize}
\item {Proveniência:(Lat. \textunderscore influere\textunderscore )}
\end{itemize}
Fazer fluír para dentro.
Insinuar; incutir.
Estimular:«\textunderscore ...a apagada estrella que tantas fatalidades influira.\textunderscore »Camillo, \textunderscore Caveira\textunderscore , 14.
Enthusiasmar: \textunderscore influir alguém com applausos\textunderscore .
Ter acção, autoridade moral ou predomínio.
Dominar a vontade alheia: \textunderscore influir nos eleitores\textunderscore .
\section{Influxo}
\begin{itemize}
\item {fónica:cso}
\end{itemize}
\begin{itemize}
\item {Grp. gram.:m.}
\end{itemize}
\begin{itemize}
\item {Proveniência:(Lat. \textunderscore influxus\textunderscore )}
\end{itemize}
Acto ou effeito de influir.
Influência.
Preamar.
Affluência.
\section{In-fólio}
\begin{itemize}
\item {Grp. gram.:adj.}
\end{itemize}
\begin{itemize}
\item {Grp. gram.:M.}
\end{itemize}
\begin{itemize}
\item {Proveniência:(Do lat. \textunderscore in\textunderscore  + \textunderscore folium\textunderscore )}
\end{itemize}
Diz-se de um livro ou de um formato, em que cada fôlha de impressão é apenas dobrada em duas. Cf. Herculano, \textunderscore Opúsc.\textunderscore , III, 65.
Livro, que tem êsse formato.
\section{Informação}
\begin{itemize}
\item {Grp. gram.:f.}
\end{itemize}
\begin{itemize}
\item {Proveniência:(Lat. \textunderscore informatio\textunderscore )}
\end{itemize}
Acto ou effeito de informar.
Transmissão de notícia ou de conhecimentos: \textunderscore pedir informações\textunderscore .
Communicação.
Instrucção.
Indagação; devassa.
\section{Informador}
\begin{itemize}
\item {Grp. gram.:m.  e  adj.}
\end{itemize}
\begin{itemize}
\item {Proveniência:(Lat. \textunderscore informans\textunderscore )}
\end{itemize}
Pessôa que informa.
\section{Informante}
\begin{itemize}
\item {Grp. gram.:m., f.  e  adj.}
\end{itemize}
\begin{itemize}
\item {Proveniência:(Lat. \textunderscore informans\textunderscore )}
\end{itemize}
Pessôa que informa.
\section{Informar}
\begin{itemize}
\item {Grp. gram.:v. t.}
\end{itemize}
\begin{itemize}
\item {Grp. gram.:V. i.}
\end{itemize}
\begin{itemize}
\item {Proveniência:(Lat. \textunderscore informare\textunderscore )}
\end{itemize}
Dar fórma a, enformar.
Tornar existente, real.
Dar conhecimento a: \textunderscore informei-o da catástrophe\textunderscore .
Dar parecer sôbre; esclarecer: \textunderscore informar um processo\textunderscore .
O mesmo que \textunderscore enformar\textunderscore ^2.
\section{Informativo}
\begin{itemize}
\item {Grp. gram.:adj.}
\end{itemize}
Destinado a informar ou noticiar.
\section{Informe}
\begin{itemize}
\item {Grp. gram.:m.}
\end{itemize}
\begin{itemize}
\item {Proveniência:(De \textunderscore informar\textunderscore )}
\end{itemize}
O mesmo que \textunderscore informação\textunderscore .
\section{Informe}
\begin{itemize}
\item {Grp. gram.:adj.}
\end{itemize}
\begin{itemize}
\item {Proveniência:(Lat. \textunderscore informis\textunderscore )}
\end{itemize}
Que não tem fórma ou feitio.
Rude, grosseiro.
Grande; colossal; monstruoso.
\section{Informemente}
\begin{itemize}
\item {Grp. gram.:adv.}
\end{itemize}
De modo informe^2.
\section{Informidade}
\begin{itemize}
\item {Grp. gram.:f.}
\end{itemize}
\begin{itemize}
\item {Proveniência:(Lat. \textunderscore informitas\textunderscore )}
\end{itemize}
Estado daquelle ou daquillo que é informe.
Deformidade.
Ausência de formalidades.
\section{Infortificável}
\begin{itemize}
\item {Grp. gram.:adj.}
\end{itemize}
\begin{itemize}
\item {Proveniência:(De \textunderscore in...\textunderscore  + \textunderscore fortificar\textunderscore )}
\end{itemize}
Que se não póde fortificar.
\section{Infortuna}
\begin{itemize}
\item {Grp. gram.:f.}
\end{itemize}
\begin{itemize}
\item {Proveniência:(De \textunderscore in...\textunderscore  + \textunderscore fortuna\textunderscore )}
\end{itemize}
O mesmo que \textunderscore desfortuna\textunderscore .
Apparição de um astro, a que se attribuía influência funesta.
\section{Infortunadamente}
\begin{itemize}
\item {Grp. gram.:adv.}
\end{itemize}
De modo infortunado, desgraçadamente.
\section{Infortunado}
\begin{itemize}
\item {Grp. gram.:adj.}
\end{itemize}
\begin{itemize}
\item {Proveniência:(De \textunderscore infortunar\textunderscore )}
\end{itemize}
Que não tem fortuna; infeliz.
\section{Infortunar}
\begin{itemize}
\item {Grp. gram.:v. t.}
\end{itemize}
\begin{itemize}
\item {Proveniência:(Lat. \textunderscore infortunare\textunderscore )}
\end{itemize}
Tornar infeliz; causar infortúnio a.
\section{Infortunidade}
\begin{itemize}
\item {Grp. gram.:f.}
\end{itemize}
\begin{itemize}
\item {Utilização:Des.}
\end{itemize}
\begin{itemize}
\item {Proveniência:(Lat. \textunderscore infortunitas\textunderscore )}
\end{itemize}
O mesmo que \textunderscore infortúnio\textunderscore .
\section{Infortúnio}
\begin{itemize}
\item {Grp. gram.:m.}
\end{itemize}
\begin{itemize}
\item {Proveniência:(Lat. \textunderscore infortunium\textunderscore )}
\end{itemize}
Infelicidade.
Calamidade.
Desventura.
\section{Infortunoso}
\begin{itemize}
\item {Grp. gram.:adj.}
\end{itemize}
\begin{itemize}
\item {Proveniência:(De \textunderscore in...\textunderscore  + \textunderscore fortunoso\textunderscore )}
\end{itemize}
Que não é fortunoso.
\section{Infra...}
\begin{itemize}
\item {Grp. gram.:pref.}
\end{itemize}
\begin{itemize}
\item {Proveniência:(Lat. \textunderscore infra\textunderscore )}
\end{itemize}
Designativo de \textunderscore abaixo\textunderscore  ou na \textunderscore parte inferior\textunderscore .
\section{Infracção}
\begin{itemize}
\item {Grp. gram.:f.}
\end{itemize}
\begin{itemize}
\item {Proveniência:(Lat. \textunderscore infractio\textunderscore )}
\end{itemize}
Acto ou effeito de infringir.
\section{Infracretáceo}
\begin{itemize}
\item {Grp. gram.:adj.}
\end{itemize}
\begin{itemize}
\item {Utilização:Geol.}
\end{itemize}
\begin{itemize}
\item {Proveniência:(De \textunderscore infra...\textunderscore  + \textunderscore cretáceo\textunderscore )}
\end{itemize}
Que está abaixo da camada cretácea.
\section{Infracto}
\begin{itemize}
\item {Grp. gram.:adj.}
\end{itemize}
\begin{itemize}
\item {Utilização:Poét.}
\end{itemize}
\begin{itemize}
\item {Proveniência:(Lat. \textunderscore infractus\textunderscore )}
\end{itemize}
Quebrado.
Alquebrado, abatido.
\section{Infractor}
\begin{itemize}
\item {Grp. gram.:m.}
\end{itemize}
\begin{itemize}
\item {Proveniência:(Lat. \textunderscore infractor\textunderscore )}
\end{itemize}
Aquelle que infringe.
\section{Infra-escavação}
\begin{itemize}
\item {Grp. gram.:f.}
\end{itemize}
Cavidade, produzida ordinariamente pela fôrça da água corrente, junto da base dos pègões.
\section{Infrajurássico}
\begin{itemize}
\item {Grp. gram.:adj.}
\end{itemize}
\begin{itemize}
\item {Utilização:Geol.}
\end{itemize}
\begin{itemize}
\item {Proveniência:(De \textunderscore infra...\textunderscore  + \textunderscore jurássico\textunderscore )}
\end{itemize}
Diz-se do terreno, situado abaixo do jurássico.
\section{Infralapsário}
\begin{itemize}
\item {Grp. gram.:m.}
\end{itemize}
\begin{itemize}
\item {Proveniência:(Do lat. \textunderscore infra\textunderscore  + \textunderscore lapsus\textunderscore )}
\end{itemize}
Sectário da doutrina dos que sustentam que Deus, depois do peccado do primeiro homem, destinara á condemnação certo número de vindoiros.
\section{Infralativo}
\begin{itemize}
\item {Grp. gram.:m.}
\end{itemize}
\begin{itemize}
\item {Utilização:Des.}
\end{itemize}
Grau inferior, por opposição a superlativo. Cf. Cortesão, \textunderscore Subs.\textunderscore 
\section{Infraliásico}
\begin{itemize}
\item {Grp. gram.:adj.}
\end{itemize}
\begin{itemize}
\item {Utilização:Geol.}
\end{itemize}
\begin{itemize}
\item {Proveniência:(De \textunderscore infra...\textunderscore  + \textunderscore liásico\textunderscore )}
\end{itemize}
Diz-se do terreno, que fica abaixo do liásico.
\section{Infrangível}
\begin{itemize}
\item {Grp. gram.:adj.}
\end{itemize}
\begin{itemize}
\item {Proveniência:(Do lat. \textunderscore in...\textunderscore  + \textunderscore frangere\textunderscore )}
\end{itemize}
Que se não póde quebrar.
\section{Infra-oitava}
\begin{itemize}
\item {Grp. gram.:f.}
\end{itemize}
\begin{itemize}
\item {Utilização:T. eccles}
\end{itemize}
Dias, comprehendidos entre uma festa e a sua oitava.
\section{Infrascrito}
\begin{itemize}
\item {Grp. gram.:adj.}
\end{itemize}
\begin{itemize}
\item {Proveniência:(Lat. \textunderscore infrascriptus\textunderscore )}
\end{itemize}
Escrito abaixo daquillo que se está tratando.
\section{Infrene}
\begin{itemize}
\item {Grp. gram.:adj.}
\end{itemize}
\begin{itemize}
\item {Utilização:Fig.}
\end{itemize}
\begin{itemize}
\item {Proveniência:(Lat. \textunderscore infrenis\textunderscore )}
\end{itemize}
Desenfreado; desordenado; descommedido: \textunderscore paixões infrenes\textunderscore .
\section{Infrequência}
\begin{itemize}
\item {fónica:cu-en}
\end{itemize}
\begin{itemize}
\item {Grp. gram.:f.}
\end{itemize}
\begin{itemize}
\item {Proveniência:(Lat. \textunderscore infrequentia\textunderscore )}
\end{itemize}
Falta de frequência.
\section{Infrequentado}
\begin{itemize}
\item {fónica:cu-en}
\end{itemize}
\begin{itemize}
\item {Grp. gram.:adj.}
\end{itemize}
\begin{itemize}
\item {Proveniência:(Lat. \textunderscore infrequentatus\textunderscore )}
\end{itemize}
Que não é frequentado: \textunderscore ruas infrequentadas\textunderscore .
\section{Infrequente}
\begin{itemize}
\item {fónica:cu-en}
\end{itemize}
\begin{itemize}
\item {Grp. gram.:adj.}
\end{itemize}
\begin{itemize}
\item {Proveniência:(Lat. \textunderscore infrequens\textunderscore )}
\end{itemize}
Que não é frequente.
\section{Infrequentemente}
\begin{itemize}
\item {fónica:cu-en}
\end{itemize}
\begin{itemize}
\item {Grp. gram.:adv.}
\end{itemize}
De modo frequente; sem frequência.
\section{Infringente}
\begin{itemize}
\item {Grp. gram.:adj.}
\end{itemize}
\begin{itemize}
\item {Proveniência:(Lat. \textunderscore infringens\textunderscore )}
\end{itemize}
Que infringe.
\section{Infringir}
\begin{itemize}
\item {Grp. gram.:v. t.}
\end{itemize}
\begin{itemize}
\item {Proveniência:(Lat. \textunderscore infringere\textunderscore )}
\end{itemize}
Quebrantar; postergar; transgredir: \textunderscore infringir a lei\textunderscore .
\section{Infringível}
\begin{itemize}
\item {Grp. gram.:adj.}
\end{itemize}
Que se não póde infringir.
\section{Infructescência}
\begin{itemize}
\item {Grp. gram.:f.}
\end{itemize}
\begin{itemize}
\item {Utilização:Bot.}
\end{itemize}
\begin{itemize}
\item {Proveniência:(Do lat. \textunderscore fruetescere\textunderscore )}
\end{itemize}
Forma de fructificação, constituida por mais de um fruto.
\section{Infructescente}
\begin{itemize}
\item {Grp. gram.:adj.}
\end{itemize}
\begin{itemize}
\item {Utilização:Bot.}
\end{itemize}
Em que há infructescência.
\section{Infructífero}
\begin{itemize}
\item {Grp. gram.:adj.}
\end{itemize}
\begin{itemize}
\item {Proveniência:(Lat. \textunderscore infructifer\textunderscore )}
\end{itemize}
Que não dá fruto.
Que não dá resultado; infecundo; frustrado; inútil: \textunderscore trabalho infructífero\textunderscore .
\section{Infructuosamente}
\begin{itemize}
\item {Grp. gram.:adv.}
\end{itemize}
De modo infructuoso.
\section{Infructuosidade}
\begin{itemize}
\item {Grp. gram.:f.}
\end{itemize}
Qualidade de infructuoso.
\section{Infructuoso}
\begin{itemize}
\item {Grp. gram.:adj.}
\end{itemize}
\begin{itemize}
\item {Proveniência:(Lat. \textunderscore infructuosus\textunderscore )}
\end{itemize}
Que não tem fruto.
Infructífero; inútil.
\section{Infrutescência}
\begin{itemize}
\item {Grp. gram.:f.}
\end{itemize}
\begin{itemize}
\item {Utilização:Bot.}
\end{itemize}
\begin{itemize}
\item {Proveniência:(Do lat. \textunderscore fruetescere\textunderscore )}
\end{itemize}
Forma de frutificação, constituida por mais de um fruto.
\section{Infrutescente}
\begin{itemize}
\item {Grp. gram.:adj.}
\end{itemize}
\begin{itemize}
\item {Utilização:Bot.}
\end{itemize}
Em que há infrutescência.
\section{Infrutífero}
\begin{itemize}
\item {Grp. gram.:adj.}
\end{itemize}
\begin{itemize}
\item {Proveniência:(Lat. \textunderscore infructifer\textunderscore )}
\end{itemize}
Que não dá fruto.
Que não dá resultado; infecundo; frustrado; inútil: \textunderscore trabalho infrutífero\textunderscore .
\section{Infrutuosamente}
\begin{itemize}
\item {Grp. gram.:adv.}
\end{itemize}
De modo infrutuoso.
\section{Infrutuosidade}
\begin{itemize}
\item {Grp. gram.:f.}
\end{itemize}
Qualidade de infrutuoso.
\section{Infrutuoso}
\begin{itemize}
\item {Grp. gram.:adj.}
\end{itemize}
\begin{itemize}
\item {Proveniência:(Lat. \textunderscore infructuosus\textunderscore )}
\end{itemize}
Que não tem fruto.
Infrutífero; inútil.
\section{Ínfula}
\begin{itemize}
\item {Grp. gram.:f.}
\end{itemize}
\begin{itemize}
\item {Proveniência:(Lat. \textunderscore infula\textunderscore )}
\end{itemize}
Faixa franjada, insígnia dos sacerdotes de Apollo.
\section{Infumígeno}
\begin{itemize}
\item {Grp. gram.:adj.}
\end{itemize}
\begin{itemize}
\item {Utilização:Neol.}
\end{itemize}
\begin{itemize}
\item {Proveniência:(De \textunderscore in...\textunderscore  + \textunderscore fumo\textunderscore  + gr. \textunderscore genos\textunderscore )}
\end{itemize}
Que não deita fumo.
Diz-se especialmente da pólvora, que arde sem fazer fumo.
\section{Infulminável}
\begin{itemize}
\item {Grp. gram.:adj.}
\end{itemize}
\begin{itemize}
\item {Proveniência:(De \textunderscore in...\textunderscore  + \textunderscore fulminar\textunderscore )}
\end{itemize}
Que não póde sêr fulminado.
\section{Infumável}
\begin{itemize}
\item {Grp. gram.:adj.}
\end{itemize}
\begin{itemize}
\item {Proveniência:(De \textunderscore in...\textunderscore  + \textunderscore fumável\textunderscore )}
\end{itemize}
Que não é bom para se fumar: \textunderscore cigarros infumáveis\textunderscore .
\section{Infumo}
\begin{itemize}
\item {Grp. gram.:m.}
\end{itemize}
O mesmo que \textunderscore dembo\textunderscore .
\section{Infundado}
\begin{itemize}
\item {Grp. gram.:adj.}
\end{itemize}
\begin{itemize}
\item {Proveniência:(De \textunderscore in...\textunderscore  + \textunderscore fundado\textunderscore )}
\end{itemize}
Que não é fundado; que não tem fundamento ou razão de sêr: \textunderscore accusações infundadas\textunderscore .
\section{Infunde}
\begin{itemize}
\item {Grp. gram.:m.}
\end{itemize}
Massa de mandioca com môlho, us. na África occidental portuguesa.
\section{Infúndi}
\begin{itemize}
\item {Grp. gram.:m.}
\end{itemize}
(V.infunde)
\section{Infundibuliforme}
\begin{itemize}
\item {Grp. gram.:adj.}
\end{itemize}
\begin{itemize}
\item {Proveniência:(Lat. \textunderscore infundibuliformis\textunderscore )}
\end{itemize}
Que tem fórma de funil.
\section{Infundíbulo}
\begin{itemize}
\item {Grp. gram.:m.}
\end{itemize}
\begin{itemize}
\item {Proveniência:(Lat. \textunderscore infundibulum\textunderscore )}
\end{itemize}
O mesmo que \textunderscore funil\textunderscore .
\section{Infundiça}
\begin{itemize}
\item {Grp. gram.:f.}
\end{itemize}
\begin{itemize}
\item {Proveniência:(Do rad. de \textunderscore infundir\textunderscore )}
\end{itemize}
Barrela, feita de urina, em que se põe de môlho a roupa muito suja, para depois se lavar mais facilmente.
\section{Infundice}
\begin{itemize}
\item {Grp. gram.:f.}
\end{itemize}
\begin{itemize}
\item {Proveniência:(Do rad. de \textunderscore infundir\textunderscore )}
\end{itemize}
Barrela, feita de urina, em que se põe de môlho a roupa muito suja, para depois se lavar mais facilmente.
\section{Infundir}
\begin{itemize}
\item {Grp. gram.:v. t.}
\end{itemize}
\begin{itemize}
\item {Proveniência:(Lat. \textunderscore infundere\textunderscore )}
\end{itemize}
Deitar dentro.
Misturar.
Entornar; derramar.
Insinuar; incutir; insuflar: \textunderscore infundir receios\textunderscore .
Pôr de infusão, meter num líquido: \textunderscore infundir roupa na lixívia\textunderscore .
\section{Infunicar}
\begin{itemize}
\item {Grp. gram.:v. t.}
\end{itemize}
\begin{itemize}
\item {Utilização:Chul.}
\end{itemize}
Desfigurar; mascarar.
\section{Infurção}
\begin{itemize}
\item {Grp. gram.:f.}
\end{itemize}
\begin{itemize}
\item {Utilização:Ant.}
\end{itemize}
Tributo.
Pagamento de rendeiro.
\section{Infusa}
\begin{itemize}
\item {Grp. gram.:f.}
\end{itemize}
\begin{itemize}
\item {Proveniência:(De \textunderscore infuso\textunderscore )}
\end{itemize}
Espécie de bilha.
\section{Infusão}
\begin{itemize}
\item {Grp. gram.:f.}
\end{itemize}
\begin{itemize}
\item {Proveniência:(Lat. \textunderscore infusio\textunderscore )}
\end{itemize}
Acto ou effeito de infundir.
Conservação, por algum tempo, de uma substância em água quente ou em líquido análogo, para lhe extrahir princípios medicamentosos ou outra substância, agradável ou alimentícia.
Maceração pharmacêutica.
\section{Infusibilidade}
\begin{itemize}
\item {Grp. gram.:f.}
\end{itemize}
Qualidade daquillo que é infusível.
\section{Infusível}
\begin{itemize}
\item {Grp. gram.:adj.}
\end{itemize}
\begin{itemize}
\item {Proveniência:(De \textunderscore in...\textunderscore  + \textunderscore fusível\textunderscore )}
\end{itemize}
Que não é fusível; que se não derrete.
\section{Infuso}
\begin{itemize}
\item {Grp. gram.:m.}
\end{itemize}
\begin{itemize}
\item {Proveniência:(Lat. \textunderscore infusus\textunderscore )}
\end{itemize}
Producto medicamentoso de uma infusão.
Líquido, em que se faz infusão.
\section{Infusórios}
\begin{itemize}
\item {Grp. gram.:m. pl.}
\end{itemize}
\begin{itemize}
\item {Utilização:Zool.}
\end{itemize}
\begin{itemize}
\item {Proveniência:(Lat. \textunderscore infusorium\textunderscore )}
\end{itemize}
Classe de animálculos, que se desenvolvem em infusões, e que só pelo microscópio se pódem observar.
Fósseis microscópicos, que se encontram nas águas doces e marítimas.
\section{Infustamento}
\begin{itemize}
\item {Grp. gram.:m.}
\end{itemize}
\begin{itemize}
\item {Proveniência:(Do b. lat. \textunderscore fustalia\textunderscore )}
\end{itemize}
Cheiro peculiar ás vasilhas de vinho.
\section{Infusura}
\begin{itemize}
\item {Grp. gram.:f.}
\end{itemize}
\begin{itemize}
\item {Utilização:Veter.}
\end{itemize}
\begin{itemize}
\item {Proveniência:(De \textunderscore infuso\textunderscore )}
\end{itemize}
Fluxão mórbida de humores, nos quadrúpedes.
\section{Inga}
\begin{itemize}
\item {Grp. gram.:m.}
\end{itemize}
Designação de várias plantas leguminosas da Ásia e da América, cuja casca é tónica e medicinal.
\section{Ingá}
\begin{itemize}
\item {Grp. gram.:m.}
\end{itemize}
\begin{itemize}
\item {Grp. gram.:m.  ou  f.}
\end{itemize}
Planta brasileira. Cf. Crespo, \textunderscore Miniat.\textunderscore , 16.
Fruto da ingazeira.
Ingazeira.
(Do tupi)
\section{Ingaí}
\begin{itemize}
\item {Grp. gram.:m.}
\end{itemize}
\begin{itemize}
\item {Utilização:Bras}
\end{itemize}
Árvore silvestre, de madeira amarela.
\section{Ingaíbos}
\begin{itemize}
\item {Grp. gram.:m. pl.}
\end{itemize}
\begin{itemize}
\item {Utilização:Bras}
\end{itemize}
Tríbo de aborígenes do Pará.
\section{Inganhável}
\begin{itemize}
\item {Grp. gram.:adj.}
\end{itemize}
\begin{itemize}
\item {Proveniência:(De \textunderscore in...\textunderscore  + \textunderscore ganhável\textunderscore )}
\end{itemize}
Que não é ganhável.
\section{Ingarilho}
\begin{itemize}
\item {Grp. gram.:m.}
\end{itemize}
\begin{itemize}
\item {Utilização:Prov.}
\end{itemize}
\begin{itemize}
\item {Utilização:trasm.}
\end{itemize}
Janota magro e presumido; bonifatre. Cf. Camillo, \textunderscore Filha do Arcediago\textunderscore .
\section{Ingazeira}
\begin{itemize}
\item {Grp. gram.:f.}
\end{itemize}
\begin{itemize}
\item {Utilização:Bras}
\end{itemize}
\begin{itemize}
\item {Proveniência:(De \textunderscore ingá\textunderscore )}
\end{itemize}
Árvore leguminosa da América.
\section{Ingazeiro}
\begin{itemize}
\item {Grp. gram.:m.}
\end{itemize}
O mesmo que \textunderscore ingazeira\textunderscore . Cf. Crespo, \textunderscore Miniaturas\textunderscore , 116.
\section{Inge}
\begin{itemize}
\item {Grp. gram.:m.}
\end{itemize}
\begin{itemize}
\item {Utilização:Ant.}
\end{itemize}
O mesmo que \textunderscore ingu\textunderscore .
\section{Ingeminar}
\textunderscore v. t.\textunderscore  (e der.)
(Corr. pop. de \textunderscore examinar\textunderscore , etc.)
\section{Ingenho}
\textunderscore m.\textunderscore  (e der.)
(V. \textunderscore engenho\textunderscore , etc.)
\section{Ingenioso}
\begin{itemize}
\item {Grp. gram.:adj.}
\end{itemize}
\begin{itemize}
\item {Utilização:Ant.}
\end{itemize}
O mesmo que \textunderscore engenhoso\textunderscore . Cf. \textunderscore Usque\textunderscore , 20.
\section{Ingénito}
\begin{itemize}
\item {Grp. gram.:adj.}
\end{itemize}
\begin{itemize}
\item {Proveniência:(Lat. \textunderscore ingenitus\textunderscore )}
\end{itemize}
Não gerado; innato.
Que nasceu com o indivíduo: \textunderscore bondade ingênita\textunderscore .
\section{Ingente}
\begin{itemize}
\item {Grp. gram.:adj.}
\end{itemize}
\begin{itemize}
\item {Proveniência:(Lat. \textunderscore ingens\textunderscore )}
\end{itemize}
Grande; enorme; desmedido.
Estrondoso.
\section{Inábil}
\begin{itemize}
\item {Grp. gram.:adj.}
\end{itemize}
\begin{itemize}
\item {Proveniência:(Lat. \textunderscore inhabilis\textunderscore )}
\end{itemize}
Não hábil; que não tem aptidão ou competencia; incapaz.
\section{Inabilidade}
\begin{itemize}
\item {Grp. gram.:f.}
\end{itemize}
\begin{itemize}
\item {Proveniência:(De \textunderscore in...\textunderscore  + \textunderscore habilidade\textunderscore )}
\end{itemize}
Qualidade de inábil; falta de habilidade.
\section{Inabilitação}
\begin{itemize}
\item {Grp. gram.:f.}
\end{itemize}
Acto ou efeito de inabilitar.
\section{Inabilitar}
\begin{itemize}
\item {Grp. gram.:v. t.}
\end{itemize}
\begin{itemize}
\item {Proveniência:(De \textunderscore in...\textunderscore  + \textunderscore habilitar\textunderscore )}
\end{itemize}
Tornar inábil.
Impedir.
Tirar a faculdade ou certos meios a.
\section{Inabilmente}
\begin{itemize}
\item {Grp. gram.:adv.}
\end{itemize}
De modo inábil.
\section{Inabitado}
\begin{itemize}
\item {Grp. gram.:adj.}
\end{itemize}
\begin{itemize}
\item {Proveniência:(Lat. \textunderscore inhabitatus\textunderscore )}
\end{itemize}
Não habitado.
\section{Inabitável}
\begin{itemize}
\item {Grp. gram.:adj.}
\end{itemize}
\begin{itemize}
\item {Proveniência:(Lat. \textunderscore inhabitabilis\textunderscore )}
\end{itemize}
Que não póde sêr habitado.
\section{Inabitual}
\begin{itemize}
\item {Grp. gram.:adj.}
\end{itemize}
\begin{itemize}
\item {Proveniência:(De \textunderscore in...\textunderscore  + \textunderscore habitual\textunderscore )}
\end{itemize}
Que não é habitual.
\section{Inalação}
\begin{itemize}
\item {Grp. gram.:f.}
\end{itemize}
\begin{itemize}
\item {Proveniência:(Lat. \textunderscore inhalatio\textunderscore )}
\end{itemize}
Acto ou efeito de inalar.
\section{Inalador}
\begin{itemize}
\item {Grp. gram.:m.  e  adj.}
\end{itemize}
\begin{itemize}
\item {Proveniência:(De \textunderscore inalar\textunderscore )}
\end{itemize}
Aquilo que é próprio para inalações.
\section{Inalante}
\begin{itemize}
\item {Grp. gram.:adj.}
\end{itemize}
\begin{itemize}
\item {Proveniência:(Lat. \textunderscore inhalans\textunderscore )}
\end{itemize}
Que inala.
\section{Inalar}
\begin{itemize}
\item {Grp. gram.:v. t.}
\end{itemize}
\begin{itemize}
\item {Proveniência:(Lat. \textunderscore inhalare\textunderscore )}
\end{itemize}
Absorver com o hálito; aspirar.
Receber.
\section{Inarmonia}
\begin{itemize}
\item {Grp. gram.:f.}
\end{itemize}
\begin{itemize}
\item {Proveniência:(De \textunderscore in...\textunderscore  + \textunderscore harmonia\textunderscore )}
\end{itemize}
O mesmo que \textunderscore desarmonia\textunderscore .
\section{Inarmonicamente}
\begin{itemize}
\item {Grp. gram.:adv.}
\end{itemize}
De modo inarmónico.
\section{Inarmónico}
\begin{itemize}
\item {Grp. gram.:adj.}
\end{itemize}
\begin{itemize}
\item {Proveniência:(De \textunderscore in...\textunderscore  + \textunderscore harmónico\textunderscore )}
\end{itemize}
Em que não há harmonia.
\section{Inerência}
\begin{itemize}
\item {Grp. gram.:f.}
\end{itemize}
Qualidade daquilo que é inerente.
\section{Inerente}
\begin{itemize}
\item {Grp. gram.:adj.}
\end{itemize}
\begin{itemize}
\item {Proveniência:(Lat. \textunderscore inhaerens\textunderscore )}
\end{itemize}
Que inere; ligado naturalmente; inseparável: \textunderscore encargos inerentes a um lugar público\textunderscore .
\section{Inerentemente}
\begin{itemize}
\item {Grp. gram.:adv.}
\end{itemize}
De modo inerente.
\section{Inerir}
\begin{itemize}
\item {Grp. gram.:v. i.}
\end{itemize}
\begin{itemize}
\item {Proveniência:(Lat. \textunderscore inhaerere\textunderscore )}
\end{itemize}
Estar ligado intimamente; sêr inseparável.
\section{Ingenuação}
\begin{itemize}
\item {Grp. gram.:f.}
\end{itemize}
\begin{itemize}
\item {Utilização:Ant.}
\end{itemize}
Acto ou effeito de ingenuar.
Libertação; alforria.
Isenção de impostos e das mais condições de servo. Cf. Herculano, \textunderscore Hist. de Port.\textunderscore , III, 286.
(B. lat. \textunderscore ingenuatio\textunderscore )
\section{Ingenuar}
\begin{itemize}
\item {Grp. gram.:v. t.}
\end{itemize}
\begin{itemize}
\item {Utilização:Ant.}
\end{itemize}
Dar liberdade a.
Dar carta de foral a.
(B. lat. \textunderscore ingenuare\textunderscore )
\section{Ingénua}
\begin{itemize}
\item {Grp. gram.:f.}
\end{itemize}
\begin{itemize}
\item {Proveniência:(De \textunderscore ingénuo\textunderscore )}
\end{itemize}
Actriz, cujo papel é caracterizado pela ingenuidade e juventude.
\section{Ingênua}
\begin{itemize}
\item {Grp. gram.:f.}
\end{itemize}
\begin{itemize}
\item {Proveniência:(De \textunderscore ingênuo\textunderscore )}
\end{itemize}
Actriz, cujo papel é caracterizado pela ingenuidade e juventude.
\section{Ingenuamente}
\begin{itemize}
\item {Grp. gram.:adv.}
\end{itemize}
\begin{itemize}
\item {Proveniência:(De \textunderscore ingênuo\textunderscore )}
\end{itemize}
Com ingenuidade.
\section{Ingenuidade}
\begin{itemize}
\item {fónica:nu-i}
\end{itemize}
\begin{itemize}
\item {Grp. gram.:f.}
\end{itemize}
\begin{itemize}
\item {Utilização:Ant.}
\end{itemize}
\begin{itemize}
\item {Proveniência:(Lat. \textunderscore ingenuitas\textunderscore )}
\end{itemize}
Qualidade daquelle ou aquillo que é ingênuo.
Simplicidade extrema.
O mesmo que \textunderscore ingenuação\textunderscore . Cf. Herculano, \textunderscore Hist. de Port.\textunderscore , IV, 41.
\section{Ingénuo}
\begin{itemize}
\item {Grp. gram.:adj.}
\end{itemize}
\begin{itemize}
\item {Utilização:Ant.}
\end{itemize}
\begin{itemize}
\item {Grp. gram.:M.}
\end{itemize}
\begin{itemize}
\item {Proveniência:(Lat. \textunderscore ingenuus\textunderscore )}
\end{itemize}
Natural; simples.
Em que não há artifício ou malícia.
Innocente.
Puro, sem mistura de sangue plebeu:«\textunderscore antiga varonia, que depois de mil annos se acha ingenua.\textunderscore »P. Carvalho, \textunderscore Corogr. Port.\textunderscore , I, 391.
Indivíduo ingénuo.
\section{Ingênuo}
\begin{itemize}
\item {Grp. gram.:adj.}
\end{itemize}
\begin{itemize}
\item {Utilização:Ant.}
\end{itemize}
\begin{itemize}
\item {Grp. gram.:M.}
\end{itemize}
\begin{itemize}
\item {Proveniência:(Lat. \textunderscore ingenuus\textunderscore )}
\end{itemize}
Natural; simples.
Em que não há artifício ou malícia.
Innocente.
Puro, sem mistura de sangue plebeu:«\textunderscore antiga varonia, que depois de mil annos se acha ingenua.\textunderscore »P. Carvalho, \textunderscore Corogr. Port.\textunderscore , I, 391.
Indivíduo ingênuo.
\section{Ingerência}
\begin{itemize}
\item {Grp. gram.:f.}
\end{itemize}
Acto ou effeito de ingerir.
Intervenção; influência.
\section{Ingerimento}
\begin{itemize}
\item {Grp. gram.:m.}
\end{itemize}
O mesmo que \textunderscore ingestão\textunderscore . Cf. Aguiar, \textunderscore Proc. de Vin.\textunderscore , 16.
\section{Ingerir}
\begin{itemize}
\item {Grp. gram.:v. t.}
\end{itemize}
\begin{itemize}
\item {Grp. gram.:V. p.}
\end{itemize}
\begin{itemize}
\item {Proveniência:(Lat. \textunderscore ingerere\textunderscore )}
\end{itemize}
Introduzir.
Engulir.
Introduzir-se.
Intervir.
\section{Ingestão}
\begin{itemize}
\item {Grp. gram.:f.}
\end{itemize}
\begin{itemize}
\item {Proveniência:(Lat. \textunderscore ingestio\textunderscore )}
\end{itemize}
Acto de ingerir; deglutição.
\section{Inglelê}
\begin{itemize}
\item {Grp. gram.:m.}
\end{itemize}
Árvore medicinal da ilha de San-Thomé.
\section{Inglês}
\begin{itemize}
\item {Grp. gram.:adj.}
\end{itemize}
\begin{itemize}
\item {Grp. gram.:M.}
\end{itemize}
\begin{itemize}
\item {Proveniência:(Ingl. \textunderscore english\textunderscore . Cp. lat. \textunderscore angli\textunderscore )}
\end{itemize}
Relativo á Inglaterra.
Indivíduo, natural da Inglaterra ou alli naturalizado.
A língua dos ingleses.
\section{Inglesa}
\begin{itemize}
\item {Grp. gram.:f}
\end{itemize}
\begin{itemize}
\item {Proveniência:(De \textunderscore inglês\textunderscore )}
\end{itemize}
Espécie de dança. Cf. Camillo, \textunderscore Noites de Insómn.\textunderscore , VII, 80.
\section{Inglesamente}
\begin{itemize}
\item {Grp. gram.:adv.}
\end{itemize}
\begin{itemize}
\item {Proveniência:(De \textunderscore inglês\textunderscore )}
\end{itemize}
Á maneira dos Inglêses.
\section{Inglesar}
\begin{itemize}
\item {Grp. gram.:v. t.}
\end{itemize}
\begin{itemize}
\item {Proveniência:(De \textunderscore inglês\textunderscore )}
\end{itemize}
Dar feição inglesa a.
\section{Inglesia}
\begin{itemize}
\item {Grp. gram.:f.}
\end{itemize}
\begin{itemize}
\item {Proveniência:(De \textunderscore inglês\textunderscore )}
\end{itemize}
O mesmo que \textunderscore ingresia\textunderscore , mas menos usado.
\section{Inglesismo}
\begin{itemize}
\item {Grp. gram.:m.}
\end{itemize}
O mesmo que \textunderscore anglicismo\textunderscore . Cf. Filinto, XVIII, 160.
\section{Inglesmente}
\begin{itemize}
\item {Grp. gram.:adv.}
\end{itemize}
O mesmo ou melhor que \textunderscore inglesamente\textunderscore . Cp. \textunderscore portuguêsmente\textunderscore .
\section{Ingloriamente}
\begin{itemize}
\item {Grp. gram.:adv.}
\end{itemize}
De modo inglório; obscuramente.
\section{Inglório}
\begin{itemize}
\item {Grp. gram.:adj.}
\end{itemize}
\begin{itemize}
\item {Utilização:Ext.}
\end{itemize}
\begin{itemize}
\item {Proveniência:(Lat. \textunderscore inglorius\textunderscore )}
\end{itemize}
Em que não há glória; que não dá glória.
Modesto; obscuro.
\section{Ingloriosamente}
\begin{itemize}
\item {Grp. gram.:adv.}
\end{itemize}
O mesmo que \textunderscore ingloriamente\textunderscore .
\section{Inglorioso}
\begin{itemize}
\item {Grp. gram.:adj.}
\end{itemize}
O mesmo que \textunderscore inglório\textunderscore .
(B. lat. \textunderscore ingloriosus\textunderscore )
\section{Ingluvial}
\begin{itemize}
\item {Grp. gram.:adj.}
\end{itemize}
\begin{itemize}
\item {Proveniência:(Do lat. \textunderscore ingluvies\textunderscore , papo)}
\end{itemize}
Relativo ao papo das aves.
Diz-se especialmente da indigestão, peculiar a certas aves, e que se manifesta pela dilatação e dureza do papo.
\section{Inglúvias}
\begin{itemize}
\item {Grp. gram.:f. pl.}
\end{itemize}
\begin{itemize}
\item {Proveniência:(Lat. \textunderscore ingluvies\textunderscore )}
\end{itemize}
Papo ou primeiro estômago das aves.
Garganta.
Espaço, entre os ramos da maxilla inferior e a larynge, nos mammíferos.
\section{Ingluviosamente}
\begin{itemize}
\item {Grp. gram.:adv.}
\end{itemize}
De modo ingluvioso.
\section{Ingluvioso}
\begin{itemize}
\item {Grp. gram.:adj.}
\end{itemize}
\begin{itemize}
\item {Proveniência:(Lat. \textunderscore inglobiosus\textunderscore )}
\end{itemize}
Que come muito.
Voraz.
\section{Ingonha}
\begin{itemize}
\item {Grp. gram.:f.}
\end{itemize}
Bebida agradável, que os Negros da Senegâmbia extrahem de um fruto semelhante ao alperche.
\section{Ingovernável}
\begin{itemize}
\item {Grp. gram.:adj.}
\end{itemize}
\begin{itemize}
\item {Proveniência:(De \textunderscore in...\textunderscore  + \textunderscore governável\textunderscore )}
\end{itemize}
Que se não póde governar.
Indisciplinável; insubmisso: \textunderscore um rapaz ingovernável\textunderscore .
\section{Ingraciosamente}
\begin{itemize}
\item {Grp. gram.:adv.}
\end{itemize}
O mesmo que \textunderscore desgraciosamente\textunderscore .
\section{Ingracioso}
\begin{itemize}
\item {Grp. gram.:adj.}
\end{itemize}
O mesmo que \textunderscore desgracioso\textunderscore .
\section{Ingramatical}
\begin{itemize}
\item {Grp. gram.:adj.}
\end{itemize}
\begin{itemize}
\item {Proveniência:(De \textunderscore in...\textunderscore  + \textunderscore gramatical\textunderscore )}
\end{itemize}
Contrário ás regras da gramática.
\section{Ingrammatical}
\begin{itemize}
\item {Grp. gram.:adj.}
\end{itemize}
\begin{itemize}
\item {Proveniência:(De \textunderscore in...\textunderscore  + \textunderscore grammatical\textunderscore )}
\end{itemize}
Contrário ás regras da grammática.
\section{Ingranzeu}
\begin{itemize}
\item {Grp. gram.:m.}
\end{itemize}
\begin{itemize}
\item {Utilização:Pop.}
\end{itemize}
O mesmo que \textunderscore ingresia\textunderscore .
\section{Ingrão}
\begin{itemize}
\item {Grp. gram.:m.}
\end{itemize}
Espécie de centeio branco.
(Cp. \textunderscore ingre\textunderscore ^2)
\section{Ingratamente}
\begin{itemize}
\item {Grp. gram.:adv.}
\end{itemize}
\begin{itemize}
\item {Proveniência:(De \textunderscore ingrato\textunderscore )}
\end{itemize}
Com ingratidão.
\section{Ingratão}
\begin{itemize}
\item {Grp. gram.:m.}
\end{itemize}
\begin{itemize}
\item {Utilização:Fam.}
\end{itemize}
Homem muito ingrato.
\section{Ingratatão}
\begin{itemize}
\item {Grp. gram.:m.}
\end{itemize}
\begin{itemize}
\item {Utilização:Pop.}
\end{itemize}
O mesmo que \textunderscore ingratão\textunderscore .
\section{Ingratatona}
\begin{itemize}
\item {Grp. gram.:f.}
\end{itemize}
\begin{itemize}
\item {Utilização:Pop.}
\end{itemize}
O mesmo que \textunderscore ingratona\textunderscore .
\section{Ingratidão}
\begin{itemize}
\item {Grp. gram.:f.}
\end{itemize}
\begin{itemize}
\item {Proveniência:(Lat. \textunderscore ingratitudo\textunderscore )}
\end{itemize}
Qualidade de quem é ingrato.
Falta de gratidão.
\section{Ingrato}
\begin{itemize}
\item {Grp. gram.:adj.}
\end{itemize}
\begin{itemize}
\item {Utilização:Fig.}
\end{itemize}
\begin{itemize}
\item {Grp. gram.:M.}
\end{itemize}
\begin{itemize}
\item {Proveniência:(Lat. \textunderscore ingratus\textunderscore )}
\end{itemize}
Que não é aprazível; desagradável.
Molesto.
Estéril, improductivo.
Que não é agradecido, que não reconhece o benefício que recebeu.
Individuo, que se esqueceu de benefícios recebidos ou que offende quem o beneficiou.
\section{Ingratona}
\begin{itemize}
\item {Grp. gram.:f.}
\end{itemize}
\begin{itemize}
\item {Utilização:Fam.}
\end{itemize}
\begin{itemize}
\item {Proveniência:(De \textunderscore ingratão\textunderscore )}
\end{itemize}
Mulher muito ingrata. Cf. Garrett, \textunderscore Fábulas\textunderscore , 66.
\section{Ingre}
\begin{itemize}
\item {Grp. gram.:m.}
\end{itemize}
\begin{itemize}
\item {Utilização:Prov.}
\end{itemize}
\begin{itemize}
\item {Utilização:beir.}
\end{itemize}
\begin{itemize}
\item {Proveniência:(Do cast. \textunderscore ingle\textunderscore ?)}
\end{itemize}
Planta annual, de haste longa e lisa, o que produz uma espécie de baga tinctória, parecida á do sabugueiro.
\section{Ingre}
\begin{itemize}
\item {Grp. gram.:adj.}
\end{itemize}
\begin{itemize}
\item {Utilização:T. da Bairrada}
\end{itemize}
Diz-se das coisas, que se puseram a cozer e que ficaram rijas ou mal cozidas.
Diz-se da pedra solta, sem argamassa; applicada na formação de paredes.
\section{Ingrediente}
\begin{itemize}
\item {Grp. gram.:adj.}
\end{itemize}
\begin{itemize}
\item {Proveniência:(Lat. \textunderscore ingrediens\textunderscore )}
\end{itemize}
Substância, que faz parte de um medicamento, de uma iguaria, etc.
\section{Ingremancia}
\begin{itemize}
\item {Grp. gram.:f.}
\end{itemize}
\begin{itemize}
\item {Utilização:Pop.}
\end{itemize}
Exquisitice, ratice.
\section{Íngreme}
\begin{itemize}
\item {Grp. gram.:adj.}
\end{itemize}
\begin{itemize}
\item {Utilização:Fig.}
\end{itemize}
Escarpado; que tem grande declive.
Que é diffícil de subir.
Diffícil, trabalhoso.--Nas provincias, ouve-se amiúde ingríme, em vez da fórma culta ingreme.
\section{Ingrême}
\begin{itemize}
\item {Grp. gram.:adj.}
\end{itemize}
O mesmo que \textunderscore ingrime\textunderscore .
\section{Ingremidade}
\begin{itemize}
\item {Grp. gram.:f.}
\end{itemize}
Qualidade do que é íngrime.
\section{Ingremidez}
\begin{itemize}
\item {Grp. gram.:f.}
\end{itemize}
O mesmo que \textunderscore ingremidade\textunderscore .
\section{Ingrenço}
\begin{itemize}
\item {Grp. gram.:m.}
\end{itemize}
\begin{itemize}
\item {Utilização:Prov.}
\end{itemize}
\begin{itemize}
\item {Utilização:minh.}
\end{itemize}
Pessôa, que só serve para embaraçar os outros; empecilho.
\section{Ingrês}
\begin{itemize}
\item {Grp. gram.:m.  e  adj.}
\end{itemize}
\begin{itemize}
\item {Utilização:Ant.}
\end{itemize}
\begin{itemize}
\item {Grp. gram.:M.}
\end{itemize}
\begin{itemize}
\item {Grp. gram.:m.}
\end{itemize}
\begin{itemize}
\item {Utilização:Ant.}
\end{itemize}
O mesmo que \textunderscore inglês\textunderscore :«\textunderscore ...que parece muito ingrês num pelote portugues.\textunderscore »\textunderscore Anfitriões\textunderscore , act. I, sc. VI.
Espécie de tecido antigo:«\textunderscore chapeirão de ingrês com fila de momperle.\textunderscore »Herculano, \textunderscore Lendas\textunderscore , I, 96.
Variedade de pano. Cf. Herculano, \textunderscore Cister\textunderscore , 184.(V.engrês)
\section{Ingresia}
\begin{itemize}
\item {Grp. gram.:f.}
\end{itemize}
\begin{itemize}
\item {Proveniência:(De \textunderscore ingrês\textunderscore )}
\end{itemize}
Barulho; berreiro; alarido; falácia confusa.
Linguagem arrevesada e inintelligível.
\section{Ingressão}
\begin{itemize}
\item {Grp. gram.:f.}
\end{itemize}
\begin{itemize}
\item {Proveniência:(Lat. \textunderscore ingressio\textunderscore )}
\end{itemize}
O mesmo que ingresso.
Advento.
\section{Ingressar}
\begin{itemize}
\item {Grp. gram.:v. i.}
\end{itemize}
\begin{itemize}
\item {Utilização:Neol.}
\end{itemize}
Fazer ingresso; entrar.
\section{Ingresso}
\begin{itemize}
\item {Grp. gram.:m.}
\end{itemize}
\begin{itemize}
\item {Proveniência:(Lat. \textunderscore ingressus\textunderscore )}
\end{itemize}
Acto de entrar; entrada.
Admissão.
Introito; início.
\section{Ingrimanço}
\begin{itemize}
\item {Grp. gram.:m.}
\end{itemize}
O mesmo que \textunderscore ingresia\textunderscore . Cf. Filinto, II. 203.
\section{Ingrime}
\begin{itemize}
\item {Grp. gram.:m.}
\end{itemize}
\begin{itemize}
\item {Utilização:Prov.}
\end{itemize}
\begin{itemize}
\item {Utilização:minh.}
\end{itemize}
\begin{itemize}
\item {Grp. gram.:Adj.}
\end{itemize}
\begin{itemize}
\item {Utilização:Prov.}
\end{itemize}
\begin{itemize}
\item {Utilização:minh.}
\end{itemize}
Alho hortense, com um só dente.
Inteiriço, feito de uma só peça.
Diz-se do alho, que é formado de um só corpo e não de várias partes ou dentes.
Gordo, forte: \textunderscore rapaz ingrime\textunderscore .
\section{Ingu}
\begin{itemize}
\item {Grp. gram.:m.}
\end{itemize}
O mesmo que \textunderscore assa-fétida\textunderscore .
\section{Íngua}
\begin{itemize}
\item {Grp. gram.:f.}
\end{itemize}
\begin{itemize}
\item {Proveniência:(Do lat. \textunderscore inguen\textunderscore )}
\end{itemize}
Engurgitamento de glândula lymphática, na virilha ou no pescoço ou na axilla.
Bubão na virilha.
\section{Inguaçu}
\begin{itemize}
\item {Grp. gram.:m.}
\end{itemize}
\begin{itemize}
\item {Utilização:Bras}
\end{itemize}
Árvore silvestre, que se emprega em carpintaria.
\section{Inguarina}
\begin{itemize}
\item {Grp. gram.:f.}
\end{itemize}
\begin{itemize}
\item {Utilização:Prov.}
\end{itemize}
\begin{itemize}
\item {Utilização:trasm.}
\end{itemize}
\begin{itemize}
\item {Utilização:Deprec.}
\end{itemize}
Veste, mais ou menos semelhante a uma blusa.
Opa, sobrepelliz, etc.
\section{Inguefo}
\begin{itemize}
\item {fónica:gué}
\end{itemize}
\begin{itemize}
\item {Grp. gram.:m.}
\end{itemize}
Planta africana, trepadeira, de caule verde, flexível e frágil, (\textunderscore piper clussii\textunderscore , De-Cand.).
\section{Inguento}
\begin{itemize}
\item {fónica:gu-en}
\end{itemize}
\begin{itemize}
\item {Grp. gram.:m.}
\end{itemize}
\begin{itemize}
\item {Utilização:pop.}
\end{itemize}
\begin{itemize}
\item {Utilização:Ant.}
\end{itemize}
O mesmo que \textunderscore unguento\textunderscore . Cf. Usque, 41.
\section{Inguiba}
\begin{itemize}
\item {Grp. gram.:f.}
\end{itemize}
Árvore brasileira, própria para construcções.
\section{Inguina}
\begin{itemize}
\item {Grp. gram.:f.}
\end{itemize}
\begin{itemize}
\item {Utilização:Pop.}
\end{itemize}
O mesmo que \textunderscore inguinação\textunderscore .
\section{Inguinação}
\begin{itemize}
\item {Grp. gram.:f.}
\end{itemize}
\begin{itemize}
\item {Utilização:beir.}
\end{itemize}
\begin{itemize}
\item {Utilização:Pop.}
\end{itemize}
\begin{itemize}
\item {Proveniência:(De \textunderscore guina\textunderscore , se não é corr. de \textunderscore indignação\textunderscore )}
\end{itemize}
Grande desejo de vingança; vontade de castigar; frenesi.
\section{Inguinal}
\begin{itemize}
\item {fónica:gu-i}
\end{itemize}
\begin{itemize}
\item {Grp. gram.:adj.}
\end{itemize}
\begin{itemize}
\item {Proveniência:(Lat. \textunderscore inguinalis\textunderscore )}
\end{itemize}
Relativo á virilha.
\section{Inguino-escrotal}
\begin{itemize}
\item {Grp. gram.:adj.}
\end{itemize}
\begin{itemize}
\item {Utilização:Anat.}
\end{itemize}
Relativo á virilha e ao escroto: \textunderscore região inguino-escrotal\textunderscore .
\section{Ingurgitação}
\begin{itemize}
\item {Grp. gram.:f.}
\end{itemize}
\begin{itemize}
\item {Proveniência:(Lat. \textunderscore ingurgitatio\textunderscore )}
\end{itemize}
Acto ou effeito de ingurgitar.
\section{Ingurgitamento}
\begin{itemize}
\item {Grp. gram.:m.}
\end{itemize}
\begin{itemize}
\item {Proveniência:(De \textunderscore ingurgitar\textunderscore )}
\end{itemize}
O mesmo que \textunderscore ingurgitação\textunderscore .
Distensão de um vaso no organismo.
Enfartamento, obstrucção.
\section{Ingurgitar}
\begin{itemize}
\item {Grp. gram.:v. t.}
\end{itemize}
\begin{itemize}
\item {Grp. gram.:V. i.  e  p.}
\end{itemize}
Devorar, engulir soffregamente.
Encher muito; obstruír.
Encher-se.
Soffrer obstrucção de um vaso ou ducto excretor.
Entumecer.
\section{Ingurunga}
\begin{itemize}
\item {Grp. gram.:f.}
\end{itemize}
\begin{itemize}
\item {Utilização:Bras}
\end{itemize}
Terreno muito accidentado e quási intransitável.
\section{Inha}
\begin{itemize}
\item {Grp. gram.:pron. f.}
\end{itemize}
\begin{itemize}
\item {Utilização:Ant.}
\end{itemize}
O mesmo que \textunderscore minha\textunderscore . Cf. Prestes, \textunderscore Autos\textunderscore , 461.
\section{Inhabaca}
\begin{itemize}
\item {Grp. gram.:m.}
\end{itemize}
\begin{itemize}
\item {Utilização:T. da Afr. Or. Port}
\end{itemize}
Indígena nobre.
\section{Inhábil}
\begin{itemize}
\item {fónica:iná}
\end{itemize}
\begin{itemize}
\item {Grp. gram.:adj.}
\end{itemize}
\begin{itemize}
\item {Proveniência:(Lat. \textunderscore inhabilis\textunderscore )}
\end{itemize}
Não hábil; que não tem aptidão ou competencia; incapaz.
\section{Inhabilidade}
\begin{itemize}
\item {fónica:ina}
\end{itemize}
\begin{itemize}
\item {Grp. gram.:f.}
\end{itemize}
\begin{itemize}
\item {Proveniência:(De \textunderscore in...\textunderscore  + \textunderscore habilidade\textunderscore )}
\end{itemize}
Qualidade de inhábil; falta de habilidade.
\section{Inhabilitação}
\begin{itemize}
\item {fónica:ina}
\end{itemize}
\begin{itemize}
\item {Grp. gram.:f.}
\end{itemize}
Acto ou effeito de inhabilitar.
\section{Inhabilitar}
\begin{itemize}
\item {fónica:ina}
\end{itemize}
\begin{itemize}
\item {Grp. gram.:v. t.}
\end{itemize}
\begin{itemize}
\item {Proveniência:(De \textunderscore in...\textunderscore  + \textunderscore habilitar\textunderscore )}
\end{itemize}
Tornar inhábil.
Impedir.
Tirar a faculdade ou certos meios a.
\section{Inhabilmente}
\begin{itemize}
\item {fónica:ina}
\end{itemize}
\begin{itemize}
\item {Grp. gram.:adv.}
\end{itemize}
De modo inhábil.
\section{Inhabitado}
\begin{itemize}
\item {fónica:ina}
\end{itemize}
\begin{itemize}
\item {Grp. gram.:adj.}
\end{itemize}
\begin{itemize}
\item {Proveniência:(Lat. \textunderscore inhabitatus\textunderscore )}
\end{itemize}
Não habitado.
\section{Inhabitável}
\begin{itemize}
\item {fónica:ina}
\end{itemize}
\begin{itemize}
\item {Grp. gram.:adj.}
\end{itemize}
\begin{itemize}
\item {Proveniência:(Lat. \textunderscore inhabitabilis\textunderscore )}
\end{itemize}
Que não póde sêr habitado.
\section{Inhabitual}
\begin{itemize}
\item {fónica:ina}
\end{itemize}
\begin{itemize}
\item {Grp. gram.:adj.}
\end{itemize}
\begin{itemize}
\item {Proveniência:(De \textunderscore in...\textunderscore  + \textunderscore habitual\textunderscore )}
\end{itemize}
Que não é habitual.
\section{Inhaca}
\begin{itemize}
\item {Grp. gram.:f.}
\end{itemize}
\begin{itemize}
\item {Utilização:Bras}
\end{itemize}
Cheiro desagradável.
\section{Inhacosso}
\begin{itemize}
\item {fónica:cô}
\end{itemize}
\begin{itemize}
\item {Grp. gram.:m.}
\end{itemize}
Espécie de antílope da Zambézia.
\section{Inhacuana}
\begin{itemize}
\item {Grp. gram.:m.}
\end{itemize}
\begin{itemize}
\item {Utilização:T. de Moçambique}
\end{itemize}
Chefe indígena de povoações, em alguns pontos da Zambézia.
\section{Inhaíba}
\begin{itemize}
\item {Grp. gram.:f.}
\end{itemize}
\begin{itemize}
\item {Utilização:Bras}
\end{itemize}
Árvore silvestre.
\section{Inhalação}
\begin{itemize}
\item {fónica:ina}
\end{itemize}
\begin{itemize}
\item {Grp. gram.:f.}
\end{itemize}
\begin{itemize}
\item {Proveniência:(Lat. \textunderscore inhalatio\textunderscore )}
\end{itemize}
Acto ou effeito de inhalar.
\section{Inhalador}
\begin{itemize}
\item {fónica:ina}
\end{itemize}
\begin{itemize}
\item {Grp. gram.:m.  e  adj.}
\end{itemize}
\begin{itemize}
\item {Proveniência:(De \textunderscore inhalar\textunderscore )}
\end{itemize}
Aquillo que é próprio para inhalações.
\section{Inhalante}
\begin{itemize}
\item {fónica:ina}
\end{itemize}
\begin{itemize}
\item {Grp. gram.:adj.}
\end{itemize}
\begin{itemize}
\item {Proveniência:(Lat. \textunderscore inhalans\textunderscore )}
\end{itemize}
Que inhala.
\section{Inhalar}
\begin{itemize}
\item {fónica:ina}
\end{itemize}
\begin{itemize}
\item {Grp. gram.:v. t.}
\end{itemize}
\begin{itemize}
\item {Proveniência:(Lat. \textunderscore inhalare\textunderscore )}
\end{itemize}
Absorver com o hálito; aspirar.
Receber.
\section{Inhambu}
\begin{itemize}
\item {Grp. gram.:m.}
\end{itemize}
\begin{itemize}
\item {Utilização:Bras}
\end{itemize}
O mesmo que \textunderscore nambu\textunderscore .
\section{Inhame}
\begin{itemize}
\item {Grp. gram.:m.}
\end{itemize}
Planta asparagínea, de raíz farinhenta e fôlhas cordiformes.
\section{Inhanha}
\begin{itemize}
\item {Grp. gram.:m.}
\end{itemize}
O mesmo que \textunderscore inhenho\textunderscore .
\section{Inhanho}
\begin{itemize}
\item {Grp. gram.:m.}
\end{itemize}
O mesmo que \textunderscore inhenho\textunderscore .
\section{Inhapecanga}
\begin{itemize}
\item {Grp. gram.:f.}
\end{itemize}
O mesmo que \textunderscore japecanga\textunderscore .
\section{Inharé}
\begin{itemize}
\item {Grp. gram.:m.}
\end{itemize}
\begin{itemize}
\item {Utilização:Bras}
\end{itemize}
Planta, o mesmo que \textunderscore mururé\textunderscore .
\section{Inharmonia}
\begin{itemize}
\item {fónica:inar}
\end{itemize}
\begin{itemize}
\item {Grp. gram.:f.}
\end{itemize}
\begin{itemize}
\item {Proveniência:(De \textunderscore in...\textunderscore  + \textunderscore harmonia\textunderscore )}
\end{itemize}
O mesmo que \textunderscore desharmonia\textunderscore .
\section{Inharmonicamente}
\begin{itemize}
\item {fónica:inar}
\end{itemize}
\begin{itemize}
\item {Grp. gram.:adv.}
\end{itemize}
De modo inharmónico.
\section{Inharmónico}
\begin{itemize}
\item {fónica:inar}
\end{itemize}
\begin{itemize}
\item {Grp. gram.:adj.}
\end{itemize}
\begin{itemize}
\item {Proveniência:(De \textunderscore in...\textunderscore  + \textunderscore harmónico\textunderscore )}
\end{itemize}
Em que não há harmonia.
\section{Inhaúba}
\begin{itemize}
\item {Grp. gram.:f.}
\end{itemize}
\begin{itemize}
\item {Utilização:Bras}
\end{itemize}
O mesmo que \textunderscore inhaíba\textunderscore .
\section{Inhé}
\begin{itemize}
\item {Grp. gram.:m.}
\end{itemize}
Gênero de árvores africanas, amomáceas, (\textunderscore xylopia africana\textunderscore ).
\section{Inhé-bobó}
\begin{itemize}
\item {Grp. gram.:m.}
\end{itemize}
Grande árvore medicinal, espécie de inhé.
\section{Inhé-branco}
\begin{itemize}
\item {Grp. gram.:m.}
\end{itemize}
Espécie de inhé.
\section{Inheiguaras}
\begin{itemize}
\item {Grp. gram.:m. pl.}
\end{itemize}
Índios das margens do Tocantins.
\section{Inhenha}
\begin{itemize}
\item {Grp. gram.:m.}
\end{itemize}
\begin{itemize}
\item {Utilização:Pop.}
\end{itemize}
\begin{itemize}
\item {Utilização:Ant.}
\end{itemize}
\begin{itemize}
\item {Proveniência:(Do lat. \textunderscore inanis\textunderscore ? Cp. cast. \textunderscore ñoño\textunderscore )}
\end{itemize}
Indivíduo muito acanhado; ímbecil; parvo.
Decrépito.
\section{Inhenho}
\begin{itemize}
\item {Grp. gram.:m.  e  adj.}
\end{itemize}
\begin{itemize}
\item {Utilização:Ant.}
\end{itemize}
\begin{itemize}
\item {Proveniência:(Do lat. \textunderscore inanis\textunderscore ? Cp. cast. \textunderscore ñoño\textunderscore )}
\end{itemize}
Indivíduo muito acanhado; ímbecil; parvo.
Decrépito.
\section{Inhé-preto}
\begin{itemize}
\item {Grp. gram.:m.}
\end{itemize}
Espécie de inhé.
\section{Inherência}
\begin{itemize}
\item {fónica:ine}
\end{itemize}
\begin{itemize}
\item {Grp. gram.:f.}
\end{itemize}
Qualidade daquillo que é inherente.
\section{Inherente}
\begin{itemize}
\item {fónica:ine}
\end{itemize}
\begin{itemize}
\item {Grp. gram.:adj.}
\end{itemize}
\begin{itemize}
\item {Proveniência:(Lat. \textunderscore inhaerens\textunderscore )}
\end{itemize}
Que inhere; ligado naturalmente; inseparável: \textunderscore encargos inherentes a um lugar público\textunderscore .
\section{Inherentemente}
\begin{itemize}
\item {fónica:ine}
\end{itemize}
\begin{itemize}
\item {Grp. gram.:adv.}
\end{itemize}
De modo inherente.
\section{Inherir}
\begin{itemize}
\item {fónica:ine}
\end{itemize}
\begin{itemize}
\item {Grp. gram.:v. i.}
\end{itemize}
\begin{itemize}
\item {Proveniência:(Lat. \textunderscore inhaerere\textunderscore )}
\end{itemize}
Estar ligado intimamente; sêr inseparável.
\section{Inhibição}
\begin{itemize}
\item {fónica:ini}
\end{itemize}
\begin{itemize}
\item {Grp. gram.:f.}
\end{itemize}
\begin{itemize}
\item {Utilização:Med.}
\end{itemize}
\begin{itemize}
\item {Proveniência:(Lat. \textunderscore inhibitio\textunderscore )}
\end{itemize}
Acto de inhibir.
Suppressão ou deminuição da actividade de uma parte do organismo, por effeito de uma excitação nervosa.
\section{Inhibir}
\begin{itemize}
\item {fónica:ini}
\end{itemize}
\begin{itemize}
\item {Grp. gram.:v. t.}
\end{itemize}
\begin{itemize}
\item {Proveniência:(Lat. \textunderscore inhibere\textunderscore )}
\end{itemize}
Impedir.
Prohibir.
Impossibilitar.
\section{Inhibitivo}
\begin{itemize}
\item {fónica:ini}
\end{itemize}
\begin{itemize}
\item {Grp. gram.:adj.}
\end{itemize}
O mesmo que \textunderscore inhibitório\textunderscore .
\section{Inhibitória}
\begin{itemize}
\item {fónica:ini}
\end{itemize}
\begin{itemize}
\item {Grp. gram.:f.}
\end{itemize}
\begin{itemize}
\item {Proveniência:(De \textunderscore inhibitório\textunderscore )}
\end{itemize}
Embaraço, difficuldade. Cf. Sousa, \textunderscore Vida do Arceb.\textunderscore , I, l. 3, 476.
\section{Inhibitório}
\begin{itemize}
\item {fónica:ini}
\end{itemize}
\begin{itemize}
\item {Grp. gram.:adj.}
\end{itemize}
\begin{itemize}
\item {Proveniência:(De \textunderscore inhibir\textunderscore )}
\end{itemize}
Que inhibe: \textunderscore condições inhibitórias\textunderscore .
\section{...inho}
\textunderscore suf.\textunderscore , \textunderscore m.\textunderscore  e \textunderscore adj.\textunderscore ,
(designativo de \textunderscore deminuição\textunderscore )
\section{Inhonestamente}
\begin{itemize}
\item {fónica:ino}
\end{itemize}
\begin{itemize}
\item {Grp. gram.:adv.}
\end{itemize}
O mesmo que \textunderscore deshonestamente\textunderscore .
\section{Inhonestidade}
\begin{itemize}
\item {fónica:ino}
\end{itemize}
\begin{itemize}
\item {Grp. gram.:f.}
\end{itemize}
O mesmo que \textunderscore deshonestidade\textunderscore .
\section{Inhonesto}
\begin{itemize}
\item {fónica:ino}
\end{itemize}
\begin{itemize}
\item {Grp. gram.:f.}
\end{itemize}
O mesmo que \textunderscore deshonesto\textunderscore .
\section{Inhospedeiro}
\begin{itemize}
\item {fónica:inos}
\end{itemize}
\begin{itemize}
\item {Grp. gram.:adj.}
\end{itemize}
O mesmo que \textunderscore inhóspito\textunderscore . Cf. Castilho, \textunderscore Fastos\textunderscore , II, 157.
\section{Inhospitaleiramente}
\begin{itemize}
\item {fónica:inos}
\end{itemize}
\begin{itemize}
\item {Grp. gram.:adv.}
\end{itemize}
De modo inhospitaleiro.
Sem vontade de receber estranjeiros.
\section{Inhospitaleiro}
\begin{itemize}
\item {fónica:inos}
\end{itemize}
\begin{itemize}
\item {Grp. gram.:adj.}
\end{itemize}
\begin{itemize}
\item {Proveniência:(De \textunderscore in...\textunderscore  + \textunderscore hospitaleiro\textunderscore )}
\end{itemize}
Que não é hospitaleiro.
Que é desfavorável a estranjeiros ou que os não recebe.
O mesmo que \textunderscore inhóspito\textunderscore .
\section{Inhospitalidade}
\begin{itemize}
\item {fónica:inos}
\end{itemize}
\begin{itemize}
\item {Grp. gram.:f.}
\end{itemize}
\begin{itemize}
\item {Proveniência:(De \textunderscore in...\textunderscore  + \textunderscore hospitalidade\textunderscore )}
\end{itemize}
Falta de hospitalidade.
Recusa de receber estranjeiros.
\section{Inhóspito}
\begin{itemize}
\item {fónica:inós}
\end{itemize}
\begin{itemize}
\item {Grp. gram.:adj.}
\end{itemize}
\begin{itemize}
\item {Proveniência:(Lat. \textunderscore inhospitus\textunderscore )}
\end{itemize}
Que não é apto para hospedar.
Que não pratíca a hospitalidade. Em que se não póde viver: \textunderscore terras inhóspitas\textunderscore .
\section{Inarrável}
\begin{itemize}
\item {Grp. gram.:adj.}
\end{itemize}
\begin{itemize}
\item {Proveniência:(De \textunderscore in...\textunderscore  + \textunderscore narrável\textunderscore )}
\end{itemize}
Que se não póde narrar; indizível.
Inenarrável.
\section{Inascível}
\begin{itemize}
\item {Grp. gram.:adj.}
\end{itemize}
\begin{itemize}
\item {Proveniência:(Lat. \textunderscore innascibilis\textunderscore )}
\end{itemize}
Que não póde nascer.
\section{Inatismo}
\begin{itemize}
\item {Grp. gram.:m.}
\end{itemize}
\begin{itemize}
\item {Proveniência:(De \textunderscore inato\textunderscore )}
\end{itemize}
Sistema filosófico dos que entendem que as ideias são innatas, existindo no entendimento em modo latente, por fórma que conhecer é só recordar.
\section{Inato}
\begin{itemize}
\item {Grp. gram.:adj.}
\end{itemize}
\begin{itemize}
\item {Proveniência:(Lat. \textunderscore innatus\textunderscore )}
\end{itemize}
Congênito; que nasce com o indivíduo.
Inerente: \textunderscore bondade inata\textunderscore .
Que não nasceu, que não teve princípio, (falando-se de Deus).
\section{Inatural}
\begin{itemize}
\item {Grp. gram.:adj.}
\end{itemize}
\begin{itemize}
\item {Proveniência:(De \textunderscore in...\textunderscore  + \textunderscore natural\textunderscore )}
\end{itemize}
Que não é natural.
\section{Inaufragável}
\begin{itemize}
\item {Grp. gram.:adj.}
\end{itemize}
\begin{itemize}
\item {Proveniência:(De \textunderscore in...\textunderscore  + \textunderscore naufragável\textunderscore )}
\end{itemize}
Que não póde naufragar:« \textunderscore barquinhas inaufragáveis.\textunderscore »Castilho.
\section{Inavegabilidade}
\begin{itemize}
\item {Grp. gram.:f.}
\end{itemize}
Qualidade de inavegável.
\section{Inavegável}
\begin{itemize}
\item {Grp. gram.:adj.}
\end{itemize}
\begin{itemize}
\item {Proveniência:(Lat. \textunderscore innavigabilis\textunderscore )}
\end{itemize}
Que não é navegável.
\section{Inavigabilidade}
\begin{itemize}
\item {Grp. gram.:f.}
\end{itemize}
O mesmo ou melhor que \textunderscore inavegabilidade\textunderscore . Cf. F. Borges, \textunderscore Diccion. Jur.\textunderscore 
(Cp. lat. \textunderscore navigare\textunderscore )
\section{Inecessário}
\begin{itemize}
\item {Grp. gram.:adj.}
\end{itemize}
\begin{itemize}
\item {Proveniência:(De \textunderscore in...\textunderscore  + \textunderscore necessário\textunderscore )}
\end{itemize}
Que não é necessário; prescindível.
\section{Inegável}
\begin{itemize}
\item {Grp. gram.:adj.}
\end{itemize}
\begin{itemize}
\item {Proveniência:(De \textunderscore in...\textunderscore  + \textunderscore negável\textunderscore )}
\end{itemize}
Que não póde negar; incontestável; evidente.
\section{Inegavelmente}
\begin{itemize}
\item {Grp. gram.:adv.}
\end{itemize}
De modo inegável.
\section{Inegociável}
\begin{itemize}
\item {Grp. gram.:adj.}
\end{itemize}
\begin{itemize}
\item {Proveniência:(De \textunderscore in...\textunderscore  + \textunderscore negociável\textunderscore )}
\end{itemize}
Que não é negociável; que se não póde contratar; que não é objecto de comércio: \textunderscore a honra é inegociável\textunderscore .
\section{Inervar}
\textunderscore v. t.\textunderscore  (e der.)
(V. \textunderscore enervar\textunderscore ^1, etc)
\section{Inérveo}
\begin{itemize}
\item {Grp. gram.:adj.}
\end{itemize}
\begin{itemize}
\item {Utilização:Bot.}
\end{itemize}
\begin{itemize}
\item {Proveniência:(De \textunderscore in...\textunderscore  + \textunderscore nervo\textunderscore )}
\end{itemize}
Que não tem nervura.
\section{Inhuma}
\begin{itemize}
\item {Grp. gram.:f.}
\end{itemize}
\begin{itemize}
\item {Utilização:Bras. do N}
\end{itemize}
O mesmo que \textunderscore anhuma\textunderscore .
\section{Inhumação}
\begin{itemize}
\item {fónica:inu}
\end{itemize}
\begin{itemize}
\item {Grp. gram.:f.}
\end{itemize}
Acto ou effeito de inhumar.
\section{Inhumanamente}
\begin{itemize}
\item {fónica:inu}
\end{itemize}
\begin{itemize}
\item {Grp. gram.:adv.}
\end{itemize}
De modo inhumano.
\section{Inhumanidade}
\begin{itemize}
\item {fónica:inu}
\end{itemize}
\begin{itemize}
\item {Grp. gram.:f.}
\end{itemize}
\begin{itemize}
\item {Proveniência:(Lat. \textunderscore inhumanitas\textunderscore )}
\end{itemize}
O mesmo que \textunderscore deshumanidade\textunderscore .
\section{Inhumano}
\begin{itemize}
\item {fónica:inu}
\end{itemize}
\begin{itemize}
\item {Grp. gram.:adj.}
\end{itemize}
\begin{itemize}
\item {Utilização:Ant.}
\end{itemize}
\begin{itemize}
\item {Proveniência:(Lat. \textunderscore inhumanus\textunderscore )}
\end{itemize}
O mesmo que \textunderscore deshumano\textunderscore .
Sobrehumano.
\section{Inhumar}
\begin{itemize}
\item {fónica:inu}
\end{itemize}
\begin{itemize}
\item {Grp. gram.:v. t.}
\end{itemize}
\begin{itemize}
\item {Proveniência:(Lat. \textunderscore inhumare\textunderscore )}
\end{itemize}
Enterrar, cobrir de terra; sepultar.
\section{Iníaco}
\begin{itemize}
\item {Grp. gram.:adj.}
\end{itemize}
\begin{itemize}
\item {Utilização:Anat.}
\end{itemize}
Relativo ao ínio.
\section{Iniala}
\begin{itemize}
\item {Grp. gram.:m.}
\end{itemize}
Espécie de antílope da África.
\section{Inibição}
\begin{itemize}
\item {Grp. gram.:f.}
\end{itemize}
\begin{itemize}
\item {Utilização:Med.}
\end{itemize}
\begin{itemize}
\item {Proveniência:(Lat. \textunderscore inhibitio\textunderscore )}
\end{itemize}
Acto de inibir.
Suppressão ou deminuição da actividade de uma parte do organismo, por efeito de uma excitação nervosa.
\section{Inibir}
\begin{itemize}
\item {Grp. gram.:v. t.}
\end{itemize}
\begin{itemize}
\item {Proveniência:(Lat. \textunderscore inhibere\textunderscore )}
\end{itemize}
Impedir.
Proibir.
Impossibilitar.
\section{Inibitivo}
\begin{itemize}
\item {Grp. gram.:adj.}
\end{itemize}
O mesmo que \textunderscore inibitório\textunderscore .
\section{Inibitória}
\begin{itemize}
\item {Grp. gram.:f.}
\end{itemize}
\begin{itemize}
\item {Proveniência:(De \textunderscore inibitório\textunderscore )}
\end{itemize}
Embaraço, dificuldade. Cf. Sousa, \textunderscore Vida do Arceb.\textunderscore , I, l. 3, 476.
\section{Inibitório}
\begin{itemize}
\item {Grp. gram.:adj.}
\end{itemize}
\begin{itemize}
\item {Proveniência:(De \textunderscore inibir\textunderscore )}
\end{itemize}
Que inibe: \textunderscore condições inibitórias\textunderscore .
\section{Iniciação}
\begin{itemize}
\item {Grp. gram.:f.}
\end{itemize}
\begin{itemize}
\item {Proveniência:(Lat. \textunderscore initiatio\textunderscore )}
\end{itemize}
Acto ou effeito de iniciar.
Acto de receber as primeiras notícias de coisas mysteriosas ou desconhecidas.
Admissão de um indivíduo em lojas maçónicas.
Acto de começar qualquer coisa.
\section{Iniciado}
\begin{itemize}
\item {Grp. gram.:m.}
\end{itemize}
\begin{itemize}
\item {Proveniência:(Lat. \textunderscore initiatus\textunderscore )}
\end{itemize}
Neóphyto de uma seita ou ordem.
\section{Iniciador}
\begin{itemize}
\item {Grp. gram.:m.  e  adj.}
\end{itemize}
\begin{itemize}
\item {Proveniência:(Lat. \textunderscore initiator\textunderscore )}
\end{itemize}
O que inicia.
\section{Inicial}
\begin{itemize}
\item {Grp. gram.:adj.}
\end{itemize}
\begin{itemize}
\item {Grp. gram.:F.}
\end{itemize}
\begin{itemize}
\item {Proveniência:(Lat. \textunderscore initialis\textunderscore )}
\end{itemize}
Que inicia.
Que está no princípio.
Primeira letra de uma palavra ou de um nome.
\section{Inicialmente}
\begin{itemize}
\item {Grp. gram.:adv.}
\end{itemize}
De modo inicial.
\section{Iniciando}
\begin{itemize}
\item {Grp. gram.:m.}
\end{itemize}
\begin{itemize}
\item {Proveniência:(Lat. \textunderscore initiandus\textunderscore )}
\end{itemize}
Aquelle que há de ser iniciado ou admittido ás ceremónias de uma ordem ou seita. Cf. Castilho, \textunderscore Fastos\textunderscore , II, 611, 660 e 661.
\section{Iniciar}
\begin{itemize}
\item {Grp. gram.:v. t.}
\end{itemize}
\begin{itemize}
\item {Proveniência:(Lat. \textunderscore initiare\textunderscore )}
\end{itemize}
Começar, principiar.
Preparar ou admittir aos mystérios e ceremónias de uma ordem ou seita.
Proporcionar as primeiras noções de alguma coisa a.
Dar fórma a.
Inaugurar.
\section{Iniciativa}
\begin{itemize}
\item {Grp. gram.:f.}
\end{itemize}
\begin{itemize}
\item {Utilização:Ext.}
\end{itemize}
\begin{itemize}
\item {Proveniência:(De \textunderscore iniciativo\textunderscore )}
\end{itemize}
Acto, com que alguém mostra sêr o primeiro em suscitar, propagar ou pôr em prática uma ideia: \textunderscore tomar a iniciativa de uma empresa\textunderscore .
Actividade, diligência: \textunderscore é homem de grande iniciativa\textunderscore .
\section{Iniciativo}
\begin{itemize}
\item {Grp. gram.:adj.}
\end{itemize}
\begin{itemize}
\item {Proveniência:(De \textunderscore iniciar\textunderscore )}
\end{itemize}
O mesmo que inicial.
Que tem carácter de iniciativa.
\section{Início}
\begin{itemize}
\item {Grp. gram.:m.}
\end{itemize}
\begin{itemize}
\item {Proveniência:(Lat. \textunderscore initium\textunderscore )}
\end{itemize}
Começo.
Exórdio.
Inauguração.
\section{Inigualável}
\begin{itemize}
\item {Grp. gram.:adj.}
\end{itemize}
\begin{itemize}
\item {Proveniência:(De \textunderscore in...\textunderscore  + \textunderscore igualável\textunderscore )}
\end{itemize}
Que se não póde igualar.
\section{Inilludível}
\begin{itemize}
\item {Grp. gram.:adj.}
\end{itemize}
\begin{itemize}
\item {Proveniência:(De \textunderscore in...\textunderscore  + \textunderscore illudível\textunderscore )}
\end{itemize}
Que não é illudível.
Que não deixa ou não admitte dúvidas; evidente.
\section{Inilludivelmente}
\begin{itemize}
\item {Grp. gram.:adv.}
\end{itemize}
De modo inilludível; evidentemente.
\section{Iniludível}
\begin{itemize}
\item {Grp. gram.:adj.}
\end{itemize}
\begin{itemize}
\item {Proveniência:(De \textunderscore in...\textunderscore  + \textunderscore iludível\textunderscore )}
\end{itemize}
Que não é iludível.
Que não deixa ou não admite dúvidas; evidente.
\section{Iniludivelmente}
\begin{itemize}
\item {Grp. gram.:adv.}
\end{itemize}
De modo iniludível; evidentemente.
\section{Inimaginável}
\begin{itemize}
\item {Grp. gram.:adj.}
\end{itemize}
\begin{itemize}
\item {Proveniência:(De \textunderscore in...\textunderscore  + \textunderscore imaginável\textunderscore )}
\end{itemize}
Que se não póde imaginar; incrivel.
\section{Inimboja}
\begin{itemize}
\item {Grp. gram.:f.}
\end{itemize}
O mesmo que \textunderscore bonduque\textunderscore .
\section{Inimicícia}
\begin{itemize}
\item {Grp. gram.:f.}
\end{itemize}
\begin{itemize}
\item {Utilização:Des.}
\end{itemize}
\begin{itemize}
\item {Proveniência:(Lat. \textunderscore inimicitia\textunderscore )}
\end{itemize}
O mesmo que \textunderscore inimizade\textunderscore . Cf. Lusíadas, VIII, 8 e 64.
\section{Inimicíssimo}
\begin{itemize}
\item {Grp. gram.:adj.}
\end{itemize}
\begin{itemize}
\item {Proveniência:(Do lat. \textunderscore inímicus\textunderscore )}
\end{itemize}
Muito inimigo, muito adverso.
\section{Inimigo}
\begin{itemize}
\item {Grp. gram.:adj.}
\end{itemize}
\begin{itemize}
\item {Grp. gram.:M.}
\end{itemize}
\begin{itemize}
\item {Utilização:Pop.}
\end{itemize}
\begin{itemize}
\item {Proveniência:(Lat. \textunderscore inimicus\textunderscore )}
\end{itemize}
Não amigo; adversário.
Nocivo; \textunderscore aquelle insecto é inimigo das plantas\textunderscore .
Que não gosta ou que aborrece.
Indivíduo, que tem ódio a alguém: \textunderscore quem é que não tem inimigos?\textunderscore 
Aquelle que milita em campo ou partido opposto ao de outrem.
Tropa ou nação, com que se anda em guerra: \textunderscore o inimigo atacou a nossa praça\textunderscore .
Rapaz traquina.
Diabo.
\section{Inimistar}
\begin{itemize}
\item {Grp. gram.:v. t.}
\end{itemize}
O mesmo que \textunderscore inimizar\textunderscore . Cf. \textunderscore Viriato Trág.\textunderscore , III, 28.
(Cast. \textunderscore inimistar\textunderscore )
\section{Inimitado}
\begin{itemize}
\item {Grp. gram.:adj.}
\end{itemize}
\begin{itemize}
\item {Proveniência:(De \textunderscore in...\textunderscore  + \textunderscore imitar\textunderscore )}
\end{itemize}
Que não é imitado. Cf. Filinto, I, 207.
\section{Inimitável}
\begin{itemize}
\item {Grp. gram.:adj.}
\end{itemize}
\begin{itemize}
\item {Proveniência:(Lat. \textunderscore inimitabilis\textunderscore )}
\end{itemize}
Que não é imitável.
\section{Inimitavelmente}
\begin{itemize}
\item {Grp. gram.:adv.}
\end{itemize}
De modo inimitável.
\section{Inimizade}
\begin{itemize}
\item {Grp. gram.:f.}
\end{itemize}
\begin{itemize}
\item {Proveniência:(Do b. lat. \textunderscore inimicitas\textunderscore )}
\end{itemize}
Falta de amizade.
Qualidade de quem é inimigo.
Malquerença.
\section{Inimizar}
\begin{itemize}
\item {Grp. gram.:v. t.}
\end{itemize}
\begin{itemize}
\item {Proveniência:(De \textunderscore inimigo\textunderscore )}
\end{itemize}
Tornar inimigo; indispor.
\section{Inimizío}
\begin{itemize}
\item {Grp. gram.:m.}
\end{itemize}
\begin{itemize}
\item {Utilização:Ant.}
\end{itemize}
\begin{itemize}
\item {Proveniência:(Do lat. \textunderscore inimicitia\textunderscore )}
\end{itemize}
Inimizade.
Enrêdo; intriga.
\section{Ininteligível}
\begin{itemize}
\item {Grp. gram.:adj.}
\end{itemize}
\begin{itemize}
\item {Proveniência:(De \textunderscore in...\textunderscore  + \textunderscore inteligível\textunderscore )}
\end{itemize}
Que não é inteligível.
Que se não percebe: \textunderscore palavras ininteligíveis\textunderscore .
Obscuro; misterioso; problemático.
Superior á razão humana.
\section{Ininteligivelmente}
\begin{itemize}
\item {Grp. gram.:adv.}
\end{itemize}
De modo ininteligível.
\section{Inintelligível}
\begin{itemize}
\item {Grp. gram.:adj.}
\end{itemize}
\begin{itemize}
\item {Proveniência:(De \textunderscore in...\textunderscore  + \textunderscore intelligível\textunderscore )}
\end{itemize}
Que não é intelligível.
Que se não percebe: \textunderscore palavras inintelligíveis\textunderscore .
Obscuro; mysterioso; problemático.
Superior á razão humana.
\section{Inintelligivelmente}
\begin{itemize}
\item {Grp. gram.:adv.}
\end{itemize}
De modo inintelligível.
\section{Ininterrompidamente}
\begin{itemize}
\item {Grp. gram.:adv.}
\end{itemize}
\begin{itemize}
\item {Proveniência:(De \textunderscore ininterrompido\textunderscore )}
\end{itemize}
Sem interrupção; continuamente.
\section{Ininterrompido}
\begin{itemize}
\item {Grp. gram.:adj.}
\end{itemize}
\begin{itemize}
\item {Proveniência:(De \textunderscore in...\textunderscore  + \textunderscore interrompido\textunderscore )}
\end{itemize}
O mesmo que \textunderscore ininterrupto\textunderscore .
\section{Ininterrupção}
\begin{itemize}
\item {Grp. gram.:f.}
\end{itemize}
\begin{itemize}
\item {Proveniência:(De \textunderscore in...\textunderscore  + \textunderscore interrupção\textunderscore )}
\end{itemize}
Falta de interrupção.
Continuidade; sequência: \textunderscore fallar duas horas sem interrupção\textunderscore .
\section{Ininterruptamente}
\begin{itemize}
\item {Grp. gram.:adv.}
\end{itemize}
De modo ininterrúpto.
\section{Ininterrupto}
\begin{itemize}
\item {Grp. gram.:adj.}
\end{itemize}
\begin{itemize}
\item {Proveniência:(De \textunderscore in...\textunderscore  + \textunderscore interrupto\textunderscore )}
\end{itemize}
Não interrompido; contínuo; constante: \textunderscore assiduidade ininterrupta\textunderscore .
\section{Ininvestigável}
\begin{itemize}
\item {Grp. gram.:adj.}
\end{itemize}
\begin{itemize}
\item {Proveniência:(Lat. \textunderscore ininvestigabilis\textunderscore )}
\end{itemize}
Que não é investigável.
Que se não póde investigar; imperscrutável.
\section{Ínio}
\begin{itemize}
\item {Grp. gram.:m.}
\end{itemize}
O mesmo ou melhor que \textunderscore ínion\textunderscore .
\section{Iniodimia}
\begin{itemize}
\item {Grp. gram.:f.}
\end{itemize}
Conformação dos iniódimos.
\section{Iniodiminiano}
\begin{itemize}
\item {Grp. gram.:adj.}
\end{itemize}
Relativo á iniodimia.
\section{Iniodímico}
\begin{itemize}
\item {Grp. gram.:adj.}
\end{itemize}
Que tem o carácter da iniodimia.
\section{Iniódimo}
\begin{itemize}
\item {Grp. gram.:m.}
\end{itemize}
Monstro, composto de dois indivíduos ligados pela nuca.
(Por \textunderscore iniodídimo\textunderscore , do gr. \textunderscore inion\textunderscore  + \textunderscore didumos\textunderscore )
\section{Iniodymia}
\begin{itemize}
\item {Grp. gram.:f.}
\end{itemize}
Conformação dos iniódymos.
\section{Iniodyminiano}
\begin{itemize}
\item {Grp. gram.:adj.}
\end{itemize}
Relativo á iniodymia.
\section{Iniodýmico}
\begin{itemize}
\item {Grp. gram.:adj.}
\end{itemize}
Que tem o carácter da iniodymia.
\section{Iniódymo}
\begin{itemize}
\item {Grp. gram.:m.}
\end{itemize}
Monstro, composto de dois indivíduos ligados pela nuca.
(Por \textunderscore iniodídymo\textunderscore , do gr. \textunderscore inion\textunderscore  + \textunderscore didumos\textunderscore )
\section{Ínion}
\begin{itemize}
\item {Grp. gram.:m.}
\end{itemize}
\begin{itemize}
\item {Utilização:Anat.}
\end{itemize}
\begin{itemize}
\item {Proveniência:(Gr. \textunderscore ínion\textunderscore , nuca)}
\end{itemize}
Vértice da protuberância occipital externa.
\section{Iniquamente}
\begin{itemize}
\item {Grp. gram.:adv.}
\end{itemize}
De modo iníquo; com iniquidade.
\section{Iniquar-se}
\begin{itemize}
\item {Grp. gram.:v. p.}
\end{itemize}
\begin{itemize}
\item {Utilização:Ant.}
\end{itemize}
\begin{itemize}
\item {Proveniência:(Lat. \textunderscore iniquare\textunderscore )}
\end{itemize}
Tornar-se iníquo.
Fazer-se ruím ou de má qualidade.
\section{Iniquícia}
\begin{itemize}
\item {Grp. gram.:f.}
\end{itemize}
\begin{itemize}
\item {Utilização:Ant.}
\end{itemize}
O mesmo que \textunderscore iniquidade\textunderscore .
\section{Iniquidade}
\begin{itemize}
\item {fónica:qu-i}
\end{itemize}
\begin{itemize}
\item {Grp. gram.:f.}
\end{itemize}
\begin{itemize}
\item {Proveniência:(Lat. \textunderscore iniquitas\textunderscore )}
\end{itemize}
Qualidade daquelle ou daquillo que é iníquo.
Acto ou dito iníquo.
\section{Iníquo}
\begin{itemize}
\item {Grp. gram.:adj.}
\end{itemize}
\begin{itemize}
\item {Proveniência:(Lat. \textunderscore iniquus\textunderscore )}
\end{itemize}
Opposto á equidade; injusto.
Perverso.
\section{Injá}
\begin{itemize}
\item {Grp. gram.:m.}
\end{itemize}
\begin{itemize}
\item {Utilização:Prov.}
\end{itemize}
O mesmo que \textunderscore rabeta\textunderscore .
\section{Injecção}
\begin{itemize}
\item {Grp. gram.:f.}
\end{itemize}
\begin{itemize}
\item {Proveniência:(Do lat. \textunderscore injectio\textunderscore )}
\end{itemize}
Acto ou effeito de injectar.
Líquido para se injectar.
Repleção dos vasos capillares.
\section{Injectar}
\begin{itemize}
\item {Grp. gram.:v. t.}
\end{itemize}
\begin{itemize}
\item {Proveniência:(Lat. \textunderscore injectare\textunderscore )}
\end{itemize}
Introduzir com seringa ou outro instrumento (um líquido) em uma cavidade do corpo.
Tornar còrado pelo affluxo do sangue: \textunderscore olhos injectados\textunderscore .
Tornar còrada e mais resistente a madeira, fazendo penetrar nella (certos líquidos).
\section{Injectiva}
\begin{itemize}
\item {Grp. gram.:f.}
\end{itemize}
\begin{itemize}
\item {Utilização:Prov.}
\end{itemize}
\begin{itemize}
\item {Utilização:trasm.}
\end{itemize}
\begin{itemize}
\item {Proveniência:(Do lat. \textunderscore injectus\textunderscore )}
\end{itemize}
Expediente, meio, recurso.
\section{Injecto}
\begin{itemize}
\item {Grp. gram.:m.}
\end{itemize}
\begin{itemize}
\item {Utilização:Des.}
\end{itemize}
\begin{itemize}
\item {Proveniência:(Lat. \textunderscore injectus\textunderscore )}
\end{itemize}
Preparação anatómica de um órgão injectado.
\section{Injector}
\begin{itemize}
\item {Grp. gram.:adj.}
\end{itemize}
\begin{itemize}
\item {Grp. gram.:M.}
\end{itemize}
Que injecta.
Parte de um apparelho, destinado á sulfuração dos vinhos.
Apparelho, para auxiliar a tiragem das fornalhas, nas máquinas de vapor.
\section{Injucundo}
\begin{itemize}
\item {Grp. gram.:adj.}
\end{itemize}
\begin{itemize}
\item {Proveniência:(Lat. \textunderscore injucundus\textunderscore )}
\end{itemize}
Que não é jucundo; que desagrada, que é molesto.
\section{Injudicioso}
\begin{itemize}
\item {Grp. gram.:adj.}
\end{itemize}
\begin{itemize}
\item {Proveniência:(De \textunderscore in...\textunderscore  + \textunderscore judicioso\textunderscore )}
\end{itemize}
Que não é judicioso; insensato. Cf. Camillo, \textunderscore Estrêll. Funestas\textunderscore , 27.
\section{Injunção}
\begin{itemize}
\item {Grp. gram.:f.}
\end{itemize}
\begin{itemize}
\item {Proveniência:(Lat. \textunderscore injunctio\textunderscore )}
\end{itemize}
Acto ou efeito de injungir; imposição.
\section{Injuncção}
\begin{itemize}
\item {Grp. gram.:f.}
\end{itemize}
\begin{itemize}
\item {Proveniência:(Lat. \textunderscore injunctio\textunderscore )}
\end{itemize}
Acto ou effeito de injungir; imposição.
\section{Injunctivo}
\begin{itemize}
\item {Grp. gram.:adj.}
\end{itemize}
\begin{itemize}
\item {Proveniência:(Do lat. \textunderscore injunctus\textunderscore )}
\end{itemize}
Obrigatório; imperativo.
\section{Injungir}
\begin{itemize}
\item {Grp. gram.:v. t.}
\end{itemize}
\begin{itemize}
\item {Proveniência:(Lat. \textunderscore injungere\textunderscore )}
\end{itemize}
Impôr; obrigar a.
\section{Injuntivo}
\begin{itemize}
\item {Grp. gram.:adj.}
\end{itemize}
\begin{itemize}
\item {Proveniência:(Do lat. \textunderscore injunctus\textunderscore )}
\end{itemize}
Obrigatório; imperativo.
\section{Injúria}
\begin{itemize}
\item {Grp. gram.:f.}
\end{itemize}
\begin{itemize}
\item {Proveniência:(Lat. \textunderscore injuria\textunderscore )}
\end{itemize}
Aquillo que é contra o direito.
Aquillo que é injusto.
Expressão ou acto que offende alguém.
Insulto.
Detrimento.
\section{Injuriador}
\begin{itemize}
\item {Grp. gram.:m.  e  adj.}
\end{itemize}
O que injuría.
\section{Injuriante}
\begin{itemize}
\item {Grp. gram.:adj.}
\end{itemize}
\begin{itemize}
\item {Proveniência:(Lat. \textunderscore injurians\textunderscore )}
\end{itemize}
Que injuría.
\section{Injuriar}
\begin{itemize}
\item {Grp. gram.:v. t.}
\end{itemize}
\begin{itemize}
\item {Proveniência:(Lat. \textunderscore injuriare\textunderscore )}
\end{itemize}
Fazer injúria a; offender; insultar.
Diffamar.
Damnificar.
\section{Injurídico}
\begin{itemize}
\item {Grp. gram.:adj.}
\end{itemize}
\begin{itemize}
\item {Proveniência:(De \textunderscore in...\textunderscore  + \textunderscore jurídico\textunderscore )}
\end{itemize}
Que não é jurídico; illegal.
\section{Injúrio}
\begin{itemize}
\item {Grp. gram.:adj.}
\end{itemize}
\begin{itemize}
\item {Utilização:Des.}
\end{itemize}
O mesmo que \textunderscore injurioso\textunderscore . Cf. Filinto, XXII, 78.
\section{Injuriosamente}
\begin{itemize}
\item {Grp. gram.:adv.}
\end{itemize}
De modo injurioso; com injúria.
\section{Injurioso}
\begin{itemize}
\item {Grp. gram.:adj.}
\end{itemize}
\begin{itemize}
\item {Proveniência:(Lat. \textunderscore injuriosus\textunderscore )}
\end{itemize}
Em que há injúria; offensivo; infamante.
\section{Injustamente}
\begin{itemize}
\item {Grp. gram.:adv.}
\end{itemize}
De modo injusto; contra a justiça ou contra o direito: \textunderscore condemnado injustamente\textunderscore .
\section{Injustiça}
\begin{itemize}
\item {Grp. gram.:f.}
\end{itemize}
\begin{itemize}
\item {Proveniência:(Lat. \textunderscore injustitia\textunderscore )}
\end{itemize}
Falta de justiça; iniquidade.
Offensa da equidade ou do direito.
\section{Injustiçoso}
\begin{itemize}
\item {Grp. gram.:adj.}
\end{itemize}
Que pratíca injustiças.
Iníquo.
\section{Injustificável}
\begin{itemize}
\item {Grp. gram.:adj.}
\end{itemize}
\begin{itemize}
\item {Proveniência:(De \textunderscore in...\textunderscore  + \textunderscore justificável\textunderscore )}
\end{itemize}
Que não é justificável; que se não póde justificar.
\section{Injusto}
\begin{itemize}
\item {Grp. gram.:adj.}
\end{itemize}
\begin{itemize}
\item {Grp. gram.:M.}
\end{itemize}
\begin{itemize}
\item {Proveniência:(Lat. \textunderscore injustus\textunderscore )}
\end{itemize}
Que não é justo.
Iníquo; contrário á justiça; infundado.
Aquelle ou aquillo que não é justo.
\section{Inlapidado}
\begin{itemize}
\item {Grp. gram.:adj.}
\end{itemize}
\begin{itemize}
\item {Proveniência:(De \textunderscore in...\textunderscore  + \textunderscore lapidar\textunderscore )}
\end{itemize}
Que não é ou não foi lapidado.
\section{Inliçar}
\textunderscore v. t.\textunderscore  (e der.)
O mesmo que \textunderscore illiçar\textunderscore , etc.
\section{Inlusir}
\begin{itemize}
\item {Grp. gram.:v. t.}
\end{itemize}
\begin{itemize}
\item {Utilização:Prov.}
\end{itemize}
Illudir.
Induzir.
Enganar com modos hypócritas.
(Por \textunderscore illusir\textunderscore , de \textunderscore illuso\textunderscore )
\section{Inlustre}
\begin{itemize}
\item {Grp. gram.:adj.}
\end{itemize}
\begin{itemize}
\item {Utilização:Pop.}
\end{itemize}
O mesmo que \textunderscore illustre\textunderscore .
(Cp. b. lat. \textunderscore inluster\textunderscore )
\section{Innarrável}
\begin{itemize}
\item {Grp. gram.:adj.}
\end{itemize}
\begin{itemize}
\item {Proveniência:(De \textunderscore in...\textunderscore  + \textunderscore narrável\textunderscore )}
\end{itemize}
Que se não póde narrar; indizível.
Inenarrável.
\section{Innascível}
\begin{itemize}
\item {Grp. gram.:adj.}
\end{itemize}
\begin{itemize}
\item {Proveniência:(Lat. \textunderscore innascibilis\textunderscore )}
\end{itemize}
Que não póde nascer.
\section{Innatismo}
\begin{itemize}
\item {Grp. gram.:m.}
\end{itemize}
\begin{itemize}
\item {Proveniência:(De \textunderscore innato\textunderscore )}
\end{itemize}
Systema philosóphico dos que entendem que as ideias são innatas, existindo no entendimento em modo latente, por fórma que conhecer é só recordar.
\section{Innato}
\begin{itemize}
\item {Grp. gram.:adj.}
\end{itemize}
\begin{itemize}
\item {Proveniência:(Lat. \textunderscore innatus\textunderscore )}
\end{itemize}
Congênito; que nasce com o indivíduo.
Inherente: \textunderscore bondade innata\textunderscore .
Que não nasceu, que não teve princípio, (falando-se de Deus).
\section{Innatural}
\begin{itemize}
\item {Grp. gram.:adj.}
\end{itemize}
\begin{itemize}
\item {Proveniência:(De \textunderscore in...\textunderscore  + \textunderscore natural\textunderscore )}
\end{itemize}
Que não é natural.
\section{Innaufragável}
\begin{itemize}
\item {Grp. gram.:adj.}
\end{itemize}
\begin{itemize}
\item {Proveniência:(De \textunderscore in...\textunderscore  + \textunderscore naufragável\textunderscore )}
\end{itemize}
Que não póde naufragar:« \textunderscore barquinhas innaufragáveis.\textunderscore »Castilho.
\section{Innavegabilidade}
\begin{itemize}
\item {Grp. gram.:f.}
\end{itemize}
Qualidade de innavegável.
\section{Innavegável}
\begin{itemize}
\item {Grp. gram.:adj.}
\end{itemize}
\begin{itemize}
\item {Proveniência:(Lat. \textunderscore innavigabilis\textunderscore )}
\end{itemize}
Que não é navegável.
\section{Innavigabilidade}
\begin{itemize}
\item {Grp. gram.:f.}
\end{itemize}
O mesmo ou melhor que \textunderscore innavegabilidade\textunderscore . Cf. F. Borges, \textunderscore Diccion. Jur.\textunderscore 
(Cp. lat. \textunderscore navigare\textunderscore )
\section{Innecessário}
\begin{itemize}
\item {Grp. gram.:adj.}
\end{itemize}
\begin{itemize}
\item {Proveniência:(De \textunderscore in...\textunderscore  + \textunderscore necessário\textunderscore )}
\end{itemize}
Que não é necessário; prescindível.
\section{Innegável}
\begin{itemize}
\item {Grp. gram.:adj.}
\end{itemize}
\begin{itemize}
\item {Proveniência:(De \textunderscore in...\textunderscore  + \textunderscore negável\textunderscore )}
\end{itemize}
Que não póde negar; incontestável; evidente.
\section{Innegavelmente}
\begin{itemize}
\item {Grp. gram.:adv.}
\end{itemize}
De modo innegável.
\section{Innegociável}
\begin{itemize}
\item {Grp. gram.:adj.}
\end{itemize}
\begin{itemize}
\item {Proveniência:(De \textunderscore in...\textunderscore  + \textunderscore negociável\textunderscore )}
\end{itemize}
Que não é negociável; que se não póde contratar; que não é objecto de commércio: \textunderscore a honra é innegociável\textunderscore .
\section{Innervar}
\textunderscore v. t.\textunderscore  (e der.)
(V. \textunderscore ennervar\textunderscore , etc)
\section{Innérveo}
\begin{itemize}
\item {Grp. gram.:adj.}
\end{itemize}
\begin{itemize}
\item {Utilização:Bot.}
\end{itemize}
\begin{itemize}
\item {Proveniência:(De \textunderscore in...\textunderscore  + \textunderscore nervo\textunderscore )}
\end{itemize}
Que não tem nervura.
\section{Innocência}
\begin{itemize}
\item {Grp. gram.:f.}
\end{itemize}
\begin{itemize}
\item {Proveniência:(Lat. \textunderscore innocentia\textunderscore )}
\end{itemize}
Qualidade daquelle ou daquillo que é inocente.
Estado de quem não peccou ou de quem não tem culpas.
Virgindade.
Simplicidade; ingenuidade.
\section{Innocentar}
\begin{itemize}
\item {Grp. gram.:v. t.}
\end{itemize}
\begin{itemize}
\item {Utilização:Neol.}
\end{itemize}
Considerar innocente; desculpar.
\section{Innocente}
\begin{itemize}
\item {Grp. gram.:adj.}
\end{itemize}
\begin{itemize}
\item {Grp. gram.:M.  e  f.}
\end{itemize}
\begin{itemize}
\item {Proveniência:(Lat. \textunderscore innocens\textunderscore )}
\end{itemize}
Que não faz mal ou damno.
Innoffensivo.
Que não é culpado.
Immaculado, puro.
Ingênuo; simples.
Criança; pessôa innocente: \textunderscore a degolação dos innocentes\textunderscore .
\section{Innocentemente}
\begin{itemize}
\item {Grp. gram.:adv.}
\end{itemize}
De modo innocente.
\section{Innocuidade}
\begin{itemize}
\item {fónica:cu-i}
\end{itemize}
\begin{itemize}
\item {Grp. gram.:f.}
\end{itemize}
Qualidade daquillo que é innócuo.
\section{Innócuo}
\begin{itemize}
\item {Grp. gram.:adj.}
\end{itemize}
\begin{itemize}
\item {Proveniência:(Lat. \textunderscore inoquus\textunderscore )}
\end{itemize}
Que não prejudica, que não faz damno; inoffensivo; innocente.
\section{Innominado}
\begin{itemize}
\item {Grp. gram.:adj.}
\end{itemize}
\begin{itemize}
\item {Proveniência:(Lat. \textunderscore innominatus\textunderscore )}
\end{itemize}
Não nomeado; que não tem nome.
Que não é designado.
\section{Innominável}
\begin{itemize}
\item {Grp. gram.:adj.}
\end{itemize}
\begin{itemize}
\item {Proveniência:(Lat. \textunderscore innominabilis\textunderscore )}
\end{itemize}
Que se não póde designar por um nome.
\section{Innovação}
\begin{itemize}
\item {Grp. gram.:f.}
\end{itemize}
\begin{itemize}
\item {Proveniência:(Lat. \textunderscore innovatio\textunderscore )}
\end{itemize}
Acto ou effeito de innovar.
\section{Innovador}
\begin{itemize}
\item {Grp. gram.:adj.}
\end{itemize}
\begin{itemize}
\item {Grp. gram.:M.}
\end{itemize}
Que innova.
Aquelle que innova.
\section{Innovar}
\begin{itemize}
\item {Grp. gram.:v. t.}
\end{itemize}
\begin{itemize}
\item {Utilização:Des.}
\end{itemize}
\begin{itemize}
\item {Proveniência:(Lat. \textunderscore innovare\textunderscore )}
\end{itemize}
Tornar novo.
Renovar.
Introduzir novidades em.
Consertar.
\section{Innóxio}
\begin{itemize}
\item {Grp. gram.:adj.}
\end{itemize}
\begin{itemize}
\item {Proveniência:(Lat. \textunderscore innoxius\textunderscore )}
\end{itemize}
O mesmo que \textunderscore innócuo\textunderscore .
\section{Innúbil}
\begin{itemize}
\item {Grp. gram.:adj.}
\end{itemize}
\begin{itemize}
\item {Proveniência:(De \textunderscore in...\textunderscore  + \textunderscore núbil\textunderscore )}
\end{itemize}
Que não é núbil; que ainda não está em idade de casar.
\section{Ínnubo}
\begin{itemize}
\item {Grp. gram.:adj.}
\end{itemize}
\begin{itemize}
\item {Proveniência:(Lat. \textunderscore innubus\textunderscore )}
\end{itemize}
O mesmo que \textunderscore innupto\textunderscore .
\section{Innumerabilidade}
\begin{itemize}
\item {Grp. gram.:f.}
\end{itemize}
\begin{itemize}
\item {Proveniência:(Lat. \textunderscore innumerabilitas\textunderscore )}
\end{itemize}
Qualidade daquillo que é inumerável.
\section{Innumerável}
\begin{itemize}
\item {Grp. gram.:adj.}
\end{itemize}
\begin{itemize}
\item {Proveniência:(Lat. \textunderscore innumerabilis\textunderscore )}
\end{itemize}
Que não é numerável.
Que se não póde numerar ou contar.
Infinito em número: \textunderscore as estrêllas innumeráveis\textunderscore .
Extraordinariamente numeroso: \textunderscore multidão innumerável\textunderscore .
\section{Innumeravelmente}
\begin{itemize}
\item {Grp. gram.:adv.}
\end{itemize}
De modo innumerável.
\section{Innúmero}
\begin{itemize}
\item {Grp. gram.:adj.}
\end{itemize}
\begin{itemize}
\item {Proveniência:(Lat. \textunderscore innumerus\textunderscore )}
\end{itemize}
O mesmo que \textunderscore innumerável\textunderscore .
\section{Innumeroso}
\begin{itemize}
\item {Grp. gram.:adj.}
\end{itemize}
O mesmo que \textunderscore innumerável\textunderscore .
\section{Innupto}
\begin{itemize}
\item {Grp. gram.:adj.}
\end{itemize}
\begin{itemize}
\item {Proveniência:(Lat. \textunderscore inuptus\textunderscore )}
\end{itemize}
Que não é casado; que está solteiro ou é celibatario.
\section{Innutrível}
\begin{itemize}
\item {Grp. gram.:adj.}
\end{itemize}
\begin{itemize}
\item {Proveniência:(De \textunderscore in...\textunderscore  + \textunderscore nutrível\textunderscore )}
\end{itemize}
Que não nutre ou que nutre pouco; que não é nutritivo.
\section{...ino}
\begin{itemize}
\item {Grp. gram.:suf. adj.}
\end{itemize}
(designativo de \textunderscore deminuição\textunderscore , \textunderscore pertença\textunderscore , \textunderscore relação\textunderscore )
\section{Inobediência}
\begin{itemize}
\item {Grp. gram.:f.}
\end{itemize}
\begin{itemize}
\item {Proveniência:(Lat. \textunderscore inobedientia\textunderscore )}
\end{itemize}
O mesmo que \textunderscore desobediência\textunderscore .
\section{Inobediente}
\begin{itemize}
\item {Grp. gram.:adj.}
\end{itemize}
\begin{itemize}
\item {Proveniência:(Lat. \textunderscore inobediens\textunderscore )}
\end{itemize}
O mesmo que \textunderscore desobediente\textunderscore .
\section{Inobliterável}
\begin{itemize}
\item {Grp. gram.:adj.}
\end{itemize}
\begin{itemize}
\item {Proveniência:(De \textunderscore in...\textunderscore  + \textunderscore obliterável\textunderscore )}
\end{itemize}
Que se não póde obliterar.
\section{Inobservado}
\begin{itemize}
\item {Grp. gram.:adj.}
\end{itemize}
\begin{itemize}
\item {Proveniência:(Lat. \textunderscore inobservatus\textunderscore )}
\end{itemize}
Não observado.
Que nunca se viu.
\section{Inobservância}
\begin{itemize}
\item {Grp. gram.:f.}
\end{itemize}
\begin{itemize}
\item {Proveniência:(Lat. \textunderscore inobservantia\textunderscore )}
\end{itemize}
Qualidade de quem não é observante; falta de observância.
\section{Inobservante}
\begin{itemize}
\item {Grp. gram.:adj.}
\end{itemize}
\begin{itemize}
\item {Proveniência:(Lat. \textunderscore inobservans\textunderscore )}
\end{itemize}
Que não observa; que não cumpre.
\section{Inobservável}
\begin{itemize}
\item {Grp. gram.:adj.}
\end{itemize}
\begin{itemize}
\item {Proveniência:(Lat. \textunderscore inobservabilis\textunderscore )}
\end{itemize}
Que se não póde observar ou cumprir.
\section{Inocarpina}
\begin{itemize}
\item {Grp. gram.:f.}
\end{itemize}
Substância còrante, extrahida do inocarpo.
\section{Inocarpo}
\begin{itemize}
\item {Grp. gram.:m.}
\end{itemize}
\begin{itemize}
\item {Proveniência:(Do gr. \textunderscore is\textunderscore , \textunderscore inos\textunderscore  + \textunderscore karpos\textunderscore )}
\end{itemize}
Gênero de árvores resinosas da Ásia e da Oceânia.
\section{Inocclusão}
\begin{itemize}
\item {Grp. gram.:f.}
\end{itemize}
\begin{itemize}
\item {Utilização:Med.}
\end{itemize}
\begin{itemize}
\item {Proveniência:(De \textunderscore in...\textunderscore  + \textunderscore occlusão\textunderscore )}
\end{itemize}
Cerramento incompleto, ou falta de cerramento, de orifícios naturaes: \textunderscore inocclusão mitral\textunderscore .
\section{Inoccupado}
\begin{itemize}
\item {Grp. gram.:adj.}
\end{itemize}
\begin{itemize}
\item {Proveniência:(De \textunderscore in...\textunderscore  + \textunderscore occupado\textunderscore )}
\end{itemize}
Que não está nem foi occupado.
Em que não se exerceram actos de occupação; desoccupado.
\section{Inocência}
\begin{itemize}
\item {Grp. gram.:f.}
\end{itemize}
\begin{itemize}
\item {Proveniência:(Lat. \textunderscore innocentia\textunderscore )}
\end{itemize}
Qualidade daquele ou daquilo que é inocente.
Estado de quem não pecou ou de quem não tem culpas.
Virgindade.
Simplicidade; ingenuidade.
\section{Inocentar}
\begin{itemize}
\item {Grp. gram.:v. t.}
\end{itemize}
\begin{itemize}
\item {Utilização:Neol.}
\end{itemize}
Considerar inocente; desculpar.
\section{Inocente}
\begin{itemize}
\item {Grp. gram.:adj.}
\end{itemize}
\begin{itemize}
\item {Grp. gram.:M.  e  f.}
\end{itemize}
\begin{itemize}
\item {Proveniência:(Lat. \textunderscore innocens\textunderscore )}
\end{itemize}
Que não faz mal ou dano.
Inofensivo.
Que não é culpado.
Imaculado, puro.
Ingênuo; simples.
Criança; pessôa inocente: \textunderscore a degolação dos inocentes\textunderscore .
\section{Inocentemente}
\begin{itemize}
\item {Grp. gram.:adv.}
\end{itemize}
De modo inocente.
\section{Inocioso}
\begin{itemize}
\item {Grp. gram.:adj.}
\end{itemize}
\begin{itemize}
\item {Proveniência:(De \textunderscore in...\textunderscore  + \textunderscore ocioso\textunderscore )}
\end{itemize}
Não ocioso.
\section{Inoclusão}
\begin{itemize}
\item {Grp. gram.:f.}
\end{itemize}
\begin{itemize}
\item {Utilização:Med.}
\end{itemize}
\begin{itemize}
\item {Proveniência:(De \textunderscore in...\textunderscore  + \textunderscore oclusão\textunderscore )}
\end{itemize}
Cerramento incompleto, ou falta de cerramento, de orifícios naturaes: \textunderscore inoclusão mitral\textunderscore .
\section{Inocuidade}
\begin{itemize}
\item {fónica:cu-i}
\end{itemize}
\begin{itemize}
\item {Grp. gram.:f.}
\end{itemize}
Qualidade daquilo que é inócuo.
\section{Inoculabilidade}
\begin{itemize}
\item {Grp. gram.:f.}
\end{itemize}
Qualidade daquillo que é inoculável.
\section{Inoculação}
\begin{itemize}
\item {Grp. gram.:f.}
\end{itemize}
\begin{itemize}
\item {Proveniência:(Lat. \textunderscore inoculatio\textunderscore )}
\end{itemize}
Acto ou effeito de inocular.
\section{Inoculador}
\begin{itemize}
\item {Grp. gram.:adj.}
\end{itemize}
\begin{itemize}
\item {Grp. gram.:M.}
\end{itemize}
\begin{itemize}
\item {Proveniência:(Lat. \textunderscore inoculator\textunderscore )}
\end{itemize}
Que inocula.
Aquelle que inocula.
\section{Inocular}
\begin{itemize}
\item {Grp. gram.:v. t.}
\end{itemize}
\begin{itemize}
\item {Utilização:Fig.}
\end{itemize}
\begin{itemize}
\item {Utilização:P. us.}
\end{itemize}
\begin{itemize}
\item {Proveniência:(Lat. \textunderscore inoculare\textunderscore )}
\end{itemize}
Inserir; introduzir no organismo: \textunderscore inocular cafeína\textunderscore .
Transmittir; diffundir: \textunderscore inocular vícios\textunderscore .
Contagiar.
Enxertar de borbulha ou de gomo.
\section{Inoculável}
\begin{itemize}
\item {Grp. gram.:adj.}
\end{itemize}
Que se póde inocular.
\section{Inoculista}
\begin{itemize}
\item {Grp. gram.:m.}
\end{itemize}
\begin{itemize}
\item {Utilização:Des.}
\end{itemize}
\begin{itemize}
\item {Proveniência:(De \textunderscore inocular\textunderscore )}
\end{itemize}
Defensor do systema de enxertia de borbulha.
\section{Inócuo}
\begin{itemize}
\item {Grp. gram.:adj.}
\end{itemize}
\begin{itemize}
\item {Proveniência:(Lat. \textunderscore inoquus\textunderscore )}
\end{itemize}
Que não prejudica, que não faz dano; inofensivo; inocente.
\section{Inocupado}
\begin{itemize}
\item {Grp. gram.:adj.}
\end{itemize}
\begin{itemize}
\item {Proveniência:(De \textunderscore in...\textunderscore  + \textunderscore ocupado\textunderscore )}
\end{itemize}
Que não está nem foi ocupado.
Em que não se exerceram actos de ocupação; desocupado.
\section{Inodoro}
\begin{itemize}
\item {Grp. gram.:adj.}
\end{itemize}
\begin{itemize}
\item {Proveniência:(Lat. \textunderscore inodorus\textunderscore )}
\end{itemize}
Que não tem odor; que não exhala cheiro.
\section{Inodular}
\begin{itemize}
\item {Grp. gram.:adj.}
\end{itemize}
Relativo a inódula.
\section{Inódula}
\begin{itemize}
\item {Grp. gram.:f.}
\end{itemize}
\begin{itemize}
\item {Proveniência:(Do gr. \textunderscore is\textunderscore , \textunderscore inos\textunderscore )}
\end{itemize}
Tecido fibroso, que se desenvolve nas chagas, determinando ou activando a cicatrização.
\section{Inofensivamente}
\begin{itemize}
\item {Grp. gram.:adv.}
\end{itemize}
De modo inofensivo.
Sem fazer dano; inocentemente.
\section{Inofensivo}
\begin{itemize}
\item {Grp. gram.:adj.}
\end{itemize}
\begin{itemize}
\item {Proveniência:(De \textunderscore in...\textunderscore  + \textunderscore ofensivo\textunderscore )}
\end{itemize}
Que não é ofensivo; que não dá mau resultado.
Que não faz mal; inocente: \textunderscore comidas inofensivas\textunderscore .
\section{Inoffensivamente}
\begin{itemize}
\item {Grp. gram.:adv.}
\end{itemize}
De modo inoffensivo.
Sem fazer damno; innocentemente.
\section{Inoffensivo}
\begin{itemize}
\item {Grp. gram.:adj.}
\end{itemize}
\begin{itemize}
\item {Proveniência:(De \textunderscore in...\textunderscore  + \textunderscore offensivo\textunderscore )}
\end{itemize}
Que não é offensivo; que não dá mau resultado.
Que não faz mal; innocente: \textunderscore comidas inoffensivas\textunderscore .
\section{Inofficiosamente}
\begin{itemize}
\item {Grp. gram.:adv.}
\end{itemize}
De modo inofficioso.
Prejudicialmente.
\section{Inofficioso}
\begin{itemize}
\item {Grp. gram.:adj.}
\end{itemize}
\begin{itemize}
\item {Proveniência:(Lat. \textunderscore inofficiosus\textunderscore )}
\end{itemize}
Que não é officioso.
Nocivo.
Que vai prejudicar terceira pessôa.
Que prejudica, sem razão conhecida.
\section{Inoficiosamente}
\begin{itemize}
\item {Grp. gram.:adv.}
\end{itemize}
De modo inofficioso.
Prejudicialmente.
\section{Inoficioso}
\begin{itemize}
\item {Grp. gram.:adj.}
\end{itemize}
\begin{itemize}
\item {Proveniência:(Lat. \textunderscore inofficiosus\textunderscore )}
\end{itemize}
Que não é oficioso.
Nocivo.
Que vai prejudicar terceira pessôa.
Que prejudica, sem razão conhecida.
\section{Inolente}
\begin{itemize}
\item {Grp. gram.:adj.}
\end{itemize}
\begin{itemize}
\item {Proveniência:(De \textunderscore in...\textunderscore  + \textunderscore olente\textunderscore )}
\end{itemize}
Que não tem cheiro; inodoro.
\section{Inolvidável}
\begin{itemize}
\item {Grp. gram.:adj.}
\end{itemize}
\begin{itemize}
\item {Proveniência:(De \textunderscore in...\textunderscore  + \textunderscore olvidável\textunderscore )}
\end{itemize}
Que se não póde ou que se não deve olvidar.
Digno do sêr lembrado.
\section{Inominado}
\begin{itemize}
\item {Grp. gram.:adj.}
\end{itemize}
\begin{itemize}
\item {Proveniência:(Lat. \textunderscore innominatus\textunderscore )}
\end{itemize}
Não nomeado; que não tem nome.
Que não é designado.
\section{Inominável}
\begin{itemize}
\item {Grp. gram.:adj.}
\end{itemize}
\begin{itemize}
\item {Proveniência:(Lat. \textunderscore innominabilis\textunderscore )}
\end{itemize}
Que se não póde designar por um nome.
\section{Inonestamente}
\begin{itemize}
\item {Grp. gram.:adv.}
\end{itemize}
O mesmo que \textunderscore desonestamente\textunderscore .
\section{Inonestidade}
\begin{itemize}
\item {Grp. gram.:f.}
\end{itemize}
O mesmo que \textunderscore desonestidade\textunderscore .
\section{Inonesto}
\begin{itemize}
\item {Grp. gram.:f.}
\end{itemize}
O mesmo que \textunderscore desonesto\textunderscore .
\section{Inoperação}
\begin{itemize}
\item {Grp. gram.:f.}
\end{itemize}
\begin{itemize}
\item {Proveniência:(Do lat. \textunderscore inoperari\textunderscore )}
\end{itemize}
Obra, producto, (em sentido theológico). Cf. Bernárdez, \textunderscore Luz e Calor\textunderscore , 428.
\section{Inopexia}
\begin{itemize}
\item {fónica:csi}
\end{itemize}
\begin{itemize}
\item {Grp. gram.:f.}
\end{itemize}
Exaggêro da coagulabilidade do sangue.
\section{Inópia}
\begin{itemize}
\item {Grp. gram.:f.}
\end{itemize}
\begin{itemize}
\item {Utilização:Fig.}
\end{itemize}
\begin{itemize}
\item {Proveniência:(Lat. \textunderscore inopia\textunderscore )}
\end{itemize}
Falta de riqueza; penúria.
Defeito.
\section{Inopinadamente}
\begin{itemize}
\item {Grp. gram.:adv.}
\end{itemize}
De modo inopinado; subitamente; imprevistamente.
\section{Inopinado}
\begin{itemize}
\item {Grp. gram.:adj.}
\end{itemize}
\begin{itemize}
\item {Grp. gram.:M.}
\end{itemize}
\begin{itemize}
\item {Utilização:Rhet.}
\end{itemize}
\begin{itemize}
\item {Proveniência:(Lat. \textunderscore inopinatus\textunderscore )}
\end{itemize}
Imprevisto; repentino.
Extraordinário.
Suspensão.
\section{Inopinável}
\begin{itemize}
\item {Grp. gram.:adj.}
\end{itemize}
\begin{itemize}
\item {Proveniência:(Lat. \textunderscore inopinabilis\textunderscore )}
\end{itemize}
Que se não póde prever.
Que se não póde apreciar:«\textunderscore inopináveis grandezas\textunderscore ». \textunderscore Luz e Calor\textunderscore .
\section{Inopino}
\begin{itemize}
\item {Grp. gram.:adj.}
\end{itemize}
\begin{itemize}
\item {Utilização:Poét.}
\end{itemize}
\begin{itemize}
\item {Proveniência:(Lat. \textunderscore inopinus\textunderscore )}
\end{itemize}
O mesmo que \textunderscore inopinado\textunderscore .
\section{Inopioso}
\begin{itemize}
\item {Grp. gram.:adj.}
\end{itemize}
\begin{itemize}
\item {Proveniência:(Lat. \textunderscore inopiosus\textunderscore )}
\end{itemize}
Que tem inópia; que é pobre.
\section{Inoportunamente}
\begin{itemize}
\item {Grp. gram.:adv.}
\end{itemize}
De modo inoportuno.
Sem oportunidade; fóra do tempo próprio ou conveniente.
\section{Inoportunidade}
\begin{itemize}
\item {Grp. gram.:f.}
\end{itemize}
\begin{itemize}
\item {Proveniência:(Lat. \textunderscore inopportunitas\textunderscore )}
\end{itemize}
Qualidade de inoportuno.
Falta de oportunidade.
\section{Inoportuno}
\begin{itemize}
\item {Grp. gram.:adj.}
\end{itemize}
\begin{itemize}
\item {Proveniência:(Lat. \textunderscore inopportunus\textunderscore )}
\end{itemize}
Não oportuno; que vem, ou que sucede, ou que se faz, fóra de tempo, fóra de ocasião própria ou conveniente: \textunderscore censura inoportuna\textunderscore .
\section{Inopportunamente}
\begin{itemize}
\item {Grp. gram.:adv.}
\end{itemize}
De modo inopportuno.
Sem opportunidade; fóra do tempo próprio ou conveniente.
\section{Inopportunidade}
\begin{itemize}
\item {Grp. gram.:f.}
\end{itemize}
\begin{itemize}
\item {Proveniência:(Lat. \textunderscore inopportunitas\textunderscore )}
\end{itemize}
Qualidade de inopportuno.
Falta de opportunidade.
\section{Inopportuno}
\begin{itemize}
\item {Grp. gram.:adj.}
\end{itemize}
\begin{itemize}
\item {Proveniência:(Lat. \textunderscore inopportunus\textunderscore )}
\end{itemize}
Não opportuno; que vem, ou que succede, ou que se faz, fóra de tempo, fóra de occasião própria ou conveniente: \textunderscore censura inopportuna\textunderscore .
\section{Inopprimido}
\begin{itemize}
\item {Grp. gram.:adj.}
\end{itemize}
\begin{itemize}
\item {Proveniência:(De \textunderscore in...\textunderscore  + \textunderscore opprimido\textunderscore )}
\end{itemize}
Não opprimido, desopprimido.
\section{Inoprimido}
\begin{itemize}
\item {Grp. gram.:adj.}
\end{itemize}
\begin{itemize}
\item {Proveniência:(De \textunderscore in...\textunderscore  + \textunderscore oprimido\textunderscore )}
\end{itemize}
Não oprimido, desoprimido.
\section{Inóquo}
\begin{itemize}
\item {Grp. gram.:adj.}
\end{itemize}
\begin{itemize}
\item {Proveniência:(Lat. \textunderscore inoquus\textunderscore )}
\end{itemize}
Que não prejudica, que não faz dano; inofensivo; inocente.
\section{Inorar}
\begin{itemize}
\item {Grp. gram.:v. t.}
\end{itemize}
\begin{itemize}
\item {Utilização:Ant.}
\end{itemize}
\begin{itemize}
\item {Utilização:Pop.}
\end{itemize}
O mesmo que \textunderscore ignorar\textunderscore . Cf. Simão Machado, f. 29.
Estranhar, notar, censurar.
(Contr. de \textunderscore ignorar\textunderscore )
\section{Inorgânico}
\begin{itemize}
\item {Grp. gram.:adj.}
\end{itemize}
\begin{itemize}
\item {Proveniência:(De \textunderscore in...\textunderscore  + \textunderscore orgânico\textunderscore )}
\end{itemize}
Não orgânico; que não é organizado; que não tem órgãos.
Que não tem vida: \textunderscore corpos inorgânicos\textunderscore .
\section{Inorganismo}
\begin{itemize}
\item {Grp. gram.:m.}
\end{itemize}
\begin{itemize}
\item {Proveniência:(De \textunderscore in...\textunderscore  + \textunderscore organismo\textunderscore )}
\end{itemize}
Ausência de fórma orgânica.
Substância, desprovida de órgãos.
Substância, que não é animal nem vegetal, mas mineral.
\section{Inorganizado}
\begin{itemize}
\item {Grp. gram.:adj.}
\end{itemize}
\begin{itemize}
\item {Proveniência:(De \textunderscore in...\textunderscore  + \textunderscore organizado\textunderscore )}
\end{itemize}
Que não é organizado; inorgânico.
\section{Inosculação}
\begin{itemize}
\item {Grp. gram.:f.}
\end{itemize}
\begin{itemize}
\item {Utilização:Anat.}
\end{itemize}
\begin{itemize}
\item {Proveniência:(Do lat. \textunderscore in\textunderscore  + \textunderscore osculum\textunderscore )}
\end{itemize}
Anastomose em arco.
\section{Inósico}
\begin{itemize}
\item {Grp. gram.:adj.}
\end{itemize}
\begin{itemize}
\item {Proveniência:(Do gr. \textunderscore is\textunderscore , \textunderscore inos\textunderscore )}
\end{itemize}
Diz-se de um ácido, extrahido do tecido muscular dos mammíferos.
\section{Inosita}
\begin{itemize}
\item {Grp. gram.:f.}
\end{itemize}
\begin{itemize}
\item {Proveniência:(Al. \textunderscore inosit\textunderscore )}
\end{itemize}
Substância branca, de sabor açucarado.
\section{Inosite}
\begin{itemize}
\item {Grp. gram.:f.}
\end{itemize}
\begin{itemize}
\item {Proveniência:(Al. \textunderscore inosit\textunderscore )}
\end{itemize}
Substância branca, de sabor açucarado.
\section{Inosituria}
\begin{itemize}
\item {Grp. gram.:f.}
\end{itemize}
O mesmo que \textunderscore inosuria\textunderscore .
\section{Inospedeiro}
\begin{itemize}
\item {Grp. gram.:adj.}
\end{itemize}
O mesmo que \textunderscore inóspito\textunderscore . Cf. Castilho, \textunderscore Fastos\textunderscore , II, 157.
\section{Inospitaleiramente}
\begin{itemize}
\item {Grp. gram.:adv.}
\end{itemize}
De modo inospitaleiro.
Sem vontade de receber estranjeiros.
\section{Inospitaleiro}
\begin{itemize}
\item {Grp. gram.:adj.}
\end{itemize}
\begin{itemize}
\item {Proveniência:(De \textunderscore in...\textunderscore  + \textunderscore hospitaleiro\textunderscore )}
\end{itemize}
Que não é hospitaleiro.
Que é desfavorável a estranjeiros ou que os não recebe.
O mesmo que \textunderscore inóspito\textunderscore .
\section{Inospitalidade}
\begin{itemize}
\item {Grp. gram.:f.}
\end{itemize}
\begin{itemize}
\item {Proveniência:(De \textunderscore in...\textunderscore  + \textunderscore hospitalidade\textunderscore )}
\end{itemize}
Falta de hospitalidade.
Recusa de receber estranjeiros.
\section{Inóspito}
\begin{itemize}
\item {Grp. gram.:adj.}
\end{itemize}
\begin{itemize}
\item {Proveniência:(Lat. \textunderscore inhospitus\textunderscore )}
\end{itemize}
Que não é apto para hospedar.
Que não pratíca a hospitalidade. Em que se não póde viver: \textunderscore terras inóspitas\textunderscore .
\section{Inosuria}
\begin{itemize}
\item {Grp. gram.:f.}
\end{itemize}
\begin{itemize}
\item {Proveniência:(Do rad. de \textunderscore inosite\textunderscore  + gr. \textunderscore ouron\textunderscore )}
\end{itemize}
Doença, determinada pela presença da inosite na urina.
\section{Inosúrico}
\begin{itemize}
\item {Grp. gram.:adj.}
\end{itemize}
\begin{itemize}
\item {Grp. gram.:M.}
\end{itemize}
Relativo á inosuria.
Aquelle que padece inosuria.
\section{Inovação}
\begin{itemize}
\item {Grp. gram.:f.}
\end{itemize}
\begin{itemize}
\item {Proveniência:(Lat. \textunderscore innovatio\textunderscore )}
\end{itemize}
Acto ou efeito de inovar.
\section{Inovador}
\begin{itemize}
\item {Grp. gram.:adj.}
\end{itemize}
\begin{itemize}
\item {Grp. gram.:M.}
\end{itemize}
Que inova.
Aquele que inova.
\section{Inovar}
\begin{itemize}
\item {Grp. gram.:v. t.}
\end{itemize}
\begin{itemize}
\item {Utilização:Des.}
\end{itemize}
\begin{itemize}
\item {Proveniência:(Lat. \textunderscore innovare\textunderscore )}
\end{itemize}
Tornar novo.
Renovar.
Introduzir novidades em.
Consertar.
\section{Inoxidável}
\begin{itemize}
\item {fónica:csi}
\end{itemize}
\begin{itemize}
\item {Grp. gram.:adj.}
\end{itemize}
\begin{itemize}
\item {Proveniência:(De \textunderscore in...\textunderscore  + \textunderscore oxidável\textunderscore )}
\end{itemize}
Que não é oxidável; que se não oxida ou que se não póde oxidar.
\section{Inóxio}
\begin{itemize}
\item {Grp. gram.:adj.}
\end{itemize}
\begin{itemize}
\item {Proveniência:(Lat. \textunderscore innoxius\textunderscore )}
\end{itemize}
O mesmo que \textunderscore inócuo\textunderscore .
\section{Inoxydável}
\begin{itemize}
\item {Grp. gram.:adj.}
\end{itemize}
\begin{itemize}
\item {Proveniência:(De \textunderscore in...\textunderscore  + \textunderscore oxydável\textunderscore )}
\end{itemize}
Que não é oxydável; que se não oxyda ou que se não póde oxydar.
\section{Inqualificável}
\begin{itemize}
\item {Grp. gram.:adj.}
\end{itemize}
\begin{itemize}
\item {Proveniência:(De \textunderscore in...\textunderscore  + \textunderscore qualificável\textunderscore )}
\end{itemize}
Que não é qualificável.
Indigno; vilíssimo: \textunderscore acções inqualificáveis\textunderscore .
\section{Inquartação}
\begin{itemize}
\item {Grp. gram.:f.}
\end{itemize}
\begin{itemize}
\item {Proveniência:(De \textunderscore inquartar\textunderscore )}
\end{itemize}
Liga metállica, em que o oiro está para com a prata na relação de 1 quarto para 3 quartos.
\section{Inquartar}
\begin{itemize}
\item {Grp. gram.:v. t.}
\end{itemize}
\begin{itemize}
\item {Proveniência:(De \textunderscore quarto\textunderscore )}
\end{itemize}
Dar inquartação a (o oiro). Cf. M. I. F. Mendonça, \textunderscore Vocab. Techn.\textunderscore 
\section{In-quarto}
\begin{itemize}
\item {Grp. gram.:m.}
\end{itemize}
\begin{itemize}
\item {Proveniência:(T. lat.)}
\end{itemize}
Volume, cujas fôlhas foram impressas a oito páginas cada uma.
Formato de livro, duplo do oitavo.
\section{Inquebrantável}
\begin{itemize}
\item {Grp. gram.:adj.}
\end{itemize}
\begin{itemize}
\item {Proveniência:(De \textunderscore in...\textunderscore  + \textunderscore quebrantar\textunderscore )}
\end{itemize}
Que se não póde quebrantar; inflexível; persistente.
Indefesso.
\section{Inquerição}
\begin{itemize}
\item {Grp. gram.:f.}
\end{itemize}
Acto de inquerir.
\section{Inquerideira}
\begin{itemize}
\item {Grp. gram.:f.}
\end{itemize}
\begin{itemize}
\item {Proveniência:(De \textunderscore inquerir\textunderscore )}
\end{itemize}
Corda, com que se aperta a carga dos animaes.
\section{Inquerir}
\begin{itemize}
\item {Grp. gram.:v. t.}
\end{itemize}
\begin{itemize}
\item {Proveniência:(Do gr. \textunderscore enkheirein\textunderscore ?)}
\end{itemize}
Apertar (a carga).
\section{Inquérito}
\begin{itemize}
\item {Grp. gram.:m.}
\end{itemize}
\begin{itemize}
\item {Proveniência:(Do rad. do lat. \textunderscore quaeritare\textunderscore )}
\end{itemize}
Acto ou effeito de inquirir.
Syndicância; devassa.
\section{Inquestionável}
\begin{itemize}
\item {Grp. gram.:adj.}
\end{itemize}
\begin{itemize}
\item {Proveniência:(De \textunderscore in...\textunderscore  + \textunderscore questionável\textunderscore )}
\end{itemize}
Que não é questionável.
Indiscutível; inconcusso.
\section{Inquestionavelmente}
\begin{itemize}
\item {Grp. gram.:adv.}
\end{itemize}
De modo inquestionável.
\section{Inquietação}
\begin{itemize}
\item {Grp. gram.:f.}
\end{itemize}
\begin{itemize}
\item {Proveniência:(Lat. \textunderscore inquietatio\textunderscore )}
\end{itemize}
Estado de inquieto; falta de quietação.
Excitação; agitação.
\section{Inquietador}
\begin{itemize}
\item {Grp. gram.:adj.}
\end{itemize}
\begin{itemize}
\item {Grp. gram.:M.}
\end{itemize}
\begin{itemize}
\item {Proveniência:(Lat. \textunderscore inquietator\textunderscore )}
\end{itemize}
Que inquieta.
Aquelle que inquieta.
\section{Inquietamente}
\begin{itemize}
\item {Grp. gram.:adv.}
\end{itemize}
De modo inquieto.
\section{Inquietamento}
\begin{itemize}
\item {Grp. gram.:m.}
\end{itemize}
O mesmo que \textunderscore inquietação\textunderscore .
\section{Inquietar}
\begin{itemize}
\item {Grp. gram.:v. t.}
\end{itemize}
\begin{itemize}
\item {Grp. gram.:V. p.}
\end{itemize}
\begin{itemize}
\item {Proveniência:(Lat. \textunderscore inquietare\textunderscore )}
\end{itemize}
Tornar inquieto.
Tirar o sossêgo a.
Excitar; amotinar.
Amofinar; perturbar.
Hostilizar.
Estar inquieto.
Têr grandes cuidados.
Amofinar-se, apoquentar-se.
\section{Inquieto}
\begin{itemize}
\item {Grp. gram.:adj.}
\end{itemize}
\begin{itemize}
\item {Proveniência:(Lat. \textunderscore inquietus\textunderscore )}
\end{itemize}
Não quieto; desassossegado; turbulento; agitado.
Apprehensivo.
\section{Inquietude}
\begin{itemize}
\item {Grp. gram.:f.}
\end{itemize}
\begin{itemize}
\item {Proveniência:(De \textunderscore in...\textunderscore  + \textunderscore quietude\textunderscore )}
\end{itemize}
O mesmo que \textunderscore inquietação\textunderscore . Cf. Latino, \textunderscore Elogios\textunderscore , 280.
\section{Inquilina}
\begin{itemize}
\item {Grp. gram.:f.}
\end{itemize}
\begin{itemize}
\item {Proveniência:(De \textunderscore inquilino\textunderscore )}
\end{itemize}
Mulher, que tomou casa de arrendamento e habita nella.
\section{Inquilinagem}
\begin{itemize}
\item {Grp. gram.:f.}
\end{itemize}
O mesmo que \textunderscore inquilinato\textunderscore .
\section{Inquilinar}
\begin{itemize}
\item {Grp. gram.:v. i.}
\end{itemize}
\begin{itemize}
\item {Utilização:Des.}
\end{itemize}
Sêr inquilino; estabelecer morada. Cf. Filinto, XXII, 81.
\section{Inquilinato}
\begin{itemize}
\item {Grp. gram.:m.}
\end{itemize}
\begin{itemize}
\item {Proveniência:(Lat. \textunderscore inquilinatus\textunderscore )}
\end{itemize}
Estado de quem reside em casa alugada.
\section{Inquilino}
\begin{itemize}
\item {Grp. gram.:m.}
\end{itemize}
\begin{itemize}
\item {Utilização:Ant.}
\end{itemize}
\begin{itemize}
\item {Proveniência:(Lat. \textunderscore inquilinus\textunderscore )}
\end{itemize}
Aquelle que reside em casa arrendada, especialmente o que é chefe de família.
O mesmo que \textunderscore emphyteuta\textunderscore  ou senhorio directo de um prédio.
\section{Inquimba}
\begin{itemize}
\item {Grp. gram.:m.}
\end{itemize}
Lingua, falada nas margens do Zaire.
\section{Inquimbas}
\begin{itemize}
\item {Grp. gram.:m. pl.}
\end{itemize}
Adoradores do ídolo Inquimba, no Congo.
\section{Inquinação}
\begin{itemize}
\item {Grp. gram.:f.}
\end{itemize}
\begin{itemize}
\item {Proveniência:(Lat. \textunderscore inquinatio\textunderscore )}
\end{itemize}
Acto ou effeito de inquinar.
\section{Inquinador}
\begin{itemize}
\item {Grp. gram.:adj.}
\end{itemize}
\begin{itemize}
\item {Proveniência:(Lat. \textunderscore inquinator\textunderscore )}
\end{itemize}
Que inquina, que suja.
\section{Inquinamento}
\begin{itemize}
\item {Grp. gram.:m.}
\end{itemize}
\begin{itemize}
\item {Proveniência:(Lat. \textunderscore inquinamentum\textunderscore )}
\end{itemize}
O mesmo que \textunderscore inquinação\textunderscore .
\section{Inquinar}
\begin{itemize}
\item {Grp. gram.:v. t.}
\end{itemize}
\begin{itemize}
\item {Proveniência:(Lat. \textunderscore inquinare\textunderscore )}
\end{itemize}
Cobrir de manchas; sujar; polluír.
Corromper.
Infectar: \textunderscore águas inquinadas\textunderscore .
\section{Inquirição}
\begin{itemize}
\item {Grp. gram.:f.}
\end{itemize}
Acto ou effeito de inquirir; inquérito; syndicância.
Interrogatório judicial.
\section{Inquiridor}
\begin{itemize}
\item {Grp. gram.:adj.}
\end{itemize}
\begin{itemize}
\item {Grp. gram.:M.}
\end{itemize}
\begin{itemize}
\item {Utilização:Ant.}
\end{itemize}
\begin{itemize}
\item {Proveniência:(De \textunderscore inquirir\textunderscore )}
\end{itemize}
Que inquire.
Aquelle que inquire.
Official de justiça, que inquiria testemunhas.
\section{Inquiridoria}
\begin{itemize}
\item {Grp. gram.:f.}
\end{itemize}
\begin{itemize}
\item {Utilização:Ant.}
\end{itemize}
\begin{itemize}
\item {Proveniência:(De \textunderscore inquiridor\textunderscore )}
\end{itemize}
Cargo de inquiridor; inquirição.
\section{Inquirimento}
\begin{itemize}
\item {Grp. gram.:m.}
\end{itemize}
O mesmo que \textunderscore inquirição\textunderscore .
\section{Inquirir}
\begin{itemize}
\item {Grp. gram.:v. t.}
\end{itemize}
\begin{itemize}
\item {Proveniência:(Lat. \textunderscore inquirere\textunderscore )}
\end{itemize}
Procurar; investigar.
Colher informações de.
Interrogar.
Interrogar judicialmente (testemunhas).
\section{Inquisa}
\begin{itemize}
\item {Grp. gram.:f.}
\end{itemize}
\begin{itemize}
\item {Utilização:Ant.}
\end{itemize}
O mesmo que \textunderscore inquirição\textunderscore .
(Cp. lat. \textunderscore inquisitus\textunderscore )
\section{Inquisição}
\begin{itemize}
\item {Grp. gram.:f.}
\end{itemize}
\begin{itemize}
\item {Proveniência:(Lat. \textunderscore inquisitio\textunderscore )}
\end{itemize}
O mesmo que \textunderscore inquirição\textunderscore .
Antigo tribunal ecclesiástico, instituído para investigar e punir os crimes contra a fé cathólica.
Santo-offício.
Cárcere do santo-offício.
\section{Inquisidor}
\begin{itemize}
\item {Grp. gram.:m.}
\end{itemize}
\begin{itemize}
\item {Proveniência:(Lat. \textunderscore inquisitor\textunderscore )}
\end{itemize}
Juiz do tribunal da Inquisição.
Membro do santo-offício.
\section{Inquisitivo}
\begin{itemize}
\item {Grp. gram.:adj.}
\end{itemize}
\begin{itemize}
\item {Proveniência:(Do lat. \textunderscore inquisitus\textunderscore )}
\end{itemize}
Relativo a inquirição.
Interrogativo. Cf. Garrett, \textunderscore Helena\textunderscore , 22.
\section{Inquisitorial}
\begin{itemize}
\item {Grp. gram.:adj.}
\end{itemize}
\begin{itemize}
\item {Utilização:Fig.}
\end{itemize}
\begin{itemize}
\item {Utilização:Ext.}
\end{itemize}
\begin{itemize}
\item {Proveniência:(De \textunderscore inquisitório\textunderscore )}
\end{itemize}
Relativo á Inquisição ou aos inquisidores.
Deshumano; severo; terrível.
Muito arrogante.
\section{Inquisitório}
\begin{itemize}
\item {Grp. gram.:adj.}
\end{itemize}
O mesmo que \textunderscore inquisitorial\textunderscore .
(Do lat. \textunderscore inquisitus)\textunderscore .
\section{Inradiante}
\begin{itemize}
\item {Grp. gram.:adj.}
\end{itemize}
\begin{itemize}
\item {Utilização:Bot.}
\end{itemize}
\begin{itemize}
\item {Proveniência:(De \textunderscore in...\textunderscore  + \textunderscore radiante\textunderscore )}
\end{itemize}
Que não é radiante.
\section{Inremediável}
\begin{itemize}
\item {Grp. gram.:adj.}
\end{itemize}
\begin{itemize}
\item {Utilização:Ant.}
\end{itemize}
O mesmo que \textunderscore irremediável\textunderscore .
\section{Inrestaurável}
\begin{itemize}
\item {Grp. gram.:adj.}
\end{itemize}
\begin{itemize}
\item {Proveniência:(De \textunderscore in...\textunderscore  + \textunderscore restaurável\textunderscore )}
\end{itemize}
Que se não póde restaurar.
\section{Insabidade}
\begin{itemize}
\item {Grp. gram.:f.}
\end{itemize}
\begin{itemize}
\item {Utilização:Ant.}
\end{itemize}
Qualidade de quem é insabido.
\section{Insabido}
\begin{itemize}
\item {Grp. gram.:adj.}
\end{itemize}
\begin{itemize}
\item {Utilização:Ant.}
\end{itemize}
\begin{itemize}
\item {Proveniência:(De \textunderscore in...\textunderscore  + \textunderscore sabido\textunderscore )}
\end{itemize}
O mesmo que \textunderscore ignorante\textunderscore .
\section{Insacável}
\begin{itemize}
\item {Grp. gram.:adj.}
\end{itemize}
\begin{itemize}
\item {Utilização:Ant.}
\end{itemize}
\begin{itemize}
\item {Proveniência:(De \textunderscore in...\textunderscore  + \textunderscore sacar\textunderscore )}
\end{itemize}
O mesmo que \textunderscore inexhaurível\textunderscore .
\section{Insaciabilidade}
\begin{itemize}
\item {Grp. gram.:f.}
\end{itemize}
\begin{itemize}
\item {Proveniência:(Lat. \textunderscore insatiabilitas\textunderscore )}
\end{itemize}
Qualidade de insaciável.
\section{Insaciado}
\begin{itemize}
\item {Grp. gram.:adj.}
\end{itemize}
\begin{itemize}
\item {Proveniência:(Lat. \textunderscore insaciatus\textunderscore )}
\end{itemize}
Não saciado.
\section{Insaciável}
\begin{itemize}
\item {Grp. gram.:adj.}
\end{itemize}
\begin{itemize}
\item {Proveniência:(Lat. \textunderscore insatiabilis\textunderscore )}
\end{itemize}
Não saciável; que se não sacia, que se não farta.
Muito ambicioso.
Avaro; sôffrego.
\section{Insaciavelmente}
\begin{itemize}
\item {Grp. gram.:adv.}
\end{itemize}
De modo insaciável.
\section{Insaciedade}
\begin{itemize}
\item {Grp. gram.:f.}
\end{itemize}
\begin{itemize}
\item {Proveniência:(Lat. \textunderscore insatietas\textunderscore )}
\end{itemize}
Appetite insaciável.
\section{Insalivação}
\begin{itemize}
\item {Grp. gram.:f.}
\end{itemize}
Acto ou effeito de insalivar.
\section{Insalivar}
\begin{itemize}
\item {Grp. gram.:v. t.}
\end{itemize}
Impregnar de saliva (os alimentos).
\section{Insalubérrimo}
\begin{itemize}
\item {Grp. gram.:adj.}
\end{itemize}
\begin{itemize}
\item {Proveniência:(Do lat. \textunderscore insaluber\textunderscore )}
\end{itemize}
Muito insalubre.
\section{Insalubre}
\begin{itemize}
\item {Grp. gram.:adj.}
\end{itemize}
\begin{itemize}
\item {Proveniência:(Lat. \textunderscore insaluber\textunderscore )}
\end{itemize}
Que não é salubre; que causa doença; doentio: \textunderscore clima insalubre\textunderscore .
\section{Insalubremente}
\begin{itemize}
\item {Grp. gram.:adv.}
\end{itemize}
De modo insalubre.
\section{Insalubridade}
\begin{itemize}
\item {Grp. gram.:f.}
\end{itemize}
Qualidade daquillo que é insalubre.
\section{Insalutífero}
\begin{itemize}
\item {Grp. gram.:adj.}
\end{itemize}
\begin{itemize}
\item {Proveniência:(De \textunderscore in...\textunderscore  + \textunderscore salutífero\textunderscore )}
\end{itemize}
O mesmo que \textunderscore insalubre\textunderscore .
\section{Insanabilidade}
\begin{itemize}
\item {Grp. gram.:f.}
\end{itemize}
\begin{itemize}
\item {Proveniência:(Do lat. \textunderscore insanabilis\textunderscore )}
\end{itemize}
Qualidade daquelle que é insanável.
\section{Insanamente}
\begin{itemize}
\item {Grp. gram.:adv.}
\end{itemize}
De modo insano; loucamente; com insânia.
\section{Insanável}
\begin{itemize}
\item {Grp. gram.:adj.}
\end{itemize}
\begin{itemize}
\item {Utilização:Fig.}
\end{itemize}
\begin{itemize}
\item {Proveniência:(Lat. \textunderscore insanabilis\textunderscore )}
\end{itemize}
Que se não póde sanar; incurável.
Que não tem remédio; que se não póde supprir ou emendar: \textunderscore faltas insanáveis\textunderscore .
\section{Insanavelmente}
\begin{itemize}
\item {Grp. gram.:adv.}
\end{itemize}
De modo insanável.
\section{Insaneável}
\begin{itemize}
\item {Grp. gram.:adj.}
\end{itemize}
\begin{itemize}
\item {Proveniência:(De \textunderscore in...\textunderscore  + \textunderscore saneável\textunderscore )}
\end{itemize}
Que se não póde sanear. Cf. Camillo, \textunderscore Estrêll. Propicias\textunderscore , 178.
\section{Insânia}
\begin{itemize}
\item {Grp. gram.:f.}
\end{itemize}
\begin{itemize}
\item {Proveniência:(Lat. \textunderscore insania\textunderscore )}
\end{itemize}
Demência; loucura; destempêro.
Falta de siso.
\section{Insanidade}
\begin{itemize}
\item {Grp. gram.:f.}
\end{itemize}
\begin{itemize}
\item {Proveniência:(Lat. \textunderscore insanitas\textunderscore )}
\end{itemize}
Qualidade de insano.
Falta de senso.
Demência.
\section{Insano}
\begin{itemize}
\item {Grp. gram.:adj.}
\end{itemize}
\begin{itemize}
\item {Utilização:Fig.}
\end{itemize}
\begin{itemize}
\item {Proveniência:(Lat. \textunderscore insanus\textunderscore )}
\end{itemize}
Demente.
Tolo; insensato.
Excessivo; custoso: \textunderscore trabalho insano\textunderscore .
\section{Insaponificável}
\begin{itemize}
\item {Grp. gram.:adj.}
\end{itemize}
\begin{itemize}
\item {Proveniência:(De \textunderscore in...\textunderscore  + \textunderscore saponificável\textunderscore )}
\end{itemize}
Não saponificável.
\section{Insatisfeito}
\begin{itemize}
\item {Grp. gram.:adj.}
\end{itemize}
\begin{itemize}
\item {Proveniência:(De \textunderscore in...\textunderscore  + \textunderscore satisfeito\textunderscore )}
\end{itemize}
Que não está satisfeito.
\section{Insaturável}
\begin{itemize}
\item {Grp. gram.:adj.}
\end{itemize}
\begin{itemize}
\item {Proveniência:(Lat. \textunderscore insaturabilis\textunderscore )}
\end{itemize}
Que não é saturável.
O mesmo que \textunderscore insaciável\textunderscore .
\section{Insaturavelmente}
\begin{itemize}
\item {Grp. gram.:adv.}
\end{itemize}
De modo insaturável.
\section{Inscícia}
\begin{itemize}
\item {Grp. gram.:f.}
\end{itemize}
\begin{itemize}
\item {Utilização:P. us.}
\end{itemize}
\begin{itemize}
\item {Proveniência:(Lat. \textunderscore inscitia\textunderscore )}
\end{itemize}
Falta de saber; imperícia; ignorância.
\section{Insciência}
\begin{itemize}
\item {Grp. gram.:f.}
\end{itemize}
\begin{itemize}
\item {Utilização:Ext.}
\end{itemize}
\begin{itemize}
\item {Proveniência:(Lat. \textunderscore inscientia\textunderscore )}
\end{itemize}
Qualidade de insciente; falta de sciência.
Ineptidão.
\section{Insciente}
\begin{itemize}
\item {Grp. gram.:adj.}
\end{itemize}
\begin{itemize}
\item {Proveniência:(Lat. \textunderscore insciens\textunderscore )}
\end{itemize}
Não sciente; que não sabe.
Ignorante; inepto.
\section{Inscientemente}
\begin{itemize}
\item {Grp. gram.:adv.}
\end{itemize}
De modo insciente.
\section{Ínscio}
\begin{itemize}
\item {Grp. gram.:adj.}
\end{itemize}
\begin{itemize}
\item {Proveniência:(Lat. \textunderscore inscius\textunderscore )}
\end{itemize}
O mesmo que \textunderscore insciente\textunderscore .
\section{Inscrever}
\begin{itemize}
\item {Grp. gram.:v. t.}
\end{itemize}
\begin{itemize}
\item {Proveniência:(Lat. \textunderscore inscribere\textunderscore )}
\end{itemize}
Escrever em ou sôbre.
Registar; commemorar.
\section{Inscrição}
\begin{itemize}
\item {Grp. gram.:f.}
\end{itemize}
\begin{itemize}
\item {Proveniência:(Lat. \textunderscore inscriptio\textunderscore )}
\end{itemize}
Acto ou efeito de inscrever.
Legenda: \textunderscore inscrições coneiformes\textunderscore .
Título da dívida pública.
\section{Inscripção}
\begin{itemize}
\item {Grp. gram.:f.}
\end{itemize}
\begin{itemize}
\item {Proveniência:(Lat. \textunderscore inscriptio\textunderscore )}
\end{itemize}
Acto ou effeito de inscrever.
Legenda: \textunderscore inscripções coneiformes\textunderscore .
Título da dívida pública.
\section{Inscriptível}
\begin{itemize}
\item {Grp. gram.:adj.}
\end{itemize}
\begin{itemize}
\item {Proveniência:(De \textunderscore inscripto\textunderscore )}
\end{itemize}
Que se póde inscrever.
\section{Inscripto}
\begin{itemize}
\item {Grp. gram.:adj.}
\end{itemize}
\begin{itemize}
\item {Utilização:Mathem.}
\end{itemize}
\begin{itemize}
\item {Grp. gram.:M.}
\end{itemize}
\begin{itemize}
\item {Utilização:T. de Lisbôa}
\end{itemize}
\begin{itemize}
\item {Proveniência:(Lat. \textunderscore inscriptus\textunderscore )}
\end{itemize}
Diz-se das figuras descritas dentro de outras.
Indivíduo, que trabalha na carga e descarga dos navios.
\section{Inscritível}
\begin{itemize}
\item {Grp. gram.:adj.}
\end{itemize}
\begin{itemize}
\item {Proveniência:(De \textunderscore inscrito\textunderscore )}
\end{itemize}
Que se póde inscrever.
\section{Inscrito}
\begin{itemize}
\item {Grp. gram.:adj.}
\end{itemize}
\begin{itemize}
\item {Utilização:Mathem.}
\end{itemize}
\begin{itemize}
\item {Grp. gram.:M.}
\end{itemize}
\begin{itemize}
\item {Utilização:T. de Lisbôa}
\end{itemize}
\begin{itemize}
\item {Proveniência:(Lat. \textunderscore inscriptus\textunderscore )}
\end{itemize}
Diz-se das figuras descritas dentro de outras.
Indivíduo, que trabalha na carga e descarga dos navios.
\section{Insculpir}
\begin{itemize}
\item {Grp. gram.:v. t.}
\end{itemize}
\begin{itemize}
\item {Proveniência:(Lat. \textunderscore insculpere\textunderscore )}
\end{itemize}
Esculpir em; gravar; inscrever.
\section{Insculptor}
\begin{itemize}
\item {Grp. gram.:m.}
\end{itemize}
\begin{itemize}
\item {Proveniência:(Do lat. \textunderscore insculptus\textunderscore )}
\end{itemize}
Aquelle que insculpe.
\section{Insculptura}
\begin{itemize}
\item {Grp. gram.:f.}
\end{itemize}
\begin{itemize}
\item {Proveniência:(Do lat. \textunderscore insculptus\textunderscore )}
\end{itemize}
Arte ou trabalho de insculptor.
\section{Inscultor}
\begin{itemize}
\item {Grp. gram.:m.}
\end{itemize}
\begin{itemize}
\item {Proveniência:(Do lat. \textunderscore insculptus\textunderscore )}
\end{itemize}
Aquele que insculpe.
\section{Inscultura}
\begin{itemize}
\item {Grp. gram.:f.}
\end{itemize}
\begin{itemize}
\item {Proveniência:(Do lat. \textunderscore insculptus\textunderscore )}
\end{itemize}
Arte ou trabalho de insculptor.
\section{Insecável}
\begin{itemize}
\item {Grp. gram.:adj.}
\end{itemize}
\begin{itemize}
\item {Proveniência:(De \textunderscore in...\textunderscore  + \textunderscore secar\textunderscore )}
\end{itemize}
Que não póde secar; que se não esgota.
\section{Insecável}
\begin{itemize}
\item {Grp. gram.:adj.}
\end{itemize}
\begin{itemize}
\item {Proveniência:(Do lat. \textunderscore in\textunderscore  + \textunderscore secare\textunderscore )}
\end{itemize}
Que se não póde cortar; que é indivisível. Cf. Latino, \textunderscore Or. da Corôa\textunderscore , LXXVIII.
\section{Insecticida}
\begin{itemize}
\item {Grp. gram.:m.  e  adj.}
\end{itemize}
\begin{itemize}
\item {Proveniência:(Do lat. \textunderscore insectum\textunderscore  + \textunderscore caedere\textunderscore )}
\end{itemize}
Aquillo que destrói insectos.
\section{Insecticídio}
\begin{itemize}
\item {Grp. gram.:m.}
\end{itemize}
\begin{itemize}
\item {Proveniência:(Do lat. \textunderscore insectum\textunderscore  + \textunderscore caedere\textunderscore )}
\end{itemize}
Morte, que se dá a um insecto.
\section{Insectífero}
\begin{itemize}
\item {Grp. gram.:adj.}
\end{itemize}
\begin{itemize}
\item {Proveniência:(Do lat. \textunderscore insectum\textunderscore  + \textunderscore ferre\textunderscore )}
\end{itemize}
Que produz ou tem insectos.
\section{Insectífugo}
\begin{itemize}
\item {Grp. gram.:adj.}
\end{itemize}
\begin{itemize}
\item {Proveniência:(Do lat. \textunderscore insectum\textunderscore  + \textunderscore fugere\textunderscore )}
\end{itemize}
Que afugenta os insectos.
\section{Inséctil}
\begin{itemize}
\item {Grp. gram.:adj.}
\end{itemize}
\begin{itemize}
\item {Proveniência:(Do rad. do lat. \textunderscore insectum\textunderscore )}
\end{itemize}
Não dividido; que se não divide.
\section{Insectírodo}
\begin{itemize}
\item {fónica:ro}
\end{itemize}
\begin{itemize}
\item {Grp. gram.:adj.}
\end{itemize}
\begin{itemize}
\item {Grp. gram.:M. pl.}
\end{itemize}
\begin{itemize}
\item {Proveniência:(Do lat. \textunderscore insectum\textunderscore  + \textunderscore rodere\textunderscore )}
\end{itemize}
Que rói insectos; insectívoro.
Família de insectos hymenópteros, que se desenvolvem dentro de outros insectos.
\section{Insectírrodo}
\begin{itemize}
\item {Grp. gram.:adj.}
\end{itemize}
\begin{itemize}
\item {Grp. gram.:M. pl.}
\end{itemize}
\begin{itemize}
\item {Proveniência:(Do lat. \textunderscore insectum\textunderscore  + \textunderscore rodere\textunderscore )}
\end{itemize}
Que rói insectos; insectívoro.
Família de insectos himenópteros, que se desenvolvem dentro de outros insectos.
\section{Insectívoro}
\begin{itemize}
\item {Grp. gram.:adj.}
\end{itemize}
\begin{itemize}
\item {Grp. gram.:M.}
\end{itemize}
\begin{itemize}
\item {Proveniência:(Do lat. \textunderscore insectum\textunderscore  + \textunderscore vorare\textunderscore )}
\end{itemize}
Que come insectos; que se alimenta de insectos.
Animal que come insectos ou delles se alimenta.
\section{Insecto}
\begin{itemize}
\item {Grp. gram.:m.}
\end{itemize}
\begin{itemize}
\item {Utilização:Fig.}
\end{itemize}
\begin{itemize}
\item {Proveniência:(Lat. \textunderscore insectum\textunderscore )}
\end{itemize}
Pequeno animal invertebrado, cujo corpo é dividido em anéis.
Classe do reino animal, a qual comprehende os animaes articulados que têm seis pés.
Pessôa insignificante, miserável.
\section{Insectófilo}
\begin{itemize}
\item {Grp. gram.:adj.}
\end{itemize}
\begin{itemize}
\item {Proveniência:(De \textunderscore insecto\textunderscore  + gr. \textunderscore philos\textunderscore )}
\end{itemize}
Que ama a insectologia.
\section{Insectologia}
\begin{itemize}
\item {Grp. gram.:f.}
\end{itemize}
\begin{itemize}
\item {Proveniência:(De \textunderscore insecto\textunderscore  + gr. \textunderscore logos\textunderscore )}
\end{itemize}
O mesmo que \textunderscore entomologia\textunderscore .
\section{Insectológico}
\begin{itemize}
\item {Grp. gram.:adj.}
\end{itemize}
Relativo á insectologia.
\section{Insectologista}
\begin{itemize}
\item {Grp. gram.:m.}
\end{itemize}
Aquelle que é versado em \textunderscore insectologia\textunderscore .
\section{Insectóphilo}
\begin{itemize}
\item {Grp. gram.:adj.}
\end{itemize}
\begin{itemize}
\item {Proveniência:(De \textunderscore insecto\textunderscore  + gr. \textunderscore philos\textunderscore )}
\end{itemize}
Que ama a insectologia.
\section{Inseduzível}
\begin{itemize}
\item {Grp. gram.:adj.}
\end{itemize}
\begin{itemize}
\item {Utilização:Ext.}
\end{itemize}
\begin{itemize}
\item {Proveniência:(De in + \textunderscore seduzivel\textunderscore )}
\end{itemize}
Não seduzivel; que se não deixa seduzir.
Incorruptível.
\section{Insegurança}
\begin{itemize}
\item {Grp. gram.:f.}
\end{itemize}
\begin{itemize}
\item {Proveniência:(De \textunderscore in...\textunderscore  + \textunderscore segurança\textunderscore )}
\end{itemize}
Falta de segurança.
Qualidade de inseguro.
\section{Inseguridade}
\begin{itemize}
\item {Grp. gram.:f.}
\end{itemize}
\begin{itemize}
\item {Proveniência:(De \textunderscore in...\textunderscore  + \textunderscore seguridade\textunderscore )}
\end{itemize}
Falta de segurança.
\section{Inseguro}
\begin{itemize}
\item {Grp. gram.:adj.}
\end{itemize}
Que não é seguro. Cf. Latino, \textunderscore Humboldt\textunderscore , 173.
\section{Inseminação}
\begin{itemize}
\item {Grp. gram.:f.}
\end{itemize}
\begin{itemize}
\item {Proveniência:(Do lat. \textunderscore inseminare\textunderscore )}
\end{itemize}
Antiga prática supersticiosa, que consistia em revolver a terra e lançar nella qualquer coisa tirada de um lugar onde havia doença, e semear alli uma planta, que serviria para a cura da mesma doença.
\section{Insensatez}
\begin{itemize}
\item {Grp. gram.:f.}
\end{itemize}
Qualidade daquelle ou daquillo que é insensato.
Falta de sensatez; expressão ou acção insensata.
\section{Insensato}
\begin{itemize}
\item {Grp. gram.:adj.}
\end{itemize}
\begin{itemize}
\item {Proveniência:(Lat. \textunderscore insensatus\textunderscore )}
\end{itemize}
Não sensato; que não tem senso: \textunderscore homem insensato\textunderscore .
Que revela falta de senso ou de juizo; contrário á razão ou ao bom senso: \textunderscore palavras insensatas\textunderscore .
\section{Insensibilidade}
\begin{itemize}
\item {Grp. gram.:f.}
\end{itemize}
\begin{itemize}
\item {Proveniência:(Lat. \textunderscore insensibilitas\textunderscore )}
\end{itemize}
Qualidade daquelle ou daquillo que é insensível; falta de sensibilidade.
\section{Insensibilizar}
\begin{itemize}
\item {Grp. gram.:v. t.}
\end{itemize}
Tornar insensivel. Cf. Camillo, \textunderscore Myst. de Lisb.\textunderscore , I, 192.
\section{Insensitivo}
\begin{itemize}
\item {Grp. gram.:adj.}
\end{itemize}
\begin{itemize}
\item {Utilização:P. us.}
\end{itemize}
Que não é sensitivo.
\section{Insensível}
\begin{itemize}
\item {Grp. gram.:adj.}
\end{itemize}
\begin{itemize}
\item {Proveniência:(Lat. \textunderscore insensibilis\textunderscore )}
\end{itemize}
Que não é sensível; que não tem sensibilidade.
Indifferente; impassível.
Que se não póde observar pelos sentidos.
\section{Insensivelmente}
\begin{itemize}
\item {Grp. gram.:adv.}
\end{itemize}
De modo insensível.
\section{Inseparabilidade}
\begin{itemize}
\item {Grp. gram.:f.}
\end{itemize}
\begin{itemize}
\item {Proveniência:(Lat. \textunderscore inseparabilitas\textunderscore )}
\end{itemize}
Qualidade daquelle ou daquillo que é inseparável.
\section{Inseparável}
\begin{itemize}
\item {Grp. gram.:adj.}
\end{itemize}
\begin{itemize}
\item {Proveniência:(Lat. \textunderscore inseparabilis\textunderscore )}
\end{itemize}
Não separável; que se não separa.
Que anda, está ou existe sempre juntamente com outro ou outrem: \textunderscore amigos inseparáveis\textunderscore .
\section{Inseparavelmente}
\begin{itemize}
\item {Grp. gram.:adv.}
\end{itemize}
De modo inseparável.
\section{Insepulto}
\begin{itemize}
\item {Grp. gram.:adj.}
\end{itemize}
\begin{itemize}
\item {Proveniência:(Lat. \textunderscore insepultus\textunderscore )}
\end{itemize}
Não sepulto.
\section{Inserção}
\begin{itemize}
\item {Grp. gram.:f.}
\end{itemize}
\begin{itemize}
\item {Proveniência:(Lat. \textunderscore insertio\textunderscore )}
\end{itemize}
Acto ou effeito de inserir.
\section{Inserir}
\begin{itemize}
\item {Grp. gram.:v. t.}
\end{itemize}
\begin{itemize}
\item {Proveniência:(Lat. \textunderscore inserere\textunderscore )}
\end{itemize}
Introduzir; cravar: \textunderscore inserir uma estaca\textunderscore .
Gravar; fixar.
Intercalar.
Inscrever.
Registar.
Estampar, entre outras coisas: \textunderscore inserir um artigo no jornal\textunderscore .
\section{Insertação}
\begin{itemize}
\item {Grp. gram.:f.}
\end{itemize}
Acto de inserir.
(Lat. \textunderscore insertatio\textunderscore ).
\section{Insertar}
\textunderscore v. t.\textunderscore  (e der.) \textunderscore Ant.\textunderscore 
O mesmo que \textunderscore enxertar\textunderscore , etc.
\section{Inserto}
\begin{itemize}
\item {Grp. gram.:adj.}
\end{itemize}
\begin{itemize}
\item {Proveniência:(Lat. \textunderscore insertus\textunderscore )}
\end{itemize}
Que se inseriu.
Publicado, entre outras coisas: \textunderscore inserto num jornal\textunderscore .
\section{Inserve}
\begin{itemize}
\item {Grp. gram.:adj.}
\end{itemize}
\begin{itemize}
\item {Utilização:Prov.}
\end{itemize}
\begin{itemize}
\item {Utilização:minh.}
\end{itemize}
Que não tem mistura.
\section{Inservível}
\begin{itemize}
\item {Grp. gram.:adj.}
\end{itemize}
\begin{itemize}
\item {Utilização:Neol.}
\end{itemize}
\begin{itemize}
\item {Proveniência:(De \textunderscore in...\textunderscore  + \textunderscore servivel\textunderscore )}
\end{itemize}
Que não serve, que não presta utilidade ou serviço.
\section{Insessão}
\begin{itemize}
\item {Grp. gram.:f.}
\end{itemize}
\begin{itemize}
\item {Utilização:Med.}
\end{itemize}
\begin{itemize}
\item {Utilização:ant.}
\end{itemize}
\begin{itemize}
\item {Proveniência:(Do lat. \textunderscore insessus\textunderscore )}
\end{itemize}
Meio banho.
\section{Insexual}
\begin{itemize}
\item {fónica:csu}
\end{itemize}
\begin{itemize}
\item {Grp. gram.:adj.}
\end{itemize}
\begin{itemize}
\item {Proveniência:(De \textunderscore in...\textunderscore  + \textunderscore sexual\textunderscore )}
\end{itemize}
Avesso ás tendências naturaes dos sexos.
\section{Insexualidade}
\begin{itemize}
\item {fónica:csu}
\end{itemize}
\begin{itemize}
\item {Grp. gram.:f.}
\end{itemize}
Qualidade de insexual.
\section{Insídia}
\begin{itemize}
\item {Grp. gram.:f.}
\end{itemize}
\begin{itemize}
\item {Proveniência:(Lat. \textunderscore insidia\textunderscore )}
\end{itemize}
Emboscada; cilada.
Estratagema.
Perfídia.
\section{Insidiador}
\begin{itemize}
\item {Grp. gram.:m.  e  adj.}
\end{itemize}
\begin{itemize}
\item {Proveniência:(Lat. \textunderscore insidiator\textunderscore )}
\end{itemize}
O que insidia.
\section{Insidiar}
\begin{itemize}
\item {Grp. gram.:v. t.}
\end{itemize}
\begin{itemize}
\item {Proveniência:(Lat. \textunderscore insidiare\textunderscore )}
\end{itemize}
Armar insidias a; preparar ciladas a.
Procurar seduzir ou corromper.
\section{Insidiosamente}
\begin{itemize}
\item {Grp. gram.:adv.}
\end{itemize}
De modo insidioso; perfidamente; á traição.
\section{Insidioso}
\begin{itemize}
\item {Grp. gram.:adj.}
\end{itemize}
\begin{itemize}
\item {Proveniência:(Lat. \textunderscore insidiosus\textunderscore )}
\end{itemize}
Que tem o costume de armar insidias.
Fallaz; pérfido.
\section{Insigne}
\begin{itemize}
\item {Grp. gram.:adj.}
\end{itemize}
\begin{itemize}
\item {Proveniência:(Lat. \textunderscore insignis\textunderscore )}
\end{itemize}
Notável; eminente; famoso: \textunderscore insigne artista\textunderscore .
Extraordinário; incrível: \textunderscore um insigne disparate\textunderscore .
\section{Insignemente}
\begin{itemize}
\item {Grp. gram.:adv.}
\end{itemize}
De modo insigne.
\section{Insígnia}
\begin{itemize}
\item {Grp. gram.:f.}
\end{itemize}
\begin{itemize}
\item {Proveniência:(Do lat. \textunderscore insigne\textunderscore )}
\end{itemize}
Emblema, signal distintivo.
Venera.
Estandarte.
\section{Insignificância}
\begin{itemize}
\item {Grp. gram.:f.}
\end{itemize}
Qualidade daquelle ou daquillo que é insignificante.
Bagatela; ninharia.
\section{Insignificante}
\begin{itemize}
\item {Grp. gram.:adj.}
\end{itemize}
\begin{itemize}
\item {Proveniência:(De \textunderscore in...\textunderscore  + \textunderscore significante\textunderscore )}
\end{itemize}
Que nada significa.
Que não tem valor ou importância: \textunderscore quantia insignificante\textunderscore .
\section{Insignificativo}
\begin{itemize}
\item {Grp. gram.:adj.}
\end{itemize}
\begin{itemize}
\item {Proveniência:(Lat. \textunderscore insignificativus\textunderscore )}
\end{itemize}
Que não é significativo.
\section{Insignios}
\begin{itemize}
\item {Grp. gram.:m. pl.}
\end{itemize}
\begin{itemize}
\item {Utilização:Ant.}
\end{itemize}
\begin{itemize}
\item {Proveniência:(Do lat. \textunderscore in...\textunderscore  + \textunderscore signum\textunderscore )}
\end{itemize}
Signaes ou demonstração da posse que se dava judicialmente.
\section{Insimulação}
\begin{itemize}
\item {Grp. gram.:f.}
\end{itemize}
Acto ou effeito de insimular.
\section{Insimular}
\begin{itemize}
\item {Grp. gram.:v. t.}
\end{itemize}
\begin{itemize}
\item {Proveniência:(Lat. \textunderscore insimulare\textunderscore )}
\end{itemize}
Attribuir um crime a.
Denunciar; accusar falsamente.
\section{Insinuação}
\begin{itemize}
\item {Grp. gram.:f.}
\end{itemize}
\begin{itemize}
\item {Utilização:Des.}
\end{itemize}
\begin{itemize}
\item {Proveniência:(Lat. \textunderscore insinuatio\textunderscore )}
\end{itemize}
Acto ou effeito de insinuar.
Aquillo que se insinua ou se dá a perceber.
Censura ou accusação indirecta ou disfarçada.
Remoque; advertência, admoestação amigável ou branda.
Suggestão.
Lembrança.
Menção de circunstância ou cláusula em documento público.
Confirmação authêntica de uma doação.
\section{Insinuador}
\begin{itemize}
\item {Grp. gram.:m.  e  adj.}
\end{itemize}
\begin{itemize}
\item {Proveniência:(Lat. \textunderscore insinuator\textunderscore )}
\end{itemize}
O que insinua.
\section{Insinuante}
\begin{itemize}
\item {Grp. gram.:adj.}
\end{itemize}
\begin{itemize}
\item {Grp. gram.:M.}
\end{itemize}
\begin{itemize}
\item {Proveniência:(Lat. \textunderscore insinuans\textunderscore )}
\end{itemize}
Que insinua ou que se insinua.
Lhano; sympático.
Modo caricioso, attrahente. Cf. Rebello, \textunderscore Mocidade\textunderscore , III, 223.
\section{Insinuar}
\begin{itemize}
\item {Grp. gram.:v. t.}
\end{itemize}
\begin{itemize}
\item {Grp. gram.:V. p.}
\end{itemize}
\begin{itemize}
\item {Proveniência:(Lat. \textunderscore insinuare\textunderscore )}
\end{itemize}
Fazer entrar no seio, no coração.
Instillar no ânimo de outrem: \textunderscore insinuar uma resolução\textunderscore .
Admoestar, aconselhar.
Introduzir.
Ensinar.
Registar em escritura pública ou em notas de tabellião.
Introduzir-se no ânimo.
Penetrar nos interstícios.
Tornar-se sympáthico, granjear estima.
\section{Insinuativa}
\begin{itemize}
\item {Grp. gram.:f.}
\end{itemize}
\begin{itemize}
\item {Proveniência:(De \textunderscore insinuativo\textunderscore )}
\end{itemize}
Faculdade de se tornar insinuante.
\section{Insinuativo}
\begin{itemize}
\item {Grp. gram.:adj.}
\end{itemize}
O mesmo que \textunderscore insinuante\textunderscore .
\section{Insipidamente}
\begin{itemize}
\item {Grp. gram.:adv.}
\end{itemize}
De modo insípido.
\section{Insipidar}
\begin{itemize}
\item {Grp. gram.:v. t.}
\end{itemize}
\begin{itemize}
\item {Utilização:Neol.}
\end{itemize}
Tornar insípido. Cf. Eça, \textunderscore Fradique\textunderscore .
\section{Insipidez}
\begin{itemize}
\item {Grp. gram.:f.}
\end{itemize}
Qualidade daquelle ou daquillo que é insípido.
\section{Insípido}
\begin{itemize}
\item {Grp. gram.:adj.}
\end{itemize}
\begin{itemize}
\item {Utilização:Fig.}
\end{itemize}
\begin{itemize}
\item {Proveniência:(Lat. \textunderscore insipidus\textunderscore )}
\end{itemize}
Que não tem sabor.
Insulso.
Sem-sabor.
Que não tem graça; monótono: \textunderscore prosa insípida\textunderscore .
\section{Insipiência}
\begin{itemize}
\item {Grp. gram.:f.}
\end{itemize}
\begin{itemize}
\item {Proveniência:(Lat. \textunderscore insipientia\textunderscore )}
\end{itemize}
Qualidade de insipiente. Cf. \textunderscore Luz e Calor\textunderscore , 209.
\section{Insipiente}
\begin{itemize}
\item {Grp. gram.:adj.}
\end{itemize}
\begin{itemize}
\item {Proveniência:(Lat. \textunderscore insipiens\textunderscore )}
\end{itemize}
Não sapiente; ignorante.
Insensato.
\section{Insistência}
\begin{itemize}
\item {Grp. gram.:f.}
\end{itemize}
Acto de insistir; teimosia; contumácia; importunidade.
\section{Insistente}
\begin{itemize}
\item {Grp. gram.:adj.}
\end{itemize}
\begin{itemize}
\item {Proveniência:(Lat. \textunderscore insistens\textunderscore )}
\end{itemize}
Que insiste.
Teimoso; obstinado; importuno.
\section{Insistir}
\begin{itemize}
\item {Grp. gram.:v. i.}
\end{itemize}
\begin{itemize}
\item {Proveniência:(Lat. \textunderscore insistere\textunderscore )}
\end{itemize}
Manter-se firme.
Têr pertinácia, tenacidade; teimar; perseverar.
\section{Ínsito}
\begin{itemize}
\item {Grp. gram.:adj.}
\end{itemize}
\begin{itemize}
\item {Utilização:Fig.}
\end{itemize}
\begin{itemize}
\item {Proveniência:(Lat. \textunderscore insitus\textunderscore )}
\end{itemize}
Inserido.
Inherente; innato; congênito.
Gravado no espírito.
\section{Insobriedade}
\begin{itemize}
\item {Grp. gram.:f.}
\end{itemize}
\begin{itemize}
\item {Proveniência:(De \textunderscore in...\textunderscore  + \textunderscore sobriedade\textunderscore )}
\end{itemize}
Falta de sobriedade.
\section{Insóbrio}
\begin{itemize}
\item {Grp. gram.:adj.}
\end{itemize}
\begin{itemize}
\item {Proveniência:(De \textunderscore in...\textunderscore  + \textunderscore sóbrio\textunderscore )}
\end{itemize}
Que não é sóbrio.
\section{Insociabilidade}
\begin{itemize}
\item {Grp. gram.:f.}
\end{itemize}
\begin{itemize}
\item {Proveniência:(Do lat. \textunderscore insociabilis\textunderscore )}
\end{itemize}
Qualidade de insociável.
\section{Insocial}
\begin{itemize}
\item {Grp. gram.:adj.}
\end{itemize}
\begin{itemize}
\item {Proveniência:(Lat. \textunderscore insocialis\textunderscore )}
\end{itemize}
Que não é social; estranho á vida da sociedade.
\section{Insociável}
\begin{itemize}
\item {Grp. gram.:adj.}
\end{itemize}
\begin{itemize}
\item {Proveniência:(Lat. \textunderscore insociabilis\textunderscore )}
\end{itemize}
Que não é sociável; que não vive em sociedade.
Que não é tratável ou lhano; misanthropo.
\section{Insociavelmente}
\begin{itemize}
\item {Grp. gram.:adv.}
\end{itemize}
De modo insociável.
\section{Insoffreável}
\begin{itemize}
\item {Grp. gram.:adj.}
\end{itemize}
\begin{itemize}
\item {Proveniência:(De \textunderscore in...\textunderscore  + \textunderscore soffreável\textunderscore )}
\end{itemize}
Que se não póde soffrear.
\section{Insoffridamente}
\begin{itemize}
\item {Grp. gram.:adv.}
\end{itemize}
De modo insoffrido; com impaciência.
\section{Insoffrido}
\begin{itemize}
\item {Grp. gram.:adj.}
\end{itemize}
\begin{itemize}
\item {Proveniência:(De \textunderscore in...\textunderscore  + \textunderscore soffrido\textunderscore )}
\end{itemize}
Pouco soffredor; impaciente.
Inquieto; indomável: \textunderscore cavallo insoffrido\textunderscore .
\section{Insoffrimento}
\begin{itemize}
\item {Grp. gram.:m.}
\end{itemize}
\begin{itemize}
\item {Proveniência:(De \textunderscore in...\textunderscore  + \textunderscore soffrimento\textunderscore )}
\end{itemize}
Estado de quem é insoffrido.
\section{Insoffrível}
\begin{itemize}
\item {Grp. gram.:adj.}
\end{itemize}
\begin{itemize}
\item {Proveniência:(De \textunderscore in...\textunderscore  + \textunderscore soffrível\textunderscore )}
\end{itemize}
Que se não póde soffrer; intolerável.
\section{Insofreável}
\begin{itemize}
\item {Grp. gram.:adj.}
\end{itemize}
\begin{itemize}
\item {Proveniência:(De \textunderscore in...\textunderscore  + \textunderscore sofreável\textunderscore )}
\end{itemize}
Que se não póde sofrear.
\section{Insofridamente}
\begin{itemize}
\item {Grp. gram.:adv.}
\end{itemize}
De modo insofrido; com impaciência.
\section{Insofrido}
\begin{itemize}
\item {Grp. gram.:adj.}
\end{itemize}
\begin{itemize}
\item {Proveniência:(De \textunderscore in...\textunderscore  + \textunderscore sofrido\textunderscore )}
\end{itemize}
Pouco sofredor; impaciente.
Inquieto; indomável: \textunderscore cavalo insofrido\textunderscore .
\section{Insofrimento}
\begin{itemize}
\item {Grp. gram.:m.}
\end{itemize}
\begin{itemize}
\item {Proveniência:(De \textunderscore in...\textunderscore  + \textunderscore sofrimento\textunderscore )}
\end{itemize}
Estado de quem é insofrido.
\section{Insofrível}
\begin{itemize}
\item {Grp. gram.:adj.}
\end{itemize}
\begin{itemize}
\item {Proveniência:(De \textunderscore in...\textunderscore  + \textunderscore sofrível\textunderscore )}
\end{itemize}
Que se não póde sofrer; intolerável.
\section{Insolação}
\begin{itemize}
\item {Grp. gram.:f.}
\end{itemize}
\begin{itemize}
\item {Proveniência:(Lat. \textunderscore insolatio\textunderscore )}
\end{itemize}
Acto ou effeito de insolar.
Acção ou effeito do sol nos corpos orgânicos ou inorgânicos.
Resultado mórbido da exposição ao ardor do sol.
Acto de expôr ao sol, como meio therapêutico.
Desecação de substâncias medicamentosas, sob a acção do sol.
O calor, que o sol transmitte á terra.
\section{Insolar}
\begin{itemize}
\item {Grp. gram.:v. t.}
\end{itemize}
\begin{itemize}
\item {Proveniência:(Lat. \textunderscore insolare\textunderscore )}
\end{itemize}
Expor ou secar ao sol.
Tornar doente pela acção do sol.
\section{Insoldável}
\begin{itemize}
\item {Grp. gram.:adj.}
\end{itemize}
\begin{itemize}
\item {Proveniência:(De \textunderscore in...\textunderscore  + \textunderscore soldável\textunderscore )}
\end{itemize}
Que se não póde soldar.
\section{Insolência}
\begin{itemize}
\item {Grp. gram.:f.}
\end{itemize}
\begin{itemize}
\item {Proveniência:(Lat. \textunderscore insolentia\textunderscore )}
\end{itemize}
Qualidade daquelle ou daquillo que é insolente.
Procedimento insólito.
Orgulho desmedido.
Inconveniência grave.
Acto ou palavra insolente.
Palavra ou palavras injuriosas; aggravo verbal.
Má-criação; dito grosseiro.
\section{Insolente}
\begin{itemize}
\item {Grp. gram.:adj.}
\end{itemize}
\begin{itemize}
\item {Utilização:Fig.}
\end{itemize}
\begin{itemize}
\item {Proveniência:(Lat. \textunderscore insolens\textunderscore )}
\end{itemize}
O mesmo que insólito.
Atrevido; grosseiro.
Injurioso.
Malcriado.
Cruel.
\section{Insolentemente}
\begin{itemize}
\item {Grp. gram.:adv.}
\end{itemize}
De modo insolente.
\section{Insolidariedade}
\begin{itemize}
\item {Grp. gram.:f.}
\end{itemize}
\begin{itemize}
\item {Proveniência:(De \textunderscore in...\textunderscore  + \textunderscore solidariedade\textunderscore )}
\end{itemize}
Falta de solidariedade.
\section{Insolitamente}
\begin{itemize}
\item {fónica:só}
\end{itemize}
\begin{itemize}
\item {Grp. gram.:adv.}
\end{itemize}
De modo insólito; extraordinariamente.
\section{Insólito}
\begin{itemize}
\item {Grp. gram.:adj.}
\end{itemize}
\begin{itemize}
\item {Proveniência:(Lat. \textunderscore insolitus\textunderscore )}
\end{itemize}
Não habitual; extraordinário; incrível.
\section{Insolubilidade}
\begin{itemize}
\item {Grp. gram.:f.}
\end{itemize}
\begin{itemize}
\item {Proveniência:(Lat. \textunderscore insolubilitas\textunderscore )}
\end{itemize}
Qualidade daquillo que é insolúvel.
\section{Insolubilizar}
\begin{itemize}
\item {Grp. gram.:v. t.}
\end{itemize}
Tornar insolúvel. Cf. \textunderscore Techn. Rur.\textunderscore , 24 e 128.
\section{Insolúvel}
\begin{itemize}
\item {Grp. gram.:adj.}
\end{itemize}
\begin{itemize}
\item {Utilização:Fig.}
\end{itemize}
\begin{itemize}
\item {Proveniência:(Lat. \textunderscore insolubilis\textunderscore )}
\end{itemize}
Que não é solúvel; que se não dissolve.
Que se não desata: \textunderscore nó insolúvel\textunderscore .
Que se não póde resolver.
Que se não póde annular: \textunderscore contrato insolúvel\textunderscore .
Que se não póde pagar ou cobrar, (falando-se de dívidas).
\section{Insoluvelmente}
\begin{itemize}
\item {Grp. gram.:adv.}
\end{itemize}
De modo insolúvel.
\section{Insolvência}
\begin{itemize}
\item {Grp. gram.:f.}
\end{itemize}
Qualidade de insolvente.
\section{Insolvente}
\begin{itemize}
\item {Grp. gram.:m., f.  e  adj.}
\end{itemize}
\begin{itemize}
\item {Proveniência:(Do lat. \textunderscore in...\textunderscore  + \textunderscore solvens\textunderscore )}
\end{itemize}
Pessôa que não tem meios para pagar o que deve.
\section{Insolvível}
\begin{itemize}
\item {Grp. gram.:adj.}
\end{itemize}
\begin{itemize}
\item {Proveniência:(De \textunderscore in...\textunderscore  + \textunderscore solvível\textunderscore )}
\end{itemize}
Que não póde sêr pago: \textunderscore dívida insolvível\textunderscore .
\section{Insombrio}
\begin{itemize}
\item {Grp. gram.:adj.}
\end{itemize}
\begin{itemize}
\item {Utilização:bras}
\end{itemize}
\begin{itemize}
\item {Utilização:Neol.}
\end{itemize}
Não sombrio.
\section{Insomne}
\begin{itemize}
\item {Grp. gram.:adj.}
\end{itemize}
\begin{itemize}
\item {Utilização:Poét.}
\end{itemize}
\begin{itemize}
\item {Proveniência:(Lat. \textunderscore insomnis\textunderscore )}
\end{itemize}
Que tem insómnias; que passa a noite sem poder dormir.
\section{Insómnia}
\begin{itemize}
\item {Grp. gram.:f.}
\end{itemize}
\begin{itemize}
\item {Proveniência:(Lat. \textunderscore insomnia\textunderscore )}
\end{itemize}
Privação de somno; vigília; difficuldade em dormir.
\section{Insomnioso}
\begin{itemize}
\item {Grp. gram.:adj.}
\end{itemize}
Que tem insómnia.
Que é sujeito a insómnias.
\section{Insomnolência}
\begin{itemize}
\item {Grp. gram.:f.}
\end{itemize}
\begin{itemize}
\item {Proveniência:(De \textunderscore in...\textunderscore  + \textunderscore somnolência\textunderscore )}
\end{itemize}
O mesmo que \textunderscore insómnia\textunderscore .
\section{Insondabilidade}
\begin{itemize}
\item {Grp. gram.:f.}
\end{itemize}
Qualidade de insondável.
\section{Insondado}
\begin{itemize}
\item {Grp. gram.:adj.}
\end{itemize}
\begin{itemize}
\item {Utilização:Fig.}
\end{itemize}
\begin{itemize}
\item {Proveniência:(De \textunderscore in...\textunderscore  + \textunderscore sondado\textunderscore )}
\end{itemize}
Não sondado.
Ainda não estudado; desconhecido.
\section{Insondável}
\begin{itemize}
\item {Grp. gram.:adj.}
\end{itemize}
\begin{itemize}
\item {Utilização:Fig.}
\end{itemize}
\begin{itemize}
\item {Proveniência:(De \textunderscore in...\textunderscore  + \textunderscore sondável\textunderscore )}
\end{itemize}
Que não é sondável.
Inexplicável; mysterioso.
\section{Insónia}
\begin{itemize}
\item {Grp. gram.:f.}
\end{itemize}
\begin{itemize}
\item {Proveniência:(Lat. \textunderscore insomnia\textunderscore )}
\end{itemize}
Privação de sono; vigília; dificuldade em dormir.
\section{Insonioso}
\begin{itemize}
\item {Grp. gram.:adj.}
\end{itemize}
Que tem insónia.
Que é sujeito a insónias.
\section{Insonolência}
\begin{itemize}
\item {Grp. gram.:f.}
\end{itemize}
\begin{itemize}
\item {Proveniência:(De \textunderscore in...\textunderscore  + \textunderscore sonolência\textunderscore )}
\end{itemize}
O mesmo que \textunderscore insónia\textunderscore .
\section{Insonoridade}
\begin{itemize}
\item {Grp. gram.:f.}
\end{itemize}
\begin{itemize}
\item {Proveniência:(De \textunderscore in...\textunderscore  + \textunderscore sonoridade\textunderscore )}
\end{itemize}
Falta de sonoridade.
\section{Insonoro}
\begin{itemize}
\item {Grp. gram.:adj.}
\end{itemize}
\begin{itemize}
\item {Proveniência:(De \textunderscore in...\textunderscore  + \textunderscore sonoro\textunderscore )}
\end{itemize}
Não sonoro; desharmonioso.
\section{Insonte}
\begin{itemize}
\item {Grp. gram.:adj.}
\end{itemize}
\begin{itemize}
\item {Utilização:Poét.}
\end{itemize}
\begin{itemize}
\item {Proveniência:(Lat. \textunderscore insons\textunderscore )}
\end{itemize}
Innócuo; innocente; que não tem culpas.
\section{Insophismável}
\begin{itemize}
\item {Grp. gram.:adj.}
\end{itemize}
\begin{itemize}
\item {Proveniência:(De \textunderscore in...\textunderscore  + \textunderscore sophismável\textunderscore )}
\end{itemize}
Que se não póde sophismar. Cf. Castilho, \textunderscore Tosquia\textunderscore .
\section{Insopitável}
\begin{itemize}
\item {Grp. gram.:adj.}
\end{itemize}
\begin{itemize}
\item {Proveniência:(De \textunderscore in...\textunderscore  + \textunderscore sopitável\textunderscore )}
\end{itemize}
Que se não póde sopitar ou adormentar.
\section{Insossar}
\begin{itemize}
\item {Grp. gram.:v. t.}
\end{itemize}
Tornar insôsso.
\section{Insossêgo}
\begin{itemize}
\item {Grp. gram.:m.}
\end{itemize}
O mesmo que \textunderscore desassossêgo\textunderscore .
\section{Insôsso}
\begin{itemize}
\item {Grp. gram.:m.}
\end{itemize}
\begin{itemize}
\item {Proveniência:(Do lat. \textunderscore insulsus\textunderscore )}
\end{itemize}
Que não tem sal; que tem pouco sal.
Insulso.
\section{Insôsso}
\begin{itemize}
\item {Grp. gram.:adj.}
\end{itemize}
O mesmo que \textunderscore sosso\textunderscore .
\section{Inspecção}
\begin{itemize}
\item {Grp. gram.:f.}
\end{itemize}
\begin{itemize}
\item {Proveniência:(Lat. \textunderscore inspectio\textunderscore )}
\end{itemize}
Acto de vêr.
Lance de olhos.
Exame: \textunderscore inspecção de recrutas\textunderscore .
Superintendência.
Cargo de inspector.
Repartição ou collectividade, encarregada de inspeccionar: \textunderscore dirigiu-se á inspecção do sêllo\textunderscore .
\section{Inspeccionador}
\begin{itemize}
\item {Grp. gram.:m.}
\end{itemize}
Instrumento, destinado a inspeccionar o estado do vinho, dentro da vasilha.
\section{Inspeccionamento}
\begin{itemize}
\item {Grp. gram.:m.}
\end{itemize}
Acto de inspeccionar.
\section{Inspeccionar}
\begin{itemize}
\item {Grp. gram.:v. t.}
\end{itemize}
\begin{itemize}
\item {Proveniência:(Do lat. \textunderscore inspectio\textunderscore )}
\end{itemize}
Examinar; fazer inspecção a; vistorizar.
\section{Inspectar}
\begin{itemize}
\item {Grp. gram.:v. t.}
\end{itemize}
\begin{itemize}
\item {Proveniência:(Lat. \textunderscore inspectare\textunderscore )}
\end{itemize}
Inspeccionar miudamente; inspeccionar.
\section{Inspector}
\begin{itemize}
\item {Grp. gram.:adj.}
\end{itemize}
\begin{itemize}
\item {Grp. gram.:M.}
\end{itemize}
\begin{itemize}
\item {Proveniência:(Lat. \textunderscore inspector\textunderscore )}
\end{itemize}
Que vê, que observa, que fiscaliza ou inspecciona.
Aquelle que, por dever official, observa e inspecciona serviços públicos, dando ás autoridades, de que depende, informações desses serviços: \textunderscore inspector de escolas\textunderscore .
\section{Inspectoria}
\begin{itemize}
\item {Grp. gram.:f.}
\end{itemize}
\begin{itemize}
\item {Utilização:Bras}
\end{itemize}
\begin{itemize}
\item {Utilização:Neol.}
\end{itemize}
Cargo ou dignidade de inspector.
\section{Inspiração}
\begin{itemize}
\item {Grp. gram.:f.}
\end{itemize}
\begin{itemize}
\item {Proveniência:(Lat. \textunderscore inspiratio\textunderscore )}
\end{itemize}
Acto ou effeito de inspirar ou de sêr inspirado.
Movimentos da alma, actos ou pensamentos, devidos a insuflação divina, comparável á insuflação que introduz o ar nos pulmões.
Enthusiasmo, que domina os poétas, os músicos, os pintores.
Suggestão.
Coisa inspirada.
Coisa ou pessôa, que inspira.
Aquillo que numa composição artística revela grande talento ou gênio.
\section{Inspirador}
\begin{itemize}
\item {Grp. gram.:adj.}
\end{itemize}
\begin{itemize}
\item {Grp. gram.:M.}
\end{itemize}
\begin{itemize}
\item {Proveniência:(Lat. \textunderscore inspirator\textunderscore )}
\end{itemize}
Que inspira.
Que suggere; que enthusiasma.
Aquelle que inspira.
\section{Inspirar}
\begin{itemize}
\item {Grp. gram.:v. t.}
\end{itemize}
\begin{itemize}
\item {Utilização:Fig.}
\end{itemize}
\begin{itemize}
\item {Proveniência:(Lat. \textunderscore inspirare\textunderscore )}
\end{itemize}
Introduzir o ar em (os pulmões).
Causar inspiração a; suggerir: \textunderscore inspirar audácia\textunderscore .
\section{Inspirativo}
\begin{itemize}
\item {Grp. gram.:adj.}
\end{itemize}
\begin{itemize}
\item {Proveniência:(Do lat. \textunderscore inspiratus\textunderscore )}
\end{itemize}
Que inspira.
\section{Inspiratório}
\begin{itemize}
\item {Grp. gram.:adj.}
\end{itemize}
Próprio para inspirar; que leva o ar aos pulmões.
\section{Inspissação}
\begin{itemize}
\item {Grp. gram.:f.}
\end{itemize}
Acto ou effeito de inspissar.
\section{Inspissamento}
\begin{itemize}
\item {Grp. gram.:m.}
\end{itemize}
Acto ou effeito de inspissar.
\section{Inspissar}
\begin{itemize}
\item {Grp. gram.:v. t.}
\end{itemize}
\begin{itemize}
\item {Proveniência:(Do lat. \textunderscore inspisare\textunderscore )}
\end{itemize}
Tornar espêsso.
\section{Instabilidade}
\begin{itemize}
\item {Grp. gram.:f.}
\end{itemize}
\begin{itemize}
\item {Proveniência:(Lat. \textunderscore instabilitas\textunderscore )}
\end{itemize}
Qualidade de instável; falta de estabilidade.
\section{Instalação}
\begin{itemize}
\item {Grp. gram.:f.}
\end{itemize}
Acto ou efeito de instalar.
\section{Instalador}
\begin{itemize}
\item {Grp. gram.:m.  e  adj.}
\end{itemize}
O que instala.
\section{Instalar}
\begin{itemize}
\item {Grp. gram.:v. t.}
\end{itemize}
\begin{itemize}
\item {Proveniência:(Do b. lat. \textunderscore stallum\textunderscore )}
\end{itemize}
Estabelecer.
Inaugurar.
Alojar.
Dar hospedagem a.
Organizar o domicílio de.
Dar posse de um cargo a.
\section{Installação}
\begin{itemize}
\item {Grp. gram.:f.}
\end{itemize}
Acto ou effeito de installar.
\section{Installador}
\begin{itemize}
\item {Grp. gram.:m.  e  adj.}
\end{itemize}
O que installa.
\section{Installar}
\begin{itemize}
\item {Grp. gram.:v. t.}
\end{itemize}
\begin{itemize}
\item {Proveniência:(Do b. lat. \textunderscore stallum\textunderscore )}
\end{itemize}
Estabelecer.
Inaugurar.
Alojar.
Dar hospedagem a.
Organizar o domicílio de.
Dar posse de um cargo a.
\section{Instaminado}
\begin{itemize}
\item {Grp. gram.:adj.}
\end{itemize}
\begin{itemize}
\item {Utilização:Bot.}
\end{itemize}
\begin{itemize}
\item {Proveniência:(Do lat. \textunderscore in...\textunderscore  + \textunderscore stamen\textunderscore )}
\end{itemize}
Que não tem estames.
\section{Instância}
\begin{itemize}
\item {Grp. gram.:f.}
\end{itemize}
\begin{itemize}
\item {Proveniência:(Lat. \textunderscore instantia\textunderscore )}
\end{itemize}
Acto ou effeito de instar.
Qualidade daquillo que é instante.
Pedido urgente e repetido.
Perseverança.
Fôro, jurisdicção: \textunderscore tribunaes de primeira instância\textunderscore .
\section{Instantaneamente}
\begin{itemize}
\item {Grp. gram.:adv.}
\end{itemize}
De modo instantâneo.
\section{Instataneidade}
\begin{itemize}
\item {Grp. gram.:f.}
\end{itemize}
Qualidade de instantâneo.
\section{Instantâneo}
\begin{itemize}
\item {Grp. gram.:adj.}
\end{itemize}
Que succede num instante; momentâneo.
Rápido; súbito.
\section{Instante}
\begin{itemize}
\item {Grp. gram.:adj.}
\end{itemize}
\begin{itemize}
\item {Grp. gram.:M.}
\end{itemize}
\begin{itemize}
\item {Proveniência:(Lat. \textunderscore instans\textunderscore )}
\end{itemize}
Que está imminente.
Vehemente.
Em que há empenho: \textunderscore pedidos instantes\textunderscore .
Pertinácia ou insistência.
Espaço de um segundo.
Momento.
Occasião: \textunderscore chegou naquelle instante\textunderscore .
Pequena duração: \textunderscore demorou-se apenas um instante\textunderscore .
\section{Instantemente}
\begin{itemize}
\item {Grp. gram.:adv.}
\end{itemize}
\begin{itemize}
\item {Proveniência:(De \textunderscore instante\textunderscore )}
\end{itemize}
Com instância; com encarecimento; com urgência.
\section{Instar}
\begin{itemize}
\item {Grp. gram.:v. i.}
\end{itemize}
\begin{itemize}
\item {Grp. gram.:V. t.}
\end{itemize}
\begin{itemize}
\item {Proveniência:(Lat. \textunderscore instare\textunderscore )}
\end{itemize}
Estar imminente.
Pedir encarecidamente.
Sêr necessário.
Dirigir solicitações reiteradas a.
\section{Instauração}
\begin{itemize}
\item {Grp. gram.:f.}
\end{itemize}
\begin{itemize}
\item {Proveniência:(Lat. \textunderscore instauratio\textunderscore )}
\end{itemize}
Acto ou effeito de instaurar.
\section{Instaurador}
\begin{itemize}
\item {Grp. gram.:m.  e  adj.}
\end{itemize}
\begin{itemize}
\item {Proveniência:(Lat. \textunderscore instaurator\textunderscore )}
\end{itemize}
O que instaura.
\section{Instaurar}
\begin{itemize}
\item {Grp. gram.:v. t.}
\end{itemize}
\begin{itemize}
\item {Proveniência:(Lat. \textunderscore instaurare\textunderscore )}
\end{itemize}
Renovar; restaurar.
Inaugurar.
Fundar.
Formar: \textunderscore instaurar um processo criminal\textunderscore .
\section{Instaurativo}
\begin{itemize}
\item {Grp. gram.:adj.}
\end{itemize}
\begin{itemize}
\item {Utilização:Des.}
\end{itemize}
\begin{itemize}
\item {Proveniência:(Lat. \textunderscore instaurativus\textunderscore )}
\end{itemize}
Que envolve instauração.
Próprio para instaurar.
\section{Instável}
\begin{itemize}
\item {Grp. gram.:adj.}
\end{itemize}
\begin{itemize}
\item {Proveniência:(Lat. \textunderscore instabilis\textunderscore )}
\end{itemize}
Não estável.
Movediço.
Inconstante.
Que desapparece facilmente.
\section{Instavelmente}
\begin{itemize}
\item {Grp. gram.:adv.}
\end{itemize}
De modo instável.
\section{Instigação}
\begin{itemize}
\item {Grp. gram.:f.}
\end{itemize}
\begin{itemize}
\item {Proveniência:(Lat. \textunderscore instigatio\textunderscore )}
\end{itemize}
Acto ou effeito de instigar.
Suggestão.
Estímulo; incitamento.
\section{Instigador}
\begin{itemize}
\item {Grp. gram.:adj.}
\end{itemize}
\begin{itemize}
\item {Grp. gram.:M.}
\end{itemize}
\begin{itemize}
\item {Proveniência:(Lat. \textunderscore instigator\textunderscore )}
\end{itemize}
Que instiga.
Aquelle que instiga.
\section{Instigar}
\begin{itemize}
\item {Grp. gram.:v. t.}
\end{itemize}
\begin{itemize}
\item {Proveniência:(Lat. \textunderscore instigare\textunderscore )}
\end{itemize}
Estimular.
Impellir.
Incitar.
Açular.
Induzir.
\section{Instilação}
\begin{itemize}
\item {Grp. gram.:f.}
\end{itemize}
\begin{itemize}
\item {Proveniência:(Lat. \textunderscore instillatio\textunderscore )}
\end{itemize}
Acto ou efeito de instilar.
\section{Instilar}
\begin{itemize}
\item {Grp. gram.:v. t.}
\end{itemize}
\begin{itemize}
\item {Utilização:Fig.}
\end{itemize}
\begin{itemize}
\item {Proveniência:(Lat. \textunderscore instillare\textunderscore )}
\end{itemize}
Introduzir, gota a gota.
Insuflar; induzir; persuadir: \textunderscore instilar vinganças\textunderscore .
\section{Instillação}
\begin{itemize}
\item {Grp. gram.:f.}
\end{itemize}
\begin{itemize}
\item {Proveniência:(Lat. \textunderscore instillatio\textunderscore )}
\end{itemize}
Acto ou effeito de instillar.
\section{Instillar}
\begin{itemize}
\item {Grp. gram.:v. t.}
\end{itemize}
\begin{itemize}
\item {Utilização:Fig.}
\end{itemize}
\begin{itemize}
\item {Proveniência:(Lat. \textunderscore instillare\textunderscore )}
\end{itemize}
Introduzir, gota a gota.
Insuflar; induzir; persuadir: \textunderscore instillar vinganças\textunderscore .
\section{Instincto}
\begin{itemize}
\item {Grp. gram.:m.}
\end{itemize}
\begin{itemize}
\item {Proveniência:(Lat. \textunderscore instinctus\textunderscore )}
\end{itemize}
Instigação ou impulso natural, independente da reflexão.
Tendência ingênita dos animaes.
Inspiração.
\section{Instintivamente}
\begin{itemize}
\item {Grp. gram.:adv.}
\end{itemize}
De modo instintivo; naturalmente; espontaneamente.
\section{Instintivo}
\begin{itemize}
\item {Grp. gram.:adj.}
\end{itemize}
Relativo a instinto.
Impensado; espontâneo; natural: \textunderscore movimento instintivo\textunderscore .
\section{Instinto}
\begin{itemize}
\item {Grp. gram.:m.}
\end{itemize}
\begin{itemize}
\item {Proveniência:(Lat. \textunderscore instinctus\textunderscore )}
\end{itemize}
Instigação ou impulso natural, independente da reflexão.
Tendência ingênita dos animaes.
Inspiração.
\section{Ínstita}
\begin{itemize}
\item {Grp. gram.:f.}
\end{itemize}
\begin{itemize}
\item {Proveniência:(Lat. \textunderscore instita\textunderscore )}
\end{itemize}
Guarnição na fimbria do vestuário das antigas damas romanas.
\section{Institor}
\begin{itemize}
\item {Grp. gram.:m.}
\end{itemize}
\begin{itemize}
\item {Proveniência:(Lat. \textunderscore institor\textunderscore )}
\end{itemize}
Aquelle que dirige ou administra negócio ou empresa industrial, por nomeação ou escolha do proprietário ou gerente da mesma empresa.
\section{Institório}
\begin{itemize}
\item {Grp. gram.:adj.}
\end{itemize}
\begin{itemize}
\item {Proveniência:(Lat. \textunderscore institórius\textunderscore )}
\end{itemize}
Relativo a institor.
\section{Institucional}
\begin{itemize}
\item {Grp. gram.:adj.}
\end{itemize}
\begin{itemize}
\item {Utilização:bras}
\end{itemize}
\begin{itemize}
\item {Utilização:Neol.}
\end{itemize}
Relativo a uma instituição. Cf. \textunderscore Jorn. do Brasil\textunderscore , de 11-II-905.
\section{Institucionalmente}
\begin{itemize}
\item {Grp. gram.:adv.}
\end{itemize}
De modo institucional.
De acôrdo com uma instituição.
\section{Instituição}
\begin{itemize}
\item {fónica:tu-i}
\end{itemize}
\begin{itemize}
\item {Grp. gram.:f.}
\end{itemize}
\begin{itemize}
\item {Utilização:Ant.}
\end{itemize}
\begin{itemize}
\item {Grp. gram.:Pl.}
\end{itemize}
\begin{itemize}
\item {Proveniência:(Lat. \textunderscore instituitio\textunderscore )}
\end{itemize}
Acto ou effeito de instituir.
Instituto; coisa instituida ou estabelecida: \textunderscore a instituição do registo civil\textunderscore .
Nomeação de herdeiro.
Educação, ensino.
Leis fundamentaes de uma sociedade política: \textunderscore respeitar as instituições\textunderscore .
Regras, norma.
\section{Instituidor}
\begin{itemize}
\item {fónica:tu-i}
\end{itemize}
\begin{itemize}
\item {Grp. gram.:m.  e  adj.}
\end{itemize}
O que institue.
\section{Instituir}
\begin{itemize}
\item {Grp. gram.:v. t.}
\end{itemize}
\begin{itemize}
\item {Proveniência:(Lat. \textunderscore instituere\textunderscore )}
\end{itemize}
Fundar, criar.
Estabelecer: \textunderscore instituir um hospício\textunderscore .
Nomear por herdeiro.
Doutrinar; disciplinar.
Assinalar; aprazar.
\section{Instituto}
\begin{itemize}
\item {Grp. gram.:m.}
\end{itemize}
\begin{itemize}
\item {Proveniência:(Lat. \textunderscore institutum\textunderscore )}
\end{itemize}
Coisa instituida.
Constituição de uma Ordem religiosa.
Regulamentação.
Corporação literária, scientífica ou artística.
Intento.
\section{Instrução}
\begin{itemize}
\item {Grp. gram.:f.}
\end{itemize}
\begin{itemize}
\item {Grp. gram.:Pl.}
\end{itemize}
Acto ou efeito de instruir.
Preleção ou explicação, que se ministra para instruir.
Complexo de conhecimentos adquiridos: \textunderscore têr instrução\textunderscore .
Informações ou diligências, que esclarecem uma causa.
\section{Instrucção}
\begin{itemize}
\item {Grp. gram.:f.}
\end{itemize}
\begin{itemize}
\item {Grp. gram.:Pl.}
\end{itemize}
Acto ou effeito de instruir.
Prelecção ou explicação, que se ministra para instruir.
Complexo de conhecimentos adquiridos: \textunderscore têr instrucção\textunderscore .
Informações ou diligências, que esclarecem uma causa.
\section{Instructivo}
\begin{itemize}
\item {Grp. gram.:adj.}
\end{itemize}
\begin{itemize}
\item {Proveniência:(De \textunderscore instructo\textunderscore )}
\end{itemize}
Próprio para instruir; que contém ensinamento: \textunderscore passeios instructivos\textunderscore .
\section{Instructo}
\begin{itemize}
\item {Grp. gram.:adj.}
\end{itemize}
\begin{itemize}
\item {Utilização:ant.}
\end{itemize}
\begin{itemize}
\item {Utilização:Poét.}
\end{itemize}
\begin{itemize}
\item {Proveniência:(Lat. \textunderscore instructus\textunderscore )}
\end{itemize}
O mesmo que \textunderscore instruido\textunderscore :«\textunderscore ...instructos em muitas linguas\textunderscore ». Filinto, \textunderscore D. Man.\textunderscore , I, 67. Cf. \textunderscore Lusiadas\textunderscore , II, 57.
\section{Instructor}
\begin{itemize}
\item {Grp. gram.:m.  e  adj.}
\end{itemize}
\begin{itemize}
\item {Utilização:T. de Turquel}
\end{itemize}
\begin{itemize}
\item {Proveniência:(Lat. \textunderscore instructor\textunderscore )}
\end{itemize}
Aquelle que instrue.
Aquelle que dá instrucções ou ensino; aquelle que adestra.
Espertalhão.
\section{Instructura}
\begin{itemize}
\item {Grp. gram.:f.}
\end{itemize}
\begin{itemize}
\item {Proveniência:(Lat. \textunderscore instructura\textunderscore )}
\end{itemize}
Construcção mechânica de um edifício.
\section{Instruido}
\begin{itemize}
\item {Grp. gram.:adj.}
\end{itemize}
\begin{itemize}
\item {Proveniência:(De \textunderscore instruir\textunderscore )}
\end{itemize}
Que tem instrucção; que possue muitos conhecimentos.
Informado ou esclarecido sôbre um assumpto.
Que acompanha ou reforça certo documento ou petição: \textunderscore instrumento instruido com várias certidões\textunderscore .
\section{Instruidor}
\begin{itemize}
\item {Grp. gram.:m.  e  adj.}
\end{itemize}
O que instrue; instructor.
\section{Instruidote}
\begin{itemize}
\item {fónica:tru-i}
\end{itemize}
\begin{itemize}
\item {Grp. gram.:adj.}
\end{itemize}
Que tem alguma instrucção. Cf. Eça, \textunderscore P. Amaro\textunderscore , 332.
\section{Instruir}
\begin{itemize}
\item {Grp. gram.:v. t.}
\end{itemize}
\begin{itemize}
\item {Proveniência:(Lat. \textunderscore instruere\textunderscore )}
\end{itemize}
Ensinar; leccionar; transmittir conhecimentos a.
Adestrar.
Informar; esclarecer: \textunderscore instruir um processo\textunderscore .
\section{Instrumentação}
\begin{itemize}
\item {Grp. gram.:f.}
\end{itemize}
Acto ou effeito de instrumentar.
Modo ou arte de dispor as partes de uma peça musical.
\section{Instrumental}
\begin{itemize}
\item {Grp. gram.:adj.}
\end{itemize}
\begin{itemize}
\item {Grp. gram.:M.}
\end{itemize}
\begin{itemize}
\item {Proveniência:(De \textunderscore instrumento\textunderscore )}
\end{itemize}
Que serve de instrumento.
Relativo a instrumentos.
Instrumentos de uma orchestra.
Instrumentos de um offício mecânico, ou necessários para uma operação cirúrgica.
\section{Instrumentalista}
\begin{itemize}
\item {Grp. gram.:m.  e  f.}
\end{itemize}
Pessôa, que toca algum instrumento.
Fabricante de instrumentos.
(V. \textunderscore instrumentista\textunderscore , que é preferível)
\section{Instrumentalmente}
\begin{itemize}
\item {Grp. gram.:adv.}
\end{itemize}
De modo intrumental.
\section{Instrumentar}
\begin{itemize}
\item {Grp. gram.:v. t.}
\end{itemize}
\begin{itemize}
\item {Proveniência:(De \textunderscore instrumento\textunderscore )}
\end{itemize}
Escrever e applicar a vários instrumentos de uma orchestra (uma obra musical).
\section{Instrumentária}
\begin{itemize}
\item {Grp. gram.:adj. f.}
\end{itemize}
\begin{itemize}
\item {Utilização:Jur.}
\end{itemize}
\begin{itemize}
\item {Proveniência:(De \textunderscore instrumento\textunderscore )}
\end{itemize}
Diz-se da testemunha que assiste aos actos, cuja validade depende da presença della.
\section{Instrumentista}
\begin{itemize}
\item {Grp. gram.:m. ,  f.  e  adj.}
\end{itemize}
Pessôa, que toca algum instrumento musical.
Aquelle que compõe música instrumental; symphonista.
\section{Instrumento}
\begin{itemize}
\item {Grp. gram.:m.}
\end{itemize}
\begin{itemize}
\item {Utilização:Ant.}
\end{itemize}
\begin{itemize}
\item {Proveniência:(Lat. \textunderscore instrumentum\textunderscore )}
\end{itemize}
Qualquer agente mecânico, que se emprega para executar um trabalho ou uma operação.
Pessôa ou coisa, que serve de meio ou de auxílio para determinado fim.
Meio.
Apparelho, destinado a produzir sons musicaes.
Título escrito, para fazer valer ou comprovar algum direito.
Mobília.
\section{Instrutivo}
\begin{itemize}
\item {Grp. gram.:adj.}
\end{itemize}
\begin{itemize}
\item {Proveniência:(De \textunderscore instruto\textunderscore )}
\end{itemize}
Próprio para instruir; que contém ensinamento: \textunderscore passeios instrutivos\textunderscore .
\section{Instruto}
\begin{itemize}
\item {Grp. gram.:adj.}
\end{itemize}
\begin{itemize}
\item {Utilização:ant.}
\end{itemize}
\begin{itemize}
\item {Utilização:Poét.}
\end{itemize}
\begin{itemize}
\item {Proveniência:(Lat. \textunderscore instructus\textunderscore )}
\end{itemize}
O mesmo que \textunderscore instruido\textunderscore :«\textunderscore ...instrutos em muitas linguas\textunderscore ». Filinto, \textunderscore D. Man.\textunderscore , I, 67. Cf. \textunderscore Lusiadas\textunderscore , II, 57.
\section{Instrutor}
\begin{itemize}
\item {Grp. gram.:m.  e  adj.}
\end{itemize}
\begin{itemize}
\item {Utilização:T. de Turquel}
\end{itemize}
\begin{itemize}
\item {Proveniência:(Lat. \textunderscore instructor\textunderscore )}
\end{itemize}
Aquele que instrue.
Aquele que dá instruções ou ensino; aquele que adestra.
Espertalhão.
\section{Instrutura}
\begin{itemize}
\item {Grp. gram.:f.}
\end{itemize}
\begin{itemize}
\item {Proveniência:(Lat. \textunderscore instructura\textunderscore )}
\end{itemize}
Construcção mecânica de um edifício.
\section{Ínsua}
\begin{itemize}
\item {Grp. gram.:f.}
\end{itemize}
\begin{itemize}
\item {Proveniência:(Do lat. \textunderscore insula\textunderscore )}
\end{itemize}
Pequena ilha, banhada de algum lado por um rio e do outro ou outros por levada ou corrente que sai do mesmo rio.
Terra regadia, junto ao rio.
Ilhota.
Pequena ilha de areia, no Vouga, Mondego e Minho.
\section{Insuave}
\begin{itemize}
\item {Grp. gram.:adj.}
\end{itemize}
\begin{itemize}
\item {Proveniência:(Lat. \textunderscore insuavis\textunderscore )}
\end{itemize}
Que não é suave.
\section{Insuavidade}
\begin{itemize}
\item {Grp. gram.:f.}
\end{itemize}
\begin{itemize}
\item {Proveniência:(Lat. \textunderscore insuavitas\textunderscore )}
\end{itemize}
Falta de suavidade.
\section{Insubjugado}
\begin{itemize}
\item {Grp. gram.:adj.}
\end{itemize}
\begin{itemize}
\item {Proveniência:(De \textunderscore in...\textunderscore  + \textunderscore subjugado\textunderscore )}
\end{itemize}
Não subjugado, não vencido. Cf. Garrett, \textunderscore Romanceiro\textunderscore , I, 68.
\section{Insubmergível}
\begin{itemize}
\item {Grp. gram.:adj.}
\end{itemize}
\begin{itemize}
\item {Proveniência:(De \textunderscore in...\textunderscore  + \textunderscore submergível\textunderscore )}
\end{itemize}
Que não é submergível.
\section{Insubmersível}
\begin{itemize}
\item {Grp. gram.:adj.}
\end{itemize}
\begin{itemize}
\item {Proveniência:(De \textunderscore in...\textunderscore  + \textunderscore submersível\textunderscore )}
\end{itemize}
O mesmo que \textunderscore insubmergível\textunderscore .
\section{Insubmisso}
\begin{itemize}
\item {Grp. gram.:adj.}
\end{itemize}
\begin{itemize}
\item {Proveniência:(De \textunderscore in...\textunderscore  + \textunderscore submisso\textunderscore )}
\end{itemize}
Não submisso.
Independente.
Altivo: \textunderscore carácter insubmisso\textunderscore .
\section{Insubordinação}
\begin{itemize}
\item {Grp. gram.:f.}
\end{itemize}
\begin{itemize}
\item {Proveniência:(De \textunderscore in...\textunderscore  + \textunderscore subordinação\textunderscore )}
\end{itemize}
Falta de subordinação; estado de quem é insubordinado.
Acto de indisciplina.
\section{Insubordinadamente}
\begin{itemize}
\item {Grp. gram.:adv.}
\end{itemize}
De modo insubordinado.
Com insubordinação.
\section{Insubordinado}
\begin{itemize}
\item {Grp. gram.:m.}
\end{itemize}
\begin{itemize}
\item {Proveniência:(De \textunderscore insubordinar\textunderscore )}
\end{itemize}
Aquelle que não é subordinado.
Aquelle que faltou á subordinação e á disciplina.
\section{Insubordinar}
\begin{itemize}
\item {Grp. gram.:v. t.}
\end{itemize}
\begin{itemize}
\item {Proveniência:(De \textunderscore in...\textunderscore  + \textunderscore subordinar\textunderscore )}
\end{itemize}
Tornar insubordinado.
Amotinar; sublevar.
\section{Insubordinável}
\begin{itemize}
\item {Grp. gram.:adj.}
\end{itemize}
\begin{itemize}
\item {Proveniência:(De \textunderscore in...\textunderscore  + \textunderscore subordinável\textunderscore )}
\end{itemize}
Que se não póde subordinar.
Indócil; incorrigível.
\section{Insubornável}
\begin{itemize}
\item {Grp. gram.:adj.}
\end{itemize}
\begin{itemize}
\item {Proveniência:(De \textunderscore in...\textunderscore  + \textunderscore subornável\textunderscore )}
\end{itemize}
Que se não póde subornar; incorruptível, íntegro.
\section{Insubre}
\begin{itemize}
\item {Grp. gram.:m.  e  adj.}
\end{itemize}
\begin{itemize}
\item {Utilização:Ant.}
\end{itemize}
\begin{itemize}
\item {Proveniência:(Lat. \textunderscore insubres\textunderscore )}
\end{itemize}
O mesmo que \textunderscore milanês\textunderscore .
\section{Insúbrio}
\begin{itemize}
\item {Grp. gram.:m.  e  adj.}
\end{itemize}
\begin{itemize}
\item {Utilização:Ant.}
\end{itemize}
\begin{itemize}
\item {Proveniência:(Lat. \textunderscore insubres\textunderscore )}
\end{itemize}
O mesmo que \textunderscore milanês\textunderscore .
\section{Insubsistência}
\begin{itemize}
\item {Grp. gram.:f.}
\end{itemize}
Qualidade de insubsistente.
\section{Insubsistente}
\begin{itemize}
\item {Grp. gram.:adj.}
\end{itemize}
\begin{itemize}
\item {Proveniência:(De \textunderscore in...\textunderscore  + \textunderscore subsistente\textunderscore )}
\end{itemize}
Que não é subsistente; que não pode subsistir.
Que não tem base ou razão de sêr: \textunderscore pretensão insubsistente\textunderscore .
\section{Insubstancial}
\begin{itemize}
\item {Grp. gram.:adj.}
\end{itemize}
\begin{itemize}
\item {Proveniência:(De \textunderscore in\textunderscore .. + \textunderscore substancial\textunderscore )}
\end{itemize}
Que não é substancial; secundário.
\section{Insubstancialidade}
\begin{itemize}
\item {Grp. gram.:f.}
\end{itemize}
Qualidade de insubstancial.
\section{Insubstituivel}
\begin{itemize}
\item {Grp. gram.:adj.}
\end{itemize}
\begin{itemize}
\item {Proveniência:(De \textunderscore in...\textunderscore  + \textunderscore substituivel\textunderscore )}
\end{itemize}
Que se não póde substituir.
Inigualável.
\section{Insuccessível}
\begin{itemize}
\item {Grp. gram.:adj.}
\end{itemize}
\begin{itemize}
\item {Proveniência:(De \textunderscore in...\textunderscore  + \textunderscore successível\textunderscore )}
\end{itemize}
Que não é successível.
\section{Insuccesso}
\begin{itemize}
\item {Grp. gram.:m.}
\end{itemize}
\begin{itemize}
\item {Utilização:Gal}
\end{itemize}
\begin{itemize}
\item {Proveniência:(De \textunderscore in...\textunderscore  + \textunderscore successo\textunderscore )}
\end{itemize}
Mau resultado; falta de bom êxito.
Falta de efficácia. Cf. Palmeirim, \textunderscore Portugal\textunderscore , 59.
\section{Insucessível}
\begin{itemize}
\item {Grp. gram.:adj.}
\end{itemize}
\begin{itemize}
\item {Proveniência:(De \textunderscore in...\textunderscore  + \textunderscore sucessível\textunderscore )}
\end{itemize}
Que não é sucessível.
\section{Insucesso}
\begin{itemize}
\item {Grp. gram.:m.}
\end{itemize}
\begin{itemize}
\item {Utilização:Gal}
\end{itemize}
\begin{itemize}
\item {Proveniência:(De \textunderscore in...\textunderscore  + \textunderscore sucesso\textunderscore )}
\end{itemize}
Mau resultado; falta de bom êxito.
Falta de eficácia. Cf. Palmeirim, \textunderscore Portugal\textunderscore , 59.
\section{Insueto}
\begin{itemize}
\item {Grp. gram.:adj.}
\end{itemize}
\begin{itemize}
\item {Proveniência:(Lat. \textunderscore insuetus\textunderscore )}
\end{itemize}
O mesmo que \textunderscore desusado\textunderscore .
\section{Insufficiência}
\begin{itemize}
\item {Grp. gram.:f.}
\end{itemize}
\begin{itemize}
\item {Utilização:Fig.}
\end{itemize}
\begin{itemize}
\item {Proveniência:(Lat. \textunderscore insufficientia\textunderscore )}
\end{itemize}
Qualidade daquillo que é insufficiente.
Ineptidão; incapacidade.
\section{Insufficiente}
\begin{itemize}
\item {Grp. gram.:adj.}
\end{itemize}
\begin{itemize}
\item {Utilização:Fig.}
\end{itemize}
\begin{itemize}
\item {Proveniência:(Lat. \textunderscore insufficiens\textunderscore )}
\end{itemize}
Não sufficiente.
Inepto; incapaz.
\section{Insufficientemente}
\begin{itemize}
\item {Grp. gram.:adv.}
\end{itemize}
De modo insufficiente.
\section{Insufflação}
\begin{itemize}
\item {Grp. gram.:f.}
\end{itemize}
\begin{itemize}
\item {Proveniência:(Lat. \textunderscore insuflatio\textunderscore )}
\end{itemize}
Acto de insuflar.
\section{Insufflador}
\begin{itemize}
\item {Grp. gram.:adj.}
\end{itemize}
\begin{itemize}
\item {Grp. gram.:M.}
\end{itemize}
Que insuffla.
Apparelho, próprio para insufflações.
\section{Insufflar}
\begin{itemize}
\item {Grp. gram.:v. t.}
\end{itemize}
\begin{itemize}
\item {Utilização:Fig.}
\end{itemize}
\begin{itemize}
\item {Proveniência:(Lat. \textunderscore insufflare\textunderscore )}
\end{itemize}
Soprar para dentro.
Encher de ar, soprando.
Introduzir, soprando, (pós medicamentosos ou outras substâncias).
Insinuar, suggerir.
\section{Insuficiência}
\begin{itemize}
\item {Grp. gram.:f.}
\end{itemize}
\begin{itemize}
\item {Utilização:Fig.}
\end{itemize}
\begin{itemize}
\item {Proveniência:(Lat. \textunderscore insufficientia\textunderscore )}
\end{itemize}
Qualidade daquilo que é insuficiente.
Ineptidão; incapacidade.
\section{Insuficiente}
\begin{itemize}
\item {Grp. gram.:adj.}
\end{itemize}
\begin{itemize}
\item {Utilização:Fig.}
\end{itemize}
\begin{itemize}
\item {Proveniência:(Lat. \textunderscore insufficiens\textunderscore )}
\end{itemize}
Não suficiente.
Inepto; incapaz.
\section{Insuficientemente}
\begin{itemize}
\item {Grp. gram.:adv.}
\end{itemize}
De modo insuficiente.
\section{Insuflação}
\begin{itemize}
\item {Grp. gram.:f.}
\end{itemize}
\begin{itemize}
\item {Proveniência:(Lat. \textunderscore insuflatio\textunderscore )}
\end{itemize}
Acto de insuflar.
\section{Insuflador}
\begin{itemize}
\item {Grp. gram.:adj.}
\end{itemize}
\begin{itemize}
\item {Grp. gram.:M.}
\end{itemize}
Que insufla.
Aparelho, próprio para insuflações.
\section{Insuflar}
\begin{itemize}
\item {Grp. gram.:v. t.}
\end{itemize}
\begin{itemize}
\item {Utilização:Fig.}
\end{itemize}
\begin{itemize}
\item {Proveniência:(Lat. \textunderscore insufflare\textunderscore )}
\end{itemize}
Soprar para dentro.
Encher de ar, soprando.
Introduzir, soprando, (pós medicamentosos ou outras substâncias).
Insinuar, sugerir.
\section{Ínsula}
\begin{itemize}
\item {Grp. gram.:f.}
\end{itemize}
\begin{itemize}
\item {Utilização:Poét.}
\end{itemize}
\begin{itemize}
\item {Utilização:Des.}
\end{itemize}
\begin{itemize}
\item {Proveniência:(Lat. \textunderscore insula\textunderscore )}
\end{itemize}
O mesmo que \textunderscore ilha\textunderscore .
Moradia insulada.
\section{Insulação}
\begin{itemize}
\item {Grp. gram.:f.}
\end{itemize}
Acto ou effeito de insular^1. Cf. Camillo, \textunderscore Sc. da Foz\textunderscore , 190.
\section{Insulado}
\begin{itemize}
\item {Grp. gram.:adj.}
\end{itemize}
\begin{itemize}
\item {Proveniência:(De \textunderscore insular\textunderscore ^1)}
\end{itemize}
Separado.
Incommunicável.
Solitário: \textunderscore viver insulado\textunderscore .
\section{Insulador}
\begin{itemize}
\item {Grp. gram.:adj.}
\end{itemize}
\begin{itemize}
\item {Grp. gram.:M.}
\end{itemize}
\begin{itemize}
\item {Proveniência:(De \textunderscore insular\textunderscore ^1)}
\end{itemize}
Que insula.
Que separa.
Instrumento de phýsica, sobre que se colloca um corpo que se quere electrizar.
\section{Insulamento}
\begin{itemize}
\item {Grp. gram.:m.}
\end{itemize}
Acto ou effeito de insular^1. Cf. Camillo, \textunderscore Mulher Fatal\textunderscore , 56.
\section{Insulano}
\begin{itemize}
\item {Grp. gram.:adj.}
\end{itemize}
\begin{itemize}
\item {Grp. gram.:M.}
\end{itemize}
\begin{itemize}
\item {Proveniência:(Lat. \textunderscore insulanus\textunderscore )}
\end{itemize}
Relativo a ilha.
Aquelle que é natural de uma ilha.
\section{Insulanamente}
\begin{itemize}
\item {Grp. gram.:adv.}
\end{itemize}
\begin{itemize}
\item {Proveniência:(De \textunderscore insulano\textunderscore )}
\end{itemize}
Á maneira dos ilhéus.
\section{Insulante}
\begin{itemize}
\item {Grp. gram.:adj.}
\end{itemize}
Que insula.
\section{Insular}
\begin{itemize}
\item {Grp. gram.:v. t.}
\end{itemize}
\begin{itemize}
\item {Proveniência:(Do lat. \textunderscore insula\textunderscore )}
\end{itemize}
Tornar semelhante a uma ilha.
Tornar incommunicável.
Tornar solitário; separar da sociedade. Cf. Herculano, \textunderscore Hist. de Port.\textunderscore , IV, 177.
Pôr (um corpo) em condições de não transmittir a outro a electricidade que tem.
\section{Insular}
\begin{itemize}
\item {Grp. gram.:m.  e  adj.}
\end{itemize}
\begin{itemize}
\item {Proveniência:(Lat. \textunderscore insularis\textunderscore )}
\end{itemize}
O mesmo que \textunderscore insulano\textunderscore .
\section{Insulativo}
\begin{itemize}
\item {Grp. gram.:adj.}
\end{itemize}
\begin{itemize}
\item {Utilização:Philol.}
\end{itemize}
\begin{itemize}
\item {Proveniência:(De \textunderscore insular\textunderscore ^1)}
\end{itemize}
Diz-se das línguas, em que a raíz e noção principal persistem separadas inteiramente da derivação e flexão, como succede nas línguas monosyllábicas. Cf. C. Figueiredo, \textunderscore Manual da Sc. da Ling.\textunderscore , 202.
\section{Insulcado}
\begin{itemize}
\item {Grp. gram.:adj.}
\end{itemize}
\begin{itemize}
\item {Utilização:Fig.}
\end{itemize}
\begin{itemize}
\item {Proveniência:(De \textunderscore in...\textunderscore  + \textunderscore sulcado\textunderscore )}
\end{itemize}
Não sulcado.
Ainda não navegado.
\section{Ínsulo}
\begin{itemize}
\item {Grp. gram.:adj.}
\end{itemize}
O mesmo que \textunderscore insulado\textunderscore . Cf. Macedo, \textunderscore Burros\textunderscore , 261.
\section{Insulsamente}
\begin{itemize}
\item {Grp. gram.:adv.}
\end{itemize}
De modo insulso.
\section{Insulsaria}
\begin{itemize}
\item {Grp. gram.:f.}
\end{itemize}
O mesmo que \textunderscore insulsez\textunderscore . Cf. Camillo, \textunderscore Caveira\textunderscore , 352.
\section{Insulsez}
\begin{itemize}
\item {Grp. gram.:f.}
\end{itemize}
Qualidade de insôlso. Cf. Camillo, \textunderscore Hist. e Sentimentalismo\textunderscore , 127.
\section{Insulsidade}
\begin{itemize}
\item {Grp. gram.:f.}
\end{itemize}
O mesmo que \textunderscore insulsez\textunderscore .
\section{Insulso}
\begin{itemize}
\item {Grp. gram.:adj.}
\end{itemize}
\begin{itemize}
\item {Utilização:Ext.}
\end{itemize}
\begin{itemize}
\item {Utilização:Fig.}
\end{itemize}
\begin{itemize}
\item {Proveniência:(Lat. \textunderscore insulsus\textunderscore )}
\end{itemize}
Que não tem sal.
Insosso; insípido.
Desenxabido.
Que não tem graça.
Monótono.
\section{Insultador}
\begin{itemize}
\item {Grp. gram.:m.  e  adj.}
\end{itemize}
\begin{itemize}
\item {Grp. gram.:M.}
\end{itemize}
\begin{itemize}
\item {Proveniência:(Lat. \textunderscore insultans\textunderscore )}
\end{itemize}
Que insulta.
Que envolve injúria.
Aquelle que insulta.
\section{Insultante}
\begin{itemize}
\item {Grp. gram.:adj.}
\end{itemize}
\begin{itemize}
\item {Grp. gram.:M.}
\end{itemize}
\begin{itemize}
\item {Proveniência:(Lat. \textunderscore insultans\textunderscore )}
\end{itemize}
Que insulta.
Que envolve injúria.
Aquelle que insulta.
\section{Insultar}
\begin{itemize}
\item {Grp. gram.:v. t.}
\end{itemize}
\begin{itemize}
\item {Proveniência:(Lat. \textunderscore insultare\textunderscore )}
\end{itemize}
Dirigir insultos a.
Injuriar gravemente; afrontar.
\section{Insulto}
\begin{itemize}
\item {Grp. gram.:m.}
\end{itemize}
\begin{itemize}
\item {Proveniência:(Lat. \textunderscore insultus\textunderscore )}
\end{itemize}
Injúria violenta.
Offensa, por actos ou palavras; afronta.
Ataque repentino: \textunderscore insulto apopléctico\textunderscore .
\section{Insultuosamente}
\begin{itemize}
\item {Grp. gram.:adv.}
\end{itemize}
De modo insultuoso; offensivamente.
\section{Insultuoso}
\begin{itemize}
\item {Grp. gram.:adj.}
\end{itemize}
O mesmo que \textunderscore insultante\textunderscore .
\section{Insuperável}
\begin{itemize}
\item {Grp. gram.:adj.}
\end{itemize}
\begin{itemize}
\item {Proveniência:(Lat. \textunderscore insuperabilis\textunderscore )}
\end{itemize}
Que se não póde superar; invencível; indomável.
\section{Insuperavelmente}
\begin{itemize}
\item {Grp. gram.:adv.}
\end{itemize}
De modo insuperável.
\section{Insuportável}
\begin{itemize}
\item {Grp. gram.:adj.}
\end{itemize}
\begin{itemize}
\item {Proveniência:(De \textunderscore in...\textunderscore  + \textunderscore suportável\textunderscore )}
\end{itemize}
Que não é suportável; intolerável; muito incômodo ou molesto.
\section{Insuportavelmente}
\begin{itemize}
\item {Grp. gram.:adv.}
\end{itemize}
De modo insuportável.
\section{Insupportável}
\begin{itemize}
\item {Grp. gram.:adj.}
\end{itemize}
\begin{itemize}
\item {Proveniência:(De \textunderscore in...\textunderscore  + \textunderscore supportável\textunderscore )}
\end{itemize}
Que não é supportável; intolerável; muito incômmodo ou molesto.
\section{Insupportavelmente}
\begin{itemize}
\item {Grp. gram.:adv.}
\end{itemize}
De modo insupportável.
\section{Insupprível}
\begin{itemize}
\item {Grp. gram.:adj.}
\end{itemize}
\begin{itemize}
\item {Proveniência:(De \textunderscore in...\textunderscore  + \textunderscore supprível\textunderscore )}
\end{itemize}
Que se não póde supprir.
\section{Insuprível}
\begin{itemize}
\item {Grp. gram.:adj.}
\end{itemize}
\begin{itemize}
\item {Proveniência:(De \textunderscore in...\textunderscore  + \textunderscore suprível\textunderscore )}
\end{itemize}
Que se não póde suprir.
\section{Insurdescência}
\begin{itemize}
\item {Grp. gram.:f.}
\end{itemize}
\begin{itemize}
\item {Proveniência:(Do rad. de \textunderscore surdêz\textunderscore . Cp. \textunderscore ensurdecer\textunderscore )}
\end{itemize}
Estado de quem é surdo.
\section{Insurgente}
\begin{itemize}
\item {Grp. gram.:adj.}
\end{itemize}
\begin{itemize}
\item {Grp. gram.:M.}
\end{itemize}
\begin{itemize}
\item {Proveniência:(Lat. \textunderscore insurgens\textunderscore )}
\end{itemize}
Que se insurge.
Individuo que se insurgiu; rebelde.
\section{Insurgir}
\begin{itemize}
\item {Grp. gram.:v. t.}
\end{itemize}
\begin{itemize}
\item {Grp. gram.:V. i.}
\end{itemize}
\begin{itemize}
\item {Proveniência:(Lat. \textunderscore insurgere\textunderscore )}
\end{itemize}
Levantar; sublevar; revolucionar.
Surgir, emergir:«\textunderscore ...o fantasma do padre a insurgir das profundezas do abysmo...\textunderscore »Camillo, \textunderscore Volcões\textunderscore , 129.
\section{Insurreccionado}
\begin{itemize}
\item {Grp. gram.:m.}
\end{itemize}
\begin{itemize}
\item {Proveniência:(De \textunderscore insurreccionar\textunderscore )}
\end{itemize}
Individuo, que se insurreccionou.
\section{Insurreccional}
\begin{itemize}
\item {Grp. gram.:adj.}
\end{itemize}
\begin{itemize}
\item {Proveniência:(Do lat. \textunderscore insurrectio\textunderscore )}
\end{itemize}
Relativo a insurreição.
\section{Insurreccionalmente}
\begin{itemize}
\item {Grp. gram.:adv.}
\end{itemize}
De modo insurrecional.
\section{Insurreccionar}
\begin{itemize}
\item {Grp. gram.:v. t.}
\end{itemize}
\begin{itemize}
\item {Proveniência:(Do lat. \textunderscore insurrectio\textunderscore )}
\end{itemize}
O mesmo que \textunderscore insurgir\textunderscore .
Revoltar.
\section{Insurreccionário}
\begin{itemize}
\item {Grp. gram.:adj.}
\end{itemize}
\begin{itemize}
\item {Grp. gram.:M.}
\end{itemize}
O mesmo que \textunderscore insurreccional\textunderscore .
Aquelle que se insúrge ou se revolta.
\section{Insurrecto}
\begin{itemize}
\item {Grp. gram.:adj.}
\end{itemize}
\begin{itemize}
\item {Grp. gram.:M.}
\end{itemize}
\begin{itemize}
\item {Proveniência:(Lat. \textunderscore insurrectus\textunderscore )}
\end{itemize}
Que se insurgiu; que se rebellou: \textunderscore as tríbos insurrectas\textunderscore .
Aquelle que se insurgiu, que se rebellou; insurgente: \textunderscore os insurrectos da Catalunha\textunderscore .
\section{Insurreição}
\begin{itemize}
\item {Grp. gram.:f.}
\end{itemize}
\begin{itemize}
\item {Utilização:Fig.}
\end{itemize}
\begin{itemize}
\item {Proveniência:(Lat. \textunderscore insurrectio\textunderscore )}
\end{itemize}
Acto de insurgir: \textunderscore rebellião\textunderscore .
Os insurrectos.
Opposição vigorosa.
\section{Insusceptível}
\begin{itemize}
\item {Grp. gram.:adj.}
\end{itemize}
\begin{itemize}
\item {Proveniência:(De \textunderscore in...\textunderscore  + \textunderscore susceptível\textunderscore )}
\end{itemize}
Que não é susceptível.
\section{Insuspeito}
\begin{itemize}
\item {Grp. gram.:adj.}
\end{itemize}
\begin{itemize}
\item {Proveniência:(De \textunderscore in...\textunderscore  + \textunderscore suspeito\textunderscore )}
\end{itemize}
Não suspeito; imparcial; fidedigno.
\section{Insustentável}
\begin{itemize}
\item {Grp. gram.:adj.}
\end{itemize}
\begin{itemize}
\item {Proveniência:(De \textunderscore in...\textunderscore  + \textunderscore sustentável\textunderscore )}
\end{itemize}
Que não é sustentável; que não tem fundamento; que não póde subsistir: \textunderscore defesa insustentável\textunderscore .
\section{Intáctil}
\begin{itemize}
\item {Grp. gram.:adj.}
\end{itemize}
\begin{itemize}
\item {Proveniência:(Lat. \textunderscore intactilis\textunderscore )}
\end{itemize}
Que não é táctil; intangível.
\section{Intactilidade}
\begin{itemize}
\item {Grp. gram.:f.}
\end{itemize}
Qualidade de intáctil.
\section{Intacto}
\begin{itemize}
\item {Grp. gram.:adj.}
\end{itemize}
\begin{itemize}
\item {Utilização:Fig.}
\end{itemize}
\begin{itemize}
\item {Proveniência:(Lat. \textunderscore intactus\textunderscore )}
\end{itemize}
Não tocado; integro; illeso.
Impolluto.
\section{Intaipaba}
\begin{itemize}
\item {fónica:ta-i}
\end{itemize}
\begin{itemize}
\item {Grp. gram.:f.}
\end{itemize}
(Corr. de \textunderscore itaipava\textunderscore )
\section{Intaipava}
\begin{itemize}
\item {fónica:ta-i}
\end{itemize}
\begin{itemize}
\item {Grp. gram.:f.}
\end{itemize}
\begin{itemize}
\item {Utilização:Bras}
\end{itemize}
(Corr. de \textunderscore itaipava\textunderscore )
\section{Intan}
\textunderscore f.\textunderscore  (\textunderscore Bras.\textunderscore )
(Corr. de \textunderscore itan\textunderscore )
\section{Intangendo}
\begin{itemize}
\item {Grp. gram.:adj.}
\end{itemize}
\begin{itemize}
\item {Proveniência:(Do lat. \textunderscore in...\textunderscore  + \textunderscore tangendus\textunderscore )}
\end{itemize}
Em que se não póde tocar.
Inatacável. Cf. Castilho, \textunderscore Fastos\textunderscore , III, 163.
\section{Intangibilidade}
\begin{itemize}
\item {Grp. gram.:f.}
\end{itemize}
Qualidade de intangivel.
\section{Intangível}
\begin{itemize}
\item {Grp. gram.:adj.}
\end{itemize}
\begin{itemize}
\item {Proveniência:(De \textunderscore in...\textunderscore  + \textunderscore tangivel\textunderscore )}
\end{itemize}
Que não é tangivel; em que se não póde tocar.
Que é impalpável.
\section{Inté}
\begin{itemize}
\item {Grp. gram.:prep.}
\end{itemize}
\begin{itemize}
\item {Utilização:pleb.}
\end{itemize}
\begin{itemize}
\item {Utilização:Ant.}
\end{itemize}
O mesmo que \textunderscore até\textunderscore . Cf. Simão Mach., f. 15.
\section{Integérrimo}
\begin{itemize}
\item {Grp. gram.:adj.}
\end{itemize}
\begin{itemize}
\item {Proveniência:(Do lat. \textunderscore integer\textunderscore )}
\end{itemize}
Muito integro; muito recto; justiceiro.
\section{Íntegra}
\begin{itemize}
\item {Grp. gram.:f.}
\end{itemize}
\begin{itemize}
\item {Proveniência:(De \textunderscore integro\textunderscore )}
\end{itemize}
Contexto completo; totalidade: \textunderscore reproduzir um artigo na íntegra\textunderscore .
\section{Integração}
\begin{itemize}
\item {Grp. gram.:f.}
\end{itemize}
\begin{itemize}
\item {Proveniência:(Lat. \textunderscore integratio\textunderscore )}
\end{itemize}
Acto de integrar.
\section{Integral}
\begin{itemize}
\item {Grp. gram.:adj.}
\end{itemize}
\begin{itemize}
\item {Utilização:Mathem.}
\end{itemize}
\begin{itemize}
\item {Grp. gram.:F.}
\end{itemize}
\begin{itemize}
\item {Utilização:Mathem.}
\end{itemize}
\begin{itemize}
\item {Proveniência:(De \textunderscore íntegro\textunderscore )}
\end{itemize}
Inteiro; total.
Que intégra.
Diz-se de um cálculo, que é o inverso do differencial.
Somma dos valores finitos de uma differencial, entre os limites dados da variável.
\section{Integralmente}
\begin{itemize}
\item {Grp. gram.:adv.}
\end{itemize}
De modo integral.
\section{Integramente}
\begin{itemize}
\item {Grp. gram.:adv.}
\end{itemize}
De modo integro.
\section{Integrante}
\begin{itemize}
\item {Grp. gram.:adj.}
\end{itemize}
\begin{itemize}
\item {Utilização:Fig.}
\end{itemize}
\begin{itemize}
\item {Utilização:Phýs.}
\end{itemize}
\begin{itemize}
\item {Utilização:Gram.}
\end{itemize}
\begin{itemize}
\item {Proveniência:(Lat. \textunderscore integrans\textunderscore )}
\end{itemize}
Que intégra; que completa.
Necessário.
Que constitue um corpo simples ou composto.
Diz-se da proposição, que intégra ou completa o sentido de outra, como a segunda destas duas: \textunderscore digo-te que mentes\textunderscore .
\section{Integrar}
\begin{itemize}
\item {Grp. gram.:v. t.}
\end{itemize}
\begin{itemize}
\item {Utilização:Mathem.}
\end{itemize}
\begin{itemize}
\item {Proveniência:(Lat. \textunderscore integrare\textunderscore )}
\end{itemize}
Tornar inteiro; completar.
Determinar a integral de.
\section{Integrável}
\begin{itemize}
\item {Grp. gram.:adj.}
\end{itemize}
\begin{itemize}
\item {Proveniência:(De \textunderscore integrar\textunderscore )}
\end{itemize}
Que póde sêr integrado.
\section{Integridade}
\begin{itemize}
\item {Grp. gram.:f.}
\end{itemize}
\begin{itemize}
\item {Utilização:Fig.}
\end{itemize}
\begin{itemize}
\item {Proveniência:(Lat. \textunderscore integritas\textunderscore )}
\end{itemize}
Qualidade de íntegro; inteireza.
Innocência.
Rectidão, imparcialidade.
\section{Integrifólio}
\begin{itemize}
\item {Grp. gram.:adj.}
\end{itemize}
\begin{itemize}
\item {Utilização:Bot.}
\end{itemize}
\begin{itemize}
\item {Proveniência:(Do lat. \textunderscore integer\textunderscore  + \textunderscore folium\textunderscore )}
\end{itemize}
Que tem fôlhas inteiras.
\section{Íntegro}
\begin{itemize}
\item {Grp. gram.:adj.}
\end{itemize}
\begin{itemize}
\item {Utilização:Fig.}
\end{itemize}
\begin{itemize}
\item {Proveniência:(Lat. \textunderscore integer\textunderscore )}
\end{itemize}
O mesmo que \textunderscore inteiro\textunderscore ; completo; perfeito.
Recto; incorruptível: \textunderscore juiz íntegro\textunderscore .
Pundonoroso.
\section{Integumento}
\begin{itemize}
\item {Grp. gram.:m.}
\end{itemize}
\begin{itemize}
\item {Utilização:Des.}
\end{itemize}
\begin{itemize}
\item {Proveniência:(Lat. \textunderscore integumentum\textunderscore )}
\end{itemize}
O mesmo que \textunderscore cobertura\textunderscore .
\section{Inteiração}
\begin{itemize}
\item {Grp. gram.:f.}
\end{itemize}
Acto de inteirar.
Acto de completar ou fazer, em dinheiro, o pagamento de uma ração a bordo, a qual não foi recebida em gêneros ou só o foi em parte.
\section{Inteiramente}
\begin{itemize}
\item {Grp. gram.:adv.}
\end{itemize}
\begin{itemize}
\item {Proveniência:(De \textunderscore inteiro\textunderscore )}
\end{itemize}
Completamente; perfeitamente.
\section{Inteirar}
\begin{itemize}
\item {Grp. gram.:v. t.}
\end{itemize}
\begin{itemize}
\item {Proveniência:(Do lat. \textunderscore integrare\textunderscore )}
\end{itemize}
Tornar inteiro ou completo.
Tornar sciente.
\section{Inteireza}
\begin{itemize}
\item {Grp. gram.:f.}
\end{itemize}
Qualidade daquillo que é inteiro.
Integridade phýsica ou moral.
\section{Inteiriçar}
\begin{itemize}
\item {Grp. gram.:v. t.}
\end{itemize}
Tornar inteiriço ou hirto.
\section{Inteiriço}
\begin{itemize}
\item {Grp. gram.:adj.}
\end{itemize}
\begin{itemize}
\item {Utilização:Fig.}
\end{itemize}
\begin{itemize}
\item {Proveniência:(De \textunderscore inteiro\textunderscore )}
\end{itemize}
Feito de uma só peça: \textunderscore um barco inteiriço\textunderscore .
Hirto; inflexivel.
\section{Inteiro}
\begin{itemize}
\item {Grp. gram.:adj.}
\end{itemize}
\begin{itemize}
\item {Utilização:Arith.}
\end{itemize}
\begin{itemize}
\item {Utilização:Fig.}
\end{itemize}
\begin{itemize}
\item {Grp. gram.:M.}
\end{itemize}
\begin{itemize}
\item {Utilização:Arith.}
\end{itemize}
\begin{itemize}
\item {Proveniência:(Do lat. \textunderscore integer\textunderscore )}
\end{itemize}
Que tem todas as suas partes, toda a sua extensão.
Que não soffreu deminuição, que não foi modificado.
Inteiriço.
Exacto; completo: \textunderscore verdade inteira\textunderscore .
Illeso.
Não corrompido.
Não castrado.
Que não tem fracções: \textunderscore número inteiro\textunderscore .
Recto, incorruptivel.
Austero.
Número, que não tem fracções.
\section{Intelecção}
\begin{itemize}
\item {Grp. gram.:f.}
\end{itemize}
\begin{itemize}
\item {Proveniência:(Lat. \textunderscore intellectio\textunderscore )}
\end{itemize}
Acto de entender. Cf. \textunderscore Luz e Calor\textunderscore , 32.
\section{Intelectivamente}
\begin{itemize}
\item {Grp. gram.:adv.}
\end{itemize}
De modo intelectivo.
\section{Intelectível}
\begin{itemize}
\item {Grp. gram.:adj.}
\end{itemize}
\begin{itemize}
\item {Proveniência:(Lat. \textunderscore intellectivus\textunderscore )}
\end{itemize}
Relativo á inteligência; intelectual.
\section{Intelectivo}
\begin{itemize}
\item {Grp. gram.:adj.}
\end{itemize}
\begin{itemize}
\item {Proveniência:(Lat. \textunderscore intellectivus\textunderscore )}
\end{itemize}
Relativo á inteligência; intelectual.
\section{Intelecto}
\begin{itemize}
\item {Grp. gram.:m.}
\end{itemize}
\begin{itemize}
\item {Proveniência:(Lat. \textunderscore intellectus\textunderscore )}
\end{itemize}
Inteligência.
Faculdade de compreender.
\section{Intelectual}
\begin{itemize}
\item {Grp. gram.:adj.}
\end{itemize}
\begin{itemize}
\item {Proveniência:(Lat. \textunderscore intellectualis\textunderscore )}
\end{itemize}
Relativo ao intelecto.
Que tem inteligência culta: \textunderscore as classes intelectuaes\textunderscore .
Que tem dotes de inteligência.
\section{Intelectualidade}
\begin{itemize}
\item {Grp. gram.:f.}
\end{itemize}
\begin{itemize}
\item {Proveniência:(Lat. \textunderscore intellectualitas\textunderscore )}
\end{itemize}
O mesmo que \textunderscore intelecto\textunderscore .
Conjunto das faculdades intelectuaes.
Qualidade de intelectual.
\section{Intelectualizar}
\begin{itemize}
\item {Grp. gram.:v. t.}
\end{itemize}
\begin{itemize}
\item {Proveniência:(De \textunderscore intelectual\textunderscore )}
\end{itemize}
Elevar á categoria de coisas intelectuaes.
\section{Intelectualmente}
\begin{itemize}
\item {Grp. gram.:adv.}
\end{itemize}
De modo intelectual.
Com inteligência.
\section{Inteligência}
\begin{itemize}
\item {Grp. gram.:f.}
\end{itemize}
\begin{itemize}
\item {Utilização:Fig.}
\end{itemize}
\begin{itemize}
\item {Proveniência:(Lat. \textunderscore intelligentia\textunderscore )}
\end{itemize}
Qualidade de inteligente.
Faculdade de compreender.
Compreensão fácil: \textunderscore o rapaz tem inteligência\textunderscore .
O espírito que compreende ou concebe.
Substância espiritual, considerada como fonte de todos os conhecimentos: \textunderscore aplicar a inteligência\textunderscore .
Pessôa de grande esfera intelectual.
Acto de conhecer, de interpretar: \textunderscore a inteligência das coisas\textunderscore .
Conluío.
Uniformidade de sentimentos.
\section{Inteligente}
\begin{itemize}
\item {Grp. gram.:adj.}
\end{itemize}
\begin{itemize}
\item {Grp. gram.:M.}
\end{itemize}
\begin{itemize}
\item {Proveniência:(Lat. \textunderscore intelligens\textunderscore )}
\end{itemize}
Que tem a faculdade de perceber ou compreender; que compreende facilmente.
Hábil, dextro.
Director de toiradas.
\section{Inteligentemente}
\begin{itemize}
\item {Grp. gram.:adv.}
\end{itemize}
De modo inteligente.
\section{Inteligibilidade}
\begin{itemize}
\item {Grp. gram.:f.}
\end{itemize}
Qualidade daquilo que é inteligível.
\section{Inteligível}
\begin{itemize}
\item {Grp. gram.:adj.}
\end{itemize}
\begin{itemize}
\item {Grp. gram.:M.}
\end{itemize}
\begin{itemize}
\item {Proveniência:(Lat. \textunderscore intelligibilis\textunderscore )}
\end{itemize}
Que se póde entender; que se compreende facilmente.
Aquilo que é inteligível ou que diz respeito á inteligência.
\section{Inteligivelmente}
\begin{itemize}
\item {Grp. gram.:adv.}
\end{itemize}
De modo inteligível.
\section{Intellecção}
\begin{itemize}
\item {Grp. gram.:f.}
\end{itemize}
\begin{itemize}
\item {Proveniência:(Lat. \textunderscore intellectio\textunderscore )}
\end{itemize}
Acto de entender. Cf. \textunderscore Luz e Calor\textunderscore , 32.
\section{Intellectivamente}
\begin{itemize}
\item {Grp. gram.:adv.}
\end{itemize}
De modo intellectivo.
\section{Intellectível}
\begin{itemize}
\item {Grp. gram.:adj.}
\end{itemize}
\begin{itemize}
\item {Proveniência:(Lat. \textunderscore intellectivus\textunderscore )}
\end{itemize}
Relativo á intelligência; intellectual.
\section{Intellectivo}
\begin{itemize}
\item {Grp. gram.:adj.}
\end{itemize}
\begin{itemize}
\item {Proveniência:(Lat. \textunderscore intellectivus\textunderscore )}
\end{itemize}
Relativo á intelligência; intellectual.
\section{Intellecto}
\begin{itemize}
\item {Grp. gram.:m.}
\end{itemize}
\begin{itemize}
\item {Proveniência:(Lat. \textunderscore intellectus\textunderscore )}
\end{itemize}
Intelligência.
Faculdade de comprehender.
\section{Intellectual}
\begin{itemize}
\item {Grp. gram.:adj.}
\end{itemize}
\begin{itemize}
\item {Proveniência:(Lat. \textunderscore intellectualis\textunderscore )}
\end{itemize}
Relativo ao intellecto.
Que tem intelligência culta: \textunderscore as classes intellectuaes\textunderscore .
Que tem dotes de intelligência.
\section{Intellectualidade}
\begin{itemize}
\item {Grp. gram.:f.}
\end{itemize}
\begin{itemize}
\item {Proveniência:(Lat. \textunderscore intellectualitas\textunderscore )}
\end{itemize}
O mesmo que \textunderscore intellecto\textunderscore .
Conjunto das faculdades intellectuaes.
Qualidade de intellectual.
\section{Intellectualizar}
\begin{itemize}
\item {Grp. gram.:v. t.}
\end{itemize}
\begin{itemize}
\item {Proveniência:(De \textunderscore intellectual\textunderscore )}
\end{itemize}
Elevar á categoria de coisas intellectuaes.
\section{Intellectualmente}
\begin{itemize}
\item {Grp. gram.:adv.}
\end{itemize}
De modo intellectual.
Com intelligência.
\section{Intelligência}
\begin{itemize}
\item {Grp. gram.:f.}
\end{itemize}
\begin{itemize}
\item {Utilização:Fig.}
\end{itemize}
\begin{itemize}
\item {Proveniência:(Lat. \textunderscore intelligentia\textunderscore )}
\end{itemize}
Qualidade de intelligente.
Faculdade de comprehender.
Comprehensão fácil: \textunderscore o rapaz tem intelligência\textunderscore .
O espírito que comprehende ou concebe.
Substância espiritual, considerada como fonte de todos os conhecimentos: \textunderscore applicar a intelligência\textunderscore .
Pessôa de grande esphera intellectual.
Acto de conhecer, de interpretar: \textunderscore a intelligência das coisas\textunderscore .
Conluío.
Uniformidade de sentimentos.
\section{Intelligente}
\begin{itemize}
\item {Grp. gram.:adj.}
\end{itemize}
\begin{itemize}
\item {Grp. gram.:M.}
\end{itemize}
\begin{itemize}
\item {Proveniência:(Lat. \textunderscore intelligens\textunderscore )}
\end{itemize}
Que tem a faculdade de perceber ou comprehender; que comprehende facilmente.
Hábil, dextro.
Director de toiradas.
\section{Intelligentemente}
\begin{itemize}
\item {Grp. gram.:adv.}
\end{itemize}
De modo intelligente.
\section{Intelligibilidade}
\begin{itemize}
\item {Grp. gram.:f.}
\end{itemize}
Qualidade daquillo que é intelligível.
\section{Intelligível}
\begin{itemize}
\item {Grp. gram.:adj.}
\end{itemize}
\begin{itemize}
\item {Grp. gram.:M.}
\end{itemize}
\begin{itemize}
\item {Proveniência:(Lat. \textunderscore intelligibilis\textunderscore )}
\end{itemize}
Que se póde entender; que se comprehende facilmente.
Aquillo que é intelligível ou que diz respeito á intelligência.
\section{Intelligivelmente}
\begin{itemize}
\item {Grp. gram.:adv.}
\end{itemize}
De modo intelligível.
\section{Intemente}
\begin{itemize}
\item {Grp. gram.:adj.}
\end{itemize}
\begin{itemize}
\item {Proveniência:(De \textunderscore in...\textunderscore  + \textunderscore temente\textunderscore )}
\end{itemize}
Que não é temente; que não teme.
\section{Intemerato}
\begin{itemize}
\item {Grp. gram.:adj.}
\end{itemize}
\begin{itemize}
\item {Proveniência:(Lat. \textunderscore intemeratus\textunderscore )}
\end{itemize}
Íntegro, incorruptivel: \textunderscore virgem intemerata\textunderscore .--No sentido de \textunderscore valente\textunderscore , \textunderscore corajoso\textunderscore , vê-se amiúde empregado o voc., mas é êrro crasso, não obstante os exemplos de Camillo, como na \textunderscore Bohêmia do Espírito\textunderscore , 213.
V. \textunderscore intimorato\textunderscore , com que se não póde confundir \textunderscore intemerato\textunderscore .
\section{Intemperadamente}
\begin{itemize}
\item {Grp. gram.:adv.}
\end{itemize}
De modo intemperado; abusivamente; desregradamente.
\section{Intemperado}
\begin{itemize}
\item {Grp. gram.:adj.}
\end{itemize}
\begin{itemize}
\item {Proveniência:(Lat. \textunderscore intemperatus\textunderscore )}
\end{itemize}
Que não tem temperança; immoderado.
\section{Intemperança}
\begin{itemize}
\item {Grp. gram.:f.}
\end{itemize}
\begin{itemize}
\item {Proveniência:(Lat. \textunderscore intemperantia\textunderscore )}
\end{itemize}
Falta de temperança.
Glutonaria.
Hábito ou vício de comer e beber excessivamente.
\section{Intemperante}
\begin{itemize}
\item {Grp. gram.:adj.}
\end{itemize}
\begin{itemize}
\item {Proveniência:(Lat. \textunderscore intemperans\textunderscore )}
\end{itemize}
Que não é sóbrio.
Immoderado.
Dissoluto.
\section{Intempérie}
\begin{itemize}
\item {Grp. gram.:f.}
\end{itemize}
\begin{itemize}
\item {Utilização:Ant.}
\end{itemize}
\begin{itemize}
\item {Proveniência:(Lat. \textunderscore intemperies\textunderscore )}
\end{itemize}
Falta de bôa temperatura; mau tempo: \textunderscore a intempérie das estações\textunderscore .
Desarranjo ou má constituição dos humores orgânicos.
\section{Intempestivamente}
\begin{itemize}
\item {Grp. gram.:adv.}
\end{itemize}
De modo intempestivo.
Inopportunamente; fóra do tempo próprio; prematuramente.
\section{Intempestividade}
\begin{itemize}
\item {Grp. gram.:f.}
\end{itemize}
\begin{itemize}
\item {Proveniência:(Lat. \textunderscore intempestivitas\textunderscore )}
\end{itemize}
Qualidade daquillo que é intempestivo.
\section{Intempestivo}
\begin{itemize}
\item {Grp. gram.:adj.}
\end{itemize}
\begin{itemize}
\item {Utilização:Fig.}
\end{itemize}
\begin{itemize}
\item {Proveniência:(Lat. \textunderscore intempestivus\textunderscore )}
\end{itemize}
Que não é feito em tempo próprio ou conveniente; inopportuno.
Súbito, inopinado; prematuro.
\section{Intenção}
\begin{itemize}
\item {Grp. gram.:f.}
\end{itemize}
\begin{itemize}
\item {Proveniência:(Lat. \textunderscore intentio\textunderscore )}
\end{itemize}
Acto de tender.
Intento.
Movimento da alma para determinado fim.
Vontade; desejo.
Propósito; pensamento: \textunderscore de bôas intenções está o inferno cheio\textunderscore .
\textunderscore Segunda intenção\textunderscore , conceito reservado; ideia, que se subentende.
\section{Intencionado}
\begin{itemize}
\item {Grp. gram.:adj.}
\end{itemize}
\begin{itemize}
\item {Proveniência:(Do lat. \textunderscore intentio\textunderscore )}
\end{itemize}
Em que há certa intenção: \textunderscore indivíduo mal intencionado\textunderscore .
\section{Intencional}
\begin{itemize}
\item {Grp. gram.:adj.}
\end{itemize}
\begin{itemize}
\item {Proveniência:(Do lat. \textunderscore intentio\textunderscore )}
\end{itemize}
Relativo a intenção.
Propositado: \textunderscore offensa intencional\textunderscore .
\section{Intencionalidade}
\begin{itemize}
\item {Grp. gram.:f.}
\end{itemize}
Qualidade de que é intencional.
\section{Intencionalmente}
\begin{itemize}
\item {Grp. gram.:adv.}
\end{itemize}
De modo intencional.
Propositadamente.
\section{Intencionável}
\begin{itemize}
\item {Grp. gram.:adj.}
\end{itemize}
O mesmo que \textunderscore intencional\textunderscore .
\section{Intencionista}
\begin{itemize}
\item {Grp. gram.:m.  e  adj.}
\end{itemize}
\begin{itemize}
\item {Proveniência:(Do lat. \textunderscore intentio\textunderscore )}
\end{itemize}
Sectário da opinião de que não há acto válido, não sendo feito com intenção.
\section{Intendência}
\begin{itemize}
\item {Grp. gram.:f.}
\end{itemize}
\begin{itemize}
\item {Proveniência:(De \textunderscore intender\textunderscore )}
\end{itemize}
Direcção ou cargo de intendente.
Repartições ou edifício, em que o intendente exerce suas funcções.
\section{Intendente}
\begin{itemize}
\item {Grp. gram.:adj.}
\end{itemize}
\begin{itemize}
\item {Utilização:Des.}
\end{itemize}
\begin{itemize}
\item {Grp. gram.:M.}
\end{itemize}
\begin{itemize}
\item {Proveniência:(Lat. \textunderscore intendens\textunderscore )}
\end{itemize}
Que intende.
Aquelle que dirige ou administra alguma coisa: \textunderscore o intendente do Arsenal\textunderscore .
Antigo magistrado superior da polícia: \textunderscore o intendente Manique...\textunderscore 
\section{Intender}
\begin{itemize}
\item {Grp. gram.:v. i.}
\end{itemize}
\begin{itemize}
\item {Proveniência:(Lat. \textunderscore intendere\textunderscore )}
\end{itemize}
Superintender; exercer vigilância.
Cp. \textunderscore entender\textunderscore .
\section{Intensamente}
\begin{itemize}
\item {Grp. gram.:adv.}
\end{itemize}
De modo intenso; com intensidade.
\section{Intensão}
\begin{itemize}
\item {Grp. gram.:f.}
\end{itemize}
\begin{itemize}
\item {Proveniência:(Lat. \textunderscore intensio\textunderscore )}
\end{itemize}
Acto de intensar.
\section{Intensar}
\begin{itemize}
\item {Grp. gram.:v. t.}
\end{itemize}
Tornar intenso; aumentar a tensão de.
\section{Intensidade}
\begin{itemize}
\item {Grp. gram.:f.}
\end{itemize}
Qualidade daquillo que é intenso; grau elevado: \textunderscore a intensidade do calor\textunderscore .
\section{Intensificação}
\begin{itemize}
\item {Grp. gram.:f.}
\end{itemize}
Acto ou effeito de intensificar.
\section{Intensificar}
\begin{itemize}
\item {Grp. gram.:v. t.}
\end{itemize}
\begin{itemize}
\item {Proveniência:(Do lat. \textunderscore intensus\textunderscore  + \textunderscore facere\textunderscore )}
\end{itemize}
Tornar intenso.
\section{Intensivamente}
\begin{itemize}
\item {Grp. gram.:adv.}
\end{itemize}
De modo intensivo.
\section{Intensivo}
\begin{itemize}
\item {Grp. gram.:adj.}
\end{itemize}
\begin{itemize}
\item {Proveniência:(De \textunderscore intenso\textunderscore )}
\end{itemize}
Que dá intensão.
Que tem intensidade.
Que dá mais fôrça a uma expressão, a uma palavra, a uma ideia: \textunderscore inflexão intensiva\textunderscore .
Em que se accumulam esforços ou meios.
\section{Intenso}
\begin{itemize}
\item {Grp. gram.:adj.}
\end{itemize}
\begin{itemize}
\item {Proveniência:(Lat. \textunderscore intensus\textunderscore )}
\end{itemize}
Que tem muita tensão; vehemente; activo; enérgico.
\section{Intentado}
\begin{itemize}
\item {Grp. gram.:adj.}
\end{itemize}
\begin{itemize}
\item {Proveniência:(De \textunderscore in...\textunderscore  + \textunderscore tentado\textunderscore )}
\end{itemize}
Não tentado, não experimentado. Cf. Rui Barb., \textunderscore Réplica\textunderscore , 157.
\section{Intentar}
\begin{itemize}
\item {Grp. gram.:v. t.}
\end{itemize}
\begin{itemize}
\item {Proveniência:(Lat. \textunderscore intentare\textunderscore )}
\end{itemize}
Têr o intento de.
Planear: \textunderscore intentar desforrar-se\textunderscore .
Comprehender.
Formular.
\section{Intento}
\begin{itemize}
\item {Grp. gram.:m.}
\end{itemize}
\begin{itemize}
\item {Utilização:Des.}
\end{itemize}
\begin{itemize}
\item {Proveniência:(Lat. \textunderscore intentus\textunderscore )}
\end{itemize}
Tenção; intenção.
Plano; desígnio; propósito.
Attenção.
\section{Intentona}
\begin{itemize}
\item {Grp. gram.:f.}
\end{itemize}
\begin{itemize}
\item {Utilização:Pop.}
\end{itemize}
Plano insensato; intento insano.
Ataque ou assalto imprevisto.
Conluio de motim ou revolta.
(Cast. \textunderscore intentona\textunderscore )
\section{Inter...}
\begin{itemize}
\item {Grp. gram.:pref.}
\end{itemize}
\begin{itemize}
\item {Proveniência:(Lat. \textunderscore inter\textunderscore )}
\end{itemize}
(designativo de \textunderscore entre\textunderscore , \textunderscore em meio\textunderscore , \textunderscore dentro\textunderscore )
\section{Interaçoreano}
\begin{itemize}
\item {Grp. gram.:adj.}
\end{itemize}
\begin{itemize}
\item {Proveniência:(De \textunderscore inter...\textunderscore  + \textunderscore açoreano\textunderscore )}
\end{itemize}
Que se realiza de ilha para ilha, nos Açores.
Relativo ás relações que há ou póde haver entre as ilhas dos Açores.
\section{Interamnense}
\begin{itemize}
\item {Grp. gram.:adj.}
\end{itemize}
\begin{itemize}
\item {Proveniência:(Lat. \textunderscore interamnensis\textunderscore )}
\end{itemize}
Que vive entre rios.
Relativo á região de entre Doiro e Minho.
\section{Interanular}
\begin{itemize}
\item {Grp. gram.:adj.}
\end{itemize}
\begin{itemize}
\item {Proveniência:(De \textunderscore inter...\textunderscore  + \textunderscore anular\textunderscore )}
\end{itemize}
Situado entre anéis.
\section{Interarticular}
\begin{itemize}
\item {Grp. gram.:adj.}
\end{itemize}
\begin{itemize}
\item {Proveniência:(De \textunderscore inter...\textunderscore  + \textunderscore articular\textunderscore )}
\end{itemize}
Situado entre articulações.
\section{Intercadência}
\begin{itemize}
\item {Grp. gram.:f.}
\end{itemize}
\begin{itemize}
\item {Proveniência:(De \textunderscore inter...\textunderscore  + \textunderscore cadência\textunderscore )}
\end{itemize}
Falta de continuidade.
Perturbação nos movimentos.
Movimento irregular do pulso.
O mesmo que \textunderscore intercorrência\textunderscore , (falando-se do tempo).
\section{Intercadente}
\begin{itemize}
\item {Grp. gram.:adj.}
\end{itemize}
\begin{itemize}
\item {Proveniência:(De \textunderscore inter...\textunderscore  + \textunderscore cadente\textunderscore )}
\end{itemize}
Irregular; interrupto; alternado.
\section{Intercalação}
\begin{itemize}
\item {Grp. gram.:f.}
\end{itemize}
\begin{itemize}
\item {Proveniência:(Lat. \textunderscore intercalatio\textunderscore )}
\end{itemize}
Acto ou effeito de intercalar^1.
\section{Intercalar}
\begin{itemize}
\item {Grp. gram.:v. t.}
\end{itemize}
\begin{itemize}
\item {Proveniência:(Lat. \textunderscore intercalare\textunderscore )}
\end{itemize}
Interpor, pôr de permeio: \textunderscore intercalar fôlhas brancas num livro\textunderscore .
Inserir: \textunderscore intercalar uma notícia no jornal\textunderscore .
\section{Intercalar}
\begin{itemize}
\item {Grp. gram.:adj.}
\end{itemize}
\begin{itemize}
\item {Proveniência:(Lat. \textunderscore intercalaris\textunderscore )}
\end{itemize}
Que se intercala: \textunderscore notas intercalares\textunderscore .
\section{Interceder}
\begin{itemize}
\item {Grp. gram.:v. i.}
\end{itemize}
\begin{itemize}
\item {Proveniência:(Lat. \textunderscore intercedere\textunderscore )}
\end{itemize}
Pedir a favor de alguém.
Intervir.
Sêr intermediário a favor de outrem.
\section{Intercellular}
\begin{itemize}
\item {Grp. gram.:adj.}
\end{itemize}
\begin{itemize}
\item {Proveniência:(De \textunderscore inter...\textunderscore  + \textunderscore cellular\textunderscore )}
\end{itemize}
Que está entre as céllulas.
\section{Intercelular}
\begin{itemize}
\item {Grp. gram.:adj.}
\end{itemize}
\begin{itemize}
\item {Proveniência:(De \textunderscore inter...\textunderscore  + \textunderscore celular\textunderscore )}
\end{itemize}
Que está entre as células.
\section{Intercepção}
\begin{itemize}
\item {Grp. gram.:f.}
\end{itemize}
\begin{itemize}
\item {Proveniência:(Lat. \textunderscore interceptio\textunderscore )}
\end{itemize}
Acto ou effeito de interceptar.
\section{Interceptação}
\begin{itemize}
\item {Grp. gram.:f.}
\end{itemize}
\begin{itemize}
\item {Proveniência:(De \textunderscore interceptar\textunderscore )}
\end{itemize}
(V.intercepção)
\section{Interceptar}
\begin{itemize}
\item {Grp. gram.:v. t.}
\end{itemize}
\begin{itemize}
\item {Proveniência:(De \textunderscore intercepto\textunderscore )}
\end{itemize}
Pôr obstáculo entre ou no meio de.
Interromper.
Impedir o curso de: \textunderscore interceptar correspondências\textunderscore .
Cortar.
\section{Intercepto}
\begin{itemize}
\item {Grp. gram.:adj.}
\end{itemize}
\begin{itemize}
\item {Proveniência:(Lat. \textunderscore interceptus\textunderscore )}
\end{itemize}
Que se interceptou.
Interrompido.
\section{Interceptor}
\begin{itemize}
\item {Grp. gram.:m.  e  adj.}
\end{itemize}
O que intercepta.
\section{Intercervical}
\begin{itemize}
\item {Grp. gram.:adj.}
\end{itemize}
\begin{itemize}
\item {Proveniência:(De \textunderscore inter...\textunderscore  + \textunderscore cervical\textunderscore )}
\end{itemize}
Situado entre as vértebras cervicaes.
\section{Intercessão}
\begin{itemize}
\item {Grp. gram.:f.}
\end{itemize}
\begin{itemize}
\item {Proveniência:(Lat. \textunderscore intercessio\textunderscore )}
\end{itemize}
Acto de interceder.
\section{Intercessor}
\begin{itemize}
\item {Grp. gram.:m.  e  adj.}
\end{itemize}
\begin{itemize}
\item {Proveniência:(Lat. \textunderscore intercessor\textunderscore )}
\end{itemize}
O que intercede.
\section{Interchondral}
\begin{itemize}
\item {fónica:con}
\end{itemize}
\begin{itemize}
\item {Grp. gram.:adj.}
\end{itemize}
\begin{itemize}
\item {Proveniência:(De \textunderscore inter...\textunderscore  + gr. \textunderscore khondros\textunderscore )}
\end{itemize}
Situado entre cartilagens.
\section{Intercílio}
\begin{itemize}
\item {Grp. gram.:m.}
\end{itemize}
\begin{itemize}
\item {Proveniência:(Lat. \textunderscore intercilium\textunderscore )}
\end{itemize}
Espaço, entre as duas sobrancelhas.
\section{Interciso}
\begin{itemize}
\item {Grp. gram.:adj.}
\end{itemize}
\begin{itemize}
\item {Proveniência:(Lat. \textunderscore intercisus\textunderscore )}
\end{itemize}
Truncado.
Cortado ao meio.
Retalhado.
Dizia-se dos dias, em que, antes do calendário juliano, se consagrava metade ao culto religioso. Cf. Castilho, \textunderscore Fastos\textunderscore , I, 520 e 287.
\section{Interclavicular}
\begin{itemize}
\item {Grp. gram.:adj.}
\end{itemize}
\begin{itemize}
\item {Proveniência:(De \textunderscore inter...\textunderscore  + \textunderscore clavicular\textunderscore )}
\end{itemize}
Situado entre as clavículas.
\section{Intercolonial}
\begin{itemize}
\item {Grp. gram.:adj.}
\end{itemize}
\begin{itemize}
\item {Proveniência:(De \textunderscore inter...\textunderscore  + \textunderscore colonial\textunderscore )}
\end{itemize}
Que se faz de colónia para colónia: \textunderscore commércio intercolonial\textunderscore .
\section{Intercolumnar}
\begin{itemize}
\item {Grp. gram.:adj.}
\end{itemize}
\begin{itemize}
\item {Proveniência:(De \textunderscore inter...\textunderscore  + \textunderscore columnar\textunderscore )}
\end{itemize}
Relativo ao intercolúmnio.
\section{Intercolúmnio}
\begin{itemize}
\item {Grp. gram.:m.}
\end{itemize}
\begin{itemize}
\item {Proveniência:(Lat. \textunderscore intercolumnium\textunderscore )}
\end{itemize}
Espaço, entre columnas.
\section{Intercolunar}
\begin{itemize}
\item {Grp. gram.:adj.}
\end{itemize}
\begin{itemize}
\item {Proveniência:(De \textunderscore inter...\textunderscore  + \textunderscore colunar\textunderscore )}
\end{itemize}
Relativo ao intercolúnio.
\section{Intercolúnio}
\begin{itemize}
\item {Grp. gram.:m.}
\end{itemize}
\begin{itemize}
\item {Proveniência:(Lat. \textunderscore intercolumnium\textunderscore )}
\end{itemize}
Espaço, entre colunas.
\section{Intercondral}
\begin{itemize}
\item {Grp. gram.:adj.}
\end{itemize}
\begin{itemize}
\item {Proveniência:(De \textunderscore inter...\textunderscore  + gr. \textunderscore khondros\textunderscore )}
\end{itemize}
Situado entre cartilagens.
\section{Intercontinental}
\begin{itemize}
\item {Grp. gram.:adj.}
\end{itemize}
\begin{itemize}
\item {Proveniência:(De \textunderscore inter...\textunderscore  + \textunderscore continental\textunderscore )}
\end{itemize}
Relativo a dois ou mais continentes.
Situado entre continentes.
Que se faz de continente para continente.
\section{Intercorrência}
\begin{itemize}
\item {Grp. gram.:f.}
\end{itemize}
Qualidade daquillo que é intercorrente.
\section{Intercorrente}
\begin{itemize}
\item {Grp. gram.:adj.}
\end{itemize}
\begin{itemize}
\item {Proveniência:(Lat. \textunderscore intercurrens\textunderscore )}
\end{itemize}
Que se mete de permeio.
Que sobrevém.
Irregular, (falando-se do pulso).
\section{Intercorrentemente}
\begin{itemize}
\item {Grp. gram.:adv.}
\end{itemize}
De modo intercorrente.
\section{Intercorrer}
\begin{itemize}
\item {Grp. gram.:v. i.}
\end{itemize}
\begin{itemize}
\item {Proveniência:(Lat. \textunderscore intercurrere\textunderscore )}
\end{itemize}
Correr pelo meio ou no interior:«\textunderscore ...o rio que intercorre pela cidade...\textunderscore »Filinto, \textunderscore D. Man.\textunderscore , II, 42.
Succeder entretanto, sobrevir.
\section{Intercósmico}
\begin{itemize}
\item {Grp. gram.:adj.}
\end{itemize}
\begin{itemize}
\item {Proveniência:(De \textunderscore inter...\textunderscore  + \textunderscore cósmico\textunderscore )}
\end{itemize}
Que está ou se move entre grandes corpos celestes.
\section{Intercostal}
\begin{itemize}
\item {Grp. gram.:adj.}
\end{itemize}
\begin{itemize}
\item {Proveniência:(Lat. \textunderscore intercostalis\textunderscore )}
\end{itemize}
Situado entre as costellas.
\section{Intercurso}
\begin{itemize}
\item {Grp. gram.:m.}
\end{itemize}
\begin{itemize}
\item {Proveniência:(Lat. \textunderscore intercursus\textunderscore )}
\end{itemize}
Communicação, trato.
\section{Intercutâneo}
\begin{itemize}
\item {Grp. gram.:adj.}
\end{itemize}
\begin{itemize}
\item {Proveniência:(Lat. \textunderscore intercutaneus\textunderscore )}
\end{itemize}
O mesmo que o \textunderscore subcutâneo\textunderscore .
\section{Interdepender}
\begin{itemize}
\item {Grp. gram.:v. i.}
\end{itemize}
\begin{itemize}
\item {Utilização:Bras}
\end{itemize}
\begin{itemize}
\item {Utilização:Neol.}
\end{itemize}
\begin{itemize}
\item {Proveniência:(De \textunderscore inter...\textunderscore  + \textunderscore depender\textunderscore )}
\end{itemize}
Depender reciprocamente.
\section{Interdição}
\begin{itemize}
\item {Grp. gram.:f.}
\end{itemize}
\begin{itemize}
\item {Proveniência:(Lat. \textunderscore interdictio\textunderscore )}
\end{itemize}
Acto de interdizer.
Proibição.
Acto de privar judicialmente alguém do direito de reger sua pessôa e bens.
\section{Interdicção}
\begin{itemize}
\item {Grp. gram.:f.}
\end{itemize}
\begin{itemize}
\item {Proveniência:(Lat. \textunderscore interdictio\textunderscore )}
\end{itemize}
Acto de interdizer.
Prohibição.
Acto de privar judicialmente alguém do direito de reger sua pessôa e bens.
\section{Interdictar}
\begin{itemize}
\item {Grp. gram.:v. t.}
\end{itemize}
Pronunciar interdicto^2 contra; tornar interdicto^1. Cf. Camillo, \textunderscore Cav. em Ruinas\textunderscore , 120.
\section{Interdicto}
\begin{itemize}
\item {Grp. gram.:adj.}
\end{itemize}
\begin{itemize}
\item {Grp. gram.:M.}
\end{itemize}
\begin{itemize}
\item {Proveniência:(Lat. \textunderscore interdictus\textunderscore )}
\end{itemize}
Prohíbido.
Que está privado de reger sua pessôa e bens.
Que não póde exercer as funcções que tínha.
Em que se não pódem celebrar actos religiosos, (falando-se da igreja ou lugar sagrado, em que houve profanação ou delicto).
Aquelle que foi privado judicialmente de reger sua pessôa ou bens.
\section{Interdicto}
\begin{itemize}
\item {Grp. gram.:m.}
\end{itemize}
\begin{itemize}
\item {Proveniência:(Lat. \textunderscore interdictum\textunderscore )}
\end{itemize}
O mesmo que \textunderscore interdicção\textunderscore .
Prohibição da administração dos sacramentos, dos offícios divinos e da sepultura ecclesiástica.
\section{Interdigital}
\begin{itemize}
\item {Grp. gram.:adj.}
\end{itemize}
\begin{itemize}
\item {Proveniência:(De \textunderscore inter...\textunderscore  + \textunderscore digital\textunderscore )}
\end{itemize}
Que está entre os dedos.
\section{Interditar}
\begin{itemize}
\item {Grp. gram.:v. t.}
\end{itemize}
Pronunciar interdito^2 contra; tornar interdito^1. Cf. Camillo, \textunderscore Cav. em Ruinas\textunderscore , 120.
\section{Interdito}
\begin{itemize}
\item {Grp. gram.:adj.}
\end{itemize}
\begin{itemize}
\item {Grp. gram.:M.}
\end{itemize}
\begin{itemize}
\item {Proveniência:(Lat. \textunderscore interdictus\textunderscore )}
\end{itemize}
Proíbido.
Que está privado de reger sua pessôa e bens.
Que não póde exercer as funções que tínha.
Em que se não pódem celebrar actos religiosos, (falando-se da igreja ou lugar sagrado, em que houve profanação ou delito).
Aquele que foi privado judicialmente de reger sua pessôa ou bens.
\section{Interdito}
\begin{itemize}
\item {Grp. gram.:m.}
\end{itemize}
\begin{itemize}
\item {Proveniência:(Lat. \textunderscore interdictum\textunderscore )}
\end{itemize}
O mesmo que \textunderscore interdição\textunderscore .
Proibição da administração dos sacramentos, dos ofícios divinos e da sepultura eclesiástica.
\section{Interdizer}
\begin{itemize}
\item {Grp. gram.:v. t.}
\end{itemize}
\begin{itemize}
\item {Proveniência:(Lat. \textunderscore interdicere\textunderscore )}
\end{itemize}
Impedir, prohibir.
Privar (alguém) da administração da sua pessôa e bens.
Prohibir ecclesiasticamente a celebração de offícios divinos e de outras solennidades em: \textunderscore interdizer uma igreja\textunderscore .
\section{Interés}
\begin{itemize}
\item {Grp. gram.:m.}
\end{itemize}
\begin{itemize}
\item {Utilização:Des.}
\end{itemize}
O mesmo que \textunderscore interésse\textunderscore . Cf. Filinto, XI, 137.
(Cast. \textunderscore interés\textunderscore )
\section{Interessadamente}
\begin{itemize}
\item {Grp. gram.:adv.}
\end{itemize}
De modo interessado; utilitariamente; com egoísmo.
\section{Interessado}
\begin{itemize}
\item {Grp. gram.:m.}
\end{itemize}
\begin{itemize}
\item {Proveniência:(De \textunderscore interessar\textunderscore )}
\end{itemize}
Aquelle que tem interesse em alguma coisa.
\section{Interessal}
\begin{itemize}
\item {Grp. gram.:adj.}
\end{itemize}
\begin{itemize}
\item {Utilização:Ant.}
\end{itemize}
O mesmo que \textunderscore interesseiro\textunderscore .
Relativo a legítimos interesses: \textunderscore ...pola glória interessal dos commércios\textunderscore . G. Vicente, \textunderscore Auto da Fama\textunderscore .
\section{Interessante}
\begin{itemize}
\item {Grp. gram.:adj.}
\end{itemize}
\begin{itemize}
\item {Proveniência:(De \textunderscore interessar\textunderscore )}
\end{itemize}
Que excita interesse ou attenção.
Importante; sympáthico: \textunderscore rapariga interessante\textunderscore .
Diz-se do estado da mulher grávida.
\section{Interessar}
\begin{itemize}
\item {Grp. gram.:v. t.}
\end{itemize}
\begin{itemize}
\item {Grp. gram.:V. i.}
\end{itemize}
\begin{itemize}
\item {Proveniência:(De \textunderscore interesse\textunderscore )}
\end{itemize}
Dar interesse a.
Lucrar.
Despertar a attenção de; excitar a curiosidade de: \textunderscore este romance interessa-me\textunderscore .
Attrahir, captar.
Ferir, (em cirurgia).
Auferir interesse ou proveito.
Sêr importante.
Agradar.
\section{Interésse}
\begin{itemize}
\item {Grp. gram.:m.}
\end{itemize}
\begin{itemize}
\item {Proveniência:(Lat. \textunderscore interesse\textunderscore )}
\end{itemize}
Vantagem; proveito; ganho.
Conveniência e sentimento egoísta de cubiça ou de utilidade pessoal.
Importância.
Attractivo.
Sympathia.
Empenho, grande diligência: \textunderscore pedir com interesse\textunderscore .
Juro de capital.
\section{Interêsse}
\begin{itemize}
\item {Grp. gram.:m.}
\end{itemize}
\begin{itemize}
\item {Proveniência:(Lat. \textunderscore interesse\textunderscore )}
\end{itemize}
Vantagem; proveito; ganho.
Conveniência e sentimento egoísta de cubiça ou de utilidade pessoal.
Importância.
Attractivo.
Sympathia.
Empenho, grande diligência: \textunderscore pedir com interesse\textunderscore .
Juro de capital.
\section{Interesseiro}
\begin{itemize}
\item {Grp. gram.:adj.}
\end{itemize}
\begin{itemize}
\item {Grp. gram.:M.}
\end{itemize}
\begin{itemize}
\item {Proveniência:(De \textunderscore interesse\textunderscore )}
\end{itemize}
Que attende só ao próprio interesse.
Egoísta.
Aquelle que cuida especialmente dos seus interesses, que é egoísta.
\section{Interessente}
\begin{itemize}
\item {Grp. gram.:adj.}
\end{itemize}
\begin{itemize}
\item {Utilização:Ant.}
\end{itemize}
\begin{itemize}
\item {Proveniência:(Do lat. \textunderscore interesse\textunderscore )}
\end{itemize}
Que está entre, que está no meio.
Distante.
\section{Interessículo}
\begin{itemize}
\item {Grp. gram.:m.}
\end{itemize}
Pequeno interesse. Cf. Castilho, \textunderscore Colloq. Ald.\textunderscore , 308.
\section{Interestadual}
\begin{itemize}
\item {Grp. gram.:adj.}
\end{itemize}
\begin{itemize}
\item {Utilização:bras}
\end{itemize}
\begin{itemize}
\item {Utilização:Neol.}
\end{itemize}
\begin{itemize}
\item {Proveniência:(De \textunderscore inter...\textunderscore  + \textunderscore estadual\textunderscore )}
\end{itemize}
Que se realiza de Estado para Estado.
Que diz respeito ás relações entre os Estados da República Federal.
\section{Interfemínio}
\begin{itemize}
\item {Grp. gram.:m.}
\end{itemize}
\begin{itemize}
\item {Proveniência:(Lat. \textunderscore interfeminium\textunderscore )}
\end{itemize}
Lugar, em que se unem as côxas femininas.
Partes pudendas da mulher.
\section{Interferência}
\begin{itemize}
\item {Grp. gram.:f.}
\end{itemize}
Intervenção.
Qualidade daquillo que é interferente.
\section{Interferente}
\begin{itemize}
\item {Grp. gram.:adj.}
\end{itemize}
\begin{itemize}
\item {Proveniência:(Do lat. \textunderscore inter\textunderscore  + \textunderscore ferens\textunderscore )}
\end{itemize}
Diz-se dos raios luminosos, que apresentam faixas alternadamente brilhantes e obscuras.
\section{Interferir}
\begin{itemize}
\item {Grp. gram.:v. t.}
\end{itemize}
\begin{itemize}
\item {Utilização:Phýs.}
\end{itemize}
\begin{itemize}
\item {Proveniência:(Do lat. \textunderscore inter\textunderscore  + \textunderscore ferre\textunderscore )}
\end{itemize}
Intervir.
Produzir interferência.
\section{Interfixo}
\begin{itemize}
\item {fónica:cso}
\end{itemize}
\begin{itemize}
\item {Grp. gram.:adj.}
\end{itemize}
\begin{itemize}
\item {Proveniência:(De \textunderscore inter\textunderscore  + \textunderscore fixo\textunderscore )}
\end{itemize}
Que tem ponto de apoio entre a potência e a resistência, (falando-se de alavancas).
\section{Interfoliação}
\begin{itemize}
\item {Grp. gram.:f.}
\end{itemize}
Acto ou effeito de interfoliar.
\section{Interfoliáceo}
\begin{itemize}
\item {Grp. gram.:adj.}
\end{itemize}
\begin{itemize}
\item {Utilização:Bot.}
\end{itemize}
\begin{itemize}
\item {Proveniência:(De \textunderscore inter...\textunderscore  + \textunderscore foliáceo\textunderscore )}
\end{itemize}
Diz-se das flôres, que nascem entre cada par de fôlhas oppostas.
\section{Interfoliado}
\begin{itemize}
\item {Grp. gram.:adj.}
\end{itemize}
Que tem entrefolhas.
\section{Interfoliar}
\begin{itemize}
\item {Grp. gram.:v. t.}
\end{itemize}
\begin{itemize}
\item {Utilização:P. us.}
\end{itemize}
\begin{itemize}
\item {Grp. gram.:Adj.}
\end{itemize}
\begin{itemize}
\item {Utilização:Bot.}
\end{itemize}
\begin{itemize}
\item {Proveniência:(Do lat. \textunderscore inter\textunderscore  + \textunderscore folium\textunderscore )}
\end{itemize}
Pôr entrefolhas em.
Diz-se da parte do vegetal, comprehendida entre duas fôlhas.
\section{Intergiversável}
\begin{itemize}
\item {Grp. gram.:adj.}
\end{itemize}
\begin{itemize}
\item {Proveniência:(De \textunderscore in...\textunderscore  + \textunderscore tergiversar\textunderscore )}
\end{itemize}
Que não póde tergiversar.
\section{Interglacial}
\begin{itemize}
\item {Grp. gram.:adj.}
\end{itemize}
\begin{itemize}
\item {Utilização:Geol.}
\end{itemize}
\begin{itemize}
\item {Proveniência:(De \textunderscore inter...\textunderscore  + \textunderscore glacial\textunderscore )}
\end{itemize}
Diz-se da phase geológica, comprehendida entre dois máximos de extensão glaciária.
\section{Interglaciário}
\begin{itemize}
\item {Grp. gram.:adj.}
\end{itemize}
\begin{itemize}
\item {Utilização:Geol.}
\end{itemize}
\begin{itemize}
\item {Proveniência:(De \textunderscore inter...\textunderscore  + \textunderscore glaciário\textunderscore )}
\end{itemize}
Que está entre dois períodos glaciários.
\section{Interglobular}
\begin{itemize}
\item {Grp. gram.:adj.}
\end{itemize}
\begin{itemize}
\item {Proveniência:(De \textunderscore inter...\textunderscore  + \textunderscore globular\textunderscore )}
\end{itemize}
Que está entre glóbulos.
\section{Interim}
\begin{itemize}
\item {fónica:ínterim}
\end{itemize}
\begin{itemize}
\item {Grp. gram.:m.}
\end{itemize}
\begin{itemize}
\item {Grp. gram.:Adv.}
\end{itemize}
\begin{itemize}
\item {Proveniência:(T. lat.)}
\end{itemize}
Qualidade daquillo que é interino; tempo intermédio.
Entretanto, entrementes. Cf. Filinto, \textunderscore D. Man.\textunderscore , 252; II, 129.
\section{Interinado}
\begin{itemize}
\item {Grp. gram.:adj.}
\end{itemize}
\begin{itemize}
\item {Proveniência:(De \textunderscore interino\textunderscore )}
\end{itemize}
Exercício de um encargo interino.
\section{Interinamente}
\begin{itemize}
\item {Grp. gram.:adv.}
\end{itemize}
De modo interino.
Provisoriamente.
\section{Interinidade}
\begin{itemize}
\item {Grp. gram.:f.}
\end{itemize}
Qualidade daquelle ou daquillo que é interino.
Interinado.
\section{Interino}
\begin{itemize}
\item {Grp. gram.:adj.}
\end{itemize}
\begin{itemize}
\item {Proveniência:(Do lat. \textunderscore interim\textunderscore )}
\end{itemize}
Provisório.
Passageiro.
Que exerce funcções provisórias, na falta ou impedimento do funccionário effectivo.
\section{Interinsular}
\begin{itemize}
\item {Grp. gram.:adj.}
\end{itemize}
\begin{itemize}
\item {Proveniência:(De \textunderscore inter...\textunderscore  + \textunderscore insular\textunderscore )}
\end{itemize}
Que se realiza, de ilha para ilha ou entre várias ilhas.
Relativo ás relações que há entre várias ilhas, especialmente quando são do mesmo archipélago.
\section{Interior}
\begin{itemize}
\item {Grp. gram.:adj.}
\end{itemize}
\begin{itemize}
\item {Grp. gram.:M.}
\end{itemize}
\begin{itemize}
\item {Proveniência:(Lat. \textunderscore interior\textunderscore )}
\end{itemize}
Que está dentro; interno.
Particular, íntimo.
Situado entre terras: \textunderscore cidades interiores\textunderscore .
Aquillo que está dentro.
Parte interna de um organismo, de uma construcção, de um país: \textunderscore o interior de uma casa\textunderscore .
Seio, coração.
Índole, tendência: \textunderscore homem de mau interior\textunderscore .
\textunderscore Ministro do interior\textunderscore , diz-se nalguns países, o minístro, que tem a seu cargo os negócios da administração interna ou continental de um Estado.
\section{Interioridade}
\begin{itemize}
\item {Grp. gram.:f.}
\end{itemize}
Qualidade ou estado daquillo que é interior.
\section{Interiormente}
\begin{itemize}
\item {Grp. gram.:adv.}
\end{itemize}
De modo interior.
No ínterior; no íntimo.
\section{Interjacente}
\begin{itemize}
\item {Grp. gram.:adj.}
\end{itemize}
\begin{itemize}
\item {Proveniência:(Lat. \textunderscore interjacens\textunderscore )}
\end{itemize}
Que está entre outras coisas; interposto.
\section{Interjeccional}
\begin{itemize}
\item {Grp. gram.:adj.}
\end{itemize}
\begin{itemize}
\item {Utilização:Gram.}
\end{itemize}
\begin{itemize}
\item {Proveniência:(Do lat. \textunderscore interjectio\textunderscore )}
\end{itemize}
Relativo á interjeição; que tem fórma de interjeição.
\section{Interjectivamente}
\begin{itemize}
\item {Grp. gram.:adv.}
\end{itemize}
De modo interjectivo.
\section{Interjectivo}
\begin{itemize}
\item {Grp. gram.:adj.}
\end{itemize}
\begin{itemize}
\item {Utilização:Gram.}
\end{itemize}
\begin{itemize}
\item {Proveniência:(Do lat. \textunderscore interjectus\textunderscore )}
\end{itemize}
Expresso por interjeição; que tem natureza de interjeição.
\section{Interjeição}
\begin{itemize}
\item {Grp. gram.:f.}
\end{itemize}
\begin{itemize}
\item {Proveniência:(Lat. \textunderscore interjectio\textunderscore )}
\end{itemize}
Palavra ou locução, que se solta instintivamente, para exprimir um sentimento súbito de dôr, de alegria, de repulsão, de admiração, etc.
Grito, produzido por um sentimento súbito; exclamação.
\section{Interjeicionar}
\begin{itemize}
\item {Grp. gram.:v. t.}
\end{itemize}
Exclamar:«\textunderscore Oh! interjeicionou compungidamente o monarcha.\textunderscore »Camillo, \textunderscore Brasileira\textunderscore , 120.
\section{Interlaçar}
\begin{itemize}
\item {Grp. gram.:v. t.}
\end{itemize}
O mesmo que \textunderscore entrelaçar\textunderscore . Cf. Garrett, \textunderscore Viagens\textunderscore .
\section{Interlinear}
\begin{itemize}
\item {Grp. gram.:adj.}
\end{itemize}
\begin{itemize}
\item {Proveniência:(De \textunderscore inter...\textunderscore  + \textunderscore linear\textunderscore )}
\end{itemize}
Que está entre linhas: \textunderscore additamentos interlineares\textunderscore .
Relativo a entrelinhas.
\section{Interlobular}
\begin{itemize}
\item {Grp. gram.:adj.}
\end{itemize}
\begin{itemize}
\item {Proveniência:(De \textunderscore inter...\textunderscore  + \textunderscore lobular\textunderscore )}
\end{itemize}
Que está entre lóbulos.
\section{Interlocução}
\begin{itemize}
\item {Grp. gram.:f.}
\end{itemize}
\begin{itemize}
\item {Proveniência:(Lat. \textunderscore interlocutio\textunderscore )}
\end{itemize}
Conversação, entre duas ou mais pessôas.
Interrupção de um discurso, pelo falar de novas personagens.
\section{Interlocutor}
\begin{itemize}
\item {Grp. gram.:m.}
\end{itemize}
\begin{itemize}
\item {Proveniência:(Do lat. \textunderscore interlocutus\textunderscore )}
\end{itemize}
Aquelle que fala com outro.
Aquelle que fala em nome de outros.
\section{Interlocutória}
\begin{itemize}
\item {Grp. gram.:f.}
\end{itemize}
\begin{itemize}
\item {Utilização:Jur.}
\end{itemize}
\begin{itemize}
\item {Proveniência:(De \textunderscore interlocutório\textunderscore )}
\end{itemize}
Despacho interlocutório.
\section{Interlocutoriamente}
\begin{itemize}
\item {Grp. gram.:adv.}
\end{itemize}
De modo interlocutório.
\section{Interlocutório}
\begin{itemize}
\item {Grp. gram.:adj.}
\end{itemize}
\begin{itemize}
\item {Utilização:Jur.}
\end{itemize}
\begin{itemize}
\item {Grp. gram.:M.}
\end{itemize}
\begin{itemize}
\item {Proveniência:(Do lat. \textunderscore interlocutus\textunderscore )}
\end{itemize}
Proferido no decurso do um pleito.
Despacho, proferido no decurso de um pleito.
\section{Interlúcido}
\begin{itemize}
\item {Grp. gram.:m.}
\end{itemize}
Intervallo lúcido. Cf. Filinto, III, 101.
\section{Interlúdio}
\begin{itemize}
\item {Grp. gram.:m.}
\end{itemize}
\begin{itemize}
\item {Utilização:bras}
\end{itemize}
\begin{itemize}
\item {Utilização:Neol.}
\end{itemize}
Prelúdio musical.
\section{Interlunar}
\begin{itemize}
\item {Grp. gram.:adj.}
\end{itemize}
Relativo ao interlúnio.
\section{Interlúnio}
\begin{itemize}
\item {Grp. gram.:m.}
\end{itemize}
\begin{itemize}
\item {Proveniência:(Lat. \textunderscore interlunium\textunderscore )}
\end{itemize}
Tempo, que decorre entre o momento em que a Lua, decrescendo, deixa de sêr vista, e aquelle em que ella reapparece.
Lua-nova.
\section{Intermaxilar}
\begin{itemize}
\item {fónica:csi}
\end{itemize}
\begin{itemize}
\item {Grp. gram.:adj.}
\end{itemize}
\begin{itemize}
\item {Proveniência:(De \textunderscore inter...\textunderscore  + \textunderscore maxilar\textunderscore )}
\end{itemize}
Que está entre os ossos das maxilas.
\section{Intermaxillar}
\begin{itemize}
\item {fónica:csi}
\end{itemize}
\begin{itemize}
\item {Grp. gram.:adj.}
\end{itemize}
\begin{itemize}
\item {Proveniência:(De \textunderscore inter...\textunderscore  + \textunderscore maxillar\textunderscore )}
\end{itemize}
Que está entre os ossos das maxillas.
\section{Intermedial}
\begin{itemize}
\item {Grp. gram.:adj.}
\end{itemize}
O mesmo que \textunderscore intermediário\textunderscore .
\section{Intermediar}
\begin{itemize}
\item {Grp. gram.:v. i.}
\end{itemize}
\begin{itemize}
\item {Grp. gram.:V. t.}
\end{itemize}
\begin{itemize}
\item {Proveniência:(De \textunderscore intermédio\textunderscore )}
\end{itemize}
Estar de permeio; interceder.
Intervir.
O mesmo que \textunderscore entremear\textunderscore . Cf. Aguiar, \textunderscore Proc. de Vin.\textunderscore , 133.
\section{Intermediariamente}
\begin{itemize}
\item {Grp. gram.:adv.}
\end{itemize}
De modo intermediário.
Com intervenção.
\section{Intermediário}
\begin{itemize}
\item {Grp. gram.:adj.}
\end{itemize}
\begin{itemize}
\item {Grp. gram.:M.}
\end{itemize}
\begin{itemize}
\item {Utilização:Phot.}
\end{itemize}
\begin{itemize}
\item {Proveniência:(De \textunderscore intermédio\textunderscore )}
\end{itemize}
Que está de permeio; intermédio.
Medianeiro.
Intervenção.
Caixilho supplementar, que se colloca no caixilho focal, quando êste é maior do que as chapas.
\section{Intermédio}
\begin{itemize}
\item {Grp. gram.:adj.}
\end{itemize}
\begin{itemize}
\item {Grp. gram.:M.}
\end{itemize}
\begin{itemize}
\item {Utilização:Des.}
\end{itemize}
\begin{itemize}
\item {Proveniência:(Lat. \textunderscore intermedius\textunderscore )}
\end{itemize}
Que está de permeio; que se interpôs.
Aquillo que estabelece communicação entre duas coisas.
Medianeiro.
Intervenção.
Pequena representação, no intervallo dos actos de uma peça theatral; entreacto.
O mesmo que \textunderscore entremês\textunderscore .
\section{Intermenstruação}
\begin{itemize}
\item {Grp. gram.:f.}
\end{itemize}
\begin{itemize}
\item {Proveniência:(De \textunderscore inter...\textunderscore  + \textunderscore menstruação\textunderscore )}
\end{itemize}
Intervallo entre os mênstruos.
\section{Intermenstrual}
\begin{itemize}
\item {Grp. gram.:adj.}
\end{itemize}
\begin{itemize}
\item {Proveniência:(De \textunderscore inter...\textunderscore  + \textunderscore menstrual\textunderscore )}
\end{itemize}
Relativo á intermenstruação.
\section{Intermênstruo}
\begin{itemize}
\item {Grp. gram.:m.}
\end{itemize}
\begin{itemize}
\item {Proveniência:(Lat. \textunderscore intermenstruum\textunderscore )}
\end{itemize}
Conjuncção da Lua-nova.
O mesmo que \textunderscore interlúnio\textunderscore .
\section{Intermeter}
\begin{itemize}
\item {Grp. gram.:v. t.}
\end{itemize}
\begin{itemize}
\item {Proveniência:(De \textunderscore inter...\textunderscore  + \textunderscore meter\textunderscore )}
\end{itemize}
Meter de permeio, intrometer:«\textunderscore ...intermetendo delongas\textunderscore ». Camillo, \textunderscore Caveira\textunderscore , 351.
\section{Interminável}
\begin{itemize}
\item {Grp. gram.:adj.}
\end{itemize}
\begin{itemize}
\item {Proveniência:(Lat. \textunderscore interminabilis\textunderscore )}
\end{itemize}
Que não tem termo.
Que não póde terminar.
Desmedido.
Enorme.
Infinito.
Que se prolonga, que dura muito: \textunderscore um discurso interminável\textunderscore .
\section{Interfalangeano}
\begin{itemize}
\item {Grp. gram.:adj.}
\end{itemize}
\begin{itemize}
\item {Utilização:Anat.}
\end{itemize}
\begin{itemize}
\item {Proveniência:(De \textunderscore inter...\textunderscore  + \textunderscore falangeano\textunderscore )}
\end{itemize}
Situado entre as falanges.
Diz-se especialmente da última articulação dos membros do cavalo. Cf. Leon, \textunderscore Arte de Ferrar\textunderscore , 31.
\section{Intermiar}
\begin{itemize}
\item {Grp. gram.:v. t.}
\end{itemize}
O mesmo que \textunderscore entremear\textunderscore . Cf. Garrett, \textunderscore Catão\textunderscore , 114; Filinto, VI, 310.
\section{Interminavelmente}
\begin{itemize}
\item {Grp. gram.:adv.}
\end{itemize}
De modo interminável.
Sem fim.
\section{Intérmino}
\begin{itemize}
\item {Grp. gram.:adj.}
\end{itemize}
\begin{itemize}
\item {Utilização:Poét.}
\end{itemize}
\begin{itemize}
\item {Proveniência:(Lat. \textunderscore interminus\textunderscore )}
\end{itemize}
O mesmo que \textunderscore interminável\textunderscore .
\section{Intermissão}
\begin{itemize}
\item {Grp. gram.:f.}
\end{itemize}
\begin{itemize}
\item {Proveniência:(Lat. \textunderscore intermissio\textunderscore )}
\end{itemize}
Acto ou effeito de intermittir.
Interrupção; intervallo.
\section{Intermisturar-se}
\begin{itemize}
\item {Grp. gram.:v. p.}
\end{itemize}
\begin{itemize}
\item {Proveniência:(De \textunderscore inter...\textunderscore  + \textunderscore misturar\textunderscore )}
\end{itemize}
Misturar-se reciprocamente.
Amalgamar-se com coisas semelhantes ou differentes:«\textunderscore intermisturavam-se as simplezas da vida campestre\textunderscore ». Castilho, \textunderscore Fastos\textunderscore , I, p. XLIV.
\section{Intermitência}
\begin{itemize}
\item {Grp. gram.:f.}
\end{itemize}
Qualidade de intermitente.
Interrupção rápida.
Intervalo nas pulsações, maior que o normal.
Intervalo em acessos febris ou noutras doenças, no qual o enfermo parece curado ou quási curado.
\section{Intermitente}
\begin{itemize}
\item {Grp. gram.:adj.}
\end{itemize}
\begin{itemize}
\item {Proveniência:(Lat. \textunderscore intermittens\textunderscore )}
\end{itemize}
Que intermite.
Que tem interrupções, paragens ou intervalos: \textunderscore febre intermitente\textunderscore .
Que apresenta suspensões ou intervalos desiguaes, (falando-se de pulsações).
\section{Intermitir}
\begin{itemize}
\item {Grp. gram.:v. i.}
\end{itemize}
\begin{itemize}
\item {Proveniência:(Lat. \textunderscore intermittere\textunderscore )}
\end{itemize}
Interromper-se.
Têr intercadências.
Manifestar-se por accessos irregulares, com intervalos.
\section{Intermittência}
\begin{itemize}
\item {Grp. gram.:f.}
\end{itemize}
Qualidade de intermittente.
Interrupção rápida.
Intervallo nas pulsações, maior que o normal.
Intervallo em accessos febris ou noutras doenças, no qual o enfermo parece curado ou quási curado.
\section{Intermittente}
\begin{itemize}
\item {Grp. gram.:adj.}
\end{itemize}
\begin{itemize}
\item {Proveniência:(Lat. \textunderscore intermittens\textunderscore )}
\end{itemize}
Que intermitte.
Que tem interrupções, paragens ou intervallos: \textunderscore febre intermittente\textunderscore .
Que apresenta suspensões ou intervallos desiguaes, (falando-se de pulsações).
\section{Intermittir}
\begin{itemize}
\item {Grp. gram.:v. i.}
\end{itemize}
\begin{itemize}
\item {Proveniência:(Lat. \textunderscore intermittere\textunderscore )}
\end{itemize}
Interromper-se.
Têr intercadências.
Manifestar-se por accessos irregulares, com intervallos.
\section{Intermostrar}
\begin{itemize}
\item {Grp. gram.:v. t.}
\end{itemize}
O mesmo que \textunderscore entremostrar\textunderscore . Cf. Castilho, \textunderscore Fastos\textunderscore , I, p. XLIII.
\section{Intermóvel}
\begin{itemize}
\item {Grp. gram.:adj.}
\end{itemize}
\begin{itemize}
\item {Proveniência:(De \textunderscore inter...\textunderscore  + \textunderscore móvel\textunderscore )}
\end{itemize}
O mesmo que \textunderscore interfixo\textunderscore .
\section{Intermúndio}
\begin{itemize}
\item {Grp. gram.:m.}
\end{itemize}
\begin{itemize}
\item {Utilização:Fig.}
\end{itemize}
\begin{itemize}
\item {Proveniência:(Lat. \textunderscore intermundium\textunderscore )}
\end{itemize}
Espaço entre muitos mundos, ou entre corpos celestes.
Ermo, solidão.
\section{Intermural}
\begin{itemize}
\item {Grp. gram.:adj.}
\end{itemize}
\begin{itemize}
\item {Proveniência:(Lat. \textunderscore intermuralis\textunderscore )}
\end{itemize}
Que está entre muros.
\section{Intermuscular}
\begin{itemize}
\item {Grp. gram.:adj.}
\end{itemize}
\begin{itemize}
\item {Proveniência:(De \textunderscore inter...\textunderscore  + \textunderscore muscular\textunderscore )}
\end{itemize}
Que está entre os músculos.
\section{Intermutável}
\begin{itemize}
\item {Grp. gram.:adj.}
\end{itemize}
\begin{itemize}
\item {Proveniência:(De \textunderscore inter...\textunderscore  + \textunderscore mutável\textunderscore )}
\end{itemize}
Diz-se dos instrumentos mecânicos, que se podem substituir reciprocamente.
\section{Internação}
\begin{itemize}
\item {Grp. gram.:f.}
\end{itemize}
Acto ou effeito de internar.
\section{Internacional}
\begin{itemize}
\item {Grp. gram.:adj.}
\end{itemize}
\begin{itemize}
\item {Grp. gram.:F.}
\end{itemize}
\begin{itemize}
\item {Proveniência:(De \textunderscore inter...\textunderscore  + \textunderscore nacional\textunderscore )}
\end{itemize}
Que se realiza entre nações ou de nação para nação: \textunderscore commercio internacional\textunderscore .
Relativo ás relações entre nações: \textunderscore Direito internacional\textunderscore .
Associação dos operários das diversas nações, no interesse da sua classe.
\section{Internacionalidade}
\begin{itemize}
\item {Grp. gram.:f.}
\end{itemize}
Qualidade de internacional.
\section{Internacionalismo}
\begin{itemize}
\item {Grp. gram.:m.}
\end{itemize}
Systema de política internacional.
Princípios da internacional.
\section{Internacionalista}
\begin{itemize}
\item {Grp. gram.:adj.}
\end{itemize}
\begin{itemize}
\item {Grp. gram.:M.}
\end{itemize}
\begin{itemize}
\item {Proveniência:(De \textunderscore internacional\textunderscore )}
\end{itemize}
Relativo ao internacionalismo.
Sectário do internacionalismo.
\section{Internacionalização}
\begin{itemize}
\item {Grp. gram.:f.}
\end{itemize}
Acto ou effeito de internacionalizar.
\section{Internacionalizar}
\begin{itemize}
\item {Grp. gram.:v. t.}
\end{itemize}
Tornar internacional.
Diffundir por várias nações.
Tornar commum a várias nações.
\section{Internacionalmente}
\begin{itemize}
\item {Grp. gram.:adv.}
\end{itemize}
De modo internacional.
De nações para nações.
\section{Internado}
\begin{itemize}
\item {Grp. gram.:m.}
\end{itemize}
\begin{itemize}
\item {Proveniência:(De \textunderscore internar\textunderscore )}
\end{itemize}
Indivíduo, internado num hospício, num collégio, etc.
Internato.
\section{Internal}
\begin{itemize}
\item {Grp. gram.:adj.}
\end{itemize}
\begin{itemize}
\item {Utilização:T. de Ceilão}
\end{itemize}
O mesmo que \textunderscore interno\textunderscore .
\section{Internamente}
\begin{itemize}
\item {Grp. gram.:adv.}
\end{itemize}
\begin{itemize}
\item {Proveniência:(De \textunderscore interno\textunderscore )}
\end{itemize}
O mesmo que \textunderscore interiormente\textunderscore .
\section{Internamento}
\begin{itemize}
\item {Grp. gram.:m.}
\end{itemize}
Acto ou effeito de internar.
\section{Internar}
\begin{itemize}
\item {Grp. gram.:v. t.}
\end{itemize}
\begin{itemize}
\item {Grp. gram.:V. p.}
\end{itemize}
\begin{itemize}
\item {Proveniência:(De \textunderscore interno\textunderscore )}
\end{itemize}
Pôr dentro.
Collocar dentro de um collégio, de um asilo, etc.: \textunderscore internei o pequeno na Escola Acadêmica\textunderscore .
Forçar a residir no interior de um país: \textunderscore fez internar os conspiradores da fronteira\textunderscore .
Introduzir.
Introduzir-se, entranhar-se.
\section{Internato}
\begin{itemize}
\item {Grp. gram.:m.}
\end{itemize}
\begin{itemize}
\item {Proveniência:(De \textunderscore internar\textunderscore )}
\end{itemize}
Estabelecimento de educação ou caridade, em que vivem os alumnos ou em que se dá asylo aos necessitados.
\section{Interno}
\begin{itemize}
\item {Grp. gram.:adj.}
\end{itemize}
\begin{itemize}
\item {Grp. gram.:M.}
\end{itemize}
\begin{itemize}
\item {Proveniência:(Lat. \textunderscore internus\textunderscore )}
\end{itemize}
O mesmo que \textunderscore interior\textunderscore .
Íntimo.
Que vive dentro de um estabelecimento, em que é funccionário, alumno ou asylado.
Alumno interno de um collegio.
Funccionário interno de um estabelecimento.
\section{Internódio}
\begin{itemize}
\item {Grp. gram.:m.}
\end{itemize}
\begin{itemize}
\item {Proveniência:(Lat. \textunderscore internodium\textunderscore )}
\end{itemize}
Espaço, entre os nós de uma planta; entre-nó.
\section{Internúncio}
\begin{itemize}
\item {Grp. gram.:m.}
\end{itemize}
\begin{itemize}
\item {Proveniência:(Lat. \textunderscore internuntius\textunderscore )}
\end{itemize}
Aquelle que leva noticias de um ponto para outro.
Mensageiro.
Representante da cúria romana, nos países em que esta não tem núncio.
\section{Intero...}
\begin{itemize}
\item {Grp. gram.:pref.}
\end{itemize}
\begin{itemize}
\item {Proveniência:(De \textunderscore interior\textunderscore )}
\end{itemize}
(designativo do \textunderscore que é interior\textunderscore )
\section{Intero-anterior}
\begin{itemize}
\item {Grp. gram.:adj.}
\end{itemize}
Que está dentro e na parte anterior.
\section{Interoceânico}
\begin{itemize}
\item {Grp. gram.:adj.}
\end{itemize}
\begin{itemize}
\item {Proveniência:(De \textunderscore inter...\textunderscore  + \textunderscore oceânico\textunderscore )}
\end{itemize}
Que está entre oceanos.
Que liga oceanos: \textunderscore canal interoceânico\textunderscore .
\section{Interocular}
\begin{itemize}
\item {Grp. gram.:adj.}
\end{itemize}
\begin{itemize}
\item {Proveniência:(De \textunderscore inter...\textunderscore  + \textunderscore ocular\textunderscore )}
\end{itemize}
Que está entre os olhos.
\section{Intero-inferior}
\begin{itemize}
\item {Grp. gram.:adj.}
\end{itemize}
Que está dentro e na parte inferior.
\section{Interoposição}
\begin{itemize}
\item {Grp. gram.:f.}
\end{itemize}
\begin{itemize}
\item {Proveniência:(De \textunderscore inter...\textunderscore  + \textunderscore oposição\textunderscore )}
\end{itemize}
Estado dos objectos entrelaçados e reciprocamente opostos.
\section{Intero-posterior}
\begin{itemize}
\item {Grp. gram.:adj.}
\end{itemize}
Que está dentro e na parte posterior.
\section{Interopposição}
\begin{itemize}
\item {Grp. gram.:f.}
\end{itemize}
\begin{itemize}
\item {Proveniência:(De \textunderscore inter...\textunderscore  + \textunderscore opposição\textunderscore )}
\end{itemize}
Estado dos objectos entrelaçados e reciprocamente oppostos.
\section{Interósseo}
\begin{itemize}
\item {Grp. gram.:adj.}
\end{itemize}
\begin{itemize}
\item {Proveniência:(De \textunderscore inter...\textunderscore  + \textunderscore ósseo\textunderscore )}
\end{itemize}
Que está entre os ossos.
\section{Intero-superior}
\begin{itemize}
\item {Grp. gram.:adj.}
\end{itemize}
Que está dentro e na parte superior.
\section{Interparietal}
\begin{itemize}
\item {Grp. gram.:adj.}
\end{itemize}
\begin{itemize}
\item {Proveniência:(De \textunderscore inter...\textunderscore  + \textunderscore parietal\textunderscore )}
\end{itemize}
Que está entre os ossos parietais.
\section{Interparlamentar}
\begin{itemize}
\item {Grp. gram.:adj.}
\end{itemize}
\begin{itemize}
\item {Proveniência:(De \textunderscore inter...\textunderscore  + \textunderscore parlamentar\textunderscore )}
\end{itemize}
Que se realiza no intervallo das sessões parlamentares.
Relativo a factos, em que intervêm os representantes de vários parlamentos: \textunderscore as conferências interparlamentares da Paz\textunderscore .
\section{Interpeciolar}
\begin{itemize}
\item {Grp. gram.:adj.}
\end{itemize}
\begin{itemize}
\item {Utilização:Bot.}
\end{itemize}
\begin{itemize}
\item {Proveniência:(De \textunderscore inter...\textunderscore  + \textunderscore peciolar\textunderscore )}
\end{itemize}
Nascido entre fôlhas oppostas.
\section{Interpelação}
\begin{itemize}
\item {Grp. gram.:f.}
\end{itemize}
\begin{itemize}
\item {Proveniência:(Lat. \textunderscore interpellatio\textunderscore )}
\end{itemize}
Acto ou efeito de interpelar.
Intimação judicial, para responder á cêrca de um facto.
\section{Interpelador}
\begin{itemize}
\item {Grp. gram.:m.  e  adj.}
\end{itemize}
\begin{itemize}
\item {Proveniência:(Lat. \textunderscore interpellator\textunderscore )}
\end{itemize}
O que interpela.
\section{Interpelante}
\begin{itemize}
\item {Grp. gram.:m.  e  adj.}
\end{itemize}
\begin{itemize}
\item {Proveniência:(Lat. \textunderscore interpellans\textunderscore )}
\end{itemize}
O mesmo que \textunderscore interpelador\textunderscore .
\section{Interpelar}
\begin{itemize}
\item {Grp. gram.:v. t.}
\end{itemize}
\begin{itemize}
\item {Proveniência:(Lat. \textunderscore interpellare\textunderscore )}
\end{itemize}
Interromper (quem fala).
Perturbar.
Intimar.
Pedir nas côrtes explicações a (um ministro).
\section{Interpellação}
\begin{itemize}
\item {Grp. gram.:f.}
\end{itemize}
\begin{itemize}
\item {Proveniência:(Lat. \textunderscore interpellatio\textunderscore )}
\end{itemize}
Acto ou effeito de interpellar.
Intimação judicial, para responder á cêrca de um facto.
\section{Interpellador}
\begin{itemize}
\item {Grp. gram.:m.  e  adj.}
\end{itemize}
\begin{itemize}
\item {Proveniência:(Lat. \textunderscore interpellator\textunderscore )}
\end{itemize}
O que interpella.
\section{Interpellante}
\begin{itemize}
\item {Grp. gram.:m.  e  adj.}
\end{itemize}
\begin{itemize}
\item {Proveniência:(Lat. \textunderscore interpellans\textunderscore )}
\end{itemize}
O mesmo que \textunderscore interpellador\textunderscore .
\section{Interpellar}
\begin{itemize}
\item {Grp. gram.:v. t.}
\end{itemize}
\begin{itemize}
\item {Proveniência:(Lat. \textunderscore interpellare\textunderscore )}
\end{itemize}
Interromper (quem fala).
Perturbar.
Intimar.
Pedir nas côrtes explicações a (um ministro).
\section{Interpeninsular}
\begin{itemize}
\item {Grp. gram.:adj.}
\end{itemize}
\begin{itemize}
\item {Proveniência:(De \textunderscore inter...\textunderscore  + \textunderscore peninsular\textunderscore )}
\end{itemize}
Situado entre penínsulas.
\section{Interphalangeano}
\begin{itemize}
\item {Grp. gram.:adj.}
\end{itemize}
\begin{itemize}
\item {Utilização:anat.}
\end{itemize}
\begin{itemize}
\item {Proveniência:(De \textunderscore inter...\textunderscore  + \textunderscore phalangeano\textunderscore )}
\end{itemize}
Situado entre as phalanges.
Diz-se especialmente da última articulação dos membros do cavallo. Cf. Leon, \textunderscore Arte de Ferrar\textunderscore , 31.
\section{Interplanetário}
\begin{itemize}
\item {Grp. gram.:adj.}
\end{itemize}
\begin{itemize}
\item {Proveniência:(De \textunderscore inter...\textunderscore  + \textunderscore planetário\textunderscore )}
\end{itemize}
Que está entre planetas.
\section{Interpoimento}
\begin{itemize}
\item {fónica:po-i}
\end{itemize}
\begin{itemize}
\item {Grp. gram.:m.}
\end{itemize}
\begin{itemize}
\item {Utilização:Des.}
\end{itemize}
\begin{itemize}
\item {Proveniência:(De \textunderscore inter...\textunderscore  + \textunderscore poer\textunderscore )}
\end{itemize}
O mesmo que \textunderscore interposição\textunderscore .
\section{Interpolação}
\begin{itemize}
\item {Grp. gram.:f.}
\end{itemize}
\begin{itemize}
\item {Proveniência:(Lat. \textunderscore interpolatio\textunderscore )}
\end{itemize}
Acto ou effeito de interpolar.
\section{Interpoladamente}
\begin{itemize}
\item {Grp. gram.:adv.}
\end{itemize}
\begin{itemize}
\item {Proveniência:(De \textunderscore interpolar\textunderscore )}
\end{itemize}
Com interpolação.
\section{Interpolado}
\begin{itemize}
\item {Grp. gram.:adj.}
\end{itemize}
\begin{itemize}
\item {Grp. gram.:F. pl.}
\end{itemize}
\begin{itemize}
\item {Utilização:Heráld.}
\end{itemize}
\begin{itemize}
\item {Proveniência:(De \textunderscore interpolar\textunderscore )}
\end{itemize}
Que soffreu interrupção.
Que tem pêlos brancos, entremeados com pêlos escuros, (falando-se do cavallo).
Diz-se das figuras heráldicas de terceira ordem.
\section{Interpolador}
\begin{itemize}
\item {Grp. gram.:m.  e  adj.}
\end{itemize}
\begin{itemize}
\item {Proveniência:(Lat. \textunderscore interpolator\textunderscore )}
\end{itemize}
O que interpola.
\section{Interpolamento}
\begin{itemize}
\item {Grp. gram.:m.}
\end{itemize}
\begin{itemize}
\item {Proveniência:(Lat. \textunderscore interpolamentum\textunderscore )}
\end{itemize}
O mesmo que \textunderscore interpolação\textunderscore .
\section{Interpolar}
\begin{itemize}
\item {Grp. gram.:v. t.}
\end{itemize}
\begin{itemize}
\item {Proveniência:(Lat. \textunderscore interpolare\textunderscore )}
\end{itemize}
Revolver.
Alterar.
Alternar.
Intercalar em (um texto) palavras ou phrases, para o esclarecer ou para o adulterar.
\section{Interpolar}
\begin{itemize}
\item {Grp. gram.:adj.}
\end{itemize}
\begin{itemize}
\item {Utilização:Phýs.}
\end{itemize}
\begin{itemize}
\item {Proveniência:(De \textunderscore inter...\textunderscore  + \textunderscore polar\textunderscore )}
\end{itemize}
Situado entre os polos de uma pilha.
\section{Interpontuação}
\begin{itemize}
\item {Grp. gram.:f.}
\end{itemize}
\begin{itemize}
\item {Proveniência:(De \textunderscore inter...\textunderscore  + \textunderscore pontuação\textunderscore )}
\end{itemize}
Série de pontos que, num discurso, indicam reticência ou suppressão de uma parte do texto.
\section{Interpôr}
\begin{itemize}
\item {Grp. gram.:v. t.}
\end{itemize}
\begin{itemize}
\item {Proveniência:(Lat. \textunderscore interponere\textunderscore )}
\end{itemize}
Pôr entre.
Oppor.
Fazer intervir.
\section{Interporto}
\begin{itemize}
\item {Grp. gram.:m.}
\end{itemize}
\begin{itemize}
\item {Proveniência:(De \textunderscore inter...\textunderscore  + \textunderscore pôrto\textunderscore )}
\end{itemize}
Pôrto, que fica entre aquelle de que sái uma embarcação e o pôrto a que ella se dirige.
\section{Interposição}
\begin{itemize}
\item {Grp. gram.:f.}
\end{itemize}
\begin{itemize}
\item {Utilização:Fig.}
\end{itemize}
\begin{itemize}
\item {Proveniência:(Lat. \textunderscore interpositio\textunderscore )}
\end{itemize}
Acto ou effeito de interpor.
Interrupção.
Intervenção.
Occorrência de um obstáculo.
\section{Interpositivo}
\begin{itemize}
\item {Grp. gram.:adj.}
\end{itemize}
\begin{itemize}
\item {Utilização:Bot.}
\end{itemize}
O mesmo que \textunderscore interposto\textunderscore ^1.
\section{Interposto}
\begin{itemize}
\item {Grp. gram.:adj.}
\end{itemize}
Que se interpôs; que está de permeio.
\section{Interposto}
\begin{itemize}
\item {Grp. gram.:m.}
\end{itemize}
(V.entrepósito)
\section{Interpotente}
\begin{itemize}
\item {Grp. gram.:adj.}
\end{itemize}
\begin{itemize}
\item {Proveniência:(De \textunderscore inter...\textunderscore  + \textunderscore potente\textunderscore )}
\end{itemize}
Que tem a potência entre a resistência e o ponto de apoio, (falando-se de alavancas).
\section{Interprender}
\begin{itemize}
\item {Grp. gram.:v. t.}
\end{itemize}
O mesmo que \textunderscore entreprender\textunderscore .
\section{Interpresa}
\begin{itemize}
\item {Grp. gram.:f.}
\end{itemize}
(V.entrepresa)
\section{Interpretação}
\begin{itemize}
\item {Grp. gram.:f.}
\end{itemize}
\begin{itemize}
\item {Proveniência:(Lat. \textunderscore interpretatio\textunderscore )}
\end{itemize}
Acto, effeito ou modo de interpretar.
Commentário.
Versão.
\section{Interpretador}
\begin{itemize}
\item {Grp. gram.:m.  e  adj.}
\end{itemize}
\begin{itemize}
\item {Proveniência:(Lat. \textunderscore interpretator\textunderscore )}
\end{itemize}
O que interpreta.
\section{Interpretante}
\begin{itemize}
\item {Grp. gram.:m. ,  f.  e  adj.}
\end{itemize}
\begin{itemize}
\item {Proveniência:(Lat. \textunderscore interpretans\textunderscore )}
\end{itemize}
Pessôa que interpreta.
\section{Interpretar}
\begin{itemize}
\item {Grp. gram.:v. t.}
\end{itemize}
\begin{itemize}
\item {Proveniência:(Lat. \textunderscore interpretari\textunderscore )}
\end{itemize}
Explicar.
Traduzir.
Commentar ou esclarecer o que há de obscuro ou de antigo em.
Julgar da intenção de.
Avaliar o sentido de.
Reproduzir o pensamento ou a intenção de.
\section{Interpretativamente}
\begin{itemize}
\item {Grp. gram.:adv.}
\end{itemize}
De modo interpretativo.
\section{Interpretativo}
\begin{itemize}
\item {Grp. gram.:adj.}
\end{itemize}
\begin{itemize}
\item {Proveniência:(De \textunderscore interpretar\textunderscore )}
\end{itemize}
Que contém interpretação; susceptível de interpretação.
\section{Interpretável}
\begin{itemize}
\item {Grp. gram.:adj.}
\end{itemize}
Que se póde interpretar.
\section{Intérprete}
\begin{itemize}
\item {Grp. gram.:m.}
\end{itemize}
\begin{itemize}
\item {Proveniência:(Lat. \textunderscore interpres\textunderscore )}
\end{itemize}
Aquelle que interpreta.
Aquelle que revela ou indica o que se não conhecia ou estava occulto.
Pessôa, que serve de língua ou de intermediário, para que se comprehendam pessôas que falam idioma diverso.
Pessôa, que explica as palavras de uma língua por palavras de outra língua.
\section{Interregno}
\begin{itemize}
\item {Grp. gram.:m.}
\end{itemize}
\begin{itemize}
\item {Utilização:Fig.}
\end{itemize}
\begin{itemize}
\item {Proveniência:(Lat. \textunderscore interregnum\textunderscore )}
\end{itemize}
Intervallo entre dois reinados.
Interrupção.
\section{Interrei}
\begin{itemize}
\item {Grp. gram.:m.}
\end{itemize}
\begin{itemize}
\item {Proveniência:(Lat. \textunderscore interrex\textunderscore )}
\end{itemize}
Espécie de regente, que na antiga monarchia romana exercia as funcções do rei fallecido, até que outro rei fôsse acclamado.
Magistrado que, no tempo da república romana, fazia as vezes dos cônsules, na ausência ou falta dêstes.
\section{Interresistente}
\begin{itemize}
\item {Grp. gram.:adj.}
\end{itemize}
\begin{itemize}
\item {Proveniência:(De \textunderscore inter...\textunderscore  + \textunderscore resistente\textunderscore )}
\end{itemize}
Que tem a resistência entre a potência e o ponto de apoio, (falando-se de alavancas)
\section{Interrogação}
\begin{itemize}
\item {Grp. gram.:f.}
\end{itemize}
\begin{itemize}
\item {Proveniência:(Lat. \textunderscore interrogatio\textunderscore )}
\end{itemize}
Acto ou effeito de interrogar.
Pergunta.
Ponto ou sinal gráphico, que indica a interrogação.
\section{Interrogador}
\begin{itemize}
\item {Grp. gram.:m.  e  adj.}
\end{itemize}
\begin{itemize}
\item {Proveniência:(Lat. \textunderscore interrogator\textunderscore )}
\end{itemize}
O que interroga.
\section{Interrogamento}
\begin{itemize}
\item {Grp. gram.:m.}
\end{itemize}
\begin{itemize}
\item {Proveniência:(Lat. \textunderscore interrogamentum\textunderscore )}
\end{itemize}
O mesmo que \textunderscore interrogação\textunderscore .
\section{Interrogante}
\begin{itemize}
\item {Grp. gram.:m.  e  adj.}
\end{itemize}
\begin{itemize}
\item {Proveniência:(Lat. \textunderscore interrogans\textunderscore )}
\end{itemize}
O mesmo que \textunderscore interrogador\textunderscore .
\section{Interrogar}
\begin{itemize}
\item {Grp. gram.:v. t.}
\end{itemize}
\begin{itemize}
\item {Utilização:Fig.}
\end{itemize}
\begin{itemize}
\item {Proveniência:(Lat. \textunderscore interrogare\textunderscore )}
\end{itemize}
Fazer perguntas a.
Inquirir (testemunhas).
Investigar; examinar: \textunderscore interrogar os archivos\textunderscore .
\section{Interrogativo}
\begin{itemize}
\item {Grp. gram.:adj.}
\end{itemize}
\begin{itemize}
\item {Proveniência:(Lat. \textunderscore interrogativus\textunderscore )}
\end{itemize}
Próprio para interrogar; que envolve interrogação.
\section{Interrogatório}
\begin{itemize}
\item {Grp. gram.:adj.}
\end{itemize}
\begin{itemize}
\item {Grp. gram.:M.}
\end{itemize}
\begin{itemize}
\item {Proveniência:(Lat. \textunderscore interrogatorius\textunderscore )}
\end{itemize}
O mesmo que \textunderscore interrogativo\textunderscore .
Acto de interrogar; inquirição.
\section{Interrompedor}
\begin{itemize}
\item {Grp. gram.:m.  e  adj.}
\end{itemize}
(V.interruptor)
\section{Interromper}
\begin{itemize}
\item {Grp. gram.:v. t.}
\end{itemize}
\begin{itemize}
\item {Proveniência:(Lat. \textunderscore interrumpere\textunderscore )}
\end{itemize}
Romper ou dividir ao meio.
Suspender: \textunderscore interromper relações\textunderscore .
Fazer cessar por algum tempo: \textunderscore o temporal interrompeu as communicações telegráphicas\textunderscore .
Impedir ou cortar o discurso de: \textunderscore interromper um orador\textunderscore .
Pôr termo a.
\section{Interrompidamente}
\begin{itemize}
\item {Grp. gram.:adv.}
\end{itemize}
\begin{itemize}
\item {Proveniência:(De \textunderscore interromper\textunderscore )}
\end{itemize}
Com interrrupção.
\section{Interrompido}
\begin{itemize}
\item {Grp. gram.:adj.}
\end{itemize}
\begin{itemize}
\item {Proveniência:(De \textunderscore interromper\textunderscore )}
\end{itemize}
Que soffreu interrupção; que se suspendeu; que cessou.
\section{Interrução}
\begin{itemize}
\item {Grp. gram.:f.}
\end{itemize}
\begin{itemize}
\item {Proveniência:(Lat. \textunderscore interruptio\textunderscore )}
\end{itemize}
Acto ou effeito de interromper.
Aquillo que interrompe.
Reticência.
\section{Interrupção}
\begin{itemize}
\item {Grp. gram.:f.}
\end{itemize}
\begin{itemize}
\item {Proveniência:(Lat. \textunderscore interruptio\textunderscore )}
\end{itemize}
Acto ou effeito de interromper.
Aquillo que interrompe.
Reticência.
\section{Interruptamente}
\begin{itemize}
\item {Grp. gram.:adv.}
\end{itemize}
De modo interrupto; interrompidamente.
\section{Interrupto}
\begin{itemize}
\item {Grp. gram.:adj.}
\end{itemize}
\begin{itemize}
\item {Proveniência:(Lat. \textunderscore interruptus\textunderscore )}
\end{itemize}
O mesmo que \textunderscore interrompido\textunderscore .
\section{Interruptor}
\begin{itemize}
\item {Grp. gram.:m.  e  adj.}
\end{itemize}
\begin{itemize}
\item {Proveniência:(Lat. \textunderscore interruptor\textunderscore )}
\end{itemize}
O que interrompe.
\section{Interrutamente}
\begin{itemize}
\item {Grp. gram.:adv.}
\end{itemize}
De modo interruto; interrompidamente.
\section{Interruto}
\begin{itemize}
\item {Grp. gram.:adj.}
\end{itemize}
\begin{itemize}
\item {Proveniência:(Lat. \textunderscore interruptus\textunderscore )}
\end{itemize}
O mesmo que \textunderscore interrompido\textunderscore .
\section{Interrutor}
\begin{itemize}
\item {Grp. gram.:m.  e  adj.}
\end{itemize}
\begin{itemize}
\item {Proveniência:(Lat. \textunderscore interruptor\textunderscore )}
\end{itemize}
O que interrompe.
\section{Intersachar}
\begin{itemize}
\item {Grp. gram.:v. t.}
\end{itemize}
O mesmo que \textunderscore entresachar\textunderscore . Cf. Garrett, \textunderscore Romanceiro\textunderscore , II, p. VIII.
\section{Interscalmo}
\begin{itemize}
\item {Grp. gram.:m.}
\end{itemize}
\begin{itemize}
\item {Utilização:Náut.}
\end{itemize}
\begin{itemize}
\item {Proveniência:(Do lat. \textunderscore interscalmium\textunderscore )}
\end{itemize}
Espaço, entre as duas cavilhas ou toletes, a que se prende o remo.
\section{Intersecção}
\begin{itemize}
\item {Grp. gram.:f.}
\end{itemize}
\begin{itemize}
\item {Proveniência:(Lat. \textunderscore intersectio\textunderscore )}
\end{itemize}
Acto de cortar pelo meio.
Córte.
Ponto, em que se cruzam duas linhas ou superfícies.
\section{Interseccional}
\begin{itemize}
\item {Grp. gram.:adj.}
\end{itemize}
\begin{itemize}
\item {Proveniência:(Do lat. \textunderscore intersectio\textunderscore )}
\end{itemize}
Relativo a intersecção.
\section{Interserir}
\begin{itemize}
\item {Grp. gram.:v. t.}
\end{itemize}
\begin{itemize}
\item {Proveniência:(Lat. \textunderscore interserere\textunderscore )}
\end{itemize}
Inserir, collocar em meio.
\section{Intersilhado}
\begin{itemize}
\item {Grp. gram.:adj.}
\end{itemize}
Enfeitiçado, seduzido. Cf. Castilho, \textunderscore Collóq. Ald.\textunderscore , 221.
(Relaciona-se com \textunderscore entresilhado\textunderscore ?)
\section{Interstelar}
\begin{itemize}
\item {Grp. gram.:adj.}
\end{itemize}
\begin{itemize}
\item {Proveniência:(Do lat. \textunderscore inter\textunderscore  + \textunderscore stela\textunderscore )}
\end{itemize}
Que está entre estrêlas.
\section{Interstellar}
\begin{itemize}
\item {Grp. gram.:adj.}
\end{itemize}
\begin{itemize}
\item {Proveniência:(Do lat. \textunderscore inter\textunderscore  + \textunderscore stella\textunderscore )}
\end{itemize}
Que está entre estrêllas.
\section{Intersticial}
\begin{itemize}
\item {Grp. gram.:adj.}
\end{itemize}
Relativo a interstício.
\section{Interstício}
\begin{itemize}
\item {Grp. gram.:m.}
\end{itemize}
\begin{itemize}
\item {Utilização:Ext.}
\end{itemize}
\begin{itemize}
\item {Proveniência:(Lat. \textunderscore interstitium\textunderscore )}
\end{itemize}
Intervallo, que separa as moléculas dos corpos.
Fenda.
Intervallo, entre órgãos contíguos.
Intervallo.
\section{Intertexto}
\begin{itemize}
\item {Grp. gram.:adj.}
\end{itemize}
\begin{itemize}
\item {Proveniência:(Lat. \textunderscore intertextus\textunderscore )}
\end{itemize}
O mesmo que [[entretecido|entretecer]].
\section{Intertransversário}
\begin{itemize}
\item {Grp. gram.:adj.}
\end{itemize}
\begin{itemize}
\item {Utilização:Anat.}
\end{itemize}
\begin{itemize}
\item {Proveniência:(De \textunderscore inter...\textunderscore  + \textunderscore transverso\textunderscore )}
\end{itemize}
Que está entre as apóphyses transversaes das vértebras.
\section{Intertrigem}
\begin{itemize}
\item {Grp. gram.:m.}
\end{itemize}
\begin{itemize}
\item {Utilização:Med.}
\end{itemize}
\begin{itemize}
\item {Proveniência:(Lat. \textunderscore intertrigo\textunderscore )}
\end{itemize}
Escoriação das coxas, produzida por se andar a cavallo ou por outro motivo, especialmente pelo attrito.
\section{Intertropical}
\begin{itemize}
\item {Grp. gram.:adj.}
\end{itemize}
\begin{itemize}
\item {Proveniência:(De \textunderscore inter...\textunderscore  + \textunderscore tropical\textunderscore )}
\end{itemize}
Que está entre os trópicos.
Relativo á zona tórrida.
\section{Interutricular}
\begin{itemize}
\item {Grp. gram.:adj.}
\end{itemize}
\begin{itemize}
\item {Proveniência:(De \textunderscore inter...\textunderscore  + \textunderscore utricular\textunderscore )}
\end{itemize}
Que está entre os utrículos.
\section{Intervaladamente}
\begin{itemize}
\item {Grp. gram.:adv.}
\end{itemize}
\begin{itemize}
\item {Proveniência:(De \textunderscore intervalado\textunderscore )}
\end{itemize}
Com intervalos.
\section{Intervalado}
\begin{itemize}
\item {Grp. gram.:adj.}
\end{itemize}
Em que há intervalos.
Entremeado.
\section{Intervalladamente}
\begin{itemize}
\item {Grp. gram.:adv.}
\end{itemize}
\begin{itemize}
\item {Proveniência:(De \textunderscore intervallado\textunderscore )}
\end{itemize}
Com intervallos.
\section{Intervallado}
\begin{itemize}
\item {Grp. gram.:adj.}
\end{itemize}
Em que há intervallos.
Entremeado.
\section{Intervalar}
\begin{itemize}
\item {Grp. gram.:v. t.}
\end{itemize}
Separar por intervalos.
Abrir intervalos em.
Entremear; alternar.
\section{Intervalar}
\begin{itemize}
\item {Grp. gram.:adj.}
\end{itemize}
Que está num intervalo.
\section{Intervaleiro}
\begin{itemize}
\item {Grp. gram.:m.}
\end{itemize}
\begin{itemize}
\item {Proveniência:(De \textunderscore intervalo\textunderscore )}
\end{itemize}
Toireiro curioso, que farpeia sem conhecimento das regras mais triviaes.
\section{Intervallar}
\begin{itemize}
\item {Grp. gram.:v. t.}
\end{itemize}
Separar por intervallos.
Abrir intervallos em.
Entremear; alternar.
\section{Intervallar}
\begin{itemize}
\item {Grp. gram.:adj.}
\end{itemize}
Que está num intervallo.
\section{Intervalleiro}
\begin{itemize}
\item {Grp. gram.:m.}
\end{itemize}
\begin{itemize}
\item {Proveniência:(De \textunderscore intervallo\textunderscore )}
\end{itemize}
Toireiro curioso, que farpeia sem conhecimento das regras mais triviaes.
\section{Intervallo}
\begin{itemize}
\item {Grp. gram.:m.}
\end{itemize}
\begin{itemize}
\item {Proveniência:(Lat. \textunderscore intervallum\textunderscore )}
\end{itemize}
Distância de um ponto a outro.
Espaço, entre dois tempos ou duas épocas.
Distância entre duas notas musicaes de som differente.
Distância, que separa dois factos.
Intercadência, intermittência.
\section{Intervalo}
\begin{itemize}
\item {Grp. gram.:m.}
\end{itemize}
\begin{itemize}
\item {Proveniência:(Lat. \textunderscore intervallum\textunderscore )}
\end{itemize}
Distância de um ponto a outro.
Espaço, entre dois tempos ou duas épocas.
Distância entre duas notas musicaes de som diferente.
Distância, que separa dois factos.
Intercadência, intermitência.
\section{Intervenção}
\begin{itemize}
\item {Grp. gram.:f.}
\end{itemize}
\begin{itemize}
\item {Proveniência:(Lat. \textunderscore interventio\textunderscore )}
\end{itemize}
Acto ou effeito de intervir.
Intercessão; mediação.
\section{Intervencionista}
\begin{itemize}
\item {Grp. gram.:m.}
\end{itemize}
\begin{itemize}
\item {Proveniência:(Do lat. \textunderscore interventio\textunderscore )}
\end{itemize}
Partidário de uma intervenção.
\section{Intervenideira}
\begin{itemize}
\item {Grp. gram.:f.}
\end{itemize}
\begin{itemize}
\item {Utilização:Des.}
\end{itemize}
\begin{itemize}
\item {Proveniência:(Do lat. \textunderscore intervenire\textunderscore )}
\end{itemize}
O mesmo que \textunderscore alcoviteira\textunderscore .
\section{Interveniente}
\begin{itemize}
\item {Grp. gram.:adj.}
\end{itemize}
\begin{itemize}
\item {Grp. gram.:M.}
\end{itemize}
\begin{itemize}
\item {Proveniência:(Lat. \textunderscore interveniens\textunderscore )}
\end{itemize}
Que intervém.
Medianeiro.
Fiador de uma letra de câmbio.
\section{Interventivo}
\begin{itemize}
\item {Grp. gram.:adj.}
\end{itemize}
\begin{itemize}
\item {Proveniência:(Do lat. \textunderscore interventus\textunderscore )}
\end{itemize}
Relativo a intervenção.
\section{Interventor}
\begin{itemize}
\item {Grp. gram.:m.  e  adj.}
\end{itemize}
\begin{itemize}
\item {Proveniência:(Lat. \textunderscore interventor\textunderscore )}
\end{itemize}
O mesmo que \textunderscore interveniente\textunderscore .
\section{Interventricular}
\begin{itemize}
\item {Grp. gram.:adj.}
\end{itemize}
\begin{itemize}
\item {Utilização:Anat.}
\end{itemize}
\begin{itemize}
\item {Proveniência:(De \textunderscore interventrículo\textunderscore )}
\end{itemize}
Situado entre os dois ventrículos.
\section{Interversão}
\begin{itemize}
\item {Grp. gram.:f.}
\end{itemize}
\begin{itemize}
\item {Proveniência:(Lat. \textunderscore interversio\textunderscore )}
\end{itemize}
Acto de interverter.
\section{Intervertebral}
\begin{itemize}
\item {Grp. gram.:adj.}
\end{itemize}
\begin{itemize}
\item {Proveniência:(De \textunderscore inter...\textunderscore  + \textunderscore vertebral\textunderscore )}
\end{itemize}
Que está entre as vértebras.
\section{Interverter}
\begin{itemize}
\item {Grp. gram.:v. t.}
\end{itemize}
\begin{itemize}
\item {Proveniência:(Lat. \textunderscore intervertere\textunderscore )}
\end{itemize}
O mesmo que \textunderscore inverter\textunderscore .
\section{Intervindo}
\begin{itemize}
\item {Grp. gram.:adj.}
\end{itemize}
\begin{itemize}
\item {Proveniência:(Lat. \textunderscore intervenius\textunderscore )}
\end{itemize}
Que interveio.
\section{Intervir}
\begin{itemize}
\item {Grp. gram.:v. i.}
\end{itemize}
\begin{itemize}
\item {Proveniência:(Lat. \textunderscore intervenire\textunderscore )}
\end{itemize}
Vir ou collocar-se entre.
Interceder.
Ingerir-se.
Sobrevir.
\section{Intervocal}
\begin{itemize}
\item {Grp. gram.:adj.}
\end{itemize}
\begin{itemize}
\item {Proveniência:(Do lat. \textunderscore inter\textunderscore  + \textunderscore vocalis\textunderscore )}
\end{itemize}
Que está entre letras vogaes.
\section{Intervocálico}
\begin{itemize}
\item {Grp. gram.:adj.}
\end{itemize}
O mesmo que \textunderscore intervocal\textunderscore .
\section{Intestado}
\begin{itemize}
\item {Grp. gram.:adj.}
\end{itemize}
\begin{itemize}
\item {Proveniência:(Lat. \textunderscore intestatus\textunderscore )}
\end{itemize}
Que não fez testamento ou cujo testamento é nullo ou illegal.
\section{Intestável}
\begin{itemize}
\item {Grp. gram.:adj.}
\end{itemize}
\begin{itemize}
\item {Proveniência:(Lat. \textunderscore intestabilis\textunderscore )}
\end{itemize}
Que não póde testar; que não póde fazer testamento.
\section{Intestinal}
\begin{itemize}
\item {Grp. gram.:adj.}
\end{itemize}
Relativo a intestino^2 ou aos intestinos: \textunderscore inflammação intestinal\textunderscore .
\section{Intestino}
\begin{itemize}
\item {Grp. gram.:adj.}
\end{itemize}
\begin{itemize}
\item {Proveniência:(Lat. \textunderscore intestinus\textunderscore )}
\end{itemize}
Interior, interno, íntimo; doméstico; nacional: \textunderscore guerras intestinas\textunderscore .
\section{Intestino}
\begin{itemize}
\item {Grp. gram.:m.}
\end{itemize}
\begin{itemize}
\item {Grp. gram.:Pl.}
\end{itemize}
\begin{itemize}
\item {Proveniência:(Lat. \textunderscore intestinum\textunderscore )}
\end{itemize}
Víscera musculo-membranosa, contida na cavidade abdominal, e que se estende desde o estômago até ao ânus, comprehendendo os chamados intestinos delgados,--duodeno, jejuno e íleo,--e os intestinos grossos,--ceco, cólon e recto.
Todo o canal intestinal.
Entranhas: \textunderscore soffrer dos intestinos\textunderscore .
\section{Intexto}
\begin{itemize}
\item {Grp. gram.:adj.}
\end{itemize}
\begin{itemize}
\item {Proveniência:(Lat. \textunderscore intextus\textunderscore )}
\end{itemize}
Entrelaçado, entremeado. Cf. Filinto, IX, 265.
\section{Inticante}
\begin{itemize}
\item {Grp. gram.:adj.}
\end{itemize}
\begin{itemize}
\item {Utilização:Bras. do N}
\end{itemize}
Que intica.
\section{Inticar}
\begin{itemize}
\item {Grp. gram.:v. i.}
\end{itemize}
\begin{itemize}
\item {Utilização:bras}
\end{itemize}
\begin{itemize}
\item {Utilização:Açor}
\end{itemize}
Sêr metediço ou provocante.
\section{Intimação}
\begin{itemize}
\item {Grp. gram.:f.}
\end{itemize}
\begin{itemize}
\item {Proveniência:(Lat. \textunderscore intimatio\textunderscore )}
\end{itemize}
Acto de intimar; citação.
\section{Intimador}
\begin{itemize}
\item {Grp. gram.:m.  e  adj.}
\end{itemize}
\begin{itemize}
\item {Proveniência:(Lat. \textunderscore intimator\textunderscore )}
\end{itemize}
O que intima.
\section{Intimamente}
\begin{itemize}
\item {Grp. gram.:adv.}
\end{itemize}
De modo íntimo.
No fundo do coração.
Na mente.
Na alma.
Com intimidade: \textunderscore os dois vivem intimamente\textunderscore .
Familiarmente.
Estreitamente: \textunderscore interesses, intimamente ligados\textunderscore .
Com a maior ligação.
\section{Intimar}
\begin{itemize}
\item {Grp. gram.:v. t.}
\end{itemize}
\begin{itemize}
\item {Grp. gram.:V. i.}
\end{itemize}
\begin{itemize}
\item {Utilização:Bras. do N}
\end{itemize}
\begin{itemize}
\item {Proveniência:(Lat. \textunderscore intimare\textunderscore )}
\end{itemize}
Tornar sciente com autoridade.
Notificar.
Avisar.
Ordenar alguma coisa a.
Falar com intimativa ou com auctoridade.

Fazer provocação; dizer insultos.
\section{Intimativa}
\begin{itemize}
\item {Grp. gram.:f.}
\end{itemize}
\begin{itemize}
\item {Proveniência:(De \textunderscore intimativo\textunderscore )}
\end{itemize}
Gesto ou phrase, em que há energia ou que acompanha uma intimação.
Energia; arrogância: \textunderscore falar com intimativa\textunderscore .
\section{Intimativo}
\begin{itemize}
\item {Grp. gram.:adj.}
\end{itemize}
Que serve para intimar.
Enérgico.
\section{Intimidação}
\begin{itemize}
\item {Grp. gram.:f.}
\end{itemize}
Acto ou effeito de intimidar.
\section{Intimidade}
\begin{itemize}
\item {Grp. gram.:f.}
\end{itemize}
Qualidade de intimo: \textunderscore viver na intimidade da alguém\textunderscore .
\section{Intimidador}
\begin{itemize}
\item {Grp. gram.:adj.}
\end{itemize}
\begin{itemize}
\item {Grp. gram.:M.}
\end{itemize}
Que intimida.
Aquelle que intimida.
\section{Intimidar}
\begin{itemize}
\item {Grp. gram.:v. t.}
\end{itemize}
Tornar tímido; assustar; apavorar.
\section{Intimidativo}
\begin{itemize}
\item {Grp. gram.:adj.}
\end{itemize}
O mesmo que \textunderscore intimidador\textunderscore .
\section{Íntimo}
\begin{itemize}
\item {Grp. gram.:adj.}
\end{itemize}
\begin{itemize}
\item {Grp. gram.:M.}
\end{itemize}
\begin{itemize}
\item {Proveniência:(Lat. \textunderscore intimus\textunderscore )}
\end{itemize}
Que está muito dentro, muito no interior.
Que actua no interior dos corpos e das suas moléculas.
Muito ligado.
Muito cordeal ou affectuoso: \textunderscore amigo íntimo\textunderscore .
Doméstico; familiar.
Âmago.
A parte mais interna: \textunderscore quero-te do íntimo do coração\textunderscore .
Amigo íntimo.
\section{Intimorato}
\begin{itemize}
\item {Grp. gram.:adj.}
\end{itemize}
\begin{itemize}
\item {Proveniência:(De \textunderscore in...\textunderscore  + \textunderscore timorato\textunderscore )}
\end{itemize}
Não timorato.
Destemido; valente. Cf. Rui Rarb., \textunderscore Réplica\textunderscore , 157.
\section{Intina}
\begin{itemize}
\item {Grp. gram.:f.}
\end{itemize}
O mesmo ou melhor que \textunderscore endhymenina\textunderscore .
\section{Intincção}
\begin{itemize}
\item {Grp. gram.:f.}
\end{itemize}
\begin{itemize}
\item {Proveniência:(Lat. \textunderscore intinctio\textunderscore )}
\end{itemize}
Acto de lançar em vinho consagrado parte de uma hóstia.
\section{Intitulação}
\begin{itemize}
\item {Grp. gram.:f.}
\end{itemize}
Acto de intitular.
\section{Intitulamento}
\begin{itemize}
\item {Grp. gram.:m.}
\end{itemize}
O mesmo que \textunderscore intitulação\textunderscore .
\section{Intitular}
\begin{itemize}
\item {Grp. gram.:v. t.}
\end{itemize}
\begin{itemize}
\item {Proveniência:(Lat. \textunderscore intilulare\textunderscore )}
\end{itemize}
Dar titulo a; denominar.
\section{Intoirido}
\begin{itemize}
\item {Grp. gram.:adj.}
\end{itemize}
\begin{itemize}
\item {Proveniência:(De \textunderscore toiro\textunderscore )}
\end{itemize}
Bravo como um toiro. Cf. Camillo, \textunderscore Cav. em Ruínas\textunderscore , 128,
\section{Intolerância}
\begin{itemize}
\item {Grp. gram.:f.}
\end{itemize}
\begin{itemize}
\item {Proveniência:(Lat. \textunderscore intolerantia\textunderscore )}
\end{itemize}
Falta de tolerância; qualidade de intolerante.
Violência.
\section{Intolerante}
\begin{itemize}
\item {Grp. gram.:m.  e  adj.}
\end{itemize}
\begin{itemize}
\item {Proveniência:(Lat. \textunderscore intolerans\textunderscore )}
\end{itemize}
Pessoa, que não é tolerante; sectário do intolerantismo; contrário aos princípios de liberdade.
\section{Intolerantemente}
\begin{itemize}
\item {Grp. gram.:adv.}
\end{itemize}
De modo intolerante.
\section{Intolerantismo}
\begin{itemize}
\item {Grp. gram.:m.}
\end{itemize}
\begin{itemize}
\item {Proveniência:(De \textunderscore intolerante\textunderscore )}
\end{itemize}
Systema dos que não admittem, antes perseguem, opiniões ou crenças oppostas ás suas.
\section{Intolerável}
\begin{itemize}
\item {Grp. gram.:adj.}
\end{itemize}
\begin{itemize}
\item {Proveniência:(Lat. \textunderscore intolerabilis\textunderscore )}
\end{itemize}
Que não é tolerável, que se não póde tolerar.
Insupportável.
\section{Intoleravelmente}
\begin{itemize}
\item {Grp. gram.:adv.}
\end{itemize}
De modo intolerável.
\section{Intonação}
\begin{itemize}
\item {Grp. gram.:f.}
\end{itemize}
\begin{itemize}
\item {Proveniência:(Do lat. \textunderscore intonare\textunderscore )}
\end{itemize}
O mesmo que \textunderscore entoação\textunderscore .
\section{Intonso}
\begin{itemize}
\item {Grp. gram.:adj.}
\end{itemize}
\begin{itemize}
\item {Proveniência:(Lat. \textunderscore intonsus\textunderscore )}
\end{itemize}
Não tosquiado; hirsuto.
\section{Intorção}
\begin{itemize}
\item {Grp. gram.:f.}
\end{itemize}
\begin{itemize}
\item {Utilização:Bot.}
\end{itemize}
\begin{itemize}
\item {Proveniência:(Lat. \textunderscore intortio\textunderscore )}
\end{itemize}
Direcção, que as plantas tomam, diversa da que naturalmente deviam seguir.
\section{Intra...}
\begin{itemize}
\item {Grp. gram.:pref.}
\end{itemize}
\begin{itemize}
\item {Proveniência:(Do lat. \textunderscore intra\textunderscore )}
\end{itemize}
(designativo de \textunderscore dentro\textunderscore )
\section{Intracraniano}
\begin{itemize}
\item {Grp. gram.:adj.}
\end{itemize}
\begin{itemize}
\item {Proveniência:(De \textunderscore intra...\textunderscore  + \textunderscore craniano\textunderscore )}
\end{itemize}
Relativo ao interior do crânio.
\section{Intradilatado}
\begin{itemize}
\item {Grp. gram.:adj.}
\end{itemize}
\begin{itemize}
\item {Utilização:Bot.}
\end{itemize}
Dilatado por dentro.
\section{Intradorso}
\begin{itemize}
\item {Grp. gram.:m.}
\end{itemize}
\begin{itemize}
\item {Proveniência:(De \textunderscore intra...\textunderscore  + \textunderscore dorso\textunderscore )}
\end{itemize}
Superfície côncava interior de um arco ou de uma abóbada.
\section{Intraduzível}
\begin{itemize}
\item {Grp. gram.:adj.}
\end{itemize}
\begin{itemize}
\item {Proveniência:(De \textunderscore in...\textunderscore  + \textunderscore traduzível\textunderscore )}
\end{itemize}
Que se não póde traduzir.
\section{Intrafólio}
\begin{itemize}
\item {Grp. gram.:adj.}
\end{itemize}
\begin{itemize}
\item {Utilização:Bot.}
\end{itemize}
\begin{itemize}
\item {Proveniência:(Do lat. \textunderscore intra\textunderscore  + \textunderscore folium\textunderscore )}
\end{itemize}
Que nasce entre as fôlhas.
\section{Intra-hepático}
\begin{itemize}
\item {Grp. gram.:adj.}
\end{itemize}
Que está no interior do fígado.
\section{Intramarginal}
\begin{itemize}
\item {Grp. gram.:adj.}
\end{itemize}
\begin{itemize}
\item {Utilização:Bot.}
\end{itemize}
\begin{itemize}
\item {Proveniência:(Do lat. \textunderscore intra\textunderscore  + \textunderscore marginatus\textunderscore )}
\end{itemize}
Que está entre os bordos, (falando-se das nervuras das fôlhas e das flôres).
\section{Intramedular}
\begin{itemize}
\item {Grp. gram.:adj.}
\end{itemize}
\begin{itemize}
\item {Proveniência:(De \textunderscore intra...\textunderscore  + \textunderscore medular\textunderscore )}
\end{itemize}
Que está dentro da medula.
\section{Intramedullar}
\begin{itemize}
\item {Grp. gram.:adj.}
\end{itemize}
\begin{itemize}
\item {Proveniência:(De \textunderscore intra...\textunderscore  + \textunderscore medullar\textunderscore )}
\end{itemize}
Que está dentro da medulla.
\section{Intramento}
\begin{itemize}
\item {Grp. gram.:m.}
\end{itemize}
\begin{itemize}
\item {Utilização:Ant.}
\end{itemize}
\begin{itemize}
\item {Proveniência:(Do lat. \textunderscore intrare\textunderscore )}
\end{itemize}
O mesmo que \textunderscore entrada\textunderscore . Cf. Frei Fortun., \textunderscore Inéd.\textunderscore , 309.
\section{Intra-muros}
\begin{itemize}
\item {Grp. gram.:loc. adv.}
\end{itemize}
Da parte de dentro dos muros ou muralhas de uma povoação.
\section{Intramuscular}
\begin{itemize}
\item {Grp. gram.:adj.}
\end{itemize}
\begin{itemize}
\item {Proveniência:(De \textunderscore intra...\textunderscore  + \textunderscore muscular\textunderscore )}
\end{itemize}
Que está na espessura dos músculos.
\section{Intranquilidade}
\begin{itemize}
\item {fónica:cu-i}
\end{itemize}
\begin{itemize}
\item {Grp. gram.:f.}
\end{itemize}
\begin{itemize}
\item {Utilização:Neol.}
\end{itemize}
\begin{itemize}
\item {Proveniência:(De \textunderscore in...\textunderscore  + \textunderscore tranquilidade\textunderscore )}
\end{itemize}
Falta de tranquilidade.
\section{Intranquillidade}
\begin{itemize}
\item {fónica:cu-i}
\end{itemize}
\begin{itemize}
\item {Grp. gram.:f.}
\end{itemize}
\begin{itemize}
\item {Utilização:Neol.}
\end{itemize}
\begin{itemize}
\item {Proveniência:(De \textunderscore in...\textunderscore  + \textunderscore tranquillidade\textunderscore )}
\end{itemize}
Falta de tranquillidade.
\section{Intranquillo}
\begin{itemize}
\item {fónica:cu-i}
\end{itemize}
\begin{itemize}
\item {Grp. gram.:adj.}
\end{itemize}
\begin{itemize}
\item {Proveniência:(De \textunderscore in...\textunderscore  + \textunderscore tranquillo\textunderscore )}
\end{itemize}
Não tranquillo.
Inquieto. Cf. J. Ribeiro, \textunderscore Crepúsc. dos Deuses\textunderscore , 171.
\section{Intranquilo}
\begin{itemize}
\item {fónica:cu-i}
\end{itemize}
\begin{itemize}
\item {Grp. gram.:adj.}
\end{itemize}
\begin{itemize}
\item {Proveniência:(De \textunderscore in...\textunderscore  + \textunderscore tranquilo\textunderscore )}
\end{itemize}
Não tranquilo.
Inquieto. Cf. J. Ribeiro, \textunderscore Crepúsc. dos Deuses\textunderscore , 171.
\section{Intransferível}
\begin{itemize}
\item {Grp. gram.:adj.}
\end{itemize}
\begin{itemize}
\item {Proveniência:(De \textunderscore in...\textunderscore  + \textunderscore transferível\textunderscore )}
\end{itemize}
Que não é transferível; intransmissível.
\section{Intransigência}
\begin{itemize}
\item {Grp. gram.:f.}
\end{itemize}
\begin{itemize}
\item {Utilização:Fig.}
\end{itemize}
\begin{itemize}
\item {Proveniência:(De \textunderscore in...\textunderscore  + \textunderscore transigência\textunderscore )}
\end{itemize}
Falta de transigência; intolerância.
Austeridade de carácter.
\section{Intransigente}
\begin{itemize}
\item {Grp. gram.:m.  e  adj.}
\end{itemize}
\begin{itemize}
\item {Utilização:Fig.}
\end{itemize}
\begin{itemize}
\item {Proveniência:(De \textunderscore in...\textunderscore  + \textunderscore transigente\textunderscore )}
\end{itemize}
Pessôa, que não transige.
Intolerante.
Austero.
\section{Intransitado}
\begin{itemize}
\item {Grp. gram.:adj.}
\end{itemize}
\begin{itemize}
\item {Proveniência:(De \textunderscore in...\textunderscore  + \textunderscore transitado\textunderscore )}
\end{itemize}
Que não é transitado, (falando-se de caminhos). Cf. Camillo, \textunderscore Noites de Insómn.\textunderscore , VII, 24.
\section{Intransitável}
\begin{itemize}
\item {Grp. gram.:adj.}
\end{itemize}
\begin{itemize}
\item {Proveniência:(De \textunderscore in...\textunderscore  + \textunderscore transitável\textunderscore )}
\end{itemize}
Que não é transitável; por onde se não póde passar.
Por onde difficilmente se passa: \textunderscore ruas intransitáveis\textunderscore .
\section{Intransitivamente}
\begin{itemize}
\item {Grp. gram.:adv.}
\end{itemize}
\begin{itemize}
\item {Utilização:Gram.}
\end{itemize}
\begin{itemize}
\item {Proveniência:(De \textunderscore intransitivo\textunderscore )}
\end{itemize}
Á maneira de verbo intransitivo.
\section{Intransitivo}
\begin{itemize}
\item {Grp. gram.:adj.}
\end{itemize}
\begin{itemize}
\item {Utilização:Gram.}
\end{itemize}
\begin{itemize}
\item {Proveniência:(Lat. \textunderscore intransitivus\textunderscore )}
\end{itemize}
Diz-se dos verbos, que exprimem acção ou estado que não passa do sujeito.
O mesmo que \textunderscore intransmissível\textunderscore . Cf. Herculano, \textunderscore Hist. de Port.\textunderscore , III, 375.
\section{Intransmissibilidade}
\begin{itemize}
\item {Grp. gram.:f.}
\end{itemize}
Qualidade daquillo que é intransmissível.
\section{Intransmissível}
\begin{itemize}
\item {Grp. gram.:adj.}
\end{itemize}
\begin{itemize}
\item {Proveniência:(De \textunderscore in...\textunderscore  + \textunderscore transmissível\textunderscore )}
\end{itemize}
Que não é transmissível; que se não póde ou se não deve transferir para outrem.
\section{Intransponível}
\begin{itemize}
\item {Grp. gram.:adj.}
\end{itemize}
Que se não póde transpor: \textunderscore montes intransponíveis\textunderscore .
\section{Intransportável}
\begin{itemize}
\item {Grp. gram.:adj.}
\end{itemize}
\begin{itemize}
\item {Proveniência:(De \textunderscore in...\textunderscore  + \textunderscore transportável\textunderscore )}
\end{itemize}
Que se não póde transportar.
\section{Intraocular}
\begin{itemize}
\item {fónica:tra-o}
\end{itemize}
\begin{itemize}
\item {Grp. gram.:adj.}
\end{itemize}
\begin{itemize}
\item {Proveniência:(De \textunderscore intra...\textunderscore  + \textunderscore ocular\textunderscore )}
\end{itemize}
Que está no interior do ôlho.
\section{Intrapulmonar}
\begin{itemize}
\item {Grp. gram.:adj.}
\end{itemize}
\begin{itemize}
\item {Proveniência:(De \textunderscore intra...\textunderscore  + \textunderscore pulmonar\textunderscore )}
\end{itemize}
Que está no interior dos pulmões.
\section{Intrário}
\begin{itemize}
\item {Grp. gram.:adj.}
\end{itemize}
\begin{itemize}
\item {Utilização:Bot.}
\end{itemize}
\begin{itemize}
\item {Proveniência:(Do lat. \textunderscore intra\textunderscore )}
\end{itemize}
Diz-se do embryão, quando contido no endosperma.
\section{Intratabilidade}
\begin{itemize}
\item {Grp. gram.:f.}
\end{itemize}
Qualidade de intratável. Cf. Castilho, \textunderscore Fastos\textunderscore , II, 502.
\section{Intratado}
\begin{itemize}
\item {Grp. gram.:adj.}
\end{itemize}
\begin{itemize}
\item {Proveniência:(De \textunderscore in...\textunderscore  + \textunderscore tratado\textunderscore )}
\end{itemize}
Não tratado; evitado.
Que se não experimentou.
\section{Intratável}
\begin{itemize}
\item {Grp. gram.:adj.}
\end{itemize}
\begin{itemize}
\item {Utilização:Fig.}
\end{itemize}
\begin{itemize}
\item {Proveniência:(Do lat. \textunderscore intractabilis\textunderscore )}
\end{itemize}
Que se não póde tratar.
Orgulhoso; insociável: \textunderscore é uma criatura intratável\textunderscore .
Intransitável.
Diffícil de se fundir.
\section{Intra-thorácico}
\begin{itemize}
\item {Grp. gram.:adj.}
\end{itemize}
Que está dentro do thórax.
Relativo á parte interna do thórax.
\section{Intra-uterino}
\begin{itemize}
\item {Grp. gram.:adj.}
\end{itemize}
Relativo ao interior do útero.
Que está ou se realiza dentro do útero.
\section{Intravascular}
\begin{itemize}
\item {Grp. gram.:adj.}
\end{itemize}
\begin{itemize}
\item {Proveniência:(De \textunderscore intra...\textunderscore  + \textunderscore vascular\textunderscore )}
\end{itemize}
Relativo ao interior dos vasos do organismo.
\section{Intravertebrado}
\begin{itemize}
\item {Grp. gram.:adj.}
\end{itemize}
\begin{itemize}
\item {Utilização:Zool.}
\end{itemize}
\begin{itemize}
\item {Proveniência:(De \textunderscore intra...\textunderscore  + \textunderscore vertebrado\textunderscore )}
\end{itemize}
Que tem esqueleto vertebral no interior do corpo.
\section{Intrêmulo}
\begin{itemize}
\item {Grp. gram.:adj.}
\end{itemize}
\begin{itemize}
\item {Utilização:Fig.}
\end{itemize}
\begin{itemize}
\item {Proveniência:(Lat. \textunderscore intremulus\textunderscore )}
\end{itemize}
Que não é trêmulo.
Destemido, intrépido.
\section{Intrepidamente}
\begin{itemize}
\item {Grp. gram.:adv.}
\end{itemize}
De modo intrépido; com intrepidez.
\section{Intrepidez}
\begin{itemize}
\item {Grp. gram.:f.}
\end{itemize}
Qualidade de intrépido.
Coragem.
Ousadia.
\section{Intrepideza}
\begin{itemize}
\item {Grp. gram.:f.}
\end{itemize}
\begin{itemize}
\item {Utilização:Ant.}
\end{itemize}
O mesmo que \textunderscore intrepidez\textunderscore .
\section{Intrépido}
\begin{itemize}
\item {Grp. gram.:adj.}
\end{itemize}
\begin{itemize}
\item {Proveniência:(Lat. \textunderscore intrepidus\textunderscore )}
\end{itemize}
Que não trepida; que não tem medo; corajoso.
Audaz.
\section{Intributável}
\begin{itemize}
\item {Grp. gram.:adj.}
\end{itemize}
\begin{itemize}
\item {Proveniência:(De \textunderscore in...\textunderscore  + \textunderscore tributável\textunderscore )}
\end{itemize}
Que se não póde tributar.
\section{Intricadamente}
\begin{itemize}
\item {Grp. gram.:adv.}
\end{itemize}
De modo intricado.
Com embaraço.
\section{Intricado}
\begin{itemize}
\item {Grp. gram.:adj.}
\end{itemize}
\begin{itemize}
\item {Proveniência:(De \textunderscore intricar\textunderscore )}
\end{itemize}
Embaraçado.
Confuso; diffícil de comprehender.
\section{Intricar}
\begin{itemize}
\item {Grp. gram.:v. t.}
\end{itemize}
\begin{itemize}
\item {Proveniência:(Lat. \textunderscore intricare\textunderscore )}
\end{itemize}
Embaraçar; enredar; complicar; tornar obscuro.
\section{Intriga}
\begin{itemize}
\item {Grp. gram.:f.}
\end{itemize}
\begin{itemize}
\item {Proveniência:(De \textunderscore intrigar\textunderscore )}
\end{itemize}
Enrêdo secreto.
Cilada.
Traição.
Bisbilhotice.
Enredo de uma peça literária.
\section{Intrigante}
\begin{itemize}
\item {Grp. gram.:m., f.  e  adj.}
\end{itemize}
\begin{itemize}
\item {Proveniência:(Lat. \textunderscore intricans\textunderscore )}
\end{itemize}
Pessôa, que intriga.
\section{Intrigar}
\begin{itemize}
\item {Grp. gram.:v. t.}
\end{itemize}
\begin{itemize}
\item {Grp. gram.:V. i.}
\end{itemize}
\begin{itemize}
\item {Proveniência:(Do lat. \textunderscore intricare\textunderscore )}
\end{itemize}
Enredar occultamente.
Envolver em mexericos.
Indispor.
Inimizar.
Excitar a curiosidade de.
Armar enredos ou mexericos.
\section{Intriguista}
\begin{itemize}
\item {Grp. gram.:m., f.  e  adj.}
\end{itemize}
O mesmo que \textunderscore intrigante\textunderscore .
\section{Intrincado}
\begin{itemize}
\item {Grp. gram.:adj.}
\end{itemize}
O mesmo que \textunderscore intricado\textunderscore .
Obscuro; problemático; emmaranhado; diffícil de perceber.--É termo clássico, embora desprezado por diccionaristas. Na linguagem geral, usa-se mais do que \textunderscore intricado\textunderscore , nasalando-se a 2.^a sýllaba por influência da 1.^a, o que é phenómeno vulgar em phonética.
\section{Intrinsecado}
\begin{itemize}
\item {Grp. gram.:adj.}
\end{itemize}
\begin{itemize}
\item {Utilização:Des.}
\end{itemize}
O mesmo que \textunderscore intricado\textunderscore , obscuro. Cf. Pant. de Aveiro, \textunderscore Itiner.\textunderscore , 26 v.^o, (2.^a ed.)
\section{Intrinsecamente}
\begin{itemize}
\item {Grp. gram.:adv.}
\end{itemize}
De modo intrínseco.
Interiormente.
Intimamente.
\section{Intrínseco}
\begin{itemize}
\item {Grp. gram.:adj.}
\end{itemize}
\begin{itemize}
\item {Proveniência:(Lat. \textunderscore intrinsecus\textunderscore )}
\end{itemize}
Muito interior; íntimo; inherente.
\section{Intrita}
\begin{itemize}
\item {Grp. gram.:f.}
\end{itemize}
\begin{itemize}
\item {Proveniência:(Lat. \textunderscore intrita\textunderscore )}
\end{itemize}
Papas, com migas de pão.
\section{Intro...}
\begin{itemize}
\item {Grp. gram.:pref.}
\end{itemize}
O mesmo que \textunderscore intra...\textunderscore 
\section{Introdir}
\begin{itemize}
\item {Grp. gram.:v. t.}
\end{itemize}
\begin{itemize}
\item {Utilização:Ant.}
\end{itemize}
\begin{itemize}
\item {Proveniência:(Lat. \textunderscore introire\textunderscore )}
\end{itemize}
O mesmo que \textunderscore introduzir\textunderscore .
Meter á fôrça.--É do século XV.
\section{Introdução}
\begin{itemize}
\item {Grp. gram.:f.}
\end{itemize}
\begin{itemize}
\item {Proveniência:(Lat. \textunderscore introductio\textunderscore )}
\end{itemize}
Acto ou efeito de introduzir.
Importação.
Preâmbulo, prefácio.
Sinfonia de abertura.
\section{Introducção}
\begin{itemize}
\item {Grp. gram.:f.}
\end{itemize}
\begin{itemize}
\item {Proveniência:(Lat. \textunderscore introductio\textunderscore )}
\end{itemize}
Acto ou effeito de íntroduzir.
Importação.
Preâmbulo, prefácio.
Symphonia de abertura.
\section{Introductivo}
\begin{itemize}
\item {Grp. gram.:adj.}
\end{itemize}
\begin{itemize}
\item {Proveniência:(Do lat. \textunderscore introductus\textunderscore )}
\end{itemize}
Que serve de introducção ou comêço.
\section{Introductor}
\begin{itemize}
\item {Grp. gram.:adj.}
\end{itemize}
\begin{itemize}
\item {Grp. gram.:M.}
\end{itemize}
\begin{itemize}
\item {Proveniência:(Lat. \textunderscore introductor\textunderscore )}
\end{itemize}
Que introduz.
Indivíduo, que introduz ou apresenta alguém.
\section{Introductório}
\begin{itemize}
\item {Grp. gram.:adj.}
\end{itemize}
\begin{itemize}
\item {Proveniência:(Lat. \textunderscore introductorius\textunderscore )}
\end{itemize}
O mesmo que \textunderscore introductivo\textunderscore .
\section{Introdutivo}
\begin{itemize}
\item {Grp. gram.:adj.}
\end{itemize}
\begin{itemize}
\item {Proveniência:(Do lat. \textunderscore introductus\textunderscore )}
\end{itemize}
Que serve de introdução ou comêço.
\section{Introdutor}
\begin{itemize}
\item {Grp. gram.:adj.}
\end{itemize}
\begin{itemize}
\item {Grp. gram.:M.}
\end{itemize}
\begin{itemize}
\item {Proveniência:(Lat. \textunderscore introductor\textunderscore )}
\end{itemize}
Que introduz.
Indivíduo, que introduz ou apresenta alguém.
\section{Introdutório}
\begin{itemize}
\item {Grp. gram.:adj.}
\end{itemize}
\begin{itemize}
\item {Proveniência:(Lat. \textunderscore introductorius\textunderscore )}
\end{itemize}
O mesmo que \textunderscore introdutivo\textunderscore .
\section{Introduzir}
\begin{itemize}
\item {Grp. gram.:v. t.}
\end{itemize}
\begin{itemize}
\item {Utilização:Fig.}
\end{itemize}
\begin{itemize}
\item {Proveniência:(Lat. \textunderscore introducere\textunderscore )}
\end{itemize}
Levar para dentro, fazer entrar.
Deixar passar.
Importar: \textunderscore introduzir trigo exótico\textunderscore .
Apresentar a alguém.
Tornar adoptado: \textunderscore introduzir modas\textunderscore .
Causar: \textunderscore introduzir discórdias\textunderscore .
Estabelecer.
\section{Introito}
\begin{itemize}
\item {Grp. gram.:m.}
\end{itemize}
\begin{itemize}
\item {Proveniência:(Lat. \textunderscore introitus\textunderscore )}
\end{itemize}
Entrada; comêço.
Máquina para laminar, nas fábricas de fiação; laminador. Cf. \textunderscore Inquér. Industr.\textunderscore , p. II, l. II, 123.
\section{Intrometer}
\begin{itemize}
\item {Grp. gram.:v. t.}
\end{itemize}
\begin{itemize}
\item {Grp. gram.:V. p.}
\end{itemize}
\begin{itemize}
\item {Proveniência:(Lat. \textunderscore intromittere\textunderscore )}
\end{itemize}
Introduzir; intercalar.
Ingerir-se, intervir desapropositadamente; fazer-se metediço.
\section{Intrometido}
\begin{itemize}
\item {Grp. gram.:adj.}
\end{itemize}
\begin{itemize}
\item {Proveniência:(De \textunderscore intrometer\textunderscore )}
\end{itemize}
Atrevido; metediço.
\section{Intromissão}
\begin{itemize}
\item {Grp. gram.:f.}
\end{itemize}
\begin{itemize}
\item {Proveniência:(Do lat. \textunderscore intromissus\textunderscore )}
\end{itemize}
Acto ou effeito de intrometer.
\section{Intropelvímetro}
\begin{itemize}
\item {Grp. gram.:m.}
\end{itemize}
\begin{itemize}
\item {Proveniência:(Do lat. \textunderscore intro\textunderscore  + \textunderscore pelvis\textunderscore  + gr. \textunderscore metron\textunderscore )}
\end{itemize}
Instrumento cirúrgico, para medir o diâmetro interior da madre da mulher.
\section{Intibélia}
\begin{itemize}
\item {Grp. gram.:f.}
\end{itemize}
\begin{itemize}
\item {Proveniência:(Do lat. \textunderscore intybum\textunderscore )}
\end{itemize}
Gênero de plantas compostas, espécie de chicória.
\section{Introrso}
\begin{itemize}
\item {Grp. gram.:adj.}
\end{itemize}
\begin{itemize}
\item {Utilização:Bot.}
\end{itemize}
\begin{itemize}
\item {Proveniência:(Lat. \textunderscore introrsus\textunderscore )}
\end{itemize}
Voltado para dentro.
Diz-se especialmente da anthera, que tem a face voltada para dentro da flôr.
\section{Introsca}
\begin{itemize}
\item {Grp. gram.:m.}
\end{itemize}
\begin{itemize}
\item {Utilização:Bras}
\end{itemize}
\begin{itemize}
\item {Proveniência:(Do rad. de \textunderscore intruso\textunderscore )}
\end{itemize}
Homem intruso, intrometido.
\section{Introspecção}
\begin{itemize}
\item {Grp. gram.:f.}
\end{itemize}
\begin{itemize}
\item {Proveniência:(Lat. \textunderscore introspectio\textunderscore )}
\end{itemize}
Observação do interior.
\section{Introspectivo}
\begin{itemize}
\item {Grp. gram.:adj.}
\end{itemize}
\begin{itemize}
\item {Proveniência:(Do lat. \textunderscore introspectus\textunderscore )}
\end{itemize}
Que examina interiormente.
Que tem por objecto o interior ou o íntimo: \textunderscore observação introspectiva\textunderscore .
\section{Introversão}
\begin{itemize}
\item {Grp. gram.:f.}
\end{itemize}
\begin{itemize}
\item {Utilização:Fig.}
\end{itemize}
\begin{itemize}
\item {Proveniência:(Do lat. \textunderscore introversus\textunderscore )}
\end{itemize}
Qualidade ou estado do que é introrso.
Recolhimento de espírito; exame íntimo.
\section{Introverso}
\begin{itemize}
\item {Grp. gram.:adj.}
\end{itemize}
\begin{itemize}
\item {Proveniência:(Lat. \textunderscore introversus\textunderscore )}
\end{itemize}
Voltado para dentro.
Concentrado.
Absorto.
\section{Introverter}
\begin{itemize}
\item {Grp. gram.:v. t.}
\end{itemize}
\begin{itemize}
\item {Proveniência:(De \textunderscore intro...\textunderscore  + lat. \textunderscore vertere\textunderscore )}
\end{itemize}
Voltar para dentro; concentrar. Cf. Camillo, \textunderscore Quéda\textunderscore , 124; \textunderscore Perfil\textunderscore , 222.
\section{Introvertido}
\begin{itemize}
\item {Grp. gram.:adj.}
\end{itemize}
\begin{itemize}
\item {Proveniência:(De \textunderscore introverter\textunderscore )}
\end{itemize}
O mesmo que \textunderscore introverso\textunderscore . Cf. Latino, \textunderscore Humboldt\textunderscore , 503.
\section{Intrudo}
\begin{itemize}
\item {Grp. gram.:m.}
\end{itemize}
\begin{itemize}
\item {Utilização:Des.}
\end{itemize}
\begin{itemize}
\item {Proveniência:(Do lat. \textunderscore introitus\textunderscore )}
\end{itemize}
A parte interior ou côncava de uma abóbada; intradorso.
\section{Intrugir}
\begin{itemize}
\item {Grp. gram.:v. t.  e  i.}
\end{itemize}
\begin{itemize}
\item {Utilização:Gír.}
\end{itemize}
Comprehender.
Intrujar.
(Cp. \textunderscore intrujar\textunderscore )
\section{Intrujão}
\begin{itemize}
\item {Grp. gram.:m.  e  adj.}
\end{itemize}
\begin{itemize}
\item {Utilização:Pop.}
\end{itemize}
Aquelle que intruja.
\section{Intrujar}
\begin{itemize}
\item {Grp. gram.:v. t.}
\end{itemize}
\begin{itemize}
\item {Utilização:Gír.}
\end{itemize}
Burlar; lograr.
Desfrutar com astúcia.
(Cp. cast. \textunderscore antruejar\textunderscore , jogar o entrudo)
\section{Intrujice}
\begin{itemize}
\item {Grp. gram.:f.}
\end{itemize}
Acto de intrujar.
\section{Intrusamente}
\begin{itemize}
\item {Grp. gram.:adv.}
\end{itemize}
De modo intruso.
\section{Intrusão}
\begin{itemize}
\item {Grp. gram.:f.}
\end{itemize}
Estado daquelle ou daquillo que é intruso.
Usurpação; posse illegal.
\section{Intruso}
\begin{itemize}
\item {Grp. gram.:adj.}
\end{itemize}
\begin{itemize}
\item {Grp. gram.:M.}
\end{itemize}
\begin{itemize}
\item {Proveniência:(Lat. \textunderscore intrusus\textunderscore )}
\end{itemize}
Illegalmente empossado ou investido num cargo ou dignidade.
Usurpador: \textunderscore os reis intrusos\textunderscore .
Metediço, intrometido.
Indivíduo, que illegalmente se empossa de um cargo ou dignidade.
Indivíduo, que se intromete em lugar ou coisas, para que não foi chamado.
\section{Intuição}
\begin{itemize}
\item {fónica:tu-i}
\end{itemize}
\begin{itemize}
\item {Grp. gram.:f.}
\end{itemize}
\begin{itemize}
\item {Proveniência:(Lat. \textunderscore intuitio\textunderscore )}
\end{itemize}
Acto de ver.
Primeiro lance de olhos.
Percepção rápida.
Conhecimento claro.
Presentimento.
Visão beatífica.
\section{Intuitivamente}
\begin{itemize}
\item {fónica:tu-i}
\end{itemize}
\begin{itemize}
\item {Grp. gram.:adv.}
\end{itemize}
De modo intuitivo.
Claramente, evidentemente.
\section{Intuitivo}
\begin{itemize}
\item {fónica:tu-i}
\end{itemize}
\begin{itemize}
\item {Grp. gram.:adj.}
\end{itemize}
\begin{itemize}
\item {Proveniência:(De \textunderscore intuito\textunderscore )}
\end{itemize}
Relativo a intuição.
Que se percebe facilmente.
Percebido clara e directamente.
Evidente, incontestável.
\section{Intuito}
\begin{itemize}
\item {Grp. gram.:m.}
\end{itemize}
\begin{itemize}
\item {Proveniência:(Lat. \textunderscore intuitus\textunderscore )}
\end{itemize}
Aquillo que se tem em vista.
Intento.
Plano.
Escopo, fim.
\section{Intulá}
\begin{itemize}
\item {Grp. gram.:m.}
\end{itemize}
Planta trepadeira da Guiné, de fôlhas medicinaes.
\section{Intumecer}
\textunderscore v. t.\textunderscore , \textunderscore i.\textunderscore  e \textunderscore p.\textunderscore  (e der.)(V.entumecer)
\section{Inturgescência}
\begin{itemize}
\item {Grp. gram.:f.}
\end{itemize}
Qualidade ou estado de inturgescente.
\section{Inturgescente}
\begin{itemize}
\item {Grp. gram.:adj.}
\end{itemize}
\begin{itemize}
\item {Proveniência:(Lat. \textunderscore inturgescens\textunderscore )}
\end{itemize}
O mesmo que \textunderscore turgescente\textunderscore .
\section{Inturgescer}
\begin{itemize}
\item {Grp. gram.:v. t. ,  i.  e  p.}
\end{itemize}
\begin{itemize}
\item {Proveniência:(Lat. \textunderscore inturgescere\textunderscore )}
\end{itemize}
O mesmo que \textunderscore turgescer\textunderscore .
\section{Intuscepção}
\begin{itemize}
\item {Grp. gram.:f.}
\end{itemize}
Acção de ingerir e assimilar os alimentos.
(Por \textunderscore intussuscepção\textunderscore , do lat. \textunderscore intus\textunderscore  + \textunderscore susceptio\textunderscore )
\section{Intuspecção}
\begin{itemize}
\item {Grp. gram.:f.}
\end{itemize}
\begin{itemize}
\item {Proveniência:(Do lat. \textunderscore intus\textunderscore  + \textunderscore spectus\textunderscore )}
\end{itemize}
Observação íntima do próprio observador.
Conhecimento de si próprio.
\section{Intuspectivamnente}
\begin{itemize}
\item {Grp. gram.:adv.}
\end{itemize}
De modo intuspectivo.
\section{Intuspectivo}
\begin{itemize}
\item {Grp. gram.:adj.}
\end{itemize}
Relativo a intuspecção.
\section{Intuspecto}
\begin{itemize}
\item {Grp. gram.:m.}
\end{itemize}
O mesmo que \textunderscore intuspecção\textunderscore .
\section{Intussuscepção}
\begin{itemize}
\item {Grp. gram.:f.}
\end{itemize}
\begin{itemize}
\item {Utilização:Hist. Nat.}
\end{itemize}
\begin{itemize}
\item {Proveniência:(Do lat. \textunderscore intus\textunderscore  + \textunderscore susceptio\textunderscore )}
\end{itemize}
Penetração de novas moléculas na intimidado do organismo, entre as moléculas já existentes, no acto do crescimento: \textunderscore os corpos vivos crescem por intussuscepção\textunderscore .
Cp. \textunderscore intuscepção\textunderscore .
\section{Intybéllia}
\begin{itemize}
\item {Grp. gram.:f.}
\end{itemize}
\begin{itemize}
\item {Proveniência:(Do lat. \textunderscore intybum\textunderscore )}
\end{itemize}
Gênero de plantas compostas, espécie de chicória.
\section{Inubia}
\begin{itemize}
\item {Grp. gram.:f.}
\end{itemize}
\begin{itemize}
\item {Utilização:Bras}
\end{itemize}
Buzina dos Índios.
\section{Inúbil}
\begin{itemize}
\item {Grp. gram.:adj.}
\end{itemize}
\begin{itemize}
\item {Proveniência:(De \textunderscore in...\textunderscore  + \textunderscore núbil\textunderscore )}
\end{itemize}
Que não é núbil; que ainda não está em idade de casar.
\section{Ínubo}
\begin{itemize}
\item {Grp. gram.:adj.}
\end{itemize}
\begin{itemize}
\item {Proveniência:(Lat. \textunderscore innubus\textunderscore )}
\end{itemize}
O mesmo que \textunderscore inupto\textunderscore .
\section{Ínula}
\begin{itemize}
\item {Grp. gram.:f.}
\end{itemize}
O mesmo que \textunderscore énula-campana\textunderscore .
\section{Inúleas}
\begin{itemize}
\item {Grp. gram.:f. pl.}
\end{itemize}
\begin{itemize}
\item {Proveniência:(De \textunderscore ínula\textunderscore )}
\end{itemize}
Tribo de synanthéreas, no systema de Cassíni.
\section{Inulina}
\begin{itemize}
\item {Grp. gram.:f.}
\end{itemize}
\begin{itemize}
\item {Proveniência:(De \textunderscore ínula\textunderscore )}
\end{itemize}
O mesmo que \textunderscore dahlina\textunderscore .
\section{Inulto}
\begin{itemize}
\item {Grp. gram.:adj.}
\end{itemize}
\begin{itemize}
\item {Proveniência:(Lat. \textunderscore inultus\textunderscore )}
\end{itemize}
Que se não vingou; que não teve desforra: \textunderscore afronta inulta\textunderscore .
\section{Inultrapassável}
\begin{itemize}
\item {Grp. gram.:adj.}
\end{itemize}
\begin{itemize}
\item {Utilização:Neol.}
\end{itemize}
\begin{itemize}
\item {Proveniência:(De \textunderscore in...\textunderscore  + \textunderscore ultrapassar\textunderscore )}
\end{itemize}
Que se não póde ultrapassar.
\section{Inumação}
\begin{itemize}
\item {Grp. gram.:f.}
\end{itemize}
Acto ou efeito de inumar.
\section{Inumanamente}
\begin{itemize}
\item {Grp. gram.:adv.}
\end{itemize}
De modo inumano.
\section{Inumanidade}
\begin{itemize}
\item {Grp. gram.:f.}
\end{itemize}
\begin{itemize}
\item {Proveniência:(Lat. \textunderscore inhumanitas\textunderscore )}
\end{itemize}
O mesmo que \textunderscore desumanidade\textunderscore .
\section{Inumano}
\begin{itemize}
\item {Grp. gram.:adj.}
\end{itemize}
\begin{itemize}
\item {Utilização:Ant.}
\end{itemize}
\begin{itemize}
\item {Proveniência:(Lat. \textunderscore inhumanus\textunderscore )}
\end{itemize}
O mesmo que \textunderscore desumano\textunderscore .
Sobrehumano.
\section{Inumar}
\begin{itemize}
\item {Grp. gram.:v. t.}
\end{itemize}
\begin{itemize}
\item {Proveniência:(Lat. \textunderscore inhumare\textunderscore )}
\end{itemize}
Enterrar, cobrir de terra; sepultar.
\section{Inumerabilidade}
\begin{itemize}
\item {Grp. gram.:f.}
\end{itemize}
\begin{itemize}
\item {Proveniência:(Lat. \textunderscore innumerabilitas\textunderscore )}
\end{itemize}
Qualidade daquilo que é inumerável.
\section{Inumerável}
\begin{itemize}
\item {Grp. gram.:adj.}
\end{itemize}
\begin{itemize}
\item {Proveniência:(Lat. \textunderscore innumerabilis\textunderscore )}
\end{itemize}
Que não é numerável.
Que se não póde numerar ou contar.
Infinito em número: \textunderscore as estrêlas inumeráveis\textunderscore .
Extraordinariamente numeroso: \textunderscore multidão inumerável\textunderscore .
\section{Inumeravelmente}
\begin{itemize}
\item {Grp. gram.:adv.}
\end{itemize}
De modo inumerável.
\section{Inúmero}
\begin{itemize}
\item {Grp. gram.:adj.}
\end{itemize}
\begin{itemize}
\item {Proveniência:(Lat. \textunderscore innumerus\textunderscore )}
\end{itemize}
O mesmo que \textunderscore inumerável\textunderscore .
\section{Inumeroso}
\begin{itemize}
\item {Grp. gram.:adj.}
\end{itemize}
O mesmo que \textunderscore inumerável\textunderscore .
\section{Inundação}
\begin{itemize}
\item {Grp. gram.:f.}
\end{itemize}
\begin{itemize}
\item {Utilização:Ext.}
\end{itemize}
\begin{itemize}
\item {Utilização:Fig.}
\end{itemize}
\begin{itemize}
\item {Proveniência:(Lat. \textunderscore inundatio\textunderscore )}
\end{itemize}
Acto ou effeito de inundar.
Grande affluência de pessôas ou animaes.
Invasão bellicosa.
\section{Inundado}
\begin{itemize}
\item {Grp. gram.:m.}
\end{itemize}
\begin{itemize}
\item {Proveniência:(De \textunderscore inundar\textunderscore )}
\end{itemize}
Indivíduo, prejudicado por inundações.
\section{Inundante}
\begin{itemize}
\item {Grp. gram.:adj.}
\end{itemize}
\begin{itemize}
\item {Proveniência:(Lat. \textunderscore inundans\textunderscore )}
\end{itemize}
Que inunda.
\section{Inundar}
\begin{itemize}
\item {Grp. gram.:v. t.}
\end{itemize}
\begin{itemize}
\item {Utilização:Fig.}
\end{itemize}
\begin{itemize}
\item {Proveniência:(Lat. \textunderscore inundare\textunderscore )}
\end{itemize}
Cobrir com águas que trasbordam; alagar.
Banhar: \textunderscore o calor inundava-o de suor\textunderscore .
Invadír em tumulto.
Encher de estranhos.
Saciar.
Espalhar.
\section{Inundável}
\begin{itemize}
\item {Grp. gram.:adj.}
\end{itemize}
Que se póde inundar.
\section{Inupto}
\begin{itemize}
\item {Grp. gram.:adj.}
\end{itemize}
\begin{itemize}
\item {Proveniência:(Lat. \textunderscore inuptus\textunderscore )}
\end{itemize}
O mesmo que \textunderscore solteiro\textunderscore .
Que não é casado; que está solteiro ou é celibatario.
\section{Inurbanamente}
\begin{itemize}
\item {Grp. gram.:adv.}
\end{itemize}
De modo inurbano.
\section{Inurbanidade}
\begin{itemize}
\item {Grp. gram.:f.}
\end{itemize}
\begin{itemize}
\item {Proveniência:(Lat. \textunderscore inurbanitas\textunderscore )}
\end{itemize}
Falta de urbanidade.
\section{Inurbano}
\begin{itemize}
\item {Grp. gram.:adj.}
\end{itemize}
\begin{itemize}
\item {Proveniência:(Lat. \textunderscore inurbanus\textunderscore )}
\end{itemize}
Que não é urbano; descortês.
\section{Inusitadamente}
\begin{itemize}
\item {Grp. gram.:adv.}
\end{itemize}
De modo inusitado.
\section{Inusitado}
\begin{itemize}
\item {Grp. gram.:adj.}
\end{itemize}
\begin{itemize}
\item {Proveniência:(Lat. \textunderscore inusitatus\textunderscore )}
\end{itemize}
Não usado; desconhecido.
Esquisito. Cf. \textunderscore Lusiadas\textunderscore , II, 107.
\section{Inútil}
\begin{itemize}
\item {Grp. gram.:adj.}
\end{itemize}
\begin{itemize}
\item {Proveniência:(Lat. \textunderscore inutilis\textunderscore )}
\end{itemize}
Não útil.
Vão.
Desnecessário.
Improfícuo.
Frustrado; estéril.
\section{Inutilidade}
\begin{itemize}
\item {Grp. gram.:f.}
\end{itemize}
\begin{itemize}
\item {Proveniência:(Lat. \textunderscore inutilitas\textunderscore )}
\end{itemize}
Qualidade de inútil.
Falta de utilidade; incapacidade.
\section{Inutilizar}
\begin{itemize}
\item {Grp. gram.:v. t.}
\end{itemize}
Tornar inútil; frustrar: \textunderscore inutilizar esforços\textunderscore .
\section{Inutilmente}
\begin{itemize}
\item {Grp. gram.:adv.}
\end{itemize}
\begin{itemize}
\item {Proveniência:(De \textunderscore inútil\textunderscore )}
\end{itemize}
Sem utilidade.
\section{Inutrível}
\begin{itemize}
\item {Grp. gram.:adj.}
\end{itemize}
\begin{itemize}
\item {Proveniência:(De \textunderscore in...\textunderscore  + \textunderscore nutrível\textunderscore )}
\end{itemize}
Que não nutre ou que nutre pouco; que não é nutritivo.
\section{Invadeável}
\begin{itemize}
\item {Grp. gram.:adj.}
\end{itemize}
\begin{itemize}
\item {Proveniência:(De \textunderscore in...\textunderscore  + \textunderscore vadeável\textunderscore )}
\end{itemize}
Que não é vadeável.
\section{Invadir}
\begin{itemize}
\item {Grp. gram.:v. t.}
\end{itemize}
\begin{itemize}
\item {Utilização:Fig.}
\end{itemize}
\begin{itemize}
\item {Proveniência:(Lat. \textunderscore invadere\textunderscore )}
\end{itemize}
Entrar em.
Entrar á força em: \textunderscore os Franceses invadiram Portugal\textunderscore .
Occupar violentamente.
Conquistar.
Diffundir-se em: \textunderscore a peste invadiu a Rússia\textunderscore .
\section{Invaginação}
\begin{itemize}
\item {Grp. gram.:f.}
\end{itemize}
\begin{itemize}
\item {Proveniência:(De \textunderscore invaginar\textunderscore )}
\end{itemize}
Crescimento de órgãos vegeates, á maneira de bainha.
Operação cirúrgica, que consiste em introduzir uma na outra as extremidades do intestino cortado, a fim de se restabelecer a continuidade do canal intestinal.
Accidente pathológico, que consiste na entrada de uma porção de intestino em outra.
\section{Invaginante}
\begin{itemize}
\item {Grp. gram.:adj.}
\end{itemize}
Que se invagina.
\section{Invaginar}
\begin{itemize}
\item {Grp. gram.:v. t.}
\end{itemize}
\begin{itemize}
\item {Grp. gram.:V. p.}
\end{itemize}
\begin{itemize}
\item {Proveniência:(Do lat. \textunderscore vagina\textunderscore )}
\end{itemize}
Ligar por meio de invaginação.
Soffrer o accidente da invaginação.
\section{Invalecer}
\begin{itemize}
\item {Grp. gram.:v. i.}
\end{itemize}
\begin{itemize}
\item {Utilização:Des.}
\end{itemize}
\begin{itemize}
\item {Proveniência:(Lat. \textunderscore invalescere\textunderscore )}
\end{itemize}
Reforçar-se; adquirir fôrças.
\section{Invalescer}
\begin{itemize}
\item {Grp. gram.:v. i.}
\end{itemize}
\begin{itemize}
\item {Utilização:Des.}
\end{itemize}
\begin{itemize}
\item {Proveniência:(Lat. \textunderscore invalescere\textunderscore )}
\end{itemize}
Reforçar-se; adquirir fôrças.
\section{Invalidação}
\begin{itemize}
\item {Grp. gram.:f.}
\end{itemize}
Acto ou effeito de invalidar.
Annullação.
\section{Invalidade}
\begin{itemize}
\item {Grp. gram.:f.}
\end{itemize}
\begin{itemize}
\item {Proveniência:(De \textunderscore in...\textunderscore  + \textunderscore validade\textunderscore )}
\end{itemize}
Falta do validade; nullidade.
\section{Invalidamente}
\begin{itemize}
\item {Grp. gram.:adv.}
\end{itemize}
\begin{itemize}
\item {Proveniência:(De \textunderscore inválido\textunderscore )}
\end{itemize}
Sem valor.
Sem validez, sem saúde.
\section{Invalidar}
\begin{itemize}
\item {Grp. gram.:v. t.}
\end{itemize}
\begin{itemize}
\item {Proveniência:(De \textunderscore in...\textunderscore  + \textunderscore validar\textunderscore )}
\end{itemize}
Tornar inválido.
Annullar.
Tirar o crédito ou a importância a: \textunderscore invalidar augmentos\textunderscore .
Inhabilitar; inutilizar.
\section{Invalidez}
\begin{itemize}
\item {Grp. gram.:f.}
\end{itemize}
Qualidade de inválido.
Invalidade.
\section{Inválido}
\begin{itemize}
\item {Grp. gram.:adj.}
\end{itemize}
\begin{itemize}
\item {Utilização:Fig.}
\end{itemize}
\begin{itemize}
\item {Grp. gram.:M.}
\end{itemize}
\begin{itemize}
\item {Proveniência:(Lat. \textunderscore invalidus\textunderscore )}
\end{itemize}
Que não é válido.
Débil; enfermo.
Que não tem valor; nullo.
Indivíduo, impossibilitado de trabalhar ou de exercer a sua profissão: \textunderscore asilo de inválidos\textunderscore .
\section{Invariabilidade}
\begin{itemize}
\item {Grp. gram.:f.}
\end{itemize}
Qualidade de invariável.
\section{Invariadamente}
\begin{itemize}
\item {Grp. gram.:adv.}
\end{itemize}
O mesmo que \textunderscore invariavelmente\textunderscore .
\section{Invariável}
\begin{itemize}
\item {Grp. gram.:adj.}
\end{itemize}
\begin{itemize}
\item {Utilização:Gram.}
\end{itemize}
\begin{itemize}
\item {Proveniência:(De \textunderscore in...\textunderscore  + \textunderscore variável\textunderscore )}
\end{itemize}
Não variável; constante.
Indeclinável, ou que não tem flexão, (falando-se de vocábulos).
\section{Invariavelmente}
\begin{itemize}
\item {Grp. gram.:adv.}
\end{itemize}
De modo invariável.
Da mesma fórma; constantemente.
\section{Invasão}
\begin{itemize}
\item {Grp. gram.:f.}
\end{itemize}
\begin{itemize}
\item {Proveniência:(Lat. \textunderscore invasio\textunderscore )}
\end{itemize}
Acto ou effeito de invadir: \textunderscore a invasão francesa\textunderscore .
\section{Invasivo}
\begin{itemize}
\item {Grp. gram.:adj.}
\end{itemize}
\begin{itemize}
\item {Proveniência:(Do lat. \textunderscore invasus\textunderscore )}
\end{itemize}
Relativo a invasão.
Que hostiliza.
\section{Invasor}
\begin{itemize}
\item {Grp. gram.:adj.}
\end{itemize}
\begin{itemize}
\item {Grp. gram.:M.}
\end{itemize}
\begin{itemize}
\item {Proveniência:(Lat. \textunderscore invasor\textunderscore )}
\end{itemize}
Que invade; que escala; que conquista.
Aquelle que invade.
\section{Invectar}
\begin{itemize}
\item {Grp. gram.:v. t.}
\end{itemize}
\begin{itemize}
\item {Proveniência:(Do lat. \textunderscore invectus\textunderscore )}
\end{itemize}
O mesmo que \textunderscore invectivar\textunderscore .
\section{Invectiva}
\begin{itemize}
\item {Grp. gram.:f.}
\end{itemize}
\begin{itemize}
\item {Proveniência:(De \textunderscore invectivo\textunderscore )}
\end{itemize}
Acto ou effeito de invectivar.
\section{Invectivador}
\begin{itemize}
\item {Grp. gram.:m.  e  adj.}
\end{itemize}
O que invectiva.
\section{Invectivar}
\begin{itemize}
\item {Grp. gram.:v. t.}
\end{itemize}
\begin{itemize}
\item {Grp. gram.:V. i.}
\end{itemize}
\begin{itemize}
\item {Proveniência:(De \textunderscore invectiva\textunderscore )}
\end{itemize}
Dirigir injúrias a.
Atacar violentamente.
Censurar com acrimónia.
Tratar alguém injuriosamente.
\section{Invectivo}
\begin{itemize}
\item {Grp. gram.:adj.}
\end{itemize}
\begin{itemize}
\item {Proveniência:(Lat. \textunderscore invectivus\textunderscore )}
\end{itemize}
Injurioso; agressivo.
\section{Invedável}
\begin{itemize}
\item {Grp. gram.:adj.}
\end{itemize}
\begin{itemize}
\item {Proveniência:(De \textunderscore in...\textunderscore  + \textunderscore vedável\textunderscore )}
\end{itemize}
Que não é vedável; que se não póde vedar.
\section{Inveja}
\begin{itemize}
\item {Grp. gram.:f.}
\end{itemize}
\begin{itemize}
\item {Grp. gram.:Pl. Loc. adv.}
\end{itemize}
\begin{itemize}
\item {Proveniência:(Do lat. \textunderscore invidia\textunderscore )}
\end{itemize}
Tristeza ou desgôsto pela prosperidade ou fortuna alheia.
Desejo excessivo de possuír exclusivamente o bem de outrem.
Objecto invejado.
\textunderscore Ás invejas\textunderscore , á porfia:«\textunderscore todos os portugueses ás invejas entraram no lavor dos muros.\textunderscore »Filinto, \textunderscore D. Man.\textunderscore , II, 91.
\section{Invejando}
\begin{itemize}
\item {Grp. gram.:adj.}
\end{itemize}
Que é digno de se invejar; que se póde invejar:«\textunderscore invejandos pórticos.\textunderscore »Filinto. Cf. Júl. Dinis, \textunderscore Morgadinha\textunderscore , 149.
\section{Invejar}
\begin{itemize}
\item {Grp. gram.:v. t.}
\end{itemize}
Têr inveja de.
Cubiçar ardentemente (aquillo que pertence a outrem): \textunderscore invejar a mulhér do próximo\textunderscore .
\section{Invejável}
\begin{itemize}
\item {Grp. gram.:adj.}
\end{itemize}
Que se póde invejar; digno de muito apreço; precioso: \textunderscore talento invejável\textunderscore .
\section{Invejidade}
\begin{itemize}
\item {Grp. gram.:f.}
\end{itemize}
\begin{itemize}
\item {Utilização:Prov.}
\end{itemize}
\begin{itemize}
\item {Utilização:trasm.}
\end{itemize}
O mesmo que \textunderscore inveja\textunderscore .
\section{Invejoso}
\begin{itemize}
\item {Grp. gram.:adj.}
\end{itemize}
\begin{itemize}
\item {Grp. gram.:M.}
\end{itemize}
Que tem inveja.
Indivíduo que tem inveja.
\section{Invenal}
\begin{itemize}
\item {Grp. gram.:adj.}
\end{itemize}
\begin{itemize}
\item {Proveniência:(Lat. \textunderscore invenalis\textunderscore )}
\end{itemize}
Que se não póde vender.
\section{Invenção}
\begin{itemize}
\item {Grp. gram.:f.}
\end{itemize}
\begin{itemize}
\item {Utilização:Fig.}
\end{itemize}
\begin{itemize}
\item {Proveniência:(Lat. \textunderscore inventio\textunderscore )}
\end{itemize}
Acto ou effeito de inventar.
Faculdade inventiva.
Coisa inventada.
Parte da Rhetórica, que ensina a procurar os meios de convencer, deleitar e persuadir.
Acto de achar ou encontrar.
Achada.
Astúcia.
Manha; engano.
Fábula.
\section{Invencibilidade}
\begin{itemize}
\item {Grp. gram.:f.}
\end{itemize}
Qualidade de invencível.
\section{Invencionar}
\begin{itemize}
\item {Grp. gram.:v. i.}
\end{itemize}
\begin{itemize}
\item {Proveniência:(Do lat. \textunderscore inventio\textunderscore )}
\end{itemize}
Adornar artificiosamente.
\section{Invencioneiro}
\begin{itemize}
\item {Grp. gram.:adj.}
\end{itemize}
\begin{itemize}
\item {Utilização:Fig.}
\end{itemize}
\begin{itemize}
\item {Grp. gram.:M.}
\end{itemize}
\begin{itemize}
\item {Proveniência:(Do lat. \textunderscore inventio\textunderscore )}
\end{itemize}
Extravagante; esquisito; affectado.
Enganoso.
Astucioso; embusteiro.
Indivíduo invencioneiro.
\section{Invencionice}
\begin{itemize}
\item {Grp. gram.:f.}
\end{itemize}
\begin{itemize}
\item {Proveniência:(Do lat. \textunderscore inventio\textunderscore )}
\end{itemize}
Enrêdo.
Embuste.
Acto ou dito invencioneiro.
\section{Invencível}
\begin{itemize}
\item {Grp. gram.:adj.}
\end{itemize}
\begin{itemize}
\item {Proveniência:(Lat. \textunderscore invencibilis\textunderscore )}
\end{itemize}
Que não póde sêr vencido.
Inexequível.
Irresistível.
Que se não póde eliminar ou fazer desapparecer: \textunderscore discórdia invencível\textunderscore .
\section{Invencivelmente}
\begin{itemize}
\item {Grp. gram.:adv.}
\end{itemize}
De modo invencível.
\section{Invendível}
\begin{itemize}
\item {Grp. gram.:adj.}
\end{itemize}
\begin{itemize}
\item {Proveniência:(Lat. \textunderscore invendibilis\textunderscore )}
\end{itemize}
Que se não póde vender.
\section{Inventador}
\begin{itemize}
\item {Grp. gram.:m.}
\end{itemize}
\begin{itemize}
\item {Utilização:P. us.}
\end{itemize}
O mesmo que \textunderscore inventor\textunderscore .
\section{Inventar}
\begin{itemize}
\item {Grp. gram.:v. t.}
\end{itemize}
\begin{itemize}
\item {Proveniência:(De \textunderscore invento\textunderscore )}
\end{itemize}
Criar na imaginação.
Sêr o primeiro em têr a ideia de: \textunderscore quem inventou a pólvora?\textunderscore 
Iniciar; inaugurar; idear.
Contar falsamente.
Urdir, tramar: \textunderscore inventar histórias\textunderscore .
Descobrir.
\section{Inventariação}
\begin{itemize}
\item {Grp. gram.:f.}
\end{itemize}
Acto de inventariar.
\section{Inventariante}
\begin{itemize}
\item {Grp. gram.:m.  e  adj.}
\end{itemize}
\begin{itemize}
\item {Proveniência:(De \textunderscore inventariar\textunderscore )}
\end{itemize}
Pessôa que inventaria.
Aquelle que deu a relação dos bens inventariados.
\section{Inventariar}
\begin{itemize}
\item {Grp. gram.:v. t.}
\end{itemize}
\begin{itemize}
\item {Utilização:Fig.}
\end{itemize}
Fazer inventário de.
Registar.
Relacionar; catalogar.
Descrever minuciosamente.
\section{Inventário}
\begin{itemize}
\item {Grp. gram.:m.}
\end{itemize}
\begin{itemize}
\item {Proveniência:(Lat. \textunderscore inventarium\textunderscore )}
\end{itemize}
Relação ou registo dos bens, deixados por alguém que morreu.
Relação de bens sequestrados.
Relação de bens.
Relação; catálogo.
Descripção minuciosa.
Enumeração de coisas, mais ou menos longa.
\section{Inventiva}
\begin{itemize}
\item {Grp. gram.:f.}
\end{itemize}
\begin{itemize}
\item {Proveniência:(De \textunderscore inventivo\textunderscore )}
\end{itemize}
Faculdade de inventar; imaginação.
Invento.
\section{Inventivo}
\begin{itemize}
\item {Grp. gram.:adj.}
\end{itemize}
\begin{itemize}
\item {Proveniência:(Do lat. \textunderscore inventus\textunderscore )}
\end{itemize}
Relativo a invenção.
Em que há engenho.
Que revela imaginação viva.
\section{Invento}
\begin{itemize}
\item {Grp. gram.:m.}
\end{itemize}
\begin{itemize}
\item {Proveniência:(Lat. \textunderscore inventum\textunderscore )}
\end{itemize}
Coisa inventada; invenção.
\section{Inventor}
\begin{itemize}
\item {Grp. gram.:adj.}
\end{itemize}
\begin{itemize}
\item {Grp. gram.:M.}
\end{itemize}
\begin{itemize}
\item {Proveniência:(Lat. \textunderscore inventor\textunderscore )}
\end{itemize}
Inventivo.
Aquelle que inventa.
Aquelle que é engenhoso.
Autor; inaugurador.
\section{Inventriz}
\begin{itemize}
\item {Grp. gram.:f.  e  adj. f.}
\end{itemize}
(Flexão fem. de \textunderscore inventor\textunderscore ). Cf. Castilho, \textunderscore Fastos\textunderscore , III, 171.
\section{Inverecundo}
\begin{itemize}
\item {Grp. gram.:adj.}
\end{itemize}
\begin{itemize}
\item {Utilização:Des.}
\end{itemize}
Que não tem vergonha, descarado. Cf. Brás L. de Abreu, \textunderscore Portugal Méd.\textunderscore , 337.
\section{Inverificável}
\begin{itemize}
\item {Grp. gram.:adj.}
\end{itemize}
\begin{itemize}
\item {Proveniência:(De \textunderscore in...\textunderscore  + \textunderscore verificável\textunderscore )}
\end{itemize}
Não verificável; que não póde verificar-se ou que difficilmente se verifica.
\section{Inverisímil}
\begin{itemize}
\item {fónica:si}
\end{itemize}
\begin{itemize}
\item {Grp. gram.:adj.}
\end{itemize}
O mesmo que \textunderscore inverosímil\textunderscore , etc.
\section{Inverissímil}
\begin{itemize}
\item {Grp. gram.:adj.}
\end{itemize}
O mesmo que \textunderscore inverosímil\textunderscore , etc.
\section{Inverna}
\begin{itemize}
\item {Grp. gram.:f.}
\end{itemize}
\begin{itemize}
\item {Utilização:Pop.}
\end{itemize}
O mesmo que \textunderscore invernada\textunderscore .
\section{Invernáculo}
\begin{itemize}
\item {Grp. gram.:adj.}
\end{itemize}
\begin{itemize}
\item {Proveniência:(De \textunderscore in...\textunderscore  + \textunderscore vernáculo\textunderscore )}
\end{itemize}
Que não é vernáculo.
\section{Invernada}
\begin{itemize}
\item {Grp. gram.:f.}
\end{itemize}
\begin{itemize}
\item {Utilização:Bras}
\end{itemize}
\begin{itemize}
\item {Proveniência:(De \textunderscore inverno\textunderscore )}
\end{itemize}
Invernia; duração do tempo invernoso.
Curral de novílhos para engorda.
\section{Invernadoiro}
\begin{itemize}
\item {Grp. gram.:m.}
\end{itemize}
\begin{itemize}
\item {Utilização:Bot.}
\end{itemize}
\begin{itemize}
\item {Proveniência:(De \textunderscore invernar\textunderscore )}
\end{itemize}
Lugar apropriado para nelle se passar o inverno.
Parte das plantas, que abrigam os renovos no inverno.
\section{Invernadouro}
\begin{itemize}
\item {Grp. gram.:m.}
\end{itemize}
\begin{itemize}
\item {Utilização:Bot.}
\end{itemize}
\begin{itemize}
\item {Proveniência:(De \textunderscore invernar\textunderscore )}
\end{itemize}
Lugar apropriado para nelle se passar o inverno.
Parte das plantas, que abrigam os renovos no inverno.
\section{Invernal}
\begin{itemize}
\item {Grp. gram.:adj.}
\end{itemize}
\begin{itemize}
\item {Proveniência:(Do lat. \textunderscore hibernalis\textunderscore )}
\end{itemize}
Relativo ao inverno.
\section{Invernar}
\begin{itemize}
\item {Grp. gram.:v. i.}
\end{itemize}
\begin{itemize}
\item {Proveniência:(Lat. \textunderscore hibernare\textunderscore )}
\end{itemize}
Passar o inverno.
Estar nos quartéis de inverno; hibernar.
Fazer mau tempo: \textunderscore êste anno, invernou muito\textunderscore .
\section{Inverneira}
\begin{itemize}
\item {Grp. gram.:f.}
\end{itemize}
\begin{itemize}
\item {Utilização:Prov.}
\end{itemize}
\begin{itemize}
\item {Utilização:minh.}
\end{itemize}
\begin{itemize}
\item {Proveniência:(De \textunderscore inverno\textunderscore )}
\end{itemize}
O mesmo que \textunderscore invernia\textunderscore .
Variedade de pêra portuguesa.
Habitação, situada em valle profundo e abrigado das tormentas.
\section{Invernia}
\begin{itemize}
\item {Grp. gram.:f.}
\end{itemize}
Inverno rigoroso.
\section{Inverniço}
\begin{itemize}
\item {Grp. gram.:adj.}
\end{itemize}
Próprio do inverno.
Que se come de inverno.
Que cresce de inverno.
\section{Invernista}
\begin{itemize}
\item {Grp. gram.:m.}
\end{itemize}
\begin{itemize}
\item {Utilização:Bras}
\end{itemize}
\begin{itemize}
\item {Proveniência:(De \textunderscore inverno\textunderscore )}
\end{itemize}
Aquelle que proporciona campos para invernada de gados.
\section{Inverno}
\begin{itemize}
\item {Grp. gram.:m.}
\end{itemize}
\begin{itemize}
\item {Utilização:Fig.}
\end{itemize}
\begin{itemize}
\item {Proveniência:(Do lat. \textunderscore hibernus\textunderscore )}
\end{itemize}
Uma das quatro estações do anno, entre o outono e a primavera.
Tempo chuvoso e frio.
Velhice.
\section{Invernoso}
\begin{itemize}
\item {Grp. gram.:adj.}
\end{itemize}
Relativo ao inverno; próprio do inverno: \textunderscore tempo invernoso\textunderscore .
\section{Inverosímil}
\begin{itemize}
\item {fónica:si}
\end{itemize}
\begin{itemize}
\item {Grp. gram.:m.  e  adj.}
\end{itemize}
\begin{itemize}
\item {Proveniência:(De \textunderscore in...\textunderscore  + \textunderscore verosímil\textunderscore )}
\end{itemize}
Aquillo que não é verosímil.
Inacreditável.
\section{Inverossímil}
\begin{itemize}
\item {Grp. gram.:m.  e  adj.}
\end{itemize}
\begin{itemize}
\item {Proveniência:(De \textunderscore in...\textunderscore  + \textunderscore verosímil\textunderscore )}
\end{itemize}
Aquillo que não é verosímil.
Inacreditável.
\section{Inverosimilhança}
\begin{itemize}
\item {fónica:si}
\end{itemize}
\begin{itemize}
\item {Grp. gram.:f.}
\end{itemize}
\begin{itemize}
\item {Proveniência:(De \textunderscore in...\textunderscore  + \textunderscore verosimilhança\textunderscore )}
\end{itemize}
Falta de verosimilhança.
\section{Inverosimilmente}
\begin{itemize}
\item {fónica:si}
\end{itemize}
\begin{itemize}
\item {Grp. gram.:adv.}
\end{itemize}
De modo inverosímil.
\section{Inverossimilhança}
\begin{itemize}
\item {Grp. gram.:f.}
\end{itemize}
\begin{itemize}
\item {Proveniência:(De \textunderscore in...\textunderscore  + \textunderscore verossimilhança\textunderscore )}
\end{itemize}
Falta de verossimilhança.
\section{Inverossimilmente}
\begin{itemize}
\item {Grp. gram.:adv.}
\end{itemize}
De modo inverossímil.
\section{Inversa}
\begin{itemize}
\item {Grp. gram.:f.}
\end{itemize}
\begin{itemize}
\item {Proveniência:(De \textunderscore inverso\textunderscore )}
\end{itemize}
Proposição, cujos termos estão invertidos.
\section{Inversamente}
\begin{itemize}
\item {Grp. gram.:adv.}
\end{itemize}
De modo inverso.
\section{Inversão}
\begin{itemize}
\item {Grp. gram.:f.}
\end{itemize}
\begin{itemize}
\item {Proveniência:(Lat. \textunderscore inversio\textunderscore )}
\end{itemize}
Acto ou effeito de inverter.
\section{Inversivo}
\begin{itemize}
\item {Grp. gram.:adj.}
\end{itemize}
\begin{itemize}
\item {Proveniência:(De \textunderscore inverso\textunderscore )}
\end{itemize}
Em que há inversão.
\section{Inverso}
\begin{itemize}
\item {Grp. gram.:m.}
\end{itemize}
\begin{itemize}
\item {Proveniência:(Lat. \textunderscore inversus\textunderscore )}
\end{itemize}
O mesmo que \textunderscore inversão\textunderscore ; invés.
\section{Inversor}
\begin{itemize}
\item {Grp. gram.:m.  e  adj.}
\end{itemize}
\begin{itemize}
\item {Proveniência:(Lat. \textunderscore inversor\textunderscore )}
\end{itemize}
O que inverte.
\section{Invertebrado}
\begin{itemize}
\item {Grp. gram.:m.  e  adj.}
\end{itemize}
\begin{itemize}
\item {Utilização:Zool.}
\end{itemize}
\begin{itemize}
\item {Proveniência:(De \textunderscore in...\textunderscore  + \textunderscore vertebrado\textunderscore )}
\end{itemize}
Animal, que não tem vértebras.
\section{Invertedor}
\begin{itemize}
\item {Grp. gram.:m.  e  adj.}
\end{itemize}
\begin{itemize}
\item {Proveniência:(De \textunderscore inverter\textunderscore )}
\end{itemize}
O mesmo que \textunderscore inversor\textunderscore . Cf. Castilho, \textunderscore Metam.\textunderscore , XXXIII.
\section{Inverter}
\begin{itemize}
\item {Grp. gram.:v. t.}
\end{itemize}
\begin{itemize}
\item {Proveniência:(Lat. \textunderscore invertere\textunderscore )}
\end{itemize}
Collocar num sentido, direcção ou ordem, opposta a outra ordem, direcção ou sentido.
Oppor.
Pôr ás avessas.
Alterar.
Transpor.
\section{Invertido}
\begin{itemize}
\item {Grp. gram.:adj.}
\end{itemize}
\begin{itemize}
\item {Grp. gram.:M.}
\end{itemize}
\begin{itemize}
\item {Utilização:Pop.}
\end{itemize}
\begin{itemize}
\item {Proveniência:(De \textunderscore inverter\textunderscore )}
\end{itemize}
Que se inverteu.
Diz-se o homem, em que outro exerce acções libidinosas.
\section{Invertível}
\begin{itemize}
\item {Grp. gram.:adj.}
\end{itemize}
\begin{itemize}
\item {Proveniência:(Lat. \textunderscore invertibilis\textunderscore )}
\end{itemize}
Que se póde inverter.
\section{Invés}
\begin{itemize}
\item {Grp. gram.:m.}
\end{itemize}
\begin{itemize}
\item {Grp. gram.:Loc. adv.}
\end{itemize}
\begin{itemize}
\item {Proveniência:(Do b. lat. \textunderscore ínverse\textunderscore )}
\end{itemize}
O mesmo que \textunderscore avesso\textunderscore .
O lado opposto.
\textunderscore ao invés\textunderscore , ao contrário. Cf. Michaëlis, \textunderscore Nota aos Son. Anon.\textunderscore , 7.
\section{Invèsamento}
\begin{itemize}
\item {Grp. gram.:m.}
\end{itemize}
\begin{itemize}
\item {Utilização:Ant.}
\end{itemize}
\begin{itemize}
\item {Proveniência:(De \textunderscore invés\textunderscore )}
\end{itemize}
Contrariedade, transtôrno.
\section{Investida}
\begin{itemize}
\item {Grp. gram.:f.}
\end{itemize}
\begin{itemize}
\item {Utilização:Fam.}
\end{itemize}
\begin{itemize}
\item {Proveniência:(De \textunderscore investido\textunderscore )}
\end{itemize}
Acto de investir, de atacar.
Tentativa.
Motejo.
\section{Investidura}
\begin{itemize}
\item {Grp. gram.:f.}
\end{itemize}
\begin{itemize}
\item {Proveniência:(Do lat. \textunderscore investitura\textunderscore )}
\end{itemize}
Acto de dar posse.
Posse.
Ceremónia, em que se dá posse ou se faz o provimento de um cargo ou dignidade.
\section{Investigação}
\begin{itemize}
\item {Grp. gram.:f.}
\end{itemize}
\begin{itemize}
\item {Proveniência:(Lat. \textunderscore investigatio\textunderscore )}
\end{itemize}
Acto de investigar, de inquirir, de pesquisar.
\section{Investigador}
\begin{itemize}
\item {Grp. gram.:m.  e  adj.}
\end{itemize}
\begin{itemize}
\item {Proveniência:(Lat. \textunderscore investigator\textunderscore )}
\end{itemize}
O que investiga.
\section{Investigante}
\begin{itemize}
\item {Grp. gram.:adj.}
\end{itemize}
\begin{itemize}
\item {Proveniência:(Lat. \textunderscore investigans\textunderscore )}
\end{itemize}
Que investiga.
\section{Investigar}
\begin{itemize}
\item {Grp. gram.:v. t.}
\end{itemize}
\begin{itemize}
\item {Proveniência:(Lat. \textunderscore investigare\textunderscore )}
\end{itemize}
Seguir os vestígios de.
Procurar.
Indagar, inquirir.
Descobrir, achar.
\section{Investigável}
\begin{itemize}
\item {Grp. gram.:adj.}
\end{itemize}
\begin{itemize}
\item {Proveniência:(Lat. \textunderscore investigabilis\textunderscore )}
\end{itemize}
Que se póde investigar.
\section{Investimento}
\begin{itemize}
\item {Grp. gram.:m.}
\end{itemize}
Acto ou effeito de investir.
\section{Investir}
\begin{itemize}
\item {Grp. gram.:v. t.}
\end{itemize}
\begin{itemize}
\item {Utilização:Fam.}
\end{itemize}
\begin{itemize}
\item {Grp. gram.:V. i.}
\end{itemize}
\begin{itemize}
\item {Proveniência:(Lat. \textunderscore investire\textunderscore )}
\end{itemize}
Revestir de poder ou autoridade.
Dar posse de um cargo ou dignidade a.
Assaltar, atacar.
Motejar de.
Arrojar-se com ímpeto; dar assalto.
\section{Inveteração}
\begin{itemize}
\item {Grp. gram.:f.}
\end{itemize}
\begin{itemize}
\item {Proveniência:(Lat. \textunderscore inveteratio\textunderscore )}
\end{itemize}
Facto de se inveterar, de envelhecer.
\section{Inveterar}
\begin{itemize}
\item {Grp. gram.:v. t.}
\end{itemize}
\begin{itemize}
\item {Proveniência:(Lat. \textunderscore inveterare\textunderscore )}
\end{itemize}
Tornar velho, antigo.
Enraízar pelo decurso do tempo.
Introduzir nos hábitos; arraigar.
\section{Inviabilidade}
\begin{itemize}
\item {Grp. gram.:f.}
\end{itemize}
Qualidade do que é inviável.
\section{Inviável}
\begin{itemize}
\item {Grp. gram.:adj.}
\end{itemize}
\begin{itemize}
\item {Proveniência:(De \textunderscore in...\textunderscore  + \textunderscore viável\textunderscore )}
\end{itemize}
Que não é viável.
\section{Invicção}
\begin{itemize}
\item {Grp. gram.:f.}
\end{itemize}
\begin{itemize}
\item {Utilização:Prov.}
\end{itemize}
\begin{itemize}
\item {Utilização:trasm.}
\end{itemize}
\begin{itemize}
\item {Proveniência:(Do rad. do lat. \textunderscore invictus\textunderscore )}
\end{itemize}
Enthusiasmo.
Paixão; pertinácia.
\section{Inviccionar-se}
\begin{itemize}
\item {Grp. gram.:v. p.}
\end{itemize}
\begin{itemize}
\item {Utilização:Pop.}
\end{itemize}
\begin{itemize}
\item {Proveniência:(De \textunderscore invicção\textunderscore )}
\end{itemize}
Têr invicção.
Insistir; sêr pertinaz.
Preoccupar-se freneticamente.
\section{Invicto}
\begin{itemize}
\item {Grp. gram.:adj.}
\end{itemize}
\begin{itemize}
\item {Proveniência:(Lat. \textunderscore invictus\textunderscore )}
\end{itemize}
Não vencido; que não póde sêr vencido: \textunderscore o Pôrto, a cidade invicta...\textunderscore 
\section{Invidadoiro}
\begin{itemize}
\item {Grp. gram.:m.}
\end{itemize}
\begin{itemize}
\item {Utilização:Prov.}
\end{itemize}
\begin{itemize}
\item {Utilização:minh.}
\end{itemize}
\begin{itemize}
\item {Proveniência:(De \textunderscore invidar\textunderscore )}
\end{itemize}
Grande baraço, para segurar rêdes.
\section{Invidador}
\begin{itemize}
\item {Grp. gram.:m.}
\end{itemize}
Aquelle que invida.
\section{Invidadouro}
\begin{itemize}
\item {Grp. gram.:m.}
\end{itemize}
\begin{itemize}
\item {Utilização:Prov.}
\end{itemize}
\begin{itemize}
\item {Utilização:minh.}
\end{itemize}
\begin{itemize}
\item {Proveniência:(De \textunderscore invidar\textunderscore )}
\end{itemize}
Grande baraço, para segurar rêdes.
\section{Invidamento}
\begin{itemize}
\item {Grp. gram.:m.}
\end{itemize}
Acto de invidar.
\section{Invidar}
\begin{itemize}
\item {Grp. gram.:v. t.}
\end{itemize}
\begin{itemize}
\item {Proveniência:(Lat. \textunderscore invitare\textunderscore )}
\end{itemize}
Chamar.
Convidar.
Provocar.
Fazer invite a, no jôgo.
Recorrer com esfôço ou empenho a.
Empregar dedicadamente: \textunderscore invidar diligências\textunderscore .
\section{Invide}
\begin{itemize}
\item {Grp. gram.:m.}
\end{itemize}
Acto de invidar; invite.
\section{Invídia}
\begin{itemize}
\item {Grp. gram.:f.}
\end{itemize}
\begin{itemize}
\item {Utilização:Poét.}
\end{itemize}
\begin{itemize}
\item {Proveniência:(Lat. \textunderscore invidia\textunderscore )}
\end{itemize}
O mesmo que \textunderscore inveja\textunderscore . Cf. B. Pato, \textunderscore Livro do Monte\textunderscore , 154.
\section{Invidiar}
\begin{itemize}
\item {Grp. gram.:v. t.}
\end{itemize}
\begin{itemize}
\item {Utilização:Ant.}
\end{itemize}
\begin{itemize}
\item {Proveniência:(De \textunderscore invídia\textunderscore )}
\end{itemize}
O mesmo que \textunderscore invejar\textunderscore .
\section{Ínvido}
\begin{itemize}
\item {Grp. gram.:adj.}
\end{itemize}
\begin{itemize}
\item {Utilização:Poét.}
\end{itemize}
\begin{itemize}
\item {Proveniência:(Lat. \textunderscore invidus\textunderscore )}
\end{itemize}
O mesmo que \textunderscore invejoso\textunderscore .
\section{Invído}
\begin{itemize}
\item {Grp. gram.:m.}
\end{itemize}
\begin{itemize}
\item {Utilização:Prov.}
\end{itemize}
\begin{itemize}
\item {Utilização:alent.}
\end{itemize}
O mesmo que \textunderscore invide\textunderscore .
\section{Invigilância}
\begin{itemize}
\item {Grp. gram.:f.}
\end{itemize}
\begin{itemize}
\item {Proveniência:(De \textunderscore in...\textunderscore  + \textunderscore vigilancia\textunderscore )}
\end{itemize}
Falta de vigilância; desmazêlo.
\section{Invigilante}
\begin{itemize}
\item {Grp. gram.:adj.}
\end{itemize}
\begin{itemize}
\item {Proveniência:(Lat. \textunderscore invigilans\textunderscore )}
\end{itemize}
Que não é vigilante; descuidado.
\section{Invingado}
\begin{itemize}
\item {Grp. gram.:adj.}
\end{itemize}
\begin{itemize}
\item {Proveniência:(De \textunderscore in...\textunderscore  + \textunderscore vingar\textunderscore )}
\end{itemize}
Que se não vingou. Cf. Rui Barb., \textunderscore Réplica\textunderscore , 157.
\section{Ínvio}
\begin{itemize}
\item {Grp. gram.:adj.}
\end{itemize}
\begin{itemize}
\item {Proveniência:(Lat. \textunderscore invius\textunderscore )}
\end{itemize}
Em que não há caminho: \textunderscore florestas ínvias\textunderscore .
Intransitável: \textunderscore caminhos ínvios\textunderscore .
\section{Inviolabilidade}
\begin{itemize}
\item {Grp. gram.:f.}
\end{itemize}
Qualidade de inviolável.
\section{Inviolado}
\begin{itemize}
\item {Grp. gram.:adj.}
\end{itemize}
\begin{itemize}
\item {Proveniência:(Lat. \textunderscore inviolatus\textunderscore )}
\end{itemize}
Não violado; íntegro; puro; immaculado.
\section{Inviolável}
\begin{itemize}
\item {Grp. gram.:adj.}
\end{itemize}
\begin{itemize}
\item {Proveniência:(Lat. \textunderscore inviolabilis\textunderscore )}
\end{itemize}
Que se não póde violar; que não deve sêr violado.
Que não está sujeito á justiça commum; privilegiado.
\section{Inviolavelmente}
\begin{itemize}
\item {Grp. gram.:adv.}
\end{itemize}
De modo inviolável.
\section{Inviolentado}
\begin{itemize}
\item {Grp. gram.:adj.}
\end{itemize}
\begin{itemize}
\item {Proveniência:(De \textunderscore in...\textunderscore  + \textunderscore violentado\textunderscore )}
\end{itemize}
Não violentado; que procede voluntariamente.
\section{Inviperar-se}
\begin{itemize}
\item {Grp. gram.:v. p.}
\end{itemize}
\begin{itemize}
\item {Proveniência:(Do lat. \textunderscore vipera\textunderscore )}
\end{itemize}
Assanhar-se como a víbora.
Encolerizar-se.
\section{Inviril}
\begin{itemize}
\item {Grp. gram.:adj.}
\end{itemize}
\begin{itemize}
\item {Utilização:bras}
\end{itemize}
\begin{itemize}
\item {Utilização:Neol.}
\end{itemize}
\begin{itemize}
\item {Proveniência:(De \textunderscore in...\textunderscore  + \textunderscore viril\textunderscore )}
\end{itemize}
Não viril; effeminado.--Us. por Tobias Barreto.
\section{Invirilidade}
\begin{itemize}
\item {Grp. gram.:f.}
\end{itemize}
Qualidade ou estado de inviril.
\section{Invirtuoso}
\begin{itemize}
\item {Grp. gram.:adj.}
\end{itemize}
\begin{itemize}
\item {Utilização:Des.}
\end{itemize}
\begin{itemize}
\item {Proveniência:(De \textunderscore in...\textunderscore  + \textunderscore virtuoso\textunderscore )}
\end{itemize}
Não virtuoso.
\section{Inviscar}
\textunderscore v. t.\textunderscore  e \textunderscore p.\textunderscore (e der.)(V.enviscar)
\section{Inviscerar}
\begin{itemize}
\item {Grp. gram.:v. t.}
\end{itemize}
\begin{itemize}
\item {Proveniência:(Lat. \textunderscore inviscerare\textunderscore )}
\end{itemize}
Introduzir nas vísceras; entranhar.
\section{Invisibilidade}
\begin{itemize}
\item {Grp. gram.:f.}
\end{itemize}
\begin{itemize}
\item {Proveniência:(Lat. \textunderscore invisibilitas\textunderscore )}
\end{itemize}
Qualidade daquillo que é invisível.
\section{Invisível}
\begin{itemize}
\item {Grp. gram.:adj.}
\end{itemize}
\begin{itemize}
\item {Utilização:Fam.}
\end{itemize}
\begin{itemize}
\item {Grp. gram.:M.}
\end{itemize}
\begin{itemize}
\item {Grp. gram.:F.}
\end{itemize}
\begin{itemize}
\item {Utilização:Fam.}
\end{itemize}
\begin{itemize}
\item {Proveniência:(Do lat. \textunderscore invisibilis\textunderscore )}
\end{itemize}
Que se não póde vêr.
Que é pouco accessível, que não recebe visita ou que não dá audiências.
Aquillo que se não vê.
Rede tenuíssima de cabello, com que as senhoras amparam a parte frisada do penteado.
\section{Invisivelmente}
\begin{itemize}
\item {Grp. gram.:adv.}
\end{itemize}
De modo invisível.
\section{Inviso}
\begin{itemize}
\item {Grp. gram.:adj.}
\end{itemize}
\begin{itemize}
\item {Utilização:Poét.}
\end{itemize}
\begin{itemize}
\item {Proveniência:(Lat. \textunderscore invisus\textunderscore )}
\end{itemize}
Não visto.
Invejado, aborrecido.
\section{Invitação}
\begin{itemize}
\item {Grp. gram.:f.}
\end{itemize}
O mesmo que \textunderscore invitamento\textunderscore .
\section{Invitador}
\begin{itemize}
\item {Grp. gram.:m.}
\end{itemize}
\begin{itemize}
\item {Proveniência:(Lat. \textunderscore invitator\textunderscore )}
\end{itemize}
Servo, que, entre os Romanos distribuía os convites para os banquetes.
\section{Invitamento}
\begin{itemize}
\item {Grp. gram.:m.}
\end{itemize}
Acto ou effeito de invitar.
\section{Invitar}
\begin{itemize}
\item {Grp. gram.:v. t.}
\end{itemize}
\begin{itemize}
\item {Proveniência:(Lat. \textunderscore invitare\textunderscore )}
\end{itemize}
Convidar.
\section{Invitatório}
\begin{itemize}
\item {Grp. gram.:adj.}
\end{itemize}
\begin{itemize}
\item {Grp. gram.:M.}
\end{itemize}
\begin{itemize}
\item {Proveniência:(Lat. \textunderscore invitatorius\textunderscore )}
\end{itemize}
Próprio para convidar.
Que encerra convite.
Antiphona, que se diz no princípio das Matinas.
Invocação.
\section{Invite}
\begin{itemize}
\item {Grp. gram.:m.}
\end{itemize}
\begin{itemize}
\item {Proveniência:(De \textunderscore invitar\textunderscore )}
\end{itemize}
Convite.
Acto de dobrar a parada, no jôgo.
\section{Invito}
\begin{itemize}
\item {Grp. gram.:adj.}
\end{itemize}
\begin{itemize}
\item {Proveniência:(Lat. \textunderscore invitus\textunderscore )}
\end{itemize}
Que procede contra a própria vontade.
Forçado, contrariado.
Involuntário.
\section{Invitrescível}
\begin{itemize}
\item {Grp. gram.:adj.}
\end{itemize}
\begin{itemize}
\item {Proveniência:(De \textunderscore in...\textunderscore  + \textunderscore vitrescível\textunderscore )}
\end{itemize}
Que não é vitrificável.
\section{Invocação}
\begin{itemize}
\item {Grp. gram.:f.}
\end{itemize}
\begin{itemize}
\item {Proveniência:(Lat. \textunderscore invocatio\textunderscore )}
\end{itemize}
Acto ou effeito de invocar.
Acto de chamar em soccorro.
Allegação.
\section{Invocador}
\begin{itemize}
\item {Grp. gram.:m.  e  adj.}
\end{itemize}
\begin{itemize}
\item {Proveniência:(Lat. \textunderscore invocator\textunderscore )}
\end{itemize}
O que invoca.
\section{Invocar}
\begin{itemize}
\item {Grp. gram.:v. t.}
\end{itemize}
\begin{itemize}
\item {Proveniência:(Lat. \textunderscore invocare\textunderscore )}
\end{itemize}
Chamar.
Implorar a protecção ou o auxílio de; supplicar.
\section{Invocativamente}
\begin{itemize}
\item {Grp. gram.:adv.}
\end{itemize}
De modo invocativo; á maneira de súpplica.
\section{Invocativo}
\begin{itemize}
\item {Grp. gram.:adj.}
\end{itemize}
\begin{itemize}
\item {Proveniência:(Lat. \textunderscore invocativus\textunderscore )}
\end{itemize}
Que invoca; próprio para invocar; que encerra invocação.
\section{Invocatória}
\begin{itemize}
\item {Grp. gram.:f.}
\end{itemize}
\begin{itemize}
\item {Proveniência:(De \textunderscore invocatório\textunderscore )}
\end{itemize}
O mesmo que \textunderscore invocação\textunderscore .
\section{Invocatório}
\begin{itemize}
\item {Grp. gram.:adj.}
\end{itemize}
O mesmo que \textunderscore invocativo\textunderscore .
\section{Invocável}
\begin{itemize}
\item {Grp. gram.:adj.}
\end{itemize}
Que se póde invocar.
\section{Invogal}
\begin{itemize}
\item {Grp. gram.:f.  e  adj.}
\end{itemize}
\begin{itemize}
\item {Proveniência:(De \textunderscore in...\textunderscore  + \textunderscore vogal\textunderscore )}
\end{itemize}
Letra, que não é vogal, letra consoante. Cf. João de Deus, \textunderscore Cartilha\textunderscore .
\section{Involução}
\begin{itemize}
\item {Grp. gram.:f.}
\end{itemize}
\begin{itemize}
\item {Utilização:Mathem.}
\end{itemize}
\begin{itemize}
\item {Proveniência:(Lat. \textunderscore involutio\textunderscore )}
\end{itemize}
Movimento regressivo.
\textunderscore Eixo de involução\textunderscore , recta fixa, que representa o lugar geométrico dos pontos de intersecção dos differentes pares de rectas, que unem dois pares de pontos homólogos, (tratando-se de duas rectas, situadas no mesmo plano e divididas homographicamente).
\section{Involucelado}
\begin{itemize}
\item {Grp. gram.:adj.}
\end{itemize}
Que é provido de involucelo.
\section{Involucelo}
\begin{itemize}
\item {Grp. gram.:adj.}
\end{itemize}
\begin{itemize}
\item {Utilização:Bot.}
\end{itemize}
Invólucro parcial de cada flôr ou de cada feixe de flôres.
(Dem. de \textunderscore invólucro\textunderscore )
\section{Involucellado}
\begin{itemize}
\item {Grp. gram.:adj.}
\end{itemize}
Que é provido de involucello.
\section{Involucello}
\begin{itemize}
\item {Grp. gram.:adj.}
\end{itemize}
\begin{itemize}
\item {Utilização:Bot.}
\end{itemize}
Invólucro parcial de cada flôr ou de cada feixe de flôres.
(Dem. de \textunderscore invólucro\textunderscore )
\section{Involucrado}
\begin{itemize}
\item {Grp. gram.:adj.}
\end{itemize}
Em que há invólucro.
\section{Involucral}
\begin{itemize}
\item {Grp. gram.:adj.}
\end{itemize}
\begin{itemize}
\item {Proveniência:(De \textunderscore invólucro\textunderscore )}
\end{itemize}
Relativo a invólucro vegetal.
\section{Involucriforme}
\begin{itemize}
\item {Grp. gram.:adj.}
\end{itemize}
\begin{itemize}
\item {Utilização:Bot.}
\end{itemize}
\begin{itemize}
\item {Proveniência:(De \textunderscore invólucro\textunderscore  + \textunderscore fórma\textunderscore )}
\end{itemize}
Semelhante ao invólucro.
\section{Invólucro}
\begin{itemize}
\item {Grp. gram.:m.}
\end{itemize}
\begin{itemize}
\item {Proveniência:(Lat. \textunderscore involucrum\textunderscore )}
\end{itemize}
Aquillo que envolve, cobre ou reveste; embrulho.--Em Portugal, há quem mande lêr \textunderscore involúcro\textunderscore ; em todo Brasil, diz-se \textunderscore invólucro\textunderscore .
\section{Involuntariamente}
\begin{itemize}
\item {Grp. gram.:adv.}
\end{itemize}
De modo involuntário; contra a vontade.
\section{Involuntário}
\begin{itemize}
\item {Grp. gram.:adj.}
\end{itemize}
\begin{itemize}
\item {Proveniência:(Lat. \textunderscore involuntarius\textunderscore )}
\end{itemize}
Não voluntário; opposto á vontade ou independente della: \textunderscore homicídio involuntário\textunderscore .
\section{Involutório}
\begin{itemize}
\item {Grp. gram.:m.}
\end{itemize}
\begin{itemize}
\item {Proveniência:(Do lat. \textunderscore involutus\textunderscore )}
\end{itemize}
O mesmo que \textunderscore envoltório\textunderscore .
\section{Involutoso}
\begin{itemize}
\item {Grp. gram.:adj.}
\end{itemize}
\begin{itemize}
\item {Utilização:Bot.}
\end{itemize}
\begin{itemize}
\item {Proveniência:(Do lat. \textunderscore involutus\textunderscore )}
\end{itemize}
Que tem os bordos enrolados para dentro.
\section{Involver}
\textunderscore v. t.\textunderscore  (e der.)
(V. \textunderscore envolver\textunderscore , etc.)
\section{Invulnerabilidade}
\begin{itemize}
\item {Grp. gram.:f.}
\end{itemize}
Qualidade de invulnerável.
\section{Invulnerado}
\begin{itemize}
\item {Grp. gram.:adj.}
\end{itemize}
\begin{itemize}
\item {Utilização:Fig.}
\end{itemize}
\begin{itemize}
\item {Proveniência:(Lat. \textunderscore invulneratus\textunderscore )}
\end{itemize}
Que não está ferido.
Intacto.
\section{Invulnerável}
\begin{itemize}
\item {Grp. gram.:adj.}
\end{itemize}
\begin{itemize}
\item {Utilização:Fig.}
\end{itemize}
\begin{itemize}
\item {Proveniência:(Lat. \textunderscore invulnerabilis\textunderscore )}
\end{itemize}
Que não é vulnerável.
Que não póde sêr atacado.
Irrespondível.
Immaculado.
\section{Inxerir}
\begin{itemize}
\item {Grp. gram.:v. t.}
\end{itemize}
O mesmo que \textunderscore inserir\textunderscore . Cf. Filinto, \textunderscore D. Man.\textunderscore , II, 210.
\section{Inxidro}
\begin{itemize}
\item {Grp. gram.:m.}
\end{itemize}
\begin{itemize}
\item {Utilização:Prov.}
\end{itemize}
Pequeno pomar.
(Cp. \textunderscore êixido\textunderscore )
\section{Inzoável}
\begin{itemize}
\item {Grp. gram.:m.  e  f.}
\end{itemize}
\begin{itemize}
\item {Utilização:Prov.}
\end{itemize}
\begin{itemize}
\item {Utilização:trasm.}
\end{itemize}
Pessôa pretensiosa, affectada no falar.
\section{Inzoneiro}
\begin{itemize}
\item {Grp. gram.:m.}
\end{itemize}
\begin{itemize}
\item {Utilização:Bras}
\end{itemize}
(Corr. de \textunderscore onzeneiro\textunderscore )
\section{Iobó}
\begin{itemize}
\item {Grp. gram.:m.}
\end{itemize}
Árvore santhomense, de sementes medicinaes.
\section{Iochroma}
\begin{itemize}
\item {Grp. gram.:f.}
\end{itemize}
\begin{itemize}
\item {Proveniência:(Do gr. \textunderscore ion\textunderscore  + \textunderscore khroma\textunderscore )}
\end{itemize}
Gênero de plantas solanáceas.
\section{Iocroma}
\begin{itemize}
\item {Grp. gram.:f.}
\end{itemize}
\begin{itemize}
\item {Proveniência:(Do gr. \textunderscore ion\textunderscore  + \textunderscore khroma\textunderscore )}
\end{itemize}
Gênero de plantas solanáceas.
\section{Iodalose}
\begin{itemize}
\item {Grp. gram.:f.}
\end{itemize}
\begin{itemize}
\item {Utilização:Pharm.}
\end{itemize}
Combinação de iodo e peptona.
\section{Iodamilo}
\begin{itemize}
\item {Grp. gram.:m.}
\end{itemize}
Substância quimica, que se obtém por meio da destilação do álcool amílico com o iodo e com o fósforo.
\section{Iodamylo}
\begin{itemize}
\item {Grp. gram.:m.}
\end{itemize}
Substância chimica, que se obtém por meio da destillação do álcool amýlico com o iodo e com o phósphoro.
\section{Iodar}
\begin{itemize}
\item {Grp. gram.:v. t.}
\end{itemize}
Cobrir ou misturar com iodo.
\section{Iodato}
\begin{itemize}
\item {Grp. gram.:m.}
\end{itemize}
\begin{itemize}
\item {Proveniência:(De \textunderscore iodo\textunderscore )}
\end{itemize}
Combinação do ácido iódico com uma base.
\section{Iode}
\begin{itemize}
\item {Grp. gram.:m.}
\end{itemize}
O mesmo ou melhor que \textunderscore iodo\textunderscore .
\section{Iodeto}
\begin{itemize}
\item {fónica:dê}
\end{itemize}
\begin{itemize}
\item {Grp. gram.:m.}
\end{itemize}
Combinação do iodo com um metal ou outro metalloide.
\section{Iodhydrato}
\begin{itemize}
\item {Grp. gram.:m.}
\end{itemize}
Sal, formado pela combinação do ácido iodhýdrico com uma base.
\section{Iodhýdrico}
\begin{itemize}
\item {Grp. gram.:adj.}
\end{itemize}
\begin{itemize}
\item {Proveniência:(De \textunderscore iodo\textunderscore  + \textunderscore hydrogênio\textunderscore )}
\end{itemize}
Diz-se de um ácido, composto de iodo e de hydrogênio.
Diz-se de um éther, resultante da acção do iodeto sôbre duas partes de álcool.
\section{Iodhydrina}
\begin{itemize}
\item {Grp. gram.:f.}
\end{itemize}
Substância chimica, resultante da combinação da glycerina com o ácido iodhýdrico.
\section{Iodidrato}
\begin{itemize}
\item {Grp. gram.:m.}
\end{itemize}
Sal, formado pela combinação do ácido iodídrico com uma base.
\section{Iodídrico}
\begin{itemize}
\item {Grp. gram.:adj.}
\end{itemize}
\begin{itemize}
\item {Proveniência:(De \textunderscore iodo\textunderscore  + \textunderscore hidrogênio\textunderscore )}
\end{itemize}
Diz-se de um ácido, composto de iodo e de hidrogênio.
Diz-se de um éter, resultante da acção do iodeto sôbre duas partes de álcool.
\section{Iodidrina}
\begin{itemize}
\item {Grp. gram.:f.}
\end{itemize}
Substância quimica, resultante da combinação da glicerina com o ácido iodídrico.
\section{Iódico}
\begin{itemize}
\item {Grp. gram.:adj.}
\end{itemize}
\begin{itemize}
\item {Proveniência:(De \textunderscore iodo\textunderscore )}
\end{itemize}
Diz-se do segundo ácido, que o iodo produz, unindo-se ao oxygênio.
\section{Iodífero}
\begin{itemize}
\item {Grp. gram.:adj.}
\end{itemize}
\begin{itemize}
\item {Proveniência:(De \textunderscore iodo\textunderscore  + lat. \textunderscore ferre\textunderscore )}
\end{itemize}
Que contém iodo.
\section{Iodina}
\begin{itemize}
\item {Grp. gram.:f.}
\end{itemize}
(V.iodo)
\section{Iodipina}
\begin{itemize}
\item {Grp. gram.:f.}
\end{itemize}
\begin{itemize}
\item {Utilização:Pharm.}
\end{itemize}
Medicamento, em que se combina o iodo e o óleo de sésamo.
\section{Iodismo}
\begin{itemize}
\item {Grp. gram.:m.}
\end{itemize}
Accidentes, resultantes do abuso do iodo.
Espécie de embriaguez, produzida pela ingestão do iodo em grande quantidade.
\section{Iodo}
\begin{itemize}
\item {Grp. gram.:m.}
\end{itemize}
\begin{itemize}
\item {Proveniência:(Do gr. \textunderscore iodes\textunderscore )}
\end{itemize}
Substância simples, que é um metalloide pardo-azulado, como a plumbagina.
\section{Iodobórico}
\begin{itemize}
\item {Grp. gram.:adj.}
\end{itemize}
\begin{itemize}
\item {Proveniência:(De \textunderscore iodo\textunderscore  + \textunderscore bórico\textunderscore )}
\end{itemize}
Diz-se de um ácido, resultante da combinação do ácido bórico com o iódico.
\section{Iodocalário}
\begin{itemize}
\item {Grp. gram.:adj.}
\end{itemize}
\begin{itemize}
\item {Proveniência:(De \textunderscore iodo\textunderscore  + \textunderscore calcário\textunderscore )}
\end{itemize}
Diz-se de um xarope de phosphato de cal solúvel e de iodeto de cálcio.
\section{Iodochloreto}
\begin{itemize}
\item {fónica:clorê}
\end{itemize}
\begin{itemize}
\item {Grp. gram.:m.}
\end{itemize}
\begin{itemize}
\item {Proveniência:(De \textunderscore iodo\textunderscore  + \textunderscore chloreto\textunderscore )}
\end{itemize}
Combinação do iodeto com o chloreto.
\section{Iodocloreto}
\begin{itemize}
\item {Grp. gram.:m.}
\end{itemize}
\begin{itemize}
\item {Proveniência:(De \textunderscore iodo\textunderscore  + \textunderscore cloreto\textunderscore )}
\end{itemize}
Combinação do iodeto com o cloreto.
\section{Iodoformado}
\begin{itemize}
\item {Grp. gram.:adj.}
\end{itemize}
Que contém iodofórmio.
Misturado com iodofórmio.
\section{Iodofórmio}
\begin{itemize}
\item {Grp. gram.:m}
\end{itemize}
\begin{itemize}
\item {Proveniência:(De \textunderscore iodo\textunderscore  + \textunderscore fórma\textunderscore )}
\end{itemize}
Composto sólido, resultante da acção do iodo sôbre o álcool.
\section{Iodoformogênio}
\begin{itemize}
\item {Grp. gram.:m.}
\end{itemize}
Combinação de iodofórmio e albumina, succedâneo do iodofórmio.
\section{Iodol}
\begin{itemize}
\item {Grp. gram.:m.}
\end{itemize}
Um dos succedâneos do iodofórmio, empregado contra as escrófulas e a sýphilis terciária.
\section{Iodómetro}
\begin{itemize}
\item {Grp. gram.:m.}
\end{itemize}
\begin{itemize}
\item {Proveniência:(De \textunderscore iodo\textunderscore  + gr. \textunderscore metron\textunderscore )}
\end{itemize}
Apparelho, para a inhalação do iodo, no tratamento da tísica pulmonar.
\section{Iodomorfina}
\begin{itemize}
\item {Grp. gram.:f.}
\end{itemize}
\begin{itemize}
\item {Proveniência:(De \textunderscore iodo\textunderscore  + \textunderscore morfina\textunderscore )}
\end{itemize}
Combinação de iodo com morfina.
\section{Iodomorphina}
\begin{itemize}
\item {Grp. gram.:f.}
\end{itemize}
\begin{itemize}
\item {Proveniência:(De \textunderscore iodo\textunderscore  + \textunderscore morphina\textunderscore )}
\end{itemize}
Combinação de iodo com morphina.
\section{Iodona}
\begin{itemize}
\item {Grp. gram.:f.}
\end{itemize}
O mesmo que \textunderscore iodalose\textunderscore .
\section{Iodonítrico}
\begin{itemize}
\item {Grp. gram.:adj.}
\end{itemize}
\begin{itemize}
\item {Proveniência:(De \textunderscore iodo\textunderscore  + \textunderscore nítrico\textunderscore )}
\end{itemize}
Diz-se de um ácido, resultante da combinação do ácido iódico com o nítrico.
\section{Iodoso}
\begin{itemize}
\item {Grp. gram.:adj.}
\end{itemize}
\begin{itemize}
\item {Proveniência:(De \textunderscore iodo\textunderscore )}
\end{itemize}
Diz-se de um dos ácidos, resultantes da combinação do iodo com o oxygênio.
\section{Iodossulfúrico}
\begin{itemize}
\item {Grp. gram.:adj.}
\end{itemize}
\begin{itemize}
\item {Proveniência:(De \textunderscore iodo\textunderscore  + \textunderscore sulfúrico\textunderscore )}
\end{itemize}
Diz-se de um ácido, resultante da combinação do ácido iódico com o sulfúrico.
\section{Iodosulfúrico}
\begin{itemize}
\item {fónica:sul}
\end{itemize}
\begin{itemize}
\item {Grp. gram.:adj.}
\end{itemize}
\begin{itemize}
\item {Proveniência:(De \textunderscore iodo\textunderscore  + \textunderscore sulfúrico\textunderscore )}
\end{itemize}
Diz-se de um ácido, resultante da combinação do ácido iódico com o sulfúrico.
\section{Iodoterapia}
\begin{itemize}
\item {Grp. gram.:f.}
\end{itemize}
\begin{itemize}
\item {Utilização:Med.}
\end{itemize}
Emprêgo terapêutico do iodo e seus compostos.
\section{Iodotherapia}
\begin{itemize}
\item {Grp. gram.:f.}
\end{itemize}
\begin{itemize}
\item {Utilização:Med.}
\end{itemize}
Emprêgo therapêutico do iodo e seus compostos.
\section{Iodureto}
\begin{itemize}
\item {fónica:durê}
\end{itemize}
\begin{itemize}
\item {Grp. gram.:m.}
\end{itemize}
(V.iodeto)
\section{Iofobia}
\begin{itemize}
\item {Grp. gram.:f.}
\end{itemize}
\begin{itemize}
\item {Proveniência:(Do gr. \textunderscore ios\textunderscore  + \textunderscore phobos\textunderscore )}
\end{itemize}
Temor mórbido dos venenos.
\section{Ioió}
\begin{itemize}
\item {Grp. gram.:m.}
\end{itemize}
O mesmo que \textunderscore nhonhô\textunderscore ; menino.
\section{Iô-iô}
\begin{itemize}
\item {Grp. gram.:m.}
\end{itemize}
\begin{itemize}
\item {Utilização:Bras}
\end{itemize}
O mesmo que \textunderscore nhonhô\textunderscore ; menino.
\section{Iónico}
\begin{itemize}
\item {Grp. gram.:adj.}
\end{itemize}
(V.jónico)
\section{Iónio}
\begin{itemize}
\item {Grp. gram.:adj.}
\end{itemize}
(V.jónio)
\section{Iophobia}
\begin{itemize}
\item {Grp. gram.:f.}
\end{itemize}
\begin{itemize}
\item {Proveniência:(Do gr. \textunderscore ios\textunderscore  + \textunderscore phobos\textunderscore )}
\end{itemize}
Temor mórbido dos venenos.
\section{Inópsida}
\begin{itemize}
\item {Grp. gram.:f.}
\end{itemize}
\begin{itemize}
\item {Proveniência:(Do gr. \textunderscore ion\textunderscore  + \textunderscore opsis\textunderscore )}
\end{itemize}
Espécie de orchídea da Ásia tropical.
\section{Inopsídio}
\begin{itemize}
\item {Grp. gram.:m.}
\end{itemize}
\begin{itemize}
\item {Proveniência:(Do rad. de \textunderscore inópsida\textunderscore )}
\end{itemize}
Planta crucífera, cultivada em Portugal.
\section{Inópsido}
\begin{itemize}
\item {Grp. gram.:m.}
\end{itemize}
\begin{itemize}
\item {Proveniência:(Do gr. \textunderscore ion\textunderscore  + \textunderscore opsis\textunderscore )}
\end{itemize}
Espécie de orchídea da Ásia tropical.
\section{Iorímans}
\begin{itemize}
\item {Grp. gram.:m. pl.}
\end{itemize}
Nação de Índios do Brasil, ao norte do rio Iapura, aflluente do Amazonas.
\section{Iota}
\begin{itemize}
\item {Grp. gram.:m.}
\end{itemize}
Nome de letra, que no alphabeto grego corresponde ao nosso \textunderscore i\textunderscore .
\section{Iotacismo}
\begin{itemize}
\item {Grp. gram.:m.}
\end{itemize}
\begin{itemize}
\item {Utilização:Gram.}
\end{itemize}
Emprêgo excessivo do \textunderscore i\textunderscore .
Confusão, resultante, do emprêgo arbitrário do \textunderscore i\textunderscore  por \textunderscore j\textunderscore , e vice-versa.
Difficuldade na pronúncia do \textunderscore j\textunderscore  ou do \textunderscore g\textunderscore , doce.
(G. \textunderscore iotakismos\textunderscore )
\section{Ióthio}
\begin{itemize}
\item {Grp. gram.:m.}
\end{itemize}
\begin{itemize}
\item {Utilização:Pharm.}
\end{itemize}
Líquido xaroposo, que é um succedâneo do iodeto de potássio.
\section{Iótio}
\begin{itemize}
\item {Grp. gram.:m.}
\end{itemize}
\begin{itemize}
\item {Utilização:Pharm.}
\end{itemize}
Líquido xaroposo, que é um succedâneo do iodeto de potássio.
\section{Iotisação}
\begin{itemize}
\item {Grp. gram.:f.}
\end{itemize}
\begin{itemize}
\item {Utilização:Philol.}
\end{itemize}
\begin{itemize}
\item {Proveniência:(De \textunderscore iota\textunderscore )}
\end{itemize}
Desenvolvimento de um \textunderscore i\textunderscore  entre \textunderscore a\textunderscore  ou \textunderscore e\textunderscore  final de uma palavra e \textunderscore a\textunderscore  ou \textunderscore e\textunderscore  tónicos iniciaes de palavra immedita: \textunderscore A i água\textunderscore  = \textunderscore a água\textunderscore ; \textunderscore a i eito\textunderscore  = \textunderscore a eito\textunderscore .
\section{Ipadu}
\begin{itemize}
\item {Grp. gram.:m.}
\end{itemize}
O mesmo que \textunderscore coca\textunderscore .
\section{Ipé}
\begin{itemize}
\item {Grp. gram.:m.}
\end{itemize}
Gênero de árvores bignoniáceas do Brasil e da África.
\section{Ipeca}
\begin{itemize}
\item {Grp. gram.:f.}
\end{itemize}
(Abrev. de \textunderscore ipecacuanha\textunderscore )
\section{Ipecacuanha}
\begin{itemize}
\item {Grp. gram.:f.}
\end{itemize}
Nome de várias plantas medicinaes, umas violáceas e outras rubiáceas.
\section{Ipemerim}
\begin{itemize}
\item {Grp. gram.:m.}
\end{itemize}
\begin{itemize}
\item {Utilização:Bras}
\end{itemize}
Árvore silvestre, empregada em construcções.
\section{Ipes}
\begin{itemize}
\item {Grp. gram.:m. pl.}
\end{itemize}
\begin{itemize}
\item {Proveniência:(Do gr. \textunderscore ips\textunderscore )}
\end{itemize}
Gênero de insectos coleópteros pentâmeros.
\section{Ipéuna}
\begin{itemize}
\item {Grp. gram.:f.}
\end{itemize}
\begin{itemize}
\item {Utilização:Bras}
\end{itemize}
Árvore silvestre, bôa para construcções.
Cp. \textunderscore ipeúva\textunderscore .--Se não são árvores distintas, é possível que, das duas fórmas, uma se tenha tomado erradamente por outra. Cf. B. C. Rubim, \textunderscore Vocab. Bras.\textunderscore 
\section{Ipeúva}
\begin{itemize}
\item {Grp. gram.:f.}
\end{itemize}
\begin{itemize}
\item {Utilização:Bras}
\end{itemize}
Espécie de ipé.
O mesmo que \textunderscore cinco-fôlhas\textunderscore .
\section{Ipim}
\begin{itemize}
\item {Grp. gram.:m.}
\end{itemize}
Espécie de mandioca do Peru.
\section{Ipo}
\begin{itemize}
\item {Grp. gram.:m.}
\end{itemize}
\begin{itemize}
\item {Proveniência:(T. mal.)}
\end{itemize}
Árvore venenosa do archipélago da Malásia.
Veneno, extrahido da resina daquella árvore, e para o qual se não conhecia antídoto.
\section{Ipoméa}
\begin{itemize}
\item {Grp. gram.:f.}
\end{itemize}
\begin{itemize}
\item {Utilização:Bras}
\end{itemize}
\begin{itemize}
\item {Proveniência:(Do gr. \textunderscore ips\textunderscore  + \textunderscore homoios\textunderscore )}
\end{itemize}
Planta trepadeira, que dá flôres brancas e raiadas de azul.
\section{Ipomeia}
\begin{itemize}
\item {Grp. gram.:f.}
\end{itemize}
\begin{itemize}
\item {Utilização:Bras}
\end{itemize}
\begin{itemize}
\item {Proveniência:(Do gr. \textunderscore ips\textunderscore  + \textunderscore homoios\textunderscore )}
\end{itemize}
Planta trepadeira, que dá flôres brancas e raiadas de azul.
\section{Ipre}
\begin{itemize}
\item {Grp. gram.:m.}
\end{itemize}
Espécie de tecido antigo. Cf. Herculano, \textunderscore Cister\textunderscore , 75; Rebello, \textunderscore Contos e Lendas\textunderscore , 150; Arn. Gama, \textunderscore Baldaia\textunderscore , 373.
\section{Ipres}
\begin{itemize}
\item {Grp. gram.:m.}
\end{itemize}
Espécie de tecido antigo. Cf. Herculano, \textunderscore Cister\textunderscore , 75; Rebello, \textunderscore Contos e Lendas\textunderscore , 150; Arn. Gama, \textunderscore Baldaia\textunderscore , 373.
\section{Ípsido}
\begin{itemize}
\item {Grp. gram.:adj.}
\end{itemize}
\begin{itemize}
\item {Grp. gram.:M. pl.}
\end{itemize}
\begin{itemize}
\item {Proveniência:(Do gr. \textunderscore ips\textunderscore  + \textunderscore eidos\textunderscore )}
\end{itemize}
Semelhante aos ipes.
Insectos, que têm por typo o gênero ipes.
\section{Ipsiladora}
\begin{itemize}
\item {Grp. gram.:f.}
\end{itemize}
Apparelho, para projectar gás antiséptico nos ferimentos.
\section{Ipsilene}
\begin{itemize}
\item {Grp. gram.:m.}
\end{itemize}
O mesmo que \textunderscore ipsilênio\textunderscore .
\section{Ipsilênio}
\begin{itemize}
\item {Grp. gram.:m.}
\end{itemize}
\begin{itemize}
\item {Utilização:Neol.}
\end{itemize}
Qualquer desinfectante, em que entra chloreto antiséptico.
\section{Ipsilo}
\begin{itemize}
\item {Grp. gram.:m.}
\end{itemize}
\begin{itemize}
\item {Utilização:Neol.}
\end{itemize}
Chloreto de ethylo puro.
\section{Ipsilos}
\begin{itemize}
\item {Grp. gram.:m.}
\end{itemize}
\begin{itemize}
\item {Utilização:Neol.}
\end{itemize}
Chloreto de ethylo puro.
\section{Ípsola}
\begin{itemize}
\item {Grp. gram.:f.}
\end{itemize}
Espécie de lan, que se fabríca em Constantinopla.
\section{Ipu}
\begin{itemize}
\item {Grp. gram.:m.}
\end{itemize}
\begin{itemize}
\item {Utilização:Bras}
\end{itemize}
O mesmo que \textunderscore jalapa\textunderscore ^1.
\section{Ipu}
\begin{itemize}
\item {Grp. gram.:m.}
\end{itemize}
\begin{itemize}
\item {Utilização:Bras. do Ceará}
\end{itemize}
Terreno húmido, adjacente ás montanhas e por onde corre a água que dellas deriva.
\section{Ipueira}
\begin{itemize}
\item {Grp. gram.:f.}
\end{itemize}
\begin{itemize}
\item {Utilização:Bras}
\end{itemize}
\begin{itemize}
\item {Proveniência:(De \textunderscore ipu\textunderscore ^2)}
\end{itemize}
Charco, formado pelo trasbordamento dos rios em lugares baixos.
\section{Ipurinans}
\begin{itemize}
\item {Grp. gram.:m. pl.}
\end{itemize}
Tríbo numerosa e aguerrida das margens do Purus, no Brasil.
\section{Iquetária}
\begin{itemize}
\item {Grp. gram.:f.}
\end{itemize}
Planta escrofularínea do Brasil.
\section{Iquitos}
\begin{itemize}
\item {Grp. gram.:m.}
\end{itemize}
Índios do Peru.
\section{Ir}
\begin{itemize}
\item {Grp. gram.:v. i.}
\end{itemize}
\begin{itemize}
\item {Grp. gram.:Loc.}
\end{itemize}
\begin{itemize}
\item {Utilização:fam.}
\end{itemize}
\begin{itemize}
\item {Grp. gram.:Loc.}
\end{itemize}
\begin{itemize}
\item {Utilização:fam.}
\end{itemize}
\begin{itemize}
\item {Grp. gram.:Loc.}
\end{itemize}
\begin{itemize}
\item {Utilização:burl.}
\end{itemize}
\begin{itemize}
\item {Grp. gram.:Loc.}
\end{itemize}
\begin{itemize}
\item {Utilização:burl.}
\end{itemize}
\begin{itemize}
\item {Proveniência:(Lat. \textunderscore ire\textunderscore )}
\end{itemize}
Deslocar-se, mover-se de um lado para outro.
Mover-se, afastando-se.
Dirigir-se: \textunderscore foi ao Pôrto\textunderscore .
Caminhar: \textunderscore vai depressa, rapaz\textunderscore .
Correr.
Progredir.
Passar: \textunderscore vai ali o João\textunderscore .
Jornadear.
Comportar-se: \textunderscore vais mal, Francisco\textunderscore .
Distar: \textunderscore daqui á aldeia vão quatro kilómetros\textunderscore .
Acontecer: \textunderscore isso foi em tempo\textunderscore .
Attingir.
\textunderscore Ir aos ares\textunderscore , enfurecer-se.
\textunderscore Ir á serra\textunderscore , encavacar.
\textunderscore Ir aos fungões de\textunderscore , esbofetear.
\textunderscore Ir á pavana de\textunderscore , sovar.
\textunderscore Ir para os anjinhos\textunderscore , morrer.
\textunderscore Ir tudo raso\textunderscore , haver pancadaria, tumulto.--Muitas outras significações tem êste verbo, que com rigor só se conhecerão pela construcção ou contexto das proposições.
\section{Ira}
\begin{itemize}
\item {Grp. gram.:f.}
\end{itemize}
\begin{itemize}
\item {Proveniência:(Lat. \textunderscore ira\textunderscore )}
\end{itemize}
Mágoa ou paixão que a injúria desperta na pessôa injuriada.
Raiva, cólera.
Desejo de vingança.
\section{Irã}
\begin{itemize}
\item {Grp. gram.:f.}
\end{itemize}
\begin{itemize}
\item {Utilização:Prov.}
\end{itemize}
\begin{itemize}
\item {Utilização:beir.}
\end{itemize}
Espécie de castanha.
\section{Iracarura}
\begin{itemize}
\item {Grp. gram.:f.}
\end{itemize}
Árvore brasileira, própria para construcções.
\section{Iracundamente}
\begin{itemize}
\item {Grp. gram.:adv.}
\end{itemize}
De modo iracundo.
\section{Iracúndia}
\begin{itemize}
\item {Grp. gram.:f.}
\end{itemize}
\begin{itemize}
\item {Proveniência:(Lat. \textunderscore iracundia\textunderscore )}
\end{itemize}
Qualidade de quem é iracundo.
\section{Iracundo}
\begin{itemize}
\item {Grp. gram.:adj.}
\end{itemize}
\begin{itemize}
\item {Proveniência:(Lat. \textunderscore iracundus\textunderscore )}
\end{itemize}
Que tem tendência para ira.
Irascível.
\section{Iradamente}
\begin{itemize}
\item {Grp. gram.:adv.}
\end{itemize}
De modo irado, com ira; colericamente.
\section{Irade}
\begin{itemize}
\item {Grp. gram.:m.}
\end{itemize}
Decreto do sultão da Turquia.
(Do turco)
\section{Irado}
\begin{itemize}
\item {Grp. gram.:adj.}
\end{itemize}
\begin{itemize}
\item {Proveniência:(De \textunderscore irar\textunderscore )}
\end{itemize}
Enraivecido; colérico; assanhado.
\section{Iraíba}
\begin{itemize}
\item {Grp. gram.:f.}
\end{itemize}
Espécie de palmeira do Brasil.
\section{Iramá}
\begin{itemize}
\item {Grp. gram.:adv.}
\end{itemize}
\begin{itemize}
\item {Utilização:Açor}
\end{itemize}
O mesmo que \textunderscore eramá\textunderscore .
\section{Iran}
\begin{itemize}
\item {Grp. gram.:f.}
\end{itemize}
\begin{itemize}
\item {Utilização:Prov.}
\end{itemize}
\begin{itemize}
\item {Utilização:beir.}
\end{itemize}
Espécie de castanha.
\section{Iraniano}
\begin{itemize}
\item {Grp. gram.:adj.}
\end{itemize}
Relativo ao Iran.
\section{Irânico}
\begin{itemize}
\item {Grp. gram.:adj.}
\end{itemize}
Relativo ao Iran.
\section{Irapuia}
\begin{itemize}
\item {Grp. gram.:m.}
\end{itemize}
\begin{itemize}
\item {Utilização:Bras}
\end{itemize}
Espécie de abelha.
\section{Irar}
\begin{itemize}
\item {Grp. gram.:v. t.}
\end{itemize}
Produzir ira em; irritar.
\section{Irara}
\begin{itemize}
\item {Grp. gram.:f.}
\end{itemize}
\begin{itemize}
\item {Utilização:Bras}
\end{itemize}
Espécie de mammífero carnívoro.
O mesmo que \textunderscore irará\textunderscore ?
\section{Irará}
\begin{itemize}
\item {Grp. gram.:m.}
\end{itemize}
Espécie de animaes mustellídeos.
Cp. \textunderscore irara\textunderscore .
\section{Irascibilidade}
\begin{itemize}
\item {Grp. gram.:f.}
\end{itemize}
Qualidade de quem é irascível.
\section{Irascível}
\begin{itemize}
\item {Grp. gram.:adj.}
\end{itemize}
\begin{itemize}
\item {Proveniência:(Lat. \textunderscore irascibilis\textunderscore )}
\end{itemize}
Que se irrita facilmente.
\section{Iratassiôa}
\begin{itemize}
\item {Grp. gram.:f.}
\end{itemize}
\begin{itemize}
\item {Utilização:Bras. do N}
\end{itemize}
Raiz cheirosa, com que se perfumam roupas.
\section{Irenarca}
\begin{itemize}
\item {Grp. gram.:m.}
\end{itemize}
\begin{itemize}
\item {Proveniência:(Do gr. \textunderscore eirene\textunderscore  + \textunderscore arkhein\textunderscore )}
\end{itemize}
Official, encarregado de manter a paz, nas províncias do Império Romano do Oriente. Cf. Herculano, \textunderscore Hist. de Port.\textunderscore , I, 22.
\section{Irene}
\begin{itemize}
\item {Grp. gram.:f.}
\end{itemize}
\begin{itemize}
\item {Proveniência:(Do gr. \textunderscore eirene\textunderscore )}
\end{itemize}
Planeta, descoberto em 1851.
\section{Ireno}
\begin{itemize}
\item {Grp. gram.:m.}
\end{itemize}
\begin{itemize}
\item {Proveniência:(Do gr. \textunderscore eirene\textunderscore )}
\end{itemize}
Chefe dos moços espartanos, nos exercícios gymnásticos e militares.
Moço espartano de 20 a 30 annos.
\section{Irerez}
\begin{itemize}
\item {Grp. gram.:m.}
\end{itemize}
Ave palmípede.
\section{Ires}
\begin{itemize}
\item {Grp. gram.:f.}
\end{itemize}
\begin{itemize}
\item {Utilização:Des.}
\end{itemize}
Certo peixe dos rios do Minho. Cf. P. Carvalho, \textunderscore Congr. Port.\textunderscore , I, 297 e 311.
(Por \textunderscore íris\textunderscore ?)
\section{Iresina}
\begin{itemize}
\item {Grp. gram.:f.}
\end{itemize}
\begin{itemize}
\item {Proveniência:(Do gr. \textunderscore eyros\textunderscore , lan)}
\end{itemize}
Planta chenopodiácea.
\section{Irgadilho}
\begin{itemize}
\item {Grp. gram.:m.}
\end{itemize}
\begin{itemize}
\item {Utilização:Prov.}
\end{itemize}
Dobadoira.
\section{Irguiço}
\begin{itemize}
\item {Grp. gram.:m.}
\end{itemize}
\begin{itemize}
\item {Utilização:Prov.}
\end{itemize}
\begin{itemize}
\item {Utilização:dur.}
\end{itemize}
Caruma sêca.
\section{Iriante}
\begin{itemize}
\item {Grp. gram.:adj.}
\end{itemize}
Que iria; scintillante.
\section{Iriar}
\begin{itemize}
\item {Grp. gram.:v. t.}
\end{itemize}
Dar as côres do íris á.
Abrilhantar; matizar.
\section{Iriarana}
\begin{itemize}
\item {Grp. gram.:f.}
\end{itemize}
\begin{itemize}
\item {Utilização:Bras}
\end{itemize}
Arvore do sertão.
\section{Iriartéa}
\begin{itemize}
\item {Grp. gram.:f.}
\end{itemize}
\begin{itemize}
\item {Proveniência:(De \textunderscore Iriarte\textunderscore , n. p.)}
\end{itemize}
Espécie de areca da América tropical.
\section{Iriarteia}
\begin{itemize}
\item {Grp. gram.:f.}
\end{itemize}
\begin{itemize}
\item {Proveniência:(De \textunderscore Iriarte\textunderscore , n. p.)}
\end{itemize}
Espécie de areca da América tropical.
\section{Iricuzeiro}
\begin{itemize}
\item {Grp. gram.:m.}
\end{itemize}
Arvore sertaneja do Brasil.
\section{Iridação}
\begin{itemize}
\item {Grp. gram.:f.}
\end{itemize}
Propriedade, que têm certos animaes, de produzir no órgão da vista a impressão das côres do iris.
(Cp. \textunderscore iridescente\textunderscore )
\section{Iridáceas}
\begin{itemize}
\item {Grp. gram.:f. pl.}
\end{itemize}
O mesmo ou melhor que \textunderscore irídias\textunderscore .
\section{Iridapso}
\begin{itemize}
\item {Grp. gram.:m.}
\end{itemize}
(V.artocarpo)
\section{Irídeas}
\begin{itemize}
\item {Grp. gram.:f. pl.}
\end{itemize}
\begin{itemize}
\item {Proveniência:(De \textunderscore íris\textunderscore  + gr. \textunderscore eidos\textunderscore )}
\end{itemize}
Família de plantas, que têm por typo o gênero íris.
\section{Iridectomia}
\begin{itemize}
\item {Grp. gram.:f.}
\end{itemize}
\begin{itemize}
\item {Utilização:Cir.}
\end{itemize}
\begin{itemize}
\item {Proveniência:(Do gr. \textunderscore iris\textunderscore  + \textunderscore tome\textunderscore )}
\end{itemize}
Operação, em que se faz a excísão de um pedaço da íris.
\section{Iridectopia}
\begin{itemize}
\item {Grp. gram.:f.}
\end{itemize}
\begin{itemize}
\item {Utilização:Anat.}
\end{itemize}
\begin{itemize}
\item {Proveniência:(Do gr. \textunderscore iris\textunderscore  + \textunderscore ek\textunderscore  + \textunderscore topos\textunderscore )}
\end{itemize}
Falsa posição da íris.
\section{Iridemia}
\begin{itemize}
\item {Grp. gram.:f.}
\end{itemize}
O mesmo que \textunderscore aniria\textunderscore .
\section{Iridescente}
\begin{itemize}
\item {Grp. gram.:adj.}
\end{itemize}
\begin{itemize}
\item {Utilização:Neol.}
\end{itemize}
\begin{itemize}
\item {Proveniência:(Fr. \textunderscore eridescente\textunderscore )}
\end{itemize}
Que reflecte as côres do arco-íris.
\section{Iridiano}
\begin{itemize}
\item {Grp. gram.:adj.}
\end{itemize}
\begin{itemize}
\item {Utilização:Anat.}
\end{itemize}
Relativo á íris.
\section{Iridífero}
\begin{itemize}
\item {Grp. gram.:adj.}
\end{itemize}
\begin{itemize}
\item {Proveniência:(De \textunderscore irídio\textunderscore  + lat. \textunderscore ferre\textunderscore )}
\end{itemize}
Que contém irídio.
\section{Irídico}
\begin{itemize}
\item {Grp. gram.:adj.}
\end{itemize}
Relativo ao irídio.
\section{Iridina}
\begin{itemize}
\item {Grp. gram.:f.}
\end{itemize}
Gênero de molluscos acéphalos.
\section{Iridíneas}
\begin{itemize}
\item {Grp. gram.:f. pl.}
\end{itemize}
\begin{itemize}
\item {Utilização:Bot.}
\end{itemize}
Gênero de irídeas.
\section{Irídio}
\begin{itemize}
\item {Grp. gram.:m.}
\end{itemize}
Metal friável e esbranquiçado, que dá soluções de cores variadíssimas.
(Do. rad. de \textunderscore íris\textunderscore )
\section{Iridite}
\begin{itemize}
\item {Grp. gram.:f.}
\end{itemize}
O mesmo que \textunderscore irite\textunderscore .
\section{Iridodiálise}
\begin{itemize}
\item {Grp. gram.:f.}
\end{itemize}
\begin{itemize}
\item {Proveniência:(Do gr. \textunderscore íris\textunderscore  + \textunderscore dialusis\textunderscore )}
\end{itemize}
Estado, em que a íris se apresenta desprendida, pelo seu bordo ciliar, da sua inserção.
\section{Iridodiályse}
\begin{itemize}
\item {Grp. gram.:f.}
\end{itemize}
\begin{itemize}
\item {Proveniência:(Do gr. \textunderscore íris\textunderscore  + \textunderscore dialusis\textunderscore )}
\end{itemize}
Estado, em que a íris se apresenta desprendida, pelo seu bordo ciliar, da sua inserção.
\section{Iridodónese}
\begin{itemize}
\item {Grp. gram.:f.}
\end{itemize}
\begin{itemize}
\item {Proveniência:(Do gr. \textunderscore íris\textunderscore  + \textunderscore donestai\textunderscore , ondular)}
\end{itemize}
Ondulação da íris, por falta de apoio no crystallino.
\section{Iridoplegia}
\begin{itemize}
\item {Grp. gram.:f.}
\end{itemize}
\begin{itemize}
\item {Proveniência:(Do gr. \textunderscore iris\textunderscore  + \textunderscore plessein\textunderscore )}
\end{itemize}
Paralysia da íris.
\section{Iridoplégico}
\begin{itemize}
\item {Grp. gram.:adj.}
\end{itemize}
Relativo á iridoplegia.
\section{Iridotomia}
\begin{itemize}
\item {Grp. gram.:f.}
\end{itemize}
\begin{itemize}
\item {Proveniência:(Do gr. \textunderscore iris\textunderscore  + \textunderscore tomè\textunderscore )}
\end{itemize}
Incisão cirúrgica na íris.
\section{Irijus}
\begin{itemize}
\item {Grp. gram.:m. pl.}
\end{itemize}
Tríbo de aborígenes do Pará.
\section{Iriribá}
\begin{itemize}
\item {Grp. gram.:m.}
\end{itemize}
\begin{itemize}
\item {Utilização:Bras}
\end{itemize}
Árvore do sertão.
\section{Íris}
\begin{itemize}
\item {Grp. gram.:m.}
\end{itemize}
\begin{itemize}
\item {Utilização:Anat.}
\end{itemize}
\begin{itemize}
\item {Utilização:Bot.}
\end{itemize}
\begin{itemize}
\item {Utilização:Zool.}
\end{itemize}
\begin{itemize}
\item {Utilização:Fig.}
\end{itemize}
\begin{itemize}
\item {Proveniência:(Do gr. \textunderscore Iris\textunderscore , n. p.)}
\end{itemize}
Meteóro luminoso, produzido na atmosphera, em fórma de arco, pela decomposição dos raios solares.
Cores, que, através de algumas lunetas, se observam em volta dos objectos.
Quartzo irisado.
Membrana circular e colorida, situada no interior do olho, e da qual procede a côr dos olhos.
Planta, que serve de typo as irídeas.
Espécie de borboleta diurna.
Paz.
Bonança.
Alegria.
\section{Irisação}
\begin{itemize}
\item {Grp. gram.:f.}
\end{itemize}
Acto ou effeito de irisar.
\section{Irisante}
\begin{itemize}
\item {Grp. gram.:adj.}
\end{itemize}
Que irisa. Cf. Gonç. Dias, \textunderscore Poesias\textunderscore , 163,
\section{Irisar}
\begin{itemize}
\item {Grp. gram.:v. t.}
\end{itemize}
O mesmo que \textunderscore iriar\textunderscore .
\section{Iristomia}
\begin{itemize}
\item {Grp. gram.:f.}
\end{itemize}
\begin{itemize}
\item {Utilização:Cir.}
\end{itemize}
\begin{itemize}
\item {Proveniência:(De \textunderscore íris\textunderscore  + gr. \textunderscore tome\textunderscore )}
\end{itemize}
Operação cirúrgica da extracção da íris.
\section{Irite}
\begin{itemize}
\item {Grp. gram.:f.}
\end{itemize}
Inflammação da membrana íris.
\section{Iriva}
\begin{itemize}
\item {Grp. gram.:f.}
\end{itemize}
\begin{itemize}
\item {Utilização:Prov.}
\end{itemize}
\begin{itemize}
\item {Utilização:alent.}
\end{itemize}
Calúmnia; blasphêmia.
\section{Iriz}
\begin{itemize}
\item {Grp. gram.:m.}
\end{itemize}
\begin{itemize}
\item {Utilização:Bras}
\end{itemize}
Epiphytia, peculiar ao cafezeiro.
\section{Irizar}
\begin{itemize}
\item {Grp. gram.:v. i.}
\end{itemize}
\begin{itemize}
\item {Utilização:Bras}
\end{itemize}
Sêr atacado de iriz, (falando-se do cafezeiro).
\section{Irlanda}
\begin{itemize}
\item {Grp. gram.:f.}
\end{itemize}
Tecido fino de algodão ou linho. Cf. \textunderscore Tarifa das Alfând. do Brasil\textunderscore , 62.
\section{Irlandês}
\begin{itemize}
\item {Grp. gram.:adj.}
\end{itemize}
\begin{itemize}
\item {Grp. gram.:M.}
\end{itemize}
Relativo á Irlanda.
Aquelle que é natural da Irlanda.
Língua novi-céltica, falada pelo povo na Irlanda.
\section{Irmã}
\begin{itemize}
\item {Grp. gram.:f.}
\end{itemize}
\begin{itemize}
\item {Proveniência:(De \textunderscore irmão\textunderscore )}
\end{itemize}
Aquella que, em relação a outrem, é filha do mesmo pai ou da mesma mãe, ou só do mesmo pai, ou só da mesma mãe.
Mulher, que faz parte de uma confraria.
Freira, que não exerce cargos superiores.
\section{Irman}
\begin{itemize}
\item {Grp. gram.:f.}
\end{itemize}
\begin{itemize}
\item {Proveniência:(De \textunderscore irmão\textunderscore )}
\end{itemize}
Aquella que, em relação a outrem, é filha do mesmo pai ou da mesma mãe, ou só do mesmo pai, ou só da mesma mãe.
Mulher, que faz parte de uma confraria.
Freira, que não exerce cargos superiores.
\section{Irmanar}
\begin{itemize}
\item {Grp. gram.:v. t.}
\end{itemize}
Tornar irmão, semelhante.
Igual.
Emparelhar.
\section{Irmandade}
\begin{itemize}
\item {Grp. gram.:f.}
\end{itemize}
\begin{itemize}
\item {Proveniência:(Do lat. \textunderscore germanitas\textunderscore )}
\end{itemize}
Parentesco de irmãos.
Confraternidade, intimidade.
Associação, liga, confraria.
\section{Irmanmente}
\begin{itemize}
\item {Grp. gram.:adv.}
\end{itemize}
\begin{itemize}
\item {Proveniência:(De \textunderscore irmão\textunderscore )}
\end{itemize}
Á maneira de irmãos; com igualdade: \textunderscore viver irmanmente\textunderscore .
\section{Irmanzinhas-dos-pobres}
\begin{itemize}
\item {Grp. gram.:f. pl.}
\end{itemize}
Congregação religiosa, fundada em 1840 e destinada a tratar de velhos pobres.
\section{Irmão}
\begin{itemize}
\item {Grp. gram.:m.}
\end{itemize}
\begin{itemize}
\item {Utilização:Fig.}
\end{itemize}
\begin{itemize}
\item {Utilização:T. de Turquel}
\end{itemize}
\begin{itemize}
\item {Utilização:Ant.}
\end{itemize}
\begin{itemize}
\item {Grp. gram.:Adj.}
\end{itemize}
\begin{itemize}
\item {Proveniência:(Do lat. \textunderscore germanus\textunderscore )}
\end{itemize}
Aquelle que, em relação a outrem, é filho do mesmo pai e da mesma mãe, ou só do mesmo pai, ou só da mesma mãe.
Cada um dos membros de uma confraria.
Correligionário.
Membro da maçonaria.
Amigo inseparável.
Frade, que não exercia cargos superiores.
Mendigo.
\textunderscore Irmão das portas\textunderscore , o mesmo que \textunderscore mendigo\textunderscore .
\textunderscore Meio irmão\textunderscore , irmão, que é filho de outro pai ou de outra mãe.
Igual; que emparelha com outro.
\section{Irmo}
\begin{itemize}
\item {Grp. gram.:m.}
\end{itemize}
\begin{itemize}
\item {Utilização:Gír.}
\end{itemize}
Irmão.
\section{Iró}
\begin{itemize}
\item {Grp. gram.:f.}
\end{itemize}
(V. \textunderscore eiró\textunderscore ^1)
\section{Ironia}
\begin{itemize}
\item {Grp. gram.:f.}
\end{itemize}
\begin{itemize}
\item {Proveniência:(Lat. \textunderscore ironia\textunderscore )}
\end{itemize}
Figura de rhetórica, em que se exprime o contrário do que as palavras naturalmente significam.
Sarcasmo; zombaria.
\section{Ironicamente}
\begin{itemize}
\item {Grp. gram.:adv.}
\end{itemize}
De modo irónico.
\section{Irónico}
\begin{itemize}
\item {Grp. gram.:adj.}
\end{itemize}
\begin{itemize}
\item {Proveniência:(Lat. \textunderscore ironicus\textunderscore )}
\end{itemize}
Em que há ironia; sarcástico; zombeteiro.
\section{Ironizar}
\begin{itemize}
\item {Grp. gram.:v. t.}
\end{itemize}
\begin{itemize}
\item {Utilização:Neol.}
\end{itemize}
Tornar irónico.
Exprimir com ironia. Cf. A. Cândido, \textunderscore Phil. Polít.\textunderscore , 113.
\section{Iroquês}
\begin{itemize}
\item {Grp. gram.:m.}
\end{itemize}
\begin{itemize}
\item {Grp. gram.:Adj.}
\end{itemize}
\begin{itemize}
\item {Grp. gram.:M. pl.}
\end{itemize}
Uma das línguas dos indigenas da América do Norte.
Relativo aos Iroqueses.
Antiga tribo da America do Norte, espalhada hoje ao Sul de Canadá e ao norte dos Estados-Unidos.
\section{Irós}
\begin{itemize}
\item {Grp. gram.:f.}
\end{itemize}
(V. \textunderscore eiró\textunderscore ^1)
\section{Irosamente}
\begin{itemize}
\item {Grp. gram.:adv.}
\end{itemize}
De modo iroso; iradamente.
\section{Irosina}
\begin{itemize}
\item {Grp. gram.:f.}
\end{itemize}
\begin{itemize}
\item {Proveniência:(Do gr. \textunderscore iros\textunderscore )}
\end{itemize}
Gênero de plantas amarantáceas da América e da Austrália.
\section{Iroso}
\begin{itemize}
\item {Grp. gram.:adj.}
\end{itemize}
\begin{itemize}
\item {Utilização:Fig.}
\end{itemize}
Em que há ira; irado; irascivel.
Tempestuoso: \textunderscore tempo iroso\textunderscore .
\section{Iróz}
\begin{itemize}
\item {Grp. gram.:f.}
\end{itemize}
(V. \textunderscore eiró\textunderscore ^1)
\section{Irra!}
\begin{itemize}
\item {Grp. gram.:interj.}
\end{itemize}
\begin{itemize}
\item {Utilização:Pleb.}
\end{itemize}
Significa repulsão, raiva, desprêzo.
Apre! é demais!
\section{Irracional}
\begin{itemize}
\item {Grp. gram.:adj.}
\end{itemize}
\begin{itemize}
\item {Utilização:Mathem.}
\end{itemize}
\begin{itemize}
\item {Grp. gram.:M.}
\end{itemize}
\begin{itemize}
\item {Proveniência:(Lat. \textunderscore irrationalis\textunderscore )}
\end{itemize}
Que não é racional.
Que não tem a faculdade do raciocinio.
Opposto á razão.
Diz-se da quantidade, cuja relação com a unidade não póde sêr expressa em números.
Diz-se da expressão algébrica, cujos radicaes se não pódem eliminar.
Animal, que não tem a faculdade do raciocinio; bruto: \textunderscore tratar bem os irracionaes\textunderscore .
\section{Irracionalidade}
\begin{itemize}
\item {Grp. gram.:f.}
\end{itemize}
\begin{itemize}
\item {Proveniência:(Lat. \textunderscore irrationalitas\textunderscore )}
\end{itemize}
Qualidade de irracional.
Falta de raciocinio.
\section{Irracionalmente}
\begin{itemize}
\item {Grp. gram.:adv.}
\end{itemize}
De modo irracional.
\section{Irracionável}
\begin{itemize}
\item {Grp. gram.:adj.}
\end{itemize}
\begin{itemize}
\item {Proveniência:(Do lat. \textunderscore irrationabilis\textunderscore )}
\end{itemize}
O mesmo que \textunderscore irracional\textunderscore .
Que não tem fundamento ou causa.
Opposto á bôa razão.
\section{Irracionavelmente}
\begin{itemize}
\item {Grp. gram.:adv.}
\end{itemize}
De modo irracionável.
\section{Irradiação}
\begin{itemize}
\item {Grp. gram.:f.}
\end{itemize}
Acto ou effeito de irradiar.
\section{Irradiador}
\begin{itemize}
\item {Grp. gram.:adj.}
\end{itemize}
Que irradia.
\section{Irradiante}
\begin{itemize}
\item {Grp. gram.:adj.}
\end{itemize}
Que irradia.
Brilhante.
\section{Irradiar}
\begin{itemize}
\item {Grp. gram.:v. t.}
\end{itemize}
\begin{itemize}
\item {Grp. gram.:V. i.}
\end{itemize}
\begin{itemize}
\item {Proveniência:(Lat. \textunderscore irradiare\textunderscore )}
\end{itemize}
Lançar de si (raios luminosos).
Diffundir, em sentido centrifugo.
Espalhar: \textunderscore irradiar benefícios\textunderscore .
Expedir raios luminosos.
Propagar-se, partindo de um ponto central.
Espalhar-se; desenvolver-se: \textunderscore a peste irradiou da Ásia\textunderscore .
\section{Irradiator}
\begin{itemize}
\item {Grp. gram.:m.}
\end{itemize}
\begin{itemize}
\item {Proveniência:(Lat. hyp. \textunderscore irradiator\textunderscore )}
\end{itemize}
Apparelho com muitos tubos, em contacto com o ar, e onde esfria a água, para fazer esfriar o cylindro, nos motores de petróleo. Cf. Benevides, \textunderscore Automóveis\textunderscore .
\section{Irradioso}
\begin{itemize}
\item {Grp. gram.:adj.}
\end{itemize}
\begin{itemize}
\item {Proveniência:(De \textunderscore in...\textunderscore  + \textunderscore radioso\textunderscore )}
\end{itemize}
Não radioso.
\section{Irreal}
\begin{itemize}
\item {Grp. gram.:adj.}
\end{itemize}
\begin{itemize}
\item {Proveniência:(De \textunderscore in...\textunderscore  + \textunderscore real\textunderscore )}
\end{itemize}
Que não existe realmente; imaginário.
\section{Irrealizabilidade}
\begin{itemize}
\item {Grp. gram.:f.}
\end{itemize}
Qualidade de irrealizável. Cf. S. Romero, \textunderscore M. de Assis\textunderscore , 230.
\section{Irrealizável}
\begin{itemize}
\item {Grp. gram.:adj.}
\end{itemize}
\begin{itemize}
\item {Proveniência:(De \textunderscore in...\textunderscore  + \textunderscore realizável\textunderscore )}
\end{itemize}
Não realizável, que se não póde realizar; inexequivel: \textunderscore planos irrealizáveis\textunderscore .
\section{Irreclamável}
\begin{itemize}
\item {Grp. gram.:adj.}
\end{itemize}
\begin{itemize}
\item {Proveniência:(De \textunderscore in...\textunderscore  + \textunderscore reclamável\textunderscore )}
\end{itemize}
Que não póde ou não deve sêr reclamado.
\section{Irreconciliado}
\begin{itemize}
\item {Grp. gram.:adj.}
\end{itemize}
\begin{itemize}
\item {Proveniência:(De \textunderscore in...\textunderscore  + \textunderscore reconciliado\textunderscore )}
\end{itemize}
Não reconciliado.
\section{Irreconciliável}
\begin{itemize}
\item {Grp. gram.:adj.}
\end{itemize}
\begin{itemize}
\item {Proveniência:(De \textunderscore in...\textunderscore  + \textunderscore reconciliável\textunderscore )}
\end{itemize}
Que se não póde reconciliar.
\section{Irreconciliavelmente}
\begin{itemize}
\item {Grp. gram.:adv.}
\end{itemize}
De modo irreconciliável.
\section{Irrecuperável}
\begin{itemize}
\item {Grp. gram.:adj.}
\end{itemize}
\begin{itemize}
\item {Proveniência:(De \textunderscore in...\textunderscore  + \textunderscore recuperável\textunderscore )}
\end{itemize}
Que se não póde recuperar.
\section{Irrecuperavelmente}
\begin{itemize}
\item {Grp. gram.:adv.}
\end{itemize}
De modo irrecuperável.
\section{Irrecusável}
\begin{itemize}
\item {Grp. gram.:adj.}
\end{itemize}
\begin{itemize}
\item {Proveniência:(Lat. \textunderscore irrecusabilis\textunderscore )}
\end{itemize}
Que se não póde recusar; incontestável.
\section{Irrecusavelmente}
\begin{itemize}
\item {Grp. gram.:adv.}
\end{itemize}
De modo irrecusável.
\section{Irredentismo}
\begin{itemize}
\item {Grp. gram.:m.}
\end{itemize}
\begin{itemize}
\item {Proveniência:(Do it. \textunderscore irredento\textunderscore , não resgatado)}
\end{itemize}
Partido ou doutrina dos que entendem que a Itália deve abranger, além das suas fronteiras actuaes, as regiões que estão ligadas pela língua e pelos costumes, mas separadas pela politica.
\section{Irredentista}
\begin{itemize}
\item {Grp. gram.:adj.}
\end{itemize}
\begin{itemize}
\item {Grp. gram.:M.}
\end{itemize}
Relativo ao irredentismo.
Partidário do irredentismo.
\section{Irredimível}
\begin{itemize}
\item {Grp. gram.:adj.}
\end{itemize}
\begin{itemize}
\item {Proveniência:(De \textunderscore in...\textunderscore  + \textunderscore redimível\textunderscore )}
\end{itemize}
Que se não póde remir.
\section{Irreductível}
\begin{itemize}
\item {Grp. gram.:adj.}
\end{itemize}
O mesmo que \textunderscore irreduzível\textunderscore .
\section{Irredutível}
\begin{itemize}
\item {Grp. gram.:adj.}
\end{itemize}
O mesmo que \textunderscore irreduzível\textunderscore .
\section{Irreduzível}
\begin{itemize}
\item {Grp. gram.:adj.}
\end{itemize}
\begin{itemize}
\item {Proveniência:(De \textunderscore in...\textunderscore  + \textunderscore reduzível\textunderscore )}
\end{itemize}
Que não é reduzivel; que se não póde domar.
Que se não póde decompor.
Que não póde voltar ao primitivo estado.
\section{Irreelegível}
\begin{itemize}
\item {Grp. gram.:adj.}
\end{itemize}
\begin{itemize}
\item {Proveniência:(De \textunderscore in...\textunderscore  + \textunderscore reelegível\textunderscore )}
\end{itemize}
Que se não póde reeleger.
\section{Irreflectidamente}
\begin{itemize}
\item {Grp. gram.:adv.}
\end{itemize}
De modo irreflectido.
Impensadamente; sem reflexão; instintivamente.
\section{Irreflectido}
\begin{itemize}
\item {Grp. gram.:adj.}
\end{itemize}
\begin{itemize}
\item {Proveniência:(De \textunderscore in...\textunderscore  + \textunderscore reflectido\textunderscore )}
\end{itemize}
Que não reflecte, que não pondera.
Que revela falta de reflexão; impensado: \textunderscore resolução irreflectida\textunderscore .
\section{Irreflexão}
\begin{itemize}
\item {fónica:csão}
\end{itemize}
\begin{itemize}
\item {Grp. gram.:f.}
\end{itemize}
\begin{itemize}
\item {Proveniência:(De \textunderscore in...\textunderscore  + \textunderscore reflexão\textunderscore )}
\end{itemize}
Falta de reflexão; imprudência; precipitação.
\section{Irreflexivo}
\begin{itemize}
\item {fónica:csi}
\end{itemize}
\begin{itemize}
\item {Grp. gram.:adj.}
\end{itemize}
\begin{itemize}
\item {Proveniência:(De \textunderscore in...\textunderscore  + \textunderscore reflexivo\textunderscore )}
\end{itemize}
Que não usa de reflexão, que não pondera.
Inconsiderado.
\section{Irreflexo}
\begin{itemize}
\item {fónica:cso}
\end{itemize}
\begin{itemize}
\item {Grp. gram.:adj.}
\end{itemize}
\begin{itemize}
\item {Proveniência:(Lat. \textunderscore irreflexus\textunderscore )}
\end{itemize}
Que não faz reflexo.
Irreflectido.
\section{Irreformável}
\begin{itemize}
\item {Grp. gram.:adj.}
\end{itemize}
\begin{itemize}
\item {Proveniência:(Lat. \textunderscore irreformabilis\textunderscore )}
\end{itemize}
Não reformável; que se não póde reformar.
\section{Irrefragável}
\begin{itemize}
\item {Grp. gram.:adj.}
\end{itemize}
\begin{itemize}
\item {Proveniência:(Lat. \textunderscore irrefragabilis\textunderscore )}
\end{itemize}
Irrecusável; incontestável.
\section{Irrefragavelmente}
\begin{itemize}
\item {Grp. gram.:adv.}
\end{itemize}
De modo irrefragável.
\section{Irrefreável}
\begin{itemize}
\item {Grp. gram.:adj.}
\end{itemize}
\begin{itemize}
\item {Proveniência:(De \textunderscore in...\textunderscore  + \textunderscore refreável\textunderscore )}
\end{itemize}
Não refreável; que se não póde refrear.
\section{Irrefutabilidade}
\begin{itemize}
\item {Grp. gram.:f.}
\end{itemize}
Qualidade de irrefutável.
\section{Irrefutado}
\begin{itemize}
\item {Grp. gram.:adj.}
\end{itemize}
\begin{itemize}
\item {Proveniência:(Lat. \textunderscore irrefutatus\textunderscore )}
\end{itemize}
Não refutado; incontestado.
\section{Irrefutável}
\begin{itemize}
\item {Grp. gram.:adj.}
\end{itemize}
\begin{itemize}
\item {Proveniência:(Lat. \textunderscore irrefutabilis\textunderscore )}
\end{itemize}
Que se não póde refutar; irrecusável; evidente.
\section{Irrefutavelmente}
\begin{itemize}
\item {Grp. gram.:adv.}
\end{itemize}
De modo irrefutável.
\section{Irregenerado}
\begin{itemize}
\item {Grp. gram.:adj.}
\end{itemize}
\begin{itemize}
\item {Proveniência:(De \textunderscore in...\textunderscore  + \textunderscore regenerado\textunderscore )}
\end{itemize}
Não regenerado; impenitente.
\section{Irregenerável}
\begin{itemize}
\item {Grp. gram.:adj.}
\end{itemize}
\begin{itemize}
\item {Proveniência:(De \textunderscore in...\textunderscore  + \textunderscore regenerável\textunderscore )}
\end{itemize}
Que se não póde regenerar; incorrigível.
\section{Irregível}
\begin{itemize}
\item {Grp. gram.:adj.}
\end{itemize}
\begin{itemize}
\item {Proveniência:(Lat. \textunderscore irregibilis\textunderscore )}
\end{itemize}
Que se não póde reger ou domar, indomável.
Indócil; incorrigível.
\section{Irregressível}
\begin{itemize}
\item {Grp. gram.:adj.}
\end{itemize}
\begin{itemize}
\item {Proveniência:(Lat. \textunderscore irregressibilis\textunderscore )}
\end{itemize}
Que não póde regressar.
Donde não póde haver regresso.
\section{Irregular}
\begin{itemize}
\item {Grp. gram.:adj.}
\end{itemize}
\begin{itemize}
\item {Grp. gram.:M.}
\end{itemize}
\begin{itemize}
\item {Proveniência:(De \textunderscore in...\textunderscore  + \textunderscore regular\textunderscore )}
\end{itemize}
Não regular.
Opposto á justiça ou á lei: \textunderscore procedimento irregular\textunderscore .
Que não obedece ás regras: \textunderscore verbo irregular\textunderscore .
Arbitrário.
Desharmónico, desigual.
Aquelle que incorreu numa irregularidade canónica.
\section{Irregularidade}
\begin{itemize}
\item {Grp. gram.:f.}
\end{itemize}
Qualidade de irregular; falta de regularidade.
\section{Irregularmente}
\begin{itemize}
\item {Grp. gram.:adv.}
\end{itemize}
De modo irregular.
\section{Irreligião}
\begin{itemize}
\item {Grp. gram.:f.}
\end{itemize}
\begin{itemize}
\item {Proveniência:(Lat. \textunderscore irreligio\textunderscore )}
\end{itemize}
Falta de religião ou piedade.
Irreligiosidade; impiedade; atheísmo.
\section{Irreligiosamente}
\begin{itemize}
\item {Grp. gram.:adv.}
\end{itemize}
De modo irreligioso.
\section{Irreligiosidade}
\begin{itemize}
\item {Grp. gram.:f.}
\end{itemize}
\begin{itemize}
\item {Proveniência:(Lat. \textunderscore irreligiositas\textunderscore )}
\end{itemize}
Qualidade de irreligioso.
Acção irreligiosa.
\section{Irreligioso}
\begin{itemize}
\item {Grp. gram.:adj.}
\end{itemize}
\begin{itemize}
\item {Proveniência:(Lat. \textunderscore irreligiosus\textunderscore )}
\end{itemize}
Não religioso.
Incrédulo; ímpio; atheu.
\section{Irremeável}
\begin{itemize}
\item {Grp. gram.:adj.}
\end{itemize}
\begin{itemize}
\item {Proveniência:(Lat. \textunderscore irremeabilis\textunderscore )}
\end{itemize}
Por onde se não póde passar de novo.
Irregressível.
\section{Irremediável}
\begin{itemize}
\item {Grp. gram.:adj.}
\end{itemize}
\begin{itemize}
\item {Utilização:Fig.}
\end{itemize}
\begin{itemize}
\item {Proveniência:(Lat. \textunderscore irremediabilis\textunderscore )}
\end{itemize}
Não remediável.
Para que não póde haver allívio: \textunderscore mágoa irremediável\textunderscore .
Infallível; inevitável.
Irrecusável.
\section{Irremediavelmente}
\begin{itemize}
\item {Grp. gram.:adv.}
\end{itemize}
De modo irremediável.
\section{Irremissibilidade}
\begin{itemize}
\item {Grp. gram.:f.}
\end{itemize}
Qualidade de irremissível.
\section{Irremissível}
\begin{itemize}
\item {Grp. gram.:adj.}
\end{itemize}
\begin{itemize}
\item {Proveniência:(Lat. \textunderscore irremissibilis\textunderscore )}
\end{itemize}
Não perdoável, não remissível.
Infallível.
Irremediável.
\section{Irremissivelmente}
\begin{itemize}
\item {Grp. gram.:adv.}
\end{itemize}
De modo irremissível.
\section{Irremitente}
\begin{itemize}
\item {Grp. gram.:adj.}
\end{itemize}
\begin{itemize}
\item {Proveniência:(De \textunderscore in...\textunderscore  + \textunderscore remitente\textunderscore )}
\end{itemize}
Que não é remitente.
\section{Irremittente}
\begin{itemize}
\item {Grp. gram.:adj.}
\end{itemize}
\begin{itemize}
\item {Proveniência:(De \textunderscore in...\textunderscore  + \textunderscore remittente\textunderscore )}
\end{itemize}
Que não é remittente.
\section{Irremovível}
\begin{itemize}
\item {Grp. gram.:adj.}
\end{itemize}
\begin{itemize}
\item {Proveniência:(De \textunderscore in...\textunderscore  + \textunderscore removível\textunderscore )}
\end{itemize}
Não removível: \textunderscore embaraço irremovível\textunderscore .
Que se não póde evitar; que não tem remédio.
\section{Irremunerado}
\begin{itemize}
\item {Grp. gram.:adj.}
\end{itemize}
\begin{itemize}
\item {Proveniência:(Lat. \textunderscore irremuneratus\textunderscore )}
\end{itemize}
Não remunerado; que não teve remuneração.
Que não foi recompensado.
\section{Irremunerável}
\begin{itemize}
\item {Grp. gram.:adj.}
\end{itemize}
\begin{itemize}
\item {Proveniência:(Lat. \textunderscore irremunerabilis\textunderscore )}
\end{itemize}
Que não é remunerável.
Impagável.
\section{Irrenunciável}
\begin{itemize}
\item {Grp. gram.:adj.}
\end{itemize}
\begin{itemize}
\item {Proveniência:(De \textunderscore in...\textunderscore  + \textunderscore renunciável\textunderscore )}
\end{itemize}
Que se não póde renunciar.
\section{Irreparabilidade}
\begin{itemize}
\item {Grp. gram.:f.}
\end{itemize}
Qualidade de irreparável.
\section{Irreparável}
\begin{itemize}
\item {Grp. gram.:adj.}
\end{itemize}
\begin{itemize}
\item {Proveniência:(Lat. \textunderscore irreparabilis\textunderscore )}
\end{itemize}
Que se não póde recuperar.
Irremediável.
\section{Irreparavelmente}
\begin{itemize}
\item {Grp. gram.:adv.}
\end{itemize}
De modo irreparável.
\section{Irrepartível}
\begin{itemize}
\item {Grp. gram.:adj.}
\end{itemize}
\begin{itemize}
\item {Proveniência:(De \textunderscore in...\textunderscore  + \textunderscore repartível\textunderscore )}
\end{itemize}
Que não é repartível.
\section{Irreplegível}
\begin{itemize}
\item {Grp. gram.:adj.}
\end{itemize}
\begin{itemize}
\item {Utilização:Des.}
\end{itemize}
\begin{itemize}
\item {Proveniência:(Do lat. \textunderscore in...\textunderscore  + \textunderscore replere\textunderscore )}
\end{itemize}
Que se não póde encher.
Insaciável. Cf. Bernárdez, \textunderscore Nova Floresta\textunderscore .
\section{Irrepleto}
\begin{itemize}
\item {Grp. gram.:adj.}
\end{itemize}
\begin{itemize}
\item {Proveniência:(Lat. \textunderscore irrepletus\textunderscore )}
\end{itemize}
Que não está cheio.
Que não está saciado.
\section{Irreplicável}
\begin{itemize}
\item {Grp. gram.:adj.}
\end{itemize}
\begin{itemize}
\item {Proveniência:(De \textunderscore in...\textunderscore  + \textunderscore replicar\textunderscore )}
\end{itemize}
A que se não póde replicar; que não admitte réplica.
\section{Irreplicavelmente}
\begin{itemize}
\item {Grp. gram.:adv.}
\end{itemize}
De modo irreplicável.
\section{Irrepreensibilidade}
\begin{itemize}
\item {Grp. gram.:f.}
\end{itemize}
Qualidade de irrepreensível.
\section{Irrepreensível}
\begin{itemize}
\item {Grp. gram.:adj.}
\end{itemize}
\begin{itemize}
\item {Proveniência:(Lat. \textunderscore irreprehensibilis\textunderscore )}
\end{itemize}
Que não póde ou não deve sêr repreendido.
Perfeito; correcto: \textunderscore prosa irrepreensível\textunderscore .
Imaculado: \textunderscore vida irrepreensível\textunderscore .
\section{Irrepreensivelmente}
\begin{itemize}
\item {Grp. gram.:adv.}
\end{itemize}
De modo irrepreensível.
\section{Irreprehensibilidade}
\begin{itemize}
\item {Grp. gram.:f.}
\end{itemize}
Qualidade de irreprehensível.
\section{Irreprehensível}
\begin{itemize}
\item {Grp. gram.:adj.}
\end{itemize}
\begin{itemize}
\item {Proveniência:(Lat. \textunderscore irreprehensibilis\textunderscore )}
\end{itemize}
Que não póde ou não deve sêr reprehendido.
Perfeito; correcto: \textunderscore prosa irreprehensível\textunderscore .
Immaculado: \textunderscore vida irreprehensível\textunderscore .
\section{Irreprehensivelmente}
\begin{itemize}
\item {Grp. gram.:adv.}
\end{itemize}
De modo irreprehensível.
\section{Irrepresentável}
\begin{itemize}
\item {Grp. gram.:adj.}
\end{itemize}
\begin{itemize}
\item {Proveniência:(De \textunderscore in...\textunderscore  + \textunderscore representável\textunderscore )}
\end{itemize}
Que não póde sêr representado ou têr representante.
\section{Irrepressível}
\begin{itemize}
\item {Grp. gram.:adj.}
\end{itemize}
O mesmo que \textunderscore irreprimível\textunderscore .
\section{Irreprimível}
\begin{itemize}
\item {Grp. gram.:adj.}
\end{itemize}
\begin{itemize}
\item {Proveniência:(De \textunderscore in...\textunderscore  + \textunderscore reprimível\textunderscore )}
\end{itemize}
Que não é reprimível; que se não póde reprimir ou conter: \textunderscore cólera irreprimível\textunderscore .
\section{Irrequietismo}
\begin{itemize}
\item {Grp. gram.:m.}
\end{itemize}
Estado ou qualidade de irrequieto.
\section{Irrequieto}
\begin{itemize}
\item {Grp. gram.:adj.}
\end{itemize}
\begin{itemize}
\item {Proveniência:(Lat. \textunderscore irrequietus\textunderscore )}
\end{itemize}
Que não tem descanso; que nunca pára.
Turbulento; buliçoso: \textunderscore rapaz irrequieto\textunderscore .
\section{Irrequietude}
\begin{itemize}
\item {Grp. gram.:f.}
\end{itemize}
O mesmo que \textunderscore irrequietismo\textunderscore .
\section{Irresignável}
\begin{itemize}
\item {Grp. gram.:adj.}
\end{itemize}
\begin{itemize}
\item {Proveniência:(De \textunderscore in...\textunderscore  + \textunderscore resignável\textunderscore )}
\end{itemize}
Que não póde resignar-se ou conformar-se.
Que não póde sêr renunciado: \textunderscore encargo irresignável\textunderscore .
\section{Irresistência}
\begin{itemize}
\item {Grp. gram.:f.}
\end{itemize}
Qualidade de irresistente; falta de resistência.
\section{Irresistente}
\begin{itemize}
\item {Grp. gram.:adj.}
\end{itemize}
\begin{itemize}
\item {Proveniência:(De \textunderscore in...\textunderscore  + \textunderscore resistente\textunderscore )}
\end{itemize}
Que não é resistente; que não resiste.
\section{Irresistibilidade}
\begin{itemize}
\item {Grp. gram.:f.}
\end{itemize}
Qualidade de irresistível.
\section{Irresistível}
\begin{itemize}
\item {Grp. gram.:adj.}
\end{itemize}
\begin{itemize}
\item {Proveniência:(De \textunderscore in...\textunderscore  + \textunderscore resistível\textunderscore )}
\end{itemize}
A que não se póde resistir.
Que seduz: \textunderscore belleza irresistível\textunderscore .
Invencível.
Convincente.
Fatal.
\section{Irresistivelmente}
\begin{itemize}
\item {Grp. gram.:adv.}
\end{itemize}
De modo irresistível.
\section{Irresolução}
\begin{itemize}
\item {Grp. gram.:f.}
\end{itemize}
\begin{itemize}
\item {Proveniência:(De \textunderscore in...\textunderscore  + \textunderscore resolução\textunderscore )}
\end{itemize}
Qualidade de irresoluto; indecisão.
\section{Irresolutamente}
\begin{itemize}
\item {Grp. gram.:adv.}
\end{itemize}
De modo irresoluto; com hesitação.
\section{Irresoluto}
\begin{itemize}
\item {Grp. gram.:adj.}
\end{itemize}
\begin{itemize}
\item {Proveniência:(Lat. \textunderscore irresolutus\textunderscore )}
\end{itemize}
Não resoluto; hesitante.
Que ainda não foi resolvido: \textunderscore negócio irresoluto\textunderscore .
\section{Irresolúvel}
\begin{itemize}
\item {Grp. gram.:adj.}
\end{itemize}
\begin{itemize}
\item {Proveniência:(Lat. \textunderscore irresolubilis\textunderscore )}
\end{itemize}
Não resolúvel; insolúvel.
\section{Irresolvível}
\begin{itemize}
\item {Grp. gram.:adj.}
\end{itemize}
O mesmo que \textunderscore irresolúvel\textunderscore .
\section{Irrespeitável}
\begin{itemize}
\item {Grp. gram.:adj.}
\end{itemize}
\begin{itemize}
\item {Proveniência:(De \textunderscore in...\textunderscore  + \textunderscore respeitável\textunderscore )}
\end{itemize}
Que não é digno de respeito.
\section{Irrespeitosamente}
\begin{itemize}
\item {Grp. gram.:adv.}
\end{itemize}
De modo irrespeito.
\section{Irrespeito}
\begin{itemize}
\item {Grp. gram.:m.}
\end{itemize}
\begin{itemize}
\item {Grp. gram.:adj.}
\end{itemize}
\begin{itemize}
\item {Proveniência:(De \textunderscore in...\textunderscore  + \textunderscore respeito\textunderscore )}
\end{itemize}
Falta de respeito; desacatamento.
Que não observa respeito.
Irreverente. Cf. B. Moreno, \textunderscore Com. do Campo\textunderscore , II, 200.
\section{Irrespirabilidade}
\begin{itemize}
\item {Grp. gram.:f.}
\end{itemize}
Qualidade de irrespirável; difficuldade de respirar.
\section{Irrespirável}
\begin{itemize}
\item {Grp. gram.:adj.}
\end{itemize}
\begin{itemize}
\item {Proveniência:(Lat. \textunderscore irrespirabilis\textunderscore )}
\end{itemize}
Que não é respirável:«\textunderscore o coração humano é irrespirável como uma sentina.\textunderscore »Camillo, \textunderscore Mulher Fatal\textunderscore , 120.
\section{Irrespondível}
\begin{itemize}
\item {Grp. gram.:adj.}
\end{itemize}
\begin{itemize}
\item {Proveniência:(De \textunderscore in...\textunderscore  + \textunderscore respondível\textunderscore )}
\end{itemize}
Que não é respondível; irreplicável.
\section{Irresponsabilidade}
\begin{itemize}
\item {Grp. gram.:f.}
\end{itemize}
Qualidade de irresponsável; falta de responsabilidade.
\section{Irresponsável}
\begin{itemize}
\item {Grp. gram.:adj.}
\end{itemize}
\begin{itemize}
\item {Proveniência:(De \textunderscore in...\textunderscore  + \textunderscore responsável\textunderscore )}
\end{itemize}
Que não é responsável.
\section{Irresponsavelmente}
\begin{itemize}
\item {Grp. gram.:adv.}
\end{itemize}
De modo irresponsável.
\section{Irrestricto}
\begin{itemize}
\item {Grp. gram.:adj.}
\end{itemize}
\begin{itemize}
\item {Proveniência:(De \textunderscore in...\textunderscore  + \textunderscore restricto\textunderscore )}
\end{itemize}
Que não é restricto; illimitado.
\section{Irrestringível}
\begin{itemize}
\item {Grp. gram.:adj.}
\end{itemize}
\begin{itemize}
\item {Proveniência:(De \textunderscore in...\textunderscore  + \textunderscore restringível\textunderscore )}
\end{itemize}
Que não é restringível.
\section{Irrestrito}
\begin{itemize}
\item {Grp. gram.:adj.}
\end{itemize}
\begin{itemize}
\item {Proveniência:(De \textunderscore in...\textunderscore  + \textunderscore restrito\textunderscore )}
\end{itemize}
Que não é restrito; ilimitado.
\section{Irretorquível}
\begin{itemize}
\item {Grp. gram.:adj.}
\end{itemize}
\begin{itemize}
\item {Proveniência:(De \textunderscore in...\textunderscore  + \textunderscore retorquível\textunderscore )}
\end{itemize}
A que se não póde retorquir; irrespondível.
\section{Irretractável}
\begin{itemize}
\item {Grp. gram.:adj.}
\end{itemize}
\begin{itemize}
\item {Proveniência:(Lat. \textunderscore irretratabilis\textunderscore )}
\end{itemize}
Que não é retractável; irrevogável; immutável.
\section{Irretractavelmente}
\begin{itemize}
\item {Grp. gram.:adv.}
\end{itemize}
De modo irretractável.
\section{Irretratável}
\begin{itemize}
\item {Grp. gram.:adj.}
\end{itemize}
\begin{itemize}
\item {Proveniência:(Lat. \textunderscore irretratabilis\textunderscore )}
\end{itemize}
Que não é retratável; irrevogável; imutável.
\section{Irretratavelmente}
\begin{itemize}
\item {Grp. gram.:adv.}
\end{itemize}
De modo irretratável.
\section{Irreverência}
\begin{itemize}
\item {Grp. gram.:f.}
\end{itemize}
\begin{itemize}
\item {Proveniência:(Lat. \textunderscore irreverentia\textunderscore )}
\end{itemize}
Falta de reverência.
Acto irreverente.
Qualidade de irreverente.
\section{Irreverenciar}
\begin{itemize}
\item {Grp. gram.:v. t.}
\end{itemize}
Commeter irreverencia contra. Cf. Pant. de Aveiro, \textunderscore Itiner.\textunderscore , 165, (2.^a ed.).
\section{Irreverenciosamente}
\begin{itemize}
\item {Grp. gram.:adv.}
\end{itemize}
De modo irreverencioso; com desacato.
\section{Irreverencioso}
\begin{itemize}
\item {Grp. gram.:adj.}
\end{itemize}
\begin{itemize}
\item {Proveniência:(De \textunderscore in...\textunderscore  + \textunderscore reverencioso\textunderscore )}
\end{itemize}
Não reverencioso.
Incivil.
\section{Irreverente}
\begin{itemize}
\item {Grp. gram.:adj.}
\end{itemize}
\begin{itemize}
\item {Proveniência:(Lat. \textunderscore irreverens\textunderscore )}
\end{itemize}
O mesmo que \textunderscore irreverencioso\textunderscore .
\section{Irreverentemente}
\begin{itemize}
\item {Grp. gram.:adv.}
\end{itemize}
De modo irreverente.
\section{Irrevocabilidade}
\begin{itemize}
\item {Grp. gram.:f.}
\end{itemize}
O mesmo que \textunderscore irrevogabilidade\textunderscore .
\section{Irrevocável}
\begin{itemize}
\item {Grp. gram.:adj.}
\end{itemize}
O mesmo que \textunderscore irrevogável\textunderscore . Cf. \textunderscore Luz e Calor\textunderscore , 427.
\section{Irrevogabilidade}
\begin{itemize}
\item {Grp. gram.:f.}
\end{itemize}
Qualidade de irrevogável.
\section{Irrevogável}
\begin{itemize}
\item {Grp. gram.:adj.}
\end{itemize}
\begin{itemize}
\item {Proveniência:(Do lat. \textunderscore irrevocabilis\textunderscore )}
\end{itemize}
Não revogável; que se não póde annullar: \textunderscore decisão irrevogável\textunderscore .
\section{Irrevogavelmente}
\begin{itemize}
\item {Grp. gram.:adv.}
\end{itemize}
De modo irrevogável.
\section{Irriariadan}
\begin{itemize}
\item {Grp. gram.:m.}
\end{itemize}
Árvore da Guiana inglesa.
\section{Írribus!}
\begin{itemize}
\item {Grp. gram.:interj.}
\end{itemize}
\begin{itemize}
\item {Utilização:Chul.}
\end{itemize}
O mesmo que \textunderscore irrório!\textunderscore .
\section{Irrigação}
\begin{itemize}
\item {Grp. gram.:f.}
\end{itemize}
\begin{itemize}
\item {Proveniência:(Lat. \textunderscore irrigatio\textunderscore )}
\end{itemize}
Acto de irrigar; rega.
\section{Irrigador}
\begin{itemize}
\item {Grp. gram.:adj.}
\end{itemize}
\begin{itemize}
\item {Grp. gram.:M.}
\end{itemize}
\begin{itemize}
\item {Proveniência:(Lat. \textunderscore irrigator\textunderscore )}
\end{itemize}
Que irriga.
Vaso para regar; regador.
Instrumento para irrigações medicinaes.
\section{Irrigar}
\begin{itemize}
\item {Grp. gram.:v. t.}
\end{itemize}
\begin{itemize}
\item {Proveniência:(Lat. \textunderscore irrigare\textunderscore )}
\end{itemize}
Dirigir regos de água para.
Regar.
Banhar.
\section{Irrigatório}
\begin{itemize}
\item {Grp. gram.:adj.}
\end{itemize}
Próprio para irrigar.
\section{Irrigável}
\begin{itemize}
\item {Grp. gram.:adj.}
\end{itemize}
Que se póde irrigar.
\section{Irríguo}
\begin{itemize}
\item {Grp. gram.:adj.}
\end{itemize}
\begin{itemize}
\item {Utilização:Hydrogr.}
\end{itemize}
\begin{itemize}
\item {Proveniência:(Lat. \textunderscore irriguus\textunderscore )}
\end{itemize}
Que é banhado por águas ou atravessado por correntes.
\section{Irrisão}
\begin{itemize}
\item {Grp. gram.:f.}
\end{itemize}
\begin{itemize}
\item {Proveniência:(Lat. \textunderscore irrisio\textunderscore )}
\end{itemize}
Acto de zombar.
Escárneo.
Objecto do escárneo.
\section{Irrisor}
\begin{itemize}
\item {Grp. gram.:m.  e  adj.}
\end{itemize}
\begin{itemize}
\item {Proveniência:(Lat. \textunderscore irrisor\textunderscore )}
\end{itemize}
O que escarnece.
\section{Irrisoriamente}
\begin{itemize}
\item {Grp. gram.:adv.}
\end{itemize}
De modo irrisório; com escárneo.
\section{Irrisório}
\begin{itemize}
\item {Grp. gram.:adj.}
\end{itemize}
\begin{itemize}
\item {Proveniência:(Lat. \textunderscore irrisorius\textunderscore )}
\end{itemize}
Que envolve irrisão.
Que provoca riso ou motejo: \textunderscore traje irrisório\textunderscore .
\section{Irritabilidade}
\begin{itemize}
\item {Grp. gram.:f.}
\end{itemize}
\begin{itemize}
\item {Proveniência:(Lat. \textunderscore irritabilitas\textunderscore )}
\end{itemize}
Qualidade de irritável.
\section{Irritação}
\begin{itemize}
\item {Grp. gram.:f.}
\end{itemize}
\begin{itemize}
\item {Proveniência:(Lat. \textunderscore irritatio\textunderscore )}
\end{itemize}
Acto ou effeito de irritar.
Exacerbação; excitação.
\section{Irritadiço}
\begin{itemize}
\item {Grp. gram.:adj.}
\end{itemize}
Que se irrita facilmente.
\section{Irritado}
\begin{itemize}
\item {Grp. gram.:adj.}
\end{itemize}
\begin{itemize}
\item {Proveniência:(Lat. \textunderscore irritatus\textunderscore )}
\end{itemize}
Que se irrita; irado.
Excitado.
\section{Irritador}
\begin{itemize}
\item {Grp. gram.:adj.}
\end{itemize}
\begin{itemize}
\item {Grp. gram.:M.}
\end{itemize}
\begin{itemize}
\item {Proveniência:(Lat. \textunderscore irritator\textunderscore )}
\end{itemize}
Que irrita; irritante.
Aquelle que irrita.
\section{Irritamente}
\begin{itemize}
\item {Grp. gram.:adv.}
\end{itemize}
De modo irrito; sem validade; debalde.
\section{Irritamento}
\begin{itemize}
\item {Grp. gram.:m.}
\end{itemize}
\begin{itemize}
\item {Proveniência:(Lat. \textunderscore irritamentum\textunderscore )}
\end{itemize}
O mesmo que \textunderscore irritação\textunderscore .
\section{Irritante}
\begin{itemize}
\item {Grp. gram.:adj.}
\end{itemize}
\begin{itemize}
\item {Grp. gram.:M.}
\end{itemize}
\begin{itemize}
\item {Proveniência:(Lat. \textunderscore irritans\textunderscore )}
\end{itemize}
Que causa irritação; excitante.
Substância estimulante.
\section{Irritar}
\begin{itemize}
\item {Grp. gram.:v. t.}
\end{itemize}
\begin{itemize}
\item {Proveniência:(Lat. \textunderscore irritare\textunderscore )}
\end{itemize}
Tornar irado.
Provocar á ira; encolerizar.
Excitar, estimular; exacerbar.
\section{Irritar}
\begin{itemize}
\item {Grp. gram.:v. t.}
\end{itemize}
\begin{itemize}
\item {Utilização:Des.}
\end{itemize}
\begin{itemize}
\item {Proveniência:(De \textunderscore írrito\textunderscore )}
\end{itemize}
Tornar nullo; tirar a fôrça ou valor a (documentos, títulos, etc.).
\section{Irritativo}
\begin{itemize}
\item {Grp. gram.:adj.}
\end{itemize}
O mesmo que \textunderscore irritante\textunderscore .
\section{Irritável}
\begin{itemize}
\item {Grp. gram.:adj.}
\end{itemize}
\begin{itemize}
\item {Proveniência:(Lat. \textunderscore irritabilis\textunderscore )}
\end{itemize}
O mesmo que \textunderscore irascível\textunderscore .
\section{Írrito}
\begin{itemize}
\item {Grp. gram.:adj.}
\end{itemize}
\begin{itemize}
\item {Proveniência:(Lat. \textunderscore irritus\textunderscore )}
\end{itemize}
Que não teve effeito; inútil; vão.
\section{Irrivalizável}
\begin{itemize}
\item {Grp. gram.:adj.}
\end{itemize}
\begin{itemize}
\item {Proveniência:(De \textunderscore in...\textunderscore  + \textunderscore rivalizável\textunderscore )}
\end{itemize}
Que não póde têr rival; inigualável:«\textunderscore Martins Sarmento... archeólogo irrivalizável\textunderscore ». Camillo, \textunderscore Echos Humorísticos\textunderscore , 6.
\section{Irrivalizavelmente}
\begin{itemize}
\item {Grp. gram.:adv.}
\end{itemize}
De modo irrivalizável.
\section{Irrogação}
\begin{itemize}
\item {Grp. gram.:f.}
\end{itemize}
\begin{itemize}
\item {Proveniência:(Lat. \textunderscore irrogatio\textunderscore )}
\end{itemize}
Acto ou effeito de irrogar.
\section{Irrogar}
\begin{itemize}
\item {Grp. gram.:v. t.}
\end{itemize}
\begin{itemize}
\item {Proveniência:(Lat. \textunderscore irrogare\textunderscore )}
\end{itemize}
Impor a alguém.
Infligir.
Attribuír, fazer recaír sôbre alguém: \textunderscore irrogar responsabilidades\textunderscore .
\section{Irromper}
\begin{itemize}
\item {Grp. gram.:v. t.}
\end{itemize}
\begin{itemize}
\item {Proveniência:(Lat. \textunderscore irrumpere\textunderscore )}
\end{itemize}
Entrar impetuosamente.
Arrojar-se.
Brotar; surgir: \textunderscore a água irrompe do solo\textunderscore .
\section{Irrónia}
\begin{itemize}
\item {Grp. gram.:f.}
\end{itemize}
\begin{itemize}
\item {Utilização:Prov.}
\end{itemize}
Má índole, mau carácter.
\section{Irroração}
\begin{itemize}
\item {Grp. gram.:f.}
\end{itemize}
\begin{itemize}
\item {Proveniência:(Lat. \textunderscore irroratio\textunderscore )}
\end{itemize}
Acto ou effeito de irrorar.
\section{Irrorar}
\begin{itemize}
\item {Grp. gram.:v. t.}
\end{itemize}
\begin{itemize}
\item {Proveniência:(Lat. \textunderscore irrorare\textunderscore )}
\end{itemize}
Aspergir com orvalho; borrifar; humedecer.
\section{Irrório!}
\begin{itemize}
\item {Grp. gram.:interj.}
\end{itemize}
\begin{itemize}
\item {Utilização:Chul.}
\end{itemize}
O mesmo que \textunderscore irra!\textunderscore . Cf. Filinto, V, 165; XIII, 55 e 261; XIX, 246.
\section{Irrupção}
\begin{itemize}
\item {Grp. gram.:f.}
\end{itemize}
\begin{itemize}
\item {Proveniência:(Lat. \textunderscore írruptio\textunderscore )}
\end{itemize}
Acto ou effeito de irromper.
\section{Irucurana}
\begin{itemize}
\item {Grp. gram.:f.}
\end{itemize}
O mesmo que \textunderscore airi\textunderscore .
\section{Isabel}
\begin{itemize}
\item {Grp. gram.:adj.}
\end{itemize}
\begin{itemize}
\item {Grp. gram.:F.}
\end{itemize}
Que tem a côr de entre branco e amarelo, (falando-se do cavallo).
Variedade de videira açoreana, \textunderscore ou\textunderscore , antes, americana.
\section{Isabelista}
\begin{itemize}
\item {Grp. gram.:m.}
\end{itemize}
Partidário da raínha Isabel, em Espanha.
\section{Isadelfia}
\begin{itemize}
\item {Grp. gram.:f.}
\end{itemize}
Qualidade de isadelfo.
\section{Isadelfo}
\begin{itemize}
\item {Grp. gram.:adj.}
\end{itemize}
\begin{itemize}
\item {Proveniência:(Do gr. \textunderscore isos\textunderscore  + \textunderscore adelphos\textunderscore )}
\end{itemize}
Que tem dois corpos iguaes, perfeitamente desenvolvidos, e ligados entre si por partes pouco importantes, (falando-se de monstros duplos, ou de plantas, cujos estames estão reunidos em dois feixes iguaes).
\section{Isadelphia}
\begin{itemize}
\item {Grp. gram.:f.}
\end{itemize}
Qualidade de isadelpho.
\section{Isadelpho}
\begin{itemize}
\item {Grp. gram.:adj.}
\end{itemize}
\begin{itemize}
\item {Proveniência:(Do gr. \textunderscore isos\textunderscore  + \textunderscore adelphos\textunderscore )}
\end{itemize}
Que tem dois corpos iguaes, perfeitamente desenvolvidos, e ligados entre si por partes pouco importantes, (falando-se de monstros duplos, ou de plantas, cujos estames estão reunidos em dois feixes iguaes).
\section{Isagoge}
\begin{itemize}
\item {Grp. gram.:f.}
\end{itemize}
\begin{itemize}
\item {Utilização:P. us.}
\end{itemize}
\begin{itemize}
\item {Proveniência:(Do gr. \textunderscore isos\textunderscore  + \textunderscore goge\textunderscore )}
\end{itemize}
Proêmio.
Introducção.
Noções rudimentares.
\section{Isagógico}
\begin{itemize}
\item {Grp. gram.:adj.}
\end{itemize}
Relativo á isagoge.
\section{Isantho}
\begin{itemize}
\item {Grp. gram.:m.}
\end{itemize}
\begin{itemize}
\item {Proveniência:(Do gr. \textunderscore isos\textunderscore  + \textunderscore anthos\textunderscore )}
\end{itemize}
Planta labiada da América do Norte.
\section{Isantina}
\begin{itemize}
\item {Grp. gram.:f.}
\end{itemize}
Substância, extrahida do anil.
\section{Isanto}
\begin{itemize}
\item {Grp. gram.:m.}
\end{itemize}
\begin{itemize}
\item {Proveniência:(Do gr. \textunderscore isos\textunderscore  + \textunderscore anthos\textunderscore )}
\end{itemize}
Planta labiada da América do Norte.
\section{Isa-quente}
\begin{itemize}
\item {Grp. gram.:m.}
\end{itemize}
Árvore africana, de frutos comestíveis e sementes oleosas.
\section{Isatídeas}
\begin{itemize}
\item {Grp. gram.:f. pl.}
\end{itemize}
Tríbo de plantas crucíferas, que tem por typo a isátis.
\section{Isátis}
\begin{itemize}
\item {Grp. gram.:f.}
\end{itemize}
\begin{itemize}
\item {Proveniência:(Do gr. \textunderscore isadro\textunderscore )}
\end{itemize}
Planta crucífera, (\textunderscore isatis tinctoria\textunderscore ).
\section{Isaura}
\begin{itemize}
\item {Grp. gram.:f.}
\end{itemize}
\begin{itemize}
\item {Proveniência:(De \textunderscore Isaura\textunderscore , n. p.)}
\end{itemize}
Gênero de polypeiros, cuja espécie týpica habita no Egypto.
\section{Isca}
\begin{itemize}
\item {Grp. gram.:f.}
\end{itemize}
\begin{itemize}
\item {Utilização:Pop.}
\end{itemize}
\begin{itemize}
\item {Utilização:Pop.}
\end{itemize}
\begin{itemize}
\item {Utilização:Prov.}
\end{itemize}
\begin{itemize}
\item {Utilização:T. da Bairrada}
\end{itemize}
\begin{itemize}
\item {Utilização:Fig.}
\end{itemize}
\begin{itemize}
\item {Grp. gram.:Interj.}
\end{itemize}
\begin{itemize}
\item {Utilização:bras}
\end{itemize}
\begin{itemize}
\item {Proveniência:(Do lat. \textunderscore esca\textunderscore )}
\end{itemize}
Qualquer substância, que se põe no anzol, para attrahir e pescar peixes.
Combustível, que recebe as faíscas do fuzil, e com o qual se communica fogo.
Tira de fígado, temperada com vinagre, alho, etc, e frita em banha de porco.
Pequena porção, biscato.
Pedaço de febra de bacalhau.
Espécie de cardo, de que se tira a substância combustível, que no Alentejo se chama bugalho.
Chamariz, engôdo, negaça.
Voz, com que se estimulam os cães.
\section{Iscaço}
\begin{itemize}
\item {Grp. gram.:m.}
\end{itemize}
\begin{itemize}
\item {Utilização:T. de Espinho}
\end{itemize}
\begin{itemize}
\item {Proveniência:(De \textunderscore isca\textunderscore )}
\end{itemize}
Estrume de cabeças de sardinha e de outros peixes.
\section{Iscar}
\begin{itemize}
\item {Grp. gram.:v. t.}
\end{itemize}
\begin{itemize}
\item {Utilização:Fig.}
\end{itemize}
\begin{itemize}
\item {Utilização:Bras}
\end{itemize}
\begin{itemize}
\item {Proveniência:(Do lat. \textunderscore escare\textunderscore )}
\end{itemize}
Pôr isca em: \textunderscore iscar o anzol\textunderscore .
Untar.
Engodar.
Contaminar.
O mesmo que \textunderscore estumar\textunderscore .
\section{Iscariotista}
\begin{itemize}
\item {Grp. gram.:m.}
\end{itemize}
\begin{itemize}
\item {Proveniência:(De \textunderscore Iscariote\textunderscore , cognome de Judas)}
\end{itemize}
Membro de uma seita, que venerava Judas Iscariote, Caím e outras más personagens da \textunderscore Bíblia\textunderscore .
\section{Ischemia}
\begin{itemize}
\item {fónica:que}
\end{itemize}
\begin{itemize}
\item {Grp. gram.:f.}
\end{itemize}
\begin{itemize}
\item {Utilização:Med.}
\end{itemize}
\begin{itemize}
\item {Proveniência:(Do gr. \textunderscore iskhein\textunderscore  + \textunderscore haima\textunderscore )}
\end{itemize}
Suspensão da circulação do sangue.
\section{Ischêmico}
\begin{itemize}
\item {fónica:quê}
\end{itemize}
\begin{itemize}
\item {Grp. gram.:adj.}
\end{itemize}
Relativo á ischêmia.
Que susta o movimento do sangue nos vasos orgânicos.
\section{Ischiadelphos}
\begin{itemize}
\item {fónica:qui}
\end{itemize}
\begin{itemize}
\item {Grp. gram.:m.  e  adj. pl.}
\end{itemize}
\begin{itemize}
\item {Proveniência:(Do gr. \textunderscore iskhion\textunderscore  + \textunderscore adelphos\textunderscore )}
\end{itemize}
Monstros duplos, cujos corpos, oppostos um ao outro, estão ligados pela bacia.
\section{Ischiádico}
\begin{itemize}
\item {fónica:qui}
\end{itemize}
\begin{itemize}
\item {Grp. gram.:adj.}
\end{itemize}
O mesmo que \textunderscore ischiático\textunderscore . Cf. C. Guerreiro, \textunderscore Versif. Port.\textunderscore , 434.
\section{Ischiagra}
\begin{itemize}
\item {fónica:qui}
\end{itemize}
\begin{itemize}
\item {Grp. gram.:f.}
\end{itemize}
\begin{itemize}
\item {Utilização:Med.}
\end{itemize}
\begin{itemize}
\item {Proveniência:(Do gr. \textunderscore iskhion\textunderscore  + \textunderscore agra\textunderscore )}
\end{itemize}
Dôr fixa nos quadris.
Dôr sciática.
\section{Ischial}
\begin{itemize}
\item {fónica:qui}
\end{itemize}
\begin{itemize}
\item {Grp. gram.:adj.}
\end{itemize}
Relativo ao íschion.
\section{Ischiático}
\begin{itemize}
\item {fónica:qui}
\end{itemize}
\begin{itemize}
\item {Grp. gram.:adj.}
\end{itemize}
Relativo ao íschion; sciático: \textunderscore dôr ischiática\textunderscore .
\section{Íschio}
\begin{itemize}
\item {fónica:qui}
\end{itemize}
\begin{itemize}
\item {Grp. gram.:m.}
\end{itemize}
O mesmo ou melhor que \textunderscore íschion\textunderscore .
\section{Ischio-anal}
\begin{itemize}
\item {Grp. gram.:adj.}
\end{itemize}
Relativo ao íschion e ao ânus.
\section{Ischiocele}
\begin{itemize}
\item {fónica:qui}
\end{itemize}
\begin{itemize}
\item {Grp. gram.:f.}
\end{itemize}
\begin{itemize}
\item {Utilização:Med.}
\end{itemize}
\begin{itemize}
\item {Proveniência:(Do gr. \textunderscore iskhion\textunderscore  + \textunderscore kele\textunderscore )}
\end{itemize}
Hérnia, produzida através da chanfradura ischiática.
\section{Ischio-femoral}
\begin{itemize}
\item {fónica:qui}
\end{itemize}
\begin{itemize}
\item {Grp. gram.:adj.}
\end{itemize}
Relativo ao íschion e ao fêmur.
\section{Íschion}
\begin{itemize}
\item {fónica:qui}
\end{itemize}
\begin{itemize}
\item {Grp. gram.:m.}
\end{itemize}
\begin{itemize}
\item {Utilização:Anat.}
\end{itemize}
\begin{itemize}
\item {Proveniência:(Gr. \textunderscore iskhion\textunderscore )}
\end{itemize}
Uma das três partes do osso ilíaco, em que se articula o osso da coxa.
Quadril.
\section{Ischiopagia}
\begin{itemize}
\item {fónica:qui}
\end{itemize}
\begin{itemize}
\item {Grp. gram.:f.}
\end{itemize}
Estado de ischiópagos.
\section{Ischiópagos}
\begin{itemize}
\item {fónica:qui}
\end{itemize}
\begin{itemize}
\item {Grp. gram.:m.  e  adj. pl.}
\end{itemize}
\begin{itemize}
\item {Proveniência:(Do gr. \textunderscore iskhion\textunderscore  + \textunderscore pagein\textunderscore )}
\end{itemize}
Diz-se dos monstros, compostos de dois indivíduos, reunidos pela região hypogástrica, e tendo um umbigo commum.
\section{Íschio-perinal}
\begin{itemize}
\item {Grp. gram.:adj.}
\end{itemize}
Relativo ao íschion e ao perinéu.
\section{Íschio-tibial}
\begin{itemize}
\item {Grp. gram.:adj.}
\end{itemize}
Relativo ao íschion e á tíbia.
\section{Ischnophonia}
\begin{itemize}
\item {Grp. gram.:f.}
\end{itemize}
\begin{itemize}
\item {Proveniência:(Gr. \textunderscore ískhnophonia\textunderscore )}
\end{itemize}
Fraqueza da voz.
\section{Ischurético}
\begin{itemize}
\item {fónica:cu}
\end{itemize}
\begin{itemize}
\item {Grp. gram.:adj.}
\end{itemize}
\begin{itemize}
\item {Proveniência:(Do rad. de \textunderscore eschuria\textunderscore )}
\end{itemize}
Próprio para a cura da ischuria.
\section{Ischuria}
\begin{itemize}
\item {fónica:cu}
\end{itemize}
\begin{itemize}
\item {Grp. gram.:f.}
\end{itemize}
\begin{itemize}
\item {Proveniência:(Gr. \textunderscore iskhouria\textunderscore )}
\end{itemize}
Suppressão ou retenção da urina.
\section{Iscnofonia}
\begin{itemize}
\item {Grp. gram.:f.}
\end{itemize}
\begin{itemize}
\item {Proveniência:(Gr. \textunderscore ískhnophonia\textunderscore )}
\end{itemize}
Fraqueza da voz.
\section{Isco}
\begin{itemize}
\item {Grp. gram.:m.}
\end{itemize}
\begin{itemize}
\item {Utilização:Pop.}
\end{itemize}
\begin{itemize}
\item {Proveniência:(Do rad. de \textunderscore isca\textunderscore )}
\end{itemize}
Porção de massa, que se separa da massa geral de uma fornada, e que se deixa continuar a fermentar, para os trabalhos da panificação; fermento.
Isca.
\section{Iscurético}
\begin{itemize}
\item {Grp. gram.:adj.}
\end{itemize}
\begin{itemize}
\item {Proveniência:(Do rad. de \textunderscore eschuria\textunderscore )}
\end{itemize}
Próprio para a cura da ischuria.
\section{Iscuria}
\begin{itemize}
\item {Grp. gram.:f.}
\end{itemize}
\begin{itemize}
\item {Proveniência:(Gr. \textunderscore iskhouria\textunderscore )}
\end{itemize}
Suppressão ou retenção da urina.
\section{Isenção}
\begin{itemize}
\item {Grp. gram.:f.}
\end{itemize}
\begin{itemize}
\item {Utilização:Fig.}
\end{itemize}
\begin{itemize}
\item {Proveniência:(Lat. \textunderscore exemptio\textunderscore )}
\end{itemize}
Acto de eximir.
Nobreza de carácter.
Abnegação; imparcialidade.
Acto de esquivar-se.
Altivez.
\section{Isentamente}
\begin{itemize}
\item {Grp. gram.:adv.}
\end{itemize}
\begin{itemize}
\item {Proveniência:(De \textunderscore isento\textunderscore )}
\end{itemize}
Desinteressadamente; com esquivança.
\section{Isentar}
\begin{itemize}
\item {Grp. gram.:v. t.}
\end{itemize}
Tornar isento.
Exceptuar.
Privilegiar.
Dispensar.
\section{Isentidão}
\begin{itemize}
\item {Grp. gram.:f.}
\end{itemize}
\begin{itemize}
\item {Utilização:Ant.}
\end{itemize}
\begin{itemize}
\item {Proveniência:(De \textunderscore isento\textunderscore )}
\end{itemize}
O mesmo que \textunderscore isenção\textunderscore . Cf. Heitor Pinto.
\section{Isento}
\begin{itemize}
\item {Grp. gram.:adj.}
\end{itemize}
\begin{itemize}
\item {Utilização:Fig.}
\end{itemize}
\begin{itemize}
\item {Grp. gram.:M.}
\end{itemize}
\begin{itemize}
\item {Proveniência:(Do lat. \textunderscore exemptus\textunderscore )}
\end{itemize}
Desobrigado; dispensado: \textunderscore isento de contribuições\textunderscore .
Privilegiado.
Que tem esquivança, que se não mostra sensível a finezas ou galanteios.
Que é imparcial, a despeito dos seus interesses; incorruptível.
Lugar, que não estava sujeito ás jurisdicções ordinárias: \textunderscore o mosteiro de Santa-Cruz era um isento, dentro do bispado de Coimbra\textunderscore .
\section{Isíaco}
\begin{itemize}
\item {Grp. gram.:adj.}
\end{itemize}
Relativo á deusa egýpcia Ísis ou aos monumentos que a ella se referem.
\section{Isidora}
\begin{itemize}
\item {Grp. gram.:f.}
\end{itemize}
Gênero de plantas rubiáceas.
\section{Ísis}
\begin{itemize}
\item {Grp. gram.:f.}
\end{itemize}
Pequeno planeta, descoberto em 1856.
\section{Isitérias}
\begin{itemize}
\item {Grp. gram.:f. pl.}
\end{itemize}
Antigas festas gregas, no primeiro mês do anno áttico, para se celebrar a entrada dos novos magistrados no exercício das suas funcções. Cf. Castilho, \textunderscore Fastos\textunderscore , I, 538.
\section{Islamismo}
\begin{itemize}
\item {Grp. gram.:m.}
\end{itemize}
\begin{itemize}
\item {Proveniência:(De \textunderscore islão\textunderscore )}
\end{itemize}
Religião muçulmana.
Os Muçulmanos.
\section{Islamita}
\begin{itemize}
\item {Grp. gram.:m.  e  f.}
\end{itemize}
\begin{itemize}
\item {Proveniência:(De \textunderscore islão\textunderscore )}
\end{itemize}
Pessôa, que segue o islamismo.
\section{Islamítico}
\begin{itemize}
\item {Grp. gram.:adj.}
\end{itemize}
Relativo aos islamitas. Cf. Herculano, \textunderscore Hist. de Port.\textunderscore , III, 205.
\section{Islandês}
\begin{itemize}
\item {Grp. gram.:adj.}
\end{itemize}
\begin{itemize}
\item {Grp. gram.:M.}
\end{itemize}
Relativo á Islândia.
Aquelle que é natural da Islândia.
Idioma, falado na Islândia e irmão do suéco e do dinamarquês.
\section{Islão}
\begin{itemize}
\item {Grp. gram.:m.}
\end{itemize}
\begin{itemize}
\item {Proveniência:(Do ár. \textunderscore islam\textunderscore , submissão á vontade divina)}
\end{itemize}
Religião dos Mahometanos.
Conjunto dos países muçulmanos.
\section{Islenho}
\begin{itemize}
\item {Grp. gram.:m.  e  adj.}
\end{itemize}
\begin{itemize}
\item {Utilização:Ant.}
\end{itemize}
O mesmo que \textunderscore insulano\textunderscore .
(Cast. \textunderscore isleño\textunderscore , de \textunderscore isla\textunderscore , ilha)
\section{Isleno}
\begin{itemize}
\item {Grp. gram.:m.  e  adj.}
\end{itemize}
\begin{itemize}
\item {Utilização:Ant.}
\end{itemize}
O mesmo que \textunderscore islenho\textunderscore .
\section{Islim}
\begin{itemize}
\item {Grp. gram.:m.}
\end{itemize}
\begin{itemize}
\item {Utilização:Ant.}
\end{itemize}
O mesmo que \textunderscore islão\textunderscore :«\textunderscore Livros dislim, eu m'leeo\textunderscore ». \textunderscore Cancioneiro da Vaticana\textunderscore .
\section{Ismaeliano}
\begin{itemize}
\item {fónica:ma-e}
\end{itemize}
\begin{itemize}
\item {Grp. gram.:m.}
\end{itemize}
Sectário do ismaelismo.
\section{Ismaelismo}
\begin{itemize}
\item {fónica:ma-e}
\end{itemize}
\begin{itemize}
\item {Grp. gram.:m.}
\end{itemize}
Seita muçulmana, que se formou no século VIII.
\section{Ismaelitas}
\begin{itemize}
\item {fónica:ma-e}
\end{itemize}
\begin{itemize}
\item {Grp. gram.:m. pl.}
\end{itemize}
Um dos nomes, com que se designam os Árabes, como descendentes de Ismael, filho de Abrahão.
\section{Ismaelítico}
\begin{itemize}
\item {fónica:ma-e}
\end{itemize}
\begin{itemize}
\item {Grp. gram.:adj.}
\end{itemize}
Relativo aos ismaelitas. Cf. Herculano, \textunderscore Opúsc.\textunderscore , III, 274.
\section{Ismênia}
\begin{itemize}
\item {Grp. gram.:f.}
\end{itemize}
Planta brasileira dos jardins.
\section{Isnáchia}
\begin{itemize}
\item {fónica:qui}
\end{itemize}
\begin{itemize}
\item {Grp. gram.:f.}
\end{itemize}
\begin{itemize}
\item {Proveniência:(Do germ. \textunderscore nachen\textunderscore )}
\end{itemize}
Antiga embarcação, do fundo largo, destinada especialmente ao transporte de vehículos, entre alguns povos do Norte. Cf. Herculano, \textunderscore Tomada de Silves\textunderscore , nos \textunderscore Quadros Hist.\textunderscore  de Castilho.
\section{Isnáquia}
\begin{itemize}
\item {Grp. gram.:f.}
\end{itemize}
\begin{itemize}
\item {Proveniência:(Do germ. \textunderscore nachen\textunderscore )}
\end{itemize}
Antiga embarcação, do fundo largo, destinada especialmente ao transporte de veículos, entre alguns povos do Norte. Cf. Herculano, \textunderscore Tomada de Silves\textunderscore , nos \textunderscore Quadros Hist.\textunderscore  de Castilho.
\section{Isnárdia}
\begin{itemize}
\item {Grp. gram.:f.}
\end{itemize}
\begin{itemize}
\item {Proveniência:(De \textunderscore Isnard\textunderscore , n. p.)}
\end{itemize}
Gênero de plantas onothéreas.
\section{Iso...}
\begin{itemize}
\item {Grp. gram.:pref.}
\end{itemize}
\begin{itemize}
\item {Proveniência:(Do gr. \textunderscore isos\textunderscore )}
\end{itemize}
(designativo de \textunderscore igualdade\textunderscore )
\section{Isoáxico}
\begin{itemize}
\item {Grp. gram.:adj.}
\end{itemize}
\begin{itemize}
\item {Utilização:Geol.}
\end{itemize}
\begin{itemize}
\item {Proveniência:(Do gr. \textunderscore isos\textunderscore  + lat. \textunderscore axis\textunderscore )}
\end{itemize}
Diz-se dos crystaes, que têm iguaes eixos.
\section{Isóbafia}
\begin{itemize}
\item {Grp. gram.:f.}
\end{itemize}
\begin{itemize}
\item {Proveniência:(Do gr. \textunderscore isos\textunderscore  + \textunderscore baphein\textunderscore )}
\end{itemize}
Estado de um corpo, que só reflecte uma côr.
\section{Isobaphia}
\begin{itemize}
\item {Grp. gram.:f.}
\end{itemize}
\begin{itemize}
\item {Proveniência:(Do gr. \textunderscore isos\textunderscore  + \textunderscore baphein\textunderscore )}
\end{itemize}
Estado de um corpo, que só reflecte uma côr.
\section{Isobárico}
\begin{itemize}
\item {Grp. gram.:adj.}
\end{itemize}
O mesmo que \textunderscore isobarométrico\textunderscore .
\section{Isóbaro}
\begin{itemize}
\item {Grp. gram.:adj.}
\end{itemize}
O mesmo que \textunderscore isobarométrico\textunderscore .
\section{Isobarométrico}
\begin{itemize}
\item {Grp. gram.:adj.}
\end{itemize}
\begin{itemize}
\item {Proveniência:(De \textunderscore iso...\textunderscore  + \textunderscore barométrico\textunderscore )}
\end{itemize}
Que apresenta as mesmas amplitudes barométricas.
\section{Isocarda}
\begin{itemize}
\item {Grp. gram.:f.}
\end{itemize}
\begin{itemize}
\item {Proveniência:(Do gr. \textunderscore isos\textunderscore  + \textunderscore kardia\textunderscore )}
\end{itemize}
Mollusco, de concha muito espêssa e cordiforme.
\section{Isocárpeo}
\begin{itemize}
\item {Grp. gram.:adj.}
\end{itemize}
\begin{itemize}
\item {Utilização:Bot.}
\end{itemize}
\begin{itemize}
\item {Proveniência:(Do gr. \textunderscore isos\textunderscore  + \textunderscore karpos\textunderscore )}
\end{itemize}
Diz-se das plantas, em que as divisões dos frutos são em número igual ao das divisões do periantho.
\section{Isochimênico}
\begin{itemize}
\item {fónica:qui}
\end{itemize}
\begin{itemize}
\item {Grp. gram.:adj.}
\end{itemize}
\begin{itemize}
\item {Proveniência:(Do gr. \textunderscore isos\textunderscore  + \textunderscore kheimainein\textunderscore )}
\end{itemize}
Diz-se, em meteorologia, da linha que passa por todos os pontos do globo, que no inverno têm a mesma temperatura.
\section{Isochímeno}
\begin{itemize}
\item {fónica:qui}
\end{itemize}
\begin{itemize}
\item {Grp. gram.:adj.}
\end{itemize}
\begin{itemize}
\item {Proveniência:(Do gr. \textunderscore isos\textunderscore  + \textunderscore kheimainein\textunderscore )}
\end{itemize}
Diz-se, em meteorologia, da linha que passa por todos os pontos do globo, que no inverno têm a mesma temperatura.
\section{Isóchoro}
\begin{itemize}
\item {fónica:co}
\end{itemize}
\begin{itemize}
\item {Grp. gram.:m.  e  adj.}
\end{itemize}
\begin{itemize}
\item {Proveniência:(Do gr. \textunderscore isos\textunderscore  + \textunderscore khoros\textunderscore )}
\end{itemize}
Dizia-se do verso hexâmetro, quando composto só de espondeus.
\section{Isochristas}
\begin{itemize}
\item {Grp. gram.:m. pl.}
\end{itemize}
\begin{itemize}
\item {Proveniência:(Do gr. \textunderscore isos\textunderscore  + \textunderscore khristos\textunderscore )}
\end{itemize}
Seita religiosa, que sustentava serem os Apóstolos iguaes a Christo.
\section{Isochromático}
\begin{itemize}
\item {Grp. gram.:adj.}
\end{itemize}
Que tem coloração uniforme.
Relativo á isochromia.
\section{Isochromia}
\begin{itemize}
\item {Grp. gram.:f.}
\end{itemize}
\begin{itemize}
\item {Proveniência:(Do gr. \textunderscore isos\textunderscore  + \textunderscore khroma\textunderscore )}
\end{itemize}
O mesmo que \textunderscore lithochromia\textunderscore .
\section{Isochrónico}
\begin{itemize}
\item {Grp. gram.:adj.}
\end{itemize}
O mesmo que \textunderscore isóchrono\textunderscore .
\section{Isochronismo}
\begin{itemize}
\item {Grp. gram.:m.}
\end{itemize}
Qualidade de isóchrono.
\section{Isóchrono}
\begin{itemize}
\item {Grp. gram.:adj.}
\end{itemize}
\begin{itemize}
\item {Proveniência:(Lat. \textunderscore isochronus\textunderscore )}
\end{itemize}
Que se realiza em tempos iguaes, ou ao mesmo tempo.
\section{Isoclinal}
\begin{itemize}
\item {Grp. gram.:adj.}
\end{itemize}
O mesmo que \textunderscore isóclino\textunderscore .
\section{Isóclino}
\begin{itemize}
\item {Grp. gram.:adj.}
\end{itemize}
\begin{itemize}
\item {Proveniência:(Do gr. \textunderscore isos\textunderscore  + \textunderscore klinein\textunderscore )}
\end{itemize}
O mesmo que \textunderscore isogónico\textunderscore .
Que tem a mesma inclinação.
\section{Isocolo}
\begin{itemize}
\item {Grp. gram.:adj.}
\end{itemize}
\begin{itemize}
\item {Utilização:Gram.}
\end{itemize}
\begin{itemize}
\item {Grp. gram.:m.}
\end{itemize}
\begin{itemize}
\item {Utilização:Gram.}
\end{itemize}
\begin{itemize}
\item {Proveniência:(Do gr. \textunderscore isos\textunderscore  + \textunderscore kolon\textunderscore )}
\end{itemize}
Diz-se do período, cujos membros são iguaes.
Período, cujos membros são iguaes.
(Cp. \textunderscore isocólon\textunderscore )
\section{Isocólon}
\begin{itemize}
\item {Grp. gram.:m.}
\end{itemize}
\begin{itemize}
\item {Utilização:Gram.}
\end{itemize}
\begin{itemize}
\item {Proveniência:(Do gr. \textunderscore isos\textunderscore  + \textunderscore kolon\textunderscore )}
\end{itemize}
Período, cujos membros são iguaes.
\section{Isócoro}
\begin{itemize}
\item {Grp. gram.:m.  e  adj.}
\end{itemize}
\begin{itemize}
\item {Proveniência:(Do gr. \textunderscore isos\textunderscore  + \textunderscore khoros\textunderscore )}
\end{itemize}
Dizia-se do verso hexâmetro, quando composto só de espondeus.
\section{Isocristas}
\begin{itemize}
\item {Grp. gram.:m. pl.}
\end{itemize}
\begin{itemize}
\item {Proveniência:(Do gr. \textunderscore isos\textunderscore  + \textunderscore khristos\textunderscore )}
\end{itemize}
Seita religiosa, que sustentava serem os Apóstolos iguaes a Cristo.
\section{Isocromático}
\begin{itemize}
\item {Grp. gram.:adj.}
\end{itemize}
Que tem coloração uniforme.
Relativo á isocromia.
\section{Isocromia}
\begin{itemize}
\item {Grp. gram.:f.}
\end{itemize}
\begin{itemize}
\item {Proveniência:(Do gr. \textunderscore isos\textunderscore  + \textunderscore khroma\textunderscore )}
\end{itemize}
O mesmo que \textunderscore litocromia\textunderscore .
\section{Isocrónico}
\begin{itemize}
\item {Grp. gram.:adj.}
\end{itemize}
O mesmo que \textunderscore isócrono\textunderscore .
\section{Isocronismo}
\begin{itemize}
\item {Grp. gram.:m.}
\end{itemize}
Qualidade de isócrono.
\section{Isócrono}
\begin{itemize}
\item {Grp. gram.:adj.}
\end{itemize}
\begin{itemize}
\item {Proveniência:(Lat. \textunderscore isochronus\textunderscore )}
\end{itemize}
Que se realiza em tempos iguaes, ou ao mesmo tempo.
\section{Isodáctilo}
\begin{itemize}
\item {Grp. gram.:adj.}
\end{itemize}
\begin{itemize}
\item {Utilização:Zool.}
\end{itemize}
\begin{itemize}
\item {Proveniência:(Do gr. \textunderscore isos\textunderscore  + \textunderscore daktulos\textunderscore )}
\end{itemize}
Que tem os dedos todos iguaes.
\section{Isodáctylo}
\begin{itemize}
\item {Grp. gram.:adj.}
\end{itemize}
\begin{itemize}
\item {Utilização:Zool.}
\end{itemize}
\begin{itemize}
\item {Proveniência:(Do gr. \textunderscore isos\textunderscore  + \textunderscore daktulos\textunderscore )}
\end{itemize}
Que tem os dedos todos iguaes.
\section{Isodínamas}
\begin{itemize}
\item {Grp. gram.:f. pl.}
\end{itemize}
\begin{itemize}
\item {Utilização:Bot.}
\end{itemize}
\begin{itemize}
\item {Proveniência:(Do gr. \textunderscore isos\textunderscore  + \textunderscore duanimis\textunderscore )}
\end{itemize}
Nome, proposto por Cassini, em substituição de \textunderscore dicotiledóneas\textunderscore .
\section{Isodinâmico}
\begin{itemize}
\item {Grp. gram.:adj.}
\end{itemize}
\begin{itemize}
\item {Proveniência:(De \textunderscore iso...\textunderscore  + \textunderscore dinâmico\textunderscore )}
\end{itemize}
Que tem a mesma fôrça, a mesma intensidade.
\section{Isódomo}
\begin{itemize}
\item {Grp. gram.:adj.}
\end{itemize}
\begin{itemize}
\item {Proveniência:(Lat. \textunderscore isodomos\textunderscore )}
\end{itemize}
Diz-se do edifício, em que todas as pedras foram cortadas em esquadria, e com a mesma altura, formando fiadas regulares e iguaes.
\section{Isodonte}
\begin{itemize}
\item {Grp. gram.:adj.}
\end{itemize}
\begin{itemize}
\item {Utilização:Zool.}
\end{itemize}
\begin{itemize}
\item {Proveniência:(Do gr. \textunderscore isos\textunderscore  + \textunderscore odous\textunderscore , \textunderscore odontos\textunderscore )}
\end{itemize}
Cujos dentes são todos iguaes ou semelhantes.
\section{Isodýnamas}
\begin{itemize}
\item {Grp. gram.:f. pl.}
\end{itemize}
\begin{itemize}
\item {Utilização:Bot.}
\end{itemize}
\begin{itemize}
\item {Proveniência:(Do gr. \textunderscore isos\textunderscore  + \textunderscore duanimis\textunderscore )}
\end{itemize}
Nome, proposto por Cassini, em substituição de \textunderscore dicotyledóneas\textunderscore .
\section{Isodynâmico}
\begin{itemize}
\item {Grp. gram.:adj.}
\end{itemize}
\begin{itemize}
\item {Proveniência:(De \textunderscore iso...\textunderscore  + \textunderscore dynâmico\textunderscore )}
\end{itemize}
Que tem a mesma fôrça, a mesma intensidade.
\section{Isoédrico}
\begin{itemize}
\item {Grp. gram.:adj.}
\end{itemize}
\begin{itemize}
\item {Proveniência:(Do gr. \textunderscore isos\textunderscore  + \textunderscore edra\textunderscore )}
\end{itemize}
Que tem faces semelhantes.
\section{Isoétas}
\begin{itemize}
\item {Grp. gram.:f. pl.}
\end{itemize}
\begin{itemize}
\item {Proveniência:(Do gr. \textunderscore isos\textunderscore  + \textunderscore etos\textunderscore )}
\end{itemize}
Plantas cryptogâmicas, da fam. das lycopodiáceas.
\section{Isogeotérmico}
\begin{itemize}
\item {Grp. gram.:adj.}
\end{itemize}
\begin{itemize}
\item {Proveniência:(Do gr. \textunderscore isos\textunderscore  + \textunderscore gaia\textunderscore  + \textunderscore termos\textunderscore )}
\end{itemize}
Diz-se, em meteorologia, da linha que passa por todos os pontos, em que a temperatura média do sol é a mesma.
\section{Isogeotermo}
\begin{itemize}
\item {Grp. gram.:adj.}
\end{itemize}
\begin{itemize}
\item {Proveniência:(Do gr. \textunderscore isos\textunderscore  + \textunderscore gaia\textunderscore  + \textunderscore termos\textunderscore )}
\end{itemize}
Diz-se, em meteorologia, da linha que passa por todos os pontos, em que a temperatura média do sol é a mesma.
\section{Isogeothérmico}
\begin{itemize}
\item {Grp. gram.:adj.}
\end{itemize}
\begin{itemize}
\item {Proveniência:(Do gr. \textunderscore isos\textunderscore  + \textunderscore gaia\textunderscore  + \textunderscore termos\textunderscore )}
\end{itemize}
Diz-se, em meteorologia, da linha que passa por todos os pontos, em que a temperatura média do sol é a mesma.
\section{Isogeothermo}
\begin{itemize}
\item {Grp. gram.:adj.}
\end{itemize}
\begin{itemize}
\item {Proveniência:(Do gr. \textunderscore isos\textunderscore  + \textunderscore gaia\textunderscore  + \textunderscore termos\textunderscore )}
\end{itemize}
Diz-se, em meteorologia, da linha que passa por todos os pontos, em que a temperatura média do sol é a mesma.
\section{Isógino}
\begin{itemize}
\item {Grp. gram.:adj.}
\end{itemize}
\begin{itemize}
\item {Utilização:Bot.}
\end{itemize}
\begin{itemize}
\item {Proveniência:(Do gr. \textunderscore isos\textunderscore  + \textunderscore gune\textunderscore )}
\end{itemize}
Que tem carpellas e pétalas em números iguaes.
\section{Isogónico}
\begin{itemize}
\item {Grp. gram.:adj.}
\end{itemize}
\begin{itemize}
\item {Proveniência:(Do gr. \textunderscore isos\textunderscore  + \textunderscore gonos\textunderscore )}
\end{itemize}
Que tem ângulos iguaes.
Que tem a mesma inclinação.
\section{Isógono}
\begin{itemize}
\item {Grp. gram.:adj.}
\end{itemize}
\begin{itemize}
\item {Proveniência:(Do gr. \textunderscore isos\textunderscore  + \textunderscore gonos\textunderscore )}
\end{itemize}
Que tem ângulos iguaes.
Que tem a mesma inclinação.
\section{Isografia}
\begin{itemize}
\item {Grp. gram.:f.}
\end{itemize}
\begin{itemize}
\item {Proveniência:(Do gr. \textunderscore isos\textunderscore  + \textunderscore graphein\textunderscore )}
\end{itemize}
Reproducção exacta de letra manuscrita; fac-símile.
\section{Isográfico}
\begin{itemize}
\item {Grp. gram.:adj.}
\end{itemize}
Relativo á isografia.
\section{Isographia}
\begin{itemize}
\item {Grp. gram.:f.}
\end{itemize}
\begin{itemize}
\item {Proveniência:(Do gr. \textunderscore isos\textunderscore  + \textunderscore graphein\textunderscore )}
\end{itemize}
Reproducção exacta de letra manuscrita; fac-símile.
\section{Isográphico}
\begin{itemize}
\item {Grp. gram.:adj.}
\end{itemize}
Relativo á isographia.
\section{Isógyno}
\begin{itemize}
\item {Grp. gram.:adj.}
\end{itemize}
\begin{itemize}
\item {Utilização:Bot.}
\end{itemize}
\begin{itemize}
\item {Proveniência:(Do gr. \textunderscore isos\textunderscore  + \textunderscore gune\textunderscore )}
\end{itemize}
Que tem carpellas e pétalas em números iguaes.
\section{Isolar}
\textunderscore v. t.\textunderscore  (e der)
(V. \textunderscore insular\textunderscore ^1)
\section{Isolépis}
\begin{itemize}
\item {Grp. gram.:f.}
\end{itemize}
\begin{itemize}
\item {Proveniência:(Do gr. \textunderscore isos\textunderscore  + \textunderscore lepis\textunderscore )}
\end{itemize}
Gênero de plantas cyperáceas.
\section{Isolítero}
\begin{itemize}
\item {Grp. gram.:adj.}
\end{itemize}
\begin{itemize}
\item {Utilização:Gram.}
\end{itemize}
\begin{itemize}
\item {Proveniência:(T. hýbr., do gr. \textunderscore isos\textunderscore  + lat. \textunderscore litera\textunderscore )}
\end{itemize}
Diz-se do vocábulo, que se escreve com as mesmas letras que outro; homógrapho. Cf. J. F. Castilho, \textunderscore Orthogr. Port.\textunderscore , 165.
\section{Isólogo}
\begin{itemize}
\item {Grp. gram.:adj.}
\end{itemize}
\begin{itemize}
\item {Utilização:Chím.}
\end{itemize}
\begin{itemize}
\item {Proveniência:(Do gr. \textunderscore isos\textunderscore  + \textunderscore logos\textunderscore )}
\end{itemize}
Que tem composição análoga.
Diz-se da série que comprehende todas as séries homólogas ou grupos de carbonetos.
\section{Isómalo}
\begin{itemize}
\item {Grp. gram.:m.}
\end{itemize}
\begin{itemize}
\item {Proveniência:(Gr. \textunderscore isomalos\textunderscore )}
\end{itemize}
Gênero de insectos coleópteros pentâmeros.
\section{Isomeria}
\begin{itemize}
\item {Grp. gram.:f.}
\end{itemize}
(V.isomerismo)
\section{Isomericoscópio}
\begin{itemize}
\item {Grp. gram.:m.}
\end{itemize}
\begin{itemize}
\item {Proveniência:(Do gr. \textunderscore isos\textunderscore  + \textunderscore meros\textunderscore  + \textunderscore skopein\textunderscore )}
\end{itemize}
Apparelho, inventado pelo prof. A. Luso, e que mostra, além da rotação e translação da Terra, o deslocamento dos equinóccios e a mudança das constellações zodiacaes, da estrêlla polar, etc.
\section{Isomérico}
\begin{itemize}
\item {Grp. gram.:adj.}
\end{itemize}
\begin{itemize}
\item {Proveniência:(De \textunderscore isómero\textunderscore )}
\end{itemize}
Relativo ao isomerismo.
\section{Isómeris}
\begin{itemize}
\item {Grp. gram.:m.}
\end{itemize}
Planta crucífera.
\section{Isomerismo}
\begin{itemize}
\item {Grp. gram.:m.}
\end{itemize}
Qualidade de isómero.
\section{Isómero}
\begin{itemize}
\item {Grp. gram.:adj.}
\end{itemize}
\begin{itemize}
\item {Proveniência:(Do gr. \textunderscore isos\textunderscore  + \textunderscore meros\textunderscore )}
\end{itemize}
Que é formado de partes semelhantes.
Que tem propriedades differentes e composição idêntica.
\section{Isófago}
\begin{itemize}
\item {Grp. gram.:m.  e  adj.}
\end{itemize}
\begin{itemize}
\item {Proveniência:(Do gr. \textunderscore isos\textunderscore  + \textunderscore phagein\textunderscore )}
\end{itemize}
O mesmo que \textunderscore canibal\textunderscore :«\textunderscore junto destes vivem... os isófagos... bárbaros...\textunderscore »\textunderscore Ethiopia Or.\textunderscore , l. I, c. I.
\section{Isofíleo}
\begin{itemize}
\item {Grp. gram.:adj.}
\end{itemize}
\begin{itemize}
\item {Utilização:Bot.}
\end{itemize}
\begin{itemize}
\item {Proveniência:(Do gr. \textunderscore isos\textunderscore  + \textunderscore phullon\textunderscore )}
\end{itemize}
Diz-se das plantas que têm fôlhas iguaes.
\section{Isofone}
\begin{itemize}
\item {Grp. gram.:adj.}
\end{itemize}
\begin{itemize}
\item {Proveniência:(Do gr. \textunderscore isos\textunderscore  + \textunderscore phone\textunderscore )}
\end{itemize}
Que tem a voz semelhante á de outrem ou igual timbre de voz.
\section{Isofónico}
\begin{itemize}
\item {Grp. gram.:adj.}
\end{itemize}
\begin{itemize}
\item {Proveniência:(Do gr. \textunderscore isos\textunderscore  + \textunderscore phone\textunderscore )}
\end{itemize}
Que tem a voz semelhante á de outrem ou igual timbre de voz.
\section{Isometria}
\begin{itemize}
\item {Grp. gram.:f.}
\end{itemize}
\begin{itemize}
\item {Proveniência:(Do gr. \textunderscore isos\textunderscore  + \textunderscore metron\textunderscore )}
\end{itemize}
Igualdade de dimensões.
\section{Isométrico}
\begin{itemize}
\item {Grp. gram.:adj.}
\end{itemize}
Que tem isometria.
\section{Isomorfia}
\begin{itemize}
\item {Grp. gram.:f.}
\end{itemize}
O mesmo que \textunderscore isomorfismo\textunderscore .
\section{Isomorfismo}
\begin{itemize}
\item {Grp. gram.:m.}
\end{itemize}
Qualidade de isomorfo.
\section{Isomorfo}
\begin{itemize}
\item {Grp. gram.:adj.}
\end{itemize}
\begin{itemize}
\item {Utilização:Geol.}
\end{itemize}
\begin{itemize}
\item {Proveniência:(Do gr. \textunderscore isos\textunderscore  + \textunderscore morphe\textunderscore )}
\end{itemize}
Que tem a mesma fórma crystalina.
\section{Isomorphia}
\begin{itemize}
\item {Grp. gram.:f.}
\end{itemize}
O mesmo que \textunderscore isomorphismo\textunderscore .
\section{Isomorphismo}
\begin{itemize}
\item {Grp. gram.:m.}
\end{itemize}
Qualidade de isomorpho.
\section{Isomorpho}
\begin{itemize}
\item {Grp. gram.:adj.}
\end{itemize}
\begin{itemize}
\item {Utilização:Geol.}
\end{itemize}
\begin{itemize}
\item {Proveniência:(Do gr. \textunderscore isos\textunderscore  + \textunderscore morphe\textunderscore )}
\end{itemize}
Que tem a mesma fórma crystallina.
\section{Isonandro}
\begin{itemize}
\item {Grp. gram.:m.}
\end{itemize}
\begin{itemize}
\item {Proveniência:(Do gr. \textunderscore isos\textunderscore  + \textunderscore aner\textunderscore , \textunderscore andros\textunderscore )}
\end{itemize}
Bella arvore sapotácea da Malásia.
\section{Isonema}
\begin{itemize}
\item {Grp. gram.:f.}
\end{itemize}
\begin{itemize}
\item {Proveniência:(Do gr. \textunderscore isos\textunderscore  + \textunderscore nema\textunderscore )}
\end{itemize}
Gênero de plantas apocíneas.
\section{Isonomia}
\begin{itemize}
\item {Grp. gram.:f.}
\end{itemize}
Igualdade perante a lei.
Qualidade de insónomo.
\section{Isónomo}
\begin{itemize}
\item {Grp. gram.:adj.}
\end{itemize}
\begin{itemize}
\item {Utilização:Geol.}
\end{itemize}
\begin{itemize}
\item {Proveniência:(Do gr. \textunderscore isos\textunderscore  + \textunderscore nomos\textunderscore )}
\end{itemize}
Que crystalliza, segundo a mesma lei.
\section{Isopata}
\begin{itemize}
\item {Grp. gram.:m.}
\end{itemize}
\begin{itemize}
\item {Proveniência:(Do gr. \textunderscore isos\textunderscore  + \textunderscore pathos\textunderscore )}
\end{itemize}
Aquele que exerce a isopatia.
\section{Isopatha}
\begin{itemize}
\item {Grp. gram.:m.}
\end{itemize}
\begin{itemize}
\item {Proveniência:(Do gr. \textunderscore isos\textunderscore  + \textunderscore pathos\textunderscore )}
\end{itemize}
Aquelle que exerce a isopathia.
\section{Isopathia}
\begin{itemize}
\item {Grp. gram.:f.}
\end{itemize}
\begin{itemize}
\item {Proveniência:(De \textunderscore isopatha\textunderscore )}
\end{itemize}
Systema de curar doenças por meios semelhantes á causa dellas.
\section{Isopatia}
\begin{itemize}
\item {Grp. gram.:f.}
\end{itemize}
\begin{itemize}
\item {Proveniência:(De \textunderscore isopata\textunderscore )}
\end{itemize}
Sistema de curar doenças por meios semelhantes á causa delas.
\section{Isoperimétrico}
\begin{itemize}
\item {Grp. gram.:adj.}
\end{itemize}
\begin{itemize}
\item {Utilização:Mathem.}
\end{itemize}
\begin{itemize}
\item {Proveniência:(De \textunderscore iso...\textunderscore  + \textunderscore perímetro\textunderscore )}
\end{itemize}
Que tem perímetro igual.
\section{Isòpétalo}
\begin{itemize}
\item {Grp. gram.:adj.}
\end{itemize}
\begin{itemize}
\item {Utilização:Bot.}
\end{itemize}
\begin{itemize}
\item {Proveniência:(De \textunderscore isó...\textunderscore  + \textunderscore pétala\textunderscore )}
\end{itemize}
Que tem pétalas iguaes.
\section{Isóphago}
\begin{itemize}
\item {Grp. gram.:m.  e  adj.}
\end{itemize}
\begin{itemize}
\item {Proveniência:(Do gr. \textunderscore isos\textunderscore  + \textunderscore phagein\textunderscore )}
\end{itemize}
O mesmo que \textunderscore cannibal\textunderscore :«\textunderscore junto destes vivem... os isóphagos... bárbaros...\textunderscore »\textunderscore Ethiopia Or.\textunderscore , l. I, c. I.
\section{Isophone}
\begin{itemize}
\item {Grp. gram.:adj.}
\end{itemize}
\begin{itemize}
\item {Proveniência:(Do gr. \textunderscore isos\textunderscore  + \textunderscore phone\textunderscore )}
\end{itemize}
Que tem a voz semelhante á de outrem ou igual timbre de voz.
\section{Isophono}
\begin{itemize}
\item {Grp. gram.:adj.}
\end{itemize}
\begin{itemize}
\item {Proveniência:(Do gr. \textunderscore isos\textunderscore  + \textunderscore phone\textunderscore )}
\end{itemize}
Que tem a voz semelhante á de outrem ou igual timbre de voz.
\section{Isophýlleo}
\begin{itemize}
\item {Grp. gram.:adj.}
\end{itemize}
\begin{itemize}
\item {Utilização:Bot.}
\end{itemize}
\begin{itemize}
\item {Proveniência:(Do gr. \textunderscore isos\textunderscore  + \textunderscore phullon\textunderscore )}
\end{itemize}
Diz-se das plantas que têm fôlhas iguaes.
\section{Isopiro}
\begin{itemize}
\item {Grp. gram.:m.}
\end{itemize}
\begin{itemize}
\item {Proveniência:(Lat. \textunderscore isopyron\textunderscore )}
\end{itemize}
Planta ranunculácea.
\section{Isopódeo}
\begin{itemize}
\item {Grp. gram.:adj.}
\end{itemize}
O mesmo que \textunderscore isópode\textunderscore .
\section{Isópode}
\begin{itemize}
\item {Grp. gram.:M. pl.}
\end{itemize}
\begin{itemize}
\item {Proveniência:(Do gr. \textunderscore isos\textunderscore  + \textunderscore pous\textunderscore , \textunderscore podos\textunderscore )}
\end{itemize}
\textunderscore adj. Zool.\textunderscore *
Que tem as patas iguaes ou semelhantes.
Ordem de crustáceos, que tem por typo o bicho de conta.
\section{Isópteros}
\begin{itemize}
\item {Grp. gram.:m. pl.}
\end{itemize}
\begin{itemize}
\item {Proveniência:(Do gr. \textunderscore isos\textunderscore  + \textunderscore pteron\textunderscore )}
\end{itemize}
Gênero de insectos coleópteros heterómeros.
\section{Isopyro}
\begin{itemize}
\item {Grp. gram.:m.}
\end{itemize}
\begin{itemize}
\item {Proveniência:(Lat. \textunderscore isopyron\textunderscore )}
\end{itemize}
Planta ranunculácea.
\section{Isoramuno}
\begin{itemize}
\item {Grp. gram.:m.}
\end{itemize}
Arvore do Malabar, cujo suco se applicava nas doenças pulmonares.
\section{Isósceles}
\begin{itemize}
\item {Grp. gram.:adj.}
\end{itemize}
\begin{itemize}
\item {Utilização:Geom.}
\end{itemize}
\begin{itemize}
\item {Proveniência:(Lat. \textunderscore isosceles\textunderscore )}
\end{itemize}
Que tem dois lados iguaes: \textunderscore triângulo isósceles\textunderscore .
\section{Isoscelia}
\begin{itemize}
\item {Grp. gram.:f.}
\end{itemize}
Propriedade do triângulo isósceles.
\section{Isosférico}
\begin{itemize}
\item {Grp. gram.:adj.}
\end{itemize}
\begin{itemize}
\item {Proveniência:(Do gr. \textunderscore isos\textunderscore  + \textunderscore sphaira\textunderscore )}
\end{itemize}
Que tem esfera igual.
\section{Isosista}
\begin{itemize}
\item {fónica:sis}
\end{itemize}
\begin{itemize}
\item {Grp. gram.:adj.}
\end{itemize}
\begin{itemize}
\item {Utilização:Geol.}
\end{itemize}
\begin{itemize}
\item {Proveniência:(Do gr. \textunderscore isos\textunderscore  + \textunderscore seismo\textunderscore )}
\end{itemize}
Diz-se da curva, que liga todos os pontos, em que se manifesta um movimento sismico com igual intensidade.
\section{Isosphérico}
\begin{itemize}
\item {Grp. gram.:adj.}
\end{itemize}
\begin{itemize}
\item {Proveniência:(Do gr. \textunderscore isos\textunderscore  + \textunderscore sphaira\textunderscore )}
\end{itemize}
Que tem esphera igual.
\section{Isossista}
\begin{itemize}
\item {Grp. gram.:adj.}
\end{itemize}
\begin{itemize}
\item {Utilização:Geol.}
\end{itemize}
\begin{itemize}
\item {Proveniência:(Do gr. \textunderscore isos\textunderscore  + \textunderscore seismo\textunderscore )}
\end{itemize}
Diz-se da curva, que liga todos os pontos, em que se manifesta um movimento sismico com igual intensidade.
\section{Isostêmone}
\begin{itemize}
\item {Grp. gram.:adj.}
\end{itemize}
\begin{itemize}
\item {Utilização:Bot.}
\end{itemize}
\begin{itemize}
\item {Proveniência:(Do gr. \textunderscore isos\textunderscore  + \textunderscore stemon\textunderscore )}
\end{itemize}
Diz-se das flôres, em que o número dos estames é igual ao das pétalas.
\section{Isótele}
\begin{itemize}
\item {Grp. gram.:m.}
\end{itemize}
\begin{itemize}
\item {Proveniência:(Gr. \textunderscore isoteles\textunderscore )}
\end{itemize}
Aquelle, que, entre os Athenienses, occupava o grau intermédio ao cidadão e ao estrangeiro domiciliado, e tinha todos os direitos de cidade.
\section{Isotelia}
\begin{itemize}
\item {Grp. gram.:f.}
\end{itemize}
Estado ou qualidade de isótele.
\section{Isotérico}
\begin{itemize}
\item {Grp. gram.:adj.}
\end{itemize}
O mesmo que \textunderscore isótero\textunderscore .
\section{Isotérmico}
\begin{itemize}
\item {Grp. gram.:adj.}
\end{itemize}
\begin{itemize}
\item {Proveniência:(De \textunderscore iso...\textunderscore  + \textunderscore térmico\textunderscore )}
\end{itemize}
Que tem igual temperatura.
Diz-se, em meteorologia, da linha que passa por todos os pontos, em que a temperatura média anual é a mesma.
\section{Isotermo}
\begin{itemize}
\item {Grp. gram.:adj.}
\end{itemize}
O mesmo ou melhor que \textunderscore isotérmico\textunderscore .
\section{Isótero}
\begin{itemize}
\item {Grp. gram.:adj.}
\end{itemize}
\begin{itemize}
\item {Proveniência:(Do gr. \textunderscore isos\textunderscore  + \textunderscore theros\textunderscore )}
\end{itemize}
Diz-se, em meteorologia, da linha que passa pelos pontos da Terra, em que a temperatura média é a mesma no estio.
\section{Isothérico}
\begin{itemize}
\item {Grp. gram.:adj.}
\end{itemize}
O mesmo que \textunderscore isóthero\textunderscore .
\section{Isothérmico}
\begin{itemize}
\item {Grp. gram.:adj.}
\end{itemize}
\begin{itemize}
\item {Proveniência:(De \textunderscore iso...\textunderscore  + \textunderscore thérmico\textunderscore )}
\end{itemize}
Que tem igual temperatura.
Diz-se, em meteorologia, da linha que passa por todos os pontos, em que a temperatura média annual é a mesma.
\section{Isothermo}
\begin{itemize}
\item {Grp. gram.:adj.}
\end{itemize}
O mesmo ou melhor que \textunderscore isothérmico\textunderscore .
\section{Isóthero}
\begin{itemize}
\item {Grp. gram.:adj.}
\end{itemize}
\begin{itemize}
\item {Proveniência:(Do gr. \textunderscore isos\textunderscore  + \textunderscore theros\textunderscore )}
\end{itemize}
Diz-se, em meteorologia, da linha que passa pelos pontos da Terra, em que a temperatura média é a mesma no estio.
\section{Isotonia}
\begin{itemize}
\item {Grp. gram.:f.}
\end{itemize}
\begin{itemize}
\item {Utilização:Phýs.}
\end{itemize}
\begin{itemize}
\item {Proveniência:(Do gr. \textunderscore isos\textunderscore  + \textunderscore tonos\textunderscore )}
\end{itemize}
Exquilíbrio mollecular de dois liquidos, que tem a mesma tensão osmótica.
\section{Isotónico}
\begin{itemize}
\item {Grp. gram.:adj.}
\end{itemize}
Em que há isotonia.
\section{Isótono}
\begin{itemize}
\item {Grp. gram.:adj.}
\end{itemize}
\begin{itemize}
\item {Utilização:Gram.}
\end{itemize}
\begin{itemize}
\item {Proveniência:(Do gr. \textunderscore isos\textunderscore  + \textunderscore tonos\textunderscore )}
\end{itemize}
Que tem o mesmo tom ou o mesmo accento tónico.
\section{Isotrópico}
\begin{itemize}
\item {Grp. gram.:adj.}
\end{itemize}
\begin{itemize}
\item {Utilização:Phýs.}
\end{itemize}
\begin{itemize}
\item {Proveniência:(Do gr. \textunderscore isos\textunderscore  + \textunderscore tropein\textunderscore )}
\end{itemize}
Diz-se de qualquer meio transparente, em que a luz actua igualmente em todas as direcções.
\section{Isótropo}
\begin{itemize}
\item {Grp. gram.:adj.}
\end{itemize}
\begin{itemize}
\item {Utilização:Miner.}
\end{itemize}
\begin{itemize}
\item {Proveniência:(Do gr. \textunderscore isos\textunderscore  + \textunderscore tropein\textunderscore )}
\end{itemize}
Diz-se do corpo, que em todas as direcções apresenta as mesmas propriedades ópticas.
\section{Isqueiro}
\begin{itemize}
\item {Grp. gram.:m.}
\end{itemize}
\begin{itemize}
\item {Utilização:Bras}
\end{itemize}
Pequena caixa, feita de chifre, onde os fumadores guardam a isca.
\section{Isquemia}
\begin{itemize}
\item {Grp. gram.:f.}
\end{itemize}
\begin{itemize}
\item {Utilização:Med.}
\end{itemize}
\begin{itemize}
\item {Proveniência:(Do gr. \textunderscore iskhein\textunderscore  + \textunderscore haima\textunderscore )}
\end{itemize}
Suspensão da circulação do sangue.
\section{Isquêmico}
\begin{itemize}
\item {Grp. gram.:adj.}
\end{itemize}
Relativo á isquêmia.
Que susta o movimento do sangue nos vasos orgânicos.
\section{Isquiadelfos}
\begin{itemize}
\item {Grp. gram.:m.  e  adj. pl.}
\end{itemize}
\begin{itemize}
\item {Proveniência:(Do gr. \textunderscore iskhion\textunderscore  + \textunderscore adelphos\textunderscore )}
\end{itemize}
Monstros duplos, cujos corpos, opostos um ao outro, estão ligados pela bacia.
\section{Isquiádico}
\begin{itemize}
\item {Grp. gram.:adj.}
\end{itemize}
O mesmo que \textunderscore isquiático\textunderscore . Cf. C. Guerreiro, \textunderscore Versif. Port.\textunderscore , 434.
\section{Isquial}
\begin{itemize}
\item {Grp. gram.:adj.}
\end{itemize}
Relativo ao ísquion.
\section{Isquiático}
\begin{itemize}
\item {Grp. gram.:adj.}
\end{itemize}
Relativo ao ísquion; sciático: \textunderscore dôr isquiática\textunderscore .
\section{Isoquimênico}
\begin{itemize}
\item {Grp. gram.:adj.}
\end{itemize}
\begin{itemize}
\item {Proveniência:(Do gr. \textunderscore isos\textunderscore  + \textunderscore kheimainein\textunderscore )}
\end{itemize}
Diz-se, em meteorologia, da linha que passa por todos os pontos do globo, que no inverno têm a mesma temperatura.
\section{Isoquímeno}
\begin{itemize}
\item {Grp. gram.:adj.}
\end{itemize}
\begin{itemize}
\item {Proveniência:(Do gr. \textunderscore isos\textunderscore  + \textunderscore kheimainein\textunderscore )}
\end{itemize}
Diz-se, em meteorologia, da linha que passa por todos os pontos do globo, que no inverno têm a mesma temperatura.
\section{Ísquio}
\begin{itemize}
\item {Grp. gram.:m.}
\end{itemize}
O mesmo ou melhor que \textunderscore ísquion\textunderscore .
\section{Isquio-femoral}
\begin{itemize}
\item {Grp. gram.:adj.}
\end{itemize}
Relativo ao ísquion e ao fêmur.
\section{Ísquion}
\begin{itemize}
\item {Grp. gram.:m.}
\end{itemize}
\begin{itemize}
\item {Utilização:Anat.}
\end{itemize}
\begin{itemize}
\item {Proveniência:(Gr. \textunderscore iskhion\textunderscore )}
\end{itemize}
Uma das três partes do osso ilíaco, em que se articula o osso da coxa.
Quadril.
\section{Isquiopagia}
\begin{itemize}
\item {Grp. gram.:f.}
\end{itemize}
Estado de isquiópagos.
\section{Isquiópagos}
\begin{itemize}
\item {Grp. gram.:m.  e  adj. pl.}
\end{itemize}
\begin{itemize}
\item {Proveniência:(Do gr. \textunderscore iskhion\textunderscore  + \textunderscore pagein\textunderscore )}
\end{itemize}
Diz-se dos monstros, compostos de dois indivíduos, reunidos pela região hipogástrica, e tendo um umbigo comum.
\section{Israel}
\begin{itemize}
\item {fónica:ra-el}
\end{itemize}
\begin{itemize}
\item {Grp. gram.:m.}
\end{itemize}
\begin{itemize}
\item {Proveniência:(De \textunderscore Israel\textunderscore , n. p.)}
\end{itemize}
Designação collectiva dos Israelitas; Hebreus.
\section{Israelita}
\begin{itemize}
\item {fónica:ra-e}
\end{itemize}
\begin{itemize}
\item {Grp. gram.:m.}
\end{itemize}
\begin{itemize}
\item {Grp. gram.:Adj.}
\end{itemize}
Individuo pertencente ao povo de Israel.
Relativo aos Israelitas.
\section{Israelítico}
\begin{itemize}
\item {fónica:ra-e}
\end{itemize}
\begin{itemize}
\item {Grp. gram.:adj.}
\end{itemize}
Relativo, aos Israelitas.
\section{Issar}
\begin{itemize}
\item {Grp. gram.:v. t.}
\end{itemize}
(Talvez preferível a \textunderscore içar\textunderscore )
\section{Issicariba}
\begin{itemize}
\item {Grp. gram.:f.}
\end{itemize}
Árvore therebintácea da América.
\section{Isso}
\begin{itemize}
\item {Grp. gram.:pron.}
\end{itemize}
\begin{itemize}
\item {Proveniência:(Do lat. \textunderscore ipsum\textunderscore )}
\end{itemize}
Èsse objecto, êsses objectos.
Essa coisa; êsse negócio.
\section{Ísthmicas}
\begin{itemize}
\item {Grp. gram.:f. pl.}
\end{itemize}
\begin{itemize}
\item {Proveniência:(De \textunderscore isthmico\textunderscore )}
\end{itemize}
Odes, compostas por Píndaro, em honra dos vencedores nos jogos ísthmicos.
\section{Ísthmico}
\begin{itemize}
\item {Grp. gram.:adj.}
\end{itemize}
\begin{itemize}
\item {Grp. gram.:M. pl.}
\end{itemize}
\begin{itemize}
\item {Proveniência:(De \textunderscore isthmo\textunderscore )}
\end{itemize}
Relativo ou semelhante a isthmo.
Relativo ao isthtmo de Corintho.
Jogos, que se celebravam, de três em três annos, no isthmo de Corintho.
\section{Ísthmio}
\begin{itemize}
\item {Grp. gram.:adj.}
\end{itemize}
Relativo aos jogos ísthmicos. Cf. Castilho, \textunderscore Fastos\textunderscore , I, 541.
\section{Isthmo}
\begin{itemize}
\item {Grp. gram.:m.}
\end{itemize}
\begin{itemize}
\item {Proveniência:(Gr. \textunderscore isthmos\textunderscore )}
\end{itemize}
Língua ou faixa estreita de terra, que liga duas partes do continente e separa dois mares.
Objecto, cuja configuração é semelhante á de um isthmo.
\section{Ístmicas}
\begin{itemize}
\item {Grp. gram.:f. pl.}
\end{itemize}
\begin{itemize}
\item {Proveniência:(De \textunderscore istmico\textunderscore )}
\end{itemize}
Odes, compostas por Píndaro, em honra dos vencedores nos jogos ístmicos.
\section{Ístmico}
\begin{itemize}
\item {Grp. gram.:adj.}
\end{itemize}
\begin{itemize}
\item {Grp. gram.:M. pl.}
\end{itemize}
\begin{itemize}
\item {Proveniência:(De \textunderscore istmo\textunderscore )}
\end{itemize}
Relativo ou semelhante a istmo.
Relativo ao istmo de Corinto.
Jogos, que se celebravam, de três em três anos, no istmo de Corinto.
\section{Ístmio}
\begin{itemize}
\item {Grp. gram.:adj.}
\end{itemize}
Relativo aos jogos ístmicos. Cf. Castilho, \textunderscore Fastos\textunderscore , I, 541.
\section{Istmo}
\begin{itemize}
\item {Grp. gram.:m.}
\end{itemize}
\begin{itemize}
\item {Proveniência:(Gr. \textunderscore isthmos\textunderscore )}
\end{itemize}
Língua ou faixa estreita de terra, que liga duas partes do continente e separa dois mares.
Objecto, cuja configuração é semelhante á de um istmo.
\section{Isto}
\begin{itemize}
\item {Grp. gram.:pron.}
\end{itemize}
\begin{itemize}
\item {Proveniência:(Do lat. \textunderscore ístud\textunderscore )}
\end{itemize}
Êste objecto, êstes objectos; êste negócio; esta coisa.
\section{Isuretina}
\begin{itemize}
\item {Grp. gram.:f.}
\end{itemize}
\begin{itemize}
\item {Utilização:Chím.}
\end{itemize}
\begin{itemize}
\item {Proveniência:(Do gr. \textunderscore isos\textunderscore  + \textunderscore ouron\textunderscore )}
\end{itemize}
Corpo isómero da ureia.
\section{Itá}
\begin{itemize}
\item {Grp. gram.:m.}
\end{itemize}
\begin{itemize}
\item {Utilização:Bras}
\end{itemize}
\begin{itemize}
\item {Proveniência:(T. tupi)}
\end{itemize}
Rochedo; pedra.
\section{Itabirite}
\begin{itemize}
\item {Grp. gram.:f.}
\end{itemize}
O mesmo que \textunderscore itabirito\textunderscore .
\section{Itabirito}
\begin{itemize}
\item {Grp. gram.:m.}
\end{itemize}
\begin{itemize}
\item {Utilização:Bras}
\end{itemize}
O mesmo que \textunderscore jacutinga\textunderscore ^2 ou \textunderscore hematita\textunderscore .
\section{Itacava}
\begin{itemize}
\item {Grp. gram.:f.}
\end{itemize}
\begin{itemize}
\item {Utilização:Bras}
\end{itemize}
Árvore silvestre, que dá bôa madeira para construcções.
\section{Itacense}
\begin{itemize}
\item {Grp. gram.:adj.}
\end{itemize}
\begin{itemize}
\item {Proveniência:(Lat. \textunderscore ithacensis\textunderscore )}
\end{itemize}
Relativo a Ítaca.
\section{Itacolumito}
\begin{itemize}
\item {Grp. gram.:m.}
\end{itemize}
Espécie de arenito talcoso, que, nalgumas regiões, é jazigo de diamantes.
\section{Itacuan}
\begin{itemize}
\item {Grp. gram.:m.}
\end{itemize}
\begin{itemize}
\item {Utilização:Bras. do N}
\end{itemize}
\begin{itemize}
\item {Proveniência:(Do guar. \textunderscore ita\textunderscore  + \textunderscore cuan\textunderscore )}
\end{itemize}
Nome de certa pedra amarela, com que se alisam as panelas feitas á mão.
\section{Itaíba}
\begin{itemize}
\item {Grp. gram.:f.}
\end{itemize}
Árvore leguminosa das regiões tropicaes.
\section{Itaimbé}
\begin{itemize}
\item {fónica:ta-im}
\end{itemize}
\begin{itemize}
\item {Grp. gram.:m.}
\end{itemize}
\begin{itemize}
\item {Utilização:Bras. do S}
\end{itemize}
\begin{itemize}
\item {Proveniência:(T. tupi)}
\end{itemize}
Despenhadeiro.
\section{Itaipava}
\begin{itemize}
\item {fónica:ta-i}
\end{itemize}
\begin{itemize}
\item {Grp. gram.:f.}
\end{itemize}
\begin{itemize}
\item {Utilização:Bras}
\end{itemize}
\begin{itemize}
\item {Proveniência:(T. tupi)}
\end{itemize}
Rocha, por onde passam águas, que em seguida formam cataracta.
\section{Italianada}
\begin{itemize}
\item {Grp. gram.:f.}
\end{itemize}
\begin{itemize}
\item {Utilização:T. de Turquel}
\end{itemize}
\begin{itemize}
\item {Proveniência:(De \textunderscore italiano\textunderscore )}
\end{itemize}
Linguagem inintellígivel.
\section{Italianamente}
\begin{itemize}
\item {Grp. gram.:adv.}
\end{itemize}
\begin{itemize}
\item {Proveniência:(De \textunderscore italiano\textunderscore )}
\end{itemize}
Á maneira dos Italianos.
\section{Italianismo}
\begin{itemize}
\item {Grp. gram.:m.}
\end{itemize}
\begin{itemize}
\item {Proveniência:(De \textunderscore italiano\textunderscore )}
\end{itemize}
Imitação affectada da língua ou dos costumes italianos.
Palavra, que, procedente do italiano, entrou noutra língua, como \textunderscore alarma\textunderscore , \textunderscore aguarela\textunderscore , \textunderscore soprano\textunderscore , etc.
Affecto exaggerado a coisas italianas. Cf. Latino, \textunderscore Elogios\textunderscore , 44.
\section{Italianizar}
\begin{itemize}
\item {Grp. gram.:v. t.}
\end{itemize}
Dar feição italiana a.
\section{Italiano}
\begin{itemize}
\item {Grp. gram.:adj.}
\end{itemize}
\begin{itemize}
\item {Grp. gram.:M.}
\end{itemize}
Relativo á Italia.
Aquelle que é natural da Italia.
Língua, falada pelos Italianos.
\section{Itálico}
\begin{itemize}
\item {Grp. gram.:adj.}
\end{itemize}
\begin{itemize}
\item {Utilização:Typ.}
\end{itemize}
\begin{itemize}
\item {Grp. gram.:M.}
\end{itemize}
\begin{itemize}
\item {Proveniência:(Lat. \textunderscore italicus\textunderscore )}
\end{itemize}
Relativo á Italia.
Diz-se do typo, que imita a letra manuscrita.
Fórma de letra, também conhecida por \textunderscore grypho\textunderscore ^2.
\section{Italiotas}
\begin{itemize}
\item {Grp. gram.:m. pl.}
\end{itemize}
\begin{itemize}
\item {Proveniência:(De \textunderscore Itália\textunderscore , n. p.)}
\end{itemize}
Habitantes da Itália, antes da dominação romana.
\section{Ítalo}
\begin{itemize}
\item {Grp. gram.:adj.}
\end{itemize}
\begin{itemize}
\item {Grp. gram.:M.}
\end{itemize}
\begin{itemize}
\item {Proveniência:(Lat. \textunderscore italus\textunderscore )}
\end{itemize}
Relativo á Italia.
Latino; romano; italiano.
Habitante da Itália. Cf. Castilho, \textunderscore Fastos\textunderscore , I, 137.
\section{Italo...}
\begin{itemize}
\item {Grp. gram.:pref.}
\end{itemize}
(designativo de \textunderscore italiano\textunderscore  ou \textunderscore relativo aos Italianos\textunderscore )
\section{Italo-celta}
\begin{itemize}
\item {Grp. gram.:adj.}
\end{itemize}
Relativo ás civilizações romana e céltica. Cf. Latino, \textunderscore Elogios\textunderscore , 70 e 75.
\section{Ítalo-céltico}
\begin{itemize}
\item {Grp. gram.:adj.}
\end{itemize}
Relativo ás civilizações romana e céltica. Cf. Latino, \textunderscore Elogios\textunderscore , 70 e 75.
\section{Italo-gaulês}
\begin{itemize}
\item {Grp. gram.:adj.}
\end{itemize}
Relativo aos povos da Itália e da Gállia.
\section{Italo-germânico}
\begin{itemize}
\item {Grp. gram.:adj.}
\end{itemize}
Relativo aos povos da Itália e da Germânia.
\section{Italo-gótico}
\begin{itemize}
\item {Grp. gram.:adj.}
\end{itemize}
Relativo a Italianos e Godos.
Diz-se especialmente da escrita romana, alterada pelos Godos, quando êstes dominaram na Itália.
\section{Italo-grego}
\begin{itemize}
\item {Grp. gram.:adj.}
\end{itemize}
O mesmo que \textunderscore greco-latino\textunderscore .
\section{Itamaca}
\begin{itemize}
\item {Grp. gram.:f.}
\end{itemize}
\begin{itemize}
\item {Utilização:Bras}
\end{itemize}
Rêde, usada por indígenas do Alto Amazonas.
\section{Itambé}
\begin{itemize}
\item {Grp. gram.:m.}
\end{itemize}
O mesmo que \textunderscore itaimbé\textunderscore .
\section{Itamotinga}
\begin{itemize}
\item {Grp. gram.:f.}
\end{itemize}
\begin{itemize}
\item {Utilização:Bras}
\end{itemize}
Variedade de pedra brilhante, que se acha num dos confluentes do rio Arinos.
\section{Itan}
\begin{itemize}
\item {Grp. gram.:f.}
\end{itemize}
\begin{itemize}
\item {Utilização:Bras. do N}
\end{itemize}
Ornato de pedra, dos que se encontram nas urnas funerárias dos antigos povos aborígenes.
Espécie de concha bivalve.
(Do tupi)
\section{Itanha}
\begin{itemize}
\item {Grp. gram.:f.}
\end{itemize}
\begin{itemize}
\item {Utilização:Bras}
\end{itemize}
Espécie de sapo grande, com duas saliências na cabeça, á maneira de chires.
\section{Itapeva}
\begin{itemize}
\item {Grp. gram.:f.}
\end{itemize}
\begin{itemize}
\item {Utilização:Bras. do N}
\end{itemize}
\begin{itemize}
\item {Proveniência:(T. tupi)}
\end{itemize}
Espécie de recife, parallelo á margem do rio.
\section{Itapicura}
\begin{itemize}
\item {Grp. gram.:f.}
\end{itemize}
\begin{itemize}
\item {Utilização:Bras}
\end{itemize}
Árvore do sertão.
\section{Itapicuro}
\begin{itemize}
\item {Grp. gram.:m.}
\end{itemize}
\begin{itemize}
\item {Utilização:Bras}
\end{itemize}
Árvore do sertão.
\section{Itapuá}
\begin{itemize}
\item {Grp. gram.:m.}
\end{itemize}
\begin{itemize}
\item {Utilização:Bras}
\end{itemize}
Espécie de farpa, para a pesca do pirarucu.
\section{Itatapriás}
\begin{itemize}
\item {Grp. gram.:m. pl.}
\end{itemize}
Tríbo de Índios brasileiros, nas margens do Canapá, affluente do Madeira.
\section{Itatiba}
\begin{itemize}
\item {Grp. gram.:f.}
\end{itemize}
Madeira do Brasil.
\section{Itaúba}
\begin{itemize}
\item {Grp. gram.:f.}
\end{itemize}
\begin{itemize}
\item {Utilização:Bras}
\end{itemize}
Árvore, que dá bôa madeira para construcções.
\section{Ité}
\begin{itemize}
\item {Grp. gram.:adj.}
\end{itemize}
\begin{itemize}
\item {Utilização:Bras}
\end{itemize}
Insipido, que não tem bom sabor.
\section{Item}
\begin{itemize}
\item {fónica:itẽu}
\end{itemize}
\begin{itemize}
\item {Grp. gram.:adv.}
\end{itemize}
\begin{itemize}
\item {Grp. gram.:M.}
\end{itemize}
\begin{itemize}
\item {Proveniência:(T. lat.)}
\end{itemize}
Da mesma fórma; também.
Cada um dos artigos de uma exposição escrita ou requerimento.
Parcella.
\section{Iteração}
\begin{itemize}
\item {Grp. gram.:f.}
\end{itemize}
\begin{itemize}
\item {Proveniência:(Lat. \textunderscore iteratio\textunderscore )}
\end{itemize}
Acto de iterar ou repetir.
\section{Iterar}
\begin{itemize}
\item {Grp. gram.:v. t.}
\end{itemize}
\begin{itemize}
\item {Proveniência:(Lat. \textunderscore iterare\textunderscore )}
\end{itemize}
O mesmo que \textunderscore repetir\textunderscore .
\section{Iterativamente}
\begin{itemize}
\item {Grp. gram.:adv.}
\end{itemize}
De modo iterativo; repetidas vezes.
\section{Iterativo}
\begin{itemize}
\item {Grp. gram.:adj.}
\end{itemize}
\begin{itemize}
\item {Proveniência:(Lat. \textunderscore iterativus\textunderscore )}
\end{itemize}
Próprio para iterar; repetido.
Frequentativo.
\section{Iterável}
\begin{itemize}
\item {Grp. gram.:adj.}
\end{itemize}
\begin{itemize}
\item {Proveniência:(Lat. \textunderscore iterabilis\textunderscore )}
\end{itemize}
Que se póde ou se deve iterar.
\section{Ithacense}
\begin{itemize}
\item {Grp. gram.:adj.}
\end{itemize}
\begin{itemize}
\item {Proveniência:(Lat. \textunderscore ithacensis\textunderscore )}
\end{itemize}
Relativo a Íthaca.
\section{Itinerário}
\begin{itemize}
\item {Grp. gram.:adj.}
\end{itemize}
\begin{itemize}
\item {Grp. gram.:M.}
\end{itemize}
\begin{itemize}
\item {Proveniência:(Lat. \textunderscore itinerarius\textunderscore )}
\end{itemize}
Relativo a caminhos: \textunderscore medida itinerária\textunderscore .
Descripção de caminho.
Viagem.
Livro em que se descreve uma viagem.
Roteiro.
\section{Itrol}
\begin{itemize}
\item {Grp. gram.:m.}
\end{itemize}
Medicamento contra a blennorrhagia.
\section{Itu}
\begin{itemize}
\item {Grp. gram.:m.}
\end{itemize}
O mesmo que \textunderscore pau-ferro\textunderscore .
\section{Ituá}
\begin{itemize}
\item {Grp. gram.:m.}
\end{itemize}
Planta brasileira, de fibras têxteis.
\section{Itupava}
\begin{itemize}
\item {Grp. gram.:f.}
\end{itemize}
\begin{itemize}
\item {Utilização:Bras}
\end{itemize}
O mesmo que \textunderscore cachoeira\textunderscore .
\section{Iú}
\begin{itemize}
\item {Grp. gram.:m.}
\end{itemize}
Gênero de palmeiras do Brasil.
\section{Iúaca}
\begin{itemize}
\item {Grp. gram.:f.}
\end{itemize}
\begin{itemize}
\item {Proveniência:(T. car.)}
\end{itemize}
Gênero de plantas liliáceas das regiões quentes da América.
\section{Iuçá}
\begin{itemize}
\item {Grp. gram.:m.}
\end{itemize}
\begin{itemize}
\item {Utilização:Bras}
\end{itemize}
Comichão.
Cócegas.
Frieira.
(Do tupi \textunderscore juçara\textunderscore )
\section{Iucagir}
\begin{itemize}
\item {Grp. gram.:m.}
\end{itemize}
Língua hyperbórea, agglutinativa.
\section{Iuracares}
\begin{itemize}
\item {Grp. gram.:m. pl.}
\end{itemize}
Índios americanos, disseminados nas florestas da Bolívia.
\section{Iuraco}
\begin{itemize}
\item {Grp. gram.:m.}
\end{itemize}
Língua uralo-altaica, do grupo samoiedo.
\section{Iva}
\begin{itemize}
\item {Grp. gram.:f.}
\end{itemize}
Planta labiada, espécie de genipi, (\textunderscore ptarmica moschata\textunderscore , De-Candolle, ou \textunderscore ajuga iva\textunderscore , Schreb.).
\section{Ivantigi}
\begin{itemize}
\item {Grp. gram.:m.}
\end{itemize}
Árvore tiliácea do Brasil.
\section{Iverapeme}
\begin{itemize}
\item {Grp. gram.:m.}
\end{itemize}
\begin{itemize}
\item {Utilização:Bras}
\end{itemize}
Maça, com que os indígenas matavam os prisioneiros. Cf. Gonç. Dias, \textunderscore Poesias\textunderscore , 17 e 49.
\section{Ivurarema}
\begin{itemize}
\item {Grp. gram.:f.}
\end{itemize}
\begin{itemize}
\item {Utilização:Bras}
\end{itemize}
O mesmo que \textunderscore tapiá\textunderscore .
\section{Ixantho}
\begin{itemize}
\item {Grp. gram.:m.}
\end{itemize}
\begin{itemize}
\item {Proveniência:(Do gr. \textunderscore ixos\textunderscore  + \textunderscore anthos\textunderscore )}
\end{itemize}
Gênero de plantas gencianáceas.
\section{Ixanto}
\begin{itemize}
\item {fónica:csi}
\end{itemize}
\begin{itemize}
\item {Grp. gram.:m.}
\end{itemize}
\begin{itemize}
\item {Proveniência:(Do gr. \textunderscore ixos\textunderscore  + \textunderscore anthos\textunderscore )}
\end{itemize}
Gênero de plantas gencianáceas.
\section{Ixe!}
\begin{itemize}
\item {Grp. gram.:interj.}
\end{itemize}
\begin{itemize}
\item {Utilização:Bras}
\end{itemize}
(Serve para mostrar ironia)
\section{Íxia}
\begin{itemize}
\item {fónica:csi}
\end{itemize}
\begin{itemize}
\item {Grp. gram.:f.}
\end{itemize}
\begin{itemize}
\item {Proveniência:(Do gr. \textunderscore ixo\textunderscore , collar)}
\end{itemize}
Gênero de plantas bulbosas, cultivadas em jardins.
\section{Ixiolena}
\begin{itemize}
\item {fónica:csi}
\end{itemize}
\begin{itemize}
\item {Grp. gram.:f.}
\end{itemize}
\begin{itemize}
\item {Proveniência:(Do gr. \textunderscore ixioeis\textunderscore )}
\end{itemize}
Gênero de plantas australianas.
\section{Ixódidas}
\begin{itemize}
\item {fónica:csó}
\end{itemize}
\begin{itemize}
\item {Grp. gram.:m. pl.}
\end{itemize}
\begin{itemize}
\item {Utilização:Zool.}
\end{itemize}
\begin{itemize}
\item {Proveniência:(De \textunderscore ixodo\textunderscore )}
\end{itemize}
Família de animaes aracnídeos.
\section{Ixodo}
\begin{itemize}
\item {fónica:cso}
\end{itemize}
\begin{itemize}
\item {Grp. gram.:m.}
\end{itemize}
\begin{itemize}
\item {Utilização:Zool.}
\end{itemize}
\begin{itemize}
\item {Proveniência:(Do gr. \textunderscore ixodes\textunderscore )}
\end{itemize}
Gênero de arachnídeos.
\section{Ixófago}
\begin{itemize}
\item {fónica:cso}
\end{itemize}
\begin{itemize}
\item {Grp. gram.:adj.}
\end{itemize}
\begin{itemize}
\item {Proveniência:(Do gr. \textunderscore ixos\textunderscore  + \textunderscore phagein\textunderscore )}
\end{itemize}
Que come visco.
\section{Ixóphago}
\begin{itemize}
\item {Grp. gram.:adj.}
\end{itemize}
\begin{itemize}
\item {Proveniência:(Do gr. \textunderscore ixos\textunderscore  + \textunderscore phagein\textunderscore )}
\end{itemize}
Que come visco.
\section{Ixora}
\begin{itemize}
\item {fónica:cso}
\end{itemize}
\begin{itemize}
\item {Grp. gram.:f.}
\end{itemize}
Gênero de plantas, cujas flôres aromáticas os Índios do Malabar offerecem a um ídolo que chamam \textunderscore Ixora\textunderscore , ou antes \textunderscore Ixuara\textunderscore .
\section{Ixoscopia}
\begin{itemize}
\item {fónica:csos}
\end{itemize}
\begin{itemize}
\item {Grp. gram.:f.}
\end{itemize}
O mesmo que \textunderscore radioscopia\textunderscore . Cf. Verg. Machado, \textunderscore Raios X\textunderscore .
\section{Iza}
\begin{itemize}
\item {Grp. gram.:f.}
\end{itemize}
Árvore santhomense, de raíz medicinal.
\section{Izal}
\begin{itemize}
\item {Grp. gram.:m.}
\end{itemize}
Substância antiséptica, do gênero da creolina.
\section{Izaquente}
\begin{itemize}
\item {Grp. gram.:f.}
\end{itemize}
O mesmo que \textunderscore iza\textunderscore .
\section{Izar}
\begin{itemize}
\item {Grp. gram.:m.}
\end{itemize}
Instrumento de caça, usado entre as cabilas da Argélia.
\section{Izuqueiro}
\begin{itemize}
\item {Grp. gram.:adj.}
\end{itemize}
\begin{itemize}
\item {Utilização:Prov.}
\end{itemize}
\begin{itemize}
\item {Utilização:beir.}
\end{itemize}
Diz-se do archote, que, por estar húmido, não arde. (Colhido em Lamego)
\section{Ienissei}
\begin{itemize}
\item {Grp. gram.:m.}
\end{itemize}
\end{document}