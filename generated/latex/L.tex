
\begin{itemize}
\item {Proveniência: }
\end{itemize}\documentclass{article}
\usepackage[portuguese]{babel}
\title{L}
\begin{document}
Gênero de plantas myrtáceas.
\section{Laboratorial}
\begin{itemize}
\item {Grp. gram.:adj.}
\end{itemize}
Relativo a laboratório:«\textunderscore todos os meios laboratoriaes...\textunderscore »R. Jorge, na \textunderscore Luta\textunderscore , de 6-VI-913.
\section{Lágrima-de-sangue}
\begin{itemize}
\item {Grp. gram.:f.}
\end{itemize}
\begin{itemize}
\item {Utilização:Bot.}
\end{itemize}
Planta ranunculácea, (\textunderscore baeticus\textunderscore , Coss.).
\section{Lâmpado}
\begin{itemize}
\item {Grp. gram.:m.}
\end{itemize}
\begin{itemize}
\item {Utilização:Ant.}
\end{itemize}
O mesmo que \textunderscore relâmpago\textunderscore .
\section{Lanedo}
\begin{itemize}
\item {fónica:nê}
\end{itemize}
\begin{itemize}
\item {Grp. gram.:m.}
\end{itemize}
\begin{itemize}
\item {Utilização:T. de Odemira}
\end{itemize}
Algazarra.
Desordem.
(Cp. \textunderscore laneiro\textunderscore ^1)
\section{Laranjim}
\begin{itemize}
\item {Grp. gram.:m.}
\end{itemize}
\begin{itemize}
\item {Proveniência:(De \textunderscore laranja\textunderscore )}
\end{itemize}
Amêndoa confeccionada, cujo núcleo é constituído por um pedaço de casca de laranja.
\section{Legitimónio}
\begin{itemize}
\item {Grp. gram.:m.}
\end{itemize}
\begin{itemize}
\item {Utilização:Des.}
\end{itemize}
O mesmo que \textunderscore património\textunderscore . Cf. Prestes, \textunderscore Autos\textunderscore , 144, (ed. de 1871).
(Formação arbitrária, de \textunderscore legítima\textunderscore  e da terminação de \textunderscore património\textunderscore )
\section{Leprose}
\begin{itemize}
\item {Grp. gram.:f.}
\end{itemize}
\begin{itemize}
\item {Utilização:Med.}
\end{itemize}
Nome, que se dá á lepra, como moléstia específica.
\section{Linotipia}
\begin{itemize}
\item {Grp. gram.:f.}
\end{itemize}
\begin{itemize}
\item {Proveniência:(De \textunderscore linótipo\textunderscore )}
\end{itemize}
Arte de linotipar.
\section{Linotypia}
\begin{itemize}
\item {Grp. gram.:f.}
\end{itemize}
\begin{itemize}
\item {Proveniência:(De \textunderscore linótypo\textunderscore )}
\end{itemize}
Arte de linotypar.
\section{Louçanear}
\begin{itemize}
\item {Grp. gram.:v. i.}
\end{itemize}
Mostrar-se loução; garrir:«\textunderscore louçanear por bailes.\textunderscore »Camillo, \textunderscore Bruxa\textunderscore , 148, (ed. 1891).
\section{Lucas}
\begin{itemize}
\item {Grp. gram.:m.}
\end{itemize}
\begin{itemize}
\item {Utilização:Pop.}
\end{itemize}
Palerma; o mesmo que \textunderscore matias\textunderscore : \textunderscore não te faças lucas\textunderscore .
\section{L}
\begin{itemize}
\item {fónica:éleoulê}
\end{itemize}
\begin{itemize}
\item {Grp. gram.:m.}
\end{itemize}
\begin{itemize}
\item {Grp. gram.:Adj.}
\end{itemize}
Décima segunda letra do alphabeto português.
Que numa série occupa o duodécimo lugar.
Cincoenta, em numeração romana.
\section{La}
\begin{itemize}
\item {fónica:lâ}
\end{itemize}
\begin{itemize}
\item {Grp. gram.:f.}
\end{itemize}
\begin{itemize}
\item {Grp. gram.:Art.}
\end{itemize}
\begin{itemize}
\item {Utilização:Ant.}
\end{itemize}
\begin{itemize}
\item {Proveniência:(Do lat. \textunderscore illa\textunderscore )}
\end{itemize}
Pronome, que se pospõe aos verbos terminados em \textunderscore r\textunderscore , \textunderscore s\textunderscore  ou \textunderscore z\textunderscore : \textunderscore queremos louvá-la\textunderscore ; \textunderscore vós viste-la chegar\textunderscore ; \textunderscore êlle fê-la melhor\textunderscore .
O mesmo que \textunderscore a\textunderscore ^2, (ainda hoje us. na Madeira e em algumas das nossas povoações raianas) Na linguagem popular, também se ouve, em vez de \textunderscore a\textunderscore , depois de algumas palavras terminadas em \textunderscore s\textunderscore , á parte os verbos: \textunderscore João, mai-la filha...\textunderscore 
\section{La}
\begin{itemize}
\item {fónica:lá}
\end{itemize}
\begin{itemize}
\item {Grp. gram.:m.}
\end{itemize}
Sexta nota da moderna escala musical.
Sinal representativo desta nota.
Designação vulgar da segunda corda nos violinos, da primeira nos rabecões, da primeira nas violetas, da segunda nos violoncellos e da terceira nos contrabaixos.
(Da 1.^a sýllaba do lat. \textunderscore labi\textunderscore , aproveitada por G. de Arezzo, com as primeiras sýllabas de algumas palavras de um hymno religioso, para a formação da antiga escala musical)
\section{Lá}
\begin{itemize}
\item {Grp. gram.:adv.}
\end{itemize}
\begin{itemize}
\item {Proveniência:(Do lat. \textunderscore illac\textunderscore )}
\end{itemize}
Naquelle lugar: \textunderscore não conheço lá ninguém\textunderscore .
Entre aquella gente.
Naquelles povos.
Para aquelle lugar; áquelle lugar: \textunderscore ainda lá não fui\textunderscore .
Ao longe.
Além.
Pois.
Nesse tempo.
Contigo.
Convosco.
Afinal.--Emprega-se também expletivamente ou com redundância, para dar mais fôrça á phrase.
\section{Labaça}
\begin{itemize}
\item {Grp. gram.:f.}
\end{itemize}
\begin{itemize}
\item {Proveniência:(Do lat. hyp. \textunderscore lapathius\textunderscore , de \textunderscore lapathus\textunderscore )}
\end{itemize}
Nome de várias plantas polygóneas.
\section{Labaça}
\begin{itemize}
\item {Grp. gram.:f.}
\end{itemize}
\begin{itemize}
\item {Utilização:Prov.}
\end{itemize}
\begin{itemize}
\item {Utilização:trasm.}
\end{itemize}
\begin{itemize}
\item {Utilização:fam.}
\end{itemize}
O mesmo que \textunderscore lábia\textunderscore ^1.
\section{Labaçal}
\begin{itemize}
\item {Grp. gram.:m.}
\end{itemize}
\begin{itemize}
\item {Proveniência:(De \textunderscore labaça\textunderscore ^1)}
\end{itemize}
Terreno, onde crescem labaças.
\section{Labaceiro}
\begin{itemize}
\item {Grp. gram.:adj.}
\end{itemize}
\begin{itemize}
\item {Utilização:Prov.}
\end{itemize}
\begin{itemize}
\item {Utilização:trasm.}
\end{itemize}
\begin{itemize}
\item {Proveniência:(De \textunderscore labaça\textunderscore ^2)}
\end{itemize}
Que tem lábia.
\section{Labaceiro}
\begin{itemize}
\item {Grp. gram.:adj.}
\end{itemize}
O mesmo que \textunderscore lambaceiro\textunderscore .
\section{Labaçol}
\begin{itemize}
\item {Grp. gram.:m.}
\end{itemize}
\begin{itemize}
\item {Proveniência:(Do rad. de \textunderscore labaça\textunderscore ^1)}
\end{itemize}
Variedade de labaça.
\section{Labadismo}
\begin{itemize}
\item {Grp. gram.:m.}
\end{itemize}
\begin{itemize}
\item {Proveniência:(De \textunderscore Labadie\textunderscore , n. p.)}
\end{itemize}
Doutrina dos theólogos franceses, que preconizavam a abolição da jerarchia ecclesiástica, a communidade dos bens e a faculdade de supprir a Bíblia pela inspiração individual.
\section{Labadista}
\begin{itemize}
\item {Grp. gram.:m.}
\end{itemize}
Partidário do labadismo.
\section{Labarda}
\begin{itemize}
\item {Grp. gram.:f.}
\end{itemize}
\begin{itemize}
\item {Utilização:Açor}
\end{itemize}
Pequeno peixe, de listras verdes, que também se chama \textunderscore rainha\textunderscore , por formar cardume com os peixes reis.
\section{Labarda}
\begin{itemize}
\item {Grp. gram.:f.}
\end{itemize}
\begin{itemize}
\item {Utilização:Ant.}
\end{itemize}
O mesmo que \textunderscore alabarda\textunderscore . Cf. Pant. de Aveiro, \textunderscore Itiner.\textunderscore , 281 (2.^a ed.)
\section{Labaréda}
\begin{itemize}
\item {Grp. gram.:f.}
\end{itemize}
\begin{itemize}
\item {Utilização:Ext.}
\end{itemize}
\begin{itemize}
\item {Utilização:Fig.}
\end{itemize}
\begin{itemize}
\item {Grp. gram.:M.}
\end{itemize}
\begin{itemize}
\item {Utilização:Pop.}
\end{itemize}
\begin{itemize}
\item {Proveniência:(Do cast. \textunderscore llamarada\textunderscore )}
\end{itemize}
Grande língua de fogo, grande chamma.
Ardor: \textunderscore as labaredas do amor\textunderscore .
Intensidade.
Homem azafamado, ferefolha.
\section{Labarêda}
\begin{itemize}
\item {Grp. gram.:f.}
\end{itemize}
\begin{itemize}
\item {Utilização:Ext.}
\end{itemize}
\begin{itemize}
\item {Utilização:Fig.}
\end{itemize}
\begin{itemize}
\item {Grp. gram.:M.}
\end{itemize}
\begin{itemize}
\item {Utilização:Pop.}
\end{itemize}
\begin{itemize}
\item {Proveniência:(Do cast. \textunderscore llamarada\textunderscore )}
\end{itemize}
Grande língua de fogo, grande chamma.
Ardor: \textunderscore as labaredas do amor\textunderscore .
Intensidade.
Homem azafamado, ferefolha.
\section{Lábaro}
\begin{itemize}
\item {Grp. gram.:m.}
\end{itemize}
\begin{itemize}
\item {Utilização:Ext.}
\end{itemize}
\begin{itemize}
\item {Proveniência:(Lat. \textunderscore labarum\textunderscore )}
\end{itemize}
Estandarte dos exércitos romanos, no tempo do Império.
Estandarte.
\section{Labátia}
\begin{itemize}
\item {Grp. gram.:f.}
\end{itemize}
\begin{itemize}
\item {Proveniência:(De \textunderscore Labat\textunderscore , n. p.)}
\end{itemize}
Árvore sapotácea da América.
\section{Labbo}
\begin{itemize}
\item {Grp. gram.:m.}
\end{itemize}
Ave palmípede, o mesmo que \textunderscore estercorário\textunderscore .
\section{Labdacismo}
\begin{itemize}
\item {Grp. gram.:m.}
\end{itemize}
O mesmo que \textunderscore lambdacismo\textunderscore .
\section{Lábdano}
\begin{itemize}
\item {Grp. gram.:m.}
\end{itemize}
\begin{itemize}
\item {Proveniência:(Do ingl. \textunderscore labdanum\textunderscore )}
\end{itemize}
Resina de algumas plantas cystíneas.
\section{Labefactação}
\begin{itemize}
\item {Grp. gram.:f.}
\end{itemize}
\begin{itemize}
\item {Proveniência:(Lat. \textunderscore labefactatio\textunderscore )}
\end{itemize}
Acto de labefactar.
\section{Labefactado}
\begin{itemize}
\item {Grp. gram.:adj.}
\end{itemize}
\begin{itemize}
\item {Utilização:Des.}
\end{itemize}
\begin{itemize}
\item {Proveniência:(De \textunderscore labefactar\textunderscore )}
\end{itemize}
Manchado.
Vicioso.
Arruinado.
\section{Labefactar}
\begin{itemize}
\item {Grp. gram.:v. t.}
\end{itemize}
\begin{itemize}
\item {Utilização:Des.}
\end{itemize}
\begin{itemize}
\item {Proveniência:(Lat. \textunderscore labefactare\textunderscore )}
\end{itemize}
Arruinar, destruir.
\section{Labego}
\begin{itemize}
\item {Grp. gram.:m.}
\end{itemize}
Charrua, que lavra fundo e é puxada por mais de uma junta de bois.
(Por \textunderscore lavrego\textunderscore , do rad. de \textunderscore lavrar\textunderscore )
\section{Labelado}
\begin{itemize}
\item {Grp. gram.:adj.}
\end{itemize}
\begin{itemize}
\item {Proveniência:(De \textunderscore labello\textunderscore )}
\end{itemize}
Que tem fórma de lábio.
\section{Labellado}
\begin{itemize}
\item {Grp. gram.:adj.}
\end{itemize}
\begin{itemize}
\item {Proveniência:(De \textunderscore labello\textunderscore )}
\end{itemize}
Que tem fórma de lábio.
\section{Labello}
\begin{itemize}
\item {Grp. gram.:m.}
\end{itemize}
\begin{itemize}
\item {Utilização:Bot.}
\end{itemize}
\begin{itemize}
\item {Proveniência:(Lat. \textunderscore labellum\textunderscore )}
\end{itemize}
Pequeno lábio.
Segmento inferior de um invólucro floral.
\section{Labelo}
\begin{itemize}
\item {Grp. gram.:m.}
\end{itemize}
\begin{itemize}
\item {Utilização:Bot.}
\end{itemize}
\begin{itemize}
\item {Proveniência:(Lat. \textunderscore labellum\textunderscore )}
\end{itemize}
Pequeno lábio.
Segmento inferior de um invólucro floral.
\section{Labéo}
\begin{itemize}
\item {Grp. gram.:f.}
\end{itemize}
\begin{itemize}
\item {Proveniência:(Do lat. hyp. \textunderscore labellum\textunderscore , de \textunderscore labes\textunderscore )}
\end{itemize}
Mancha na reputação; desdoiro; deshonra.
\section{Labéu}
\begin{itemize}
\item {Grp. gram.:f.}
\end{itemize}
\begin{itemize}
\item {Proveniência:(Do lat. hyp. \textunderscore labellum\textunderscore , de \textunderscore labes\textunderscore )}
\end{itemize}
Mancha na reputação; desdoiro; deshonra.
\section{Lábia}
\begin{itemize}
\item {Grp. gram.:f.}
\end{itemize}
\begin{itemize}
\item {Utilização:Fam.}
\end{itemize}
Astúcia; ronha; manha; solércia.
Falas mellífluas ou adocicadas, para captar agrado ou favores.
(Parece relacionar-se com o vocabulário dos ciganos em França)
\section{Lábia}
\begin{itemize}
\item {Grp. gram.:f.}
\end{itemize}
\begin{itemize}
\item {Utilização:T. de Vianna}
\end{itemize}
O mesmo que \textunderscore relicário\textunderscore .
\section{Labiadas}
\begin{itemize}
\item {Grp. gram.:f. pl.}
\end{itemize}
\begin{itemize}
\item {Utilização:Bot.}
\end{itemize}
\begin{itemize}
\item {Proveniência:(De \textunderscore labiado\textunderscore )}
\end{itemize}
Família de plantas, de corollas monopétalas.
\section{Labiado}
\begin{itemize}
\item {Grp. gram.:adj.}
\end{itemize}
\begin{itemize}
\item {Utilização:Hist. Nat.}
\end{itemize}
Que tem fórma de lábio; que é formado de lábios.
\section{Labiados}
\begin{itemize}
\item {Grp. gram.:m. pl.}
\end{itemize}
\begin{itemize}
\item {Utilização:Zool.}
\end{itemize}
\begin{itemize}
\item {Proveniência:(De \textunderscore labiado\textunderscore )}
\end{itemize}
Animaes, de lábios alongados, grossos ou de côr differente da do resto do corpo.
\section{Labial}
\begin{itemize}
\item {Grp. gram.:adj.}
\end{itemize}
\begin{itemize}
\item {Grp. gram.:F.}
\end{itemize}
\begin{itemize}
\item {Proveniência:(Lat. \textunderscore labialis\textunderscore )}
\end{itemize}
Relativo aos lábios.
Que está nos lábios.
Que se pronuncía com os lábios: \textunderscore letras labiaes\textunderscore .
Letra labial.
\section{Labialização}
\begin{itemize}
\item {Grp. gram.:f.}
\end{itemize}
Acto de \textunderscore labializar\textunderscore .
\section{Labializar}
\begin{itemize}
\item {Grp. gram.:v.}
\end{itemize}
\begin{itemize}
\item {Utilização:t. Gram.}
\end{itemize}
Tornar labial; pronunciar com os lábios.
\section{Labiatifloras}
\begin{itemize}
\item {Grp. gram.:f. pl.}
\end{itemize}
\begin{itemize}
\item {Utilização:Bot.}
\end{itemize}
\begin{itemize}
\item {Proveniência:(Do lat. \textunderscore labiatus\textunderscore  + \textunderscore flos\textunderscore )}
\end{itemize}
Grupo de synanthéreas, na classificação de De-Candolle.
\section{Lábido}
\begin{itemize}
\item {Grp. gram.:m.}
\end{itemize}
\begin{itemize}
\item {Proveniência:(Do gr. \textunderscore labis\textunderscore , \textunderscore labidos\textunderscore )}
\end{itemize}
Gênero de insectos hymenópteros, da tríbo das formigas.
\section{Labiduro}
\begin{itemize}
\item {Grp. gram.:adj.}
\end{itemize}
\begin{itemize}
\item {Utilização:Zool.}
\end{itemize}
\begin{itemize}
\item {Proveniência:(Do gr. \textunderscore labis\textunderscore , \textunderscore labidos\textunderscore  + \textunderscore ouron\textunderscore )}
\end{itemize}
Diz-se dos animaes, cuja cauda termina em fórma de tenaz.
\section{Lábil}
\begin{itemize}
\item {Grp. gram.:adj.}
\end{itemize}
\begin{itemize}
\item {Utilização:Poét.}
\end{itemize}
\begin{itemize}
\item {Proveniência:(Lat. \textunderscore labilis\textunderscore )}
\end{itemize}
Que escorrega facilmente, fraco; transitório.
\section{Labímetro}
\begin{itemize}
\item {Grp. gram.:m.}
\end{itemize}
Instrumento cirúrgico, que se adapta aos braços dos fórceps, para indicar a abertura delles, bem como a das respectivas conchas.
\section{Lábio}
\begin{itemize}
\item {Grp. gram.:m.}
\end{itemize}
\begin{itemize}
\item {Utilização:Fig.}
\end{itemize}
\begin{itemize}
\item {Proveniência:(Lat. \textunderscore labium\textunderscore )}
\end{itemize}
Parte exterior e vermelha, que fórma o contôrno da bôca.
Parte ou objecto semelhante ao lábio.
Bôca; lóbulo.
Linguagem.
\section{Lábio-dental}
\begin{itemize}
\item {Grp. gram.:adj.}
\end{itemize}
\begin{itemize}
\item {Utilização:Gram.}
\end{itemize}
O mesmo ou melhor que \textunderscore dento-labial\textunderscore .
\section{Labio-nasal}
\begin{itemize}
\item {Grp. gram.:adj.}
\end{itemize}
\begin{itemize}
\item {Utilização:Gram.}
\end{itemize}
Diz-se da letra \textunderscore m\textunderscore , porque se pronuncía com os lábios e o nariz.
\section{Labioso}
\begin{itemize}
\item {Grp. gram.:adj.}
\end{itemize}
Que tem grandes lábios; beiçudo.
\section{Labioso}
\begin{itemize}
\item {Grp. gram.:adj.}
\end{itemize}
Que tem lábia; em que que há lábia.
\section{Labirintado}
\begin{itemize}
\item {Grp. gram.:adj.}
\end{itemize}
\begin{itemize}
\item {Utilização:Bot.}
\end{itemize}
Diz-se das fôlhas, que apresentam línhas entremeadas, á maneira de labirinto.
\section{Labirintar}
\begin{itemize}
\item {Grp. gram.:v. i.}
\end{itemize}
Andar num labirintho.
Enredar-se, confundir-se:«\textunderscore ...no qual labirinthava durante dias um mundo de loucos\textunderscore ». D. Ant. da Costa, \textunderscore Três Mundos\textunderscore , 179.
\section{Labiríntico}
\begin{itemize}
\item {Grp. gram.:adj.}
\end{itemize}
Relativo a labirinto; confuso; intricado.
\section{Labirintiforme}
\begin{itemize}
\item {Grp. gram.:adj.}
\end{itemize}
\begin{itemize}
\item {Proveniência:(De \textunderscore labirintho\textunderscore  + \textunderscore forma\textunderscore )}
\end{itemize}
Que tem fórma de labirinto.
\section{Labirinto}
\begin{itemize}
\item {Grp. gram.:m.}
\end{itemize}
\begin{itemize}
\item {Utilização:Fig.}
\end{itemize}
\begin{itemize}
\item {Utilização:Pop.}
\end{itemize}
\begin{itemize}
\item {Utilização:Bras. do N}
\end{itemize}
\begin{itemize}
\item {Proveniência:(Gr. \textunderscore laburinthos\textunderscore )}
\end{itemize}
Edifício, construido com taes divisões e recantos, que é dificílimo achar-lhe a saída.
Jardim ou plantação em fórma de labirinto.
Questão complicada, obscura.
Confusão.
Parte interna do ouvido.
Alarido; tumulto.
Certo trabalho de agulha, também chamado \textunderscore crivo\textunderscore .
\section{Labirintodonte}
\begin{itemize}
\item {Grp. gram.:m.}
\end{itemize}
\begin{itemize}
\item {Proveniência:(Do gr. \textunderscore laburinthos\textunderscore  + \textunderscore odous\textunderscore )}
\end{itemize}
Reptil anfíbio, fóssil, da série paleozoica.
\section{Labisóme}
\begin{itemize}
\item {Grp. gram.:m.}
\end{itemize}
\begin{itemize}
\item {Utilização:ant.}
\end{itemize}
\begin{itemize}
\item {Utilização:Pop.}
\end{itemize}
O mesmo que \textunderscore lobishomem\textunderscore . Cf. \textunderscore Arte de Furtar\textunderscore , c. XXVII.
\section{Labita}
\begin{itemize}
\item {Grp. gram.:f.}
\end{itemize}
\begin{itemize}
\item {Utilização:Gír.}
\end{itemize}
O mesmo que \textunderscore casaca\textunderscore ^1.
(V. \textunderscore levita\textunderscore ^2)
\section{Labo}
\begin{itemize}
\item {Grp. gram.:m.}
\end{itemize}
Ave palmípede, o mesmo que \textunderscore estercorário\textunderscore .
\section{Labor}
\begin{itemize}
\item {Grp. gram.:m.}
\end{itemize}
\begin{itemize}
\item {Proveniência:(Lat. \textunderscore labor\textunderscore )}
\end{itemize}
Lavor; trabalho, faina.
\section{Laboração}
\begin{itemize}
\item {Grp. gram.:f.}
\end{itemize}
Acto ou effeito de laborar.
\section{Laborador}
\begin{itemize}
\item {Grp. gram.:m.}
\end{itemize}
\begin{itemize}
\item {Proveniência:(Lat. \textunderscore laborator\textunderscore )}
\end{itemize}
Aquelle que labora.
\section{Laborar}
\begin{itemize}
\item {Grp. gram.:v. i.}
\end{itemize}
\begin{itemize}
\item {Proveniência:(Lat. \textunderscore laborare\textunderscore )}
\end{itemize}
Trabalhar.
Empregar-se.
Fazer a cultura da terra.
Lidar; manobrar.
\section{Laboratório}
\begin{itemize}
\item {Grp. gram.:m.}
\end{itemize}
\begin{itemize}
\item {Utilização:Fig.}
\end{itemize}
\begin{itemize}
\item {Proveniência:(De \textunderscore laborar\textunderscore )}
\end{itemize}
Lugar, onde se fazem experiências biológicas ou operações chímicas ou pharmacêuticas.
Lugar de grandes operações ou transformações.
\section{Labórdia}
\begin{itemize}
\item {Grp. gram.:f.}
\end{itemize}
\begin{itemize}
\item {Proveniência:(De \textunderscore Laborde\textunderscore , n. p.)}
\end{itemize}
Gênero de plantas loganiáceas.
\section{Laboreira}
\begin{itemize}
\item {Grp. gram.:f.}
\end{itemize}
\begin{itemize}
\item {Proveniência:(Lat. \textunderscore leporaria\textunderscore )}
\end{itemize}
Planta da serra de Sintra.
\section{Laborinha}
\begin{itemize}
\item {Grp. gram.:f.}
\end{itemize}
Espécie de jôgo popular.
\section{Laborinho}
\begin{itemize}
\item {Grp. gram.:m.}
\end{itemize}
\begin{itemize}
\item {Utilização:Prov.}
\end{itemize}
\begin{itemize}
\item {Proveniência:(Lat. \textunderscore leporinus.\textunderscore  Cp. \textunderscore laboreira\textunderscore )}
\end{itemize}
Pastagem, constítuída por plantas, do gênero \textunderscore festuca ovina\textunderscore .
\section{Laboriosamente}
\begin{itemize}
\item {Grp. gram.:adv.}
\end{itemize}
De modo laborioso.
Com trabalho; á custa de fadigas.
\section{Laboriosidade}
\begin{itemize}
\item {Grp. gram.:f.}
\end{itemize}
Qualidade de laborioso.
\section{Laborioso}
\begin{itemize}
\item {Grp. gram.:adj.}
\end{itemize}
\begin{itemize}
\item {Proveniência:(Lat. \textunderscore laboriosus\textunderscore )}
\end{itemize}
Que labora, que trabalha.
Trabalhoso: \textunderscore vida laboriosa\textunderscore .
Trabalhador.
Activo; incansável: \textunderscore homem laborioso\textunderscore .
\section{Laborjeiro}
\begin{itemize}
\item {Grp. gram.:m.}
\end{itemize}
Casta de uva.
(Relaciona-se com \textunderscore Labrujeira\textunderscore , n. p.?)
\section{Labortano}
\begin{itemize}
\item {Grp. gram.:m.}
\end{itemize}
Um dos dialectos vasconços, na França meridional.
\section{Labrador}
\begin{itemize}
\item {Grp. gram.:m.}
\end{itemize}
O mesmo que \textunderscore labradorite\textunderscore .
\section{Labradórico}
\begin{itemize}
\item {Grp. gram.:adj.}
\end{itemize}
\begin{itemize}
\item {Utilização:Miner.}
\end{itemize}
Relativo ao labrador, rocha.
\section{Labradorita}
\begin{itemize}
\item {Grp. gram.:f.}
\end{itemize}
Feldspatho, de reflexos opalinos, que se encontra na costa do Labrador.
\section{Labradorite}
\begin{itemize}
\item {Grp. gram.:f.}
\end{itemize}
Feldspatho, de reflexos opalinos, que se encontra na costa do Labrador.
\section{Labrear}
\begin{itemize}
\item {Grp. gram.:v. t.}
\end{itemize}
\begin{itemize}
\item {Utilização:Bras. do N}
\end{itemize}
Sujar, emporcalhar.
\section{Labregamente}
\begin{itemize}
\item {Grp. gram.:adv.}
\end{itemize}
Com modos de labrego.
Rusticamente.
\section{Labrego}
\begin{itemize}
\item {fónica:brê}
\end{itemize}
\begin{itemize}
\item {Grp. gram.:m.  e  adj.}
\end{itemize}
\begin{itemize}
\item {Utilização:Fig.}
\end{itemize}
\begin{itemize}
\item {Utilização:Açor}
\end{itemize}
Homem rústico.
Aldeão; camponês.
Homem sem educação.
Espécie de arado, que tem um varredoiro, para limpar da terra as raizes.
O mesmo ou melhor que \textunderscore labego\textunderscore .
Lobishomem; diabo.
(Por \textunderscore lavrêgo\textunderscore , de \textunderscore lavrar\textunderscore , que na Beira e noutros pontos se pronuncía \textunderscore labrar\textunderscore )
\section{Labrego}
\begin{itemize}
\item {fónica:brê}
\end{itemize}
\begin{itemize}
\item {Grp. gram.:m.}
\end{itemize}
Charrua, que lavra fundo e é puxada por mais de uma junta de bois.
(Por \textunderscore lavrego\textunderscore , do rad. de \textunderscore lavrar\textunderscore )
\section{Labrestada}
\begin{itemize}
\item {Grp. gram.:f.}
\end{itemize}
Acto de labrestar.
Vergastada.
\section{Labrestar}
\begin{itemize}
\item {Grp. gram.:v. t.}
\end{itemize}
\begin{itemize}
\item {Utilização:Prov.}
\end{itemize}
\begin{itemize}
\item {Utilização:trasm.}
\end{itemize}
Roubar.
\section{Labresto}
\begin{itemize}
\item {fónica:brês}
\end{itemize}
\begin{itemize}
\item {Grp. gram.:m.}
\end{itemize}
(V.lampsana)
\section{Labro}
\begin{itemize}
\item {Grp. gram.:m.}
\end{itemize}
\begin{itemize}
\item {Proveniência:(Lat. \textunderscore labrum\textunderscore )}
\end{itemize}
Lábio superior dos mammíferos.
Extremidade do bico dos insectos.
Extremidade exterior do uma concha univalve.
Gênero de peixes, que têm os lábios carnosos.
\section{Labroides}
\begin{itemize}
\item {Grp. gram.:m. pl.}
\end{itemize}
Família de peixes, que tem por typo o labro.
\section{Labroso}
\begin{itemize}
\item {Grp. gram.:adj.}
\end{itemize}
Diz-se da concha univalve, cuja extremidade externa é grossa e revirada.
(Lat, \textunderscore labrosus\textunderscore )
\section{Labrosta}
\begin{itemize}
\item {Grp. gram.:m.  e  adj.}
\end{itemize}
Labrego, rústico.
(Por \textunderscore lavrosta\textunderscore , de \textunderscore lavrar\textunderscore . Cp. \textunderscore labrego\textunderscore ^1)
\section{Labroste}
\begin{itemize}
\item {Grp. gram.:m.  e  adj.}
\end{itemize}
O mesmo que \textunderscore labrosta\textunderscore .
\section{Labruge}
\begin{itemize}
\item {Grp. gram.:m.}
\end{itemize}
\begin{itemize}
\item {Utilização:Bras}
\end{itemize}
Espécie de loireiro.
\section{Labrusca}
\begin{itemize}
\item {Grp. gram.:f.}
\end{itemize}
\begin{itemize}
\item {Proveniência:(Lat. \textunderscore labrusca\textunderscore )}
\end{itemize}
Variedade de uva preta.
\section{Labrusco}
\begin{itemize}
\item {Grp. gram.:adj.}
\end{itemize}
\begin{itemize}
\item {Utilização:T. de Turquel}
\end{itemize}
\begin{itemize}
\item {Grp. gram.:M.}
\end{itemize}
\begin{itemize}
\item {Proveniência:(Lat. \textunderscore labruscus\textunderscore )}
\end{itemize}
Imbecil; grosseiro.
Agreste.
Inculto.
Sujo, lambuzado.
Casta de uva branca do Cartaxo.
Casta de uva preta, o mesmo que \textunderscore labrusca.\textunderscore 
\section{Labugante}
\begin{itemize}
\item {Grp. gram.:m.}
\end{itemize}
\begin{itemize}
\item {Grp. gram.:m.}
\end{itemize}
O mesmo que \textunderscore lavagante\textunderscore .
Crustáceo decápode, marítimo, um pouco mais pequeno que a lagosta e munido de duas fortes torqueses nos braços (\textunderscore homarus vulgaris\textunderscore ).
(Cp. cast. \textunderscore lobogante\textunderscore )
\section{Laburno}
\begin{itemize}
\item {Grp. gram.:m.}
\end{itemize}
\begin{itemize}
\item {Proveniência:(Lat. \textunderscore laburnum\textunderscore )}
\end{itemize}
Planta leguminosa.
\section{Labuta}
\begin{itemize}
\item {Grp. gram.:f.}
\end{itemize}
Acto ou effeito de labutar.
\section{Labutação}
\begin{itemize}
\item {Grp. gram.:f.}
\end{itemize}
Acto ou effeito de labutar.
\section{Labutar}
\begin{itemize}
\item {Grp. gram.:v. i.}
\end{itemize}
\begin{itemize}
\item {Utilização:Fig.}
\end{itemize}
\begin{itemize}
\item {Proveniência:(Do rad. de \textunderscore labor\textunderscore ?)}
\end{itemize}
Trabalhar insistentemente.
Lidar.
Funccionar activamente.
Empenhar-se, esforçar-se.
\section{Labuzar}
\begin{itemize}
\item {Grp. gram.:v. t.}
\end{itemize}
O mesmo que \textunderscore lambuzar\textunderscore .
\section{Labyrinthado}
\begin{itemize}
\item {Grp. gram.:adj.}
\end{itemize}
\begin{itemize}
\item {Utilização:Bot.}
\end{itemize}
Diz-se das fôlhas, que apresentam línhas entremeadas, á maneira de labyrintho.
\section{Labyrinthar}
\begin{itemize}
\item {Grp. gram.:v. i.}
\end{itemize}
Andar num labyrintho.
Enredar-se, confundir-se:«\textunderscore ...no qual labyrinthava durante dias um mundo de loucos\textunderscore ». D. Ant. da Costa, \textunderscore Três Mundos\textunderscore , 179.
\section{Labyrínthico}
\begin{itemize}
\item {Grp. gram.:adj.}
\end{itemize}
Relativo a labyrintho; confuso; intricado.
\section{Labyrinthiforme}
\begin{itemize}
\item {Grp. gram.:adj.}
\end{itemize}
\begin{itemize}
\item {Proveniência:(De \textunderscore labyrintho\textunderscore  + \textunderscore forma\textunderscore )}
\end{itemize}
Que tem fórma de labyrintho.
\section{Labyrintho}
\begin{itemize}
\item {Grp. gram.:m.}
\end{itemize}
\begin{itemize}
\item {Utilização:Fig.}
\end{itemize}
\begin{itemize}
\item {Utilização:Pop.}
\end{itemize}
\begin{itemize}
\item {Utilização:Bras. do N}
\end{itemize}
\begin{itemize}
\item {Proveniência:(Gr. \textunderscore laburinthos\textunderscore )}
\end{itemize}
Edifício, construido com taes divisões e recantos, que é dificílimo achar-lhe a saída.
Jardim ou plantação em fórma de labyrintho.
Questão complicada, obscura.
Confusão.
Parte interna do ouvido.
Alarido; tumulto.
Certo trabalho de agulha, também chamado \textunderscore crivo\textunderscore .
\section{Labyrinthodonte}
\begin{itemize}
\item {Grp. gram.:m.}
\end{itemize}
\begin{itemize}
\item {Proveniência:(Do gr. \textunderscore laburinthos\textunderscore  + \textunderscore odous\textunderscore )}
\end{itemize}
Reptil amphíbio, fóssil, da série paleozoica.
\section{Laca}
\begin{itemize}
\item {Grp. gram.:f.}
\end{itemize}
Resina ou fécula vermelha, extrahida das sementes de algumas plantas leguminosas.
Verniz da China, negro ou vermelho.
Tinta da fécula do pau brasil, que, misturada com cochonilha, tem applicação na pintura.
(Ár. \textunderscore lac\textunderscore )
\section{Laça}
\begin{itemize}
\item {Grp. gram.:f.}
\end{itemize}
\begin{itemize}
\item {Utilização:Prov.}
\end{itemize}
\begin{itemize}
\item {Utilização:minh.}
\end{itemize}
Laçada ou aselha.
(Cp. \textunderscore laço\textunderscore ^1)
\section{Lacacan}
\begin{itemize}
\item {Grp. gram.:m.}
\end{itemize}
Planta herbácea caboverdeana.
\section{Laçaço}
\begin{itemize}
\item {Grp. gram.:m.}
\end{itemize}
\begin{itemize}
\item {Utilização:Bras}
\end{itemize}
Pancada com o laço.
\section{Lacada}
\begin{itemize}
\item {Grp. gram.:f.}
\end{itemize}
\begin{itemize}
\item {Utilização:Prov.}
\end{itemize}
\begin{itemize}
\item {Utilização:minh.}
\end{itemize}
\begin{itemize}
\item {Proveniência:(De \textunderscore lacar\textunderscore )}
\end{itemize}
Pancada ou quéda da roda do carro em lugar fundo de estrada.
\section{Laçada}
\begin{itemize}
\item {Grp. gram.:f.}
\end{itemize}
Laço, que se desata facilmente; aselha.
\section{Laçador}
\begin{itemize}
\item {Grp. gram.:m.}
\end{itemize}
\begin{itemize}
\item {Utilização:Bras. do S}
\end{itemize}
Aquelle que prende cavallos a laço.
\section{Lacafá}
\begin{itemize}
\item {Grp. gram.:m.}
\end{itemize}
\begin{itemize}
\item {Utilização:Ant.}
\end{itemize}
Quantidade numérica, equivalente a cem mil, na ilha de Ainão. Cf. \textunderscore Peregrinação\textunderscore , XLV.
\section{Lacaia}
\begin{itemize}
\item {Grp. gram.:f.}
\end{itemize}
\begin{itemize}
\item {Utilização:Ant.}
\end{itemize}
\begin{itemize}
\item {Proveniência:(De \textunderscore lacaio\textunderscore )}
\end{itemize}
Mulher, que, em peças theatraes, representa papel de moça, criada ou aia, quási sempre finória e espertalhona.
Criada, que, fóra de casa, acompanha a ama.
\section{Lacaiada}
\begin{itemize}
\item {Grp. gram.:f.}
\end{itemize}
Acto ou dito próprio de lacaio.
Grupo de lacaios.
\section{Lacaiar}
\begin{itemize}
\item {Grp. gram.:v. t.}
\end{itemize}
\begin{itemize}
\item {Utilização:bras}
\end{itemize}
\begin{itemize}
\item {Utilização:Neol.}
\end{itemize}
Servir de lacaio a.
\section{Lacaiesco}
\begin{itemize}
\item {fónica:ês}
\end{itemize}
\begin{itemize}
\item {Grp. gram.:adj.}
\end{itemize}
Relativo a lacaio; próprio de lacaio. Cf. Castilho, \textunderscore D. Quixote\textunderscore , II, 372.
\section{Lacaio}
\begin{itemize}
\item {Grp. gram.:m.}
\end{itemize}
\begin{itemize}
\item {Utilização:Fig.}
\end{itemize}
Criado, que acompanha o amo, com libré ou sem ella.
Trintanário ou criado, que vai ao lado do cocheiro ou na traseira da sege.
Casta de uva.
Homem desprezível.
Homem servil, amouco.
(Cp. fr. \textunderscore laquais\textunderscore )
\section{Lacaio}
\begin{itemize}
\item {Grp. gram.:m.}
\end{itemize}
\begin{itemize}
\item {Utilização:Prov.}
\end{itemize}
\begin{itemize}
\item {Utilização:trasm.}
\end{itemize}
O mesmo que \textunderscore lacrau\textunderscore .
(Colhido em V. P. de Aguiar)
\section{Lacanhal}
\begin{itemize}
\item {Grp. gram.:m.}
\end{itemize}
\begin{itemize}
\item {Utilização:Prov.}
\end{itemize}
\begin{itemize}
\item {Utilização:trasm.}
\end{itemize}
O mesmo que \textunderscore atoleiro\textunderscore .
(Cp. \textunderscore lacoso\textunderscore )
\section{Lacão}
\begin{itemize}
\item {Grp. gram.:m.}
\end{itemize}
\begin{itemize}
\item {Utilização:Prov.}
\end{itemize}
\begin{itemize}
\item {Utilização:Ant.}
\end{itemize}
O mesmo que \textunderscore presunto\textunderscore .
Pernil de porco.
\section{Lácar}
\begin{itemize}
\item {Grp. gram.:m.}
\end{itemize}
\begin{itemize}
\item {Utilização:Ant.}
\end{itemize}
O mesmo que \textunderscore lacre\textunderscore .
\section{Lacar}
\begin{itemize}
\item {Grp. gram.:v. i.}
\end{itemize}
\begin{itemize}
\item {Utilização:Prov.}
\end{itemize}
Esbarrondar-se; desmoronar-se; esboroar-se.
\section{Laçar}
\begin{itemize}
\item {Grp. gram.:v. t.}
\end{itemize}
Prender com laço.
Fazer laço em.
Enlaçar.
\section{Laçarada}
\begin{itemize}
\item {Grp. gram.:f.}
\end{itemize}
Conjunto de laços para enfeite.
\section{Laçaria}
\begin{itemize}
\item {Grp. gram.:f.}
\end{itemize}
Ornatos em fórma de laço.
Ornatos em pedra ou talha.
Porção de laços; fitas enlaçadas.
\section{Laçarotes}
\begin{itemize}
\item {Grp. gram.:m. pl.}
\end{itemize}
\begin{itemize}
\item {Utilização:Fam.}
\end{itemize}
\begin{itemize}
\item {Proveniência:(De \textunderscore laço\textunderscore ^1)}
\end{itemize}
Grande porção de laços ou enfeites vistosos; laçarada.
\section{Laçarrão}
\begin{itemize}
\item {Grp. gram.:m.}
\end{itemize}
Grande laço; laço muito vistoso, para enfeite. Cf. Castilho, \textunderscore Fausto\textunderscore , 164.
\section{Lacear}
\begin{itemize}
\item {Grp. gram.:v. t.}
\end{itemize}
Enfeitar com laços.
\section{Lacedemónias}
\begin{itemize}
\item {Grp. gram.:f. pl.}
\end{itemize}
\begin{itemize}
\item {Proveniência:(De \textunderscore lacedemónio\textunderscore )}
\end{itemize}
Festas de Esparta, feitas pelas mulheres, com exclusão dos homens.
\section{Lacedemónico}
\begin{itemize}
\item {Grp. gram.:adj.}
\end{itemize}
O mesmo que \textunderscore lacedemónio\textunderscore , adj.
\section{Lacedemónio}
\begin{itemize}
\item {Grp. gram.:adj.}
\end{itemize}
\begin{itemize}
\item {Grp. gram.:M.}
\end{itemize}
\begin{itemize}
\item {Proveniência:(Lat. \textunderscore lacedemonius\textunderscore )}
\end{itemize}
Relativo á Lacedemónia ou a Esparta.
Habitante de Lacedemónia; espartano; lacónio.
\section{Laceira}
\begin{itemize}
\item {Grp. gram.:f.}
\end{itemize}
\begin{itemize}
\item {Utilização:T. da Bairrada}
\end{itemize}
\begin{itemize}
\item {Proveniência:(De \textunderscore laço\textunderscore ^2)}
\end{itemize}
Nateiro, lodeiro.
Camada, mais ou menos espêssa, que se fórma á superfície do leite, da cal, de gorduras derretidas, etc.
\section{Laceração}
\begin{itemize}
\item {Grp. gram.:f.}
\end{itemize}
\begin{itemize}
\item {Proveniência:(Lat. \textunderscore laceratio\textunderscore )}
\end{itemize}
Acto ou effeito de lacerar.
\section{Lacerante}
\begin{itemize}
\item {Grp. gram.:adj.}
\end{itemize}
\begin{itemize}
\item {Proveniência:(Lat. \textunderscore lacerans\textunderscore )}
\end{itemize}
O mesmo que \textunderscore dilacerante\textunderscore .
\section{Lacerar}
\begin{itemize}
\item {Grp. gram.:v. t.}
\end{itemize}
\begin{itemize}
\item {Proveniência:(Lat. \textunderscore lacerare\textunderscore )}
\end{itemize}
O mesmo que \textunderscore dilacerar\textunderscore .
\section{Lacerável}
\begin{itemize}
\item {Grp. gram.:adj.}
\end{itemize}
\begin{itemize}
\item {Proveniência:(Lat. \textunderscore lacerabilis\textunderscore )}
\end{itemize}
Que se póde lacerar ou rasgar.
\section{Lacerna}
\begin{itemize}
\item {Grp. gram.:f.}
\end{itemize}
\begin{itemize}
\item {Proveniência:(Lat. \textunderscore lacerna\textunderscore )}
\end{itemize}
Gabão, pesado que os Romanos usavam no inverno.
O mesmo que \textunderscore murça\textunderscore  ou \textunderscore birro\textunderscore . Cf. Frei Nic. de S.^ta Maria, \textunderscore Chrón. dos Regrantes\textunderscore .
\section{Lácero-anterior}
\begin{itemize}
\item {Grp. gram.:adj.}
\end{itemize}
\begin{itemize}
\item {Utilização:Anat.}
\end{itemize}
\begin{itemize}
\item {Proveniência:(De \textunderscore lacerar\textunderscore  + \textunderscore anterior\textunderscore )}
\end{itemize}
Diz-se de cada um dos buracos, que há na parte anterior da base do crânio.
\section{Lácero-posterior}
\begin{itemize}
\item {Grp. gram.:adj.}
\end{itemize}
\begin{itemize}
\item {Utilização:Anat.}
\end{itemize}
\begin{itemize}
\item {Proveniência:(De \textunderscore lacerar\textunderscore  + \textunderscore posterior\textunderscore )}
\end{itemize}
Diz-se de cada um dos dois buracos, que há na parte posterior da base do crânio.
\section{Lacertiforme}
\begin{itemize}
\item {Grp. gram.:adj.}
\end{itemize}
\begin{itemize}
\item {Proveniência:(Do lat. \textunderscore lacertus\textunderscore  + \textunderscore forma\textunderscore )}
\end{itemize}
Semelhante ao lagarto.
\section{Lacertinos}
\begin{itemize}
\item {Grp. gram.:m. pl.}
\end{itemize}
\begin{itemize}
\item {Proveniência:(Do lat. \textunderscore lacertus\textunderscore )}
\end{itemize}
Família du reptís sáurios.
\section{Lacerto}
\begin{itemize}
\item {Grp. gram.:m.}
\end{itemize}
\begin{itemize}
\item {Utilização:Anat.}
\end{itemize}
\begin{itemize}
\item {Proveniência:(Lat. \textunderscore lacertus\textunderscore )}
\end{itemize}
Músculo, entre o cotovelo e o ombro.
\section{Lacete}
\begin{itemize}
\item {fónica:cê}
\end{itemize}
\begin{itemize}
\item {Grp. gram.:m.}
\end{itemize}
\begin{itemize}
\item {Proveniência:(De \textunderscore laço\textunderscore . Cp. fr. \textunderscore lacet\textunderscore )}
\end{itemize}
Parte da fechadura, por onde passa o fecho.
Curva e contra-curva de uma estrada.
Movimento da máquina de combóio, quando marcha coleando.
Empedrado, que, de espaço a espaço, se faz nas estradas macadamizadas, para que as enxurradas as não descarnem.
\section{Lacha}
\begin{itemize}
\item {Grp. gram.:f.}
\end{itemize}
\begin{itemize}
\item {Utilização:Gír.}
\end{itemize}
Vergonha, pudor; \textunderscore aquelle rapaz não tem lacha nenhuma\textunderscore .
(Da gíria de ciganos)
\section{Lacial}
\begin{itemize}
\item {Grp. gram.:adj.}
\end{itemize}
\begin{itemize}
\item {Utilização:Des.}
\end{itemize}
\begin{itemize}
\item {Proveniência:(Lat. \textunderscore latialis\textunderscore )}
\end{itemize}
Relativo ao Lácio; latino.
\section{Laciar}
\begin{itemize}
\item {Grp. gram.:adj.}
\end{itemize}
O mesmo que \textunderscore lacial\textunderscore .
\section{Lácico}
\begin{itemize}
\item {Grp. gram.:adj.}
\end{itemize}
Diz-se de um ácido, que se extrai da laca.
\section{Lacina}
\begin{itemize}
\item {Grp. gram.:f.}
\end{itemize}
Substância resinosa, que fórma a base de várias lacas do commércio.
\section{Lacinete}
\begin{itemize}
\item {fónica:nê}
\end{itemize}
\begin{itemize}
\item {Grp. gram.:m.}
\end{itemize}
\begin{itemize}
\item {Utilização:Ant.}
\end{itemize}
Pequeno lenço de algibeira.
(Por \textunderscore lencinete\textunderscore , de \textunderscore lenço\textunderscore ? De \textunderscore laço\textunderscore , pelo aspecto das pontas do lenço, pendentes da abertura da algibeira?)
\section{Lacínia}
\begin{itemize}
\item {Grp. gram.:f.}
\end{itemize}
\begin{itemize}
\item {Utilização:Bot.}
\end{itemize}
\begin{itemize}
\item {Proveniência:(Lat. \textunderscore lacinia\textunderscore )}
\end{itemize}
O mesmo que \textunderscore sépala\textunderscore .
\section{Laciniado}
\begin{itemize}
\item {Grp. gram.:adj.}
\end{itemize}
\begin{itemize}
\item {Utilização:Bot.}
\end{itemize}
\begin{itemize}
\item {Proveniência:(Do lat. \textunderscore lacinía\textunderscore )}
\end{itemize}
Recortado em tiras irregulares.
\section{Lacínula}
\begin{itemize}
\item {Grp. gram.:f.}
\end{itemize}
\begin{itemize}
\item {Utilização:Bot.}
\end{itemize}
\begin{itemize}
\item {Proveniência:(Do lat. \textunderscore lacinia\textunderscore )}
\end{itemize}
Nome, dado por Hoffmann á ponta dobrada das pétalas das umbellíferas.
\section{Lacistema}
\begin{itemize}
\item {Grp. gram.:f.}
\end{itemize}
Gênero de plantas, que serve de typo ás lacistemáceas.
\section{Lacistemáceas}
\begin{itemize}
\item {Grp. gram.:f. pl.}
\end{itemize}
\begin{itemize}
\item {Proveniência:(De \textunderscore lacistema\textunderscore )}
\end{itemize}
Família de plantas dicotyledóneas apétalas.
\section{Laclara}
\begin{itemize}
\item {Grp. gram.:f.}
\end{itemize}
\begin{itemize}
\item {Utilização:Prov.}
\end{itemize}
\begin{itemize}
\item {Utilização:alg.}
\end{itemize}
O mesmo que \textunderscore lacrau\textunderscore .
\section{Lacnóstoma}
\begin{itemize}
\item {Grp. gram.:f.}
\end{itemize}
Gênero de plantas asclepiádeas.
\section{Laço}
\begin{itemize}
\item {Grp. gram.:m.}
\end{itemize}
\begin{itemize}
\item {Utilização:Fig.}
\end{itemize}
Nó, que se desata facilmente.
Laçada.
Armadilha, para caçar.
Vínculo, prisão: \textunderscore os laços do parentesco\textunderscore .
Alliança.
Estratagema; cilada; traição.
Marca de dança mirandesa.
\section{Laço}
\begin{itemize}
\item {Grp. gram.:m.}
\end{itemize}
\begin{itemize}
\item {Utilização:Prov.}
\end{itemize}
\begin{itemize}
\item {Utilização:minh.}
\end{itemize}
\begin{itemize}
\item {Utilização:T. do Fundão}
\end{itemize}
Pellicula, que, á superfície da água, é produzida por outras substâncias, como sabão, etc.
Cal, que se estende com a colher na cal ou na parede, para depois receber a cal, que se applica com pincel.
\section{Lacobricense}
\begin{itemize}
\item {Grp. gram.:adj.}
\end{itemize}
\begin{itemize}
\item {Proveniência:(Do lat. \textunderscore Lacobrica\textunderscore , n. p.)}
\end{itemize}
Relativo á cidade de Lagos.
\section{Lacomancia}
\begin{itemize}
\item {Grp. gram.:f.}
\end{itemize}
Adivinhação por meio de dados.
\section{Lacomântico}
\begin{itemize}
\item {Grp. gram.:adj.}
\end{itemize}
Relativo á lacomancia.
\section{La-condessa}
\begin{itemize}
\item {Grp. gram.:f.}
\end{itemize}
Espécie de jôgo popular.
\section{Laconicamente}
\begin{itemize}
\item {Grp. gram.:adv.}
\end{itemize}
De modo lacónico; em resumo, em sýnthese.
\section{Lacónico}
\begin{itemize}
\item {Grp. gram.:adj.}
\end{itemize}
\begin{itemize}
\item {Grp. gram.:M.}
\end{itemize}
\begin{itemize}
\item {Proveniência:(Lat. \textunderscore laconicus\textunderscore )}
\end{itemize}
Resumido; conciso.
Estufa sêca, no tempo dos Romanos.
\section{Lacónio}
\begin{itemize}
\item {Grp. gram.:adj.}
\end{itemize}
\begin{itemize}
\item {Grp. gram.:M.}
\end{itemize}
\begin{itemize}
\item {Proveniência:(Do gr. \textunderscore lakon\textunderscore )}
\end{itemize}
Relativo á Lacónia.
O mesmo que \textunderscore lacónico\textunderscore , (falando-se do estílo). Cf. M. Bernárdez, \textunderscore N. Floresta\textunderscore , II, 203 e 204.
Habitante da Lacónia; lacedemónio; espartano.
\section{Laconismo}
\begin{itemize}
\item {Grp. gram.:m.}
\end{itemize}
\begin{itemize}
\item {Proveniência:(Lat. \textunderscore laconismus\textunderscore )}
\end{itemize}
Maneira de escrever ou falar laconicamente.
\section{Laconizar}
\begin{itemize}
\item {Grp. gram.:v. t.}
\end{itemize}
\begin{itemize}
\item {Proveniência:(Do gr. \textunderscore lakonizein\textunderscore )}
\end{itemize}
Tornar lacónico.
Expor em resumo; synthetizar.
\section{Laços-espanhoes}
\begin{itemize}
\item {Grp. gram.:m. pl.}
\end{itemize}
\begin{itemize}
\item {Utilização:Bras}
\end{itemize}
Planta, o mesmo que \textunderscore gaìllárdia\textunderscore .
\section{Lacoso}
\begin{itemize}
\item {Grp. gram.:adj.}
\end{itemize}
\begin{itemize}
\item {Utilização:Prov.}
\end{itemize}
\begin{itemize}
\item {Utilização:trasm.}
\end{itemize}
\begin{itemize}
\item {Proveniência:(Do lat. \textunderscore lacus\textunderscore )}
\end{itemize}
Pantanoso; húmido.
\section{Lacátomo}
\begin{itemize}
\item {Grp. gram.:m.}
\end{itemize}
\begin{itemize}
\item {Utilização:Geogr.}
\end{itemize}
\begin{itemize}
\item {Utilização:ant.}
\end{itemize}
\begin{itemize}
\item {Proveniência:(Lat. \textunderscore lacotomus\textunderscore )}
\end{itemize}
Linha recta, que corta o equador.
\section{Lacra}
\begin{itemize}
\item {Grp. gram.:f.}
\end{itemize}
(V.laca)
\section{Lacrador}
\begin{itemize}
\item {Grp. gram.:m.}
\end{itemize}
Indivíduo, encarregado de lacrar garrafas nos armazens de vinho. Cf. Th. Ribeiro, \textunderscore Jornadas\textunderscore , I, 167.
\section{Lacraia}
\begin{itemize}
\item {Grp. gram.:f.}
\end{itemize}
\begin{itemize}
\item {Utilização:T. de Aveiro}
\end{itemize}
\begin{itemize}
\item {Utilização:Bras}
\end{itemize}
\begin{itemize}
\item {Utilização:Bras}
\end{itemize}
\begin{itemize}
\item {Utilização:Bras. do N}
\end{itemize}
Espécie de peixe.
Espécie de canôa.
Insecto das regiões do Amazonas.
O mesmo que \textunderscore centopeia\textunderscore .
\section{Lacrar}
\begin{itemize}
\item {Grp. gram.:v. t.}
\end{itemize}
Pôr lacre em; fechar com lacre: \textunderscore lacrar uma carta\textunderscore .
\section{Lacrau}
\begin{itemize}
\item {Grp. gram.:m.}
\end{itemize}
\begin{itemize}
\item {Proveniência:(Do ár. \textunderscore al-'acrab\textunderscore )}
\end{itemize}
O mesmo que \textunderscore escorpião\textunderscore .
\section{Lacrau-do-mar}
\begin{itemize}
\item {Grp. gram.:m.}
\end{itemize}
Peixe de Portugal.
\section{Lacre}
\begin{itemize}
\item {Grp. gram.:m.}
\end{itemize}
Substância resinosa, misturada com outra substância còrante, e que serve geralmente para fechar garrafas, fechar e sellar cartas, etc.
Nome de várias plantas brasileiras.
(Da mesma or. que \textunderscore laca\textunderscore )
\section{Lacreada}
\begin{itemize}
\item {Grp. gram.:f.}
\end{itemize}
\begin{itemize}
\item {Proveniência:(De \textunderscore lacrear\textunderscore )}
\end{itemize}
Ornato esmaltado de lacre ou laca.
\section{Lacrear}
\begin{itemize}
\item {Grp. gram.:v. t.}
\end{itemize}
Dar côr de lacre a.
\section{Lacrimação}
\begin{itemize}
\item {Grp. gram.:f.}
\end{itemize}
\begin{itemize}
\item {Proveniência:(Lat. \textunderscore lacrimatio\textunderscore )}
\end{itemize}
Derramamento de lágrimas.
\section{Lacrimal}
\begin{itemize}
\item {Grp. gram.:adj.}
\end{itemize}
\begin{itemize}
\item {Grp. gram.:M.}
\end{itemize}
\begin{itemize}
\item {Proveniência:(Lat. \textunderscore lacrimalis\textunderscore )}
\end{itemize}
Relativo a lágrimas.
Que tem aspecto de lágrima.
Que serve para a secreção das lágrimas: \textunderscore canalículo lacrimal\textunderscore .
Pequeno osso, dentro da órbita ocular.
\section{Lacrimante}
\begin{itemize}
\item {Grp. gram.:adj.}
\end{itemize}
\begin{itemize}
\item {Proveniência:(Lat. \textunderscore lacrimans\textunderscore )}
\end{itemize}
O mesmo que \textunderscore lacrimoso\textunderscore .
\section{Lacrimatório}
\begin{itemize}
\item {Grp. gram.:adj.}
\end{itemize}
\begin{itemize}
\item {Grp. gram.:M.}
\end{itemize}
\begin{itemize}
\item {Proveniência:(Lat. \textunderscore lacrimatorius\textunderscore )}
\end{itemize}
Relativo a lágrimas.
Pequeno vaso, que se depositava nas sepulturas romanas, e que se suppunha guardar lágrimas.
\section{Lacrimável}
\begin{itemize}
\item {Grp. gram.:adj.}
\end{itemize}
\begin{itemize}
\item {Proveniência:(Lat. \textunderscore lacrimabilis\textunderscore )}
\end{itemize}
Digno de dó; lastimável.
\section{Lacrimejar}
\begin{itemize}
\item {Proveniência:(Do lat. \textunderscore lacrima\textunderscore )}
\end{itemize}
\textunderscore v. i.\textunderscore  (e der.)
O mesmo que \textunderscore lagrimejar\textunderscore , etc.
\section{Lacrimoso}
\begin{itemize}
\item {Grp. gram.:adj.}
\end{itemize}
\begin{itemize}
\item {Utilização:Ext.}
\end{itemize}
\begin{itemize}
\item {Proveniência:(Lat. \textunderscore lacrimosus\textunderscore )}
\end{itemize}
Que chora.
Que está banhado em lágrimas.
Lastimoso; que provoca o pranto.
\section{Lacrimotomia}
\begin{itemize}
\item {Grp. gram.:f.}
\end{itemize}
\begin{itemize}
\item {Proveniência:(De \textunderscore lacrimótomo\textunderscore )}
\end{itemize}
Incisão no canalículo lacrimal.
\section{Lacrimótomo}
\begin{itemize}
\item {Grp. gram.:m.}
\end{itemize}
\begin{itemize}
\item {Proveniência:(Do gr. \textunderscore lakruma\textunderscore  + \textunderscore tome\textunderscore )}
\end{itemize}
Instrumento, para fender o canalículo lacrimal.
\section{Lacrões}
\begin{itemize}
\item {Grp. gram.:m. pl.}
\end{itemize}
Ganchos de ferro, cada um dos quaes nasce de uma chapa em que entram as extremidades da cavilha de atravessar da testa dos reparos do campanha. Cf. Leoni, \textunderscore Diccion. de Artilh.\textunderscore , inédito.
(Provavelmente por \textunderscore lacraus\textunderscore , de \textunderscore lacrau\textunderscore . Cp. cast. \textunderscore lacrán\textunderscore , que poderia têr produzido \textunderscore lacrão\textunderscore  em português)
\section{Lactação}
\begin{itemize}
\item {Grp. gram.:f.}
\end{itemize}
Acto ou effeito de lactar.
\section{Lactagol}
\begin{itemize}
\item {Grp. gram.:m.}
\end{itemize}
\begin{itemize}
\item {Utilização:Pharm.}
\end{itemize}
Extracto sêco dos grãos de algodoeiro.
\section{Lactâmida}
\begin{itemize}
\item {Grp. gram.:f.}
\end{itemize}
\begin{itemize}
\item {Proveniência:(De \textunderscore lac\textunderscore , \textunderscore lactis\textunderscore  lat. + \textunderscore amido\textunderscore )}
\end{itemize}
Amido neutro do ácido láctico.
\section{Lactante}
\begin{itemize}
\item {Grp. gram.:adj.}
\end{itemize}
\begin{itemize}
\item {Proveniência:(Lat. \textunderscore lactans\textunderscore )}
\end{itemize}
Que lacta.
\section{Lactar}
\begin{itemize}
\item {Grp. gram.:v. t.}
\end{itemize}
\begin{itemize}
\item {Grp. gram.:V. i.}
\end{itemize}
\begin{itemize}
\item {Proveniência:(Lat. \textunderscore lactare\textunderscore )}
\end{itemize}
Aleitar, amamentar.
Mamar.
\section{Lactário}
\begin{itemize}
\item {Grp. gram.:adj.}
\end{itemize}
\begin{itemize}
\item {Grp. gram.:M.}
\end{itemize}
\begin{itemize}
\item {Proveniência:(Lat. \textunderscore lactarius\textunderscore )}
\end{itemize}
Que segrega suco leitoso.
Estabelecimento para lactação.
\section{Lactato}
\begin{itemize}
\item {Grp. gram.:m.}
\end{itemize}
\begin{itemize}
\item {Utilização:Chím.}
\end{itemize}
Designação genérica dos saes, compostos pelo ácido láctico e uma base.
\section{Láctea}
\begin{itemize}
\item {Grp. gram.:f.}
\end{itemize}
\begin{itemize}
\item {Proveniência:(De \textunderscore lácteo\textunderscore )}
\end{itemize}
Sêmen dos peixes.
\section{Láctea-via}
\begin{itemize}
\item {Grp. gram.:f.}
\end{itemize}
O mesmo que \textunderscore via-láctea\textunderscore . Cf. Garrett, \textunderscore Flor. sem Fructo\textunderscore , 113.
\section{Lactente}
\begin{itemize}
\item {Grp. gram.:adj.}
\end{itemize}
\begin{itemize}
\item {Proveniência:(Lat. \textunderscore lactens\textunderscore )}
\end{itemize}
Que ainda mama; que se está amamentando. Cf. Castilho, \textunderscore Fastos\textunderscore .
\section{Lácteo}
\begin{itemize}
\item {Grp. gram.:adj.}
\end{itemize}
\begin{itemize}
\item {Proveniência:(Lat. \textunderscore lacteus\textunderscore )}
\end{itemize}
Relativo a leite.
Que tem a côr do leite.
Que produz leite.
Que tem muito suco leitoso, (falando-se das plantas).
\section{Lactescência}
\begin{itemize}
\item {Grp. gram.:f.}
\end{itemize}
Qualidade de lactescente.
\section{Lactescente}
\begin{itemize}
\item {Grp. gram.:adj.}
\end{itemize}
\begin{itemize}
\item {Proveniência:(Lat. \textunderscore lactescens\textunderscore )}
\end{itemize}
Que tem a côr do leite.
Que contém suco leitoso.
\section{Lacticinal}
\begin{itemize}
\item {Grp. gram.:adj.}
\end{itemize}
\begin{itemize}
\item {Utilização:Ant.}
\end{itemize}
O mesmo que \textunderscore lactescente\textunderscore . Cf. Orta, \textunderscore Collóq.\textunderscore , 209.
(Cp. \textunderscore lacticínio\textunderscore )
\section{Lacticínio}
\begin{itemize}
\item {Grp. gram.:m.}
\end{itemize}
\begin{itemize}
\item {Proveniência:(Lat. \textunderscore lacticinium\textunderscore )}
\end{itemize}
Preparação comestível, feita de leite, ou em que entra o leite como elemento principal.
\section{Lacticinoso}
\begin{itemize}
\item {Grp. gram.:adj.}
\end{itemize}
\begin{itemize}
\item {Proveniência:(De \textunderscore lacticínio\textunderscore )}
\end{itemize}
O mesmo que \textunderscore lactescente\textunderscore .
\section{Láctico}
\begin{itemize}
\item {Grp. gram.:adj.}
\end{itemize}
Diz-se de um ácido, que existe no soro do leite. Cf. \textunderscore Techn. Rur.\textunderscore , 20 e 117.
(Do lat, \textunderscore lac\textunderscore , \textunderscore lactis\textunderscore )
\section{Lacticolor}
\begin{itemize}
\item {Grp. gram.:adj.}
\end{itemize}
\begin{itemize}
\item {Proveniência:(Lat. \textunderscore lacticolor\textunderscore )}
\end{itemize}
Que tem a côr do leite.
Branco como o leite.
\section{Lactífago}
\begin{itemize}
\item {Grp. gram.:adj.}
\end{itemize}
\begin{itemize}
\item {Proveniência:(Do lat. \textunderscore lac\textunderscore , \textunderscore lactis\textunderscore  + gr. \textunderscore phagein\textunderscore )}
\end{itemize}
Que se alimenta de leite.
\section{Lactífero}
\begin{itemize}
\item {Grp. gram.:adj.}
\end{itemize}
\begin{itemize}
\item {Utilização:Zool.}
\end{itemize}
\begin{itemize}
\item {Proveniência:(Do lat. \textunderscore lac\textunderscore  + \textunderscore ferre\textunderscore )}
\end{itemize}
Que produz leite ou suco lactiforme.
Que tem flôres brancas como leite.
\section{Lactífico}
\begin{itemize}
\item {Grp. gram.:adj.}
\end{itemize}
O mesmo que \textunderscore lactífero\textunderscore .
\section{Lactiforme}
\begin{itemize}
\item {Grp. gram.:adj.}
\end{itemize}
\begin{itemize}
\item {Proveniência:(Do lat. \textunderscore lac\textunderscore  + \textunderscore forma\textunderscore )}
\end{itemize}
Semelhante ao leite.
\section{Lactífugo}
\begin{itemize}
\item {Grp. gram.:adj.}
\end{itemize}
\begin{itemize}
\item {Proveniência:(Do lat. \textunderscore lac\textunderscore , \textunderscore lactis\textunderscore  + \textunderscore fugere\textunderscore )}
\end{itemize}
Que faz secar o leite ás mulheres.
\section{Lactígeno}
\begin{itemize}
\item {Grp. gram.:adj.}
\end{itemize}
\begin{itemize}
\item {Proveniência:(Do lat. \textunderscore lac\textunderscore , \textunderscore lactis\textunderscore  + \textunderscore genere\textunderscore , fórma ant. de \textunderscore gignere\textunderscore )}
\end{itemize}
Que produz leite.
Galactagogo.
\section{Lactina}
\begin{itemize}
\item {Grp. gram.:f.}
\end{itemize}
O mesmo que \textunderscore lactose\textunderscore .
\section{Lactíneo}
\begin{itemize}
\item {Grp. gram.:adj.}
\end{itemize}
\begin{itemize}
\item {Proveniência:(Lat. \textunderscore lactineus\textunderscore )}
\end{itemize}
O mesmo que \textunderscore lacticolor\textunderscore .
\section{Lactíphago}
\begin{itemize}
\item {Grp. gram.:adj.}
\end{itemize}
\begin{itemize}
\item {Proveniência:(Do lat. \textunderscore lac\textunderscore , \textunderscore lactis\textunderscore  + gr. \textunderscore phagein\textunderscore )}
\end{itemize}
Que se alimenta de leite.
\section{Lactiróseo}
\begin{itemize}
\item {fónica:ró}
\end{itemize}
\begin{itemize}
\item {Grp. gram.:adj.}
\end{itemize}
\begin{itemize}
\item {Utilização:Poét.}
\end{itemize}
\begin{itemize}
\item {Proveniência:(Do lat. \textunderscore lac\textunderscore , \textunderscore lactis\textunderscore  + \textunderscore rosa\textunderscore )}
\end{itemize}
Que tem côr de leite e de rosa: \textunderscore faces lactiróseas\textunderscore .
\section{Lactirróseo}
\begin{itemize}
\item {Grp. gram.:adj.}
\end{itemize}
\begin{itemize}
\item {Utilização:Poét.}
\end{itemize}
\begin{itemize}
\item {Proveniência:(Do lat. \textunderscore lac\textunderscore , \textunderscore lactis\textunderscore  + \textunderscore rosa\textunderscore )}
\end{itemize}
Que tem côr de leite e de rosa: \textunderscore faces lactirróseas\textunderscore .
\section{Lactobacilina}
\begin{itemize}
\item {Grp. gram.:f.}
\end{itemize}
\begin{itemize}
\item {Proveniência:(Do lat. \textunderscore lac\textunderscore , \textunderscore lactis\textunderscore  + \textunderscore bacillus\textunderscore )}
\end{itemize}
Fermento de culturas do bacilo do ácido láctico, para acidificar o leite, que se há de aplicar contra as fermentações pútridas do canal digestivo.
\section{Lactobacillina}
\begin{itemize}
\item {Grp. gram.:f.}
\end{itemize}
\begin{itemize}
\item {Proveniência:(Do lat. \textunderscore lac\textunderscore , \textunderscore lactis\textunderscore  + \textunderscore bacillus\textunderscore )}
\end{itemize}
Fermento de culturas do bacillo do ácido láctico, para acidificar o leite, que se há de applicar contra as fermentações pútridas do canal digestivo.
\section{Lacto-densímetro}
\begin{itemize}
\item {Grp. gram.:m.}
\end{itemize}
\begin{itemize}
\item {Proveniência:(De \textunderscore lac\textunderscore , \textunderscore lactis\textunderscore  + gr. \textunderscore metron\textunderscore )}
\end{itemize}
Instrumento, para medir a densidade do leite e a sua pureza.
\section{Lactofenina}
\begin{itemize}
\item {Grp. gram.:f.}
\end{itemize}
Medicamento antipirético, hipnótico, etc.
\section{Lactofosfato}
\begin{itemize}
\item {Grp. gram.:m.}
\end{itemize}
\begin{itemize}
\item {Proveniência:(De \textunderscore láctico\textunderscore  + \textunderscore fosfórico\textunderscore )}
\end{itemize}
Sal duplo, formado pelo ácido láctico e pelo ácido fosfórico com uma base.
\section{Lactómetro}
\begin{itemize}
\item {Grp. gram.:m.}
\end{itemize}
\begin{itemize}
\item {Proveniência:(Do gr. \textunderscore lac\textunderscore , \textunderscore lactis\textunderscore  + gr. \textunderscore metron\textunderscore )}
\end{itemize}
O mesmo que \textunderscore galactómetro\textunderscore .
\section{Lactophenina}
\begin{itemize}
\item {Grp. gram.:f.}
\end{itemize}
Medicamento antipyrético, hypnótico, etc.
\section{Lactophosphato}
\begin{itemize}
\item {Grp. gram.:m.}
\end{itemize}
\begin{itemize}
\item {Proveniência:(De \textunderscore láctico\textunderscore  + \textunderscore phosphórico\textunderscore )}
\end{itemize}
Sal duplo, formado pelo ácido láctico e pelo ácido phosphórico com uma base.
\section{Lactoscópio}
\begin{itemize}
\item {Grp. gram.:m.}
\end{itemize}
\begin{itemize}
\item {Proveniência:(Do lat. \textunderscore lac\textunderscore , \textunderscore lactis\textunderscore  + gr. \textunderscore skopein\textunderscore )}
\end{itemize}
Instrumento, com que se determina a quantidade de manteiga que há no leite.
\section{Lactose}
\begin{itemize}
\item {Grp. gram.:f.}
\end{itemize}
\begin{itemize}
\item {Proveniência:(Do lat. \textunderscore lac\textunderscore )}
\end{itemize}
Substância privativa do leite dos mammíferos, também conhecida por \textunderscore açúcar do leite\textunderscore .
\section{Lactoso}
\begin{itemize}
\item {Grp. gram.:adj.}
\end{itemize}
\begin{itemize}
\item {Proveniência:(Lat. \textunderscore lactosus\textunderscore )}
\end{itemize}
O mesmo que \textunderscore leitoso\textunderscore .
\section{Lactucário}
\begin{itemize}
\item {Grp. gram.:m.}
\end{itemize}
\begin{itemize}
\item {Proveniência:(Lat. \textunderscore lactucarius\textunderscore )}
\end{itemize}
Suco do caule da alface, applicado em pharmácia.
\section{Lactúceas}
\begin{itemize}
\item {Grp. gram.:f. pl.}
\end{itemize}
\begin{itemize}
\item {Utilização:Bot.}
\end{itemize}
\begin{itemize}
\item {Proveniência:(Do lat. \textunderscore lactuca\textunderscore )}
\end{itemize}
Tríbo de synanthéreas, o mesmo que \textunderscore chicoriáceas\textunderscore .
\section{Lactúceo}
\begin{itemize}
\item {Grp. gram.:adj.}
\end{itemize}
\begin{itemize}
\item {Proveniência:(Do lat. \textunderscore lactuca\textunderscore )}
\end{itemize}
Relativo ou semelhante á alface.
\section{Lactúcico}
\begin{itemize}
\item {Grp. gram.:adj.}
\end{itemize}
\begin{itemize}
\item {Proveniência:(Do lat. \textunderscore lactuca\textunderscore )}
\end{itemize}
Diz-se de um ácido, que se extrai do suco das lactúceas.
\section{Lactucina}
\begin{itemize}
\item {Grp. gram.:f.}
\end{itemize}
\begin{itemize}
\item {Proveniência:(Do lat. \textunderscore lactuca\textunderscore )}
\end{itemize}
Substância, extrahída das lactúceas.
\section{Lactume}
\begin{itemize}
\item {Grp. gram.:m.}
\end{itemize}
\begin{itemize}
\item {Utilização:Med.}
\end{itemize}
\begin{itemize}
\item {Proveniência:(Do lat. \textunderscore lac\textunderscore , \textunderscore lactis\textunderscore )}
\end{itemize}
Espécie de crosta ou usagre, que apparece na cabeça e ás vezes noutras partes do corpo das crianças, em quanto mamam.
\section{Lacuna}
\begin{itemize}
\item {Grp. gram.:f.}
\end{itemize}
\begin{itemize}
\item {Proveniência:(Lat. \textunderscore lacuna\textunderscore )}
\end{itemize}
Vácuo, num corpo.
Intervallo.
Falta; omissão: \textunderscore preencher lacunas\textunderscore .
Cavidade regular de algumas plantas.
\section{Lacunar}
\begin{itemize}
\item {Grp. gram.:adj.}
\end{itemize}
Que tem lacunas.
\section{Lacunário}
\begin{itemize}
\item {Grp. gram.:m.}
\end{itemize}
\begin{itemize}
\item {Proveniência:(Lat. \textunderscore lacunarius\textunderscore )}
\end{itemize}
Espaço entre vigas.
Ornato nos intercolúmnios das architraves.
\section{Lacunoso}
\begin{itemize}
\item {Grp. gram.:adj.}
\end{itemize}
O mesmo que \textunderscore lacunar\textunderscore .
Que tem falhas ou em que falta alguma coisa.
\section{Lacustral}
\begin{itemize}
\item {Grp. gram.:adj.}
\end{itemize}
O mesmo que \textunderscore lacustre\textunderscore .
\section{Lacustre}
\begin{itemize}
\item {Grp. gram.:adj.}
\end{itemize}
\begin{itemize}
\item {Proveniência:(Lat. \textunderscore lacustris\textunderscore )}
\end{itemize}
Relativo a lago.
Que está sobre um lago: \textunderscore habitações lacustres\textunderscore .
Que vive nos lagos: \textunderscore povos lacustres\textunderscore .
\section{Lada}
\begin{itemize}
\item {Grp. gram.:f.}
\end{itemize}
\begin{itemize}
\item {Utilização:Ant.}
\end{itemize}
\begin{itemize}
\item {Proveniência:(De \textunderscore lado\textunderscore )}
\end{itemize}
Corrente navegável.
Pequena corrente.
Beira do rio, margem.
\section{Lada}
\begin{itemize}
\item {Grp. gram.:f.}
\end{itemize}
\begin{itemize}
\item {Proveniência:(Lat. \textunderscore lada\textunderscore )}
\end{itemize}
Planta, o mesmo que \textunderscore estevão\textunderscore .
\section{Ladaínha}
\begin{itemize}
\item {Grp. gram.:f.}
\end{itemize}
\begin{itemize}
\item {Utilização:Fig.}
\end{itemize}
\begin{itemize}
\item {Proveniência:(Do lat. \textunderscore litania\textunderscore )}
\end{itemize}
Oração, em que se invoca a Vírgem ou os santos, pelos seus nomes ou attributos symbólicos.
Relação fastidiosa; lenga-lenga.
\section{Ladairo}
\begin{itemize}
\item {Grp. gram.:m.}
\end{itemize}
\begin{itemize}
\item {Utilização:Prov.}
\end{itemize}
\begin{itemize}
\item {Utilização:minh.}
\end{itemize}
Procissão de penitência, por voto a algum santuário; clamor, círio.
Parlenga, aranzel.
(Port. ant. \textunderscore ledaairo\textunderscore , do lat. hyp. \textunderscore litanarius\textunderscore , de \textunderscore litania\textunderscore )
\section{Ladam}
\begin{itemize}
\item {Grp. gram.:m.}
\end{itemize}
\begin{itemize}
\item {Proveniência:(Lat. \textunderscore ladanus\textunderscore )}
\end{itemize}
Planta, o mesmo que \textunderscore ládano\textunderscore  e \textunderscore lada\textunderscore ^2.
\section{Ladania}
\begin{itemize}
\item {Grp. gram.:f.}
\end{itemize}
Fórma obsoleta de ladaínha. Cf. Frei Fortun., \textunderscore Inéditos\textunderscore , 309.
\section{Ladanífero}
\begin{itemize}
\item {Grp. gram.:adj.}
\end{itemize}
\begin{itemize}
\item {Proveniência:(Do lat. \textunderscore ladanum\textunderscore  + \textunderscore ferre\textunderscore )}
\end{itemize}
Que produz ládano.
\section{Ládano}
\begin{itemize}
\item {Grp. gram.:m.}
\end{itemize}
\begin{itemize}
\item {Proveniência:(Lat. \textunderscore ladanum\textunderscore )}
\end{itemize}
O mesmo que \textunderscore lábdano\textunderscore .
O mesmo que \textunderscore lada\textunderscore ^2.
\section{Ládão}
\begin{itemize}
\item {Grp. gram.:m.}
\end{itemize}
\begin{itemize}
\item {Proveniência:(Lat. \textunderscore ladanus\textunderscore )}
\end{itemize}
Planta, o mesmo que \textunderscore ládano\textunderscore  e \textunderscore lada\textunderscore ^2.
\section{Ladário}
\begin{itemize}
\item {Grp. gram.:m.}
\end{itemize}
O mesmo que \textunderscore ladairo\textunderscore .
\section{Ladeamento}
\begin{itemize}
\item {Grp. gram.:m.}
\end{itemize}
Acto de ladear.
\section{Ladear}
\begin{itemize}
\item {Grp. gram.:v. t.}
\end{itemize}
Seguir a par, ao lado.
Estar ao lado de: \textunderscore campo que ladeia a estrada\textunderscore .
Atacar de lado.
Tergiversar á cêrca de: \textunderscore ladear uma questão\textunderscore .
Sophismar.
Collocar por igual (a alma da peça de artilharia).
Guiar para os lados; fazer ladear: \textunderscore ladear um cavallo\textunderscore . Cf. Camillo, \textunderscore Brasileira\textunderscore , 67.
\section{Ladeira}
\begin{itemize}
\item {Grp. gram.:f.}
\end{itemize}
\begin{itemize}
\item {Proveniência:(Do lat. \textunderscore lateraria\textunderscore )}
\end{itemize}
Declive; terreno inclinado.
\section{Ladeirento}
\begin{itemize}
\item {Grp. gram.:adj.}
\end{itemize}
Em que há ladeira; que tem ladeira; íngreme.
\section{Ladeiro}
\begin{itemize}
\item {Grp. gram.:adj.}
\end{itemize}
\begin{itemize}
\item {Grp. gram.:M.}
\end{itemize}
Que pende para o lado.
Que está ao lado.
Que é chato, (falando-se de um prato).
O mesmo que \textunderscore ladeira\textunderscore .
\section{Ladeza}
\begin{itemize}
\item {Grp. gram.:f.}
\end{itemize}
\begin{itemize}
\item {Utilização:Ant.}
\end{itemize}
O mesmo que \textunderscore lado\textunderscore :«\textunderscore está em ladeza da equinoxial\textunderscore ». Esmeraldo, X.
\section{Ladilha}
\begin{itemize}
\item {Grp. gram.:f.}
\end{itemize}
\begin{itemize}
\item {Utilização:Ant.}
\end{itemize}
Piolho ladro.
(Por \textunderscore ladrilha\textunderscore , de \textunderscore ladro\textunderscore )
\section{Ladim}
\begin{itemize}
\item {Grp. gram.:adj.}
\end{itemize}
O mesmo que \textunderscore ladinho\textunderscore .
\section{Ladimo}
\begin{itemize}
\item {Grp. gram.:adj.}
\end{itemize}
\begin{itemize}
\item {Utilização:Ant.}
\end{itemize}
O mesmo que \textunderscore ladinho\textunderscore .
\section{Ladinho}
\begin{itemize}
\item {Grp. gram.:adj.}
\end{itemize}
\begin{itemize}
\item {Utilização:Ant.}
\end{itemize}
\begin{itemize}
\item {Proveniência:(Do lat. \textunderscore latinus\textunderscore )}
\end{itemize}
Legitimo, lídimo.
Puro, sem mescla.
Nóvi-latino, românico.
Latino.
O mesmo que \textunderscore rhético\textunderscore .
\section{Ladinice}
\begin{itemize}
\item {Grp. gram.:f.}
\end{itemize}
Qualidade ou acto de ladino.
\section{Ladino}
\begin{itemize}
\item {Grp. gram.:adj.}
\end{itemize}
\begin{itemize}
\item {Utilização:Ant.}
\end{itemize}
\begin{itemize}
\item {Proveniência:(Do lat. \textunderscore latinus\textunderscore )}
\end{itemize}
Astuto.
Finório; ardiloso.
Latino.
Genuíno; puro.
\section{Lado}
\begin{itemize}
\item {Grp. gram.:m.}
\end{itemize}
\begin{itemize}
\item {Proveniência:(Lat. \textunderscore latus\textunderscore )}
\end{itemize}
Parte direita ou esquerda do corpo dos homens ou dos animaes: \textunderscore uma dôr no lado esquerdo\textunderscore .
Flanco.
Lugar ou parte, situada á esquerda ou á direita de alguém, ou de alguma coisa.
Cada um dos limites de uma figura geométrica.
Direcção.
Aspecto.
Banda, parte; sítio: \textunderscore os gritos vinham daquelle lado\textunderscore .
\section{Ladra}
\begin{itemize}
\item {Grp. gram.:f.  e  adj.}
\end{itemize}
\begin{itemize}
\item {Utilização:T. de Aveiro}
\end{itemize}
\begin{itemize}
\item {Proveniência:(De \textunderscore ladro\textunderscore ^2)}
\end{itemize}
Mulher, que furta ou rouba.
Cambo.
Batel, que acompanha o barco moliceiro.
\section{Ladrado}
\begin{itemize}
\item {Grp. gram.:m.}
\end{itemize}
\begin{itemize}
\item {Utilização:Pop.}
\end{itemize}
\begin{itemize}
\item {Utilização:Fig.}
\end{itemize}
\begin{itemize}
\item {Proveniência:(De \textunderscore ladrar\textunderscore )}
\end{itemize}
O mesmo que \textunderscore latido\textunderscore .
Maledicência.
\section{Ladrador}
\begin{itemize}
\item {Grp. gram.:adj.}
\end{itemize}
\begin{itemize}
\item {Grp. gram.:M.}
\end{itemize}
\begin{itemize}
\item {Proveniência:(Lat. \textunderscore latrator\textunderscore )}
\end{itemize}
Que ladra.
Aquelle que ladra.
\section{Ladradura}
\begin{itemize}
\item {Grp. gram.:f.}
\end{itemize}
\begin{itemize}
\item {Proveniência:(De \textunderscore ladrar\textunderscore ^1)}
\end{itemize}
O mesmo que \textunderscore latido\textunderscore .
\section{Ladral}
\begin{itemize}
\item {Grp. gram.:m.}
\end{itemize}
\begin{itemize}
\item {Utilização:Prov.}
\end{itemize}
\begin{itemize}
\item {Utilização:minh.}
\end{itemize}
\begin{itemize}
\item {Utilização:Prov.}
\end{itemize}
\begin{itemize}
\item {Utilização:trasm.}
\end{itemize}
\begin{itemize}
\item {Proveniência:(Do lat. \textunderscore lateralis\textunderscore )}
\end{itemize}
Cada um dos dois costaes de madeira, que se levantam sôbre a cheda, para conducção de objectos mais ou menos soltos, como espigas de milho, batatas, etc.
O mesmo que \textunderscore taipal\textunderscore .
\section{Ladrante}
\begin{itemize}
\item {Grp. gram.:adj.}
\end{itemize}
\begin{itemize}
\item {Proveniência:(Do lat. \textunderscore latrans\textunderscore )}
\end{itemize}
Que ladra.
\section{Ladrão}
\begin{itemize}
\item {Grp. gram.:adj.}
\end{itemize}
\begin{itemize}
\item {Grp. gram.:M.}
\end{itemize}
\begin{itemize}
\item {Utilização:Fig.}
\end{itemize}
\begin{itemize}
\item {Utilização:Fam.}
\end{itemize}
\begin{itemize}
\item {Utilização:Agr.}
\end{itemize}
\begin{itemize}
\item {Utilização:Prov.}
\end{itemize}
\begin{itemize}
\item {Utilização:alent.}
\end{itemize}
\begin{itemize}
\item {Utilização:Prov.}
\end{itemize}
\begin{itemize}
\item {Utilização:alent.}
\end{itemize}
\begin{itemize}
\item {Proveniência:(Do lat. \textunderscore latro\textunderscore )}
\end{itemize}
Que furta ou rouba.
Aquelle que furta ou rouba.
Tratante.
Biltre.
Homem sem consciência.
Brejeiro, maganão.
Rebento vegetal, que prejudica o desenvolvimento da planta, roubando-lhe parte da seiva.
Abertura na levada do moínho, por onde se escôa a água que sobeja.
Baile de roda e espectivo canto.
Morrão de torcida, que se inclina acceso para um dos lados da vela, consumindo-a nesse ponto, com prejuízo.
\section{Ladrar}
\begin{itemize}
\item {Grp. gram.:v. i.}
\end{itemize}
\begin{itemize}
\item {Utilização:pop.}
\end{itemize}
\begin{itemize}
\item {Utilização:Fig.}
\end{itemize}
\begin{itemize}
\item {Proveniência:(Lat. \textunderscore latrare\textunderscore )}
\end{itemize}
Dar latidos.
Gritar desentoadamente.
\section{Ladrar}
\begin{itemize}
\item {Grp. gram.:v. i.}
\end{itemize}
\begin{itemize}
\item {Utilização:Ant.}
\end{itemize}
Fazer alarde ou ostentação de seus méritos.
(Corr. de \textunderscore alardear\textunderscore )
\section{Ladraria}
\begin{itemize}
\item {Grp. gram.:f.}
\end{itemize}
\begin{itemize}
\item {Proveniência:(De \textunderscore ladras\textunderscore )}
\end{itemize}
Doença dos porcos, produzida pelos vermes, chamados ladras.
\section{Ladras}
\begin{itemize}
\item {Grp. gram.:f. pl.}
\end{itemize}
Espécie de vermes, que accommetem o gado suíno.
\section{Ladravão}
\begin{itemize}
\item {Grp. gram.:m.}
\end{itemize}
O mesmo que \textunderscore ladravaz\textunderscore . Cf. Arn. Gama, \textunderscore Últ. Dona\textunderscore , 35.
\section{Ladravaz}
\begin{itemize}
\item {Grp. gram.:m.}
\end{itemize}
\begin{itemize}
\item {Proveniência:(Do rad. de \textunderscore ladrão\textunderscore )}
\end{itemize}
Grande ladrão.
Grande tratante.
\section{Ladriço}
\begin{itemize}
\item {Grp. gram.:m.}
\end{itemize}
Corda, que prende ao travão o pé do cavallo.
\section{Ladrido}
\begin{itemize}
\item {Grp. gram.:m.}
\end{itemize}
O mesmo que \textunderscore latido\textunderscore .
\section{Ladrilhador}
\begin{itemize}
\item {Grp. gram.:m.  e  adj.}
\end{itemize}
Aquelle que ladrilha.
\section{Ladrilhagem}
\begin{itemize}
\item {Grp. gram.:f.}
\end{itemize}
Acto ou effeito de ladrilhar.
\section{Ladrilhar}
\begin{itemize}
\item {Grp. gram.:v. t.}
\end{itemize}
Pôr ladrilhos em.
\section{Ladrilheiro}
\begin{itemize}
\item {Grp. gram.:m.}
\end{itemize}
Aquelle que faz ladrilhos.
\section{Ladrilho}
\begin{itemize}
\item {Grp. gram.:m.}
\end{itemize}
\begin{itemize}
\item {Utilização:Ext.}
\end{itemize}
\begin{itemize}
\item {Proveniência:(Do lat. hypoth. \textunderscore latericulus\textunderscore )}
\end{itemize}
Peça rectangular de barro cozido, a qual serve geralmente para pavimentos.
Tejolo.
Aquillo que tem fórma ou apparência de ladrilho.
Pedaço rectangular de marmelada.
Variedade de bolos secos.
\section{Ladripar}
\begin{itemize}
\item {Grp. gram.:v. t.}
\end{itemize}
\begin{itemize}
\item {Proveniência:(De \textunderscore ladro\textunderscore ^2)}
\end{itemize}
Surripiar; furtar (coisas de pouco valor).
\section{Ladripo}
\begin{itemize}
\item {Grp. gram.:m.}
\end{itemize}
\begin{itemize}
\item {Utilização:Prov.}
\end{itemize}
\begin{itemize}
\item {Utilização:minh.}
\end{itemize}
Aquelle que ladripa.
(Colhido em Guimarães)
\section{Ladro}
\begin{itemize}
\item {Grp. gram.:m.}
\end{itemize}
\begin{itemize}
\item {Proveniência:(Do rad. de \textunderscore ladrar\textunderscore ^1)}
\end{itemize}
O mesmo que \textunderscore latido\textunderscore .
\section{Ladro}
\begin{itemize}
\item {Grp. gram.:adj.}
\end{itemize}
\begin{itemize}
\item {Utilização:Fig.}
\end{itemize}
\begin{itemize}
\item {Proveniência:(Do lat. \textunderscore latro\textunderscore )}
\end{itemize}
Que é ladrão.
Que prende o coração.
E diz-se de uma variedade de piolho, popularmente conhecido por \textunderscore chato\textunderscore .
\section{Ladrôa}
\begin{itemize}
\item {Grp. gram.:f.}
\end{itemize}
\begin{itemize}
\item {Utilização:Des.}
\end{itemize}
O mesmo que \textunderscore ladra\textunderscore . Cf. B. Pereira, \textunderscore Prosódia\textunderscore .
\section{Ladroaço}
\begin{itemize}
\item {Grp. gram.:m.}
\end{itemize}
O mesmo que \textunderscore ladravaz\textunderscore . Cf. Camillo, \textunderscore Mar. da Fonte\textunderscore , 389.
\section{Ladroado}
\begin{itemize}
\item {Grp. gram.:adj.}
\end{itemize}
\begin{itemize}
\item {Proveniência:(De \textunderscore ladroar\textunderscore )}
\end{itemize}
Roubado:«\textunderscore pôsto na forca por queijo ladroado.\textunderscore »Filinto, XII, 108.
\section{Ladroagem}
\begin{itemize}
\item {Grp. gram.:f.}
\end{itemize}
\begin{itemize}
\item {Proveniência:(De \textunderscore ladro\textunderscore ^2)}
\end{itemize}
Vício de ladrão.
Os ladrões.
\section{Ladroar}
\begin{itemize}
\item {Grp. gram.:v. t.}
\end{itemize}
\begin{itemize}
\item {Proveniência:(De \textunderscore ladro\textunderscore ^2)}
\end{itemize}
O mesmo que \textunderscore roubar\textunderscore . Cf. Camillo, \textunderscore Coisas Leves\textunderscore , 98.
\section{Ladroeira}
\begin{itemize}
\item {Grp. gram.:f.}
\end{itemize}
\begin{itemize}
\item {Proveniência:(De \textunderscore ladrão\textunderscore )}
\end{itemize}
Acto de roubar; roubo.
Extorsão.
Esconderijo de ladrões.
Descaminho criminoso e continuado de valores.
\section{Ladroeirar}
\begin{itemize}
\item {Grp. gram.:v. i.}
\end{itemize}
Fazer ladroeiras.
\section{Ladroeiro}
\begin{itemize}
\item {Grp. gram.:m.}
\end{itemize}
\begin{itemize}
\item {Utilização:Agr.}
\end{itemize}
Rebento, que damnifica as plantas.
(Cp. \textunderscore ladrão\textunderscore )
\section{Ladroíce}
\begin{itemize}
\item {Grp. gram.:f.}
\end{itemize}
O mesmo que \textunderscore ladroeira\textunderscore .
\section{Ladrona}
\begin{itemize}
\item {Grp. gram.:f.  e  adj.}
\end{itemize}
\begin{itemize}
\item {Utilização:Burl.}
\end{itemize}
\begin{itemize}
\item {Utilização:Fam.}
\end{itemize}
O mesmo que \textunderscore ladra\textunderscore .
Mulher ou menina brejeira, maliciosa.
\section{Lafiro}
\begin{itemize}
\item {Grp. gram.:adj.}
\end{itemize}
\begin{itemize}
\item {Utilização:Prov.}
\end{itemize}
\begin{itemize}
\item {Utilização:ant.}
\end{itemize}
Taful.
Garrido.
\section{Lafrau}
\begin{itemize}
\item {Grp. gram.:m.}
\end{itemize}
\begin{itemize}
\item {Utilização:Prov.}
\end{itemize}
\begin{itemize}
\item {Utilização:trasm.}
\end{itemize}
Intrujão, que percorre as feiras, burlando os incautos com o jôgo da vermelhinha.
\section{Lafuêntea}
\begin{itemize}
\item {Grp. gram.:f.}
\end{itemize}
\begin{itemize}
\item {Proveniência:(De \textunderscore Lafuente\textunderscore , n. p.)}
\end{itemize}
Gênero de plantas escrofularíneas.
\section{Lagalhé}
\begin{itemize}
\item {Grp. gram.:m.}
\end{itemize}
\begin{itemize}
\item {Utilização:Burl.}
\end{itemize}
Pessôa insignificante; jagodes; badameco.
(Cp. \textunderscore nagalhé\textunderscore )
\section{Lagamar}
\begin{itemize}
\item {Grp. gram.:m.}
\end{itemize}
\begin{itemize}
\item {Proveniência:(De \textunderscore lago\textunderscore  + \textunderscore mar\textunderscore )}
\end{itemize}
Cova no fundo de um rio ou do mar.
Pégo.
Parte abrigada de um pôrto ou baía.
Lagôa de água salgada.
\section{Lágana}
\begin{itemize}
\item {Grp. gram.:f.}
\end{itemize}
\begin{itemize}
\item {Proveniência:(Lat. \textunderscore laganum\textunderscore )}
\end{itemize}
Bolo de farinha e azeite, usado entre os antigos Romanos. Cf. Castilho, \textunderscore Fastos\textunderscore , III, 480.--Não obstante a autoridade citada, sería preferível \textunderscore lágano\textunderscore .
\section{Laganha}
\begin{itemize}
\item {Grp. gram.:f.}
\end{itemize}
\begin{itemize}
\item {Utilização:Prov.}
\end{itemize}
\begin{itemize}
\item {Utilização:trasm.}
\end{itemize}
O mesmo que \textunderscore remela\textunderscore .
(Cast. \textunderscore lagaña\textunderscore )
\section{Laganhoso}
\begin{itemize}
\item {Grp. gram.:adj.}
\end{itemize}
Que tem laganha.
\section{Lagão}
\begin{itemize}
\item {Grp. gram.:m.}
\end{itemize}
Espécie de galera asiática.
\section{Lagar}
\begin{itemize}
\item {Grp. gram.:m.}
\end{itemize}
\begin{itemize}
\item {Proveniência:(Do rad. de \textunderscore lago\textunderscore )}
\end{itemize}
Espécie de tanque, em que se espremem e se reduzem a líquido certos frutos: \textunderscore lagar de azeite\textunderscore ; \textunderscore lagar de vinho\textunderscore .
Estabelecimento ou officina, em que está êsse tanque e os apparelhos correspondentes.
\section{Lagarada}
\begin{itemize}
\item {Grp. gram.:f.}
\end{itemize}
Porção de frutos, que um lagar contém.
\section{Lagaragem}
\begin{itemize}
\item {Grp. gram.:f.}
\end{itemize}
\begin{itemize}
\item {Proveniência:(De \textunderscore lagar\textunderscore )}
\end{itemize}
Conjunto dos serviços ou operações, para se fazer vinho ou azeite.
Retribuição, em azeite ou vinho, ao dono de um lagar, por cada lagarada.
\section{Lagareiro}
\begin{itemize}
\item {Grp. gram.:m.}
\end{itemize}
\begin{itemize}
\item {Utilização:Pop.}
\end{itemize}
\begin{itemize}
\item {Grp. gram.:Adj.}
\end{itemize}
Aquelle que trabalha em lagares, especialmente nos de azeite.
Dono de lagar.
Indivíduo, que se apresenta com o fato muito sujo.
Relativo a lagar: \textunderscore serviços lagareiros\textunderscore .
Diz-se de certos sinaes usados em lagares, no termo de Alcobaça, para distinguir as porções de azeitona, com que cada indivíduo concorreu para uma lagaragem, a fim de se dividir depois equitativamente o azeite.
\section{Lagareta}
\begin{itemize}
\item {fónica:garê}
\end{itemize}
\begin{itemize}
\item {Grp. gram.:f.}
\end{itemize}
\begin{itemize}
\item {Utilização:Prov.}
\end{itemize}
\begin{itemize}
\item {Utilização:minh.}
\end{itemize}
O mesmo que \textunderscore lagariça\textunderscore .
\section{Lagariça}
\begin{itemize}
\item {Grp. gram.:f.}
\end{itemize}
\begin{itemize}
\item {Utilização:Pop.}
\end{itemize}
Pequeno lagar.
Lagar.
Porção de líquido entornado e espalhado.
\section{Lagariço}
\begin{itemize}
\item {Grp. gram.:adj.}
\end{itemize}
\begin{itemize}
\item {Grp. gram.:M.}
\end{itemize}
\begin{itemize}
\item {Utilização:Bras}
\end{itemize}
Relativo ao lagar.
Vaso de madeira e loiça, ou todo de ferro, em que se espremem frutos. Cf. \textunderscore Tarifa das Alfând.\textunderscore , do Brasil, 119.
\section{Lagarinto}
\begin{itemize}
\item {Grp. gram.:m.}
\end{itemize}
Gênero de plantas do Cabo da Bôa-Esperança.
\section{Lágaro}
\begin{itemize}
\item {Grp. gram.:m.}
\end{itemize}
\begin{itemize}
\item {Proveniência:(Do gr. \textunderscore lagaros\textunderscore )}
\end{itemize}
Espécie de antigo verso hexámetro.
\section{Lagarta}
\begin{itemize}
\item {Grp. gram.:f.}
\end{itemize}
\begin{itemize}
\item {Proveniência:(De \textunderscore lagarto\textunderscore )}
\end{itemize}
Larva dos insectos lepidópteros.
Primeira phase da vida das borboletas, até que se convertem em chrysállidas.
Lagartixa.
\section{Lagarteira}
\begin{itemize}
\item {Grp. gram.:f.}
\end{itemize}
Buraco ou toca, onde se recolhem os lagartos.
\section{Lagarteiro}
\begin{itemize}
\item {Grp. gram.:adj.}
\end{itemize}
\begin{itemize}
\item {Utilização:Chul.}
\end{itemize}
\begin{itemize}
\item {Utilização:Cyn.}
\end{itemize}
\begin{itemize}
\item {Proveniência:(De \textunderscore lagarto\textunderscore )}
\end{itemize}
Manhoso.
Nome, que se dava a uma espécie de francelho ou falcão, que se criava em buracos.
\section{Lagartixa}
\begin{itemize}
\item {Grp. gram.:f.}
\end{itemize}
Pequeno lagarto, (\textunderscore lacerta agilis\textunderscore ).
Antiga e pequena peça de artilharia. Cf. \textunderscore Peregrinação\textunderscore , CLXXXVI.
\section{Lagarto}
\begin{itemize}
\item {Grp. gram.:m.}
\end{itemize}
\begin{itemize}
\item {Utilização:Bras. do N}
\end{itemize}
\begin{itemize}
\item {Proveniência:(Lat. hyp. \textunderscore lacartus\textunderscore , por \textunderscore lacertus\textunderscore )}
\end{itemize}
Família de reptís sáurios; sardão.
Peixe dos Açores.
Polpa da perna.
Apparelho, com que se apertam as rolhas de cortiça, para lhes dar menor diâmetro.
O mesmo que \textunderscore tendão\textunderscore .
\section{Lagartuxa}
\begin{itemize}
\item {Grp. gram.:f.}
\end{itemize}
\begin{itemize}
\item {Utilização:Prov.}
\end{itemize}
\begin{itemize}
\item {Utilização:trasm.}
\end{itemize}
O mesmo que \textunderscore lagartixa\textunderscore .
\section{Lagasca}
\begin{itemize}
\item {Grp. gram.:f.}
\end{itemize}
\begin{itemize}
\item {Proveniência:(De \textunderscore Lagasca\textunderscore , n. p.)}
\end{itemize}
Gênero de plantas da América tropical.
\section{Lage}
\begin{itemize}
\item {Grp. gram.:f.}
\end{itemize}
Pedra, de superfície plana.
Grande pedra quadrada e chata.
Rocha extensa, de superfície mais ou menos plana.
(Cp. \textunderscore laja\textunderscore , que é melhor orthogr.)
\section{Lágea}
\begin{itemize}
\item {Grp. gram.:f.}
\end{itemize}
O mesmo que \textunderscore lage\textunderscore .
\section{Lageado}
\begin{itemize}
\item {Grp. gram.:m.}
\end{itemize}
\begin{itemize}
\item {Utilização:Bras. do S}
\end{itemize}
\begin{itemize}
\item {Proveniência:(De \textunderscore lagear\textunderscore )}
\end{itemize}
Pavimento coberto de lages.
Regato, cujo leito é de rocha.
\section{Lageador}
\begin{itemize}
\item {Grp. gram.:m.}
\end{itemize}
Aquelle que lageia.
\section{Lageamento}
\begin{itemize}
\item {Grp. gram.:m.}
\end{itemize}
Acto ou effeito de lagear.
\section{Lagear}
\begin{itemize}
\item {Grp. gram.:v. t.}
\end{itemize}
Assentar lages em; revestir de lages.
Fazer o pavimento de.
\section{Lagedo}
\begin{itemize}
\item {fónica:gê}
\end{itemize}
\begin{itemize}
\item {Grp. gram.:m.}
\end{itemize}
Lageamento.
Lugar, em que há muitas lages; lage muito extensa.
\section{Lagem}
\begin{itemize}
\item {Grp. gram.:f.}
\end{itemize}
O mesmo que \textunderscore lage\textunderscore .
\section{Lagena}
\begin{itemize}
\item {Grp. gram.:f.}
\end{itemize}
\begin{itemize}
\item {Proveniência:(Lat. \textunderscore lagena\textunderscore )}
\end{itemize}
Vaso de barro, com asas.
Antigo vaso de collo estreito, semelhante a uma garrafa.
\section{Lagênia}
\begin{itemize}
\item {Grp. gram.:f.}
\end{itemize}
\begin{itemize}
\item {Proveniência:(Do lat. \textunderscore lagena\textunderscore )}
\end{itemize}
Gênero de plantas gencianáceas.
\section{Lageniforme}
\begin{itemize}
\item {Grp. gram.:adj.}
\end{itemize}
\begin{itemize}
\item {Proveniência:(Do lat. \textunderscore lagena\textunderscore  + \textunderscore forma\textunderscore )}
\end{itemize}
Semelhante a uma garrafa.
\section{Lagênula}
\begin{itemize}
\item {Grp. gram.:f.}
\end{itemize}
Pequena lagena.
Planta trepadeira da Cochinchina, cujo fruto tem a apparência de uma pequena garrafa.
\section{Lageoso}
\begin{itemize}
\item {Grp. gram.:adj.}
\end{itemize}
\begin{itemize}
\item {Utilização:Des.}
\end{itemize}
Em que há lages: \textunderscore terreno lageoso\textunderscore .
\section{Lagerstrêmia}
\begin{itemize}
\item {Grp. gram.:f.}
\end{itemize}
Arbusto ornamental, (\textunderscore lagerstroemia indica\textunderscore , Lin.).
\section{Lágidas}
\begin{itemize}
\item {Grp. gram.:m. pl.}
\end{itemize}
Dynastia egýpcia, fundada por Ptolomeu Lago.
\section{Lago}
\begin{itemize}
\item {Grp. gram.:m.}
\end{itemize}
\begin{itemize}
\item {Utilização:Ext.}
\end{itemize}
\begin{itemize}
\item {Utilização:Pop.}
\end{itemize}
\begin{itemize}
\item {Utilização:Fig.}
\end{itemize}
\begin{itemize}
\item {Proveniência:(Do lat. \textunderscore lacus\textunderscore )}
\end{itemize}
Grande espaço de água, rodeado de terra.
Tanque.
Grande quantidade de líquido, entornado no solo.
\section{Lagôa}
\begin{itemize}
\item {Grp. gram.:f.}
\end{itemize}
\begin{itemize}
\item {Utilização:Prov.}
\end{itemize}
\begin{itemize}
\item {Utilização:minh.}
\end{itemize}
\begin{itemize}
\item {Utilização:Ant.}
\end{itemize}
Pequeno lago.
Charco.
Belga grande, com água de lima.
Lameiro.
O mesmo que \textunderscore galé\textunderscore ^1. Cf. \textunderscore Conquista do Peru\textunderscore , IV.
(B. lat. \textunderscore lagona\textunderscore , do lat. \textunderscore lacuna\textunderscore )
\section{Lagocéfalo}
\begin{itemize}
\item {Grp. gram.:adj.}
\end{itemize}
\begin{itemize}
\item {Utilização:Zool.}
\end{itemize}
\begin{itemize}
\item {Proveniência:(Do gr. \textunderscore lagos\textunderscore  + \textunderscore kephale\textunderscore )}
\end{itemize}
Que tem cabeça semelhante á da lebre, isto é, que tem fendido o lábio superior.
\section{Lagocéphalo}
\begin{itemize}
\item {Grp. gram.:adj.}
\end{itemize}
\begin{itemize}
\item {Utilização:Zool.}
\end{itemize}
\begin{itemize}
\item {Proveniência:(Do gr. \textunderscore lagos\textunderscore  + \textunderscore kephale\textunderscore )}
\end{itemize}
Que tem cabeça semelhante á da lebre, isto é, que tem fendido o lábio superior.
\section{Lagoeiro}
\begin{itemize}
\item {Grp. gram.:m.}
\end{itemize}
\begin{itemize}
\item {Utilização:Pop.}
\end{itemize}
\begin{itemize}
\item {Proveniência:(Do lat. \textunderscore lacunarius\textunderscore )}
\end{itemize}
Charco.
Água pluvial estagnada.
Grande porção de água entornada.
\section{Lagoftalmia}
\begin{itemize}
\item {Grp. gram.:f.}
\end{itemize}
\begin{itemize}
\item {Proveniência:(T. registado nos diccionários, mas impróprio, devendo sêr substituído por \textunderscore lagoftalmo\textunderscore )}
\end{itemize}

\section{Lagoftalmo}
\begin{itemize}
\item {Grp. gram.:m.}
\end{itemize}
\begin{itemize}
\item {Proveniência:(Do gr. \textunderscore lagos\textunderscore , lebre, e \textunderscore ophthalmos\textunderscore , ôlho)}
\end{itemize}
Oclusão incompleta das pálpebras durante o somno.
\section{Lagomis}
\begin{itemize}
\item {Grp. gram.:m.}
\end{itemize}
\begin{itemize}
\item {Proveniência:(Do gr. \textunderscore lagos\textunderscore  + \textunderscore mus\textunderscore )}
\end{itemize}
Mamífero roedor, semelhante á lebre, e que vive na Sibéria.
\section{Lagomys}
\begin{itemize}
\item {Grp. gram.:m.}
\end{itemize}
\begin{itemize}
\item {Proveniência:(Do gr. \textunderscore lagos\textunderscore  + \textunderscore mus\textunderscore )}
\end{itemize}
Mammífero roedor, semelhante á lebre, e que vive na Sibéria.
\section{Lagópede}
\begin{itemize}
\item {Grp. gram.:m.}
\end{itemize}
\begin{itemize}
\item {Proveniência:(T. hybr., com que alguns designaram a ave lagopo)}
\end{itemize}
\begin{itemize}
\item {Proveniência:(Gr. \textunderscore lagos\textunderscore , \textunderscore lebre\textunderscore  + lat. \textunderscore pes\textunderscore , \textunderscore pedis\textunderscore , pé)}
\end{itemize}

\section{Lagophtalmia}
\begin{itemize}
\item {Grp. gram.:f.}
\end{itemize}
\begin{itemize}
\item {Proveniência:(T. registado nos diccionários, mas impróprio, devendo sêr substituído por \textunderscore lagophthalmo\textunderscore )}
\end{itemize}

\section{Lagophthalmo}
\begin{itemize}
\item {Grp. gram.:m.}
\end{itemize}
\begin{itemize}
\item {Proveniência:(Do gr. \textunderscore lagos\textunderscore , lebre, e \textunderscore ophthalmos\textunderscore , ôlho)}
\end{itemize}
Occlusão incompleta das pálpebras durante o somno.
\section{Lagopo}
\begin{itemize}
\item {Grp. gram.:m.}
\end{itemize}
\begin{itemize}
\item {Proveniência:(Lat. \textunderscore lagopus\textunderscore )}
\end{itemize}
Espécie de trevo.
Perdiz branca dos Alpes.
Gênero de aves, o mesmo que \textunderscore lagópode\textunderscore .
\section{Lagópode}
\begin{itemize}
\item {Grp. gram.:adj.}
\end{itemize}
\begin{itemize}
\item {Utilização:Bot.}
\end{itemize}
\begin{itemize}
\item {Grp. gram.:M.}
\end{itemize}
\begin{itemize}
\item {Proveniência:(Do gr. \textunderscore lagos\textunderscore  + \textunderscore pous\textunderscore , \textunderscore podos\textunderscore )}
\end{itemize}
Que tem patas semelhantes ás da lebre.
Diz-se dos órgãos vegetaes, cobertos de cotão ou pêlos, como o rhizoma de alguns fêtos.
Gênero de aves, próprias dos climas frios (\textunderscore lagopus mutus\textunderscore , Leach).
\section{Lagosta}
\begin{itemize}
\item {fónica:gôs}
\end{itemize}
\begin{itemize}
\item {Grp. gram.:f.}
\end{itemize}
\begin{itemize}
\item {Proveniência:(Do lat. \textunderscore lacusta\textunderscore )}
\end{itemize}
Crustáceo macruro, de antennas cylíndricas e compridas.
\section{Lagostim}
\begin{itemize}
\item {Grp. gram.:m.}
\end{itemize}
Pequena lagosta.
\section{Lagreta}
\begin{itemize}
\item {fónica:grê}
\end{itemize}
\begin{itemize}
\item {Grp. gram.:f.}
\end{itemize}
Peixe de Portugal.
\section{Lágria}
\begin{itemize}
\item {Grp. gram.:f.}
\end{itemize}
Gênero de insectos coleópteros heterómeros.
\section{Lagriário}
\begin{itemize}
\item {Grp. gram.:adj.}
\end{itemize}
\begin{itemize}
\item {Grp. gram.:M. pl.}
\end{itemize}
Relativo ou semelhante á lágria.
Tríbo de insectos, que tem por typo a lágria.
\section{Lagrifas}
\begin{itemize}
\item {Grp. gram.:f. pl.}
\end{itemize}
\begin{itemize}
\item {Utilização:Gír. do Pôrto.}
\end{itemize}
Olhos.
\section{Lágrima}
\begin{itemize}
\item {Grp. gram.:f.}
\end{itemize}
\begin{itemize}
\item {Utilização:Ext.}
\end{itemize}
\begin{itemize}
\item {Utilização:Fig.}
\end{itemize}
\begin{itemize}
\item {Utilização:Prov.}
\end{itemize}
\begin{itemize}
\item {Utilização:alent.}
\end{itemize}
\begin{itemize}
\item {Utilização:Pop.}
\end{itemize}
\begin{itemize}
\item {Proveniência:(Do lat. \textunderscore lacrima\textunderscore )}
\end{itemize}
Gota de humor límpido, que sái do ôlho e é determinada por uma causa phýsica ou por um abalo moral.
Gota de um líquido.
Objecto, que tem a forma dessa gota.
Pequena porção; um tudonada.
Resina ou goma, que apparece no tronco de algumas árvores, resultante da picada de certos insectos, e que prejudica as mesmas árvores.
Vinho, que se faz com o sumo, produzido só pelo pêso da uva, quando amontoada.
O mesmo que \textunderscore fúchsia\textunderscore .
\section{Lagrimação}
\begin{itemize}
\item {Grp. gram.:f.}
\end{itemize}
(V.lacrimação)
\section{Lagrimal}
\begin{itemize}
\item {Grp. gram.:m.  e  adj.}
\end{itemize}
(V.lacrimal)
\section{Lagrimal}
\begin{itemize}
\item {Grp. gram.:m.}
\end{itemize}
\begin{itemize}
\item {Proveniência:(De \textunderscore lágrima\textunderscore )}
\end{itemize}
Buraco, de 3 centímetros de diâmetro, feito nas barachas das salinas.
\section{Lagrimante}
\begin{itemize}
\item {Grp. gram.:adj.}
\end{itemize}
O mesmo que \textunderscore lacrimante\textunderscore .
\section{Lagrimar}
\begin{itemize}
\item {Grp. gram.:v. i.}
\end{itemize}
Deitar lágrimas; chorar. Cf. Camillo, \textunderscore Volcões\textunderscore , 59 e 135.
\section{Lágrimas-de-job}
\begin{itemize}
\item {Grp. gram.:f. pl.}
\end{itemize}
Planta gramínea da Índia portuguesa, (\textunderscore coix lacrima Jobi\textunderscore , Lin.).
\section{Lágrimas-de-napoleão}
\begin{itemize}
\item {Grp. gram.:f. pl.}
\end{itemize}
O mesmo que \textunderscore raios-de-júpiter\textunderscore .
\section{Lagrimatório}
\begin{itemize}
\item {Grp. gram.:adj.}
\end{itemize}
O mesmo que \textunderscore lacrimatório\textunderscore .
\section{Lagrimável}
\begin{itemize}
\item {Grp. gram.:adj.}
\end{itemize}
O mesmo que \textunderscore lacrimável\textunderscore :«\textunderscore ...nos theatros, onde se faz o auto lagrimável da apotheose do vício\textunderscore ». Camillo, \textunderscore Mulher Fatal\textunderscore , 194.
\section{Lagrimejamento}
\begin{itemize}
\item {Grp. gram.:m.}
\end{itemize}
Acto de lagrimejar.
\section{Lagrimejar}
\begin{itemize}
\item {Grp. gram.:v. i.}
\end{itemize}
Deitar algumas lágrimas; choramingar.
\section{Lagrimoso}
\begin{itemize}
\item {Grp. gram.:adj.}
\end{itemize}
\begin{itemize}
\item {Utilização:Mús.}
\end{itemize}
\begin{itemize}
\item {Proveniência:(De \textunderscore lágrima\textunderscore )}
\end{itemize}
O mesmo que \textunderscore lacrimoso\textunderscore :«\textunderscore ...das matronas impollutas e das velhas lagrimosas\textunderscore ». Camillo, \textunderscore Mulher Fatal\textunderscore , 196.
Diz-se do andamento triste, e lento.
\section{Lagueiro}
\begin{itemize}
\item {fónica:gu-ei}
\end{itemize}
\begin{itemize}
\item {Grp. gram.:m.}
\end{itemize}
\begin{itemize}
\item {Proveniência:(De \textunderscore lago\textunderscore )}
\end{itemize}
Mólho de linho, com as raízes todas para um lado, como os que se metem na água, antes de serem maçados e espadelados.
Cp. \textunderscore lagoeiro\textunderscore .
\section{Lagumeiro}
\begin{itemize}
\item {Grp. gram.:m.}
\end{itemize}
\begin{itemize}
\item {Utilização:Prov.}
\end{itemize}
O mesmo que \textunderscore lamegueiro\textunderscore .
\section{Laguna}
\begin{itemize}
\item {Grp. gram.:f.}
\end{itemize}
\begin{itemize}
\item {Proveniência:(Do lat. \textunderscore lacuna\textunderscore )}
\end{itemize}
Canal entre ilhas ou entre bancos de areia.
Variedade de cafeeiro.
\section{Laia}
\begin{itemize}
\item {Grp. gram.:f.}
\end{itemize}
\begin{itemize}
\item {Utilização:Pop.}
\end{itemize}
\begin{itemize}
\item {Utilização:Ant.}
\end{itemize}
\begin{itemize}
\item {Utilização:Prov.}
\end{itemize}
\begin{itemize}
\item {Utilização:minh.}
\end{itemize}
Qualidade; casta; modo; feitio: \textunderscore amigos desta laia, dispenso-os\textunderscore .
Lan.
Tecido de lan, semelhante ao fio de Escócia.
(Do germ.?)
\section{Laia}
\begin{itemize}
\item {Grp. gram.:f.}
\end{itemize}
\begin{itemize}
\item {Utilização:Gír.}
\end{itemize}
Prata.
\section{Laical}
\begin{itemize}
\item {Grp. gram.:adj.}
\end{itemize}
\begin{itemize}
\item {Proveniência:(De \textunderscore laico\textunderscore )}
\end{itemize}
Relativo a leigo, próprio de leigo.
Que não diz respeito á classe ecclesiástica: \textunderscore ensino laical\textunderscore .
\section{Laicalismo}
\begin{itemize}
\item {Grp. gram.:m.}
\end{itemize}
Procedimento laical.
Attriuições estranhas ao poder ecclesiástico.
\section{Laicificar}
\begin{itemize}
\item {Grp. gram.:v. t.}
\end{itemize}
\begin{itemize}
\item {Utilização:Neol.}
\end{itemize}
\begin{itemize}
\item {Proveniência:(Do lat. \textunderscore laicus\textunderscore  + \textunderscore facere\textunderscore )}
\end{itemize}
Tornar leigo ou laical.
\section{Laicismo}
\begin{itemize}
\item {Grp. gram.:m.}
\end{itemize}
\begin{itemize}
\item {Proveniência:(Do lat. \textunderscore laicus\textunderscore )}
\end{itemize}
Systema dos que pretendem para os leigos o govêrno ecclesiástico.
\section{Laico}
\begin{itemize}
\item {Grp. gram.:adj.}
\end{itemize}
\begin{itemize}
\item {Proveniência:(Lat. \textunderscore laicus\textunderscore )}
\end{itemize}
O mesmo que \textunderscore leigo\textunderscore .
Secular, (por opposição a \textunderscore ecclesiástico\textunderscore ): \textunderscore ensino laico\textunderscore .
\section{Laidamento}
\begin{itemize}
\item {Grp. gram.:m.}
\end{itemize}
\begin{itemize}
\item {Utilização:Ant.}
\end{itemize}
\begin{itemize}
\item {Proveniência:(De \textunderscore laidar\textunderscore )}
\end{itemize}
Ferida, contusão.
\section{Laidar}
\begin{itemize}
\item {Grp. gram.:v. t.}
\end{itemize}
\begin{itemize}
\item {Utilização:Ant.}
\end{itemize}
\begin{itemize}
\item {Proveniência:(Do lat. \textunderscore laedere\textunderscore )}
\end{itemize}
Causar lesão a; ferir; contundir.
\section{Laido}
\begin{itemize}
\item {Grp. gram.:adj.}
\end{itemize}
\begin{itemize}
\item {Utilização:Ant.}
\end{itemize}
\begin{itemize}
\item {Proveniência:(Do fr. \textunderscore laid\textunderscore )}
\end{itemize}
Feio.
\section{Laidrar}
\begin{itemize}
\item {Grp. gram.:v. i.}
\end{itemize}
\begin{itemize}
\item {Utilização:T. de Canavezes}
\end{itemize}
O mesmo que \textunderscore ladrar\textunderscore ^1.
\section{Lais}
\begin{itemize}
\item {Grp. gram.:m.}
\end{itemize}
\begin{itemize}
\item {Grp. gram.:Pl.}
\end{itemize}
Ponta da vêrga, num navio.
Laises.
\section{Laiva}
\begin{itemize}
\item {Grp. gram.:f.}
\end{itemize}
\begin{itemize}
\item {Utilização:Prov.}
\end{itemize}
\begin{itemize}
\item {Utilização:alg.}
\end{itemize}
O mesmo que \textunderscore lábia\textunderscore ^1.
\section{Laivar}
\begin{itemize}
\item {Grp. gram.:v. t.}
\end{itemize}
Pôr laivos em.
Sujar; besuntar.
\section{Laivo}
\begin{itemize}
\item {Grp. gram.:m.}
\end{itemize}
\begin{itemize}
\item {Utilização:Fig.}
\end{itemize}
\begin{itemize}
\item {Utilização:Gír.}
\end{itemize}
\begin{itemize}
\item {Proveniência:(Do lat. \textunderscore abes\textunderscore ?)}
\end{itemize}
Mancha; nódoa.
Veio.
Rudimentos, ligeiras noções.
Lenço.
\section{Laivoso}
\begin{itemize}
\item {Grp. gram.:adj.}
\end{itemize}
\begin{itemize}
\item {Utilização:Prov.}
\end{itemize}
\begin{itemize}
\item {Utilização:alg.}
\end{itemize}
Que tem laiva; em e há laiva.
\section{Laja}
\begin{itemize}
\item {Grp. gram.:f.}
\end{itemize}
O mesmo ou melhor que \textunderscore lage\textunderscore .
\section{Lajão}
\begin{itemize}
\item {Grp. gram.:m.}
\end{itemize}
Laja grande.
\section{Lajeira}
\begin{itemize}
\item {Grp. gram.:f.}
\end{itemize}
\begin{itemize}
\item {Utilização:Prov.}
\end{itemize}
\begin{itemize}
\item {Utilização:beir.}
\end{itemize}
Laja ampla e de superficie lisa.
\section{Lalopathia}
\begin{itemize}
\item {Grp. gram.:f.}
\end{itemize}
\begin{itemize}
\item {Utilização:Med.}
\end{itemize}
\begin{itemize}
\item {Proveniência:(Do gr. \textunderscore lalein\textunderscore  + \textunderscore pathos\textunderscore )}
\end{itemize}
Designação genérica das perturbações da palavra.
\section{Lalopatia}
\begin{itemize}
\item {Grp. gram.:f.}
\end{itemize}
\begin{itemize}
\item {Utilização:Med.}
\end{itemize}
\begin{itemize}
\item {Proveniência:(Do gr. \textunderscore lalein\textunderscore  + \textunderscore pathos\textunderscore )}
\end{itemize}
Designação genérica das perturbações da palavra.
\section{Lama}
\begin{itemize}
\item {Grp. gram.:f.}
\end{itemize}
\begin{itemize}
\item {Utilização:Fig.}
\end{itemize}
\begin{itemize}
\item {Grp. gram.:M.}
\end{itemize}
\begin{itemize}
\item {Utilização:Pop.}
\end{itemize}
\begin{itemize}
\item {Grp. gram.:Pl.}
\end{itemize}
\begin{itemize}
\item {Proveniência:(Lat. \textunderscore lama\textunderscore )}
\end{itemize}
Mistura de terra com água.
Lodo.
Labéu; miséria.
Homem imbecil, fraco, inhenho.
Lodo ou sedimentos de nascentes mineraes, utilizados therapeuticamente.
\section{Lama}
\begin{itemize}
\item {Grp. gram.:m.}
\end{itemize}
\begin{itemize}
\item {Proveniência:(Fr. \textunderscore lama\textunderscore )}
\end{itemize}
Espécie de camelo sem corcova, e mais pequeno que o camelo typo.
\section{Lama}
\begin{itemize}
\item {Grp. gram.:m.}
\end{itemize}
\begin{itemize}
\item {Proveniência:(Do thibet. \textunderscore blama\textunderscore , com \textunderscore b\textunderscore  mudo)}
\end{itemize}
Sacerdote budhista, na Mongólia e no Thibet.
\textunderscore Grão lama\textunderscore , chefe da religião budhista.
\section{Lama}
\begin{itemize}
\item {Grp. gram.:f.}
\end{itemize}
\begin{itemize}
\item {Utilização:Gír.}
\end{itemize}
Prata.
\section{Lamaçal}
\begin{itemize}
\item {Grp. gram.:m.}
\end{itemize}
\begin{itemize}
\item {Proveniência:(De \textunderscore lama\textunderscore ^1)}
\end{itemize}
Lugar, em que há muita lama; lameiro; lodaçal; atoleiro.
\section{Lamação}
\begin{itemize}
\item {Grp. gram.:m.}
\end{itemize}
(V. \textunderscore lamarão\textunderscore ^1)
\section{Lamaceira}
\begin{itemize}
\item {Grp. gram.:f.}
\end{itemize}
O mesmo que \textunderscore lamaçal\textunderscore .
\section{Lamaceiro}
\begin{itemize}
\item {Grp. gram.:m.}
\end{itemize}
O mesmo que \textunderscore lamaçal\textunderscore .
\section{Lamacento}
\begin{itemize}
\item {Grp. gram.:adj.}
\end{itemize}
\begin{itemize}
\item {Proveniência:(De \textunderscore lamaçal\textunderscore )}
\end{itemize}
Em que há muita lama; lodoso.
Relativo a lama.
Que tem semelhança com a lama.
\section{Lamagem}
\begin{itemize}
\item {Grp. gram.:f.}
\end{itemize}
\begin{itemize}
\item {Utilização:T. de Vianna}
\end{itemize}
\begin{itemize}
\item {Proveniência:(De \textunderscore lama\textunderscore ^1)}
\end{itemize}
Serviço de lamageiro.
\section{Lamagueiro}
\begin{itemize}
\item {Grp. gram.:m.}
\end{itemize}
\begin{itemize}
\item {Utilização:T. de Vianna}
\end{itemize}
\begin{itemize}
\item {Proveniência:(De \textunderscore lamagem\textunderscore )}
\end{itemize}
Pescador da Ribeira.
Marítimo, que auxilia os pilotos da barra.
\section{Lamaico}
\begin{itemize}
\item {Grp. gram.:adj.}
\end{itemize}
\begin{itemize}
\item {Proveniência:(De \textunderscore lama\textunderscore ^3)}
\end{itemize}
Relativo ao lamaísmo.
\section{Lamaísmo}
\begin{itemize}
\item {Grp. gram.:m.}
\end{itemize}
\begin{itemize}
\item {Proveniência:(De \textunderscore lama\textunderscore ^3)}
\end{itemize}
Doutrina dos lamaístas.
\section{Lamaísta}
\begin{itemize}
\item {Grp. gram.:m.}
\end{itemize}
\begin{itemize}
\item {Proveniência:(De \textunderscore lama\textunderscore ^3)}
\end{itemize}
Adorador do grão-lama.
\section{Lamantim}
\begin{itemize}
\item {Grp. gram.:m.}
\end{itemize}
\begin{itemize}
\item {Utilização:Zool.}
\end{itemize}
Gênero de vivíparos amphíbios, que, segundo alguns críticos, determinaram a lenda dos tritões, sereias, etc.
(Cast. \textunderscore lamantin\textunderscore )
\section{Lamaquitos}
\begin{itemize}
\item {Grp. gram.:m. pl.}
\end{itemize}
Tríbo indígena de Timor.
\section{Lamarão}
\begin{itemize}
\item {Grp. gram.:m.}
\end{itemize}
\begin{itemize}
\item {Proveniência:(Do rad. de \textunderscore lama\textunderscore )}
\end{itemize}
Grande lamaçal.
Grande porção de lodo, que a baixa-mar deixa a descoberto.
\section{Lamarão}
\begin{itemize}
\item {Grp. gram.:m.}
\end{itemize}
\begin{itemize}
\item {Utilização:T. da Bairrada}
\end{itemize}
Grande mentira.
Mexerico, bisbilhotice.
\section{Lamárckia}
\begin{itemize}
\item {Grp. gram.:f.}
\end{itemize}
\begin{itemize}
\item {Proveniência:(De \textunderscore Lamarck\textunderscore , n. p.)}
\end{itemize}
Gênero de plantas gramíneas.
\section{Lamarckista}
\begin{itemize}
\item {Grp. gram.:adj.}
\end{itemize}
\begin{itemize}
\item {Grp. gram.:M.}
\end{itemize}
Relativo ao naturalista Lamarck, precursor de Darwin.
Sectário de Lamarck.
\section{Lamarento}
\begin{itemize}
\item {Grp. gram.:adj.}
\end{itemize}
O mesmo que \textunderscore lamacento\textunderscore . Cf. Filinto, \textunderscore D. Man.\textunderscore , I, 265.
\section{Lamaroso}
\begin{itemize}
\item {Grp. gram.:adj.}
\end{itemize}
O mesmo que \textunderscore lamacento\textunderscore .
\section{Lamartiniano}
\begin{itemize}
\item {Grp. gram.:adj.}
\end{itemize}
Relativo a Lamartine; que tem a feição literária de Lamartine: \textunderscore lyrismo lamartiniano\textunderscore .
\section{Lamartinista}
\begin{itemize}
\item {Grp. gram.:m.  e  f.}
\end{itemize}
Pessôa, que imita ou aprecia muito o gênero literário de Lamartine. Cf. Th. Braga, \textunderscore Mod. Ideias\textunderscore , I, 250.
\section{Lamba}
\begin{itemize}
\item {Grp. gram.:m.}
\end{itemize}
\textunderscore Chorar o lamba\textunderscore , chorar com pieguice. Cf. Camillo, \textunderscore Corja\textunderscore , 174.
(Quimb. \textunderscore lamba\textunderscore , desventura)
\section{Lambaças}
\begin{itemize}
\item {Grp. gram.:m.}
\end{itemize}
\begin{itemize}
\item {Utilização:Prov.}
\end{itemize}
\begin{itemize}
\item {Utilização:trasm.}
\end{itemize}
O mesmo que \textunderscore comilão\textunderscore . (Colhido em Alijó)
(Cp. \textunderscore lambaz\textunderscore )
\section{Lambaceiro}
\begin{itemize}
\item {Grp. gram.:adj.}
\end{itemize}
\begin{itemize}
\item {Utilização:T. de Turquel}
\end{itemize}
O mesmo que \textunderscore lambaz\textunderscore .
\section{Lambada}
\begin{itemize}
\item {Grp. gram.:f.}
\end{itemize}
\begin{itemize}
\item {Utilização:Chul.}
\end{itemize}
\begin{itemize}
\item {Utilização:Fig.}
\end{itemize}
O mesmo que \textunderscore paulada\textunderscore .
Descompostura.
(Por \textunderscore lombada\textunderscore , de \textunderscore lombo\textunderscore ?)
\section{Lambamba}
\begin{itemize}
\item {Grp. gram.:m.}
\end{itemize}
\begin{itemize}
\item {Utilização:Bras}
\end{itemize}
Beberrão de cachaça.
\section{Lambança}
\begin{itemize}
\item {Grp. gram.:f.}
\end{itemize}
\begin{itemize}
\item {Utilização:Pop.}
\end{itemize}
\begin{itemize}
\item {Utilização:pop.}
\end{itemize}
\begin{itemize}
\item {Utilização:Prolóq.}
\end{itemize}
\begin{itemize}
\item {Utilização:Bras. do N}
\end{itemize}
\begin{itemize}
\item {Proveniência:(Do cast. \textunderscore alabanza\textunderscore )}
\end{itemize}
Coisa que se póde lamber ou comer.
\textunderscore Muita chança e pouca lambança\textunderscore , muita bazófia e poucos teres.
Bazófia, jactância.
Trapaça.
\section{Lambanceiro}
\begin{itemize}
\item {Grp. gram.:m.}
\end{itemize}
\begin{itemize}
\item {Utilização:Bras. do N}
\end{itemize}
\begin{itemize}
\item {Proveniência:(De \textunderscore lambança\textunderscore )}
\end{itemize}
Homem vaidoso; jactancioso.
Trapaceiro.
\section{Lambão}
\begin{itemize}
\item {Grp. gram.:m.}
\end{itemize}
\begin{itemize}
\item {Proveniência:(Do rad. de \textunderscore lamber\textunderscore )}
\end{itemize}
Aquelle que é guloso; lambareiro; glotão; comilão.
\section{Lambarar}
\begin{itemize}
\item {Grp. gram.:v. i.}
\end{itemize}
\begin{itemize}
\item {Proveniência:(Do rad. de \textunderscore lambareiro\textunderscore )}
\end{itemize}
Comer lambarices; gostar de lambarices.
\section{Lambaraz}
\begin{itemize}
\item {Grp. gram.:m.}
\end{itemize}
O mesmo que \textunderscore lambareiro\textunderscore .
\section{Lambareada}
\begin{itemize}
\item {Grp. gram.:f.}
\end{itemize}
\begin{itemize}
\item {Utilização:Prov.}
\end{itemize}
\begin{itemize}
\item {Utilização:alg.}
\end{itemize}
Loquacidade; parola.
\section{Lambareiro}
\begin{itemize}
\item {Grp. gram.:m.  e  adj.}
\end{itemize}
\begin{itemize}
\item {Utilização:Náut.}
\end{itemize}
Aquelle que é guloso.
Aquelle que gosta de lambarices.
Chocalheiro.
Cabo náutico, limitado de um lado por um sapatilho e do outro por um gato.
Talha, destinada a pôr as âncoras na posição horizontal. Cp. \textunderscore turco\textunderscore .
\section{Lambarejar}
\begin{itemize}
\item {Grp. gram.:v. i.}
\end{itemize}
\begin{itemize}
\item {Utilização:Pop.}
\end{itemize}
\begin{itemize}
\item {Proveniência:(De \textunderscore lambareiro\textunderscore )}
\end{itemize}
Provar comidas, remexendo-as, e deixando os restos embodegados.
\section{Lambari}
\begin{itemize}
\item {Grp. gram.:m.}
\end{itemize}
Peixe do Brasil.
\section{Lambarice}
\begin{itemize}
\item {Grp. gram.:f.}
\end{itemize}
\begin{itemize}
\item {Proveniência:(Do rad. de \textunderscore lambareiro\textunderscore )}
\end{itemize}
Gulodice.
Qualidade de quem é lambareiro.
\section{Lambariscar}
\begin{itemize}
\item {Grp. gram.:v. i.}
\end{itemize}
\begin{itemize}
\item {Utilização:Prov.}
\end{itemize}
O mesmo que \textunderscore lambarejar\textunderscore .
\section{Lambaz}
\begin{itemize}
\item {Grp. gram.:adj.}
\end{itemize}
\begin{itemize}
\item {Utilização:Chul.}
\end{itemize}
\begin{itemize}
\item {Utilização:Prov.}
\end{itemize}
\begin{itemize}
\item {Proveniência:(Do rad. de \textunderscore lamber\textunderscore )}
\end{itemize}
Que é lambão; glotão.
Vassoiro de cordas, usado a bordo.
Tejolo grosso.
\section{Lambazar}
\begin{itemize}
\item {Grp. gram.:v.}
\end{itemize}
\begin{itemize}
\item {Utilização:t. Náut.}
\end{itemize}
Enxugar ou varrer com o lambaz ou vassoiro de bordo.
\section{Lambda}
\begin{itemize}
\item {Grp. gram.:m.}
\end{itemize}
\begin{itemize}
\item {Utilização:Anat.}
\end{itemize}
Nome da letra, que no alphabeto grego corresponde ao nosso \textunderscore l\textunderscore .
A sutura lambdoídea.
\section{Lambdacismo}
\begin{itemize}
\item {Grp. gram.:m.}
\end{itemize}
\begin{itemize}
\item {Proveniência:(De \textunderscore lambda\textunderscore )}
\end{itemize}
Pronúncia viciosa da letra \textunderscore l\textunderscore .
Substituição, na pronúncia, do \textunderscore l\textunderscore  por \textunderscore r\textunderscore .
\section{Lâmbdático}
\begin{itemize}
\item {Grp. gram.:adj.}
\end{itemize}
\begin{itemize}
\item {Utilização:Neol.}
\end{itemize}
O mesmo que \textunderscore lâmbdico\textunderscore .
\section{Lâmbdico}
\begin{itemize}
\item {Grp. gram.:adj.}
\end{itemize}
\begin{itemize}
\item {Utilização:Anat.}
\end{itemize}
\begin{itemize}
\item {Proveniência:(De \textunderscore lambda\textunderscore )}
\end{itemize}
Diz-se do ângulo póstero superior dos parietaes.
\section{Lambdoide}
\begin{itemize}
\item {Grp. gram.:adj. f.}
\end{itemize}
\begin{itemize}
\item {Proveniência:(Do gr. \textunderscore lambda\textunderscore  + \textunderscore eidos\textunderscore )}
\end{itemize}
Diz-se da sutura occípito-parietal.
\section{Lambdoídea}
\begin{itemize}
\item {Grp. gram.:adj. f.}
\end{itemize}
\begin{itemize}
\item {Proveniência:(Do gr. \textunderscore lambda\textunderscore  + \textunderscore eidos\textunderscore )}
\end{itemize}
Diz-se da sutura occípito-parietal.
\section{Lambear}
\begin{itemize}
\item {Grp. gram.:v. t.}
\end{itemize}
\begin{itemize}
\item {Utilização:Chul.}
\end{itemize}
\begin{itemize}
\item {Proveniência:(Do rad. de \textunderscore lamber\textunderscore )}
\end{itemize}
Comer soffregamente; devorar.
\section{Lambeato}
\begin{itemize}
\item {Grp. gram.:m.}
\end{itemize}
\begin{itemize}
\item {Utilização:Ant.}
\end{itemize}
\begin{itemize}
\item {Proveniência:(Do rad. de \textunderscore lamber\textunderscore )}
\end{itemize}
Lambarice, guloseima.
\section{Lambeche}
\begin{itemize}
\item {Grp. gram.:m.}
\end{itemize}
Espécie de charrua, no Riba-Tejo.
\section{Lambedela}
\begin{itemize}
\item {Grp. gram.:f.}
\end{itemize}
\begin{itemize}
\item {Utilização:Fig.}
\end{itemize}
Acto ou effeito de lamber.
Lisonja.
Gorgeta; pechincha.
\section{Lambédine}
\begin{itemize}
\item {Grp. gram.:m.}
\end{itemize}
\begin{itemize}
\item {Utilização:Fam.}
\end{itemize}
Guloseima; bom petisco.
(Relaciona-se com \textunderscore lamber\textunderscore )
\section{Lambedor}
\begin{itemize}
\item {Grp. gram.:adj.}
\end{itemize}
\begin{itemize}
\item {Grp. gram.:M.}
\end{itemize}
Que lambe.
Aquelle que lambe.
Xarope, feito de açúcar dissolvido em suco de flôres ou frutos.
Coisa dôce.
\section{Lambedura}
\begin{itemize}
\item {Grp. gram.:f.}
\end{itemize}
O mesmo que \textunderscore lambedela\textunderscore .
\section{Lambefe}
\begin{itemize}
\item {Grp. gram.:m.}
\end{itemize}
\begin{itemize}
\item {Utilização:Prov.}
\end{itemize}
\begin{itemize}
\item {Utilização:trasm.}
\end{itemize}
Tabefe; lambada.
\section{Lambeirão}
\begin{itemize}
\item {Grp. gram.:m.}
\end{itemize}
\begin{itemize}
\item {Utilização:T. do Fundão}
\end{itemize}
Aquelle que, por preguiça, não faz nada.
\section{Lambeiro}
\begin{itemize}
\item {Grp. gram.:m.  e  adj.}
\end{itemize}
Aquelle que lambe.
Aquelle que é lambareiro.
\section{Lâmbel}
\begin{itemize}
\item {Grp. gram.:m.}
\end{itemize}
\begin{itemize}
\item {Utilização:Ant.}
\end{itemize}
Faixa, banda.
Cotica de brasão.
Pano listrado, com que se cobriam bancos.
Lençaria de pano listrado.
(Cp. \textunderscore alâmbel\textunderscore )
\section{Lambe-lhe-os-dedos}
\begin{itemize}
\item {Grp. gram.:f.}
\end{itemize}
Variedade de pêra, também conhecida por amorim ou amerim.
\section{Lambe-pratos}
\begin{itemize}
\item {Grp. gram.:m.}
\end{itemize}
\begin{itemize}
\item {Utilização:Fam.}
\end{itemize}
Indivíduo guloso; glotão; lambaz.
\section{Lamber}
\begin{itemize}
\item {Grp. gram.:v. t.}
\end{itemize}
\begin{itemize}
\item {Utilização:Fig.}
\end{itemize}
\begin{itemize}
\item {Proveniência:(Lat. \textunderscore lambere\textunderscore )}
\end{itemize}
Passar a língua sôbre: \textunderscore lamber os pratos\textunderscore .
Tocar levemente.
Polir.
Apurar excessivamente.
Desgastar.
Comer soffregamente; devorar: \textunderscore lambeu tudo que lhe puseram na mesa\textunderscore .
\section{Lambeta}
\begin{itemize}
\item {fónica:bê}
\end{itemize}
\begin{itemize}
\item {Grp. gram.:f.}
\end{itemize}
\begin{itemize}
\item {Utilização:Prov.}
\end{itemize}
\begin{itemize}
\item {Utilização:minh.}
\end{itemize}
\begin{itemize}
\item {Utilização:Pop.}
\end{itemize}
Guloseima.
Coisa de pouca duração.
Pequena refeição.
(Cp. \textunderscore lamber\textunderscore )
\section{Lambião}
\begin{itemize}
\item {Grp. gram.:m.}
\end{itemize}
\begin{itemize}
\item {Utilização:Açor}
\end{itemize}
Labareda; chammarela.
(Corr. de \textunderscore lampião\textunderscore ?)
\section{Lambicada}
\begin{itemize}
\item {Grp. gram.:f.}
\end{itemize}
\begin{itemize}
\item {Utilização:Prov.}
\end{itemize}
Aguardente, que o lambique produz de cada vez.
\section{Lambida}
\begin{itemize}
\item {Grp. gram.:f.}
\end{itemize}
O mesmo que \textunderscore lambedela\textunderscore .
Jôgo popular, espécie de bisca.
\section{Lambidela}
\begin{itemize}
\item {Grp. gram.:f.}
\end{itemize}
O mesmo que \textunderscore lambedela\textunderscore .
\section{Lambique}
\begin{itemize}
\item {Grp. gram.:m.}
\end{itemize}
O mesmo que \textunderscore alambique\textunderscore .
\section{Lambiscar}
\begin{itemize}
\item {Grp. gram.:v. t.}
\end{itemize}
\begin{itemize}
\item {Utilização:Pop.}
\end{itemize}
\begin{itemize}
\item {Proveniência:(De \textunderscore lambisco\textunderscore )}
\end{itemize}
Comer pouco, debicar.
\section{Lambisco}
\begin{itemize}
\item {Grp. gram.:m.}
\end{itemize}
\begin{itemize}
\item {Utilização:Pop.}
\end{itemize}
\begin{itemize}
\item {Proveniência:(Do rad. de \textunderscore lamber\textunderscore )}
\end{itemize}
Pequena porção de comida.
Pouca coisa.
\section{Lambisgóia}
\begin{itemize}
\item {Grp. gram.:f.}
\end{itemize}
\begin{itemize}
\item {Proveniência:(Do rad. de \textunderscore lamber\textunderscore )}
\end{itemize}
Mulher delambida, intrometida, mexeriqueira.
\section{Lambisqueiro}
\begin{itemize}
\item {Grp. gram.:m.  e  adj.}
\end{itemize}
\begin{itemize}
\item {Proveniência:(De \textunderscore lambisco\textunderscore )}
\end{itemize}
Aquelle que lambisca ou debica.
Lambareiro.
\section{Lambitana}
\begin{itemize}
\item {Grp. gram.:f.}
\end{itemize}
\begin{itemize}
\item {Utilização:Prov.}
\end{itemize}
\begin{itemize}
\item {Utilização:trasm.}
\end{itemize}
O mesmo que \textunderscore lambisgóia\textunderscore .
\section{Lambitão}
\begin{itemize}
\item {Grp. gram.:m.  e  adj.}
\end{itemize}
\begin{itemize}
\item {Utilização:Prov.}
\end{itemize}
\begin{itemize}
\item {Utilização:trasm.}
\end{itemize}
O mesmo que \textunderscore lambareiro\textunderscore .
(Cp. \textunderscore lambitar\textunderscore )
\section{Lambitar}
\begin{itemize}
\item {Grp. gram.:v. i.}
\end{itemize}
\begin{itemize}
\item {Utilização:Prov.}
\end{itemize}
\begin{itemize}
\item {Utilização:trasm.}
\end{itemize}
O mesmo que \textunderscore gulosar\textunderscore .
(Cp. \textunderscore lamber\textunderscore )
\section{Lambiteiro}
\begin{itemize}
\item {Grp. gram.:adj.}
\end{itemize}
\begin{itemize}
\item {Utilização:Prov.}
\end{itemize}
\begin{itemize}
\item {Utilização:trasm.}
\end{itemize}
\begin{itemize}
\item {Proveniência:(De \textunderscore lambitar\textunderscore )}
\end{itemize}
Lambareiro; guloso.
\section{Lambodas}
\begin{itemize}
\item {fónica:bô}
\end{itemize}
\begin{itemize}
\item {Grp. gram.:m.  e  f.}
\end{itemize}
\begin{itemize}
\item {Utilização:Prov.}
\end{itemize}
\begin{itemize}
\item {Utilização:trasm.}
\end{itemize}
Pessôa suja, immunda. (Colhido na Régua)
(Cp. \textunderscore lambodes\textunderscore )
\section{Lambodes}
\begin{itemize}
\item {Grp. gram.:m.}
\end{itemize}
\begin{itemize}
\item {Utilização:ant.}
\end{itemize}
\begin{itemize}
\item {Utilização:Chul.}
\end{itemize}
O mesmo que \textunderscore comilão\textunderscore .
(Cp. \textunderscore lambão\textunderscore )
\section{Lambra}
\begin{itemize}
\item {Grp. gram.:f.}
\end{itemize}
\begin{itemize}
\item {Utilização:Prov.}
\end{itemize}
\begin{itemize}
\item {Utilização:trasm.}
\end{itemize}
O mesmo que \textunderscore fome\textunderscore .
(Cp. cast. \textunderscore hambre\textunderscore )
\section{Lambrequins}
\begin{itemize}
\item {Grp. gram.:m. pl.}
\end{itemize}
\begin{itemize}
\item {Proveniência:(Fr. \textunderscore lambrequins\textunderscore )}
\end{itemize}
Antiga cobertura do elmo.
Ornato, que desce do elmo sôbre o escudo.
Recorte de pano ou madeira, para enfeite de pavilhão, estore, cantoneira, etc.
\section{Lambresto}
\begin{itemize}
\item {Grp. gram.:m.}
\end{itemize}
(V.lampsana)
\section{Lambrete}
\begin{itemize}
\item {fónica:brê}
\end{itemize}
\begin{itemize}
\item {Grp. gram.:m.}
\end{itemize}
\begin{itemize}
\item {Utilização:Náut.}
\end{itemize}
Régua estreita, que se amarra a cada uma das peças do apparelho da embarcação, quando esta desarma.
\section{Lambrião}
\begin{itemize}
\item {Grp. gram.:m.}
\end{itemize}
\begin{itemize}
\item {Utilização:T. da Bairrada}
\end{itemize}
\textunderscore Ficar em lambrião\textunderscore , ficar logrado numa espectativa, ficar a vêr navios.
\section{Lambris}
\begin{itemize}
\item {Grp. gram.:m. pl.}
\end{itemize}
\begin{itemize}
\item {Utilização:Neol.}
\end{itemize}
\begin{itemize}
\item {Proveniência:(Fr. \textunderscore lambris\textunderscore )}
\end{itemize}
Madeira lavrada, mármore ou estuque, com que se revestem as paredes ou o tecto de uma sala.
\section{Lambrusca}
\begin{itemize}
\item {Grp. gram.:f.}
\end{itemize}
O mesmo que \textunderscore labrusca\textunderscore .
(Cast. \textunderscore lambrusca\textunderscore )
\section{Lambuça}
\begin{itemize}
\item {Grp. gram.:f.}
\end{itemize}
(V.lambuzadela)
\section{Lambuçadela}
\begin{itemize}
\item {Grp. gram.:f.}
\end{itemize}
(V.lambuzadela)
\section{Lambuçar}
\begin{itemize}
\item {Grp. gram.:v. i.}
\end{itemize}
(V.lambuzar)
\section{Lambujar}
\begin{itemize}
\item {Grp. gram.:v. i.}
\end{itemize}
\begin{itemize}
\item {Utilização:Pop.}
\end{itemize}
Andar á lambujem; lambarar.
\section{Lambujeiro}
\begin{itemize}
\item {Grp. gram.:m.  e  adj.}
\end{itemize}
Aquelle que lambuja.
\section{Lambujem}
\begin{itemize}
\item {Grp. gram.:f.}
\end{itemize}
\begin{itemize}
\item {Utilização:T. de oleiro}
\end{itemize}
\begin{itemize}
\item {Utilização:Bras}
\end{itemize}
\begin{itemize}
\item {Proveniência:(Do rad. de \textunderscore lamber\textunderscore )}
\end{itemize}
Gulodice.
Glotonaria.
Pequeno lucro, que serve de engodo a alguém.
Barro muito fino.
Gratificação, luvas.
\section{Lambujem}
\begin{itemize}
\item {Grp. gram.:f.}
\end{itemize}
Espécie de oliveira minhota.
\section{Lambuzada}
\begin{itemize}
\item {Grp. gram.:f.}
\end{itemize}
\begin{itemize}
\item {Utilização:Pop.}
\end{itemize}
\begin{itemize}
\item {Proveniência:(De \textunderscore lambuzar\textunderscore )}
\end{itemize}
Coisa que suja.
Lambedura.
\section{Lambuzadela}
\begin{itemize}
\item {Grp. gram.:f.}
\end{itemize}
\begin{itemize}
\item {Utilização:Fig.}
\end{itemize}
Lambedela.
Acto ou effeito de lambuzar.
Mancha de comida ou bebida.
Laivos, ligeiras noções: \textunderscore uma lambuzadela de metrificação\textunderscore .
\section{Lambuzão}
\begin{itemize}
\item {Grp. gram.:m.}
\end{itemize}
\begin{itemize}
\item {Utilização:Açor}
\end{itemize}
O mesmo que \textunderscore lobishomem\textunderscore .
(Corr. de \textunderscore lobishomem\textunderscore , sob a infl. de \textunderscore lambaz\textunderscore ?)
\section{Lambuzar}
\begin{itemize}
\item {Grp. gram.:v. t.}
\end{itemize}
\begin{itemize}
\item {Proveniência:(Do rad. de \textunderscore lamber\textunderscore )}
\end{itemize}
Pôr nódoas de gordura em.
Emporcalhar.
\section{Lambuzo}
\begin{itemize}
\item {Grp. gram.:m.}
\end{itemize}
\begin{itemize}
\item {Proveniência:(T. lund.)}
\end{itemize}
Arbusto africano, sarmentoso, de fôlhas oppostas, e flôres em longas espigas axillares.
\section{Lamecense}
\begin{itemize}
\item {Grp. gram.:adj.}
\end{itemize}
\begin{itemize}
\item {Grp. gram.:M.}
\end{itemize}
\begin{itemize}
\item {Proveniência:(Do lat. \textunderscore Lameca\textunderscore , n. p.)}
\end{itemize}
Relativo a Lamego.
Habitante de Lamego.
\section{Lamecha}
\begin{itemize}
\item {Grp. gram.:m.  e  adj.}
\end{itemize}
\begin{itemize}
\item {Utilização:Fam.}
\end{itemize}
O mesmo que \textunderscore bajoujo\textunderscore .
Namorador ridículo.
\section{Lã}
\begin{itemize}
\item {Grp. gram.:f.}
\end{itemize}
\begin{itemize}
\item {Utilização:Fam.}
\end{itemize}
\begin{itemize}
\item {Proveniência:(Do lat. \textunderscore lana\textunderscore )}
\end{itemize}
Pêlo macio, espêsso e longo, que cobre a pele dos carneiros e de outros animaes.
Tecido, feito desse pêlo.
Lanugem de certas plantas.
Carapinha.
Acanhamento, excessiva timidez.
\section{Lamechismo}
\begin{itemize}
\item {Grp. gram.:m.}
\end{itemize}
\begin{itemize}
\item {Utilização:Fam.}
\end{itemize}
Qualidade de lamecha.
\section{Lamécula}
\begin{itemize}
\item {Grp. gram.:f.}
\end{itemize}
\begin{itemize}
\item {Utilização:Neol.}
\end{itemize}
Pequena lâmina.
(Dem. pouco acceitável, formado sôbre o fr. \textunderscore lame\textunderscore )
\section{Lamecular}
\begin{itemize}
\item {Grp. gram.:adj.}
\end{itemize}
Que tem fórma de lamécula.
\section{Lameda}
\begin{itemize}
\item {fónica:mê}
\end{itemize}
\begin{itemize}
\item {Grp. gram.:f.}
\end{itemize}
(Corr. de \textunderscore alameda\textunderscore )
\section{Lamegão}
\begin{itemize}
\item {Grp. gram.:m.}
\end{itemize}
\begin{itemize}
\item {Utilização:Prov.}
\end{itemize}
\begin{itemize}
\item {Utilização:T. da Régua}
\end{itemize}
Homem encorpado e parvo.
Comilão.
\section{Lamego}
\begin{itemize}
\item {fónica:mê}
\end{itemize}
\begin{itemize}
\item {Grp. gram.:m.}
\end{itemize}
Arado de varredoiro, labrego.
(Corr. de \textunderscore labego\textunderscore )
\section{Lamegueiro}
\begin{itemize}
\item {Grp. gram.:m.}
\end{itemize}
\begin{itemize}
\item {Proveniência:(De \textunderscore Lamego\textunderscore , n. p.?)}
\end{itemize}
O mesmo que \textunderscore olmeiro\textunderscore . Cf. \textunderscore Bibl. da G. do Campo\textunderscore , 309.
\section{Lameira}
\begin{itemize}
\item {Grp. gram.:f.}
\end{itemize}
\begin{itemize}
\item {Proveniência:(De \textunderscore lama\textunderscore )}
\end{itemize}
Casta de uva trasmontana.
\section{Lameira}
\begin{itemize}
\item {Grp. gram.:f.}
\end{itemize}
O mesmo que \textunderscore lameiro\textunderscore .
\section{Lameiral}
\begin{itemize}
\item {Grp. gram.:m.}
\end{itemize}
Grande lameiro.
Série de lameiros. Cf. Camillo, \textunderscore Brasileira\textunderscore , 61.
\section{Lameirar}
\begin{itemize}
\item {Grp. gram.:v. t.}
\end{itemize}
Converter em lameiro; \textunderscore limar\textunderscore ^3. Cf. Assis Teix., \textunderscore Águas\textunderscore , 196.
\section{Lameirento}
\begin{itemize}
\item {Grp. gram.:adj.}
\end{itemize}
\begin{itemize}
\item {Proveniência:(De \textunderscore lameiro\textunderscore )}
\end{itemize}
Lamacento; pantanoso.
\section{Lameiro}
\begin{itemize}
\item {Grp. gram.:m.}
\end{itemize}
\begin{itemize}
\item {Proveniência:(De \textunderscore lama\textunderscore )}
\end{itemize}
Lamaçal; pântano.
Terra alagadiça, que produz muito pasto.
\section{Lameiró}
\begin{itemize}
\item {Grp. gram.:m.}
\end{itemize}
\begin{itemize}
\item {Utilização:T. da Bairrada}
\end{itemize}
Pequeno pássaro, de rabo curto, asas cinzentas e papo roxo.
O mesmo que \textunderscore cartaxo\textunderscore ?
\section{Lamejinha}
\begin{itemize}
\item {Grp. gram.:f.}
\end{itemize}
\begin{itemize}
\item {Utilização:Prov.}
\end{itemize}
\begin{itemize}
\item {Utilização:alent.}
\end{itemize}
Espécie de amêijoa.
\section{Lâmel}
\begin{itemize}
\item {Grp. gram.:m.}
\end{itemize}
\begin{itemize}
\item {Utilização:Ant.}
\end{itemize}
O mesmo que \textunderscore lâmbel\textunderscore .
\section{Lamela}
\begin{itemize}
\item {Grp. gram.:f.}
\end{itemize}
\begin{itemize}
\item {Utilização:Bot.}
\end{itemize}
\begin{itemize}
\item {Proveniência:(Lat. \textunderscore lamella\textunderscore )}
\end{itemize}
Lâmina pequena.
Membrana delgada.
Apêndice petaloide.
\section{Lamelação}
\begin{itemize}
\item {Grp. gram.:f.}
\end{itemize}
Acto ou efeito de lamelar.
\section{Lamelado}
\begin{itemize}
\item {Grp. gram.:adj.}
\end{itemize}
\begin{itemize}
\item {Proveniência:(De \textunderscore lamelar\textunderscore ^1)}
\end{itemize}
Que tem lamelas.
\section{Lamelar}
\begin{itemize}
\item {Grp. gram.:v. t.}
\end{itemize}
\begin{itemize}
\item {Proveniência:(De \textunderscore lamela\textunderscore )}
\end{itemize}
Guarnecer com lâminas; dividir em lâminas.
\section{Lamelar}
\begin{itemize}
\item {Grp. gram.:adj.}
\end{itemize}
O mesmo que \textunderscore lameloso\textunderscore .
\section{Lamelibrânquio}
\begin{itemize}
\item {Grp. gram.:adj.}
\end{itemize}
\begin{itemize}
\item {Proveniência:(De \textunderscore lamela\textunderscore  + \textunderscore branquias\textunderscore )}
\end{itemize}
Diz-se dos peixes, cujas brânquias têm a fórma de lâminas semicirculares.
\section{Lamelicórneo}
\begin{itemize}
\item {Grp. gram.:adj.}
\end{itemize}
\begin{itemize}
\item {Utilização:Zool.}
\end{itemize}
\begin{itemize}
\item {Grp. gram.:M. pl.}
\end{itemize}
\begin{itemize}
\item {Proveniência:(De \textunderscore lamela\textunderscore  + \textunderscore córneo\textunderscore )}
\end{itemize}
Que termina em massa folhosa, (falando-se de antenas).
Família de insectos coleópteros pentâmeros.
\section{Lamelífero}
\begin{itemize}
\item {Grp. gram.:adj.}
\end{itemize}
\begin{itemize}
\item {Grp. gram.:M. pl.}
\end{itemize}
\begin{itemize}
\item {Proveniência:(Do lat. \textunderscore lamella\textunderscore  + \textunderscore ferre\textunderscore )}
\end{itemize}
Que tem lâminas.
Família de polipeiros.
\section{Lameliforme}
\begin{itemize}
\item {Grp. gram.:adj.}
\end{itemize}
\begin{itemize}
\item {Proveniência:(De \textunderscore lamela\textunderscore  + \textunderscore forma\textunderscore )}
\end{itemize}
Que tem fórma de lâmina.
\section{Lamelinha}
\begin{itemize}
\item {Grp. gram.:f.}
\end{itemize}
\begin{itemize}
\item {Proveniência:(Do lat. \textunderscore lamella\textunderscore )}
\end{itemize}
Gênero de infusórios, caracterizados por seu pequeno corpo transparente em fórma de lâmina.
\section{Lamelípede}
\begin{itemize}
\item {Grp. gram.:adj.}
\end{itemize}
\begin{itemize}
\item {Utilização:Zool.}
\end{itemize}
\begin{itemize}
\item {Proveniência:(Do lat. \textunderscore lamella\textunderscore  + \textunderscore pes\textunderscore , \textunderscore pedis\textunderscore )}
\end{itemize}
Que tem pés achatados.
\section{Lamelirrostro}
\begin{itemize}
\item {Grp. gram.:adj.}
\end{itemize}
\begin{itemize}
\item {Utilização:Zool.}
\end{itemize}
\begin{itemize}
\item {Grp. gram.:M. pl.}
\end{itemize}
\begin{itemize}
\item {Proveniência:(De \textunderscore lamela\textunderscore  + \textunderscore rostro\textunderscore )}
\end{itemize}
Que tem o bico guarnecido de lâminas.
Aves palmípedes, cujo bico é guarnecido de lâminas córneas, em fórma de dentes.
\section{Lamella}
\begin{itemize}
\item {Grp. gram.:f.}
\end{itemize}
\begin{itemize}
\item {Utilização:Bot.}
\end{itemize}
\begin{itemize}
\item {Proveniência:(Lat. \textunderscore lamella\textunderscore )}
\end{itemize}
Lâmina pequena.
Membrana delgada.
Appêndice petaloide.
\section{Lamellação}
\begin{itemize}
\item {Grp. gram.:f.}
\end{itemize}
Acto ou effeito de lamellar.
\section{Lamellado}
\begin{itemize}
\item {Proveniência:(De \textunderscore lamellar\textunderscore ^1)}
\end{itemize}
Que tem lamellas.
\section{Lamellar}
\begin{itemize}
\item {Grp. gram.:v. t.}
\end{itemize}
\begin{itemize}
\item {Proveniência:(De \textunderscore lamella\textunderscore )}
\end{itemize}
Guarnecer com lâminas; dividir em lâminas.
\section{Lamellar}
\begin{itemize}
\item {Grp. gram.:adj.}
\end{itemize}
O mesmo que \textunderscore lamelloso\textunderscore .
\section{Lamellibrânchio}
\begin{itemize}
\item {Grp. gram.:adj.}
\end{itemize}
\begin{itemize}
\item {Proveniência:(De \textunderscore lamella\textunderscore  + \textunderscore branchias\textunderscore )}
\end{itemize}
Diz-se dos peixes, cujas brânchias têm a fórma de lâminas semicirculares.
\section{Lamellicórneo}
\begin{itemize}
\item {Grp. gram.:adj.}
\end{itemize}
\begin{itemize}
\item {Utilização:Zool.}
\end{itemize}
\begin{itemize}
\item {Grp. gram.:M. pl.}
\end{itemize}
\begin{itemize}
\item {Proveniência:(De \textunderscore lamella\textunderscore  + \textunderscore córneo\textunderscore )}
\end{itemize}
Que termina em massa folhosa, (falando-se de antennas).
Família de insectos coleópteros pentâmeros.
\section{Lamellífero}
\begin{itemize}
\item {Grp. gram.:adj.}
\end{itemize}
\begin{itemize}
\item {Grp. gram.:M. pl.}
\end{itemize}
\begin{itemize}
\item {Proveniência:(Do lat. \textunderscore lamella\textunderscore  + \textunderscore ferre\textunderscore )}
\end{itemize}
Que tem lâminas.
Família de polypeiros.
\section{Lamelliforme}
\begin{itemize}
\item {Grp. gram.:adj.}
\end{itemize}
\begin{itemize}
\item {Proveniência:(De \textunderscore lamella\textunderscore  + \textunderscore forma\textunderscore )}
\end{itemize}
Que tem fórma de lâmina.
\section{Lamellinha}
\begin{itemize}
\item {Grp. gram.:f.}
\end{itemize}
\begin{itemize}
\item {Proveniência:(Do lat. \textunderscore lamella\textunderscore )}
\end{itemize}
Gênero de infusórios, caracterizados por seu pequeno corpo transparente em fórma de lâmina.
\section{Lamellípede}
\begin{itemize}
\item {Grp. gram.:adj.}
\end{itemize}
\begin{itemize}
\item {Utilização:Zool.}
\end{itemize}
\begin{itemize}
\item {Proveniência:(Do lat. \textunderscore lamella\textunderscore  + \textunderscore pes\textunderscore , \textunderscore pedis\textunderscore )}
\end{itemize}
Que tem pés achatados.
\section{Lamellirostro}
\begin{itemize}
\item {fónica:rós}
\end{itemize}
\begin{itemize}
\item {Grp. gram.:adj.}
\end{itemize}
\begin{itemize}
\item {Utilização:Zool.}
\end{itemize}
\begin{itemize}
\item {Grp. gram.:M. pl.}
\end{itemize}
\begin{itemize}
\item {Proveniência:(De \textunderscore lamella\textunderscore  + \textunderscore rostro\textunderscore )}
\end{itemize}
Que tem o bico guarnecido de lâminas.
Aves palmípedes, cujo bico é guarnecido de lâminas córneas, em fórma de dentes.
\section{Lamelloso}
\begin{itemize}
\item {Grp. gram.:adj.}
\end{itemize}
\begin{itemize}
\item {Proveniência:(De \textunderscore lamella\textunderscore )}
\end{itemize}
Que tem lâminas.
\section{Lamellosodentado}
\begin{itemize}
\item {Grp. gram.:adj.}
\end{itemize}
\begin{itemize}
\item {Utilização:Zool.}
\end{itemize}
\begin{itemize}
\item {Proveniência:(De \textunderscore lamelloso\textunderscore  + \textunderscore dente\textunderscore )}
\end{itemize}
Que tem dentes em fórma de lamellas.
\section{Lameloso}
\begin{itemize}
\item {Grp. gram.:adj.}
\end{itemize}
\begin{itemize}
\item {Proveniência:(De \textunderscore lamela\textunderscore )}
\end{itemize}
Que tem lâminas.
\section{Lamelosodentado}
\begin{itemize}
\item {Grp. gram.:adj.}
\end{itemize}
\begin{itemize}
\item {Utilização:Zool.}
\end{itemize}
\begin{itemize}
\item {Proveniência:(De \textunderscore lameloso\textunderscore  + \textunderscore dente\textunderscore )}
\end{itemize}
Que tem dentes em fórma de lamelas.
\section{Lamentação}
\begin{itemize}
\item {Grp. gram.:f.}
\end{itemize}
\begin{itemize}
\item {Proveniência:(Lat. \textunderscore lamentatio\textunderscore )}
\end{itemize}
Acto ou effeito de lamentar.
Clamor; queixa.
Canto fúnebre; elegia.
\section{Lamentador}
\begin{itemize}
\item {Grp. gram.:adj.}
\end{itemize}
\begin{itemize}
\item {Grp. gram.:M.}
\end{itemize}
\begin{itemize}
\item {Proveniência:(Lat. \textunderscore lamentator\textunderscore )}
\end{itemize}
Que lamenta; que lastíma.
Aquelle que lamenta; aquelle que se lamenta.
\section{Lamentar}
\begin{itemize}
\item {Grp. gram.:v. t.}
\end{itemize}
\begin{itemize}
\item {Proveniência:(Lat. \textunderscore lamentari\textunderscore )}
\end{itemize}
Prantear com gemidos ou gritos.
Manifestar dôr ou pesar, por causa de: \textunderscore lamentar uma desgraça\textunderscore .
Lastimar; compadecer-se de: \textunderscore lamentar um órfão\textunderscore .
Exprimir doloridamente.
\section{Lamentável}
\begin{itemize}
\item {Grp. gram.:adj.}
\end{itemize}
\begin{itemize}
\item {Proveniência:(De \textunderscore lamentar\textunderscore )}
\end{itemize}
Digno de sêr lastimado.
Digno de dó.
Que encerra lamentação.
\section{Lamentavelmente}
\begin{itemize}
\item {Grp. gram.:adv.}
\end{itemize}
De modo lamentável.
\section{Lamento}
\begin{itemize}
\item {Grp. gram.:m.}
\end{itemize}
\begin{itemize}
\item {Proveniência:(Lat. \textunderscore lamentum\textunderscore )}
\end{itemize}
O mesmo que \textunderscore lamentação\textunderscore .
Queixa, expressão de dôr ou de dó.
Pranto.
\section{Lamentoso}
\begin{itemize}
\item {Grp. gram.:adj.}
\end{itemize}
Relativo a lamento.
Que tem o caracter ou o tom de lamentação; lamentável; plangente.
\section{Lami}
\begin{itemize}
\item {Grp. gram.:m.}
\end{itemize}
Turco nobre, que, nas cidades da Palestina, desempenhava as funcções de juiz. Cf. Pant. de Aveiro, \textunderscore Itiner.\textunderscore , 47 v.^o, (2.^a ed.).
\section{Lâmia}
\begin{itemize}
\item {Grp. gram.:f.}
\end{itemize}
\begin{itemize}
\item {Proveniência:(Lat. \textunderscore lamia\textunderscore )}
\end{itemize}
Espécie de feiticeira ou vampíro, que os Gregos representavam com rosto de mulher e corpo de serpente, e de quem se dizia que devorava crianças.
Gênero de insectos longicórneos.
Grande peixe cartilaginoso.
\section{Lamigueiro}
\begin{itemize}
\item {Grp. gram.:m.}
\end{itemize}
(V.lamegueiro)
\section{Lâmina}
\begin{itemize}
\item {Grp. gram.:f.}
\end{itemize}
\begin{itemize}
\item {Utilização:Prov.}
\end{itemize}
\begin{itemize}
\item {Utilização:ant.}
\end{itemize}
\begin{itemize}
\item {Utilização:pop.}
\end{itemize}
\begin{itemize}
\item {Utilização:Fig.}
\end{itemize}
\begin{itemize}
\item {Proveniência:(Lat. \textunderscore lamina\textunderscore )}
\end{itemize}
Chapa delgada de metal.
Tira delgada e pouco espêssa de qualquer substância.
Lasca.
Fôlha de certos instrumentos cortantes.
Parte plana das fôlhas das plantas gramíneas.
Chapa de vidro, sôbre que se deita a substância que se deseja examinar com o microscópio.
O mesmo que \textunderscore quadro\textunderscore . Cf. M. Bernárdez, \textunderscore Nova Floresta\textunderscore , V, 31.
Pessôa estúpida.
\section{Laminação}
\begin{itemize}
\item {Grp. gram.:f.}
\end{itemize}
Acto ou effeito de laminar^1.
\section{Laminador}
\begin{itemize}
\item {Grp. gram.:adj.}
\end{itemize}
Aquelle que lamina.
Máquina para laminar.
\section{Laminagem}
\begin{itemize}
\item {Grp. gram.:f.}
\end{itemize}
O mesmo que \textunderscore laminação\textunderscore .
\section{Laminar}
\begin{itemize}
\item {Grp. gram.:v. t.}
\end{itemize}
Reduzir a lâminas; chapear.
\section{Laminar}
\begin{itemize}
\item {Grp. gram.:adj.}
\end{itemize}
Que tem fórma de lâmina.
Que tem lâminas ou textura laminar.
\section{Laminária}
\begin{itemize}
\item {Grp. gram.:f.}
\end{itemize}
\begin{itemize}
\item {Utilização:bras}
\end{itemize}
\begin{itemize}
\item {Utilização:Neol.}
\end{itemize}
\begin{itemize}
\item {Proveniência:(De \textunderscore lâmina\textunderscore )}
\end{itemize}
Tubo de borracha, empregada pela cirurgia na dilatação do canal uterino.
\section{Laminária-digitada}
\begin{itemize}
\item {Grp. gram.:f.}
\end{itemize}
Espécie de alga, (\textunderscore laminaria digitata\textunderscore ).
\section{Laminoso}
\begin{itemize}
\item {Grp. gram.:adj.}
\end{itemize}
\begin{itemize}
\item {Proveniência:(Lat. \textunderscore laminosus\textunderscore )}
\end{itemize}
O mesmo que \textunderscore laminar\textunderscore ^2.
\section{Lamínula}
\begin{itemize}
\item {Grp. gram.:f.}
\end{itemize}
Lâmina pequena.
\section{Lâmio-branco}
\begin{itemize}
\item {Grp. gram.:m.}
\end{itemize}
Planta, o mesmo que \textunderscore urtiga-morta\textunderscore .
\section{Lamira}
\begin{itemize}
\item {Grp. gram.:f.}
\end{itemize}
\begin{itemize}
\item {Utilização:Gír.}
\end{itemize}
O mesmo que \textunderscore libra\textunderscore , moéda.
\section{Lamiré}
\begin{itemize}
\item {Grp. gram.:m.}
\end{itemize}
\begin{itemize}
\item {Utilização:Fig.}
\end{itemize}
\begin{itemize}
\item {Utilização:Pop.}
\end{itemize}
Diapasão.
Sinal para comêço de alguma coisa.
Reprehensão.
(Do nome das três notas musicaes \textunderscore la\textunderscore , \textunderscore mi\textunderscore , \textunderscore ré\textunderscore )
\section{Lamista}
\begin{itemize}
\item {Grp. gram.:m.}
\end{itemize}
O mesmo que \textunderscore lamaísta\textunderscore .
\section{Lamnúrgios}
\begin{itemize}
\item {Grp. gram.:m. pl.}
\end{itemize}
\begin{itemize}
\item {Utilização:Zool.}
\end{itemize}
Ordem de animaes mammíferos, intermediária aos roedores e aos perissodáctylos.
\section{Lamoja}
\begin{itemize}
\item {Grp. gram.:f.}
\end{itemize}
\begin{itemize}
\item {Proveniência:(Do rad. de \textunderscore lama\textunderscore )}
\end{itemize}
Barrela, em que entra barro e água.
\section{Lamoso}
\begin{itemize}
\item {Grp. gram.:adj.}
\end{itemize}
O mesmo que \textunderscore lamacento\textunderscore . Cf. Pacheco, \textunderscore Promptuário\textunderscore , 7.
\section{Lampa}
\begin{itemize}
\item {Grp. gram.:f.}
\end{itemize}
Seda da China.
\section{Lampa}
\begin{itemize}
\item {Grp. gram.:f.}
\end{itemize}
\begin{itemize}
\item {Utilização:Pop.}
\end{itemize}
O mesmo que \textunderscore lâmpada\textunderscore . \textunderscore Levar as lampas\textunderscore , ir á frente; sobrelevar, têr superioridade.--Há quem supponha que a expressão \textunderscore levar as lampas\textunderscore  se relaciona com \textunderscore lampa\textunderscore ^3.
\section{Lampa}
\begin{itemize}
\item {Grp. gram.:m.}
\end{itemize}
\begin{itemize}
\item {Proveniência:(De \textunderscore lampo\textunderscore ^1)}
\end{itemize}
Fruto, que se apanha em a noite de San-João.
Figo lampo, que se apanhava em a noite de San-João e se levava de presente.
Variedade de figueira.
\section{Lampaça}
\begin{itemize}
\item {Grp. gram.:f.}
\end{itemize}
\begin{itemize}
\item {Utilização:Prov.}
\end{itemize}
\begin{itemize}
\item {Utilização:trasm.}
\end{itemize}
Espécie de acelga.
O mesmo que \textunderscore labaça\textunderscore ^2.
\section{Lâmpada}
\begin{itemize}
\item {Grp. gram.:f.}
\end{itemize}
\begin{itemize}
\item {Proveniência:(Lat. \textunderscore lampada\textunderscore )}
\end{itemize}
Vaso, em que se accende a luz, alimentada a óleo.
\section{Lampadário}
\begin{itemize}
\item {Grp. gram.:m.}
\end{itemize}
\begin{itemize}
\item {Proveniência:(Lat. \textunderscore lampadarius\textunderscore )}
\end{itemize}
Espécie de lustre, que tem pendentes várias lâmpadas.
Lustre, candelabro.
\section{Lampadeiro}
\begin{itemize}
\item {Grp. gram.:m.}
\end{itemize}
Fabricante de lâmpadas.
Haste de metal ou madeira, que sustenta uma lâmpada ou lâmpadas.
\section{Lampadejar}
\begin{itemize}
\item {Grp. gram.:v. i.}
\end{itemize}
\begin{itemize}
\item {Proveniência:(De \textunderscore lâmpada\textunderscore )}
\end{itemize}
Bruxulear.
Brilhar: \textunderscore lampadejavam espadas\textunderscore .
Espargir luz.
\section{Lampadomancia}
\begin{itemize}
\item {Grp. gram.:f.}
\end{itemize}
\begin{itemize}
\item {Proveniência:(Do gr. \textunderscore lampas\textunderscore  + \textunderscore manteia\textunderscore )}
\end{itemize}
Espécie de adivinhação, que consistia em tirar preságios das cores e movimentos da luz de uma lâmpada.
\section{Lampadomântico}
\begin{itemize}
\item {Grp. gram.:adj.}
\end{itemize}
Relativo á lampadomancia.
\section{Lâmpam}
\begin{itemize}
\item {Grp. gram.:adj.}
\end{itemize}
\begin{itemize}
\item {Utilização:Des.}
\end{itemize}
O mesmo que \textunderscore lampo\textunderscore ^1.
\section{Lampana}
\begin{itemize}
\item {Grp. gram.:f.}
\end{itemize}
\begin{itemize}
\item {Utilização:Burl.}
\end{itemize}
Peta; intrujice; bazófia.
O mesmo que \textunderscore bofetada\textunderscore .
\section{Lâmpão}
\begin{itemize}
\item {Grp. gram.:adj.}
\end{itemize}
\begin{itemize}
\item {Utilização:Des.}
\end{itemize}
O mesmo que \textunderscore lampo\textunderscore ^1.
\section{Lampar}
\begin{itemize}
\item {Grp. gram.:v. i.}
\end{itemize}
\begin{itemize}
\item {Utilização:Prov.}
\end{itemize}
\begin{itemize}
\item {Utilização:minh.}
\end{itemize}
\begin{itemize}
\item {Proveniência:(De \textunderscore lampo\textunderscore ^2)}
\end{itemize}
O mesmo que \textunderscore relampejar\textunderscore .
\section{Lamparão}
\begin{itemize}
\item {Grp. gram.:m.}
\end{itemize}
(V. \textunderscore laparão\textunderscore ^1)
\section{Lamparina}
\begin{itemize}
\item {Grp. gram.:f.}
\end{itemize}
\begin{itemize}
\item {Utilização:Chul.}
\end{itemize}
\begin{itemize}
\item {Proveniência:(Do cast. \textunderscore lamparilla\textunderscore ?)}
\end{itemize}
Pequena lâmpada; luminária.
Bofetada.
\section{Lampascópio}
\begin{itemize}
\item {Grp. gram.:m.}
\end{itemize}
\begin{itemize}
\item {Proveniência:(Do gr. \textunderscore lampas\textunderscore  + \textunderscore skopein\textunderscore )}
\end{itemize}
Espécie de lanterna mágica.
\section{Lâmpedo}
\begin{itemize}
\item {Grp. gram.:adj.}
\end{itemize}
\begin{itemize}
\item {Utilização:Bras}
\end{itemize}
O mesmo que \textunderscore lampo\textunderscore ^2.
\section{Lampeiro}
\begin{itemize}
\item {Grp. gram.:adj.}
\end{itemize}
\begin{itemize}
\item {Utilização:Fig.}
\end{itemize}
\begin{itemize}
\item {Proveniência:(De \textunderscore lampa\textunderscore ^2)}
\end{itemize}
O mesmo que \textunderscore lampo\textunderscore ^1.
Metediço.
Espevitado.
Que procura levar as lampas a outrem.
\section{Lampejante}
\begin{itemize}
\item {Grp. gram.:adj.}
\end{itemize}
Que lampeja.
\section{Lampejar}
\begin{itemize}
\item {Grp. gram.:v. i.}
\end{itemize}
\begin{itemize}
\item {Utilização:Prov.}
\end{itemize}
\begin{itemize}
\item {Utilização:minh.}
\end{itemize}
\begin{itemize}
\item {Proveniência:(De \textunderscore lampo\textunderscore ^2)}
\end{itemize}
Scintillar, brilhar como relâmpago.
O mesmo que \textunderscore relampaguear\textunderscore .
\section{Lampejo}
\begin{itemize}
\item {Grp. gram.:m.}
\end{itemize}
\begin{itemize}
\item {Utilização:Fig.}
\end{itemize}
Acto ou effeito de lampejar.
Manifestação rápida e brilhante de uma ideia ou de um sentimento.
\section{Lampianista}
\begin{itemize}
\item {Grp. gram.:m.}
\end{itemize}
\begin{itemize}
\item {Proveniência:(De \textunderscore lampião\textunderscore )}
\end{itemize}
O encarregado de accender, apagar e limpar os lampiões da illuminação pública.
\section{Lampião}
\begin{itemize}
\item {Grp. gram.:m.}
\end{itemize}
Grande lanterna portátil, ou fixa num tecto, esquina ou parede.
(Cp. cast. \textunderscore lampión\textunderscore )
\section{Lampinho}
\begin{itemize}
\item {Grp. gram.:adj.}
\end{itemize}
\begin{itemize}
\item {Grp. gram.:M.}
\end{itemize}
Que não tem barba.
Indivíduo que não tem barba; indivíduo imberbe.
(Cast. \textunderscore lampiño\textunderscore )
\section{Lampiro}
\begin{itemize}
\item {Grp. gram.:m.}
\end{itemize}
\begin{itemize}
\item {Proveniência:(Gr. \textunderscore lampuris\textunderscore )}
\end{itemize}
Designação cientifica do pirilampo.
\section{Lampista}
\begin{itemize}
\item {Grp. gram.:m.}
\end{itemize}
\begin{itemize}
\item {Utilização:Gal}
\end{itemize}
\begin{itemize}
\item {Proveniência:(Fr. \textunderscore lampiste\textunderscore )}
\end{itemize}
Aquelle que faz lampiões ou lanternas.
Aquelle que trata dos lampiões de uma illuminação.
\section{Lampo}
\begin{itemize}
\item {Grp. gram.:adj.}
\end{itemize}
Temporão, (falando-se especialmente de uma variedade do figos brancos).
\section{Lampo}
\begin{itemize}
\item {Grp. gram.:m.}
\end{itemize}
\begin{itemize}
\item {Utilização:Prov.}
\end{itemize}
\begin{itemize}
\item {Utilização:minh.}
\end{itemize}
O mesmo que \textunderscore relâmpago\textunderscore .
(Contr. de \textunderscore relâmpago\textunderscore )
\section{Lampona}
\begin{itemize}
\item {Grp. gram.:f.}
\end{itemize}
\begin{itemize}
\item {Utilização:T. da Bairrada}
\end{itemize}
O mesmo que \textunderscore lampana\textunderscore .
\section{Lamponeiro}
\begin{itemize}
\item {Grp. gram.:m.}
\end{itemize}
\begin{itemize}
\item {Utilização:T. da Bairrada}
\end{itemize}
Aquelle que diz lamponas; intrujão.
\section{Lampote}
\begin{itemize}
\item {Grp. gram.:m.}
\end{itemize}
Pano de algodão, fabricado nas Filippinas.
\section{Lampreão}
\begin{itemize}
\item {Grp. gram.:m.}
\end{itemize}
\begin{itemize}
\item {Utilização:Pleb.}
\end{itemize}
Pênis com orgasmo. Cf. Macedo, \textunderscore Burros\textunderscore , 260.
\section{Lamprear}
\begin{itemize}
\item {Grp. gram.:v. t.}
\end{itemize}
Deitar abaixo (um pau do jôgo da bola), sem tocar nos outros.
\section{Lampreeira}
\begin{itemize}
\item {Grp. gram.:f.}
\end{itemize}
Rede de emmalhar, que se emprega na pesca da lampreia.
\section{Lampreia}
\begin{itemize}
\item {Grp. gram.:f.}
\end{itemize}
\begin{itemize}
\item {Proveniência:(Do lat. \textunderscore lampetra\textunderscore )}
\end{itemize}
Peixe cyclóstomo, (\textunderscore petromizon marinus\textunderscore  e \textunderscore petromizon fluviatilis\textunderscore ).
\section{Lâmpride}
\begin{itemize}
\item {Grp. gram.:f.}
\end{itemize}
Gênero de peixes acanthopterýgios.
\section{Lamprite}
\begin{itemize}
\item {Grp. gram.:f.}
\end{itemize}
\begin{itemize}
\item {Proveniência:(Do gr. \textunderscore lampros\textunderscore , brilhante)}
\end{itemize}
Nome, proposto por Tschermak, em substituição de \textunderscore sulfureto\textunderscore .
\section{Lamprocária}
\begin{itemize}
\item {Grp. gram.:f.}
\end{itemize}
Gênero de plantas cyperáceas.
\section{Lamproglena}
\begin{itemize}
\item {Grp. gram.:f.}
\end{itemize}
Gênero de crustáceos pachycéphalos.
\section{Lamprómetro}
\begin{itemize}
\item {Grp. gram.:m.}
\end{itemize}
\begin{itemize}
\item {Proveniência:(Do gr. \textunderscore lampros\textunderscore  + \textunderscore metron\textunderscore )}
\end{itemize}
Instrumento, para medir a intensidade da luz.
\section{Lampróptero}
\begin{itemize}
\item {Grp. gram.:m.}
\end{itemize}
\begin{itemize}
\item {Proveniência:(Do gr. \textunderscore lampros\textunderscore  + \textunderscore pteron\textunderscore )}
\end{itemize}
Gênero de insectos hemípteros, de asas brilhantes.
\section{Lampsana}
\begin{itemize}
\item {Grp. gram.:f.}
\end{itemize}
\begin{itemize}
\item {Proveniência:(Lat. \textunderscore lampsana\textunderscore )}
\end{itemize}
Planta annual, de flôres amarelas, da fam. das compostas, (lampsana communis, Lin.).
\section{Lamptérias}
\begin{itemize}
\item {Grp. gram.:f. pl.}
\end{itemize}
\begin{itemize}
\item {Proveniência:(Do gr. \textunderscore lampter\textunderscore )}
\end{itemize}
Antigas festas, com illuminações, em honra de Baccho, depois das vindimas.
\section{Lampuga}
\begin{itemize}
\item {Grp. gram.:f.}
\end{itemize}
Gênero de peixes acanthopterýgios.
\section{Lampurda}
\begin{itemize}
\item {Grp. gram.:f.}
\end{itemize}
Gênero de plantas synanthéreas.
\section{Lampyro}
\begin{itemize}
\item {Grp. gram.:m.}
\end{itemize}
\begin{itemize}
\item {Proveniência:(Gr. \textunderscore lampuris\textunderscore )}
\end{itemize}
Designação scientifica do pyrilampo.
\section{Lamuge}
\begin{itemize}
\item {Grp. gram.:f.}
\end{itemize}
Mollusco marítimo.
\section{Lamúria}
\begin{itemize}
\item {Grp. gram.:f.}
\end{itemize}
\begin{itemize}
\item {Utilização:Fig.}
\end{itemize}
\begin{itemize}
\item {Proveniência:(Lat. \textunderscore lemuria\textunderscore ?)}
\end{itemize}
Queixa, lamentação, jeremiada.
Súpplica de mendigo.
Importunação de pretendente que se lastíma.
\section{Lamuriador}
\begin{itemize}
\item {Grp. gram.:adj.}
\end{itemize}
O mesmo que \textunderscore lamuriante\textunderscore .
\section{Lamuriante}
\begin{itemize}
\item {Grp. gram.:adj.}
\end{itemize}
\begin{itemize}
\item {Proveniência:(De \textunderscore lamuriar\textunderscore )}
\end{itemize}
Que faz lamúria, para conseguir alguma coisa.
Relativo a lamúria; lamentoso.
\section{Lamuriar}
\begin{itemize}
\item {Grp. gram.:v. i.}
\end{itemize}
Fazer lamúria.
\section{Lamuriento}
\begin{itemize}
\item {Grp. gram.:adj.}
\end{itemize}
O mesmo que \textunderscore lamuriante\textunderscore .
\section{Lamurioso}
\begin{itemize}
\item {Grp. gram.:adj.}
\end{itemize}
O mesmo que \textunderscore lamuriante\textunderscore .
\section{Lamurúxia}
\begin{itemize}
\item {fónica:csi}
\end{itemize}
\begin{itemize}
\item {Grp. gram.:f.}
\end{itemize}
Gênero de plantas escrofularíneas.
\section{Lamuta}
\begin{itemize}
\item {Grp. gram.:m.}
\end{itemize}
Uma das línguas uralo-altaicas.
\section{Lan}
\begin{itemize}
\item {Grp. gram.:f.}
\end{itemize}
\begin{itemize}
\item {Utilização:Fam.}
\end{itemize}
\begin{itemize}
\item {Proveniência:(Do lat. \textunderscore lana\textunderscore )}
\end{itemize}
Pêlo macio, espêsso e longo, que cobre a pelle dos carneiros e de outros animaes.
Tecido, feito desse pêlo.
Lanugem de certas plantas.
Carapinha.
Acanhamento, excessiva timidez.
\section{Lana-caprina}
\begin{itemize}
\item {Grp. gram.:f.}
\end{itemize}
\begin{itemize}
\item {Proveniência:(Do lat. \textunderscore lana\textunderscore  + \textunderscore caprinus\textunderscore )}
\end{itemize}
Insignificância, pouca monta; bagatela: \textunderscore questões de lana-caprina\textunderscore .
\section{Lanada}
\begin{itemize}
\item {Grp. gram.:f.}
\end{itemize}
\begin{itemize}
\item {Proveniência:(Do lat. \textunderscore lanatus\textunderscore )}
\end{itemize}
Varredoiro de pelles de ovelha, com que se limpam interiormente as peças de artilharia.
\section{Lanar}
\begin{itemize}
\item {Grp. gram.:adj.}
\end{itemize}
\begin{itemize}
\item {Proveniência:(Do lat. \textunderscore lana\textunderscore )}
\end{itemize}
Relativo a lan; lanígero.
\section{Lança}
\begin{itemize}
\item {Grp. gram.:f.}
\end{itemize}
\begin{itemize}
\item {Proveniência:(Do lat. \textunderscore lancea\textunderscore )}
\end{itemize}
Arma offensiva ou de arremêsso, formada de uma haste, que tem na extremidade uma lâmina ponteaguda.
Antenna náutica, que liga os calcetes aos pés dos mastros.
Varal de carruagem.
\section{Lançaço}
\begin{itemize}
\item {Grp. gram.:f.}
\end{itemize}
\begin{itemize}
\item {Utilização:Bras}
\end{itemize}
Lançada, golpe de lança.
\section{Lançada}
\begin{itemize}
\item {Grp. gram.:f.}
\end{itemize}
Ferimento ou pancada com lança.
\section{Lançadeira}
\begin{itemize}
\item {Grp. gram.:f.}
\end{itemize}
\begin{itemize}
\item {Proveniência:(De \textunderscore lançar\textunderscore )}
\end{itemize}
Instrumento, que contém um pequeno cylindro ou canela, em que se enleia o fio que os tecelões e tecedeiras fazem passar pelos fios do urdume.
Pequeno instrumento análogo, nas máquinas de costura.
\section{Lançadiço}
\begin{itemize}
\item {Grp. gram.:adj.}
\end{itemize}
\begin{itemize}
\item {Utilização:Pop.}
\end{itemize}
\begin{itemize}
\item {Utilização:Ant.}
\end{itemize}
\begin{itemize}
\item {Proveniência:(Do rad. de \textunderscore lançar\textunderscore )}
\end{itemize}
Que se deve deitar fóra.
Que não presta.
Astuto; manhoso. Cf. \textunderscore Inéd. da Hist. Port.\textunderscore , I, 364.
\section{Lançado}
\begin{itemize}
\item {Grp. gram.:m.}
\end{itemize}
\begin{itemize}
\item {Proveniência:(De \textunderscore lançar\textunderscore )}
\end{itemize}
Aquillo que se vomitou.
\section{Lançador}
\begin{itemize}
\item {Grp. gram.:adj.}
\end{itemize}
\begin{itemize}
\item {Grp. gram.:M.}
\end{itemize}
\begin{itemize}
\item {Utilização:Pop.}
\end{itemize}
Que lança.
Aquelle que offerece lances em leilões.
Lançarote.
\section{Lançador}
\begin{itemize}
\item {Grp. gram.:m.}
\end{itemize}
\begin{itemize}
\item {Utilização:Ant.}
\end{itemize}
Guerreiro armado de lança.
\section{Lançadura}
\begin{itemize}
\item {Grp. gram.:f.}
\end{itemize}
Acto ou effeito de lançar.
\section{Lançagem}
\begin{itemize}
\item {Grp. gram.:f.}
\end{itemize}
\begin{itemize}
\item {Utilização:P. us.}
\end{itemize}
O mesmo que \textunderscore lançadura\textunderscore .
\section{Lançamento}
\begin{itemize}
\item {Grp. gram.:m.}
\end{itemize}
Acto de lançar.
Rebento vegetal.
Conjunto de operações, na organização dos mappas dos contribuintes.
Assentamento.
Escrituração de uma verba, em livro commercial.
\section{Lançante}
\begin{itemize}
\item {Grp. gram.:adj.}
\end{itemize}
Que lança.
\section{Lancantina}
\begin{itemize}
\item {Grp. gram.:f.}
\end{itemize}
\begin{itemize}
\item {Utilização:Prov.}
\end{itemize}
\begin{itemize}
\item {Utilização:alg.}
\end{itemize}
O mesmo que \textunderscore lencantina\textunderscore .
\section{Lançar}
\begin{itemize}
\item {Grp. gram.:v. t.}
\end{itemize}
\begin{itemize}
\item {Grp. gram.:V. i.}
\end{itemize}
\begin{itemize}
\item {Proveniência:(Do b. lat. \textunderscore lanceare\textunderscore )}
\end{itemize}
Arremessar.
Soltar da mão.
Arrojar: \textunderscore lançar uma pedra\textunderscore .
Deitar, fazer caír: \textunderscore lançar alguém ao chão\textunderscore .
Fazer saír.
Vomitar.
Produzir, causar: \textunderscore lançar terror na população\textunderscore .
Imputar.
Offerecer como lanço (uma quantia).
Espalhar.
Exhalar: \textunderscore lançar cheiro\textunderscore .
Consignar, traçar, registar: \textunderscore lançar uma dívida\textunderscore .
Vomitar.
\section{Lançarote}
\begin{itemize}
\item {Grp. gram.:m.}
\end{itemize}
\begin{itemize}
\item {Proveniência:(De \textunderscore lançar\textunderscore )}
\end{itemize}
Indivíduo, que auxilia o cavallo no acto da padreação.
Tratador de cavallo de padreação.
\section{Lancasteriano}
\begin{itemize}
\item {Grp. gram.:adj.}
\end{itemize}
Diz-se de um méthodo de ensino mútuo, inventado por um pedagogo inglês, Lancaster.
\section{Lancasterita}
\begin{itemize}
\item {Grp. gram.:f.}
\end{itemize}
\begin{itemize}
\item {Proveniência:(De \textunderscore Lancaster\textunderscore , n. p.)}
\end{itemize}
Hydrocarbonato de magnésia natural, que se encontra em Lancaster, nos Estados-Unidos.
\section{Lancastriano}
\begin{itemize}
\item {Grp. gram.:adj.}
\end{itemize}
\begin{itemize}
\item {Grp. gram.:M.}
\end{itemize}
Relativo a Lancastre, na Inglaterra.
Habitante de Lancastre.
Partidário da casa de Lancastre, na história da Inglaterra.
\section{Lance}
\begin{itemize}
\item {Grp. gram.:m.}
\end{itemize}
\begin{itemize}
\item {Utilização:Bras}
\end{itemize}
Acto ou effeito de lançar.
Impulso.
Conjuntura.
Perigo.
Vicissitude.
Acontecimento.
Facto notavel ou diffícil.
\textunderscore Lance de casas\textunderscore , sequência de casas contíguas, quarteirão de casas, correnteza. Cf. \textunderscore Jorn. do Comm.\textunderscore , do Rio, 26-IV-901.
\section{Lanceada}
\begin{itemize}
\item {Grp. gram.:m.}
\end{itemize}
\begin{itemize}
\item {Utilização:Bras. do Pará}
\end{itemize}
\begin{itemize}
\item {Proveniência:(Do rad. de \textunderscore lanço\textunderscore )}
\end{itemize}
Pescaria com rêde de arrastar.
\section{Lanceador}
\begin{itemize}
\item {Grp. gram.:m.}
\end{itemize}
Aquelle que lanceia.
\section{Lancear}
\begin{itemize}
\item {Grp. gram.:v. t.}
\end{itemize}
\begin{itemize}
\item {Utilização:Fig.}
\end{itemize}
Ferir com lança.
Affligir, atormentar.
\section{Lancear}
\begin{itemize}
\item {Grp. gram.:v. t.}
\end{itemize}
\begin{itemize}
\item {Utilização:Bras}
\end{itemize}
\begin{itemize}
\item {Utilização:Bras. do N}
\end{itemize}
\begin{itemize}
\item {Proveniência:(De \textunderscore lanço\textunderscore )}
\end{itemize}
Pescar com rêde.
Fazer rodar por cima da cabeça (o laço da corda, que se atirava aos cornos do boi, para o laçar).
\section{Lanceiro}
\begin{itemize}
\item {Grp. gram.:m.}
\end{itemize}
\begin{itemize}
\item {Grp. gram.:Pl.}
\end{itemize}
\begin{itemize}
\item {Proveniência:(Do lat. \textunderscore lancearius\textunderscore )}
\end{itemize}
Casa de armas.
Panóplia.
Cabide.
Fabricante de lanças.
Soldado com lança.
Regimento de soldados com lança.
Espécie de quadrilha dançante.
\section{Lanceolado}
\begin{itemize}
\item {Grp. gram.:adj.}
\end{itemize}
\begin{itemize}
\item {Utilização:Bot.}
\end{itemize}
\begin{itemize}
\item {Proveniência:(Lat. \textunderscore lanceolatus\textunderscore )}
\end{itemize}
Semelhante ao ferro da lança: \textunderscore folhas lanceoladas\textunderscore .
\section{Lanceolar}
\begin{itemize}
\item {Grp. gram.:adj.}
\end{itemize}
O mesmo que \textunderscore lanceolado\textunderscore .
\section{Lanceta}
\begin{itemize}
\item {fónica:cê}
\end{itemize}
\begin{itemize}
\item {Grp. gram.:f.}
\end{itemize}
\begin{itemize}
\item {Proveniência:(Fr. \textunderscore lancette\textunderscore )}
\end{itemize}
Pequena lâmina lanceolada, com dois gumes, para operações cirúrgicas.
Cutello, com que se abatem algumas reses nos matadoiros.
Planta brasileira, (\textunderscore solidago vulneraria\textunderscore ).
\section{Lancetada}
\begin{itemize}
\item {Grp. gram.:f.}
\end{itemize}
Acto ou effeito de lancetar.
\section{Lancetar}
\begin{itemize}
\item {Grp. gram.:v. t.}
\end{itemize}
Ferir com lanceia; esvurmar: \textunderscore lancetar um tumor\textunderscore .
\section{Lanceteira}
\begin{itemize}
\item {Grp. gram.:f.}
\end{itemize}
\begin{itemize}
\item {Proveniência:(De \textunderscore lanceta\textunderscore )}
\end{itemize}
Espécie de lima, usada por espingardeiros e serralheiros.
\section{Lancha}
\begin{itemize}
\item {Grp. gram.:f.}
\end{itemize}
\begin{itemize}
\item {Utilização:Prov.}
\end{itemize}
\begin{itemize}
\item {Utilização:trasm.}
\end{itemize}
\begin{itemize}
\item {Utilização:Prov.}
\end{itemize}
\begin{itemize}
\item {Utilização:beir.}
\end{itemize}
Pedra xistosa e grosseira.
O mesmo que \textunderscore lage\textunderscore .
(Cast. \textunderscore lancha\textunderscore , lage)
\section{Lancha}
\begin{itemize}
\item {Grp. gram.:f.}
\end{itemize}
Pequeno barco, para serviço de navios.
Embarcação maior que êsse barco, própria para navegação costeira ou para pesca.
(Cast. \textunderscore lancha\textunderscore )
\section{Lanchada}
\begin{itemize}
\item {Grp. gram.:f.}
\end{itemize}
\begin{itemize}
\item {Utilização:Chul. da Covilhan.}
\end{itemize}
Carga de uma lancha.
Bofetada.
\section{Lanchal}
\begin{itemize}
\item {Grp. gram.:m.}
\end{itemize}
\begin{itemize}
\item {Utilização:Prov.}
\end{itemize}
\begin{itemize}
\item {Utilização:beir.}
\end{itemize}
\begin{itemize}
\item {Proveniência:(De \textunderscore lancha\textunderscore ^1)}
\end{itemize}
O mesmo que \textunderscore lagedo\textunderscore .
\section{Lanchão}
\begin{itemize}
\item {Grp. gram.:m.}
\end{itemize}
\begin{itemize}
\item {Proveniência:(De \textunderscore lancha\textunderscore ^2)}
\end{itemize}
Grande lancha.
\section{Lanchão}
\begin{itemize}
\item {Grp. gram.:m.}
\end{itemize}
\begin{itemize}
\item {Utilização:Prov.}
\end{itemize}
\begin{itemize}
\item {Utilização:trasm.}
\end{itemize}
\begin{itemize}
\item {Proveniência:(De \textunderscore lancha\textunderscore ^1. Cp. cast. \textunderscore lanchon\textunderscore , e \textunderscore lajon\textunderscore , e port. \textunderscore leixão\textunderscore )}
\end{itemize}
Grande lage xistosa.
\section{Lanchar}
\begin{itemize}
\item {Grp. gram.:v. t.}
\end{itemize}
\begin{itemize}
\item {Grp. gram.:V. i.}
\end{itemize}
Comer, como lanche.
Tomar um lanche.
\section{Lanchara}
\begin{itemize}
\item {Grp. gram.:f.}
\end{itemize}
\begin{itemize}
\item {Proveniência:(Do mal. \textunderscore lantiar\textunderscore )}
\end{itemize}
Antigo barco de guerra, no Oriente.
\section{Lanche}
\begin{itemize}
\item {Grp. gram.:m.}
\end{itemize}
\begin{itemize}
\item {Proveniência:(Do ingl. \textunderscore lunch\textunderscore )}
\end{itemize}
Pequena refeição, entre o almôço e o jantar.
\section{Lancheiro}
\begin{itemize}
\item {Grp. gram.:m.}
\end{itemize}
Cada um dos dez indivíduos, que ordinariamente tripulam uma baleeira. Cf. \textunderscore Jorn. do Comm.\textunderscore , do Rio de 29-VI-900.
\section{Lancheta}
\begin{itemize}
\item {fónica:chê}
\end{itemize}
\begin{itemize}
\item {Grp. gram.:f.}
\end{itemize}
Pequena lancha. Cf. Oliv. Martins, \textunderscore Port. nos Mares\textunderscore , 240.
\section{Lancho}
\begin{itemize}
\item {Grp. gram.:m.}
\end{itemize}
\begin{itemize}
\item {Utilização:ant.}
\end{itemize}
\begin{itemize}
\item {Utilização:Gír.}
\end{itemize}
Penedo.
(Cp. \textunderscore lancha\textunderscore ^1)
\section{Lancil}
\begin{itemize}
\item {Grp. gram.:m.}
\end{itemize}
\begin{itemize}
\item {Proveniência:(De \textunderscore lançar\textunderscore )}
\end{itemize}
Pedra de cantaria, longa e estreita, que serve para peitoris, vêrgas de janelas, resguardo de estradas, etc.
\section{Lancinante}
\begin{itemize}
\item {Grp. gram.:adj.}
\end{itemize}
\begin{itemize}
\item {Proveniência:(Lat. \textunderscore lancinans\textunderscore )}
\end{itemize}
Que lancina; afflictivo; pungente: \textunderscore dôres lancinantes\textunderscore .
\section{Lancinar}
\begin{itemize}
\item {Grp. gram.:v. t.}
\end{itemize}
\begin{itemize}
\item {Proveniência:(Lat. \textunderscore lancinare\textunderscore )}
\end{itemize}
Picar ou golpear.
Pungir; atormentar.
\section{Lanço}
\begin{itemize}
\item {Grp. gram.:m.}
\end{itemize}
Acto ou effeito de lançar.
Offerta de preço em leilão.
Secção de uma estrada, de um muro.
Extensão de uma fachada.
Porção de peixe, que uma rêde apanha.
Lado de uma rua, de um corredor.
Volta de lançadeira.
Parte de uma escada, comprehendida entre dois patamares.
Lance; relance.
\section{Lançó}
\begin{itemize}
\item {Grp. gram.:m.}
\end{itemize}
\begin{itemize}
\item {Utilização:Ant.}
\end{itemize}
\begin{itemize}
\item {Proveniência:(Do lat. \textunderscore lanceola\textunderscore )}
\end{itemize}
O mesmo que \textunderscore lanceta\textunderscore . Cf. Mestre Geraldo, 25.
\section{Lançol}
\begin{itemize}
\item {Grp. gram.:m.}
\end{itemize}
(V.lençol)
\section{Landagogolo}
\begin{itemize}
\item {Grp. gram.:m.}
\end{itemize}
(V.rula-mala)
\section{Landainas}
\begin{itemize}
\item {Grp. gram.:f. pl.}
\end{itemize}
\begin{itemize}
\item {Utilização:Prov.}
\end{itemize}
\begin{itemize}
\item {Utilização:trasm.}
\end{itemize}
Lérias; paleio, lábia.
Histórias da carochinha.
(Relaciona-se com \textunderscore ladainha\textunderscore ?)
\section{Landaineiro}
\begin{itemize}
\item {Grp. gram.:m.}
\end{itemize}
\begin{itemize}
\item {Utilização:Prov.}
\end{itemize}
\begin{itemize}
\item {Utilização:trasm.}
\end{itemize}
Aquelle que anda sempre com landainas.
\section{Landainices}
\begin{itemize}
\item {Grp. gram.:f. pl.}
\end{itemize}
\begin{itemize}
\item {Utilização:Prov.}
\end{itemize}
\begin{itemize}
\item {Utilização:trasm.}
\end{itemize}
Balelas, ditos; o mesmo que \textunderscore landainas\textunderscore .
\section{Lande}
\begin{itemize}
\item {Grp. gram.:m.}
\end{itemize}
\begin{itemize}
\item {Proveniência:(Do lat. \textunderscore glans\textunderscore , \textunderscore glandis\textunderscore )}
\end{itemize}
Bolota, glande.
\section{Lan-de-borrego}
\begin{itemize}
\item {Grp. gram.:f.}
\end{itemize}
Uma das espécies de limo, que apparecem nas salinas.
\section{Landeira}
\begin{itemize}
\item {Grp. gram.:f.}
\end{itemize}
\begin{itemize}
\item {Proveniência:(De \textunderscore lande\textunderscore )}
\end{itemize}
Montado de sobreiros.
\section{Landeiro}
\begin{itemize}
\item {Grp. gram.:adj.}
\end{itemize}
\begin{itemize}
\item {Utilização:T. de Turquel}
\end{itemize}
\begin{itemize}
\item {Proveniência:(De \textunderscore lande\textunderscore )}
\end{itemize}
Diz-se do carvalho e de outras árvores, quando são muito productivas.
\section{Landgrava}
\begin{itemize}
\item {Grp. gram.:f.}
\end{itemize}
O mesmo que \textunderscore landgravina\textunderscore .
\section{Landgrave}
\begin{itemize}
\item {Grp. gram.:m.}
\end{itemize}
\begin{itemize}
\item {Proveniência:(Do al. \textunderscore land\textunderscore  + \textunderscore graf\textunderscore )}
\end{itemize}
Título ou dignidade de alguns Príncipes alemães.
\section{Landgraviado}
\begin{itemize}
\item {Grp. gram.:m.}
\end{itemize}
Dignidade de landgrave.
\section{Landgraviato}
\begin{itemize}
\item {Grp. gram.:m.}
\end{itemize}
Dignidade de landgrave.
\section{Landgravina}
\begin{itemize}
\item {Grp. gram.:f.}
\end{itemize}
A mulher do landgrave.
\section{Landgravio}
\begin{itemize}
\item {Grp. gram.:m.}
\end{itemize}
O mesmo que \textunderscore landgraviado\textunderscore . Cf. Herculano, \textunderscore Hist. de Port.\textunderscore , II.
\section{Landi}
\begin{itemize}
\item {Grp. gram.:m.}
\end{itemize}
(V.lantim)
\section{Landim}
\begin{itemize}
\item {Grp. gram.:m.}
\end{itemize}
Língua de Lourenço-Marques, língua dos Landins.
\section{Landim}
\begin{itemize}
\item {Grp. gram.:m.}
\end{itemize}
\begin{itemize}
\item {Utilização:Bras}
\end{itemize}
O mesmo que \textunderscore lantim\textunderscore .
\section{Landino}
\begin{itemize}
\item {Grp. gram.:adj.}
\end{itemize}
Relativo aos Landins.
\section{Landins}
\begin{itemize}
\item {Grp. gram.:m. pl.}
\end{itemize}
Povos das margens do Zambeze.
\section{Landirana}
\begin{itemize}
\item {Grp. gram.:f.}
\end{itemize}
\begin{itemize}
\item {Utilização:Bras}
\end{itemize}
Árvore silvestre.
\section{Lândoa}
\begin{itemize}
\item {Grp. gram.:f.}
\end{itemize}
\begin{itemize}
\item {Utilização:Prov.}
\end{itemize}
Fenda natural nos troncos dos carvalhos, dos castanheiros, etc.
\section{Landólfia}
\begin{itemize}
\item {Grp. gram.:f.}
\end{itemize}
Colossal árvore africana, que produz borracha.
\section{Landólphia}
\begin{itemize}
\item {Grp. gram.:f.}
\end{itemize}
Colossal árvore africana, que produz borracha.
\section{Landonas}
\begin{itemize}
\item {Grp. gram.:f. pl.}
\end{itemize}
\begin{itemize}
\item {Utilização:Prov.}
\end{itemize}
\begin{itemize}
\item {Utilização:alent.}
\end{itemize}
Adulação, lisonja.
(Cp. \textunderscore landainas\textunderscore )
\section{Landoque}
\begin{itemize}
\item {Grp. gram.:m.}
\end{itemize}
\begin{itemize}
\item {Utilização:Prov.}
\end{itemize}
\begin{itemize}
\item {Utilização:alent.}
\end{itemize}
Mistura, amálgama, salgalhada.
\section{Landra}
\begin{itemize}
\item {Grp. gram.:f.}
\end{itemize}
O mesmo que \textunderscore lande\textunderscore .
\section{Landraia}
\begin{itemize}
\item {Grp. gram.:f.}
\end{itemize}
\begin{itemize}
\item {Utilização:Prov.}
\end{itemize}
\begin{itemize}
\item {Utilização:trasm.}
\end{itemize}
Mulher de má índole, antipáthica.
\section{Landre}
\begin{itemize}
\item {Grp. gram.:f.}
\end{itemize}
\begin{itemize}
\item {Utilização:Prov.}
\end{itemize}
\begin{itemize}
\item {Utilização:minh.}
\end{itemize}
\begin{itemize}
\item {Utilização:Ant.}
\end{itemize}
O mesmo que \textunderscore lande\textunderscore . Cp. Sim. Mach., 68.
\section{Landreiro}
\begin{itemize}
\item {Grp. gram.:m.}
\end{itemize}
\begin{itemize}
\item {Utilização:Prov.}
\end{itemize}
\begin{itemize}
\item {Utilização:minh.}
\end{itemize}
\begin{itemize}
\item {Proveniência:(De \textunderscore landre\textunderscore )}
\end{itemize}
Cacete; varapau.
\section{Landro}
\begin{itemize}
\item {Grp. gram.:m.}
\end{itemize}
\begin{itemize}
\item {Utilização:Prov.}
\end{itemize}
\begin{itemize}
\item {Utilização:alent.}
\end{itemize}
O mesmo que \textunderscore loendro\textunderscore .
\section{Landúkia}
\begin{itemize}
\item {Grp. gram.:f.}
\end{itemize}
\begin{itemize}
\item {Proveniência:(De \textunderscore Landuk\textunderscore , n. p.)}
\end{itemize}
Um dos gêneros de videiras, em que se dividiu a fam. das ampelídeas.
\section{Landum}
\begin{itemize}
\item {Grp. gram.:m.}
\end{itemize}
(Corr. de \textunderscore lundum\textunderscore )
\section{Laneiro}
\begin{itemize}
\item {Grp. gram.:m.}
\end{itemize}
\begin{itemize}
\item {Utilização:Prov.}
\end{itemize}
\begin{itemize}
\item {Utilização:alent.}
\end{itemize}
Altercação, disputa.
(Cp. lat. \textunderscore laniarius\textunderscore )
\section{Laneiro}
\begin{itemize}
\item {Grp. gram.:m.}
\end{itemize}
\begin{itemize}
\item {Utilização:Prov.}
\end{itemize}
\begin{itemize}
\item {Utilização:alent.}
\end{itemize}
\begin{itemize}
\item {Proveniência:(Do cast. \textunderscore lanero\textunderscore ?)}
\end{itemize}
Casa, onde se guarda lan.
\section{Lâneo}
\begin{itemize}
\item {Grp. gram.:m.}
\end{itemize}
\begin{itemize}
\item {Utilização:Ant.}
\end{itemize}
\begin{itemize}
\item {Proveniência:(Lat. \textunderscore laneus\textunderscore )}
\end{itemize}
Cobertor ou vestido de lan.
\section{Langará}
\begin{itemize}
\item {Grp. gram.:m.}
\end{itemize}
\begin{itemize}
\item {Utilização:Fam.}
\end{itemize}
\begin{itemize}
\item {Utilização:Prov.}
\end{itemize}
\begin{itemize}
\item {Utilização:trasm.}
\end{itemize}
O mesmo que \textunderscore arriosca\textunderscore  ou \textunderscore esparrela\textunderscore .
Embrulhada, barulho, questões.
\section{Langaré}
\begin{itemize}
\item {Grp. gram.:m.}
\end{itemize}
\begin{itemize}
\item {Utilização:Fam.}
\end{itemize}
Linguado ou manuscrito, feito á pressa ou sem cuidado. Cf. Camillo, \textunderscore Mar. da Fonte\textunderscore , 189.
\section{Langarear}
\begin{itemize}
\item {Grp. gram.:v. i.}
\end{itemize}
Fazer langará, altercar.
\section{Lângia}
\begin{itemize}
\item {Grp. gram.:f.}
\end{itemize}
\begin{itemize}
\item {Proveniência:(De \textunderscore Lang\textunderscore , n. p.)}
\end{itemize}
Planta amarantácea, do Cabo da Bôa-Esperança.
\section{Langóia}
\begin{itemize}
\item {Grp. gram.:f.}
\end{itemize}
Nome, que nalguns pontos da África portuguesa se dá á doença do somno. Cf. Capello e Ivens, I, 125.
\section{Langonha}
\begin{itemize}
\item {Grp. gram.:f.}
\end{itemize}
\begin{itemize}
\item {Utilização:Prov.}
\end{itemize}
\begin{itemize}
\item {Utilização:beir.}
\end{itemize}
\begin{itemize}
\item {Utilização:chul.}
\end{itemize}
O mesmo que \textunderscore esperma\textunderscore .
Qualquer substância pegajosa.
Ranho.
(Cp. \textunderscore languinhento\textunderscore )
\section{Langor}
\begin{itemize}
\item {Grp. gram.:m.}
\end{itemize}
\begin{itemize}
\item {Proveniência:(Lat. \textunderscore languor\textunderscore )}
\end{itemize}
O mesmo que \textunderscore languidez\textunderscore .
\section{Langorosamente}
\begin{itemize}
\item {Grp. gram.:adv.}
\end{itemize}
De modo langoroso.
Languidamente.
\section{Langoroso}
\begin{itemize}
\item {Grp. gram.:adj.}
\end{itemize}
Que tem langor; lânguido.
\section{Langosta}
\begin{itemize}
\item {Grp. gram.:f.}
\end{itemize}
\begin{itemize}
\item {Utilização:Prov.}
\end{itemize}
\begin{itemize}
\item {Utilização:trasm.}
\end{itemize}
Pessôa magra e desajeitada.
\section{Langotim}
\begin{itemize}
\item {Grp. gram.:m.}
\end{itemize}
\begin{itemize}
\item {Utilização:Ant.}
\end{itemize}
Pano, com que os Índios se cobrem, da cintura para baixo; tanga.
(Do indostano \textunderscore lãgoti\textunderscore )
\section{Langróia}
\begin{itemize}
\item {Grp. gram.:f.}
\end{itemize}
Lambisgóia? sirigaita?:«\textunderscore ...a criada... uma langroia muito abelhuda...\textunderscore »Camillo, \textunderscore Corja\textunderscore , 141.
(Cp. \textunderscore longórvia\textunderscore )
\section{Langronha}
\begin{itemize}
\item {Grp. gram.:f.}
\end{itemize}
\begin{itemize}
\item {Utilização:Prov.}
\end{itemize}
\begin{itemize}
\item {Utilização:dur.}
\end{itemize}
Espécie de alga, que cresce nas rochas marinhas, e cujas ramificações parecem tentáculos de polvo.
\section{Langúa}
\begin{itemize}
\item {Grp. gram.:f.}
\end{itemize}
Terreno chão e pantanoso, sem arvoredo, na África oriental.
(Do landim?)
\section{Langue}
\begin{itemize}
\item {Grp. gram.:adj.}
\end{itemize}
O mesmo que \textunderscore lânguido\textunderscore .
\section{Languecer}
\begin{itemize}
\item {Grp. gram.:v. i.}
\end{itemize}
O mesmo que \textunderscore elanguescer\textunderscore .
\section{Languedor}
\begin{itemize}
\item {Grp. gram.:m.}
\end{itemize}
Variedade de uva preta algarvia.
\section{Langueirão}
\begin{itemize}
\item {Grp. gram.:f.}
\end{itemize}
\begin{itemize}
\item {Utilização:Prov.}
\end{itemize}
\begin{itemize}
\item {Utilização:trasm.}
\end{itemize}
(Aument. de \textunderscore langueiras\textunderscore )
\section{Langueiras}
\begin{itemize}
\item {Grp. gram.:m.}
\end{itemize}
\begin{itemize}
\item {Utilização:Prov.}
\end{itemize}
\begin{itemize}
\item {Utilização:trasm.}
\end{itemize}
Sujeito alto, desajeitado e indolente.
\section{Languenhento}
\begin{itemize}
\item {Grp. gram.:adj.}
\end{itemize}
O mesmo que \textunderscore languinhento\textunderscore .
\section{Languenho}
\begin{itemize}
\item {Grp. gram.:m.}
\end{itemize}
\begin{itemize}
\item {Utilização:Bras. do N}
\end{itemize}
Carne negra.
Fragmentos de carne.
(Cp. \textunderscore languenhento\textunderscore )
\section{Languente}
\begin{itemize}
\item {Grp. gram.:adj.}
\end{itemize}
\begin{itemize}
\item {Proveniência:(Lat. \textunderscore languens\textunderscore )}
\end{itemize}
O mesmo que \textunderscore lânguido\textunderscore .
\section{Languento}
\begin{itemize}
\item {Grp. gram.:adj.}
\end{itemize}
\begin{itemize}
\item {Utilização:Pop.}
\end{itemize}
Doentio; achacadiço.
Piegas.
(Cp. \textunderscore languente\textunderscore )
\section{Languescer}
\begin{itemize}
\item {Grp. gram.:v. i.}
\end{itemize}
O mesmo que \textunderscore elanguescer\textunderscore .
\section{Languidamente}
\begin{itemize}
\item {Grp. gram.:adv.}
\end{itemize}
Com languidez; de modo lânguido.
\section{Languidescer}
\begin{itemize}
\item {Grp. gram.:v. i.}
\end{itemize}
O mesmo que \textunderscore elanguescer\textunderscore .
\section{Languidez}
\begin{itemize}
\item {Grp. gram.:f.}
\end{itemize}
Estado daquillo ou daquelle que é lânguido.
Enfraquecimento mórbido.
Fraqueza ou froixidão orgânica.
Definhamento.
Morbidez.
Prostração moral.
\section{Lânguido}
\begin{itemize}
\item {Grp. gram.:adj.}
\end{itemize}
\begin{itemize}
\item {Proveniência:(Lat. \textunderscore languidus\textunderscore )}
\end{itemize}
Debilitado, extenuado.
Fraco; adoentado.
Abatido.
Mórbido.
Froixo; voluptuoso: \textunderscore olhares lânguidos\textunderscore .
\section{Languinhento}
\begin{itemize}
\item {Grp. gram.:adj.}
\end{itemize}
\begin{itemize}
\item {Utilização:Pop.}
\end{itemize}
Debilitado.
Pegajoso.
Molle e húmido: \textunderscore a lesma é languinhenta\textunderscore .
Debiqueiro, pestinheiro.
(Do mesmo rad. que \textunderscore languente\textunderscore )
\section{Languinhosa}
\begin{itemize}
\item {Grp. gram.:f.}
\end{itemize}
Mulher delambida. Cf. \textunderscore Eufrosina\textunderscore , 332.
\section{Languinhoso}
\begin{itemize}
\item {Grp. gram.:adj.}
\end{itemize}
O mesmo que \textunderscore languinhento\textunderscore .
\section{Languir}
\begin{itemize}
\item {Grp. gram.:v. i.}
\end{itemize}
\begin{itemize}
\item {Proveniência:(Do lat. \textunderscore languere\textunderscore )}
\end{itemize}
O mesmo que \textunderscore elanguescer\textunderscore .
\section{Languor}
\begin{itemize}
\item {Grp. gram.:m.}
\end{itemize}
(V.langor)
\section{Languoroso}
\begin{itemize}
\item {Grp. gram.:adj.}
\end{itemize}
O mesmo que \textunderscore langoroso\textunderscore .
\section{Langureta}
\begin{itemize}
\item {fónica:gurê}
\end{itemize}
\begin{itemize}
\item {Grp. gram.:f.}
\end{itemize}
\begin{itemize}
\item {Utilização:Prov.}
\end{itemize}
Doce de farinha triga, ovos e açúcar.
\section{Lanha}
\begin{itemize}
\item {Grp. gram.:f.}
\end{itemize}
\begin{itemize}
\item {Proveniência:(T. asiát.)}
\end{itemize}
Côco tenro de palmeira.
\section{Lanhaço}
\begin{itemize}
\item {Grp. gram.:m.}
\end{itemize}
\begin{itemize}
\item {Utilização:Prov.}
\end{itemize}
\begin{itemize}
\item {Utilização:trasm.}
\end{itemize}
Grande lanho^1.
\section{Lanhar}
\begin{itemize}
\item {Grp. gram.:v. t.}
\end{itemize}
\begin{itemize}
\item {Utilização:Fig.}
\end{itemize}
\begin{itemize}
\item {Proveniência:(Do lat. \textunderscore laniare\textunderscore )}
\end{itemize}
Dar golpes em; ferir; maltratar.
Deturpar.
\section{Lanho}
\begin{itemize}
\item {Grp. gram.:m.}
\end{itemize}
\begin{itemize}
\item {Utilização:Bras. do N}
\end{itemize}
\begin{itemize}
\item {Proveniência:(De \textunderscore lanhar\textunderscore )}
\end{itemize}
Golpe de instrumento cortante: \textunderscore o barbeiro fez-te dois lanhos\textunderscore .
Pedaço de carne em tiras; lardo.
\section{Lanho}
\begin{itemize}
\item {Grp. gram.:m.}
\end{itemize}
\begin{itemize}
\item {Utilização:T. da Índia port}
\end{itemize}
Apreciado manjar, feito de polpa de côco verde, vinho, noz muscada, sumo de limão e açúcar. Cf. Ed. Magalhães, \textunderscore Hyg. Alimentar\textunderscore , I, 368.
(Cp. \textunderscore lanha\textunderscore )
\section{Laníadas}
\begin{itemize}
\item {Grp. gram.:f. pl.}
\end{itemize}
\begin{itemize}
\item {Proveniência:(De \textunderscore laniádeo\textunderscore )}
\end{itemize}
Família de aves, que têm por typo a pêga parda.
\section{Laniádeas}
\begin{itemize}
\item {Grp. gram.:f. pl.}
\end{itemize}
\begin{itemize}
\item {Proveniência:(De \textunderscore laniádeo\textunderscore )}
\end{itemize}
Família de aves, que têm por typo a pêga parda.
\section{Laniádeo}
\begin{itemize}
\item {Grp. gram.:adj.}
\end{itemize}
\begin{itemize}
\item {Proveniência:(Do lat. \textunderscore lanius\textunderscore  + gr. \textunderscore eidos\textunderscore )}
\end{itemize}
Semelhante á pêga parda.
\section{Laniado}
\begin{itemize}
\item {Grp. gram.:adj.}
\end{itemize}
O mesmo que \textunderscore laniádeo\textunderscore .
\section{Lanífero}
\begin{itemize}
\item {Grp. gram.:adj.}
\end{itemize}
\begin{itemize}
\item {Proveniência:(Lat. \textunderscore lanifer\textunderscore )}
\end{itemize}
Que tem lan ou lanugem.
Lanígero.
\section{Lanifício}
\begin{itemize}
\item {Grp. gram.:m.}
\end{itemize}
\begin{itemize}
\item {Proveniência:(Lat. \textunderscore lanificium\textunderscore )}
\end{itemize}
Manufactura de lans.
Obra de lan.
\section{Lanígero}
\begin{itemize}
\item {Grp. gram.:adj.}
\end{itemize}
\begin{itemize}
\item {Proveniência:(Lat. \textunderscore laniger\textunderscore )}
\end{itemize}
O mesmo que \textunderscore lanífero\textunderscore : \textunderscore gado lanígero\textunderscore .
\section{Lanilha}
\begin{itemize}
\item {Grp. gram.:f.}
\end{itemize}
Antigo tecido de lan fina:«\textunderscore ...uma saia de lanilha e outra de duquesa.\textunderscore »(De um testamento de 1694).
\section{Lânio}
\begin{itemize}
\item {Grp. gram.:m.}
\end{itemize}
\begin{itemize}
\item {Utilização:Ant.}
\end{itemize}
\begin{itemize}
\item {Proveniência:(Lat. \textunderscore laneus\textunderscore )}
\end{itemize}
Cobertor ou vestido de lan.
\section{Lanista}
\begin{itemize}
\item {Grp. gram.:m.}
\end{itemize}
\begin{itemize}
\item {Proveniência:(Lat. \textunderscore lanista\textunderscore )}
\end{itemize}
Mestre de gladiadores ou de esgrima, entre os antigos Romanos.
Contratador de gladiadores.
Aquelle que ensinava as aves a combater.
\section{Lanitite}
\begin{itemize}
\item {Grp. gram.:f.}
\end{itemize}
Composição de varios aggregados de lan e cimento, que se applica em pavimentos e em trabalhos incombustíveis.
\section{Lanolina}
\begin{itemize}
\item {Grp. gram.:f.}
\end{itemize}
\begin{itemize}
\item {Utilização:Bras}
\end{itemize}
Substância pharmacêutica, extrahída da lan do carneiro.
\section{Lanosidade}
\begin{itemize}
\item {Grp. gram.:f.}
\end{itemize}
\begin{itemize}
\item {Proveniência:(Lat. \textunderscore lanositas\textunderscore )}
\end{itemize}
Qualidade daquillo que é lanoso.
\section{Lanoso}
\begin{itemize}
\item {Grp. gram.:adj.}
\end{itemize}
\begin{itemize}
\item {Proveniência:(Lat. \textunderscore lanosus\textunderscore )}
\end{itemize}
Relativo a lan; que tem lan.
\section{Lansquené}
\begin{itemize}
\item {Grp. gram.:m.}
\end{itemize}
\begin{itemize}
\item {Proveniência:(Fr. \textunderscore lansquenet\textunderscore )}
\end{itemize}
Espécie de jôgo de asar.
\section{Lansquenete}
\begin{itemize}
\item {Grp. gram.:m.}
\end{itemize}
\begin{itemize}
\item {Utilização:Bras}
\end{itemize}
O mesmo que \textunderscore lansquené\textunderscore .
\section{Lantana}
\begin{itemize}
\item {Grp. gram.:f.}
\end{itemize}
Gênero de plantas verbenáceas, de flôres vermelhas ou da côr da laranja, (\textunderscore lantana camara\textunderscore , Lin.)
\section{Lantanina}
\begin{itemize}
\item {Grp. gram.:f.}
\end{itemize}
Alcaloide da lantana.
\section{Lantéa}
\begin{itemize}
\item {Grp. gram.:f.}
\end{itemize}
\begin{itemize}
\item {Utilização:Ant.}
\end{itemize}
Embarcação asiática, semelhante á fusta, ou o mesmo que \textunderscore lorcha\textunderscore . Cf. \textunderscore Peregrinação\textunderscore , XLIV e LXIII.
\section{Lanteia}
\begin{itemize}
\item {Grp. gram.:f.}
\end{itemize}
\begin{itemize}
\item {Utilização:Ant.}
\end{itemize}
Embarcação asiática, semelhante á fusta, ou o mesmo que \textunderscore lorcha\textunderscore . Cf. \textunderscore Peregrinação\textunderscore , XLIV e LXIII.
\section{Lantejoila}
\textunderscore f.\textunderscore  (e der)
O mesmo que \textunderscore lentejoila\textunderscore .
\section{Lanterna}
\begin{itemize}
\item {Grp. gram.:f.}
\end{itemize}
\begin{itemize}
\item {Utilização:Chul.}
\end{itemize}
\begin{itemize}
\item {Utilização:Gír.}
\end{itemize}
\begin{itemize}
\item {Proveniência:(Lat. \textunderscore lanterna\textunderscore )}
\end{itemize}
Espécie de caixa, guarnecida de substância transparente, geralmente vidro, para resguardar do vento uma luz que se colloca dentro.
Parte superior de um zimbório, aberta aos lados.
Fresta.
Garrafa de vinho.
Sapato.
\textunderscore Lanterna mágica\textunderscore , instrumento de óptica, que apresenta, a distância e em ponto grande, as figuras nelle pintadas.
\section{Lanterneiro}
\begin{itemize}
\item {Grp. gram.:m.}
\end{itemize}
Fabricante de lanternas.
Lampianista.
Pharoleiro.
Aquelle que leva lanterna em procissão.
\section{Lanterneta}
\begin{itemize}
\item {fónica:nê}
\end{itemize}
\begin{itemize}
\item {Grp. gram.:f.}
\end{itemize}
\begin{itemize}
\item {Proveniência:(De \textunderscore lanterna\textunderscore )}
\end{itemize}
Projéctil, que contém metralha.
\section{Lanternim}
\begin{itemize}
\item {Grp. gram.:m.}
\end{itemize}
\begin{itemize}
\item {Proveniência:(De \textunderscore lanterna\textunderscore )}
\end{itemize}
Carrete, que transmitte á mó o movimento das velas de um moinho.
Parte, aberta lateralmente, de um zimbório; lanterna.
\section{Lanternino}
\begin{itemize}
\item {Grp. gram.:m.}
\end{itemize}
(V.lanternim)
\section{Lanterníforo}
\begin{itemize}
\item {Grp. gram.:m.  e  adj.}
\end{itemize}
\begin{itemize}
\item {Proveniência:(T. hybr., do lat. \textunderscore lanterna\textunderscore  + gr. \textunderscore phoros\textunderscore )}
\end{itemize}
Nome, que os Jansenistas deram ao cão que figurava num emblema, colocado no frontispício de uma obra de propaganda, com a divisa: \textunderscore não ladra, mas morde\textunderscore .
\section{Lanterníphoro}
\begin{itemize}
\item {Grp. gram.:m.  e  adj.}
\end{itemize}
\begin{itemize}
\item {Proveniência:(T. hybr., do lat. \textunderscore lanterna\textunderscore  + gr. \textunderscore phoros\textunderscore )}
\end{itemize}
Nome, que os Jansenistas deram ao cão que figurava num emblema, collocado no frontispício de uma obra de propaganda, com a divisa: \textunderscore não ladra, mas morde\textunderscore .
\section{Lanthânio}
\begin{itemize}
\item {Grp. gram.:m.}
\end{itemize}
\begin{itemize}
\item {Proveniência:(Do gr. \textunderscore lanthanein\textunderscore )}
\end{itemize}
Metal, contido na cerita.
\section{Lânthano}
\begin{itemize}
\item {Grp. gram.:m.}
\end{itemize}
\begin{itemize}
\item {Proveniência:(Do gr. \textunderscore lanthanein\textunderscore )}
\end{itemize}
Metal, contido na cerita.
\section{Lantim}
\begin{itemize}
\item {Grp. gram.:m.}
\end{itemize}
Árvore gutífera do Brasil.
\section{Lantor}
\begin{itemize}
\item {Grp. gram.:m.}
\end{itemize}
\begin{itemize}
\item {Proveniência:(T. as.)}
\end{itemize}
Espécie de coqueiro.
\section{Lantunense}
\begin{itemize}
\item {Grp. gram.:adj.}
\end{itemize}
O mesmo que \textunderscore lantunita\textunderscore .
\section{Lantunita}
\begin{itemize}
\item {Grp. gram.:adj.}
\end{itemize}
\begin{itemize}
\item {Proveniência:(De \textunderscore Lantuna\textunderscore , n. da principal raça dos Berberes. Herculano escreveu \textunderscore Lamtuna\textunderscore )}
\end{itemize}
Diz-se da dynastia dos Príncipes almorávides. Cf. Herculano, \textunderscore Opúsc\textunderscore , III, 206.
\section{Lanudo}
\begin{itemize}
\item {Grp. gram.:adj.}
\end{itemize}
\begin{itemize}
\item {Proveniência:(Do lat. \textunderscore lana\textunderscore )}
\end{itemize}
O mesmo que \textunderscore lanoso\textunderscore .
\section{Lanugem}
\begin{itemize}
\item {Grp. gram.:f.}
\end{itemize}
\begin{itemize}
\item {Proveniência:(Lat. \textunderscore lanugo\textunderscore )}
\end{itemize}
Pêlo fino e pouco perceptível, que precede o apparecimento da barba ou que está no lugar da barba.
Buço.
Pêlos, que cobrem algumas folhas ou frutos: \textunderscore a lanugem dos pêssegos\textunderscore .
\section{Lanuginoso}
\begin{itemize}
\item {Grp. gram.:adj.}
\end{itemize}
\begin{itemize}
\item {Proveniência:(Lat. \textunderscore lanuginosus\textunderscore )}
\end{itemize}
Que tem lanugem.
Semelhante á lan ou ao algodão.
\section{Lanzeira}
\begin{itemize}
\item {Grp. gram.:f.}
\end{itemize}
Doença das cabras.
\section{Lanzinha}
\begin{itemize}
\item {Grp. gram.:f.}
\end{itemize}
Lan fraca ou pouco consistente, já manufacturada.
Variedade de pano de lan, fabricada na Covilhan.
\section{Lanzoar}
\begin{itemize}
\item {Grp. gram.:v. i.}
\end{itemize}
\begin{itemize}
\item {Utilização:Prov.}
\end{itemize}
\begin{itemize}
\item {Utilização:beir.}
\end{itemize}
O mesmo que \textunderscore alanzoar\textunderscore . (Colhido na Guarda)
\section{Lanzudo}
\begin{itemize}
\item {Grp. gram.:adj.}
\end{itemize}
\begin{itemize}
\item {Grp. gram.:M.  e  adj.}
\end{itemize}
\begin{itemize}
\item {Utilização:Pop.}
\end{itemize}
\begin{itemize}
\item {Proveniência:(Do rad. de \textunderscore lan\textunderscore )}
\end{itemize}
O mesmo que \textunderscore lanudo\textunderscore .
Indivíduo sem educação; grosseiro.
\section{Laos}
\begin{itemize}
\item {Grp. gram.:m. pl.}
\end{itemize}
Antigo povo, ao norte do Sião. Cf. Barros, \textunderscore Déc.\textunderscore  II, l. VI, c. 6. (Gr. \textunderscore laos\textunderscore )
\section{Laossinacta}
\begin{itemize}
\item {Grp. gram.:m.}
\end{itemize}
\begin{itemize}
\item {Proveniência:(Gr. \textunderscore laosunaktes\textunderscore )}
\end{itemize}
Empregado que, na Igreja grega, tinha por dever o convocar o povo para as assembleias.
\section{Laosynacta}
\begin{itemize}
\item {fónica:si}
\end{itemize}
\begin{itemize}
\item {Grp. gram.:m.}
\end{itemize}
\begin{itemize}
\item {Proveniência:(Gr. \textunderscore laosunaktes\textunderscore )}
\end{itemize}
Empregado que, na Igreja grega, tinha por dever o convocar o povo para as assembleias.
\section{Lapa}
\begin{itemize}
\item {Grp. gram.:f.}
\end{itemize}
\begin{itemize}
\item {Utilização:Prov.}
\end{itemize}
\begin{itemize}
\item {Utilização:Bras do N}
\end{itemize}
\begin{itemize}
\item {Proveniência:(Do gr. \textunderscore lapados\textunderscore , seg. Diez)}
\end{itemize}
Cavidade em rochedo; gruta.
Grande pedra ou laja, que, resaíndo de um rochedo, fórma debaixo de si um abrigo para gente ou animaes.
Mollúsco gasterópode, univalve.
Pedra solta, calhau.
Pedaço; fatia.
\section{Lapa}
\begin{itemize}
\item {Grp. gram.:f.}
\end{itemize}
\begin{itemize}
\item {Utilização:Pop.}
\end{itemize}
\begin{itemize}
\item {Proveniência:(Do germ. \textunderscore lappa\textunderscore , seg. Körting)}
\end{itemize}
O mesmo que \textunderscore bofetada\textunderscore .
\section{Lapada}
\begin{itemize}
\item {Grp. gram.:f.}
\end{itemize}
\begin{itemize}
\item {Utilização:Prov.}
\end{itemize}
\begin{itemize}
\item {Utilização:trasm.}
\end{itemize}
\begin{itemize}
\item {Utilização:Bras. do N}
\end{itemize}
\begin{itemize}
\item {Proveniência:(De \textunderscore lapa\textunderscore ^1)}
\end{itemize}
O mesmo que \textunderscore pedrada\textunderscore .
Chicotada; vergastada.
\section{Lapagéria}
\begin{itemize}
\item {Grp. gram.:f.}
\end{itemize}
Espécie de espargo do Chile.
\section{Lapantana}
\begin{itemize}
\item {Grp. gram.:m. ,  f.  e  adj.}
\end{itemize}
Pessôa simplória, idiota. Cf. Camillo, \textunderscore Corja\textunderscore , 199.
\section{Lapantanamente}
\begin{itemize}
\item {Grp. gram.:adv.}
\end{itemize}
Á maneira de lapantana; tolamente. Cf. Camillo, \textunderscore Mar. da Fonte\textunderscore . 22.
\section{Lapão}
\begin{itemize}
\item {Grp. gram.:m.}
\end{itemize}
\begin{itemize}
\item {Proveniência:(Do rad. de \textunderscore Lapónia\textunderscore , n. p.)}
\end{itemize}
Habitante da Lapónia.
Idioma dos Lapões.
\section{Lapão}
\begin{itemize}
\item {Grp. gram.:m.  e  adj.}
\end{itemize}
\begin{itemize}
\item {Utilização:Chul.}
\end{itemize}
\begin{itemize}
\item {Grp. gram.:M.}
\end{itemize}
\begin{itemize}
\item {Utilização:Prov.}
\end{itemize}
\begin{itemize}
\item {Utilização:trasm.}
\end{itemize}
\begin{itemize}
\item {Proveniência:(De \textunderscore lapa\textunderscore ^1)}
\end{itemize}
Labrego; lanzudo.
Lapa grande.
Lasca de pedra, em parede de alvenaria.
Armadilha, feita de uma lapa, para caçar teixugos.
\section{Lapara}
\begin{itemize}
\item {Grp. gram.:f.}
\end{itemize}
\begin{itemize}
\item {Utilização:Ant.}
\end{itemize}
Iguaria, própria dos banquetes reaes, entre os povos da Indo-China. Cf. \textunderscore Conquista do Peru\textunderscore , XIII.
\section{Laparão}
\begin{itemize}
\item {Grp. gram.:m.}
\end{itemize}
\begin{itemize}
\item {Proveniência:(Do gr. \textunderscore lapara\textunderscore , flanco)}
\end{itemize}
Inflammação dos gânglios e vasos lympháticos.
\section{Laparão}
\begin{itemize}
\item {Grp. gram.:m.}
\end{itemize}
\begin{itemize}
\item {Utilização:Pesc.}
\end{itemize}
Lapa grande, mollusco.
\section{Lapardão}
\begin{itemize}
\item {Grp. gram.:adj.}
\end{itemize}
\begin{itemize}
\item {Utilização:Prov.}
\end{itemize}
\begin{itemize}
\item {Utilização:trasm.}
\end{itemize}
Estúpido. (Colhido em V. P. de Aguiar)
(Cp. \textunderscore lapão\textunderscore ^2)
\section{Láparo}
\begin{itemize}
\item {Grp. gram.:m.}
\end{itemize}
\begin{itemize}
\item {Proveniência:(Do lat. \textunderscore lepus\textunderscore , \textunderscore leporis\textunderscore )}
\end{itemize}
Coelho pequeno.
\section{Laparocele}
\begin{itemize}
\item {Grp. gram.:m.}
\end{itemize}
\begin{itemize}
\item {Proveniência:(Do gr. \textunderscore lapara\textunderscore  + \textunderscore kele\textunderscore )}
\end{itemize}
Hérnia lombar.
\section{Laparoso}
\begin{itemize}
\item {Grp. gram.:adj.}
\end{itemize}
\begin{itemize}
\item {Utilização:Açor}
\end{itemize}
\begin{itemize}
\item {Proveniência:(De \textunderscore laparão\textunderscore ^1? ou corr. de \textunderscore leprôso\textunderscore ?)}
\end{itemize}
Asqueroso, repugnante.
\section{Laparoto}
\begin{itemize}
\item {fónica:parô}
\end{itemize}
\begin{itemize}
\item {Grp. gram.:m.}
\end{itemize}
\begin{itemize}
\item {Utilização:Prov.}
\end{itemize}
\begin{itemize}
\item {Grp. gram.:Adj.}
\end{itemize}
\begin{itemize}
\item {Utilização:Prov.}
\end{itemize}
\begin{itemize}
\item {Utilização:trasm.}
\end{itemize}
Láparo, já desenvolvido.
Rapaz gordo.
Astucioso.
\section{Laparotomia}
\begin{itemize}
\item {Grp. gram.:f.}
\end{itemize}
\begin{itemize}
\item {Utilização:Cir.}
\end{itemize}
\begin{itemize}
\item {Proveniência:(Do gr. \textunderscore lapara\textunderscore  + \textunderscore tome\textunderscore )}
\end{itemize}
Dava-se êste nome á incisão na ilharga, para a operação da hérnia lombar.
Hoje, é qualquer incisão na parede abdominal anterior, para se descobrir ou tratar uma lesão em víscera abdominal.
\section{Laparotomizar}
\begin{itemize}
\item {Grp. gram.:v. t.}
\end{itemize}
Fazer a operação da laparotomia em.
\section{Lapathina}
\begin{itemize}
\item {Grp. gram.:f.}
\end{itemize}
\begin{itemize}
\item {Utilização:Chím.}
\end{itemize}
\begin{itemize}
\item {Proveniência:(Do gr. \textunderscore lapathon\textunderscore , paciênca)}
\end{itemize}
Princípio amargo da raíz de uma planta, (\textunderscore rumex obtusifolius\textunderscore , Lin.).
\section{Lapatina}
\begin{itemize}
\item {Grp. gram.:f.}
\end{itemize}
\begin{itemize}
\item {Utilização:Chím.}
\end{itemize}
\begin{itemize}
\item {Proveniência:(Do gr. \textunderscore lapathon\textunderscore , paciênca)}
\end{itemize}
Princípio amargo da raíz de uma planta, (\textunderscore rumex obtusifolius\textunderscore , Lin.).
\section{Lapear}
\begin{itemize}
\item {Grp. gram.:v. t.}
\end{itemize}
\begin{itemize}
\item {Utilização:Bras. do N}
\end{itemize}
O mesmo que \textunderscore chicotear\textunderscore .
(Cp. \textunderscore lapada\textunderscore )
\section{Lapedo}
\begin{itemize}
\item {fónica:pê}
\end{itemize}
\begin{itemize}
\item {Grp. gram.:m.}
\end{itemize}
Lugar, em que há muitas lapas.
\section{Lapela}
\begin{itemize}
\item {Grp. gram.:f.}
\end{itemize}
Parte, voltada para fóra, nos quartos deanteiros e superiores de um casaco, fraque, jaquetão, etc.
\section{Lapes}
\begin{itemize}
\item {Grp. gram.:m.}
\end{itemize}
\begin{itemize}
\item {Utilização:Ant.}
\end{itemize}
\begin{itemize}
\item {Proveniência:(T. as.)}
\end{itemize}
Fôrro de tabuado delgado, com que no Oriente se revestiam exteriormente as embarcações, e entre o qual e o costado se punha uma substância calcárea.
\section{Lapiás}
\begin{itemize}
\item {Grp. gram.:m.}
\end{itemize}
\begin{itemize}
\item {Utilização:Geol.}
\end{itemize}
\begin{itemize}
\item {Proveniência:(T. dos Alpes)}
\end{itemize}
Rocha calcária, caprichosamente recortada pela acção chímica das chuvas.
\section{Lapicidas}
\begin{itemize}
\item {Grp. gram.:m.  e  adj. pl.}
\end{itemize}
\begin{itemize}
\item {Proveniência:(Lat. \textunderscore lapicida\textunderscore )}
\end{itemize}
Diz-se dos molluscos, que abrem buracos nas pedras, para alli se alojar.
\section{Lápico}
\begin{itemize}
\item {Grp. gram.:m.}
\end{itemize}
\begin{itemize}
\item {Grp. gram.:Adj.}
\end{itemize}
Língua dos lapões, o lapão.
Relativo aos lapões. Cf. Latino, \textunderscore Elog. Acad.\textunderscore , I, 66.
\section{Lápida}
\begin{itemize}
\item {Grp. gram.:f.}
\end{itemize}
(V.lápide)
\section{Lapidação}
\begin{itemize}
\item {Grp. gram.:f.}
\end{itemize}
\begin{itemize}
\item {Proveniência:(Lat. \textunderscore lapidatio\textunderscore )}
\end{itemize}
Acto ou effeito de lapidar.
\section{Lapidagem}
\begin{itemize}
\item {Grp. gram.:f.}
\end{itemize}
\begin{itemize}
\item {Proveniência:(De \textunderscore lapidar\textunderscore ^1)}
\end{itemize}
Acto de lapidar, ou operação, com que o lapidário indica as facêtas das pedras preciosas e as tornas polidas. Cf. Ortigão, \textunderscore Hollanda\textunderscore .
\section{Lapidar}
\begin{itemize}
\item {Grp. gram.:v. t.}
\end{itemize}
\begin{itemize}
\item {Utilização:Fig.}
\end{itemize}
\begin{itemize}
\item {Proveniência:(Lat. \textunderscore lapidare\textunderscore )}
\end{itemize}
Apedrejar.
Desbastar, polir: \textunderscore lapidar diamantes\textunderscore .
Tornar perfeito.
Dar bôa educação a.
\section{Lapidar}
\begin{itemize}
\item {Grp. gram.:adj.}
\end{itemize}
\begin{itemize}
\item {Proveniência:(Lat. \textunderscore lapidaris\textunderscore )}
\end{itemize}
Relativo a lápide.
Insculpido ou gravado em pedra.
Relativo a inscripções feitas em pedra.
\section{Lapidária}
\begin{itemize}
\item {Grp. gram.:f.}
\end{itemize}
\begin{itemize}
\item {Proveniência:(De \textunderscore lapidário\textunderscore )}
\end{itemize}
Sciência, que estuda as inscripções lapidares dos monumentos antigos.
\section{Lapidaría}
\begin{itemize}
\item {Grp. gram.:f.}
\end{itemize}
\begin{itemize}
\item {Proveniência:(De \textunderscore lapidar\textunderscore ^1)}
\end{itemize}
Arte de lapidar diamantes ou pedras preciosas.
Estabelecimento ou officina de lapidário. Cf. Ortigão, \textunderscore Hollanda\textunderscore , 63 e 258.
O mesmo que \textunderscore lapidária\textunderscore . Cf. Pacheco, \textunderscore Promptuário\textunderscore .
\section{Lapidário}
\begin{itemize}
\item {Grp. gram.:adj.}
\end{itemize}
\begin{itemize}
\item {Grp. gram.:M.}
\end{itemize}
\begin{itemize}
\item {Utilização:Des.}
\end{itemize}
\begin{itemize}
\item {Proveniência:(Lat. \textunderscore lapidarius\textunderscore )}
\end{itemize}
Relativo a inscripções lapidares.
Que se abriga entre pedras, (falando-se de alguns insectos).
Aquelle que trabalha em pedras preciosas; joalheiro.
O mesmo que [[pedreiro-livre|pedreiro:1]] ou \textunderscore mação\textunderscore . (Do séc. XVIII)
\section{Lápide}
\begin{itemize}
\item {Grp. gram.:f.}
\end{itemize}
\begin{itemize}
\item {Proveniência:(Lat. \textunderscore lapis\textunderscore , \textunderscore lapidis\textunderscore )}
\end{itemize}
Pedra, que contém uma inscripção, para commemorar um facto ou celebrar a memória de alguém.
Laja, que cobre o túmulo.
\section{Lapídeo}
\begin{itemize}
\item {Grp. gram.:adj.}
\end{itemize}
\begin{itemize}
\item {Proveniência:(Lat. \textunderscore lapideus\textunderscore )}
\end{itemize}
Que tem a dureza da pedra.
\section{Lapidescente}
\begin{itemize}
\item {Grp. gram.:adj.}
\end{itemize}
\begin{itemize}
\item {Proveniência:(Lat. \textunderscore lapidescens\textunderscore )}
\end{itemize}
Que se petrifica.
\section{Lapidícola}
\begin{itemize}
\item {Grp. gram.:adj.}
\end{itemize}
\begin{itemize}
\item {Proveniência:(Do lat. \textunderscore lapis\textunderscore  + \textunderscore colere\textunderscore )}
\end{itemize}
Diz-se dos animaes, que habitam ou fazem ninho entre pedras ou nas fendas dos rochedos.
\section{Lapidificação}
\begin{itemize}
\item {Grp. gram.:f.}
\end{itemize}
Acto ou effeito de \textunderscore lapidificar\textunderscore .
\section{Lapidificar}
\begin{itemize}
\item {Grp. gram.:v. t.}
\end{itemize}
\begin{itemize}
\item {Proveniência:(Do lat. \textunderscore lapis\textunderscore  + \textunderscore facere\textunderscore )}
\end{itemize}
O mesmo que \textunderscore petrificar\textunderscore .
\section{Lapidífico}
\begin{itemize}
\item {Grp. gram.:adj.}
\end{itemize}
\begin{itemize}
\item {Proveniência:(De \textunderscore lapidificar\textunderscore )}
\end{itemize}
Próprio para a formação de pedras.
\section{Lapidoso}
\begin{itemize}
\item {Grp. gram.:adj.}
\end{itemize}
\begin{itemize}
\item {Proveniência:(Lat. \textunderscore lapidosus\textunderscore )}
\end{itemize}
Lapídeo; em que há muitas pedras.
\section{Lapijar}
\begin{itemize}
\item {Grp. gram.:v. i.}
\end{itemize}
\begin{itemize}
\item {Proveniência:(Do rad. de \textunderscore lápis\textunderscore )}
\end{itemize}
Fazer traços com o lápis.
\section{Lapilloso}
\begin{itemize}
\item {Grp. gram.:adj.}
\end{itemize}
\begin{itemize}
\item {Utilização:Bot.}
\end{itemize}
\begin{itemize}
\item {Proveniência:(Do lat. \textunderscore lapillus\textunderscore )}
\end{itemize}
Diz-se do fruto, que tem o mesocarpo muito endurecido.
\section{Lapiloso}
\begin{itemize}
\item {Grp. gram.:adj.}
\end{itemize}
\begin{itemize}
\item {Utilização:Bot.}
\end{itemize}
\begin{itemize}
\item {Proveniência:(Do lat. \textunderscore lapillus\textunderscore )}
\end{itemize}
Diz-se do fruto, que tem o mesocarpo muito endurecido.
\section{Lapim}
\begin{itemize}
\item {Grp. gram.:m.}
\end{itemize}
\begin{itemize}
\item {Utilização:Prov.}
\end{itemize}
\begin{itemize}
\item {Utilização:minh.}
\end{itemize}
O mesmo que \textunderscore larápio\textunderscore .
\section{Lapim}
\begin{itemize}
\item {Grp. gram.:m.}
\end{itemize}
Sarja de seda, fina e preta, com que se faziam mantilhas para senhoras. Cf. Camillo, \textunderscore Mem. do Cárcere\textunderscore , II, 29.
\section{Lapina}
\begin{itemize}
\item {Grp. gram.:m.}
\end{itemize}
\begin{itemize}
\item {Utilização:Prov.}
\end{itemize}
O mesmo que \textunderscore larápio\textunderscore .
(Cp. \textunderscore larapinar\textunderscore )
\section{Lapinha}
\begin{itemize}
\item {Grp. gram.:f.}
\end{itemize}
\begin{itemize}
\item {Utilização:Ant.}
\end{itemize}
Espécie de representação popular.
\section{Lápis}
\begin{itemize}
\item {Grp. gram.:m.}
\end{itemize}
\begin{itemize}
\item {Utilização:Ext.}
\end{itemize}
\begin{itemize}
\item {Proveniência:(Lat. \textunderscore lapis\textunderscore . Cp. entretanto cast. \textunderscore lápiz\textunderscore , que parece indicar outra or.)}
\end{itemize}
Substância escura e pouco consistente, que é um carboneto do ferro ou plombagina, e que serve para escrever ou desenhar.
Qualquer substância, que tenha fórma oblonga, o com que se possa escrever ou desenhar.
\section{Lapisada}
\begin{itemize}
\item {Grp. gram.:f.}
\end{itemize}
Traço a lápis. Cf. Garrett, \textunderscore Retr. de Vénus\textunderscore , 71.
\section{Lapiseira}
\begin{itemize}
\item {Grp. gram.:f.}
\end{itemize}
Tubo ou caixa, em que se guardam os lápis.
\section{Lapiseiro}
\begin{itemize}
\item {Grp. gram.:m.}
\end{itemize}
(V.lapiseira)
\section{Lápis-lazúli}
\begin{itemize}
\item {Grp. gram.:m.}
\end{itemize}
O mesmo que \textunderscore lazulite\textunderscore .
\section{Lapláceas}
\begin{itemize}
\item {Grp. gram.:f. pl.}
\end{itemize}
\begin{itemize}
\item {Proveniência:(De \textunderscore Laplace\textunderscore , n. p.)}
\end{itemize}
Tríbo de plantas ternstremiáceas.
\section{Lapónio}
\begin{itemize}
\item {Grp. gram.:m.  e  adj.}
\end{itemize}
Indivíduo lapuz, labrego.
(Cp. \textunderscore lapão\textunderscore ^2)
\section{Lapouço}
\begin{itemize}
\item {Grp. gram.:m.}
\end{itemize}
\begin{itemize}
\item {Utilização:Prov.}
\end{itemize}
\begin{itemize}
\item {Utilização:trasm.}
\end{itemize}
\begin{itemize}
\item {Grp. gram.:Adj.}
\end{itemize}
Coêlho novo.
Homem gordo.
Sujo.
Estúpido.
(Cp. \textunderscore lapuz\textunderscore )
\section{Lapáceo}
\begin{itemize}
\item {Grp. gram.:adj.}
\end{itemize}
\begin{itemize}
\item {Utilização:Bot.}
\end{itemize}
\begin{itemize}
\item {Proveniência:(Lat. \textunderscore lappaceus\textunderscore )}
\end{itemize}
Diz-se das brácteas, que se curvam em ponta de anzol.
\section{Lappáceo}
\begin{itemize}
\item {Grp. gram.:adj.}
\end{itemize}
\begin{itemize}
\item {Utilização:Bot.}
\end{itemize}
\begin{itemize}
\item {Proveniência:(Lat. \textunderscore lappaceus\textunderscore )}
\end{itemize}
Diz-se das brácteas, que se curvam em ponta de anzol.
\section{Lapso}
\begin{itemize}
\item {Grp. gram.:m.}
\end{itemize}
\begin{itemize}
\item {Proveniência:(Lat. \textunderscore lapsus\textunderscore )}
\end{itemize}
Acto de escorregar.
Descuido; culpa: \textunderscore um lapso de memória\textunderscore .
Acto de correr (o tempo); decurso de tempo: \textunderscore aperfeiçoou-se, no lapso de déz annos\textunderscore .
\section{Lapúrdio}
\begin{itemize}
\item {Grp. gram.:m.  e  adj.}
\end{itemize}
O mesmo que \textunderscore lapuz\textunderscore .
\section{Lapuz}
\begin{itemize}
\item {Grp. gram.:m.  e  adj.}
\end{itemize}
\begin{itemize}
\item {Proveniência:(Do rad. de \textunderscore lapa\textunderscore . Cp. \textunderscore lapão\textunderscore ^2)}
\end{itemize}
Homem grosseiro, rude; labrego.
\section{Lapuzado}
\begin{itemize}
\item {Grp. gram.:adj.}
\end{itemize}
Relativo a lapuz.
\section{Lapuzice}
\begin{itemize}
\item {Grp. gram.:f.}
\end{itemize}
Qualidade de lapuz.
\section{Laque}
\begin{itemize}
\item {Grp. gram.:m.}
\end{itemize}
\begin{itemize}
\item {Utilização:T. da Índia Port}
\end{itemize}
Cem mil.--(Na Malásia e Zanzibar, corresponde a dez mil).
(Indostano \textunderscore lak\textunderscore )
\section{Laqueação}
\begin{itemize}
\item {Grp. gram.:f.}
\end{itemize}
Acto ou effeito de laquear^1.
\section{Laquear}
\begin{itemize}
\item {Grp. gram.:v. t.}
\end{itemize}
\begin{itemize}
\item {Proveniência:(Lat. \textunderscore laqueare\textunderscore )}
\end{itemize}
Enlaçar.
Ligar (artéria cortada ou ferida).
\section{Laquear}
\begin{itemize}
\item {Grp. gram.:m.}
\end{itemize}
\begin{itemize}
\item {Proveniência:(Lat. \textunderscore laqueare\textunderscore )}
\end{itemize}
Sobrecéu; dossel do leito.
\section{Laqueário}
\begin{itemize}
\item {Grp. gram.:m.}
\end{itemize}
\begin{itemize}
\item {Proveniência:(Lat. \textunderscore laquearius\textunderscore )}
\end{itemize}
Gladiador, que, na arena, impedia os movimentos do adversário, atirando-lhe uma corda, com que o prendia em nó corredio.
\section{Laqueca}
\begin{itemize}
\item {Grp. gram.:f.}
\end{itemize}
\begin{itemize}
\item {Proveniência:(Do ár. \textunderscore aquica\textunderscore )}
\end{itemize}
Pedra lustrosa e avermelhada ou alaranjada do Oriente.
\section{Laqueta}
\begin{itemize}
\item {fónica:quê}
\end{itemize}
\begin{itemize}
\item {Grp. gram.:f.}
\end{itemize}
Espécie de tecido antigo.
\section{Lar}
\begin{itemize}
\item {Grp. gram.:m.}
\end{itemize}
\begin{itemize}
\item {Utilização:Fig.}
\end{itemize}
\begin{itemize}
\item {Proveniência:(Lat. \textunderscore lar\textunderscore )}
\end{itemize}
Lugar, em que se accende fogo, na cozinha.
Parte ou face inferior do pão.
Casa de habitação.
Pátria.
Família.
\section{Laracha}
\begin{itemize}
\item {Grp. gram.:f.}
\end{itemize}
\begin{itemize}
\item {Utilização:Chul.}
\end{itemize}
\begin{itemize}
\item {Grp. gram.:M.}
\end{itemize}
Motejo; chalaça.
Aquelle que diz facécias ou procura sêr gracioso.
\section{Laracheador}
\begin{itemize}
\item {Grp. gram.:m.}
\end{itemize}
Aquelle que laracheia.
\section{Larachear}
\begin{itemize}
\item {Grp. gram.:v. i.}
\end{itemize}
\begin{itemize}
\item {Utilização:Chul.}
\end{itemize}
Dizer larachas.
\section{Larachento}
\begin{itemize}
\item {Grp. gram.:adj.}
\end{itemize}
Em que há laracha.
Motejador. Cf. Camillo, \textunderscore Volcões\textunderscore , 88.
\section{Larachista}
\begin{itemize}
\item {Grp. gram.:m.  e  f.}
\end{itemize}
Pessôa, que laracheia.
\section{Larada}
\begin{itemize}
\item {Grp. gram.:f.}
\end{itemize}
\begin{itemize}
\item {Utilização:Prov.}
\end{itemize}
\begin{itemize}
\item {Utilização:Fam.}
\end{itemize}
\begin{itemize}
\item {Utilização:Prov.}
\end{itemize}
\begin{itemize}
\item {Utilização:alg.}
\end{itemize}
\begin{itemize}
\item {Proveniência:(De \textunderscore lar\textunderscore )}
\end{itemize}
Cinza do lar.
Mancha larga, produzida por um líquido entornado.
Communidade immoral, em que vivem os pescadores unhantes da ria de Aveiro.
Porção de coisas, com que se cobre a lareira: \textunderscore uma larada de castanhas\textunderscore .
Família ou pessôas que cercam toda a lareira:«\textunderscore Fulano tem uma larada de filhos.\textunderscore »
Serão, á lareira:«\textunderscore ...nas laradas de inverno, agora é que é recontar.\textunderscore »Júl. Castilho, \textunderscore Manuelinas\textunderscore , 67.
Porção de excremento, um tanto líquido.
\section{Laraita}
\begin{itemize}
\item {Grp. gram.:f.}
\end{itemize}
\begin{itemize}
\item {Utilização:Prov.}
\end{itemize}
\begin{itemize}
\item {Utilização:trasm.}
\end{itemize}
\begin{itemize}
\item {Utilização:Prov.}
\end{itemize}
\begin{itemize}
\item {Utilização:beir.}
\end{itemize}
Porca grande e magra; galdrapa.
Fome, larica.
\section{Laranja}
\begin{itemize}
\item {Grp. gram.:f.}
\end{itemize}
\begin{itemize}
\item {Utilização:Bras}
\end{itemize}
\begin{itemize}
\item {Proveniência:(Do ár. \textunderscore naranj\textunderscore )}
\end{itemize}
Fruto da laranjeira.
Laranjeira azeda.
Variedade de pêra portuguesa.
\textunderscore Meia laranja\textunderscore , meia volta, num caminho ou numa viagem marítima:«\textunderscore navegámos á meia laranja rodando o Cabo.\textunderscore »\textunderscore Roteiro do Mar-Vermelho\textunderscore .
\section{Laranjada}
\begin{itemize}
\item {Grp. gram.:f.}
\end{itemize}
Bebida, em que entra o sumo da laranja.
Grande porção de laranjas.
Arremêsso de laranja: \textunderscore dantes, no Carnaval, jogava-se a laranjada\textunderscore .
\section{Laranjal}
\begin{itemize}
\item {Grp. gram.:m.}
\end{itemize}
\begin{itemize}
\item {Proveniência:(De \textunderscore laranja\textunderscore )}
\end{itemize}
Pomar de laranjeiras.
\section{Laranjeira}
\begin{itemize}
\item {Grp. gram.:f.}
\end{itemize}
\begin{itemize}
\item {Proveniência:(De \textunderscore laranja\textunderscore )}
\end{itemize}
Árvore sempre verde, da fam. das auranciáceas, (\textunderscore citrus aurantium\textunderscore , Lin.).
\section{Laranjeirinha}
\begin{itemize}
\item {Grp. gram.:f.}
\end{itemize}
\begin{itemize}
\item {Proveniência:(De \textunderscore laranjeira\textunderscore )}
\end{itemize}
Arbusto polygaláceo do Brasil.
\section{Laranjeiro}
\begin{itemize}
\item {Grp. gram.:adj.}
\end{itemize}
\begin{itemize}
\item {Grp. gram.:M.}
\end{itemize}
\begin{itemize}
\item {Utilização:Açor}
\end{itemize}
Diz-se de uma variedade de feijão.
Homem, que se emprega em encaixotar laranjas para embarque.
\section{Laranjinha}
\begin{itemize}
\item {Grp. gram.:f.}
\end{itemize}
\begin{itemize}
\item {Utilização:Bras}
\end{itemize}
\begin{itemize}
\item {Utilização:Bras}
\end{itemize}
\begin{itemize}
\item {Utilização:Bras}
\end{itemize}
\begin{itemize}
\item {Proveniência:(De \textunderscore laranja\textunderscore )}
\end{itemize}
Espécie de jôgo popular.
Bebida alcoólica, em que entra o sumo de laranja.
O mesmo que \textunderscore cabacinha\textunderscore .
Árvore, cuja madeira é amarela.
\section{Laranjinha-do-mato}
\begin{itemize}
\item {Grp. gram.:f.}
\end{itemize}
\begin{itemize}
\item {Utilização:Bras}
\end{itemize}
O mesmo que \textunderscore tinguaci\textunderscore .
\section{Laranjitas-de-quito}
\begin{itemize}
\item {Grp. gram.:f.}
\end{itemize}
Planta solânea do Alto Amazonas.
\section{Laranjo}
\begin{itemize}
\item {Grp. gram.:adj.}
\end{itemize}
\begin{itemize}
\item {Utilização:Bras. do S}
\end{itemize}
Diz-se do boi, que tem côr de laranja.
\section{Larapa}
\begin{itemize}
\item {Grp. gram.:f.}
\end{itemize}
\begin{itemize}
\item {Utilização:Prov.}
\end{itemize}
\begin{itemize}
\item {Utilização:alg.}
\end{itemize}
O mesmo que \textunderscore água-pé\textunderscore ^1.
\section{Larapiar}
\begin{itemize}
\item {Grp. gram.:v. t.}
\end{itemize}
\begin{itemize}
\item {Proveniência:(De \textunderscore larápio\textunderscore )}
\end{itemize}
Furtar; surripiar.
\section{Larapinar}
\begin{itemize}
\item {Grp. gram.:v. t.}
\end{itemize}
\begin{itemize}
\item {Utilização:Prov.}
\end{itemize}
\begin{itemize}
\item {Utilização:trasm.}
\end{itemize}
\begin{itemize}
\item {Proveniência:(De \textunderscore larápio\textunderscore , sob infl. de \textunderscore rapina\textunderscore )}
\end{itemize}
O mesmo que \textunderscore larapiar\textunderscore .
\section{Larápio}
\begin{itemize}
\item {Grp. gram.:m.}
\end{itemize}
\begin{itemize}
\item {Utilização:Pop.}
\end{itemize}
Gatuno; aquelle que tem o hábito de furtar.
\section{Larário}
\begin{itemize}
\item {Grp. gram.:m.}
\end{itemize}
\begin{itemize}
\item {Utilização:Fig.}
\end{itemize}
\begin{itemize}
\item {Proveniência:(Lat. \textunderscore lararium\textunderscore )}
\end{itemize}
Espécie de capella, em que as famílias romanas guardavam os deuses protectores do lar.
O lar; o seio da família.
\section{Larcão}
\begin{itemize}
\item {Grp. gram.:m.}
\end{itemize}
\begin{itemize}
\item {Utilização:Prov.}
\end{itemize}
\begin{itemize}
\item {Utilização:beir.}
\end{itemize}
Carne de porco, tirada de entre o chispe e a parte mais gorda da espádua.
(Cp. \textunderscore lacão\textunderscore )
\section{Lardeadeira}
\begin{itemize}
\item {Grp. gram.:f.}
\end{itemize}
Agulha para lardear.
\section{Lardear}
\begin{itemize}
\item {Grp. gram.:v. t.}
\end{itemize}
\begin{itemize}
\item {Utilização:Fig.}
\end{itemize}
\begin{itemize}
\item {Proveniência:(De \textunderscore lardo\textunderscore )}
\end{itemize}
Entremear com toucinho (uma peça de carne).
Entremear.
\section{Lardiforme}
\begin{itemize}
\item {Grp. gram.:adj.}
\end{itemize}
\begin{itemize}
\item {Proveniência:(De \textunderscore lardo\textunderscore  + \textunderscore fórma\textunderscore )}
\end{itemize}
Que tem fórma de lardo.
\section{Lardita}
\begin{itemize}
\item {Grp. gram.:f.}
\end{itemize}
\begin{itemize}
\item {Proveniência:(De \textunderscore lardo\textunderscore )}
\end{itemize}
Silicato de alumina, transparente, com aspecto de toicinho.
\section{Lardívoro}
\begin{itemize}
\item {Grp. gram.:adj.}
\end{itemize}
\begin{itemize}
\item {Proveniência:(Do lat. \textunderscore lardum\textunderscore  + \textunderscore vorare\textunderscore )}
\end{itemize}
Que devora toicinho.
\section{Lardizabal}
\begin{itemize}
\item {Grp. gram.:m.}
\end{itemize}
Gênero de plantas trepadeiras do Peru.
(Do \textunderscore Lardizabal\textunderscore , n. p.)
\section{Lardizabáleas}
\begin{itemize}
\item {Grp. gram.:f. pl.}
\end{itemize}
\begin{itemize}
\item {Proveniência:(De \textunderscore lardizabáleo\textunderscore )}
\end{itemize}
Família de plantas sarmentosas, originárias da China, do Japão e do Chile.
Primeira secção da fam. das monospérmeas.
\section{Lardizabáleo}
\begin{itemize}
\item {Grp. gram.:adj.}
\end{itemize}
Semelhante ao lardizabal.
\section{Lardo}
\begin{itemize}
\item {Grp. gram.:m.}
\end{itemize}
\begin{itemize}
\item {Utilização:Fig.}
\end{itemize}
\begin{itemize}
\item {Proveniência:(Lat. \textunderscore lardum\textunderscore )}
\end{itemize}
Toicinho, especialmente toicinho em tiras ou talhadinhas, para entremear peças de carne.
Condimento; ornato:«\textunderscore sem lardo de história nem de mythos.\textunderscore »Camillo, \textunderscore Corja\textunderscore , 183.
\section{Lardoeirada}
\begin{itemize}
\item {Grp. gram.:f.}
\end{itemize}
\begin{itemize}
\item {Utilização:Prov.}
\end{itemize}
\begin{itemize}
\item {Utilização:trasm.}
\end{itemize}
O mesmo que \textunderscore pancada\textunderscore .
\section{Laré}
\begin{itemize}
\item {Grp. gram.:m.}
\end{itemize}
\begin{itemize}
\item {Utilização:Prov.}
\end{itemize}
\begin{itemize}
\item {Utilização:alent.}
\end{itemize}
\begin{itemize}
\item {Grp. gram.:Loc. adv.}
\end{itemize}
Pessôa, que dança mal.
\textunderscore Ao laré\textunderscore , á tuna; vadiando.
De gorra ou de patuscada.
\section{Larear}
\begin{itemize}
\item {Grp. gram.:v. i.}
\end{itemize}
\begin{itemize}
\item {Utilização:Pop.}
\end{itemize}
Andar ao laré; vadiar.
\section{Larecer}
\begin{itemize}
\item {Grp. gram.:v. i.}
\end{itemize}
\begin{itemize}
\item {Utilização:Prov.}
\end{itemize}
\begin{itemize}
\item {Utilização:minh.}
\end{itemize}
Falar muito, tagarelar.
(Por \textunderscore lerecer\textunderscore , relacionado com \textunderscore léria\textunderscore ?)
\section{Laredo}
\begin{itemize}
\item {fónica:larê}
\end{itemize}
\begin{itemize}
\item {Utilização:Prov.}
\end{itemize}
\begin{itemize}
\item {Utilização:alg.}
\end{itemize}
Conjunto de recifes cascalhosos.
\section{Larega}
\begin{itemize}
\item {Grp. gram.:f.}
\end{itemize}
\begin{itemize}
\item {Utilização:Prov.}
\end{itemize}
\begin{itemize}
\item {Utilização:trasm.}
\end{itemize}
\begin{itemize}
\item {Utilização:Fig.}
\end{itemize}
Pequena porca.
Mulher gorda e atarracada.
\section{Larego}
\begin{itemize}
\item {Grp. gram.:m.}
\end{itemize}
\begin{itemize}
\item {Utilização:Prov.}
\end{itemize}
\begin{itemize}
\item {Utilização:trasm.}
\end{itemize}
Pequeno porco, entre leitão e cevado.
\section{Lareira}
\begin{itemize}
\item {Grp. gram.:f.}
\end{itemize}
\begin{itemize}
\item {Proveniência:(De \textunderscore lar\textunderscore )}
\end{itemize}
Laja, em que se accende o fogo; lar.
\section{Lareiras}
\begin{itemize}
\item {Grp. gram.:m.}
\end{itemize}
\begin{itemize}
\item {Utilização:Prov.}
\end{itemize}
\begin{itemize}
\item {Utilização:trasm.}
\end{itemize}
O mesmo que \textunderscore langueiras\textunderscore .
\section{Lareiro}
\begin{itemize}
\item {Grp. gram.:adj.}
\end{itemize}
Relativo a lar ou a lareira.
\section{Lareiro}
\begin{itemize}
\item {Grp. gram.:m.}
\end{itemize}
\begin{itemize}
\item {Utilização:Prov.}
\end{itemize}
\begin{itemize}
\item {Utilização:trasm.}
\end{itemize}
Cacete grande, o mesmo que \textunderscore jarundo\textunderscore .
\section{Lares}
\begin{itemize}
\item {Grp. gram.:f. pl.}
\end{itemize}
\begin{itemize}
\item {Utilização:Prov.}
\end{itemize}
\begin{itemize}
\item {Utilização:trasm.}
\end{itemize}
O mesmo que \textunderscore lárias\textunderscore .
\section{Lares}
\begin{itemize}
\item {Grp. gram.:f. pl.}
\end{itemize}
\begin{itemize}
\item {Proveniência:(Lat. \textunderscore lares\textunderscore )}
\end{itemize}
Deuses familiares, deuses protectores do lar ou da família, entre os Etruscos e entre os Romanos.
\section{Lareta}
\begin{itemize}
\item {fónica:larê}
\end{itemize}
\begin{itemize}
\item {Grp. gram.:m. ,  f.  e  adj.}
\end{itemize}
\begin{itemize}
\item {Utilização:Prov.}
\end{itemize}
\begin{itemize}
\item {Utilização:beir.}
\end{itemize}
O mesmo que \textunderscore traquina\textunderscore . (Colhido no Fundão)
\section{Larétia}
\begin{itemize}
\item {Grp. gram.:f.}
\end{itemize}
Gênero de plantas umbellíferas.
\section{Laréu}
\begin{itemize}
\item {Grp. gram.:m.}
\end{itemize}
\begin{itemize}
\item {Utilização:Prov.}
\end{itemize}
\begin{itemize}
\item {Utilização:trasm.}
\end{itemize}
O mesmo que \textunderscore léu\textunderscore .
\section{Larga}
\begin{itemize}
\item {Grp. gram.:f.}
\end{itemize}
\begin{itemize}
\item {Utilização:Fig.}
\end{itemize}
\begin{itemize}
\item {Grp. gram.:Loc. adv.}
\end{itemize}
\begin{itemize}
\item {Proveniência:(De \textunderscore largo\textunderscore )}
\end{itemize}
Acto ou effeito de largar.
Espécie de gancho de ferro, com que se prende ao banco de carpinteiro ou marceneiro a madeira em que se trabalha.
Largueza, liberdade.
Ampliação.
\textunderscore Á larga\textunderscore , com largueza; generosamente; á vontade; desafogadamente.
\section{Largada}
\begin{itemize}
\item {Grp. gram.:f.}
\end{itemize}
\begin{itemize}
\item {Utilização:Fig.}
\end{itemize}
Acto de \textunderscore largar\textunderscore : \textunderscore a largada do vapor\textunderscore .
Chiste: \textunderscore tem bôas largadas\textunderscore .
\section{Largado}
\begin{itemize}
\item {Grp. gram.:adj.}
\end{itemize}
\begin{itemize}
\item {Utilização:Bras. do S}
\end{itemize}
\begin{itemize}
\item {Utilização:Fig.}
\end{itemize}
\begin{itemize}
\item {Proveniência:(De \textunderscore largar\textunderscore )}
\end{itemize}
Folgado, indómito.
Abandonado, como indomável, (falando-se do cavallo).
Diz-se do homem incorrigível.
\section{Largamente}
\begin{itemize}
\item {Grp. gram.:adj.}
\end{itemize}
De modo largo.
Com largueza, com generosidade.
Extensamente; minuciosamente: \textunderscore discorrer largamente\textunderscore .
\section{Largar}
\begin{itemize}
\item {Grp. gram.:v. t.}
\end{itemize}
\begin{itemize}
\item {Grp. gram.:V. i.}
\end{itemize}
\begin{itemize}
\item {Utilização:Gír.}
\end{itemize}
\begin{itemize}
\item {Proveniência:(De \textunderscore largo\textunderscore )}
\end{itemize}
Soltar da mão: \textunderscore largou a bengala\textunderscore .
Dar liberdade a: \textunderscore largou o cão\textunderscore .
Deixar: \textunderscore largou as más companhias\textunderscore .
Alargar.
Ceder; conceder.
Emittir: \textunderscore largar opinião\textunderscore .
Desfraldar: \textunderscore largar bandeiras\textunderscore .
Proferir: \textunderscore largar disparates\textunderscore .
Mentir.
\section{Largata}
\begin{itemize}
\item {Grp. gram.:f.}
\end{itemize}
\begin{itemize}
\item {Utilização:Prov.}
\end{itemize}
\begin{itemize}
\item {Utilização:trasm.}
\end{itemize}
O mesmo que \textunderscore lagarta\textunderscore .
(Metáth. de \textunderscore lagarta\textunderscore )
\section{Largífico}
\begin{itemize}
\item {Grp. gram.:adj.}
\end{itemize}
\begin{itemize}
\item {Utilização:Des.}
\end{itemize}
\begin{itemize}
\item {Proveniência:(Lat. \textunderscore largificus\textunderscore )}
\end{itemize}
Abundante, copioso.
\section{Largífluo}
\begin{itemize}
\item {Grp. gram.:adj.}
\end{itemize}
\begin{itemize}
\item {Utilização:Poét.}
\end{itemize}
\begin{itemize}
\item {Proveniência:(Lat. \textunderscore largifluus\textunderscore )}
\end{itemize}
Que corre abundantemente.
\section{Largina}
\begin{itemize}
\item {Grp. gram.:f.}
\end{itemize}
Medicamento contra a blennorrhagia, e que é uma das combinações da prata.
\section{Largo}
\begin{itemize}
\item {Grp. gram.:adj.}
\end{itemize}
\begin{itemize}
\item {Utilização:Fig.}
\end{itemize}
\begin{itemize}
\item {Grp. gram.:M.}
\end{itemize}
\begin{itemize}
\item {Utilização:Gír.}
\end{itemize}
\begin{itemize}
\item {Grp. gram.:Adv.}
\end{itemize}
\begin{itemize}
\item {Proveniência:(Lat. \textunderscore largus\textunderscore )}
\end{itemize}
Que é extenso, de lado a lado: \textunderscore rio largo\textunderscore .
Que tem largura.
Amplo.
Generoso.
Minucioso.
Duradoiro; numeroso: \textunderscore por largos annos\textunderscore .
Demorado.
Importante.
Copioso: \textunderscore larga colheita de apontamentos\textunderscore .
Desenvolvido; longo: \textunderscore discurso largo\textunderscore .
Lasso; não apertado: \textunderscore corda larga\textunderscore .
Largura: \textunderscore tem dois metros de largo\textunderscore .
Pequena praça.
Alto mar.
Casaco.
Com largueza; largamente.
\section{Largueador}
\begin{itemize}
\item {Grp. gram.:m.  e  adj.}
\end{itemize}
Aquelle que largueia.
\section{Larguear}
\begin{itemize}
\item {Grp. gram.:v. t.}
\end{itemize}
\begin{itemize}
\item {Proveniência:(De \textunderscore largo\textunderscore )}
\end{itemize}
Despender largamente; prodigalizar.
Alargar.
\section{Largueirão}
\begin{itemize}
\item {Grp. gram.:adj.}
\end{itemize}
\begin{itemize}
\item {Utilização:Pop.}
\end{itemize}
Muito largo: \textunderscore casaco largueirão\textunderscore .
\section{Largueto}
\begin{itemize}
\item {fónica:guê}
\end{itemize}
\begin{itemize}
\item {Grp. gram.:adv.}
\end{itemize}
\begin{itemize}
\item {Utilização:Mús.}
\end{itemize}
\begin{itemize}
\item {Proveniência:(It. \textunderscore larghetto\textunderscore )}
\end{itemize}
Menos largo, (indicando-se um andamento musical, menos lento que o largo).
\section{Largueza}
\begin{itemize}
\item {Grp. gram.:f.}
\end{itemize}
\begin{itemize}
\item {Utilização:Fig.}
\end{itemize}
O mesmo que \textunderscore largura\textunderscore .
Generosidade; bizarria.
Dissipação.
\section{Largura}
\begin{itemize}
\item {Grp. gram.:f.}
\end{itemize}
Qualidade daquilllo que é largo.
A mais pequena das duas dimensões, (tratando-se de uma superfície).
\section{Laria}
\begin{itemize}
\item {Grp. gram.:f.}
\end{itemize}
\begin{itemize}
\item {Utilização:Gír.}
\end{itemize}
Laranja.
\section{Larião}
\begin{itemize}
\item {Grp. gram.:m.}
\end{itemize}
\begin{itemize}
\item {Utilização:Prov.}
\end{itemize}
\begin{itemize}
\item {Utilização:alg.}
\end{itemize}
O mesmo que \textunderscore leirão\textunderscore ^1.
\section{Lárias}
\begin{itemize}
\item {Grp. gram.:f. pl.}
\end{itemize}
\begin{itemize}
\item {Utilização:Prov.}
\end{itemize}
\begin{itemize}
\item {Utilização:trasm.}
\end{itemize}
\begin{itemize}
\item {Proveniência:(De \textunderscore lar\textunderscore )}
\end{itemize}
Cadeia de ferro, que pende do tecto na cozinha; cremalheira.
\section{Larica}
\begin{itemize}
\item {Grp. gram.:f.}
\end{itemize}
\begin{itemize}
\item {Utilização:Fam.}
\end{itemize}
Joio.
Fome.
\section{Lárice}
\begin{itemize}
\item {Grp. gram.:f.}
\end{itemize}
Árvore conifera, (\textunderscore pinus larix\textunderscore , Lin.).
\section{Larício}
\begin{itemize}
\item {Grp. gram.:m.}
\end{itemize}
Árvore conifera, (\textunderscore pinus larix\textunderscore , Lin.).
\section{Lárico}
\begin{itemize}
\item {Grp. gram.:adj.}
\end{itemize}
Relativo aos lares.
\section{Lariço}
\begin{itemize}
\item {Grp. gram.:m.}
\end{itemize}
(V.larício)
\section{Larida}
\begin{itemize}
\item {Grp. gram.:f.}
\end{itemize}
\begin{itemize}
\item {Utilização:Pop.}
\end{itemize}
O mesmo que \textunderscore alarida\textunderscore .
\section{Larífugo}
\begin{itemize}
\item {Grp. gram.:m.}
\end{itemize}
\begin{itemize}
\item {Utilização:Des.}
\end{itemize}
\begin{itemize}
\item {Proveniência:(Lat. \textunderscore larifugus\textunderscore )}
\end{itemize}
Aquelle que abandona os lares.
Vagabundo; vadio.
\section{Larim}
\begin{itemize}
\item {Grp. gram.:m.}
\end{itemize}
Habitante de Lara.
Antiga moéda da Índia portuguesa.
Moéda de prata na Pérsia.
\section{Larim}
\begin{itemize}
\item {Grp. gram.:m.}
\end{itemize}
Árvore espinhosa do Oriente.
\section{Laringalgia}
\begin{itemize}
\item {Grp. gram.:f.}
\end{itemize}
\begin{itemize}
\item {Proveniência:(Do gr. \textunderscore larunx\textunderscore  + \textunderscore algos\textunderscore )}
\end{itemize}
Neuralgia laríngea.
\section{Laringe}
\begin{itemize}
\item {Grp. gram.:f.}
\end{itemize}
\begin{itemize}
\item {Proveniência:(Gr. \textunderscore larunx\textunderscore )}
\end{itemize}
Parte superior da traqueia, e órgão principal da voz.
\section{Laríngeo}
\begin{itemize}
\item {Grp. gram.:adj.}
\end{itemize}
Relativo á laringe.
\section{Laringiano}
\begin{itemize}
\item {Grp. gram.:adj.}
\end{itemize}
O mesmo que \textunderscore laríngeo\textunderscore .
\section{Laringismo}
\begin{itemize}
\item {Grp. gram.:m.}
\end{itemize}
Contracção espasmódica dos músculos da laringe, por acção reflexa na epilepsia e em certos estados nervosos.
\section{Laringite}
\begin{itemize}
\item {Grp. gram.:f.}
\end{itemize}
Inflamação da laringe.
\section{Laringografia}
\begin{itemize}
\item {Grp. gram.:f.}
\end{itemize}
\begin{itemize}
\item {Proveniência:(Do gr. \textunderscore larunx\textunderscore  + \textunderscore graphein\textunderscore )}
\end{itemize}
Descripção da laringe.
\section{Laringologia}
\begin{itemize}
\item {Grp. gram.:f.}
\end{itemize}
\begin{itemize}
\item {Proveniência:(Do gr. \textunderscore larunx\textunderscore  + \textunderscore logos\textunderscore )}
\end{itemize}
Tratado á cêrca da laringe.
Teoria á cêrca da laringe.
\section{Laringoplegia}
\begin{itemize}
\item {Grp. gram.:f.}
\end{itemize}
Paralisia da laringe.
\section{Laringoscopia}
\begin{itemize}
\item {Grp. gram.:f.}
\end{itemize}
Observação interior da laringe, por meio do laringoscópio.
(Cp. \textunderscore laringoscópio\textunderscore )
\section{Laringoscópio}
\begin{itemize}
\item {Grp. gram.:m.}
\end{itemize}
\begin{itemize}
\item {Proveniência:(Do gr. \textunderscore larunx\textunderscore  + \textunderscore skopein\textunderscore )}
\end{itemize}
Instrumento, para examinar o interior da laringe.
\section{Laringotifo}
\begin{itemize}
\item {Grp. gram.:f.}
\end{itemize}
\begin{itemize}
\item {Proveniência:(De \textunderscore laringe\textunderscore  + \textunderscore tifo\textunderscore )}
\end{itemize}
Ulceração da mucosa da laringe, constituindo um accidente secundário do tifo.
\section{Laringóstomo}
\begin{itemize}
\item {Grp. gram.:adj.}
\end{itemize}
\begin{itemize}
\item {Utilização:Zool.}
\end{itemize}
\begin{itemize}
\item {Proveniência:(Do gr. \textunderscore larunx\textunderscore  + \textunderscore stoma\textunderscore )}
\end{itemize}
Diz-se do animal articulado, cuja bôca é uma espécie de tromba, formada pelo esófago.
\section{Laringotomia}
\begin{itemize}
\item {Grp. gram.:f.}
\end{itemize}
\begin{itemize}
\item {Proveniência:(Do gr. \textunderscore larunx\textunderscore  + \textunderscore tome\textunderscore )}
\end{itemize}
Incisão na laringe, para a extracção de um corpo estranho.
\section{Lárix}
\begin{itemize}
\item {Grp. gram.:f.}
\end{itemize}
O mesmo que \textunderscore lárice\textunderscore .
\section{Laró}
\begin{itemize}
\item {Grp. gram.:m.}
\end{itemize}
\begin{itemize}
\item {Utilização:Prov.}
\end{itemize}
\begin{itemize}
\item {Utilização:Prov.}
\end{itemize}
\begin{itemize}
\item {Utilização:alent.}
\end{itemize}
Uma das peças da asna.
O mesmo que \textunderscore léu\textunderscore .
O mesmo que \textunderscore laroz\textunderscore .
Intersecção de duas vertentes no telhado, formando ângulo reintrante.
\section{Larota}
\begin{itemize}
\item {Grp. gram.:f.}
\end{itemize}
\begin{itemize}
\item {Utilização:trasm}
\end{itemize}
\begin{itemize}
\item {Utilização:Gír.}
\end{itemize}
O mesmo que \textunderscore larica\textunderscore .
\section{Laroteiro}
\begin{itemize}
\item {Grp. gram.:m.  e  adj.}
\end{itemize}
\begin{itemize}
\item {Utilização:Prov.}
\end{itemize}
\begin{itemize}
\item {Utilização:trasm.}
\end{itemize}
Mandrião.
Velhaco.
\section{Laroz}
\begin{itemize}
\item {Grp. gram.:m.}
\end{itemize}
Barrote, que sustenta a tacaniça.
O mesmo que \textunderscore laró\textunderscore .
\section{Larpão}
\begin{itemize}
\item {Grp. gram.:m.}
\end{itemize}
\begin{itemize}
\item {Utilização:Prov.}
\end{itemize}
\begin{itemize}
\item {Utilização:trasm.}
\end{itemize}
Comilão.
\section{Larpar}
\begin{itemize}
\item {Grp. gram.:v. i.}
\end{itemize}
\begin{itemize}
\item {Utilização:Prov.}
\end{itemize}
\begin{itemize}
\item {Utilização:trasm.}
\end{itemize}
Comer muito.
\section{Larpeiro}
\begin{itemize}
\item {Grp. gram.:m.  e  adj.}
\end{itemize}
\begin{itemize}
\item {Utilização:Prov.}
\end{itemize}
\begin{itemize}
\item {Utilização:trasm.}
\end{itemize}
Comilão.
(Cp. gall. \textunderscore larpeiro\textunderscore )
\section{Larundos}
\begin{itemize}
\item {Grp. gram.:m. pl.}
\end{itemize}
\begin{itemize}
\item {Proveniência:(De \textunderscore Larunda\textunderscore , n. p.)}
\end{itemize}
O mesmo que \textunderscore lares\textunderscore ^2.
\section{Laruto}
\begin{itemize}
\item {Grp. gram.:m.}
\end{itemize}
\begin{itemize}
\item {Utilização:Prov.}
\end{itemize}
\begin{itemize}
\item {Utilização:trasm.}
\end{itemize}
\begin{itemize}
\item {Utilização:pop.}
\end{itemize}
O mesmo que \textunderscore langueiras\textunderscore .
\section{Larva}
\begin{itemize}
\item {Grp. gram.:f.}
\end{itemize}
\begin{itemize}
\item {Utilização:Constr.}
\end{itemize}
\begin{itemize}
\item {Proveniência:(Lat. \textunderscore larva\textunderscore )}
\end{itemize}
Primeiro estado dos insectos depois de saírem do ovo.
Barrote, que sustenta a tacaniça; laroz.
\section{Larvado}
\begin{itemize}
\item {Grp. gram.:adj.}
\end{itemize}
\begin{itemize}
\item {Utilização:Fam.}
\end{itemize}
\begin{itemize}
\item {Proveniência:(Lat. \textunderscore larvatus\textunderscore )}
\end{itemize}
Diz-se de algumas doenças, que apresentam intermittências e que, não sendo febres, têm alguma analogia como estas.
Que é doido, com intervallos lúcidos; desequilibrado; maníaco.
\section{Larval}
\begin{itemize}
\item {Grp. gram.:adj.}
\end{itemize}
\begin{itemize}
\item {Utilização:Poét.}
\end{itemize}
\begin{itemize}
\item {Proveniência:(Lat. \textunderscore larvalis\textunderscore )}
\end{itemize}
Relativo a larva.
Que é da natureza da larva.
Relativo a fantasmas.
Terrivel, assustador.
\section{Larvar}
\begin{itemize}
\item {Grp. gram.:adj.}
\end{itemize}
O mesmo ou melhor que \textunderscore larvário\textunderscore .
\section{Larvária}
\begin{itemize}
\item {Grp. gram.:f.}
\end{itemize}
\begin{itemize}
\item {Proveniência:(De \textunderscore larvário\textunderscore )}
\end{itemize}
Gênero de pólypos fósseis.
\section{Larvário}
\begin{itemize}
\item {Grp. gram.:adj.}
\end{itemize}
Relativo a larva.
\section{Larvícola}
\begin{itemize}
\item {Grp. gram.:adj.}
\end{itemize}
\begin{itemize}
\item {Proveniência:(Do lat. \textunderscore larva\textunderscore  + \textunderscore colere\textunderscore )}
\end{itemize}
Que vive no corpo da larva.
\section{Larvíparo}
\begin{itemize}
\item {Grp. gram.:adj.}
\end{itemize}
\begin{itemize}
\item {Utilização:Zool.}
\end{itemize}
\begin{itemize}
\item {Proveniência:(Do lat. \textunderscore larva\textunderscore  + \textunderscore pãrere\textunderscore )}
\end{itemize}
Diz-se dos animaes que, em vez de ovos, põem larvas.
\section{Larvívoro}
\begin{itemize}
\item {Grp. gram.:m.}
\end{itemize}
\begin{itemize}
\item {Proveniência:(Do lat. \textunderscore larva\textunderscore  + \textunderscore vorare\textunderscore )}
\end{itemize}
Espécie de melro.
\section{Laryngalgia}
\begin{itemize}
\item {Grp. gram.:f.}
\end{itemize}
\begin{itemize}
\item {Proveniência:(Do gr. \textunderscore larunx\textunderscore  + \textunderscore algos\textunderscore )}
\end{itemize}
Neuralgia larýngea.
\section{Larynge}
\begin{itemize}
\item {Grp. gram.:f.}
\end{itemize}
\begin{itemize}
\item {Proveniência:(Gr. \textunderscore larunx\textunderscore )}
\end{itemize}
Parte superior da tracheia, e órgão principal da voz.
\section{Larýngeo}
\begin{itemize}
\item {Grp. gram.:adj.}
\end{itemize}
Relativo á larynge.
\section{Laryngiano}
\begin{itemize}
\item {Grp. gram.:adj.}
\end{itemize}
O mesmo que \textunderscore larýngeo\textunderscore .
\section{Laryngismo}
\begin{itemize}
\item {Grp. gram.:m.}
\end{itemize}
Contracção espasmódica dos músculos da larynge, por acção reflexa na epilepsia e em certos estados nervosos.
\section{Laryngite}
\begin{itemize}
\item {Grp. gram.:f.}
\end{itemize}
Inflammação da larynge.
\section{Laryngographia}
\begin{itemize}
\item {Grp. gram.:f.}
\end{itemize}
\begin{itemize}
\item {Proveniência:(Do gr. \textunderscore larunx\textunderscore  + \textunderscore graphein\textunderscore )}
\end{itemize}
Descripção da larynge.
\section{Laryngologia}
\begin{itemize}
\item {Grp. gram.:f.}
\end{itemize}
\begin{itemize}
\item {Proveniência:(Do gr. \textunderscore larunx\textunderscore  + \textunderscore logos\textunderscore )}
\end{itemize}
Tratado á cêrca da larynge.
Theoria á cêrca da larynge.
\section{Laryngoplegia}
\begin{itemize}
\item {Grp. gram.:f.}
\end{itemize}
Paralysia da larynge.
\section{Laryngoscopia}
\begin{itemize}
\item {Grp. gram.:f.}
\end{itemize}
Observação interior da larynge, por meio do laryngoscópio.
(Cp. \textunderscore laringoscópio\textunderscore )
\section{Laryngoscópio}
\begin{itemize}
\item {Grp. gram.:m.}
\end{itemize}
\begin{itemize}
\item {Proveniência:(Do gr. \textunderscore larunx\textunderscore  + \textunderscore skopein\textunderscore )}
\end{itemize}
Instrumento, para examinar o interior da larynge.
\section{Laryngóstomo}
\begin{itemize}
\item {Grp. gram.:adj.}
\end{itemize}
\begin{itemize}
\item {Utilização:Zool.}
\end{itemize}
\begin{itemize}
\item {Proveniência:(Do gr. \textunderscore larunx\textunderscore  + \textunderscore stoma\textunderscore )}
\end{itemize}
Diz-se do animal articulado, cuja bôca é uma espécie de tromba, formada pelo esóphago.
\section{Laryngotomia}
\begin{itemize}
\item {Grp. gram.:f.}
\end{itemize}
\begin{itemize}
\item {Proveniência:(Do gr. \textunderscore larunx\textunderscore  + \textunderscore tome\textunderscore )}
\end{itemize}
Incisão na larynge, para a extracção de um corpo estranho.
\section{Laryngotypho}
\begin{itemize}
\item {Grp. gram.:f.}
\end{itemize}
\begin{itemize}
\item {Proveniência:(De \textunderscore larynge\textunderscore  + \textunderscore typho\textunderscore )}
\end{itemize}
Ulceração da mucosa da larynge, constituindo um accidente secundário do typho.
\section{Lasanha}
\begin{itemize}
\item {Grp. gram.:f.}
\end{itemize}
\begin{itemize}
\item {Proveniência:(It. \textunderscore lasagna\textunderscore )}
\end{itemize}
Tiras largas de massa de trigo para sopa.
\section{Lasca}
\begin{itemize}
\item {Grp. gram.:f.}
\end{itemize}
\begin{itemize}
\item {Proveniência:(De \textunderscore lascar\textunderscore )}
\end{itemize}
Fragmento ou estilhaço de madeira, pedra ou metal.
Fragmento.
Tira.
Peça de madeira, na borda dos barcos de pesca, pela qual passam as linhas das rêdes.
Espécie de jôgo de asar. Cf. \textunderscore Man. dos Jogos\textunderscore , 275.
\section{Lascar}
\begin{itemize}
\item {Grp. gram.:v. t.}
\end{itemize}
\begin{itemize}
\item {Grp. gram.:V. i.}
\end{itemize}
\begin{itemize}
\item {Utilização:Gír.}
\end{itemize}
\begin{itemize}
\item {Utilização:Prov.}
\end{itemize}
\begin{itemize}
\item {Utilização:minh.}
\end{itemize}
Partir em lascas.
Tirar lascas de.
Abrir em lascas.
Defecar, evacuar.
Fugir.
\section{Lascar}
\begin{itemize}
\item {Grp. gram.:m.}
\end{itemize}
\begin{itemize}
\item {Utilização:Ant.}
\end{itemize}
Corpo de soldados indianos.
(Persa \textunderscore laxcar\textunderscore )
\section{Lascarim}
\begin{itemize}
\item {Grp. gram.:m.}
\end{itemize}
\begin{itemize}
\item {Utilização:Ant.}
\end{itemize}
\begin{itemize}
\item {Utilização:Prov.}
\end{itemize}
\begin{itemize}
\item {Utilização:trasm.}
\end{itemize}
Soldado indiano ou moiro.
Marinheiro, que vivia a bordo com mulher e filhos.
Fedelho, doidelas, que gosta de andar descalço.
Cavallo que faz filetes.
(Persa \textunderscore laxcari\textunderscore )
\section{Lascarina}
\begin{itemize}
\item {Grp. gram.:m.}
\end{itemize}
\begin{itemize}
\item {Utilização:Prov.}
\end{itemize}
\begin{itemize}
\item {Utilização:trasm.}
\end{itemize}
O mesmo que \textunderscore lascarim\textunderscore .
\section{Lascarinho}
\begin{itemize}
\item {Grp. gram.:m.}
\end{itemize}
\begin{itemize}
\item {Utilização:T. da Bairrada}
\end{itemize}
Indivíduo desavergonhado.
\section{Lascarino}
\begin{itemize}
\item {Grp. gram.:adj.}
\end{itemize}
\begin{itemize}
\item {Grp. gram.:Adj.}
\end{itemize}
\begin{itemize}
\item {Utilização:Prov.}
\end{itemize}
\begin{itemize}
\item {Utilização:trasm.}
\end{itemize}
Larápio, ratoneiro. Cf. Rebello, \textunderscore Mocidade\textunderscore , I, 241.
Travesso, inquieto.
\section{Lascivamente}
\begin{itemize}
\item {Grp. gram.:adv.}
\end{itemize}
De modo lascivo; de modo sensual; com luxúria.
\section{Lascívia}
\begin{itemize}
\item {Grp. gram.:f.}
\end{itemize}
\begin{itemize}
\item {Proveniência:(Lat. \textunderscore lascivia\textunderscore )}
\end{itemize}
Carácter lascivo; qualidade de lascivo.
Luxúria.
\section{Lascivo}
\begin{itemize}
\item {Grp. gram.:adj.}
\end{itemize}
\begin{itemize}
\item {Proveniência:(Lat. \textunderscore lascivus\textunderscore )}
\end{itemize}
Brincalhão; travesso.
Desregrado; sensual; libidinoso.
\section{Laséguea}
\begin{itemize}
\item {Grp. gram.:f.}
\end{itemize}
\begin{itemize}
\item {Proveniência:(De \textunderscore Lasegue\textunderscore , n. p.)}
\end{itemize}
Gênero de plantas apocýneas.
\section{Lásia}
\begin{itemize}
\item {Grp. gram.:f.}
\end{itemize}
Gênero de plantas apocýneas.
\section{Lasiandra}
\begin{itemize}
\item {Grp. gram.:f.}
\end{itemize}
\begin{itemize}
\item {Proveniência:(Do gr. \textunderscore lasios\textunderscore  + \textunderscore aner\textunderscore , \textunderscore andros\textunderscore )}
\end{itemize}
Gênero de plantas melastoniaceas.
\section{Lasiântea}
\begin{itemize}
\item {Grp. gram.:f.}
\end{itemize}
\begin{itemize}
\item {Proveniência:(Do gr. \textunderscore lasios\textunderscore  + \textunderscore anthos\textunderscore )}
\end{itemize}
Gênero de plantas sinantéreas.
\section{Lasiantera}
\begin{itemize}
\item {Grp. gram.:f.}
\end{itemize}
Gênero de plantas ampelídeas.
\section{Lasiânthea}
\begin{itemize}
\item {Grp. gram.:f.}
\end{itemize}
\begin{itemize}
\item {Proveniência:(Do gr. \textunderscore lasios\textunderscore  + \textunderscore anthos\textunderscore )}
\end{itemize}
Gênero de plantas synanthéreas.
\section{Lasianthera}
\begin{itemize}
\item {Grp. gram.:f.}
\end{itemize}
Gênero de plantas ampelídeas.
\section{Lasiochlôa}
\begin{itemize}
\item {Grp. gram.:f.}
\end{itemize}
Gênero de plantas gramíneas.
\section{Lasioclôa}
\begin{itemize}
\item {Grp. gram.:f.}
\end{itemize}
Gênero de plantas gramíneas.
\section{Lasiocóride}
\begin{itemize}
\item {Grp. gram.:f.}
\end{itemize}
Espécie de plantas labiadas.
\section{Lasiocóryde}
\begin{itemize}
\item {Grp. gram.:f.}
\end{itemize}
Espécie de plantas labiadas.
\section{Lasionema}
\begin{itemize}
\item {Grp. gram.:f.}
\end{itemize}
Gênero de plantas rubiáceas.
\section{Lasionita}
\begin{itemize}
\item {Grp. gram.:f.}
\end{itemize}
\begin{itemize}
\item {Proveniência:(Do gr. \textunderscore lasios\textunderscore )}
\end{itemize}
Mineral, que toma a fórma de crystal capillar.
\section{Lasionite}
\begin{itemize}
\item {Grp. gram.:f.}
\end{itemize}
\begin{itemize}
\item {Proveniência:(Do gr. \textunderscore lasios\textunderscore )}
\end{itemize}
Mineral, que toma a fórma de crystal capillar.
\section{Lasiosperma}
\begin{itemize}
\item {Grp. gram.:f.}
\end{itemize}
\begin{itemize}
\item {Proveniência:(Do gr. \textunderscore lasios\textunderscore  + \textunderscore sperma\textunderscore )}
\end{itemize}
Gênero de plantas synanthereas.
\section{Lasiopétala}
\begin{itemize}
\item {Grp. gram.:f.}
\end{itemize}
\begin{itemize}
\item {Proveniência:(Do gr. \textunderscore lasios\textunderscore  + \textunderscore petalon\textunderscore )}
\end{itemize}
Gênero de arbustos australianos.
\section{Lasiopetáleas}
\begin{itemize}
\item {Grp. gram.:f. pl.}
\end{itemize}
\begin{itemize}
\item {Proveniência:(De \textunderscore lasiopétala\textunderscore )}
\end{itemize}
Tríbo de plantas bythneriáceas.
\section{Lasióptero}
\begin{itemize}
\item {Grp. gram.:m.}
\end{itemize}
\begin{itemize}
\item {Proveniência:(Do gr. \textunderscore lasios\textunderscore  + \textunderscore pteron\textunderscore )}
\end{itemize}
Gênero de insectos dípteros.
\section{Lassacuane}
\begin{itemize}
\item {Grp. gram.:m.}
\end{itemize}
Chefe das fôrças marítimas de Malaca, espécie de almirante, antes da conquista portuguesa. Cf. \textunderscore Comment. de Aff. de Albuq.\textunderscore 
\section{Lassar}
\begin{itemize}
\item {Grp. gram.:v. t.}
\end{itemize}
Tornar lasso. Cf. \textunderscore Tech. Rur.\textunderscore , 452 e 453.
\section{Lasseiro}
\begin{itemize}
\item {Grp. gram.:adj.}
\end{itemize}
Lasso, froixo:«\textunderscore a mim já me aperta e a ti te é lasseiro...\textunderscore »Castilho, \textunderscore Escav. Poét.\textunderscore , 35.
\section{Lassidão}
\begin{itemize}
\item {Grp. gram.:f.}
\end{itemize}
\begin{itemize}
\item {Proveniência:(Do lat. \textunderscore lassitudo\textunderscore )}
\end{itemize}
Qualidade de lasso; cansaço; prostração de fôrças.
Tédio.
\section{Lassitude}
\begin{itemize}
\item {Grp. gram.:f.}
\end{itemize}
O mesmo que \textunderscore lassidão\textunderscore .
\section{Lasso}
\begin{itemize}
\item {Grp. gram.:adj.}
\end{itemize}
\begin{itemize}
\item {Proveniência:(Lat. \textunderscore lassus\textunderscore )}
\end{itemize}
Fatigado, cansado.
Dissoluto.
Enervado.
Gasto.
Bambo.
Froixo, relaxado; laxo.
\section{Lástima}
\begin{itemize}
\item {Grp. gram.:f.}
\end{itemize}
\begin{itemize}
\item {Utilização:Deprec.}
\end{itemize}
Acto ou effeito de lastimar.
Compaixão.
Desgraça.
Aquillo que merece compaixão.
Lamentação.
Coisa ou pessôa inútil, sem préstimo.
\section{Lastimadamente}
\begin{itemize}
\item {Grp. gram.:adv.}
\end{itemize}
\begin{itemize}
\item {Proveniência:(De \textunderscore lastimar\textunderscore )}
\end{itemize}
Com lástima; lastimosamente.
\section{Lastimador}
\begin{itemize}
\item {Grp. gram.:m.  e  adj.}
\end{itemize}
O que lastíma.
\section{Lastimar}
\begin{itemize}
\item {Grp. gram.:v. t.}
\end{itemize}
\begin{itemize}
\item {Proveniência:(Do lat. hyp. \textunderscore blastimare\textunderscore )}
\end{itemize}
Têr pena de.
Compadecer-se de.
Deplorar.
Causar dôr a.
Magoar, offender, ferir. Cf. Filinto, XV, 236.
\section{Lastimável}
\begin{itemize}
\item {Grp. gram.:adj.}
\end{itemize}
Que merece compaixão; que se deve lastimar.
Digno de lástima; deplorável.
\section{Lastimavelmente}
\begin{itemize}
\item {Grp. gram.:adv.}
\end{itemize}
De modo lastimável.
\section{Lastimeiro}
\begin{itemize}
\item {Grp. gram.:adj.}
\end{itemize}
\begin{itemize}
\item {Utilização:Des.}
\end{itemize}
\begin{itemize}
\item {Utilização:Ant.}
\end{itemize}
\begin{itemize}
\item {Proveniência:(De \textunderscore lastimar\textunderscore )}
\end{itemize}
O mesmo que \textunderscore lastimoso\textunderscore .
Bravo, aguerrido.
Que fere, que magôa. Cf. G. Vicente, \textunderscore Inês Pereira\textunderscore .
\section{Lastimosamente}
\begin{itemize}
\item {Grp. gram.:adv.}
\end{itemize}
De modo lastimoso.
\section{Lastimoso}
\begin{itemize}
\item {Grp. gram.:adj.}
\end{itemize}
\begin{itemize}
\item {Proveniência:(De \textunderscore lástima\textunderscore )}
\end{itemize}
Que inspira dó: \textunderscore estado lastimoso\textunderscore .
Que envolve lástima ou lamentação: \textunderscore gritos lastimosos\textunderscore .
Que se lastima.
\section{Lastra}
\begin{itemize}
\item {Grp. gram.:f.}
\end{itemize}
\begin{itemize}
\item {Utilização:Prov.}
\end{itemize}
\begin{itemize}
\item {Utilização:trasm.}
\end{itemize}
Pedra larga, laja.
(Cp. cast. \textunderscore lastra\textunderscore  e \textunderscore lastre\textunderscore , laja)
\section{Lastração}
\begin{itemize}
\item {Grp. gram.:f.}
\end{itemize}
Acto ou effeito de lastrar.
\section{Lastrador}
\begin{itemize}
\item {Grp. gram.:m.  e  adj.}
\end{itemize}
O que lastra.
\section{Lastragem}
\begin{itemize}
\item {Grp. gram.:f.}
\end{itemize}
\begin{itemize}
\item {Utilização:Bras}
\end{itemize}
O mesmo que \textunderscore lastração\textunderscore .
\section{Lastrar}
\begin{itemize}
\item {Grp. gram.:v. t.}
\end{itemize}
\begin{itemize}
\item {Utilização:Fig.}
\end{itemize}
Pôr lastro em.
Deitar ou pôr lastro em (navio).
Tornar firme, aumentando o pêso de.
\section{Lastrear}
\begin{itemize}
\item {Grp. gram.:v. t.}
\end{itemize}
O mesmo que \textunderscore lastrar\textunderscore . Cf. \textunderscore Oriente Conquistado\textunderscore , I, 85.
\section{Lástrico}
\begin{itemize}
\item {Grp. gram.:m.}
\end{itemize}
\begin{itemize}
\item {Proveniência:(It. \textunderscore lástrico\textunderscore )}
\end{itemize}
Espécie do betão, usado na construção de terraços italianos.
\section{Lastro}
\begin{itemize}
\item {Grp. gram.:m.}
\end{itemize}
\begin{itemize}
\item {Utilização:Fig.}
\end{itemize}
\begin{itemize}
\item {Utilização:Fam.}
\end{itemize}
\begin{itemize}
\item {Proveniência:(Do al. \textunderscore last\textunderscore , pêso)}
\end{itemize}
Pêso, feito de pedra, areia, etc., que, posto no porão do navio, faz que êste se equilibre na água.
Areia, que vai na barquinha dos aeróstatos, para se lançar fóra, quando seja conveniente ou preciso que o balão se eleve mais.
Base.
Aquillo sôbre que se acamam ou se collocam outras coisas.
Qualquer comida, com que se prepara o estômago para melhor iguaria ou para bebidas.
\section{Lastroada}
\begin{itemize}
\item {Grp. gram.:f.}
\end{itemize}
\begin{itemize}
\item {Utilização:Prov.}
\end{itemize}
\begin{itemize}
\item {Utilização:minh.}
\end{itemize}
\begin{itemize}
\item {Proveniência:(De \textunderscore lastro\textunderscore )}
\end{itemize}
O mesmo que \textunderscore pedrada\textunderscore .
\section{Lata}
\begin{itemize}
\item {Grp. gram.:f.}
\end{itemize}
\begin{itemize}
\item {Utilização:Chul.}
\end{itemize}
\begin{itemize}
\item {Utilização:Prov.}
\end{itemize}
\begin{itemize}
\item {Utilização:minh.}
\end{itemize}
\begin{itemize}
\item {Utilização:Gír.}
\end{itemize}
Ferro em folha ou batido e estanhado.
Caixa de folha de ferro.
Trave, que, atravessando a nau, sustenta a coberta superior.
Latada.
Cada uma das varas ou canas transversaes da parreira.
Caibro.
Canudo de folha, para guardar papéis ou receber outras substâncias.
Qualquer utensílio de fôlha.
Cara.
Latada, parreira.
Litro.
\section{Lata}
\begin{itemize}
\item {Grp. gram.:m.}
\end{itemize}
\begin{itemize}
\item {Utilização:Prov.}
\end{itemize}
\begin{itemize}
\item {Utilização:alg.}
\end{itemize}
Maçador; indivíduo importuno.
\section{Lata}
\begin{itemize}
\item {Grp. gram.:f.}
\end{itemize}
\begin{itemize}
\item {Utilização:Prov.}
\end{itemize}
\begin{itemize}
\item {Utilização:trasm.}
\end{itemize}
Coirela, belga.
\section{Latada}
\begin{itemize}
\item {Grp. gram.:f.}
\end{itemize}
\begin{itemize}
\item {Utilização:Prov.}
\end{itemize}
\begin{itemize}
\item {Utilização:alent.}
\end{itemize}
\begin{itemize}
\item {Utilização:Gír.}
\end{itemize}
\begin{itemize}
\item {Utilização:T. de Coimbra}
\end{itemize}
\begin{itemize}
\item {Proveniência:(De \textunderscore lata\textunderscore ^1)}
\end{itemize}
Grade de canas ou de varas, para sustentar videiras ou outras plantas trepadeiras.
Parreira.
Quéda.
Bofetada.
Estúrdia e barulho, que os estudantes de Direito fazem de noite, com latas, percorrendo as ruas, para festejar o ponto das aulas.
\section{Latagão}
\begin{itemize}
\item {Grp. gram.:m.}
\end{itemize}
\begin{itemize}
\item {Utilização:Fam.}
\end{itemize}
Homem novo, robusto e encorpado.
\section{Latamente}
\begin{itemize}
\item {Grp. gram.:adv.}
\end{itemize}
De modo lato; em sentido lato; com largueza.
\section{Latana}
\begin{itemize}
\item {Grp. gram.:f.}
\end{itemize}
Palmeira do Brasil.
O mesmo que \textunderscore latânia\textunderscore ?
\section{Latane}
\begin{itemize}
\item {Grp. gram.:f.}
\end{itemize}
Casa de jogo, na China.
\section{Latâneo}
\begin{itemize}
\item {Grp. gram.:adj.}
\end{itemize}
\begin{itemize}
\item {Utilização:Ant.}
\end{itemize}
Que está ao lado; lateral.
(Refl. do lat. \textunderscore latus\textunderscore )
\section{Latânia}
\begin{itemize}
\item {Grp. gram.:f.}
\end{itemize}
Gênero de palmeiras, (\textunderscore latania rubra\textunderscore , Jacq.).
\section{Latão}
\begin{itemize}
\item {Grp. gram.:m.}
\end{itemize}
\begin{itemize}
\item {Proveniência:(De \textunderscore lata\textunderscore )}
\end{itemize}
Liga de cobre e zinco.
\section{Late}
\begin{itemize}
\item {Grp. gram.:m.}
\end{itemize}
\begin{itemize}
\item {Proveniência:(T. asiat.)}
\end{itemize}
O mesmo que \textunderscore cegonha\textunderscore .
\section{Latear}
\begin{itemize}
\item {Grp. gram.:v. t.}
\end{itemize}
Enfeitar ou guarnecer com lata ou latão.
\section{Lategaço}
\begin{itemize}
\item {Grp. gram.:m.}
\end{itemize}
O mesmo que \textunderscore lategada\textunderscore . Cf. Latino \textunderscore Hist. de Pol. e Mil.\textunderscore , I, 165.
\section{Lategada}
\begin{itemize}
\item {Grp. gram.:f.}
\end{itemize}
Pancada ou açoite com látego.
\section{Látego}
\begin{itemize}
\item {Grp. gram.:m.}
\end{itemize}
\begin{itemize}
\item {Utilização:Bras. do S}
\end{itemize}
\begin{itemize}
\item {Utilização:Fig.}
\end{itemize}
Azorrague, chicote de cordas ou de correias.
Inquerideira.
Tira de coiro cru, com que se apertam os arreios e que faz parte da cincha.
Castigo; flagello.
Estímulo.
(Cast. \textunderscore látigo\textunderscore )
\section{Latejante}
\begin{itemize}
\item {Grp. gram.:adj.}
\end{itemize}
Que lateja.
\section{Latejar}
\begin{itemize}
\item {Grp. gram.:v. i.}
\end{itemize}
Arquejar.
Palpitar; pulsar: \textunderscore latejam-lhe as fontes\textunderscore .
\section{Latejo}
\begin{itemize}
\item {Grp. gram.:m.}
\end{itemize}
Acto ou effeito de latejar.
\section{Latente}
\begin{itemize}
\item {Grp. gram.:adj.}
\end{itemize}
\begin{itemize}
\item {Utilização:Gram.}
\end{itemize}
\begin{itemize}
\item {Utilização:Ext.}
\end{itemize}
\begin{itemize}
\item {Proveniência:(Lat. \textunderscore latens\textunderscore )}
\end{itemize}
Que se não vê, que está occulto.
Que se subentende: \textunderscore concordância latente\textunderscore .
Dissimulado.
\section{Later}
\begin{itemize}
\item {Grp. gram.:v. i.}
\end{itemize}
\begin{itemize}
\item {Utilização:Ant.}
\end{itemize}
\begin{itemize}
\item {Proveniência:(Lat. \textunderscore latere\textunderscore )}
\end{itemize}
Estar occulto. Cf. F. Manuel, \textunderscore Carta de Guia\textunderscore , XXIX.
\section{Later}
\begin{itemize}
\item {Grp. gram.:v. i.}
\end{itemize}
O mesmo que \textunderscore latejar\textunderscore .--É verbo defectivo. Só vejo empregada a 3.^a pess. do sing. do pres. do indic., como poderia empregar-se a 3.^a do pl.«Em vão lhe \textunderscore late\textunderscore  o peito de insoffrido.»Filinto. VII, 248. Cf. Garrett, \textunderscore Flor. sem Fruto\textunderscore , 198.
\section{Lateral}
\begin{itemize}
\item {Grp. gram.:adj.}
\end{itemize}
\begin{itemize}
\item {Proveniência:(Lat. \textunderscore lateralis\textunderscore )}
\end{itemize}
Relativo a lado.
Transversal; que está ao lado.
\section{Lateralidade}
\begin{itemize}
\item {Grp. gram.:f.}
\end{itemize}
Qualidade de lateral.
\section{Lateralmente}
\begin{itemize}
\item {Grp. gram.:adv.}
\end{itemize}
De modo lateral; ao lado.
\section{Lateranense}
\begin{itemize}
\item {Grp. gram.:adv.}
\end{itemize}
\begin{itemize}
\item {Proveniência:(Do lat. \textunderscore lateranus\textunderscore )}
\end{itemize}
Relativo ao palácio pontifício de Latrão.
Diz-se especialmente de um concílio, que se celebrou em Latrão.
\section{Laterário}
\begin{itemize}
\item {Grp. gram.:adj.}
\end{itemize}
\begin{itemize}
\item {Utilização:Des.}
\end{itemize}
\begin{itemize}
\item {Proveniência:(Lat. \textunderscore laterarius\textunderscore )}
\end{itemize}
Relativo a tejolo.
Que serve para fazer tejolo.
\section{Làteriflexão}
\begin{itemize}
\item {fónica:csão}
\end{itemize}
\begin{itemize}
\item {Grp. gram.:f.}
\end{itemize}
\begin{itemize}
\item {Utilização:Med.}
\end{itemize}
\begin{itemize}
\item {Proveniência:(Do lat. \textunderscore latus\textunderscore  + \textunderscore flexio\textunderscore )}
\end{itemize}
Flexão lateral do tero.
\section{Làterifólio}
\begin{itemize}
\item {Grp. gram.:adj.}
\end{itemize}
\begin{itemize}
\item {Utilização:Bot.}
\end{itemize}
\begin{itemize}
\item {Proveniência:(Do lat. \textunderscore latus\textunderscore , \textunderscore lateris\textunderscore  + \textunderscore folium\textunderscore )}
\end{itemize}
Que nasce ao lado das folhas.
\section{Làterígrado}
\begin{itemize}
\item {Grp. gram.:adj.}
\end{itemize}
\begin{itemize}
\item {Utilização:Zool.}
\end{itemize}
\begin{itemize}
\item {Proveniência:(Do lat. \textunderscore latus\textunderscore , \textunderscore lateris\textunderscore  + \textunderscore gradi\textunderscore )}
\end{itemize}
Diz-se de certas aranhas, que andam sôbre o lado, como andam para trás e para deante.
\section{Làterinérveo}
\begin{itemize}
\item {Grp. gram.:adj.}
\end{itemize}
\begin{itemize}
\item {Utilização:Bot.}
\end{itemize}
\begin{itemize}
\item {Proveniência:(Do lat. \textunderscore latus\textunderscore  + \textunderscore nervum\textunderscore )}
\end{itemize}
Diz-se da fôlha, cujas nervuras partem da nervura média para a margem.
\section{Laterita}
\begin{itemize}
\item {Grp. gram.:f.}
\end{itemize}
\begin{itemize}
\item {Proveniência:(Do lat. \textunderscore later\textunderscore )}
\end{itemize}
Mineral da África meridional.
\section{Laterite}
\begin{itemize}
\item {Grp. gram.:f.}
\end{itemize}
\begin{itemize}
\item {Proveniência:(Do lat. \textunderscore later\textunderscore )}
\end{itemize}
Mineral da África meridional.
\section{Làteriversão}
\begin{itemize}
\item {Grp. gram.:f.}
\end{itemize}
\begin{itemize}
\item {Utilização:Med.}
\end{itemize}
\begin{itemize}
\item {Proveniência:(Do lat. \textunderscore latus\textunderscore , \textunderscore lateris\textunderscore  + \textunderscore versio\textunderscore )}
\end{itemize}
Reviramento do útero sôbre o lado.
\section{Látex}
\begin{itemize}
\item {Grp. gram.:m.}
\end{itemize}
\begin{itemize}
\item {Proveniência:(Lat. \textunderscore latex\textunderscore )}
\end{itemize}
Suco leitoso, que escorre de certas plantas, quando nellas se faz uma íncisão.
\section{Lathyrismo}
\begin{itemize}
\item {Grp. gram.:m.}
\end{itemize}
Entoxicação, produzida pelo lathyro.
\section{Lathyro}
\begin{itemize}
\item {Grp. gram.:m.}
\end{itemize}
\begin{itemize}
\item {Utilização:Bot.}
\end{itemize}
\begin{itemize}
\item {Proveniência:(Gr. \textunderscore lathuros\textunderscore )}
\end{itemize}
Designação scientífica do cizirão.
\section{Latíbulo}
\begin{itemize}
\item {Grp. gram.:m.}
\end{itemize}
\begin{itemize}
\item {Proveniência:(Lat. \textunderscore latibulum\textunderscore )}
\end{itemize}
O mesmo que \textunderscore esconderijo\textunderscore .
Céu, morada dos deuses. Cf. J. A. Macedo, \textunderscore Oriente\textunderscore , I, 15.
\section{Laticífero}
\begin{itemize}
\item {Grp. gram.:adj.}
\end{itemize}
\begin{itemize}
\item {Proveniência:(Do lat. \textunderscore latex\textunderscore , \textunderscore laticis\textunderscore  + \textunderscore ferre\textunderscore )}
\end{itemize}
Que derrama látex.
\section{Laticlávio}
\begin{itemize}
\item {Grp. gram.:m.}
\end{itemize}
\begin{itemize}
\item {Proveniência:(Lat. \textunderscore laticlavius\textunderscore )}
\end{itemize}
Aquelle que usava o laticlavo.
\section{Laticlavo}
\begin{itemize}
\item {Grp. gram.:m.}
\end{itemize}
\begin{itemize}
\item {Proveniência:(Lat. \textunderscore laticlavus\textunderscore )}
\end{itemize}
Vestido de púrpura, com guarnições em fórma de cabeças de cravo, e usado pelos senadores romanos.
\section{Laticollo}
\begin{itemize}
\item {Grp. gram.:adj.}
\end{itemize}
\begin{itemize}
\item {Proveniência:(Do lat. \textunderscore latus\textunderscore  + \textunderscore collum\textunderscore )}
\end{itemize}
Que tem pescoço largo.
\section{Laticolo}
\begin{itemize}
\item {Grp. gram.:adj.}
\end{itemize}
\begin{itemize}
\item {Proveniência:(Do lat. \textunderscore latus\textunderscore  + \textunderscore collum\textunderscore )}
\end{itemize}
Que tem pescoço largo.
\section{Laticórneo}
\begin{itemize}
\item {Grp. gram.:adj.}
\end{itemize}
\begin{itemize}
\item {Proveniência:(Do lat. \textunderscore latus\textunderscore  + \textunderscore cornu\textunderscore )}
\end{itemize}
Que tem cornos ou antennas largas.
\section{Latido}
\begin{itemize}
\item {Grp. gram.:m.}
\end{itemize}
\begin{itemize}
\item {Utilização:Fig.}
\end{itemize}
\begin{itemize}
\item {Utilização:Chul.}
\end{itemize}
Acto ou effeito de latir.
Remorso.
Palavras vans.
\section{Latifloro}
\begin{itemize}
\item {Grp. gram.:adj.}
\end{itemize}
\begin{itemize}
\item {Proveniência:(Do lat. \textunderscore latus\textunderscore  + \textunderscore flos\textunderscore )}
\end{itemize}
Que tem flôres largas.
\section{Latifoliado}
\begin{itemize}
\item {Grp. gram.:adj.}
\end{itemize}
\begin{itemize}
\item {Utilização:Bot.}
\end{itemize}
O mesmo que \textunderscore latifólio\textunderscore .
\section{Latifólio}
\begin{itemize}
\item {Grp. gram.:adj.}
\end{itemize}
\begin{itemize}
\item {Proveniência:(Do lat. \textunderscore latus\textunderscore  + \textunderscore folium\textunderscore )}
\end{itemize}
Que tem fôlhas largas.
\section{Latifundiado}
\begin{itemize}
\item {Grp. gram.:adj.}
\end{itemize}
Relativo a latifúndio.
\section{Latifúndio}
\begin{itemize}
\item {Grp. gram.:m.}
\end{itemize}
\begin{itemize}
\item {Proveniência:(Do lat. \textunderscore latus\textunderscore  + \textunderscore fundus\textunderscore )}
\end{itemize}
Grande propriedade rural.
\section{Latilabro}
\begin{itemize}
\item {Grp. gram.:adj.}
\end{itemize}
\begin{itemize}
\item {Utilização:Zool.}
\end{itemize}
\begin{itemize}
\item {Proveniência:(Do lat. \textunderscore latus\textunderscore  + \textunderscore labrum\textunderscore )}
\end{itemize}
Que tem lábios grossos.
\section{Latim}
\begin{itemize}
\item {Grp. gram.:m.}
\end{itemize}
\begin{itemize}
\item {Utilização:Fam.}
\end{itemize}
\begin{itemize}
\item {Utilização:Gír.}
\end{itemize}
\begin{itemize}
\item {Proveniência:(Lat. \textunderscore latinus\textunderscore )}
\end{itemize}
Língua do Lácio, língua latina, língua falada pelo antigo povo romano.
Coisa diffícil de comprehender.
A linguagem dos ladrões.
\section{Latímano}
\begin{itemize}
\item {Grp. gram.:adj.}
\end{itemize}
\begin{itemize}
\item {Proveniência:(Do lat. \textunderscore latus\textunderscore  + \textunderscore manus\textunderscore )}
\end{itemize}
Que tem mãos largas.
\section{Latina}
\begin{itemize}
\item {Grp. gram.:f.}
\end{itemize}
\begin{itemize}
\item {Utilização:Náut.}
\end{itemize}
\begin{itemize}
\item {Proveniência:(De \textunderscore latino\textunderscore )}
\end{itemize}
Vela, de fórma triangular.
\section{Latinada}
\begin{itemize}
\item {Grp. gram.:f.}
\end{itemize}
\begin{itemize}
\item {Proveniência:(De \textunderscore latino\textunderscore )}
\end{itemize}
Discurso em latim.
Êrro contra as regras do latim.
Pronúncia defeituosa ou errada de palavras latinas.
\section{Latinamente}
\begin{itemize}
\item {Grp. gram.:adv.}
\end{itemize}
\begin{itemize}
\item {Proveniência:(De \textunderscore latino\textunderscore )}
\end{itemize}
Segundo as regras do latim.
Segundo o costume dos Latinos.
\section{Latinar}
\begin{itemize}
\item {Grp. gram.:v. i.}
\end{itemize}
\begin{itemize}
\item {Proveniência:(Lat. \textunderscore latinare\textunderscore )}
\end{itemize}
Fazer traducções do latim; falar ou escrever latim.
\section{Latinaria}
\begin{itemize}
\item {Grp. gram.:f.}
\end{itemize}
Conhecimento de latim:«\textunderscore atiremos todas as latinarias ao diabo.\textunderscore »Castilho, \textunderscore Escav. Poét.\textunderscore , 71.
\section{Latinas}
\begin{itemize}
\item {Grp. gram.:f. pl.}
\end{itemize}
\begin{itemize}
\item {Proveniência:(Lat. \textunderscore latinae\textunderscore )}
\end{itemize}
Festas, que se celebravam em honra de Júpiter no Lácio, geralmente sôbre o monte Albano.
\section{Latingar}
\begin{itemize}
\item {Grp. gram.:v. i.}
\end{itemize}
\begin{itemize}
\item {Utilização:Gír.}
\end{itemize}
Comer.
\section{Latinice}
\begin{itemize}
\item {Grp. gram.:f.}
\end{itemize}
\begin{itemize}
\item {Proveniência:(De \textunderscore latino\textunderscore )}
\end{itemize}
Presumpção de saber latim. Cf. Filinto, VI, 210.
\section{Latinidade}
\begin{itemize}
\item {Grp. gram.:f.}
\end{itemize}
\begin{itemize}
\item {Proveniência:(Lat. \textunderscore latinitas\textunderscore )}
\end{itemize}
Língua latina.
Maneira de falar ou de escrever latim.
Estudo dos principaes e mais diffíceis escritores latinos.
Rigorosa construcção grammatical de uma composição latina.
\section{Latiniparla}
\begin{itemize}
\item {Grp. gram.:m.  e  f.}
\end{itemize}
\begin{itemize}
\item {Utilização:Deprec.}
\end{itemize}
\begin{itemize}
\item {Proveniência:(De \textunderscore latino\textunderscore  + \textunderscore palrar\textunderscore )}
\end{itemize}
Pessôa, que faz alarde de saber latim. Cf. Filinto, V, 17.
\section{Latinismo}
\begin{itemize}
\item {Grp. gram.:m.}
\end{itemize}
\begin{itemize}
\item {Proveniência:(De \textunderscore latino\textunderscore )}
\end{itemize}
Locução própria da língua latina.
Construcção grammatical, própria do latim.
\section{Latinista}
\begin{itemize}
\item {Grp. gram.:m.}
\end{itemize}
\begin{itemize}
\item {Proveniência:(De \textunderscore latino\textunderscore )}
\end{itemize}
Aquelle que é versado no latim.
\section{Latinização}
\begin{itemize}
\item {Grp. gram.:f.}
\end{itemize}
Acto ou effeito de latinizar.
\section{Latinizante}
\begin{itemize}
\item {Grp. gram.:adj.}
\end{itemize}
\begin{itemize}
\item {Proveniência:(Lat. \textunderscore latinizans\textunderscore )}
\end{itemize}
Que latiniza.
Sectário do rito latino, em país de herejes ou scismáticos.
\section{Latinizar}
\begin{itemize}
\item {Grp. gram.:v. t.}
\end{itemize}
\begin{itemize}
\item {Grp. gram.:V. i.}
\end{itemize}
\begin{itemize}
\item {Proveniência:(Lat. \textunderscore latinizare\textunderscore )}
\end{itemize}
Dar fórma latina a.
Alatinar.
Usar de expressões latinas; falar latim.
\section{Latino}
\begin{itemize}
\item {Grp. gram.:adj.}
\end{itemize}
\begin{itemize}
\item {Grp. gram.:M.}
\end{itemize}
\begin{itemize}
\item {Proveniência:(Lat. \textunderscore latinus\textunderscore )}
\end{itemize}
Relativo ao latim: \textunderscore grammática latina\textunderscore .
Dito ou escrito em latim.
Relativo aos povos latinos: \textunderscore o mundo latino\textunderscore .
Concernente á Igreja christan do Occidente.
Que diz respeito aos povos procedentes dos Romanos, ou encorporados no Império romano.
Que tem velas latinas, (falando-se de navios).
E diz-se de uma vela náutica de fórma triangular.
Habitante do Lácio.
Aquelle que descende dos Latinos ou dos Romanos antigos.
Aquelle que é versado em latim; latinista.
Vela latina.
\section{Latino-americano}
\begin{itemize}
\item {Grp. gram.:m.}
\end{itemize}
O mesmo que \textunderscore ibero-americano\textunderscore .
\section{Latinório}
\begin{itemize}
\item {Grp. gram.:m.}
\end{itemize}
\begin{itemize}
\item {Utilização:Fam.}
\end{itemize}
\begin{itemize}
\item {Grp. gram.:Adj.}
\end{itemize}
\begin{itemize}
\item {Proveniência:(Do rad. de \textunderscore latim\textunderscore )}
\end{itemize}
Mau latim.
Trecho latino, mal traduzido ou mal applicado.
Escrito em mau latim. Cf. Filinto, XI, 193.
\section{Latípede}
\begin{itemize}
\item {Grp. gram.:adj.}
\end{itemize}
\begin{itemize}
\item {Utilização:Zool.}
\end{itemize}
\begin{itemize}
\item {Proveniência:(Do lat. \textunderscore latus\textunderscore  + \textunderscore pes\textunderscore , \textunderscore pedis\textunderscore )}
\end{itemize}
Que tem pés largos.
\section{Latipene}
\begin{itemize}
\item {Grp. gram.:adj.}
\end{itemize}
\begin{itemize}
\item {Utilização:Zool.}
\end{itemize}
\begin{itemize}
\item {Proveniência:(Do lat. \textunderscore latus\textunderscore  + \textunderscore penna\textunderscore )}
\end{itemize}
Que tem penas largas.
\section{Latipenne}
\begin{itemize}
\item {Grp. gram.:adj.}
\end{itemize}
\begin{itemize}
\item {Utilização:Zool.}
\end{itemize}
\begin{itemize}
\item {Proveniência:(Do lat. \textunderscore latus\textunderscore  + \textunderscore penna\textunderscore )}
\end{itemize}
Que tem pennas largas.
\section{Latir}
\begin{itemize}
\item {Grp. gram.:v. i.}
\end{itemize}
\begin{itemize}
\item {Utilização:Fig.}
\end{itemize}
O mesmo que \textunderscore ladrar\textunderscore ^1.
Gannir.
Gritar.
(Cp. lat. \textunderscore latrare\textunderscore )
\section{Latirismo}
\begin{itemize}
\item {Grp. gram.:m.}
\end{itemize}
Entoxicação, produzida pelo latiro.
\section{Latiro}
\begin{itemize}
\item {Grp. gram.:m.}
\end{itemize}
\begin{itemize}
\item {Utilização:Bot.}
\end{itemize}
\begin{itemize}
\item {Proveniência:(Gr. \textunderscore lathuros\textunderscore )}
\end{itemize}
Designação científica do cizirão.
\section{Latirostro}
\begin{itemize}
\item {fónica:ros}
\end{itemize}
\begin{itemize}
\item {Grp. gram.:adj.}
\end{itemize}
\begin{itemize}
\item {Utilização:Zool.}
\end{itemize}
\begin{itemize}
\item {Grp. gram.:M. pl.}
\end{itemize}
\begin{itemize}
\item {Proveniência:(Do lat. \textunderscore latus\textunderscore  + \textunderscore rostrum\textunderscore )}
\end{itemize}
Que tem bico largo e chato.
Família de aves, caracterizadas por um bico muito achatado.
\section{Latirrostro}
\begin{itemize}
\item {Grp. gram.:adj.}
\end{itemize}
\begin{itemize}
\item {Utilização:Zool.}
\end{itemize}
\begin{itemize}
\item {Grp. gram.:M. pl.}
\end{itemize}
\begin{itemize}
\item {Proveniência:(Do lat. \textunderscore latus\textunderscore  + \textunderscore rostrum\textunderscore )}
\end{itemize}
Que tem bico largo e chato.
Família de aves, caracterizadas por um bico muito achatado.
\section{Latitude}
\begin{itemize}
\item {Grp. gram.:f.}
\end{itemize}
\begin{itemize}
\item {Utilização:Ext.}
\end{itemize}
\begin{itemize}
\item {Utilização:Fig.}
\end{itemize}
\begin{itemize}
\item {Proveniência:(Lat. \textunderscore latitudo\textunderscore )}
\end{itemize}
Arco do meridiano, comprehendido entre o Equador e a vertical de qualquer lugar com o plano do Equador.
Distância, que vai do Equador a um lugar do globo, medida em graus sôbre o meridiano dêsse lugar.
Ângulo, formado pela vertical de um lugar parallelo á ecliptica e á linha recta que passa por um astro e por um centro dado naquelle plano.
Clima.
Amplitude.
Desenvolvimento; extensão.
\section{Latitudinário}
\begin{itemize}
\item {Grp. gram.:adj.}
\end{itemize}
\begin{itemize}
\item {Proveniência:(Do lat. \textunderscore latitudo\textunderscore )}
\end{itemize}
Amplo; amplificado; extensivo.
Arbitrário na interpretação.
\section{Latitudinarismo}
\begin{itemize}
\item {Grp. gram.:m.}
\end{itemize}
\begin{itemize}
\item {Proveniência:(De \textunderscore latitudinário\textunderscore )}
\end{itemize}
Seita inglesa, que sustenta a máxima tolerância religiosa e a doutrina do que todos os homens se salvarão.
\section{Latitudinarista}
\begin{itemize}
\item {Grp. gram.:m.}
\end{itemize}
Partidário do latitudinarismo.
\section{Lato}
\begin{itemize}
\item {Grp. gram.:adj.}
\end{itemize}
\begin{itemize}
\item {Grp. gram.:M.}
\end{itemize}
\begin{itemize}
\item {Utilização:Prov.}
\end{itemize}
\begin{itemize}
\item {Utilização:minh.}
\end{itemize}
\begin{itemize}
\item {Proveniência:(Lat. \textunderscore latus\textunderscore )}
\end{itemize}
Amplo; dilatado.
Extensivo: \textunderscore em sentido lato\textunderscore .
Vara comprida. (Colhido em Barcelos)
\section{Lato}
\begin{itemize}
\item {Grp. gram.:m.}
\end{itemize}
\begin{itemize}
\item {Utilização:Prov.}
\end{itemize}
\begin{itemize}
\item {Utilização:alg.}
\end{itemize}
Baraço curto de pita.
(Contr. de \textunderscore látego\textunderscore )
\section{Latoaria}
\begin{itemize}
\item {Grp. gram.:f.}
\end{itemize}
\begin{itemize}
\item {Proveniência:(De \textunderscore latão\textunderscore )}
\end{itemize}
Offício ou officina de latoeiro.
\section{Latoeiro}
\begin{itemize}
\item {Grp. gram.:m.}
\end{itemize}
\begin{itemize}
\item {Proveniência:(De \textunderscore latão\textunderscore )}
\end{itemize}
Aquelle que trabalha em lata ou latão; funileiro.
\section{Latrante}
\begin{itemize}
\item {Grp. gram.:adj.}
\end{itemize}
\begin{itemize}
\item {Utilização:Poét.}
\end{itemize}
\begin{itemize}
\item {Proveniência:(Lat. \textunderscore latrans\textunderscore )}
\end{itemize}
O mesmo que \textunderscore ladrante\textunderscore .
\section{Latria}
\begin{itemize}
\item {Grp. gram.:f.}
\end{itemize}
\begin{itemize}
\item {Utilização:Fig.}
\end{itemize}
\begin{itemize}
\item {Proveniência:(Do gr. \textunderscore latreia\textunderscore )}
\end{itemize}
Adoração devida a Deus.
Adoração.
\section{Latrina}
\begin{itemize}
\item {Grp. gram.:f.}
\end{itemize}
\begin{itemize}
\item {Proveniência:(Lat. \textunderscore latrina\textunderscore )}
\end{itemize}
Lugar para dejecções; cloaca.
\section{Latrinário}
\begin{itemize}
\item {Grp. gram.:adj.}
\end{itemize}
\begin{itemize}
\item {Utilização:Ext.}
\end{itemize}
Relativo á latrina.
Que vive nas latrinas.
Immundo; repugnante: \textunderscore publicações latrinárias\textunderscore .
\section{Latrineiro}
\begin{itemize}
\item {Grp. gram.:m.}
\end{itemize}
Aquelle que guarda ou limpa latrinas.
\section{Latrocinar}
\begin{itemize}
\item {Grp. gram.:v. t.}
\end{itemize}
\begin{itemize}
\item {Proveniência:(Lat. \textunderscore latrocinari\textunderscore )}
\end{itemize}
Roubar violentamente.
Commeter latrocínio contra. Cf. D. Ant. da Costa, \textunderscore Três Mundos\textunderscore , 223.
\section{Latrocínio}
\begin{itemize}
\item {Grp. gram.:m.}
\end{itemize}
\begin{itemize}
\item {Proveniência:(Lat. \textunderscore latrocinium\textunderscore )}
\end{itemize}
Roubo violento, á mão armada.
\section{Lauda}
\begin{itemize}
\item {Grp. gram.:f.}
\end{itemize}
\begin{itemize}
\item {Proveniência:(Do lat. \textunderscore laus\textunderscore , \textunderscore laudis\textunderscore )}
\end{itemize}
Página de livro, escrita ou em branco.
\section{Laudabilidade}
\begin{itemize}
\item {Grp. gram.:f.}
\end{itemize}
\begin{itemize}
\item {Proveniência:(Do lat. \textunderscore laudabilis\textunderscore )}
\end{itemize}
Qualidade daquillo que é digno de louvor. Cf. \textunderscore Luz e Calor\textunderscore , 51 e 564.
\section{Laudânico}
\begin{itemize}
\item {Grp. gram.:adj.}
\end{itemize}
\begin{itemize}
\item {Proveniência:(De \textunderscore láudano\textunderscore )}
\end{itemize}
Soporífico, narcótico.
\section{Laudanizar}
\begin{itemize}
\item {Grp. gram.:v. t.}
\end{itemize}
Preparar com láudano.
\section{Láudano}
\begin{itemize}
\item {Grp. gram.:m.}
\end{itemize}
Medicamento, em que o ópio está ligado com outros ingredientes.
(Talvez da mesma or. que \textunderscore ládano\textunderscore  e \textunderscore lábdano\textunderscore )
\section{Laudatício}
\begin{itemize}
\item {Grp. gram.:adj.}
\end{itemize}
\begin{itemize}
\item {Proveniência:(Lat. \textunderscore laudaticius\textunderscore )}
\end{itemize}
O mesmo que \textunderscore laudatório\textunderscore .
\section{Laudativamente}
\begin{itemize}
\item {Grp. gram.:adv.}
\end{itemize}
De modo laudativo.
\section{Laudativo}
\begin{itemize}
\item {Grp. gram.:adj.}
\end{itemize}
\begin{itemize}
\item {Proveniência:(Lat. \textunderscore laudativus\textunderscore )}
\end{itemize}
O mesmo que \textunderscore laudatório\textunderscore .
\section{Laudatório}
\begin{itemize}
\item {Grp. gram.:adj.}
\end{itemize}
\begin{itemize}
\item {Proveniência:(Lat. \textunderscore laudatorius\textunderscore )}
\end{itemize}
Relativo a louvor.
Que contém louvor: \textunderscore palavras laudatórias\textunderscore .
Que louva.
\section{Laudável}
\begin{itemize}
\item {Grp. gram.:adj.}
\end{itemize}
\begin{itemize}
\item {Proveniência:(Do lat. \textunderscore laudabilis\textunderscore )}
\end{itemize}
Que se deve louvar; louvável.
\section{Laudavelmente}
\begin{itemize}
\item {Grp. gram.:adv.}
\end{itemize}
De modo laudável.
\section{Laúde}
\begin{itemize}
\item {Grp. gram.:m.}
\end{itemize}
\begin{itemize}
\item {Utilização:Ant.}
\end{itemize}
O mesmo que \textunderscore alaúde\textunderscore ^1.
\section{Laúde}
\begin{itemize}
\item {Grp. gram.:m.}
\end{itemize}
Espécie de embarcação usada na Pesca do atum. Cf. Ortigão, \textunderscore Culto da Arte\textunderscore .
(Cast. \textunderscore laud\textunderscore )
\section{Laude}
\begin{itemize}
\item {Grp. gram.:f.}
\end{itemize}
\begin{itemize}
\item {Utilização:Ant.}
\end{itemize}
\begin{itemize}
\item {Proveniência:(Do lat. \textunderscore laus\textunderscore , \textunderscore laudis\textunderscore )}
\end{itemize}
Canto de louvor, lôa.
\section{Laudel}
\begin{itemize}
\item {Grp. gram.:m.}
\end{itemize}
\begin{itemize}
\item {Proveniência:(Do lat. \textunderscore lodix\textunderscore ?)}
\end{itemize}
Vestidura antiga de coiro, ou acolchoada com lâminas metállicas, para preservar dos golpes de espada.
\section{Laudêmio}
\begin{itemize}
\item {Grp. gram.:m.}
\end{itemize}
\begin{itemize}
\item {Proveniência:(Do rad. do lat. \textunderscore laudare\textunderscore ?)}
\end{itemize}
Pensão que, segundo a constituição dos prazos antigos, é paga aos senhorios directos, quando o emphyteuta aliena o prédio respectivo.
\section{Laudes}
\begin{itemize}
\item {Grp. gram.:m.}
\end{itemize}
\begin{itemize}
\item {Proveniência:(Do lat. \textunderscore laus\textunderscore , \textunderscore laudis\textunderscore )}
\end{itemize}
Segunda parte das horas canónicas.
\section{Laudéus}
\begin{itemize}
\item {Grp. gram.:m.}
\end{itemize}
\begin{itemize}
\item {Utilização:Bras}
\end{itemize}
Tríbo de aborigenes, que dominou em Mato-Grosso.
\section{Laudo}
\begin{itemize}
\item {Grp. gram.:m.}
\end{itemize}
\begin{itemize}
\item {Proveniência:(Do lat. \textunderscore laudare\textunderscore )}
\end{itemize}
Parecer do louvado ou de um árbitro.
\section{Laulé}
\begin{itemize}
\item {Grp. gram.:f.}
\end{itemize}
\begin{itemize}
\item {Utilização:Ant.}
\end{itemize}
Pequena embarcação de remo, na Ásia. Cf. \textunderscore Peregrinação\textunderscore , CXVI.
\section{Laúnea}
\begin{itemize}
\item {Grp. gram.:f.}
\end{itemize}
Gênero de plantas chicoriáceas.
\section{Laura}
\begin{itemize}
\item {Grp. gram.:f.}
\end{itemize}
\begin{itemize}
\item {Utilização:Ant.}
\end{itemize}
Cada uma das cellas ou covas, que vários anachoretas occupavam no mesmo ermo. Cf. F. Alex. Lobo, III, 193.
(R. lat. \textunderscore laura\textunderscore )
\section{Lauráceas}
\begin{itemize}
\item {Grp. gram.:f. pl.}
\end{itemize}
O mesmo ou melhor que \textunderscore lauríneas\textunderscore .
\section{Laurbanense}
\begin{itemize}
\item {Grp. gram.:adj.}
\end{itemize}
Relativo a Lorvão ou ao convento de Lorvão. Cf. Herculano, \textunderscore Quest. Públ.\textunderscore , I, 200.
\section{Láurea}
\begin{itemize}
\item {Grp. gram.:f.}
\end{itemize}
\begin{itemize}
\item {Utilização:Ant.}
\end{itemize}
\begin{itemize}
\item {Utilização:Ext.}
\end{itemize}
\begin{itemize}
\item {Proveniência:(Lat. \textunderscore laurea\textunderscore )}
\end{itemize}
Corôa de loiros.
Laurel.
Galardão, prêmio.
\section{Laurear}
\begin{itemize}
\item {Grp. gram.:v. t.}
\end{itemize}
(V.larear)
\section{Laurear}
\begin{itemize}
\item {Grp. gram.:v. t.}
\end{itemize}
\begin{itemize}
\item {Utilização:Fig.}
\end{itemize}
\begin{itemize}
\item {Proveniência:(Lat. \textunderscore laureare\textunderscore )}
\end{itemize}
Coroar de loiros.
Galardoar.
Enfeitar.
Festejar.
\section{Laureio}
\begin{itemize}
\item {Grp. gram.:m.}
\end{itemize}
\begin{itemize}
\item {Utilização:Prov.}
\end{itemize}
\begin{itemize}
\item {Utilização:trasm.}
\end{itemize}
Acto de laurear^1.
\section{Laurel}
\begin{itemize}
\item {Grp. gram.:m.}
\end{itemize}
\begin{itemize}
\item {Utilização:Fig.}
\end{itemize}
\begin{itemize}
\item {Proveniência:(Do lat. \textunderscore laureola\textunderscore )}
\end{itemize}
Corôa de loiros; láurea.
Galardão: prêmio.
Homenagens.
\section{Laurência}
\begin{itemize}
\item {Grp. gram.:f.}
\end{itemize}
\begin{itemize}
\item {Proveniência:(De \textunderscore Laurenti\textunderscore , n. p.)}
\end{itemize}
Gênero do plantas lobeliáceas.
\section{Laurêncio}
\begin{itemize}
\item {Grp. gram.:adj.}
\end{itemize}
\begin{itemize}
\item {Proveniência:(Lat. \textunderscore laurentius\textunderscore )}
\end{itemize}
Relativo a Laurento, antiga cidade do Lácio. Cf. Castilho, \textunderscore Fastos\textunderscore , I, 101.
\section{Laurentiano}
\begin{itemize}
\item {Grp. gram.:adj.}
\end{itemize}
O mesmo que \textunderscore laurentino\textunderscore ^1.
\section{Laurentim}
\begin{itemize}
\item {Grp. gram.:m.}
\end{itemize}
Variedade de árvores lauráceas. Cf. Júlio Dinis, \textunderscore Fidalgos\textunderscore , II, 78.
\section{Laurentino}
\begin{itemize}
\item {Grp. gram.:adj.}
\end{itemize}
\begin{itemize}
\item {Utilização:Geol.}
\end{itemize}
\begin{itemize}
\item {Proveniência:(Do b. lat. \textunderscore Laurentius\textunderscore )}
\end{itemize}
Diz-se de uma das secçoes do terreno archaico do Canadá.
\section{Laurentino}
\begin{itemize}
\item {Grp. gram.:adj.}
\end{itemize}
O mesmo que \textunderscore láureo\textunderscore .
\section{Láureo}
\begin{itemize}
\item {Grp. gram.:adj.}
\end{itemize}
\begin{itemize}
\item {Proveniência:(Lat. \textunderscore laureus\textunderscore )}
\end{itemize}
Relativo a loiros; feito de loiros.
\section{Lauréola}
\begin{itemize}
\item {Grp. gram.:f.}
\end{itemize}
\begin{itemize}
\item {Proveniência:(Lat. \textunderscore laureola\textunderscore )}
\end{itemize}
Laurel; auréola.
Nome de algumas plantas.
\section{Láurico}
\begin{itemize}
\item {Grp. gram.:adj.}
\end{itemize}
\begin{itemize}
\item {Proveniência:(Do lat. \textunderscore laurus\textunderscore )}
\end{itemize}
Diz-se de um ácido, contido nas bagas de loireiro.
\section{Laurícomo}
\begin{itemize}
\item {Grp. gram.:adj.}
\end{itemize}
\begin{itemize}
\item {Utilização:Poét.}
\end{itemize}
\begin{itemize}
\item {Proveniência:(Lat. \textunderscore lauricomus\textunderscore )}
\end{itemize}
Coroado de loiros.
\section{Laurífero}
\begin{itemize}
\item {Grp. gram.:adj.}
\end{itemize}
\begin{itemize}
\item {Proveniência:(Lat. \textunderscore laurifer\textunderscore )}
\end{itemize}
Coroado de loiros; que tem loiros.
\section{Laurifólio}
\begin{itemize}
\item {Grp. gram.:adj.}
\end{itemize}
\begin{itemize}
\item {Proveniência:(Do lat. \textunderscore laurus\textunderscore  + \textunderscore folium\textunderscore )}
\end{itemize}
Que tem fôlhas semelhantes ás do loireiro.
\section{Laurígero}
\begin{itemize}
\item {Grp. gram.:adj.}
\end{itemize}
\begin{itemize}
\item {Proveniência:(Lat. \textunderscore lauriger\textunderscore )}
\end{itemize}
O mesmo que \textunderscore laurífero\textunderscore .
\section{Laurina}
\begin{itemize}
\item {Grp. gram.:f.}
\end{itemize}
\begin{itemize}
\item {Proveniência:(Do lat. \textunderscore laurus\textunderscore )}
\end{itemize}
Substância crystallina, que se extrai das bagas do loireiro.
\section{Lauríneas}
\begin{itemize}
\item {Grp. gram.:f. pl.}
\end{itemize}
\begin{itemize}
\item {Proveniência:(De \textunderscore laurineo\textunderscore )}
\end{itemize}
Família de plantas, que tem por typo o loireiro.
\section{Lauríneo}
\begin{itemize}
\item {Grp. gram.:adj.}
\end{itemize}
\begin{itemize}
\item {Proveniência:(Do lat. \textunderscore laurus\textunderscore )}
\end{itemize}
Relativo ou semelhante ao loireiro.
\section{Laurino}
\begin{itemize}
\item {Grp. gram.:adj.}
\end{itemize}
\begin{itemize}
\item {Proveniência:(Lat. \textunderscore laurinus\textunderscore )}
\end{itemize}
O mesmo que \textunderscore láureo\textunderscore .
\section{Laurívoro}
\begin{itemize}
\item {Grp. gram.:adj.}
\end{itemize}
\begin{itemize}
\item {Proveniência:(Do lat. \textunderscore laurus\textunderscore  + \textunderscore vorare\textunderscore )}
\end{itemize}
Sobrenome de antigos adivinhos, que mascavam folhas de loireiro, antes de pronunciar os seus vaticínios.
\section{Lauro}
\begin{itemize}
\item {Grp. gram.:adj.}
\end{itemize}
\begin{itemize}
\item {Utilização:Poét.}
\end{itemize}
\begin{itemize}
\item {Grp. gram.:M.}
\end{itemize}
\begin{itemize}
\item {Utilização:fig.}
\end{itemize}
\begin{itemize}
\item {Proveniência:(Lat. \textunderscore laurus\textunderscore )}
\end{itemize}
O mesmo que \textunderscore loiro\textunderscore ^1, adj.
O mesmo que \textunderscore laurel\textunderscore ; prêmio. Cf. Filinto, II, 33.
\section{Lausperene}
\begin{itemize}
\item {Grp. gram.:m.}
\end{itemize}
\begin{itemize}
\item {Proveniência:(Do lat. \textunderscore laus\textunderscore  + \textunderscore perennis\textunderscore )}
\end{itemize}
Exposição contínua e successiva do Santíssimo Sacramento, em todas as igrejas de Lisbôa.
\section{Lausperenne}
\begin{itemize}
\item {Grp. gram.:m.}
\end{itemize}
\begin{itemize}
\item {Proveniência:(Do lat. \textunderscore laus\textunderscore  + \textunderscore perennis\textunderscore )}
\end{itemize}
Exposição contínua e successiva do Santíssimo Sacramento, em todas as igrejas de Lisbôa.
\section{Lautamente}
\begin{itemize}
\item {Grp. gram.:adv.}
\end{itemize}
De modo lauto, com magnificência: \textunderscore banquetear-se lautamente\textunderscore .
\section{Lauto}
\begin{itemize}
\item {Grp. gram.:adj.}
\end{itemize}
\begin{itemize}
\item {Proveniência:(Lat. \textunderscore lautus\textunderscore )}
\end{itemize}
Sumptuoso; magnifico; abundante: \textunderscore jantar lauto\textunderscore .
\section{Lava}
\begin{itemize}
\item {Grp. gram.:f.}
\end{itemize}
\begin{itemize}
\item {Utilização:Fig.}
\end{itemize}
\begin{itemize}
\item {Proveniência:(It. \textunderscore lava\textunderscore )}
\end{itemize}
Matéria, em fusão, que sái ou saíu dos vulcões.
Torrente; enxurrada.
Chamma.
\section{Lavabo}
\begin{itemize}
\item {Grp. gram.:m.}
\end{itemize}
\begin{itemize}
\item {Proveniência:(T. lat.)}
\end{itemize}
Acto de lavar os dedos (o sacerdote) na celebração da Missa.
Oração, que o sacerdote pronuncia nessa occasião.
Pano, com que, depois de lavar os dedos, os limpa.
Depósito de água, com torneira, para lavagens parciais, em refeitórios, latrinas, etc.
\section{Lavação}
\begin{itemize}
\item {Grp. gram.:f.}
\end{itemize}
\begin{itemize}
\item {Proveniência:(Lat. \textunderscore lavatio\textunderscore )}
\end{itemize}
Acto ou effeito de lavar.
\section{Lavacro}
\begin{itemize}
\item {Grp. gram.:m.}
\end{itemize}
\begin{itemize}
\item {Utilização:Fig.}
\end{itemize}
\begin{itemize}
\item {Proveniência:(Lat. \textunderscore lavacrum\textunderscore )}
\end{itemize}
Banho.
Baptismo.
\section{Lavada}
\begin{itemize}
\item {Grp. gram.:f.}
\end{itemize}
\begin{itemize}
\item {Proveniência:(De \textunderscore lavado\textunderscore )}
\end{itemize}
Espécie de rêde de pesca.
\section{Lavadaria}
\begin{itemize}
\item {Grp. gram.:f.}
\end{itemize}
\begin{itemize}
\item {Proveniência:(De \textunderscore lavar\textunderscore )}
\end{itemize}
Officina ou estabelecimento, para lavagem e enxugamento de roupas.
\section{Lavadeira}
\begin{itemize}
\item {Grp. gram.:f.}
\end{itemize}
\begin{itemize}
\item {Utilização:Bras}
\end{itemize}
\begin{itemize}
\item {Utilização:Bras}
\end{itemize}
Mulher, que lava roupa.
Máquina, nas fábricas de lanifícios, para a lavagem das lans.
Insecto, o mesmo quo \textunderscore libellinha\textunderscore .
Espécie de teixugo. Cf. \textunderscore Jorn. do Comm.\textunderscore , do Rio. de 13-VII-902.
O mesmo ou melhor que \textunderscore lavandeira\textunderscore  ou \textunderscore lavandisca\textunderscore .
\section{Lavadeiro}
\begin{itemize}
\item {Grp. gram.:m.}
\end{itemize}
\begin{itemize}
\item {Grp. gram.:Adj.}
\end{itemize}
\begin{itemize}
\item {Proveniência:(De \textunderscore lavar\textunderscore )}
\end{itemize}
Cesto, com que nalgumas praias se mede a sardinha.
Fossa para depósito de águas pluviaes.
Homem, que se emprega em lavar roupa.
\textunderscore Ratinho lavadeiro\textunderscore , pequeno mammífero americano, que costuma lavar em água o alimento, antes de o levar á bôca.
\section{Lavadela}
\begin{itemize}
\item {Grp. gram.:f.}
\end{itemize}
\begin{itemize}
\item {Proveniência:(De \textunderscore lavar\textunderscore )}
\end{itemize}
Lavagem ligeira.
\section{Lava-dente}
\begin{itemize}
\item {Grp. gram.:m.}
\end{itemize}
\begin{itemize}
\item {Utilização:Pop.}
\end{itemize}
Beberete; pinga.
\section{Lava-dentes}
\begin{itemize}
\item {Grp. gram.:m.}
\end{itemize}
\begin{itemize}
\item {Utilização:Prov.}
\end{itemize}
\begin{itemize}
\item {Utilização:trasm.}
\end{itemize}
Sarabanda, descompostura.
\section{Lavadiço}
\begin{itemize}
\item {Grp. gram.:adj.}
\end{itemize}
\begin{itemize}
\item {Utilização:P. us.}
\end{itemize}
\begin{itemize}
\item {Proveniência:(De \textunderscore lavado\textunderscore )}
\end{itemize}
Que gosta de se lavar; que anda muito lavado.
\section{Lavado}
\begin{itemize}
\item {Grp. gram.:adj.}
\end{itemize}
\begin{itemize}
\item {Utilização:Fig.}
\end{itemize}
\begin{itemize}
\item {Grp. gram.:M.}
\end{itemize}
\begin{itemize}
\item {Utilização:Gír.}
\end{itemize}
\begin{itemize}
\item {Utilização:Prov.}
\end{itemize}
Franco; generoso: \textunderscore coração lavado\textunderscore .
Arejado: \textunderscore casa muito lavada\textunderscore  (de ares).
Coração de uma peça de caça, desfeito em água morna, o qual se dava aos falcões.
Quartilho de vinho.
Serviço de lavar: \textunderscore hoje é dia de lavados\textunderscore .
\section{Lavadoira}
\begin{itemize}
\item {Grp. gram.:f.}
\end{itemize}
\begin{itemize}
\item {Utilização:Prov.}
\end{itemize}
\begin{itemize}
\item {Utilização:alent.}
\end{itemize}
A junta de bois que, em meio do duas, ajuda a puxar um carro.
(Talvez por \textunderscore levadoira\textunderscore , de \textunderscore levar\textunderscore )
\section{Lavadoira}
\begin{itemize}
\item {Grp. gram.:f.}
\end{itemize}
\begin{itemize}
\item {Utilização:Marn.}
\end{itemize}
Córte vertical, espécie do degrau, formado pela tirada dos torrões, com que se fórma a vedação das salinas. Cf. \textunderscore Mus. Technol.\textunderscore ,55.
(Cp. \textunderscore lavadoiro\textunderscore )
\section{Lavadoiro}
\begin{itemize}
\item {Grp. gram.:m.}
\end{itemize}
\begin{itemize}
\item {Utilização:Marn.}
\end{itemize}
\begin{itemize}
\item {Grp. gram.:Adj.}
\end{itemize}
\begin{itemize}
\item {Utilização:Prov.}
\end{itemize}
\begin{itemize}
\item {Utilização:trasm.}
\end{itemize}
\begin{itemize}
\item {Proveniência:(De \textunderscore lavar\textunderscore )}
\end{itemize}
Tanque ou lugar, onde se lava roupa.
Cova, que os antigos marnotos abriam, junto ao tabuleiro do sal, e para onde era rido o producto da marinha, sendo alli remexido muitas vezes com rasoira e ugalho, para se lhe tirarem as impurezas.
Em que se lava roupa.
Que serve para se lavar roupa: \textunderscore tanque lavadoiro\textunderscore .
\section{Lavador}
\begin{itemize}
\item {Grp. gram.:m.}
\end{itemize}
\begin{itemize}
\item {Proveniência:(De \textunderscore lavar\textunderscore )}
\end{itemize}
Um dos instrumentos agrícolas, destinados a preparar o alimento vegetal para o arraçoamento de animaes.
\section{Lavadoura}
\begin{itemize}
\item {Grp. gram.:f.}
\end{itemize}
\begin{itemize}
\item {Utilização:Marn.}
\end{itemize}
Córte vertical, espécie do degrau, formado pela tirada dos torrões, com que se fórma a vedação das salinas. Cf. \textunderscore Mus. Technol.\textunderscore ,55.
(Cp. \textunderscore lavadouro\textunderscore )
\section{Lavadouro}
\begin{itemize}
\item {Grp. gram.:m.}
\end{itemize}
\begin{itemize}
\item {Utilização:Marn.}
\end{itemize}
\begin{itemize}
\item {Grp. gram.:Adj.}
\end{itemize}
\begin{itemize}
\item {Utilização:Prov.}
\end{itemize}
\begin{itemize}
\item {Utilização:trasm.}
\end{itemize}
\begin{itemize}
\item {Proveniência:(De \textunderscore lavar\textunderscore )}
\end{itemize}
Tanque ou lugar, onde se lava roupa.
Cova, que os antigos marnotos abriam, junto ao tabuleiro do sal, e para onde era rido o producto da marinha, sendo alli remexido muitas vezes com rasoira e ugalho, para se lhe tirarem as impurezas.
Em que se lava roupa.
Que serve para se lavar roupa: \textunderscore tanque lavadouro\textunderscore .
\section{Lavadura}
\begin{itemize}
\item {Grp. gram.:f.}
\end{itemize}
Acto de lavar.
Água, em que se lavou loiça de mesa, e que se dá como alimento a porcos.
\section{Lavagante}
\begin{itemize}
\item {Grp. gram.:m.}
\end{itemize}
Crustáceo decápode, marítimo, um pouco mais pequeno que a lagosta e munido de duas fortes torqueses nos braços (\textunderscore homarus vulgaris\textunderscore ).
(Cp. cast. \textunderscore lobogante\textunderscore )
\section{Lavagem}
\begin{itemize}
\item {Grp. gram.:f.}
\end{itemize}
\begin{itemize}
\item {Utilização:Prov.}
\end{itemize}
O mesmo que \textunderscore lavadura\textunderscore .
Comida para os porcos.
Pagamento do trabalho de lavar.
\section{Lavajado}
\begin{itemize}
\item {Grp. gram.:adj.}
\end{itemize}
\begin{itemize}
\item {Utilização:Prov.}
\end{itemize}
\begin{itemize}
\item {Utilização:alent.}
\end{itemize}
\begin{itemize}
\item {Proveniência:(De \textunderscore lavajo\textunderscore )}
\end{itemize}
Emporcalhado pela água do lavajo.
Sujo pelo lodo dos charcos: \textunderscore o cão vinha todo lavajado\textunderscore .
\section{Lavajo}
\begin{itemize}
\item {Grp. gram.:m.}
\end{itemize}
\begin{itemize}
\item {Utilização:Prov.}
\end{itemize}
\begin{itemize}
\item {Utilização:alent.}
\end{itemize}
Pequeno pântano.
Charco, atoleiro.
(Cp. \textunderscore lavagem\textunderscore )
\section{Lavajola}
\begin{itemize}
\item {Grp. gram.:f.}
\end{itemize}
\begin{itemize}
\item {Utilização:Prov.}
\end{itemize}
\begin{itemize}
\item {Utilização:beir.}
\end{itemize}
Terreno baixo, que no inverno se alaga em água.
(Cp. \textunderscore lavajo\textunderscore )
\section{Lavajona}
\begin{itemize}
\item {Grp. gram.:f.}
\end{itemize}
\begin{itemize}
\item {Utilização:Prov.}
\end{itemize}
\begin{itemize}
\item {Proveniência:(De \textunderscore lavajo\textunderscore )}
\end{itemize}
Mulher suja e desavergonhada.
\section{Lavamento}
\begin{itemize}
\item {Grp. gram.:m.}
\end{itemize}
Acto ou effeito de lavar.
\section{Lavanca}
\begin{itemize}
\item {Grp. gram.:f.}
\end{itemize}
\begin{itemize}
\item {Utilização:Ant.}
\end{itemize}
O mesmo que \textunderscore alavanca\textunderscore . Cf. B. Pereira, \textunderscore Prosódia\textunderscore , vb. \textunderscore mochlium\textunderscore .
\section{Lavanco}
\begin{itemize}
\item {Grp. gram.:m.}
\end{itemize}
O mesmo que \textunderscore ganso\textunderscore ^1; ádem.
\section{Lavanda}
\begin{itemize}
\item {Grp. gram.:f.}
\end{itemize}
\begin{itemize}
\item {Utilização:P. us.}
\end{itemize}
\begin{itemize}
\item {Proveniência:(T. it.)}
\end{itemize}
Alfazema.
\section{Lavandaria}
\begin{itemize}
\item {Grp. gram.:f.}
\end{itemize}
O mesmo que \textunderscore lavadaria\textunderscore .
\section{Lavandeira}
\begin{itemize}
\item {Grp. gram.:f.}
\end{itemize}
\begin{itemize}
\item {Utilização:Bras}
\end{itemize}
Ave aquática (\textunderscore charadrius\textunderscore ).
O mesmo que \textunderscore lavandisca\textunderscore .
O mesmo que \textunderscore borrelho\textunderscore .
Passarinho branco, de asas negras.
(Cp. fr. \textunderscore levandiere\textunderscore )
\section{Lavanderia}
\begin{itemize}
\item {Grp. gram.:f.}
\end{itemize}
\begin{itemize}
\item {Utilização:Gal}
\end{itemize}
\begin{itemize}
\item {Proveniência:(Fr. \textunderscore lavanderie\textunderscore )}
\end{itemize}
(V.lavadaria)
\section{Lavandisca}
\begin{itemize}
\item {Grp. gram.:f.}
\end{itemize}
Espécie de alvéloa.
(Cp. \textunderscore lavandeira\textunderscore )
\section{Làvapé}
\begin{itemize}
\item {Grp. gram.:m.}
\end{itemize}
\begin{itemize}
\item {Grp. gram.:Pl.}
\end{itemize}
\begin{itemize}
\item {Proveniência:(De \textunderscore lavar\textunderscore  + \textunderscore pé\textunderscore )}
\end{itemize}
Planta, espécie de centáurea.
Festa em que a Igreja, no dia de quinta-feira de Endoenças, celebra o facto de Jesus têr lavado os pés a seus discípulos.
\section{Lava-peixe}
\begin{itemize}
\item {Grp. gram.:m.  e  f.}
\end{itemize}
\begin{itemize}
\item {Utilização:Ant.}
\end{itemize}
Pessôa que, nas ribeiras e mercados, tinha por offício lavar o peixe escamado.
\section{Lava-pratos}
\begin{itemize}
\item {Grp. gram.:m.}
\end{itemize}
O mesmo que \textunderscore mamanga\textunderscore .
\section{Lavar}
\begin{itemize}
\item {Grp. gram.:v. t.}
\end{itemize}
\begin{itemize}
\item {Utilização:Fig.}
\end{itemize}
\begin{itemize}
\item {Proveniência:(Lat. \textunderscore lavare\textunderscore )}
\end{itemize}
Limpar, banhando: \textunderscore lavar a cara\textunderscore .
Tirar com água as impurezas de: \textunderscore lavar a roupa\textunderscore .
Tornar puro.
Cercar de águas.
Percorrer, regando ou banhando.
\section{Lavareda}
\begin{itemize}
\item {Grp. gram.:f.}
\end{itemize}
O mesmo que \textunderscore labarêda\textunderscore .
\section{Lavarejar}
\begin{itemize}
\item {Grp. gram.:v. i.}
\end{itemize}
\begin{itemize}
\item {Utilização:Prov.}
\end{itemize}
\begin{itemize}
\item {Utilização:trasm.}
\end{itemize}
(Corr. de \textunderscore lambarejar\textunderscore )
\section{Lavarinto}
\begin{itemize}
\item {Grp. gram.:m.}
\end{itemize}
\begin{itemize}
\item {Utilização:Bras. do N}
\end{itemize}
Fórma pop. de \textunderscore labyrintho\textunderscore .
Trabalho de agulha, também conhecido por \textunderscore crivo\textunderscore .
\section{Lavátera}
\begin{itemize}
\item {Grp. gram.:f.}
\end{itemize}
\begin{itemize}
\item {Proveniência:(De \textunderscore Lavater\textunderscore , n. p.)}
\end{itemize}
Gênero de plantas malváceas.
\section{Lavateriano}
\begin{itemize}
\item {Grp. gram.:adj.}
\end{itemize}
Relativo a Lavater.
\section{Lavático}
\begin{itemize}
\item {Grp. gram.:adj.}
\end{itemize}
\begin{itemize}
\item {Proveniência:(Do rad. de \textunderscore lavar\textunderscore )}
\end{itemize}
Que serve para clister.
\section{Lavativo}
\begin{itemize}
\item {Grp. gram.:adj.}
\end{itemize}
O mesmo que \textunderscore lavático\textunderscore .
\section{Lavatório}
\begin{itemize}
\item {Grp. gram.:m.}
\end{itemize}
\begin{itemize}
\item {Utilização:Fig.}
\end{itemize}
\begin{itemize}
\item {Proveniência:(Lat. \textunderscore lavatorium\textunderscore )}
\end{itemize}
Utensílio, que sustenta uma bacia em que se lavam as mãos e o rosto.
Acto de lavar.
Água, que os Cathólicos bebem depois da communhão.
Purificação.
\section{Lavável}
\begin{itemize}
\item {Grp. gram.:adj.}
\end{itemize}
Que se póde lavar.
\section{Lavegada}
\begin{itemize}
\item {Grp. gram.:f.}
\end{itemize}
\begin{itemize}
\item {Utilização:Ant.}
\end{itemize}
\begin{itemize}
\item {Proveniência:(De \textunderscore lavegar\textunderscore )}
\end{itemize}
O mesmo que \textunderscore arroteia\textunderscore .
\section{Lavegante}
\begin{itemize}
\item {Grp. gram.:m.}
\end{itemize}
O mesmo que \textunderscore lavagante\textunderscore .
\section{Lavegar}
\begin{itemize}
\item {Grp. gram.:v. t.}
\end{itemize}
\begin{itemize}
\item {Proveniência:(Do lat. \textunderscore laevigare\textunderscore ?)}
\end{itemize}
Lavrar com lavêgo.
\section{Lavêgo}
\begin{itemize}
\item {Grp. gram.:m.}
\end{itemize}
\begin{itemize}
\item {Proveniência:(De \textunderscore lavegar\textunderscore )}
\end{itemize}
Arado com varredoiro, o mesmo que \textunderscore labego\textunderscore .
\section{Laverca}
\begin{itemize}
\item {Grp. gram.:f.}
\end{itemize}
\begin{itemize}
\item {Utilização:Prov.}
\end{itemize}
\begin{itemize}
\item {Utilização:minh.}
\end{itemize}
Ave, espécie de cotovía.
Calhandra, (\textunderscore alauda arvensis\textunderscore , Lin.).
Pessôa muito magra.
\section{Laverco}
\begin{itemize}
\item {Grp. gram.:m.}
\end{itemize}
Macho da laverca. Cf. B. Pato, \textunderscore Livro do Monte\textunderscore .
Homem finório e trapaceiro. Cf. \textunderscore Hyssope\textunderscore , 52.
\section{Lavoeira}
\begin{itemize}
\item {Grp. gram.:f.}
\end{itemize}
\begin{itemize}
\item {Utilização:Prov.}
\end{itemize}
\begin{itemize}
\item {Utilização:trasm.}
\end{itemize}
O mesmo que \textunderscore lavoura\textunderscore .
\section{Lavoira}
\begin{itemize}
\item {Grp. gram.:f.}
\end{itemize}
\begin{itemize}
\item {Utilização:Prov.}
\end{itemize}
\begin{itemize}
\item {Utilização:alent.}
\end{itemize}
\begin{itemize}
\item {Proveniência:(Do b. lat. \textunderscore laboria\textunderscore . Cf. B. Pereira, \textunderscore Prosódia\textunderscore )}
\end{itemize}
Preparação do terreno para sementeira ou plantação.
Agricultura.
Acto de cultivar a terra.
Terra cultivada.
Exploração agricola e pecuária de uma herdade ou grupo de herdades.
\section{Lavoisiéreas}
\begin{itemize}
\item {Grp. gram.:f. pl.}
\end{itemize}
\begin{itemize}
\item {Utilização:Bot.}
\end{itemize}
\begin{itemize}
\item {Proveniência:(De \textunderscore Lavoisier\textunderscore , n. p.)}
\end{itemize}
Tríbo de melastomáceas, na classificação de De-Candolle.
\section{Lavor}
\begin{itemize}
\item {Grp. gram.:m.}
\end{itemize}
\begin{itemize}
\item {Utilização:Ext.}
\end{itemize}
\begin{itemize}
\item {Proveniência:(Lat. \textunderscore labor\textunderscore )}
\end{itemize}
Trabalho manual.
Labor.
Trabalho.
Ornato em relêvo.
Obra do agulha, feita por desenho: \textunderscore a pequena é hábil em lavores\textunderscore .
Lavrado.
Crystallização superficial nas salinas, que impede a evaporação e, portanto, a formação do sal.
\section{Lavorar}
\begin{itemize}
\item {Grp. gram.:v. t.}
\end{itemize}
\begin{itemize}
\item {Proveniência:(Lat. \textunderscore laborare\textunderscore )}
\end{itemize}
Fazer lavores em; lavrar.
\section{Lavorear}
\begin{itemize}
\item {Grp. gram.:v. t.}
\end{itemize}
\begin{itemize}
\item {Utilização:Prov.}
\end{itemize}
\begin{itemize}
\item {Utilização:trasm.}
\end{itemize}
\begin{itemize}
\item {Proveniência:(De \textunderscore lavor\textunderscore )}
\end{itemize}
Adornar muito, tornar primoroso.
\section{Lavoso}
\begin{itemize}
\item {Grp. gram.:adj.}
\end{itemize}
Relativo a lava; que é da natureza da lava.
\section{Lavoura}
\begin{itemize}
\item {Grp. gram.:f.}
\end{itemize}
\begin{itemize}
\item {Utilização:Prov.}
\end{itemize}
\begin{itemize}
\item {Utilização:alent.}
\end{itemize}
\begin{itemize}
\item {Proveniência:(Do b. lat. \textunderscore laboria\textunderscore . Cf. B. Pereira, \textunderscore Prosódia\textunderscore )}
\end{itemize}
Preparação do terreno para sementeira ou plantação.
Agricultura.
Acto de cultivar a terra.
Terra cultivada.
Exploração agricola e pecuária de uma herdade ou grupo de herdades.
\section{Lavra}
\begin{itemize}
\item {Grp. gram.:f.}
\end{itemize}
Acto de lavrar.
Lavoira.
\section{Lavração}
\begin{itemize}
\item {Grp. gram.:f.}
\end{itemize}
\begin{itemize}
\item {Utilização:T. de Ceilão}
\end{itemize}
\begin{itemize}
\item {Proveniência:(De \textunderscore lavrar\textunderscore )}
\end{itemize}
O mesmo que \textunderscore lavoira\textunderscore .
\section{Lavrada}
\begin{itemize}
\item {Grp. gram.:f.}
\end{itemize}
O mesmo que \textunderscore lavra\textunderscore .
\section{Lavradeira}
\begin{itemize}
\item {Grp. gram.:f.}
\end{itemize}
Mulher, que lavra.
Camponesa.
Mulher de lavrador.
Mulher, que faz renda ou lavores de agulha.
\section{Lavradeiro}
\begin{itemize}
\item {Grp. gram.:adj.}
\end{itemize}
\begin{itemize}
\item {Proveniência:(De \textunderscore lavrar\textunderscore )}
\end{itemize}
Que se emprega na lavoira, (falando-se de animaes).
\section{Lavradio}
\begin{itemize}
\item {Grp. gram.:adj.}
\end{itemize}
\begin{itemize}
\item {Grp. gram.:M.}
\end{itemize}
Que se póde lavrar; próprio para se lavrar: \textunderscore terra lavradia\textunderscore .
Lavoira.
\section{Lavrado}
\begin{itemize}
\item {Grp. gram.:m.}
\end{itemize}
\begin{itemize}
\item {Grp. gram.:M. pl.}
\end{itemize}
\begin{itemize}
\item {Utilização:Bras. de Minas}
\end{itemize}
Terra lavrada.
Lavor de agulha; bordado.
Contas de oiro, que formam collar; adornos de oiro e prata. Cf. Taunay, \textunderscore Innocência\textunderscore , 76.
\section{Lavrador}
\begin{itemize}
\item {Grp. gram.:m.  e  adj.}
\end{itemize}
\begin{itemize}
\item {Proveniência:(De \textunderscore lavrar\textunderscore )}
\end{itemize}
O que trabalha em lavoira.
Aquelle que possue propriedades lavradias.
Proprietário de herdades.
Dono de salinas.
\section{Lavragem}
\begin{itemize}
\item {Grp. gram.:f.}
\end{itemize}
Acto ou effeito de lavrar; lavoira.
\section{Lavramento}
\begin{itemize}
\item {Grp. gram.:m.}
\end{itemize}
Acto de lavrar.
\section{Lavrança}
\begin{itemize}
\item {Grp. gram.:f.}
\end{itemize}
\begin{itemize}
\item {Utilização:Ant.}
\end{itemize}
\begin{itemize}
\item {Proveniência:(De \textunderscore lavrar\textunderscore )}
\end{itemize}
Lavra; terra lavradia.
\section{Lavrandeira}
\begin{itemize}
\item {Grp. gram.:f.}
\end{itemize}
\begin{itemize}
\item {Utilização:Des.}
\end{itemize}
Mulher, que faz lavores; bordadeira; costureira. Cf. G. Vicente e Ribeiro Chiado.
(Cp. \textunderscore lavradeira\textunderscore )
\section{Lavrante}
\begin{itemize}
\item {Grp. gram.:adj.}
\end{itemize}
\begin{itemize}
\item {Grp. gram.:M.}
\end{itemize}
Que lavra.
Artista, que lavra em oiro ou prata.
\section{Lavrar}
\begin{itemize}
\item {Grp. gram.:v. t.}
\end{itemize}
\begin{itemize}
\item {Utilização:Ext.}
\end{itemize}
\begin{itemize}
\item {Utilização:Fig.}
\end{itemize}
\begin{itemize}
\item {Grp. gram.:V. i.}
\end{itemize}
\begin{itemize}
\item {Proveniência:(Lat. \textunderscore laborare\textunderscore )}
\end{itemize}
Fazer regos com arado em: \textunderscore lavrar a terra\textunderscore .
Cultivar (terras).
Cinzelar.
Aplainar.
Bordar.
Abrir ornatos em.
Preparar com lavores.
Construír.
Corroer.
Consignar; inscrever: \textunderscore lavrar um epitáphio\textunderscore .
Explorar (minas).
Alastrar-se: \textunderscore a epidemia vai lavrando\textunderscore .
Avultar; desenvolver-se.
\section{Laxação}
\begin{itemize}
\item {Grp. gram.:f.}
\end{itemize}
\begin{itemize}
\item {Proveniência:(Lat. \textunderscore laxatio\textunderscore )}
\end{itemize}
Acto ou effeito de laxar.
\section{Laxamente}
\begin{itemize}
\item {Grp. gram.:adv.}
\end{itemize}
De modo laxo; froixamente.
\section{Laxante}
\begin{itemize}
\item {Grp. gram.:adj.}
\end{itemize}
\begin{itemize}
\item {Grp. gram.:M.}
\end{itemize}
\begin{itemize}
\item {Proveniência:(Lat. \textunderscore laxans\textunderscore )}
\end{itemize}
Que laxa.
Purgante ligeiro.
\section{Laxar}
\begin{itemize}
\item {Grp. gram.:v. t.}
\end{itemize}
\begin{itemize}
\item {Utilização:Fig.}
\end{itemize}
\begin{itemize}
\item {Proveniência:(Lat. \textunderscore laxare\textunderscore )}
\end{itemize}
Tornar froixo.
Alargar; desimpedir.
Attenuar; alliviar.
Relaxar.
\section{Laxativo}
\begin{itemize}
\item {Grp. gram.:m.  e  adj.}
\end{itemize}
\begin{itemize}
\item {Proveniência:(Lat. \textunderscore laxativos\textunderscore )}
\end{itemize}
O mesmo que \textunderscore laxante\textunderscore .
\section{Laxidão}
\begin{itemize}
\item {fónica:csi}
\end{itemize}
\begin{itemize}
\item {Grp. gram.:f.}
\end{itemize}
\begin{itemize}
\item {Proveniência:(De \textunderscore laxo\textunderscore )}
\end{itemize}
O mesmo que \textunderscore lassidão\textunderscore .
\section{Laxifloro}
\begin{itemize}
\item {fónica:csi}
\end{itemize}
\begin{itemize}
\item {Grp. gram.:adj.}
\end{itemize}
\begin{itemize}
\item {Utilização:Bot.}
\end{itemize}
\begin{itemize}
\item {Proveniência:(Do lat. \textunderscore laxus\textunderscore  + \textunderscore flos\textunderscore , \textunderscore floris\textunderscore )}
\end{itemize}
Que tem as flôres muito afastadas umas das outras.
\section{Laxiorismo}
\begin{itemize}
\item {fónica:csi}
\end{itemize}
\begin{itemize}
\item {Grp. gram.:m.}
\end{itemize}
\begin{itemize}
\item {Utilização:Des.}
\end{itemize}
\begin{itemize}
\item {Proveniência:(Do lat. \textunderscore laxior\textunderscore )}
\end{itemize}
Opinião relaxada em moral.
\section{Laxo}
\begin{itemize}
\item {Grp. gram.:adj.}
\end{itemize}
\begin{itemize}
\item {Proveniência:(Lat. \textunderscore laxus\textunderscore )}
\end{itemize}
Desimpedido.
Alargado; froixo; bambo.
Lasso.
\section{Lazã}
(\textunderscore fem.\textunderscore  de \textunderscore lazão\textunderscore )
\section{Lazan}
(\textunderscore fem.\textunderscore  de \textunderscore lazão\textunderscore )
\section{Lazão}
\begin{itemize}
\item {Grp. gram.:adj.}
\end{itemize}
(V.alazão)
\section{Lazarado}
\begin{itemize}
\item {Grp. gram.:adj.}
\end{itemize}
\begin{itemize}
\item {Utilização:Prov.}
\end{itemize}
\begin{itemize}
\item {Utilização:beir.}
\end{itemize}
\begin{itemize}
\item {Proveniência:(De \textunderscore lazarar\textunderscore )}
\end{itemize}
O mesmo que \textunderscore faminto\textunderscore .
\section{Lazarar}
\begin{itemize}
\item {Grp. gram.:v. i.}
\end{itemize}
Ter muita fome. Cf. \textunderscore Agostinheida\textunderscore , 141.
(Cp. \textunderscore lazeirar\textunderscore )
\section{Lazarento}
\begin{itemize}
\item {Grp. gram.:m.  e  adj.}
\end{itemize}
\begin{itemize}
\item {Proveniência:(De \textunderscore lázaro\textunderscore )}
\end{itemize}
Aquelle que tem pústulas; leproso.
\section{Lazarento}
\begin{itemize}
\item {Grp. gram.:adj.}
\end{itemize}
\begin{itemize}
\item {Utilização:Pop.}
\end{itemize}
Esfomeado: o mesmo que \textunderscore lazerento\textunderscore ^2.
\section{Lazaretário}
\begin{itemize}
\item {Grp. gram.:adj.}
\end{itemize}
\begin{itemize}
\item {Utilização:Neol.}
\end{itemize}
Relativo a lazareto: \textunderscore regime lazaretário\textunderscore .
\section{Lazareto}
\begin{itemize}
\item {fónica:zarê}
\end{itemize}
\begin{itemize}
\item {Grp. gram.:m.}
\end{itemize}
\begin{itemize}
\item {Proveniência:(It. \textunderscore lazaretto\textunderscore )}
\end{itemize}
Edifício para quarentenas.
\section{Lazarina}
\begin{itemize}
\item {Grp. gram.:f.}
\end{itemize}
\begin{itemize}
\item {Proveniência:(De \textunderscore Lázaro\textunderscore , n. p.)}
\end{itemize}
Arma de fuzil, comprida e de pequeno calibre, que se fabríca na Bélgica para os pretos da África, e que é imitação grosseira das que entre nós fabricava o armeiro Lázaro, de Braga. Cf. Capello e Ivens, I, 7.
\section{Lazarismo}
\begin{itemize}
\item {Grp. gram.:m.}
\end{itemize}
Doutrina dos lazaristas.
\section{Lazarista}
\begin{itemize}
\item {Grp. gram.:m.}
\end{itemize}
\begin{itemize}
\item {Proveniência:(De \textunderscore lázaro\textunderscore )}
\end{itemize}
Membro da Ordem religiosa de San-Vicente de Paulo.
\section{Lázaro}
\begin{itemize}
\item {Grp. gram.:m.}
\end{itemize}
\begin{itemize}
\item {Utilização:Ext.}
\end{itemize}
\begin{itemize}
\item {Utilização:T. de Évora  e}
\end{itemize}
\begin{itemize}
\item {Utilização:ant.}
\end{itemize}
Aquelle que é leproso.
Aquelle que está coberto de chagas ou pústulas.
O mesmo que \textunderscore asylado\textunderscore : \textunderscore o producto do bazar foi para os lázaros\textunderscore .
(B. lat. \textunderscore lazarus\textunderscore )
\section{Lazarone}
\begin{itemize}
\item {Grp. gram.:m.}
\end{itemize}
\begin{itemize}
\item {Utilização:Ext.}
\end{itemize}
\begin{itemize}
\item {Proveniência:(It. \textunderscore lazarone\textunderscore )}
\end{itemize}
Mendigo de Nápoles.
Mendigo.
\section{Lazeira}
\begin{itemize}
\item {Grp. gram.:f.}
\end{itemize}
\begin{itemize}
\item {Utilização:Fig.}
\end{itemize}
\begin{itemize}
\item {Proveniência:(De \textunderscore lázaro\textunderscore )}
\end{itemize}
Miséria.
Desgraça.
Lepra.
Fome: \textunderscore cair de lazeira\textunderscore .
\section{Lazeirar}
\begin{itemize}
\item {Grp. gram.:v. i.}
\end{itemize}
Têr lazeira; estar esfomeado.
\section{Lazer}
\begin{itemize}
\item {Grp. gram.:m.}
\end{itemize}
\begin{itemize}
\item {Proveniência:(Do lat. \textunderscore licere\textunderscore )}
\end{itemize}
O mesmo que ócio.
Vagar; passatempo.
\section{Lazerar}
\begin{itemize}
\item {Grp. gram.:v. t.}
\end{itemize}
\begin{itemize}
\item {Utilização:Ant.}
\end{itemize}
\begin{itemize}
\item {Proveniência:(Do lat. \textunderscore lacerare\textunderscore ?)}
\end{itemize}
Expiar.
Compensar; indemnizar.
Causar damno ou soffrimento a. Cf. Usque, 50, v.^o; \textunderscore Eufrosina\textunderscore , 42.
\section{Lazerar}
\begin{itemize}
\item {Grp. gram.:v. i.}
\end{itemize}
\begin{itemize}
\item {Utilização:Pop.}
\end{itemize}
O mesmo que \textunderscore lazeirar\textunderscore .
\section{Lazerento}
\begin{itemize}
\item {Grp. gram.:adj.}
\end{itemize}
\begin{itemize}
\item {Proveniência:(De \textunderscore lazerar\textunderscore ^1)}
\end{itemize}
Leproso.
Chagado.
\section{Lazerento}
\begin{itemize}
\item {Grp. gram.:adj.}
\end{itemize}
\begin{itemize}
\item {Proveniência:(De \textunderscore lazerar\textunderscore ^2)}
\end{itemize}
Faminto.
Miserável.
\section{Lazúli}
\begin{itemize}
\item {Grp. gram.:m.}
\end{itemize}
\begin{itemize}
\item {Proveniência:(Do b. lat. \textunderscore lazur\textunderscore )}
\end{itemize}
Pedra azul, opaca e listrada de branco, com pontos amarelos.
\section{Lazulita}
\begin{itemize}
\item {Grp. gram.:f.}
\end{itemize}
\begin{itemize}
\item {Proveniência:(Do b. lat. \textunderscore lazur\textunderscore )}
\end{itemize}
Pedra azul, opaca e listrada de branco, com pontos amarelos.
\section{Lazulite}
\begin{itemize}
\item {Grp. gram.:f.}
\end{itemize}
\begin{itemize}
\item {Proveniência:(Do b. lat. \textunderscore lazur\textunderscore )}
\end{itemize}
Pedra azul, opaca e listrada de branco, com pontos amarelos.
\section{Le}
\begin{itemize}
\item {Grp. gram.:pron.}
\end{itemize}
\begin{itemize}
\item {Utilização:Ant.}
\end{itemize}
O mesmo que \textunderscore lhe\textunderscore .
\section{Lé}
Monossýllabo, us. na loc. \textunderscore cré com cré, lé com lé\textunderscore , equivalente a \textunderscore cada qual com os da sua igualha\textunderscore .
(Cp. \textunderscore cré\textunderscore ^2)
\section{Leal}
\begin{itemize}
\item {Grp. gram.:adj.}
\end{itemize}
\begin{itemize}
\item {Grp. gram.:M.}
\end{itemize}
\begin{itemize}
\item {Proveniência:(Do lat. \textunderscore legalis\textunderscore )}
\end{itemize}
Conforme com a lei.
Digno.
Honesto.
Sincero.
Fiel: \textunderscore espôsa leal\textunderscore .
Antiga moeda de prata, que na Índia portuguesa valia 12 reis.
Moéda, correspondente a 10 reis, em tempo de D. João I.
\section{Lealdação}
\begin{itemize}
\item {Grp. gram.:f.}
\end{itemize}
Acto do lealdar.
\section{Lealdade}
\begin{itemize}
\item {Grp. gram.:f.}
\end{itemize}
Qualidade de leal.
Acção leal.
\section{Lealdador}
\begin{itemize}
\item {Grp. gram.:adj.}
\end{itemize}
\begin{itemize}
\item {Grp. gram.:M.}
\end{itemize}
\begin{itemize}
\item {Proveniência:(De \textunderscore lealdar\textunderscore )}
\end{itemize}
Que lealda.
Antigo fiscal das mercadorias que entravam na cidade.
\section{Lealdamento}
\begin{itemize}
\item {Grp. gram.:m.}
\end{itemize}
O mesmo que \textunderscore lealdação\textunderscore .
\section{Lealdar}
\begin{itemize}
\item {Grp. gram.:v. t.}
\end{itemize}
\begin{itemize}
\item {Proveniência:(De \textunderscore leal\textunderscore )}
\end{itemize}
Verificar, dar ao manifesto na alfândega.
\section{Lealdoso}
\begin{itemize}
\item {Grp. gram.:adj.}
\end{itemize}
\begin{itemize}
\item {Utilização:Des.}
\end{itemize}
\begin{itemize}
\item {Proveniência:(De \textunderscore lealdade\textunderscore )}
\end{itemize}
O mesmo que \textunderscore leal\textunderscore .
\section{Lealeza}
\begin{itemize}
\item {Grp. gram.:f.}
\end{itemize}
\begin{itemize}
\item {Utilização:Ant.}
\end{itemize}
\begin{itemize}
\item {Proveniência:(De \textunderscore leal\textunderscore )}
\end{itemize}
O mesmo que \textunderscore lealdade\textunderscore . Cf. Filinto, \textunderscore D. Man.\textunderscore , I, 375.
\section{Lealismo}
\begin{itemize}
\item {Grp. gram.:m.}
\end{itemize}
\begin{itemize}
\item {Utilização:Neol.}
\end{itemize}
O mesmo que \textunderscore lealdade\textunderscore .
Obediência dos cidadãos ao respectivo Govêrno.
Acatamento, com que numa colónia são recebidas as leis e ordens da respectiva metrópole.
\section{Lealmente}
\begin{itemize}
\item {Grp. gram.:adv.}
\end{itemize}
De modo leal; fielmente.
Com dignidade; com honra.
\section{Leandra}
\begin{itemize}
\item {Grp. gram.:f.}
\end{itemize}
Gênero de arbustos melastomáceos do Brasil.
\section{Leandro}
\begin{itemize}
\item {Grp. gram.:m.}
\end{itemize}
Gênero de arbustos melastomáceos do Brasil.
\section{Leão}
\begin{itemize}
\item {Grp. gram.:m.}
\end{itemize}
\begin{itemize}
\item {Utilização:Fig.}
\end{itemize}
\begin{itemize}
\item {Utilização:Ant.}
\end{itemize}
\begin{itemize}
\item {Utilização:escol.}
\end{itemize}
\begin{itemize}
\item {Proveniência:(Do lat. \textunderscore leo\textunderscore )}
\end{itemize}
Quadrúpede carnívoro, que habita principalmente a África.
Homem ousado, valente.
Pessôa célebre.
Homem intratável, rispido.
Homem namorador, que alardeia conquistas amorosas.
Constellação.
Figura do leão nos brasões.
Pequena peça de artilharia.
Bacio da cama.
\section{Leãozete}
\begin{itemize}
\item {fónica:zê}
\end{itemize}
\begin{itemize}
\item {Grp. gram.:m.}
\end{itemize}
Pequeno leão; leão pouco encorpado. Cf. Filinto, III, 218.
\section{Lebedoiro}
\begin{itemize}
\item {Grp. gram.:m.}
\end{itemize}
\begin{itemize}
\item {Utilização:Ant.}
\end{itemize}
Terra húmida, que produz erva; lenteiro.
(Por \textunderscore levedoiro\textunderscore , de \textunderscore lêvedo\textunderscore ?)
\section{Lebedouro}
\begin{itemize}
\item {Grp. gram.:m.}
\end{itemize}
\begin{itemize}
\item {Utilização:Ant.}
\end{itemize}
Terra húmida, que produz erva; lenteiro.
(Por \textunderscore levedoiro\textunderscore , de \textunderscore lêvedo\textunderscore ?)
\section{Lebetantho}
\begin{itemize}
\item {Grp. gram.:m.}
\end{itemize}
\begin{itemize}
\item {Proveniência:(Do gr. \textunderscore lebes\textunderscore  + \textunderscore anthos\textunderscore )}
\end{itemize}
Arbusto da América austral.
\section{Lebetanto}
\begin{itemize}
\item {Grp. gram.:m.}
\end{itemize}
\begin{itemize}
\item {Proveniência:(Do gr. \textunderscore lebes\textunderscore  + \textunderscore anthos\textunderscore )}
\end{itemize}
Arbusto da América austral.
\section{Lebetina}
\begin{itemize}
\item {Grp. gram.:f.}
\end{itemize}
\begin{itemize}
\item {Proveniência:(Do gr. \textunderscore lebes\textunderscore )}
\end{itemize}
Gênero de plantas americanas.
\section{Lebetona}
\begin{itemize}
\item {Grp. gram.:f.}
\end{itemize}
Túnica de linho, sem mangas, usada pelos solitários da Thebaida.
(Por \textunderscore levitona\textunderscore , do b. lat. \textunderscore levito\textunderscore , \textunderscore levitonis\textunderscore , do rad. de \textunderscore levis\textunderscore )
\section{Leboreiro}
\begin{itemize}
\item {Grp. gram.:adj.}
\end{itemize}
\begin{itemize}
\item {Proveniência:(Do lat. \textunderscore leporarius\textunderscore )}
\end{itemize}
O mesmo que \textunderscore lebreiro\textunderscore .
\section{Lebracho}
\begin{itemize}
\item {Grp. gram.:m.}
\end{itemize}
\begin{itemize}
\item {Utilização:Pop.}
\end{itemize}
\begin{itemize}
\item {Proveniência:(De \textunderscore lebre\textunderscore )}
\end{itemize}
Lebrão novo.
\section{Lebrada}
\begin{itemize}
\item {Grp. gram.:f.}
\end{itemize}
\begin{itemize}
\item {Utilização:Pop.}
\end{itemize}
Guisado de lebre.
\section{Lebrão}
\begin{itemize}
\item {Grp. gram.:m.}
\end{itemize}
Macho da lebre.
\section{Lebre}
\begin{itemize}
\item {Grp. gram.:f.}
\end{itemize}
\begin{itemize}
\item {Utilização:Náut.}
\end{itemize}
\begin{itemize}
\item {Grp. gram.:Loc.}
\end{itemize}
\begin{itemize}
\item {Utilização:pop.}
\end{itemize}
\begin{itemize}
\item {Proveniência:(Do lat. \textunderscore lepus\textunderscore , \textunderscore leporis\textunderscore )}
\end{itemize}
Animal mammífero, da ordem dos roedores.
Constellação austral.
Peixe venenoso.
Peça de madeira, por onde passam os cabos bastardos.
\textunderscore Andar á lebre\textunderscore , não têr dinheiro. Cf. Camillo, \textunderscore Hist. e Sentimentalismo\textunderscore , 164.
\section{Lèbrechina}
\begin{itemize}
\item {Grp. gram.:f.}
\end{itemize}
\begin{itemize}
\item {Utilização:Prov.}
\end{itemize}
\begin{itemize}
\item {Utilização:trasm.}
\end{itemize}
Rapariga magra e leviana; sirigaita.
(Cp. \textunderscore lebracho\textunderscore )
\section{Lebreia}
\begin{itemize}
\item {Grp. gram.:f.}
\end{itemize}
\begin{itemize}
\item {Utilização:Bras. da Baia}
\end{itemize}
Chuvisco, garôa.
\section{Lebreiro}
\begin{itemize}
\item {Grp. gram.:adj.}
\end{itemize}
Que caça lebres.
\section{Lebrel}
\begin{itemize}
\item {Grp. gram.:m.}
\end{itemize}
\begin{itemize}
\item {Proveniência:(Do rad. de \textunderscore lebre\textunderscore )}
\end{itemize}
Cão, próprio para a caça das lebres; galgo.
\section{Lebréo}
\begin{itemize}
\item {Grp. gram.:m.}
\end{itemize}
\begin{itemize}
\item {Proveniência:(Do rad. de \textunderscore lebre\textunderscore )}
\end{itemize}
Cão, próprio para a caça das lebres; galgo.
\section{Lebréu}
\begin{itemize}
\item {Grp. gram.:m.}
\end{itemize}
\begin{itemize}
\item {Proveniência:(Do rad. de \textunderscore lebre\textunderscore )}
\end{itemize}
Cão, próprio para a caça das lebres; galgo.
\section{Lecanantho}
\begin{itemize}
\item {Grp. gram.:m.}
\end{itemize}
\begin{itemize}
\item {Proveniência:(Do gr. \textunderscore lekane\textunderscore  + \textunderscore anthos\textunderscore )}
\end{itemize}
Gênero de arbustos indianos, da fam. das rubiáceas.
\section{Lecananto}
\begin{itemize}
\item {Grp. gram.:m.}
\end{itemize}
\begin{itemize}
\item {Proveniência:(Do gr. \textunderscore lekane\textunderscore  + \textunderscore anthos\textunderscore )}
\end{itemize}
Gênero de arbustos indianos, da fam. das rubiáceas.
\section{Lecanocárpio}
\begin{itemize}
\item {Grp. gram.:m.}
\end{itemize}
Gênero de plantas chenopódeas.
\section{Lecanomancia}
\begin{itemize}
\item {Grp. gram.:f.}
\end{itemize}
\begin{itemize}
\item {Proveniência:(Do gr. \textunderscore lekane\textunderscore , bacia, e \textunderscore manteia\textunderscore , adivinhação)}
\end{itemize}
Antiga e supposta adivinhação, por meio da observação do som, produzido no fundo de uma bacia de água por pedras preciosas, e metaes que alli se lançavam.
\section{Lecanomântico}
\begin{itemize}
\item {Grp. gram.:adj.}
\end{itemize}
Relativo á lecanomancia.
\section{Lecanóreas}
\begin{itemize}
\item {Grp. gram.:f. pl.}
\end{itemize}
\begin{itemize}
\item {Utilização:Bot.}
\end{itemize}
Uma das tríbos dos líchens, segundo Fries.
\section{Leccional}
\begin{itemize}
\item {Grp. gram.:adj.}
\end{itemize}
\begin{itemize}
\item {Utilização:Neol.}
\end{itemize}
\begin{itemize}
\item {Proveniência:(Do lat. \textunderscore lectio\textunderscore )}
\end{itemize}
Relativo a lição.
\section{Leccionando}
\begin{itemize}
\item {Grp. gram.:m.  e  adj.}
\end{itemize}
\begin{itemize}
\item {Proveniência:(De \textunderscore leccionar\textunderscore )}
\end{itemize}
O que recebe lições de alguém; discípulo.
\section{Leccionar}
\begin{itemize}
\item {Grp. gram.:v. t.}
\end{itemize}
\begin{itemize}
\item {Grp. gram.:V. i.}
\end{itemize}
\begin{itemize}
\item {Proveniência:(Do lat. \textunderscore lectio\textunderscore , \textunderscore lectionis\textunderscore )}
\end{itemize}
Dar lições de: \textunderscore leccionar Mathemática\textunderscore .
Dar lições a; ensinar: \textunderscore leccionar dois rapazes\textunderscore .
Sêr leccionista: \textunderscore eu dantes leccionava\textunderscore .
\section{Leccionário}
\begin{itemize}
\item {Grp. gram.:m.}
\end{itemize}
\begin{itemize}
\item {Proveniência:(Do rad. do lat. \textunderscore lectio\textunderscore )}
\end{itemize}
Livro ecclesiástico, que contém as vidas dos santos.
\section{Leccionista}
\begin{itemize}
\item {Grp. gram.:m.}
\end{itemize}
Aquelle que lecciona; professor particular.
\section{Lechetrez}
\begin{itemize}
\item {Grp. gram.:m.}
\end{itemize}
O mesmo que \textunderscore maleiteira\textunderscore .
\section{Lechia}
\begin{itemize}
\item {Grp. gram.:f.}
\end{itemize}
Árvore sapindácea.
Fruto dessa árvore.
\section{Lecídeas}
\begin{itemize}
\item {Grp. gram.:f. pl.}
\end{itemize}
\begin{itemize}
\item {Utilização:Bot.}
\end{itemize}
Sub-ordem de líchens, no méthodo de Fries.
\section{Lecithídeas}
\begin{itemize}
\item {Grp. gram.:f. pl.}
\end{itemize}
\begin{itemize}
\item {Utilização:Bot.}
\end{itemize}
\begin{itemize}
\item {Proveniência:(Do gr. \textunderscore lekithos\textunderscore  + \textunderscore eidos\textunderscore )}
\end{itemize}
Tríbo de plantas myrtáceas, no systema do De-Candolle.
\section{Lecithina}
\begin{itemize}
\item {Grp. gram.:f.}
\end{itemize}
\begin{itemize}
\item {Proveniência:(Do gr. \textunderscore lekithos\textunderscore )}
\end{itemize}
Substância viscosa, contida nos ovos, no cérebro e noutras matérias animaes.
\section{Lecitídeas}
\begin{itemize}
\item {Grp. gram.:f. pl.}
\end{itemize}
\begin{itemize}
\item {Utilização:Bot.}
\end{itemize}
\begin{itemize}
\item {Proveniência:(Do gr. \textunderscore lekithos\textunderscore  + \textunderscore eidos\textunderscore )}
\end{itemize}
Tríbo de plantas myrtáceas, no systema do De-Candolle.
\section{Lecitina}
\begin{itemize}
\item {Grp. gram.:f.}
\end{itemize}
\begin{itemize}
\item {Proveniência:(Do gr. \textunderscore lekithos\textunderscore )}
\end{itemize}
Substância viscosa, contida nos ovos, no cérebro e noutras matérias animaes.
\section{Lécito}
\begin{itemize}
\item {Grp. gram.:m.}
\end{itemize}
\begin{itemize}
\item {Utilização:Ant.}
\end{itemize}
\begin{itemize}
\item {Proveniência:(Lat. \textunderscore lecythus\textunderscore )}
\end{itemize}
Almotolia de azeite.
Entre os Gregos, vaso em fórma de grande garrafa.
\section{Leco}
\begin{itemize}
\item {Grp. gram.:m.}
\end{itemize}
\begin{itemize}
\item {Utilização:Ant.}
\end{itemize}
Lacaio.
\section{Lecôntea}
\begin{itemize}
\item {Grp. gram.:f.}
\end{itemize}
\begin{itemize}
\item {Proveniência:(De \textunderscore Leconte\textunderscore , n. p.)}
\end{itemize}
Gênero de plantas rubiáceas.
\section{Lecóquia}
\begin{itemize}
\item {Grp. gram.:f.}
\end{itemize}
\begin{itemize}
\item {Proveniência:(De \textunderscore Lecoq\textunderscore , n. p.)}
\end{itemize}
Gênero de plantas umbellíferas da ilha de Creta.
\section{Léctica}
\begin{itemize}
\item {Grp. gram.:f.}
\end{itemize}
\begin{itemize}
\item {Utilização:Des.}
\end{itemize}
\begin{itemize}
\item {Proveniência:(Lat. \textunderscore lectica\textunderscore )}
\end{itemize}
Liteira; cadeirinha para transporte.
\section{Lecticário}
\begin{itemize}
\item {Grp. gram.:m.}
\end{itemize}
\begin{itemize}
\item {Utilização:Des.}
\end{itemize}
\begin{itemize}
\item {Proveniência:(Lat. \textunderscore lecticarius\textunderscore )}
\end{itemize}
Conductor de liteira; liteireiro.
\section{Lectícola}
\begin{itemize}
\item {Grp. gram.:adj.}
\end{itemize}
\begin{itemize}
\item {Proveniência:(Do lat. \textunderscore lectum\textunderscore  + \textunderscore colere\textunderscore )}
\end{itemize}
Que habita nos leitos, (falando-se do persevejo).
\section{Lectícula}
\begin{itemize}
\item {Grp. gram.:f.}
\end{itemize}
\begin{itemize}
\item {Utilização:Des.}
\end{itemize}
\begin{itemize}
\item {Proveniência:(Lat. \textunderscore lecticula\textunderscore )}
\end{itemize}
Cadeirinha; poltrona; pequeno leito.
\section{Lectistérnio}
\begin{itemize}
\item {Grp. gram.:m.}
\end{itemize}
\begin{itemize}
\item {Proveniência:(Lat. \textunderscore lectisternium\textunderscore )}
\end{itemize}
Banquete, que os Romanos offereciam aos deuses, collocando as imagens dêstes sôbre leitos na rua, e em frente delles iguarias de toda a espécie.
\section{Lectivo}
\begin{itemize}
\item {Grp. gram.:adj.}
\end{itemize}
\begin{itemize}
\item {Proveniência:(Do lat. \textunderscore lectus\textunderscore )}
\end{itemize}
Relativo a lições ou ao movimento escolar: \textunderscore o anno lectivo de 1913 a 1914\textunderscore .
\section{Lectocéfalo}
\begin{itemize}
\item {Grp. gram.:adj.}
\end{itemize}
\begin{itemize}
\item {Grp. gram.:M.}
\end{itemize}
\begin{itemize}
\item {Proveniência:(Do gr. \textunderscore lektos\textunderscore  + \textunderscore kephale\textunderscore )}
\end{itemize}
Que tem cabeça pequena.
Gêneros de pequenos peixes anguiliformes, que recentemente se verificou serem a larva da enguia.
\section{Lectocéphalo}
\begin{itemize}
\item {Grp. gram.:adj.}
\end{itemize}
\begin{itemize}
\item {Grp. gram.:M.}
\end{itemize}
\begin{itemize}
\item {Proveniência:(Do gr. \textunderscore lektos\textunderscore  + \textunderscore kephale\textunderscore )}
\end{itemize}
Que tem cabeça pequena.
Gêneros de pequenos peixes anguiliformes, que recentemente se verificou serem a larva da enguia.
\section{Lector}
\begin{itemize}
\item {Grp. gram.:m.}
\end{itemize}
\begin{itemize}
\item {Utilização:Ant.}
\end{itemize}
O mesmo que \textunderscore leitor\textunderscore ^1.
\section{Lectorato}
\begin{itemize}
\item {Grp. gram.:m.}
\end{itemize}
\begin{itemize}
\item {Proveniência:(Do lat. \textunderscore lector\textunderscore )}
\end{itemize}
(V.leitorado)
\section{Lécytho}
\begin{itemize}
\item {Grp. gram.:m.}
\end{itemize}
\begin{itemize}
\item {Utilização:Ant.}
\end{itemize}
\begin{itemize}
\item {Proveniência:(Lat. \textunderscore lecythus\textunderscore )}
\end{itemize}
Almotolia de azeite.
Entre os Gregos, vaso em fórma de grande garrafa.
\section{Leda}
\begin{itemize}
\item {Grp. gram.:f.}
\end{itemize}
\begin{itemize}
\item {Proveniência:(De \textunderscore Leda\textunderscore , n. p. myth.)}
\end{itemize}
Gênero de molluscos bivalves.
Planeta telescópico.
\section{Ledamente}
\begin{itemize}
\item {fónica:lê}
\end{itemize}
\begin{itemize}
\item {Grp. gram.:adv.}
\end{itemize}
De modo ledo.
Alegremente.
\section{Lediça}
\begin{itemize}
\item {Grp. gram.:f.}
\end{itemize}
\begin{itemize}
\item {Utilização:Ant.}
\end{itemize}
O mesmo que \textunderscore ledice\textunderscore . Cf. Frei Fortun., \textunderscore Inéd.\textunderscore , 309.
\section{Ledice}
\begin{itemize}
\item {Grp. gram.:f.}
\end{itemize}
\begin{itemize}
\item {Grp. gram.:Pl.}
\end{itemize}
Qualidade de ledo.
Facécias; galantarias.
\section{Ledo}
\begin{itemize}
\item {fónica:lê}
\end{itemize}
\begin{itemize}
\item {Grp. gram.:adj.}
\end{itemize}
\begin{itemize}
\item {Proveniência:(Lat. \textunderscore laetus\textunderscore )}
\end{itemize}
Risonho; alegre; jubiloso.
\section{Ledor}
\begin{itemize}
\item {Grp. gram.:m.  e  adj.}
\end{itemize}
\begin{itemize}
\item {Proveniência:(Lat. \textunderscore lector\textunderscore )}
\end{itemize}
O que lê.
\section{Leelite}
\begin{itemize}
\item {Grp. gram.:f.}
\end{itemize}
Mineral da Suécia, avermelhado e duro.
\section{Leérsia}
\begin{itemize}
\item {Grp. gram.:f.}
\end{itemize}
Gênero de plantas gramíneas.
\section{Leflíngia}
\begin{itemize}
\item {Grp. gram.:f.}
\end{itemize}
\begin{itemize}
\item {Proveniência:(De \textunderscore Lefling\textunderscore , n. p.)}
\end{itemize}
Gênero de plantas gramíneas.
\section{Legação}
\begin{itemize}
\item {Grp. gram.:m.}
\end{itemize}
Salsaparrilha, conhecida também por \textunderscore salsaparrilha do reino\textunderscore .
\section{Legação}
\begin{itemize}
\item {Grp. gram.:f.}
\end{itemize}
\begin{itemize}
\item {Proveniência:(Lat. \textunderscore legatio\textunderscore )}
\end{itemize}
Residência de um diplomata estrangeiro.
Missão diplomática.
Repartição, dirigida por um diplomata estrangeiro.
Tempo que dura uma legacia.
\section{Legacia}
\begin{itemize}
\item {Grp. gram.:f.}
\end{itemize}
\begin{itemize}
\item {Proveniência:(Do rad. do lat. \textunderscore legatus\textunderscore )}
\end{itemize}
Dignidade ou cargo de um legado.
\section{Legado}
\begin{itemize}
\item {Grp. gram.:m.  e  adj.}
\end{itemize}
\begin{itemize}
\item {Proveniência:(De \textunderscore legar\textunderscore )}
\end{itemize}
Embaixador ou enviado, que trata negócios do seu país em país estrangeiro.
Núncio pontifício.
\section{Legado}
\begin{itemize}
\item {Grp. gram.:m.}
\end{itemize}
\begin{itemize}
\item {Proveniência:(Lat. \textunderscore legatum\textunderscore )}
\end{itemize}
Aquillo que se deixa por testamento a quem não é o principal herdeiro.
\section{Legal}
\begin{itemize}
\item {Grp. gram.:adj.}
\end{itemize}
\begin{itemize}
\item {Proveniência:(Lat. \textunderscore legalis\textunderscore )}
\end{itemize}
Conforme com a lei: \textunderscore acto legal\textunderscore .
Relativo a lei: \textunderscore disposições legaes\textunderscore .
\section{Legalidade}
\begin{itemize}
\item {Grp. gram.:f.}
\end{itemize}
\begin{itemize}
\item {Proveniência:(Lat. \textunderscore legalitas\textunderscore )}
\end{itemize}
Qualidade daquillo que é legal.
\section{Legalista}
\begin{itemize}
\item {Grp. gram.:adj.}
\end{itemize}
\begin{itemize}
\item {Grp. gram.:M.}
\end{itemize}
\begin{itemize}
\item {Proveniência:(De \textunderscore legal\textunderscore )}
\end{itemize}
Relativo ás disposições legaes.
Aquelle que pugna pela observância das leis.
\section{Legalização}
\begin{itemize}
\item {Grp. gram.:f.}
\end{itemize}
Acto ou effeito de legalizar.
\section{Legalizar}
\begin{itemize}
\item {Grp. gram.:v. t.}
\end{itemize}
Tornar legal; authenticar.
\section{Legalmente}
\begin{itemize}
\item {Grp. gram.:adv.}
\end{itemize}
De modo legal; em harmonia com a lei.
\section{Legame}
\begin{itemize}
\item {Grp. gram.:m.}
\end{itemize}
\begin{itemize}
\item {Utilização:Ant.}
\end{itemize}
O mesmo que \textunderscore legado\textunderscore ^2.
Herança.
\section{Legante}
\begin{itemize}
\item {Grp. gram.:m.}
\end{itemize}
\begin{itemize}
\item {Utilização:Gír.}
\end{itemize}
Pistola.
\section{Legão}
\begin{itemize}
\item {Grp. gram.:m.}
\end{itemize}
\begin{itemize}
\item {Utilização:Prov.}
\end{itemize}
\begin{itemize}
\item {Utilização:minh.}
\end{itemize}
O mesmo que \textunderscore enxada\textunderscore .
(Cast. \textunderscore legón\textunderscore )
\section{Legar}
\begin{itemize}
\item {Grp. gram.:v. t.}
\end{itemize}
\begin{itemize}
\item {Proveniência:(Lat. \textunderscore legare\textunderscore )}
\end{itemize}
Enviar como legado.
Transmittir em testamento ou por herança, não sendo a herdeiro forçado: \textunderscore legar um prédio ao afilhado\textunderscore .
\section{Legatário}
\begin{itemize}
\item {Grp. gram.:m.}
\end{itemize}
\begin{itemize}
\item {Proveniência:(Lat. \textunderscore legatarius\textunderscore )}
\end{itemize}
Aquelle que foi contemplado com legado testamentario.
\section{Legatina}
\begin{itemize}
\item {Grp. gram.:f.}
\end{itemize}
Dizem os diccion. que é um estôfo de seda e lan. Vi porém algures que é êrro de cópia ou de composição e que a palavra é \textunderscore legatura\textunderscore .
\section{Legatário}
\begin{itemize}
\item {Grp. gram.:adj.}
\end{itemize}
Relativo a legados.
Que envolve legado.
(Cp. lat. \textunderscore legator\textunderscore )
\section{Legenda}
\begin{itemize}
\item {Grp. gram.:f.}
\end{itemize}
\begin{itemize}
\item {Proveniência:(Lat. \textunderscore legenda\textunderscore )}
\end{itemize}
Inscripção; distico; letreiro.
Vidas de santos.
\section{Legendário}
\begin{itemize}
\item {Grp. gram.:adj.}
\end{itemize}
\begin{itemize}
\item {Grp. gram.:M.}
\end{itemize}
Relativo a legenda.
Aquelle que escreve legendas.
Collecção de vidas dos santos.
\section{Legião}
\begin{itemize}
\item {Grp. gram.:f.}
\end{itemize}
\begin{itemize}
\item {Utilização:Fig.}
\end{itemize}
\begin{itemize}
\item {Proveniência:(Lat. \textunderscore legio\textunderscore )}
\end{itemize}
Corpo da antiga milícia romana, composto de infantaria e de alguma cavallaria.
Divisão de exército.
Grande quantidade de gente.
Grande quantidade de anjos ou demónios.
Grande quantidade de coisas.
\section{Legiferação}
\begin{itemize}
\item {Grp. gram.:f.}
\end{itemize}
Acto de legiferar.
\section{Legiferar}
\begin{itemize}
\item {Grp. gram.:v. i.}
\end{itemize}
\begin{itemize}
\item {Utilização:Fam.}
\end{itemize}
\begin{itemize}
\item {Utilização:Neol.}
\end{itemize}
\begin{itemize}
\item {Proveniência:(Do lat. \textunderscore lex\textunderscore , \textunderscore legis\textunderscore  + \textunderscore ferre\textunderscore )}
\end{itemize}
Fazer leis, legislar.
\section{Legífero}
\begin{itemize}
\item {Grp. gram.:m.}
\end{itemize}
Aquelle que faz leis ou que legisla:«\textunderscore desplante de pseudos legíferos da língua de Camões\textunderscore ». Pacheco, \textunderscore Promptuário\textunderscore , 2.
\section{Legionário}
\begin{itemize}
\item {Grp. gram.:adj.}
\end{itemize}
\begin{itemize}
\item {Grp. gram.:M.}
\end{itemize}
\begin{itemize}
\item {Proveniência:(Lat. \textunderscore legionarius\textunderscore )}
\end{itemize}
Relativo a legião.
Soldado de uma legião.
\section{Legislação}
\begin{itemize}
\item {Grp. gram.:f.}
\end{itemize}
\begin{itemize}
\item {Proveniência:(Lat. \textunderscore legislatio\textunderscore )}
\end{itemize}
Complexo de leis.
Estudo dos actos legislativos.
\section{Legislador}
\begin{itemize}
\item {Grp. gram.:adj.}
\end{itemize}
\begin{itemize}
\item {Grp. gram.:M.}
\end{itemize}
\begin{itemize}
\item {Proveniência:(Lat. \textunderscore legislator\textunderscore )}
\end{itemize}
Que legisla.
Que explica as leis.
Aquelle que faz leis.
Membro da câmara legislativa.
\section{Legislar}
\begin{itemize}
\item {Grp. gram.:v. t.}
\end{itemize}
\begin{itemize}
\item {Grp. gram.:V. i.}
\end{itemize}
\begin{itemize}
\item {Proveniência:(Do rad. de \textunderscore legislador\textunderscore )}
\end{itemize}
Ordenar ou preceituar por lei: \textunderscore legislar incompatibilidades\textunderscore .
Fazer leis.
\section{Legislativamente}
\begin{itemize}
\item {Grp. gram.:adv.}
\end{itemize}
De modo legislativo.
Pela fórma seguida na feitura das leis.
\section{Legislativo}
\begin{itemize}
\item {Grp. gram.:adj.}
\end{itemize}
\begin{itemize}
\item {Proveniência:(De \textunderscore legislar\textunderscore )}
\end{itemize}
Relativo ao poder de legislar.
Que diz respeito á legislação.
\section{Legislável}
\begin{itemize}
\item {Grp. gram.:adj.}
\end{itemize}
Que so póde legislar ou decretar.
\section{Legislatório}
\begin{itemize}
\item {Grp. gram.:adj.}
\end{itemize}
\begin{itemize}
\item {Proveniência:(De \textunderscore legislar\textunderscore )}
\end{itemize}
Que obriga como lei.
Relativo a legislação.
\section{Legislatura}
\begin{itemize}
\item {Grp. gram.:f.}
\end{itemize}
\begin{itemize}
\item {Proveniência:(De \textunderscore legislar\textunderscore )}
\end{itemize}
Reunião das entidades que têm o poder de legislar.
Espaço de tempo, em que se exercem os poderes de uma assembleia legislativa.
\section{Legisperito}
\begin{itemize}
\item {Grp. gram.:m.}
\end{itemize}
\begin{itemize}
\item {Proveniência:(Lat. \textunderscore legisperitus\textunderscore )}
\end{itemize}
Aquelle que é perito em leis.
\section{Legista}
\begin{itemize}
\item {Grp. gram.:m.}
\end{itemize}
\begin{itemize}
\item {Proveniência:(Do lat. \textunderscore lex\textunderscore , \textunderscore legis\textunderscore )}
\end{itemize}
O mesmo que \textunderscore legisperito\textunderscore ; jurisconsulto.
\section{Legítima}
\begin{itemize}
\item {Grp. gram.:f.}
\end{itemize}
\begin{itemize}
\item {Proveniência:(De \textunderscore legitimo\textunderscore )}
\end{itemize}
Parte da herança, de que o testador não póde dispor livremente, por pertencer legitimamente a herdeiro ascendente ou descendente.
Uma das divisões da salina.
\section{Legitimação}
\begin{itemize}
\item {Grp. gram.:f.}
\end{itemize}
Acto ou effeito do legitimar.
\section{Legitimado}
\begin{itemize}
\item {Grp. gram.:adj.}
\end{itemize}
\begin{itemize}
\item {Grp. gram.:M.}
\end{itemize}
\begin{itemize}
\item {Proveniência:(De \textunderscore legitimar\textunderscore )}
\end{itemize}
Que se tornou legítimo.
Filho natural, que o matrimónio dos pais legitimou.
\section{Legitimador}
\begin{itemize}
\item {Grp. gram.:adj.}
\end{itemize}
\begin{itemize}
\item {Grp. gram.:M.}
\end{itemize}
Que legitíma.
Aquelle que legitíma.
\section{Legitimamente}
\begin{itemize}
\item {Grp. gram.:adv.}
\end{itemize}
De modo legítimo.
Lealmente.
\section{Legitimar}
\begin{itemize}
\item {Grp. gram.:v. t.}
\end{itemize}
Tornar legítimo.
Reconhecer como legítimo ou authêntico.
Justificar.
Habilitar para certos actos ou para o gôzo de certos direitos.
\section{Legitimidade}
\begin{itemize}
\item {Grp. gram.:f.}
\end{itemize}
Qualidade de legítimo.
Estado daquelle ou daquilo que se tornou legítimo.
Estado ou qualidade daquillo que está de acôrdo com a razão, ou com a justiça, ou com a lei.
Illação ou conclusão lógica.
Direito de succeder a um monarcha, pelo princípio da primogenitura, ou pela exclusão legal do primogênito.
Doutrina política dos legitimistas.
O partido dos legitimistas.
\section{Legitimista}
\begin{itemize}
\item {Grp. gram.:adj.}
\end{itemize}
\begin{itemize}
\item {Grp. gram.:M.  e  adj.}
\end{itemize}
\begin{itemize}
\item {Proveniência:(De \textunderscore legítimo\textunderscore )}
\end{itemize}
Relativo á legitimidade.
O que advoga o direito de D. Miguel de Bragança ou dos seus descendentes ao throno de Portugal.
\section{Legítimo}
\begin{itemize}
\item {Grp. gram.:adj.}
\end{itemize}
\begin{itemize}
\item {Proveniência:(Lat. \textunderscore legitimus\textunderscore )}
\end{itemize}
Conforme á lei; legal.
Que procede de um matrimónio, (falando-se de filhos).
Válido perante a lei.
Justo.
Racional.
Genuíno: \textunderscore legítimo vinho do Pôrto\textunderscore .
Puro.
Verdadeiro.
Concludente.
\section{Legível}
\begin{itemize}
\item {Grp. gram.:adj.}
\end{itemize}
\begin{itemize}
\item {Proveniência:(Lat. \textunderscore legibilis\textunderscore )}
\end{itemize}
Que se póde ler.
Que está escrito em caracteres bem visíveis e distintos.
\section{Legivelmente}
\begin{itemize}
\item {Grp. gram.:adj.}
\end{itemize}
De modo legível.
\section{Légoa}
\begin{itemize}
\item {Grp. gram.:f.}
\end{itemize}
\begin{itemize}
\item {Utilização:Fam.}
\end{itemize}
\begin{itemize}
\item {Grp. gram.:Loc. adv.}
\end{itemize}
\begin{itemize}
\item {Proveniência:(Do lat. \textunderscore leuca\textunderscore .)}
\end{itemize}
Medida itenerária, equivalente, no systema métrico, a 5 kilómetros.
Grande distância.
\textunderscore Légua da Póvoa\textunderscore , grande distância.
\textunderscore Á légoa\textunderscore , até de longe; perfeitamente, nitidamente.
Apressadamente.
\textunderscore De légoa e meia\textunderscore , muito grande, muito extenso: \textunderscore fez um discurso de légoa e meia\textunderscore .
\section{Legra}
\begin{itemize}
\item {Grp. gram.:f.}
\end{itemize}
\begin{itemize}
\item {Utilização:Prov.}
\end{itemize}
\begin{itemize}
\item {Utilização:alent.}
\end{itemize}
Instrumento, para observar as fracturas do crânio.
Lâmina curva e cortante, para escavar madeira, e com a qual os cabreiros fazem colheres de pau e outros objectos.
(Cast. \textunderscore legra\textunderscore )
\section{Legração}
\begin{itemize}
\item {Grp. gram.:f.}
\end{itemize}
Acto de legrar.
\section{Legradura}
\begin{itemize}
\item {Grp. gram.:f.}
\end{itemize}
O mesmo que \textunderscore legração\textunderscore .
Acto de raspar ou limpar com goiva apropriada os ossos cariados ou fracturados.
\section{Legrar}
\begin{itemize}
\item {Grp. gram.:v. t.}
\end{itemize}
Operar ou examinar com a legra.
\section{Légua}
\begin{itemize}
\item {Grp. gram.:f.}
\end{itemize}
\begin{itemize}
\item {Utilização:Fam.}
\end{itemize}
\begin{itemize}
\item {Grp. gram.:Loc. adv.}
\end{itemize}
\begin{itemize}
\item {Proveniência:(Do lat. \textunderscore leuca\textunderscore .)}
\end{itemize}
Medida itenerária, equivalente, no sistema métrico, a 5 kilómetros.
Grande distância.
\textunderscore Légua da Póvoa\textunderscore , grande distância.
\textunderscore Á légua\textunderscore , até de longe; perfeitamente, nitidamente.
Apressadamente.
\textunderscore De légua e meia\textunderscore , muito grande, muito extenso: \textunderscore fez um discurso de légua e meia\textunderscore .
\section{Leguleio}
\begin{itemize}
\item {Grp. gram.:m.}
\end{itemize}
\begin{itemize}
\item {Proveniência:(Lat. \textunderscore leguleius\textunderscore )}
\end{itemize}
Aquelle que attende servilmente á letra da lei.
Advogado rábula, chicaneiro.
\section{Legulejo}
\begin{itemize}
\item {Grp. gram.:m.}
\end{itemize}
O mesmo que \textunderscore leguleio\textunderscore . Cf. Rebello, no \textunderscore Panorama\textunderscore , IX, 357.
\section{Legulismo}
\begin{itemize}
\item {Grp. gram.:m.}
\end{itemize}
Qualidade de leguleio.
Systema dos falsos jurisconsultos, que só attendem á letra das leis, descurando-lhes o espírito e o alcance. Cf. R. de Brito, \textunderscore Philos. do Dir.\textunderscore , 380.
\section{Legume}
\begin{itemize}
\item {Grp. gram.:m.}
\end{itemize}
\begin{itemize}
\item {Proveniência:(Lat. \textunderscore legumen\textunderscore )}
\end{itemize}
Fruto comestível das plantas leguminosas.
Producto de horticultura.
Hortaliça.
\section{Legumeiro}
\begin{itemize}
\item {Grp. gram.:adj.}
\end{itemize}
Que tem legumes; em que crescem legumes.
\section{Legumilha}
\begin{itemize}
\item {Grp. gram.:f.}
\end{itemize}
\begin{itemize}
\item {Utilização:Ant.}
\end{itemize}
O mesmo que \textunderscore legume\textunderscore .
\section{Legumina}
\begin{itemize}
\item {Grp. gram.:f.}
\end{itemize}
\begin{itemize}
\item {Proveniência:(De \textunderscore legume\textunderscore )}
\end{itemize}
Princípio, extrahido das sementes de várias plantas leguminosas.
\section{Leguminário}
\begin{itemize}
\item {Grp. gram.:adj.}
\end{itemize}
\begin{itemize}
\item {Proveniência:(Lat. \textunderscore legumenarius\textunderscore )}
\end{itemize}
Relativo a legume.
\section{Leguminiforme}
\begin{itemize}
\item {Grp. gram.:adj.}
\end{itemize}
\begin{itemize}
\item {Utilização:Bot.}
\end{itemize}
\begin{itemize}
\item {Proveniência:(Do lat. \textunderscore legumen\textunderscore  + \textunderscore forma\textunderscore )}
\end{itemize}
Diz-se dos órgãos vegetaes, que se parecem, mais ou menos, com um legume.
\section{Leguminívoro}
\begin{itemize}
\item {Grp. gram.:adj.}
\end{itemize}
\begin{itemize}
\item {Proveniência:(Do lat. \textunderscore legumen\textunderscore  + \textunderscore vorare\textunderscore )}
\end{itemize}
Que se alimenta de legumes.
\section{Leguminosas}
\begin{itemize}
\item {Grp. gram.:f. pl.}
\end{itemize}
\begin{itemize}
\item {Proveniência:(De \textunderscore leguminoso\textunderscore )}
\end{itemize}
Ordem de plantas, caracterizada pela fructificação em vagens.
\section{Leguminoso}
\begin{itemize}
\item {Grp. gram.:adj.}
\end{itemize}
\begin{itemize}
\item {Proveniência:(Lat. \textunderscore leguminosus\textunderscore )}
\end{itemize}
Que fructifica em vagens.
\section{Legumista}
\begin{itemize}
\item {Grp. gram.:m.}
\end{itemize}
\begin{itemize}
\item {Proveniência:(De \textunderscore legume\textunderscore )}
\end{itemize}
Aquelle que trata especialmente de plantas leguminosas.
\section{Lèguória}
\begin{itemize}
\item {fónica:gu-o}
\end{itemize}
\begin{itemize}
\item {Grp. gram.:f.}
\end{itemize}
\begin{itemize}
\item {Utilização:Prov.}
\end{itemize}
\begin{itemize}
\item {Proveniência:(Do rad. de \textunderscore légua\textunderscore )}
\end{itemize}
Pequena légua.
\section{Lei}
\begin{itemize}
\item {Grp. gram.:f.}
\end{itemize}
\begin{itemize}
\item {Utilização:T. de Pare -de-Coira}
\end{itemize}
\begin{itemize}
\item {Utilização:des.}
\end{itemize}
\begin{itemize}
\item {Grp. gram.:Loc. prep.}
\end{itemize}
\begin{itemize}
\item {Proveniência:(Do lat. \textunderscore lex\textunderscore )}
\end{itemize}
Preceito, que deriva da autoridade soberana.
Prescripção do poder legislativo.
Religião.
Regra.
Norma devida.
Relação constante, entre phenómenos, ou entre phases de um só phenómeno.
Estima; affeição: \textunderscore tenho lei á vida\textunderscore .
\textunderscore Á lei de\textunderscore , segundo a regra ou costume:«\textunderscore começou a viver á lei da nobreza\textunderscore ». Camillo, \textunderscore Retr. de Ricard.\textunderscore , 17.
\section{Leia}
\begin{itemize}
\item {Grp. gram.:f.}
\end{itemize}
\begin{itemize}
\item {Utilização:T. do Fundão}
\end{itemize}
Corda delgada.
(Por \textunderscore lia\textunderscore , de \textunderscore liar\textunderscore ?)
\section{Leiblínia}
\begin{itemize}
\item {Grp. gram.:f.}
\end{itemize}
Gênero de algas.
\section{Leibnitzianismo}
\begin{itemize}
\item {Grp. gram.:m.}
\end{itemize}
\begin{itemize}
\item {Proveniência:(De \textunderscore Leibnitz\textunderscore , n. p.)}
\end{itemize}
Philosophia idealista de Leibnitz, caracterizada pelas mónadas e pela harmonia preestável.
\section{Leibnitzianista}
\begin{itemize}
\item {Grp. gram.:m.}
\end{itemize}
Sectário do leibnitzianismo.
\section{Leicenço}
\begin{itemize}
\item {Grp. gram.:m.}
\end{itemize}
O mesmo que \textunderscore furúnculo\textunderscore .
(Relaciona-se com o lat. \textunderscore lacessitio\textunderscore , de \textunderscore lacessere\textunderscore ?)
\section{Leicéster}
\begin{itemize}
\item {Grp. gram.:m.}
\end{itemize}
(V.líster)
\section{Leigaça}
\begin{itemize}
\item {Grp. gram.:m.  e  f.}
\end{itemize}
Pessôa leiga ou estranha a certos conhecimentos. Cf. \textunderscore Agostinheida\textunderscore , 106.
(Cp. \textunderscore leigaço\textunderscore )
\section{Leigaço}
\begin{itemize}
\item {Grp. gram.:m.}
\end{itemize}
\begin{itemize}
\item {Utilização:Deprec.}
\end{itemize}
Aquelle que é muito leigo em certos assumptos; ignorantão. Cf. Filinto, IX, 219.
\section{Leigal}
\begin{itemize}
\item {Grp. gram.:adj.}
\end{itemize}
Relativo a leigo; laical.
\section{Leigar}
\begin{itemize}
\item {Grp. gram.:v. t.}
\end{itemize}
\begin{itemize}
\item {Utilização:Des.}
\end{itemize}
Tornar leigo.
\section{Leigarraço}
\begin{itemize}
\item {Grp. gram.:m.}
\end{itemize}
O mesmo que \textunderscore leigaço\textunderscore . Cf. Castilho, \textunderscore Colloq. Ald.\textunderscore , 184.
\section{Leigarrão}
\begin{itemize}
\item {Grp. gram.:m.}
\end{itemize}
O mesmo que \textunderscore leigaço\textunderscore . Cf. Castilho, \textunderscore Méd. á Fôrça\textunderscore , 17.
\section{Leigarraz}
\begin{itemize}
\item {Grp. gram.:m.}
\end{itemize}
O mesmo que \textunderscore leigaço\textunderscore :«\textunderscore verdadeiro typo de leigarraz, estúpido e servil\textunderscore ». Garrett, \textunderscore Arco de Sant.\textunderscore , I, 138.
\section{Leigo}
\begin{itemize}
\item {Grp. gram.:m.}
\end{itemize}
\begin{itemize}
\item {Grp. gram.:Adj.}
\end{itemize}
\begin{itemize}
\item {Utilização:Fig.}
\end{itemize}
\begin{itemize}
\item {Proveniência:(Do lat. \textunderscore laicus\textunderscore )}
\end{itemize}
Aquelle que não tem ordens sacras.
Que não recebeu Ordens sacras, Ordens ecclesiasticas.
Laical.
Estranho, alheio a um assumpto: \textunderscore eu cá sou leigo em Astronomia\textunderscore .
\section{Leiguice}
\begin{itemize}
\item {Grp. gram.:f.}
\end{itemize}
Dito ou acto de leigo.
\section{Leilão}
\begin{itemize}
\item {Grp. gram.:m.}
\end{itemize}
Venda pública de objectos, que se entregam a quem offerece maior preço ou lanço.
Almoéda.
Hasta pública.
(Talvez do ár. \textunderscore al-ilon\textunderscore , annúncio)
\section{Leiloamento}
\begin{itemize}
\item {Grp. gram.:m.}
\end{itemize}
Acto de leiloar.
Venda em leilão: almoéda. Cf. Camillo, \textunderscore Perfil do M. de Pombal\textunderscore , 99.
\section{Leiloar}
\begin{itemize}
\item {Grp. gram.:v. t.}
\end{itemize}
Pôr em leilão, em praça ou em almoéda.
Apregoar em leilão. Cf. Camillo, \textunderscore Brasileira\textunderscore , 237.
\section{Leiloeiro}
\begin{itemize}
\item {Grp. gram.:m.}
\end{itemize}
Pregoeiro em leilões.
Organizador de leilões.
\section{Leimonitos}
\begin{itemize}
\item {Grp. gram.:m. pl.}
\end{itemize}
Gênero de aves, da ordem dos pássaros.
\section{Leino}
\begin{itemize}
\item {Grp. gram.:adj.}
\end{itemize}
\begin{itemize}
\item {Utilização:Prov.}
\end{itemize}
\begin{itemize}
\item {Utilização:trasm.}
\end{itemize}
Bonito, catita.
\section{Leira}
\begin{itemize}
\item {Grp. gram.:f.}
\end{itemize}
\begin{itemize}
\item {Proveniência:(Do b. lat. \textunderscore larea\textunderscore )}
\end{itemize}
Sulco na terra, para se deitar a semente.
Geira.
Alfobre.
Belga.
Elevação de terra entre sulcos.
\section{Leira}
\begin{itemize}
\item {Grp. gram.:f.}
\end{itemize}
Casta de uva branca algarvia.
\section{Leira}
\begin{itemize}
\item {Grp. gram.:f.}
\end{itemize}
\begin{itemize}
\item {Utilização:Prov.}
\end{itemize}
\begin{itemize}
\item {Utilização:minh.}
\end{itemize}
Mania, telha.
\section{Leirã}
\begin{itemize}
\item {Grp. gram.:f.}
\end{itemize}
\begin{itemize}
\item {Proveniência:(De \textunderscore Leiria\textunderscore , n. p.? De \textunderscore leira\textunderscore ^1?)}
\end{itemize}
Variedade de uva branca.
\section{Leiran}
\begin{itemize}
\item {Grp. gram.:f.}
\end{itemize}
\begin{itemize}
\item {Proveniência:(De \textunderscore Leiria\textunderscore , n. p.? De \textunderscore leira\textunderscore ^1?)}
\end{itemize}
Variedade de uva branca.
\section{Leirão}
\begin{itemize}
\item {Grp. gram.:m.}
\end{itemize}
\begin{itemize}
\item {Utilização:Prov.}
\end{itemize}
\begin{itemize}
\item {Utilização:trasm.}
\end{itemize}
\begin{itemize}
\item {Utilização:Prov.}
\end{itemize}
\begin{itemize}
\item {Utilização:minh.}
\end{itemize}
Rato grande.
Castanheiro muito alto.
\section{Leirão}
\begin{itemize}
\item {Grp. gram.:m.}
\end{itemize}
\begin{itemize}
\item {Utilização:Prov.}
\end{itemize}
\begin{itemize}
\item {Utilização:beir.}
\end{itemize}
Leira pequena; espaço de terreno cultivado.
\section{Leirar}
\begin{itemize}
\item {Grp. gram.:v. t.}
\end{itemize}
Dividir em leiras; lavrar, formando leiras:«\textunderscore o Formoso e o Galante iam leirando o chão\textunderscore ». Júl. Castilho, \textunderscore Ermitério\textunderscore , 108.
\section{Leiria}
\begin{itemize}
\item {Grp. gram.:adj. f.}
\end{itemize}
O mesmo que \textunderscore leiriôa\textunderscore .
\section{Leirião}
\begin{itemize}
\item {Grp. gram.:m.}
\end{itemize}
\begin{itemize}
\item {Utilização:T. de Alcobaça}
\end{itemize}
Homem de Leiria.
\section{Leiriense}
\begin{itemize}
\item {Grp. gram.:adj.}
\end{itemize}
\begin{itemize}
\item {Grp. gram.:M.}
\end{itemize}
Relativo a Leiria.
Habitante de Leiria.
\section{Leiriôa}
\begin{itemize}
\item {Grp. gram.:adj. f.}
\end{itemize}
\begin{itemize}
\item {Proveniência:(De \textunderscore Leiria\textunderscore , n. p.)}
\end{itemize}
Diz-se de uma espécie de maçan.
\section{Leiroto}
\begin{itemize}
\item {fónica:leirô}
\end{itemize}
\begin{itemize}
\item {Grp. gram.:m.}
\end{itemize}
\begin{itemize}
\item {Utilização:Prov.}
\end{itemize}
\begin{itemize}
\item {Utilização:minh.}
\end{itemize}
Pequena leira ou coirela.
\section{Leita}
\begin{itemize}
\item {Grp. gram.:f.}
\end{itemize}
\begin{itemize}
\item {Utilização:Pesc.}
\end{itemize}
\begin{itemize}
\item {Proveniência:(De \textunderscore leite\textunderscore )}
\end{itemize}
Ova que, em vez de estructura granulosa, a tem leitosa e molle.
\section{Leitado}
\begin{itemize}
\item {Grp. gram.:adj.}
\end{itemize}
Que cria suco leitoso.
(Cp. \textunderscore leitar\textunderscore ^2)
\section{Leital}
\begin{itemize}
\item {Grp. gram.:adj.}
\end{itemize}
\begin{itemize}
\item {Utilização:Prov.}
\end{itemize}
\begin{itemize}
\item {Proveniência:(De \textunderscore leite\textunderscore )}
\end{itemize}
Diz-se de uma pedra ou penedo, em que as mulheres crendeiras vão chupar e dar três voltas em roda delle, para terem leite.
\section{Leitão}
\begin{itemize}
\item {Grp. gram.:m.}
\end{itemize}
\begin{itemize}
\item {Grp. gram.:Loc.}
\end{itemize}
\begin{itemize}
\item {Utilização:minh}
\end{itemize}
\begin{itemize}
\item {Proveniência:(Do b. lat. \textunderscore lacto\textunderscore )}
\end{itemize}
Bácoro, em-quanto mama.
Peixe plagióstomo, alvacento.
\textunderscore Em leitão\textunderscore , nu, em pêlo.
\section{Leitão}
\begin{itemize}
\item {Grp. gram.:m.}
\end{itemize}
\begin{itemize}
\item {Utilização:Prov.}
\end{itemize}
\begin{itemize}
\item {Utilização:trasm.}
\end{itemize}
\begin{itemize}
\item {Proveniência:(De \textunderscore leito\textunderscore ?)}
\end{itemize}
Pedaço de terra, que os cavadores deixam em cru, dando-lhe apenas umas cavadelas superficiaes, para encobrir o defeito do trabalho.
\section{Leitar}
\begin{itemize}
\item {Grp. gram.:adj.}
\end{itemize}
Que tem côr de leite.
\section{Leitar}
\begin{itemize}
\item {Grp. gram.:v. i.}
\end{itemize}
\begin{itemize}
\item {Proveniência:(De \textunderscore leite\textunderscore )}
\end{itemize}
Criar succo leitoso.
\section{Leitara}
\begin{itemize}
\item {Grp. gram.:f.}
\end{itemize}
O mesmo que \textunderscore leita\textunderscore .
\section{Leitarega}
\begin{itemize}
\item {Grp. gram.:f.}
\end{itemize}
\begin{itemize}
\item {Utilização:Prov.}
\end{itemize}
\begin{itemize}
\item {Utilização:trasm.}
\end{itemize}
Planta, de suco leitoso e purgativo.
O mesmo que \textunderscore leitariga\textunderscore ?
\section{Leitaria}
\begin{itemize}
\item {Grp. gram.:f.}
\end{itemize}
Edificação annexa a uma vacaria, para depósito de leite.
Estabelecimento de lacticínios.
\section{Leitariga}
\begin{itemize}
\item {Grp. gram.:f.}
\end{itemize}
(V.maleiteira)
\section{Leitazona}
\begin{itemize}
\item {Grp. gram.:f.}
\end{itemize}
\begin{itemize}
\item {Utilização:Prov.}
\end{itemize}
\begin{itemize}
\item {Utilização:trasm.}
\end{itemize}
\begin{itemize}
\item {Proveniência:(De \textunderscore leitão\textunderscore ?)}
\end{itemize}
O mesmo que \textunderscore lebre\textunderscore .
\section{Leite}
\begin{itemize}
\item {Grp. gram.:m.}
\end{itemize}
\begin{itemize}
\item {Utilização:Bras}
\end{itemize}
\begin{itemize}
\item {Proveniência:(Do lat. \textunderscore lac\textunderscore , \textunderscore lactis\textunderscore )}
\end{itemize}
Líquido opaco, branco e de sabor adocicado, produzido pelas glândulas mammárias da mulher e das fêmeas dos animaes mammíferos.
\textunderscore Leite de gallinha\textunderscore , planta liliácea.
\textunderscore Árvore de leite\textunderscore , árvore urticácea da América.
\textunderscore Leite de cal\textunderscore , cal, preparada para caiação de branco.
\textunderscore Leite de côco\textunderscore , sumo da amêndoa de côco, preparado como adubo culinário.
\textunderscore Irmão de leite\textunderscore , irmão collaço.
\section{Leitegada}
\begin{itemize}
\item {Grp. gram.:f.}
\end{itemize}
\begin{itemize}
\item {Utilização:Pop.}
\end{itemize}
\begin{itemize}
\item {Proveniência:(Do rad. de \textunderscore leitão\textunderscore )}
\end{itemize}
Conjunto dos leitões, que nasceram de um parto.
\section{Leiteira}
\begin{itemize}
\item {Grp. gram.:f.}
\end{itemize}
\begin{itemize}
\item {Proveniência:(De \textunderscore leiteiro\textunderscore )}
\end{itemize}
Vendedora de leite.
Vaso, em que se leva o leite á mesa.
\section{Leiteiro}
\begin{itemize}
\item {Grp. gram.:adj.}
\end{itemize}
\begin{itemize}
\item {Grp. gram.:M.}
\end{itemize}
Que produz leite: \textunderscore vacas leiteiras\textunderscore .
Próprio para conter leite: \textunderscore vaso leiteiro\textunderscore .
Vendedor de leite.
\section{Leitento}
\begin{itemize}
\item {Grp. gram.:adj.}
\end{itemize}
O mesmo que \textunderscore lácteo\textunderscore .
Que deita leite; que segrega líquido semelhante a leite.
\section{Leitiga}
\begin{itemize}
\item {Grp. gram.:f.}
\end{itemize}
\begin{itemize}
\item {Utilização:Des.}
\end{itemize}
\begin{itemize}
\item {Proveniência:(Do b. lat. \textunderscore lectica\textunderscore , por \textunderscore lactica\textunderscore ?)}
\end{itemize}
O mesmo que \textunderscore leitôa\textunderscore .
\section{Leitigada}
\begin{itemize}
\item {Grp. gram.:f.}
\end{itemize}
\begin{itemize}
\item {Utilização:Ant.}
\end{itemize}
\begin{itemize}
\item {Proveniência:(De \textunderscore leitiga\textunderscore )}
\end{itemize}
O mesmo que \textunderscore leitegada\textunderscore .
\section{Leito}
\begin{itemize}
\item {Grp. gram.:m.}
\end{itemize}
\begin{itemize}
\item {Utilização:Constr.}
\end{itemize}
\begin{itemize}
\item {Proveniência:(Do lat. \textunderscore lectus\textunderscore )}
\end{itemize}
Conjunto das diversas peças que constituem o móvel, sôbre que nos deitamos e dormimos.
Cama.
Armação, que sustenta o enxergão.
Álveo do rio.
Superfície do carro, em que assenta a carga.
Chedeiro.
Matrimónio: \textunderscore Pedro e João, filhos do mesmo leito\textunderscore .
Superfície superior de cada uma das camadas, que constituem a parede.
\textunderscore Leito de estrada\textunderscore  ou \textunderscore rua\textunderscore , a parte de uma estrada ou rua, comprehendida entre as bermas ou entre os passeios lateraes.
\section{Leitôa}
\begin{itemize}
\item {Grp. gram.:f.}
\end{itemize}
A fêmea do leitão.
Variedade de pêra, parecida á carvalhal.
\section{Leitoada}
\begin{itemize}
\item {Grp. gram.:f.}
\end{itemize}
\begin{itemize}
\item {Proveniência:(De \textunderscore leitão\textunderscore )}
\end{itemize}
O mesmo que \textunderscore leitegada\textunderscore .
Refeição, cuja principal iguaria são leitões.
\section{Leitoado}
\begin{itemize}
\item {Grp. gram.:adj.}
\end{itemize}
\begin{itemize}
\item {Proveniência:(De \textunderscore leitão\textunderscore )}
\end{itemize}
Nédio; gordo.
\section{Leitor}
\begin{itemize}
\item {Grp. gram.:m.}
\end{itemize}
\begin{itemize}
\item {Proveniência:(Do lat. \textunderscore lector\textunderscore )}
\end{itemize}
Aquelle que lê.
Aquelle que, nos seminários ou noutras casas religiosas, lê tratados de religião ou de moral, durante as refeições da communidade.
Aquelle que, na jerarchia ecclesiástica, tem o segundo grau das Ordens menores.
\section{Leitor}
\begin{itemize}
\item {Grp. gram.:m.}
\end{itemize}
\begin{itemize}
\item {Utilização:Prov.}
\end{itemize}
\begin{itemize}
\item {Proveniência:(De \textunderscore leite\textunderscore , ou do lat. \textunderscore lactor\textunderscore )}
\end{itemize}
Anel de pedra, que as mulheres criadeiras trazem enfiado numa fita, ao peito, como amuleto, para evitar o mau olhado no leite. (Colhido em Cernancelhe)
\section{Leitorado}
\begin{itemize}
\item {Grp. gram.:m.}
\end{itemize}
Cargo de leitor.
O segundo grau das Ordens menores, na jerarchia ecclesiástica.
O tempo da leitura:«\textunderscore ...tão estendido leytorado\textunderscore ». Sousa, \textunderscore Vida do Arceb.\textunderscore , 19.
\section{Leitoral}
\begin{itemize}
\item {Grp. gram.:adj.}
\end{itemize}
\begin{itemize}
\item {Utilização:Fam.}
\end{itemize}
Relativo a leitor:«\textunderscore ora, senhor, eu já não tenho paciencia leitoral\textunderscore ». \textunderscore Anat. Joc.\textunderscore , I, 223.
\section{Leitoso}
\begin{itemize}
\item {Grp. gram.:adj.}
\end{itemize}
\begin{itemize}
\item {Proveniência:(Do lat. \textunderscore lactosus\textunderscore )}
\end{itemize}
Lácteo.
Relativo a leite.
Que tem côr ou apparência de leite: \textunderscore faces leitosas\textunderscore .
\section{Leitras}
\begin{itemize}
\item {Grp. gram.:f. pl.}
\end{itemize}
\begin{itemize}
\item {Utilização:Prov.}
\end{itemize}
O mesmo que \textunderscore láctea\textunderscore .
Leituga.
\section{Leituado}
\begin{itemize}
\item {Grp. gram.:adj.}
\end{itemize}
O mesmo que \textunderscore lactescente\textunderscore .
\section{Leituário}
\begin{itemize}
\item {Grp. gram.:m.}
\end{itemize}
Amuleto, para dar leite e vigor ás amas.
\section{Leituga}
\begin{itemize}
\item {Grp. gram.:f.}
\end{itemize}
\begin{itemize}
\item {Proveniência:(Do lat. \textunderscore lactuca\textunderscore )}
\end{itemize}
Planta, da fam. das compostas, (\textunderscore tolpis barbata\textunderscore ).
\section{Leitura}
\begin{itemize}
\item {Grp. gram.:f.}
\end{itemize}
\begin{itemize}
\item {Proveniência:(Do lat. \textunderscore lectura\textunderscore )}
\end{itemize}
Acto ou effeito de lêr.
Arte de lêr.
Lição; aquillo que se lê.
\section{Leiú}
\begin{itemize}
\item {Grp. gram.:m.}
\end{itemize}
Espécie de borboleta do Brasil.
\section{Leiva}
\begin{itemize}
\item {Grp. gram.:f.}
\end{itemize}
\begin{itemize}
\item {Utilização:Prov.}
\end{itemize}
\begin{itemize}
\item {Utilização:T. de Turquel}
\end{itemize}
\begin{itemize}
\item {Proveniência:(Do lat. hyp. \textunderscore glebea\textunderscore , de \textunderscore gleba\textunderscore ?)}
\end{itemize}
Elevação ou manta de terra, entre dois sulcos.
Sulco aberto por arado.
Gleba.
O mesmo que \textunderscore aduela\textunderscore .
Torrão, que se tira de uma vez com enxada.
\section{Leivanco}
\begin{itemize}
\item {Grp. gram.:m.}
\end{itemize}
O mesmo que \textunderscore leivão\textunderscore .
\section{Leivão}
\begin{itemize}
\item {Grp. gram.:m.}
\end{itemize}
\begin{itemize}
\item {Utilização:Prov.}
\end{itemize}
\begin{itemize}
\item {Utilização:minh.}
\end{itemize}
Rato do monte.
\section{Leixa}
\begin{itemize}
\item {Grp. gram.:f.}
\end{itemize}
\begin{itemize}
\item {Utilização:Prov.}
\end{itemize}
Acto de leixar.
Resíduo; aquillo que ficou por colher: \textunderscore foi á leixa da azeitona\textunderscore , foi ao rebusco della.
\section{Leixamento}
\begin{itemize}
\item {Grp. gram.:m.}
\end{itemize}
\begin{itemize}
\item {Utilização:Ant.}
\end{itemize}
Acto de leixar.
\section{Leixão}
\begin{itemize}
\item {Grp. gram.:m.}
\end{itemize}
Pedra alta e insulada, na costa marítima.
Ilhota.
(Cp. \textunderscore lanchão\textunderscore ^2)
\section{Leixa-pren}
\begin{itemize}
\item {Grp. gram.:m.}
\end{itemize}
Antigo artifício poético, que consistia em começar uma estrophe pela palavra ou phrase, em que terminou a estrophe anterior.
(Provn. ant.)
\section{Leixar}
\begin{itemize}
\item {Grp. gram.:v. i.}
\end{itemize}
\begin{itemize}
\item {Utilização:Ant.}
\end{itemize}
\begin{itemize}
\item {Proveniência:(Do lat. \textunderscore laxare\textunderscore )}
\end{itemize}
O mesmo que \textunderscore deixar\textunderscore .
\section{Lela}
\begin{itemize}
\item {Grp. gram.:adj. f.}
\end{itemize}
\begin{itemize}
\item {Utilização:Prov.}
\end{itemize}
\begin{itemize}
\item {Utilização:trasm.}
\end{itemize}
Diz-se de rapariga leviana, adoidada.
(Cast. \textunderscore lelo\textunderscore )
\section{Lele}
\begin{itemize}
\item {Grp. gram.:m.}
\end{itemize}
Ave africana, (\textunderscore urobrachia axillaris\textunderscore ).
\section{Lélia}
\begin{itemize}
\item {Grp. gram.:f.}
\end{itemize}
Orchídea brasileira.
\section{Lema}
\begin{itemize}
\item {Grp. gram.:m.}
\end{itemize}
\begin{itemize}
\item {Utilização:Philos.}
\end{itemize}
\begin{itemize}
\item {Utilização:Fig.}
\end{itemize}
\begin{itemize}
\item {Proveniência:(Gr. \textunderscore lemma\textunderscore )}
\end{itemize}
Proposição, cuja demonstração prepara os teoremas ou outra proposição.
Preceito escrito.
Sentença.
Emblema: \textunderscore o amor ao próximo é o seu lema\textunderscore .
\section{Lemânea}
\begin{itemize}
\item {Grp. gram.:f.}
\end{itemize}
Gênero de algas.
\section{Lemanita}
\begin{itemize}
\item {Grp. gram.:f.}
\end{itemize}
\begin{itemize}
\item {Proveniência:(De \textunderscore Leman\textunderscore , n. p.)}
\end{itemize}
Espécie de jade, descoberto nas margens do lago Leman.
\section{Lemático}
\begin{itemize}
\item {Grp. gram.:adj.}
\end{itemize}
Relativo a lema; que tem o carácter de lema.
\section{Lembefe}
\begin{itemize}
\item {Grp. gram.:m.}
\end{itemize}
\begin{itemize}
\item {Utilização:Prov.}
\end{itemize}
\begin{itemize}
\item {Utilização:alg.}
\end{itemize}
Bofetão, tabefe.
(Por \textunderscore lambefe\textunderscore )
\section{Lembósia}
\begin{itemize}
\item {Grp. gram.:f.}
\end{itemize}
Gênero de cogumelos.
\section{Lembradiço}
\begin{itemize}
\item {Grp. gram.:m.  e  adj.}
\end{itemize}
\begin{itemize}
\item {Proveniência:(De \textunderscore lembrar\textunderscore )}
\end{itemize}
Aquelle que tem bôa memória.
\section{Lembrado}
\begin{itemize}
\item {Grp. gram.:adj.}
\end{itemize}
\begin{itemize}
\item {Proveniência:(De \textunderscore lembrar\textunderscore )}
\end{itemize}
Que tem bôa memória.
\section{Lembrador}
\begin{itemize}
\item {Grp. gram.:m.  e  adj.}
\end{itemize}
Aquelle ou aquillo que serve para lembrar ou que faz lembrar ou traz á memória.
\section{Lembramento}
\begin{itemize}
\item {Grp. gram.:m.}
\end{itemize}
\begin{itemize}
\item {Utilização:Ant.}
\end{itemize}
O mesmo que \textunderscore lembrança\textunderscore .
\section{Lembrança}
\begin{itemize}
\item {Grp. gram.:f.}
\end{itemize}
\begin{itemize}
\item {Grp. gram.:Pl.}
\end{itemize}
Acto ou effeito de lembrar.
Coisa própria para ajudar a memória.
Memória: \textunderscore coisas que não acodem á lembrança\textunderscore .
Recordação: \textunderscore lembranças do passado\textunderscore .
Alvitre.
Brinde, dádiva: \textunderscore offerecer uma lembrança\textunderscore .
Cumprimentos; recordações de affecto.
\section{Lembrar}
\begin{itemize}
\item {Grp. gram.:v. t.}
\end{itemize}
\begin{itemize}
\item {Grp. gram.:V. i}
\end{itemize}
\begin{itemize}
\item {Proveniência:(Do lat. \textunderscore memorare\textunderscore )}
\end{itemize}
Trazer á memória.
Admoestar; notar: \textunderscore lembro-lhe que procedeu mal\textunderscore .
Recordar; suggerir: \textunderscore lembro-lhe a sua promessa\textunderscore .
Vir a memória, á ideia: \textunderscore já me não lembra o que dissémos\textunderscore .
\section{Lembreada}
\begin{itemize}
\item {Grp. gram.:f.}
\end{itemize}
\begin{itemize}
\item {Utilização:alg.}
\end{itemize}
\begin{itemize}
\item {Utilização:Pop.}
\end{itemize}
O mesmo que \textunderscore lambreada\textunderscore .
\section{Lembrete}
\begin{itemize}
\item {fónica:brê}
\end{itemize}
\begin{itemize}
\item {Grp. gram.:m.}
\end{itemize}
\begin{itemize}
\item {Utilização:Fam.}
\end{itemize}
\begin{itemize}
\item {Proveniência:(Do rad. de \textunderscore lembrar\textunderscore )}
\end{itemize}
Apontamento, para ajudar a memória.
Censura; ligeira punição.
\section{Leme}
\begin{itemize}
\item {Grp. gram.:m.}
\end{itemize}
\begin{itemize}
\item {Utilização:Fig.}
\end{itemize}
\begin{itemize}
\item {Proveniência:(Do b. lat. \textunderscore limo\textunderscore ?)}
\end{itemize}
Apparelho, com que se dá direcção ás embarcações.
Direcção.
Governança.
Alavanca do reparo, em artilharia.
Ferro da dobradiça, o qual se embebe no vão da fêmea, e sôbre que joga a porta ou a janela.
\section{Lemiste}
\begin{itemize}
\item {Grp. gram.:m.}
\end{itemize}
Tecido preto de lan.
\section{Lemma}
\begin{itemize}
\item {Grp. gram.:m.}
\end{itemize}
\begin{itemize}
\item {Utilização:Philos.}
\end{itemize}
\begin{itemize}
\item {Utilização:Fig.}
\end{itemize}
\begin{itemize}
\item {Proveniência:(Gr. \textunderscore lemma\textunderscore )}
\end{itemize}
Proposição, cuja demonstração prepara os theoremas ou outra proposição.
Preceito escrito.
Sentença.
Emblema: \textunderscore o amor ao próximo é o seu lemma\textunderscore .
\section{Lemmático}
\begin{itemize}
\item {Grp. gram.:adj.}
\end{itemize}
Relativo a lemma; que tem o carácter de lemma.
\section{Lemna}
\begin{itemize}
\item {Grp. gram.:f.}
\end{itemize}
\begin{itemize}
\item {Proveniência:(Gr. \textunderscore lemma\textunderscore )}
\end{itemize}
Planta aquática, que vive nas aguas tranquillas e consta de uma espécie de fôlhas em cujo pecíolo, que é também o seu caule, há um corpo que reveste e protege a raíz.
Designação scientífica da lentilha.
\section{Lemnáceas}
\begin{itemize}
\item {Grp. gram.:f. pl.}
\end{itemize}
\begin{itemize}
\item {Proveniência:(De \textunderscore lemnáceo\textunderscore )}
\end{itemize}
Família de plantas monocotyledóneas, que têm por typo a lentilha.
\section{Lemnáceo}
\begin{itemize}
\item {Grp. gram.:adj.}
\end{itemize}
\begin{itemize}
\item {Proveniência:(De \textunderscore lemna\textunderscore )}
\end{itemize}
Semelhante á lentilha.
\section{Lêmneas}
\begin{itemize}
\item {Grp. gram.:f. pl.}
\end{itemize}
(V.lemnáceas)
\section{Lemniscata}
\begin{itemize}
\item {Grp. gram.:f.}
\end{itemize}
O mesmo que \textunderscore lemniscato\textunderscore .
\section{Lemniscato}
\begin{itemize}
\item {Grp. gram.:m.}
\end{itemize}
\begin{itemize}
\item {Proveniência:(Lat. \textunderscore lemniscatus\textunderscore )}
\end{itemize}
Curva geométrica, em fórma de 8, semelhando um laço de fita.
\section{Lemnisco}
\begin{itemize}
\item {Grp. gram.:m.}
\end{itemize}
\begin{itemize}
\item {Utilização:Ant.}
\end{itemize}
\begin{itemize}
\item {Proveniência:(Lat. \textunderscore lemniscus\textunderscore )}
\end{itemize}
Fitas, pendentes das corôas dos vencedores.
Traço horizontal entre dois pontos, ou sobreposto por dois pontos, indicando no primeiro caso as passagens traduzidas da Sagrada Escritura, e no segundo caso as transposições.
\section{Lempa}
\begin{itemize}
\item {Grp. gram.:f.}
\end{itemize}
Pérola, que se pesca em algumas ilhas do Brasil.
\section{Lêmur}
\begin{itemize}
\item {Grp. gram.:m.}
\end{itemize}
Animal quadrúmano, cujas fórmas geraes se aproximam das dos quadrúpedes propriamente ditos.
(Cp. \textunderscore lêmures\textunderscore )
\section{Lemural}
\begin{itemize}
\item {Grp. gram.:adj.}
\end{itemize}
Relativo aos lêmures. Cf. Castilho, \textunderscore Fastos\textunderscore , III, 49.
\section{Lêmures}
\begin{itemize}
\item {Grp. gram.:m. pl.}
\end{itemize}
\begin{itemize}
\item {Utilização:Zool.}
\end{itemize}
\begin{itemize}
\item {Proveniência:(Lat. \textunderscore lemures\textunderscore )}
\end{itemize}
Nome, que os Romanos davam aos fantasmas dos mortos, dos quaes se dizia que appareciam de noite.
Fantasmas; duendes, trasgos.
Fam. de quadrúmanos, cujo indivíduo é o lêmur, e cujo typo é o maque.
\section{Lemúria}
\begin{itemize}
\item {Grp. gram.:f.}
\end{itemize}
O mesmo que \textunderscore lemúrias\textunderscore .
\section{Lemuriano}
\begin{itemize}
\item {Grp. gram.:adj.}
\end{itemize}
\begin{itemize}
\item {Grp. gram.:Pl.}
\end{itemize}
Relativo ou semelhante ao lêmur ou maque.
Mammíferos, o mesmo que \textunderscore lêmures\textunderscore .
\section{Lemúrias}
\begin{itemize}
\item {Grp. gram.:f. pl.}
\end{itemize}
\begin{itemize}
\item {Proveniência:(Lat. \textunderscore lemuria\textunderscore )}
\end{itemize}
Festas, que os Romanos celebravam a 9 de Maio, para applacar os espíritos dos mortos ou para conjurar espectros e fantasmas nocturnos.
\section{Lena}
\begin{itemize}
\item {Grp. gram.:f.}
\end{itemize}
\begin{itemize}
\item {Utilização:Prov.}
\end{itemize}
\begin{itemize}
\item {Utilização:alg.}
\end{itemize}
\begin{itemize}
\item {Proveniência:(Lat. \textunderscore lena\textunderscore )}
\end{itemize}
Alcoviteira.
Conversa, cavaco.
\section{Lena}
\begin{itemize}
\item {Grp. gram.:f.}
\end{itemize}
\begin{itemize}
\item {Proveniência:(Lat. \textunderscore laena\textunderscore )}
\end{itemize}
Vestuário, que os flâmines usavam sôbre a toga.
Espécie de sobretudo, usado pelos Romanos mais distintos.
\section{Lençalho}
\begin{itemize}
\item {Grp. gram.:m.}
\end{itemize}
Lenço grande e ordinário. Cf. J. de Deus, \textunderscore Campo de Flôres\textunderscore , 499.
\section{Lencantina}
\begin{itemize}
\item {Grp. gram.:f.}
\end{itemize}
\begin{itemize}
\item {Utilização:Prov.}
\end{itemize}
\begin{itemize}
\item {Utilização:alg.}
\end{itemize}
Cantilena.
Choradeira.
O mesmo que \textunderscore alicantina\textunderscore .
\section{Lenção}
\begin{itemize}
\item {Grp. gram.:m.}
\end{itemize}
\begin{itemize}
\item {Proveniência:(De \textunderscore lenço\textunderscore )}
\end{itemize}
Antiga armadilha para pesca.
\section{Lençaria}
\begin{itemize}
\item {Grp. gram.:f.}
\end{itemize}
Fábrica ou estabelecimento de lenços.
Negócio de tecidos de linho ou de algodão.
\section{Lencinho}
\begin{itemize}
\item {Grp. gram.:m.}
\end{itemize}
Espécie de jôgo popular.
\section{Lenço}
\begin{itemize}
\item {Grp. gram.:m.}
\end{itemize}
\begin{itemize}
\item {Utilização:Ant.}
\end{itemize}
\begin{itemize}
\item {Utilização:Pop.}
\end{itemize}
\begin{itemize}
\item {Utilização:Marcen.}
\end{itemize}
\begin{itemize}
\item {Utilização:Des.}
\end{itemize}
\begin{itemize}
\item {Proveniência:(Do lat. \textunderscore linteum\textunderscore )}
\end{itemize}
Pequeno pano quadrangular, que serve para alguém se assoar, ou para cobrir a cabeça ou para resguardar o pescoço, etc.
Espécie de tecido de algodão e linho.
Espécie de imposto antigo?:«\textunderscore e de outras casas paga de lenço cada anno 130 reis.\textunderscore »(De um testamento de 1691)
Mesentério.
Cada um dos lados das gavetas.
Tela de pintura ou quadro.
\section{Lencó}
\begin{itemize}
\item {Grp. gram.:m.}
\end{itemize}
Fruta de Macau, (\textunderscore trapa bicornis\textunderscore ).
\section{Lenço-de-fivelas}
\begin{itemize}
\item {Grp. gram.:m.}
\end{itemize}
\begin{itemize}
\item {Utilização:Prov.}
\end{itemize}
\begin{itemize}
\item {Utilização:trasm.}
\end{itemize}
\begin{itemize}
\item {Utilização:chul.}
\end{itemize}
O mesmo que \textunderscore cabresto\textunderscore .
\section{Lençol}
\begin{itemize}
\item {Grp. gram.:m.}
\end{itemize}
\begin{itemize}
\item {Utilização:Fig.}
\end{itemize}
\begin{itemize}
\item {Utilização:ant.}
\end{itemize}
\begin{itemize}
\item {Utilização:Chul.}
\end{itemize}
\begin{itemize}
\item {Grp. gram.:Loc.}
\end{itemize}
\begin{itemize}
\item {Utilização:fam.}
\end{itemize}
\begin{itemize}
\item {Grp. gram.:Loc.}
\end{itemize}
\begin{itemize}
\item {Utilização:fam.}
\end{itemize}
\begin{itemize}
\item {Proveniência:(Do lat. \textunderscore linteolum\textunderscore )}
\end{itemize}
Peça de pano branco, destinada a revestir superiormente o colchão.
Peça análoga, que, sotoposta aos cobertores, cobre quem se deita sôbre aquella outra.
Mortalha.
Objecto, que tem o aspecto ou fórma de lençol.
Vela de navio. Cf. Simão Mach., f. 37 v.^o
\textunderscore Pôr em lençoes de vinho\textunderscore , dar grande sova em.
\textunderscore Estar em maus lençoes\textunderscore , estar em má situação, em difficuldades.
\section{Lenda}
\begin{itemize}
\item {Grp. gram.:f.}
\end{itemize}
\begin{itemize}
\item {Utilização:Fig.}
\end{itemize}
\begin{itemize}
\item {Proveniência:(Do lat. \textunderscore legenda\textunderscore )}
\end{itemize}
Narrativa escrita, digna de se lêr.
Narrativa de successos fantásticos.
Conto.
Tradição popular.
Mentira.
Narração fastidiosa.
\section{Lendário}
\begin{itemize}
\item {Grp. gram.:adj.}
\end{itemize}
Relativo a lenda; que tem carácter de lenda.
\section{Lêndea}
\begin{itemize}
\item {Grp. gram.:f.}
\end{itemize}
\begin{itemize}
\item {Proveniência:(Do lat. \textunderscore lens\textunderscore , \textunderscore lendis\textunderscore )}
\end{itemize}
Ovo de piolho da cabeça.
\section{Lendeaço}
\begin{itemize}
\item {Grp. gram.:m.}
\end{itemize}
Grande porção de lêndeas.
Cabeça com muitas lêndeas.
\section{Lendeoso}
\begin{itemize}
\item {Grp. gram.:adj.}
\end{itemize}
Que tem lêndeas.
\section{Lendroeira}
\begin{itemize}
\item {Grp. gram.:f.}
\end{itemize}
\begin{itemize}
\item {Utilização:Prov.}
\end{itemize}
\begin{itemize}
\item {Utilização:alent.}
\end{itemize}
O mesmo que \textunderscore loendro\textunderscore  ou \textunderscore cevadilha\textunderscore .
\section{Lendroso}
\begin{itemize}
\item {Grp. gram.:adj.}
\end{itemize}
\begin{itemize}
\item {Utilização:Ant.}
\end{itemize}
Que tem muitas lêndeas; lendeoso.
(Cp. cast. \textunderscore lendroso\textunderscore )
\section{Lene}
\begin{itemize}
\item {Grp. gram.:adj.}
\end{itemize}
\begin{itemize}
\item {Proveniência:(Lat. \textunderscore lenis\textunderscore )}
\end{itemize}
Brando; macio; suave. Cf. Latino, \textunderscore Hist. Pol. e Mil.\textunderscore , I, 312.
\section{Leneias}
\begin{itemize}
\item {Grp. gram.:f. pl.}
\end{itemize}
\begin{itemize}
\item {Proveniência:(Do lat. \textunderscore leneae\textunderscore )}
\end{itemize}
Festas gregas, em honra de Baccho.
\section{Lenga-lenga}
\begin{itemize}
\item {Grp. gram.:f.}
\end{itemize}
\begin{itemize}
\item {Utilização:Pop.}
\end{itemize}
Narração monótona, enfadonha.
\section{Lenga-lengar}
\begin{itemize}
\item {Grp. gram.:v. i.}
\end{itemize}
\begin{itemize}
\item {Utilização:Neol.}
\end{itemize}
Fazer lenga-lenga. Cf. Valentim Mag., \textunderscore Contos\textunderscore .
\section{Lengue}
\begin{itemize}
\item {Grp. gram.:m.}
\end{itemize}
Pássaro syndáctylo da África occidental.
\section{Lenha}
\begin{itemize}
\item {Grp. gram.:f.}
\end{itemize}
\begin{itemize}
\item {Utilização:Fam.}
\end{itemize}
\begin{itemize}
\item {Proveniência:(Do b. lat. \textunderscore lenia\textunderscore )}
\end{itemize}
Ramagem de árvores, de estevas ou de outras plantas, destinada a alimentar a combustão nos fornos, nas cozinhas, etc.
Achas ou cavacas, destinadas ao mesmo fim.
Pancadas.
\section{Lenhador}
\begin{itemize}
\item {Grp. gram.:m.  e  adj.}
\end{itemize}
\begin{itemize}
\item {Proveniência:(De \textunderscore lenhar\textunderscore )}
\end{itemize}
Aquelle que colhe ou corta lenha.
Aquelle que racha troncos para fazer lenha.
Rachador de lenha; lenheiro.
\section{Lenhar}
\begin{itemize}
\item {Grp. gram.:v. i.}
\end{itemize}
\begin{itemize}
\item {Utilização:Des.}
\end{itemize}
Cortar lenha.
Prover-se de lenha.
\section{Lenheiro}
\begin{itemize}
\item {Grp. gram.:m.}
\end{itemize}
\begin{itemize}
\item {Proveniência:(Do lat. \textunderscore lignarius\textunderscore )}
\end{itemize}
O mesmo que \textunderscore lenhador\textunderscore .
\section{Lenhificar-se}
\begin{itemize}
\item {Grp. gram.:v. p.}
\end{itemize}
(V.lignificar-se)
\section{Lenhite}
\begin{itemize}
\item {Grp. gram.:f.}
\end{itemize}
(V.lignite)
\section{Lenho}
\begin{itemize}
\item {Grp. gram.:m.}
\end{itemize}
\begin{itemize}
\item {Utilização:Poét.}
\end{itemize}
\begin{itemize}
\item {Proveniência:(Do lat. \textunderscore lignum\textunderscore )}
\end{itemize}
Ramo de árvore.
Pernada.
Tronco.
Navio.
\textunderscore Santo lenho\textunderscore , cruz de Christo.
\section{Lenhose}
\begin{itemize}
\item {Grp. gram.:f.}
\end{itemize}
\begin{itemize}
\item {Utilização:Bot.}
\end{itemize}
Substância carbonada, que encrusta a cellulose e é o molde da fibra. Cf. F. Lapa, \textunderscore Alman. do Lavr.\textunderscore , (1869), 11.
\section{Lenhoso}
\begin{itemize}
\item {Grp. gram.:adj.}
\end{itemize}
\begin{itemize}
\item {Grp. gram.:M.}
\end{itemize}
\begin{itemize}
\item {Proveniência:(Do lat. \textunderscore lignosus\textunderscore )}
\end{itemize}
Que tem a consistência de madeira.
Princípio orgânico das plantas.
\section{Lenidade}
\begin{itemize}
\item {Grp. gram.:f.}
\end{itemize}
\begin{itemize}
\item {Proveniência:(Lat. \textunderscore lenitas\textunderscore )}
\end{itemize}
Brandura; mansidão; suavidade.
\section{Leniente}
\begin{itemize}
\item {Grp. gram.:m.  e  adj.}
\end{itemize}
\begin{itemize}
\item {Proveniência:(Lat. \textunderscore leniens\textunderscore )}
\end{itemize}
Lenitivo.
\section{Lenificar}
\begin{itemize}
\item {Grp. gram.:v. t.}
\end{itemize}
\begin{itemize}
\item {Proveniência:(Do lat. \textunderscore lenis\textunderscore  + \textunderscore facere\textunderscore )}
\end{itemize}
Alliviar; mitigar; abrandar.
\section{Lenimento}
\begin{itemize}
\item {Grp. gram.:m.}
\end{itemize}
\begin{itemize}
\item {Proveniência:(Lat. \textunderscore lenimentum\textunderscore )}
\end{itemize}
Aquillo que embrandece.
Medicamento, que mitiga dores.
\section{Lenir}
\begin{itemize}
\item {Grp. gram.:v. t.}
\end{itemize}
\begin{itemize}
\item {Proveniência:(Lat. \textunderscore lenire\textunderscore )}
\end{itemize}
Abrandar; alliviar; suavizar.
\section{Lenirobina}
\begin{itemize}
\item {fónica:ro}
\end{itemize}
\begin{itemize}
\item {Grp. gram.:f.}
\end{itemize}
Composto chímico, usado em Medicina, contra a dermatose.
\section{Lenirrobina}
\begin{itemize}
\item {Grp. gram.:f.}
\end{itemize}
Composto químico, usado em Medicina, contra a dermatose.
\section{Lenitivo}
\begin{itemize}
\item {Grp. gram.:adj.}
\end{itemize}
\begin{itemize}
\item {Grp. gram.:M.}
\end{itemize}
\begin{itemize}
\item {Utilização:Fig.}
\end{itemize}
\begin{itemize}
\item {Proveniência:(Do lat. \textunderscore lenitus\textunderscore )}
\end{itemize}
Próprio para lenir.
Lenimento; laxante.
Allívio; consolação.
\section{Lenocínio}
\begin{itemize}
\item {Grp. gram.:m.}
\end{itemize}
\begin{itemize}
\item {Proveniência:(Lat. \textunderscore lenocinium\textunderscore )}
\end{itemize}
Acto de proporcionar, estimular ou facilitar a devassidão ou a corrupção de alguém.
\section{Lentação}
\begin{itemize}
\item {Grp. gram.:f.}
\end{itemize}
\begin{itemize}
\item {Utilização:Des.}
\end{itemize}
Lentidão, demora:«\textunderscore ...dá-la ao prelo sem mais lentação...\textunderscore »Varnhagem, no pról. do \textunderscore Diário da Naveg.\textunderscore , de Pero Lopes.
\section{Lentamente}
\begin{itemize}
\item {Grp. gram.:adv.}
\end{itemize}
De modo lento; paulatinamente; devagar.
\section{Lentar}
\begin{itemize}
\item {Grp. gram.:v. t.}
\end{itemize}
\begin{itemize}
\item {Grp. gram.:V. i.}
\end{itemize}
\begin{itemize}
\item {Utilização:Ext.}
\end{itemize}
\begin{itemize}
\item {Proveniência:(Lat. \textunderscore lentare\textunderscore )}
\end{itemize}
Tornar lento, húmido.
Tornar-se um tanto húmido.
Transpirar um pouco.
\section{Lente}
\begin{itemize}
\item {Grp. gram.:adj.}
\end{itemize}
\begin{itemize}
\item {Grp. gram.:M.}
\end{itemize}
\begin{itemize}
\item {Utilização:Ext.}
\end{itemize}
\begin{itemize}
\item {Utilização:Des.}
\end{itemize}
Que lê.
Professor da Universidade.
Professor de uma escola superior ou de um lyceu.
Leitor de qualquer obra. Cf. Pant. de Aveiro, \textunderscore Itiner.\textunderscore , 21, (2.^a ed.).
\section{Lente}
\begin{itemize}
\item {Grp. gram.:f.}
\end{itemize}
\begin{itemize}
\item {Proveniência:(Do lat. \textunderscore lens\textunderscore , \textunderscore lentis\textunderscore )}
\end{itemize}
Disco de vidro, que refrange os raios luminosos.
\section{Lenteiro}
\begin{itemize}
\item {Grp. gram.:m.}
\end{itemize}
\begin{itemize}
\item {Grp. gram.:Adj.}
\end{itemize}
\begin{itemize}
\item {Utilização:Prov.}
\end{itemize}
\begin{itemize}
\item {Proveniência:(Do rad. de \textunderscore lentar\textunderscore )}
\end{itemize}
Terra húmida; pântano; lameiro.
Húmido.
\section{Lentejar}
\begin{itemize}
\item {Grp. gram.:v. t.}
\end{itemize}
\begin{itemize}
\item {Grp. gram.:V. i.}
\end{itemize}
\begin{itemize}
\item {Proveniência:(De \textunderscore lento\textunderscore )}
\end{itemize}
Tornar húmido.
Tornar-se húmido.
\section{Lentejoila}
\begin{itemize}
\item {Grp. gram.:f.}
\end{itemize}
Pequenina chapa circular de metal, para enfeite de vestuário.
(Cast. \textunderscore lentejuela\textunderscore )
\section{Lentejoilar}
\begin{itemize}
\item {Grp. gram.:v. t.}
\end{itemize}
\begin{itemize}
\item {Utilização:Ext.}
\end{itemize}
Ornar de lentejoilas.
Ornar.
\section{Lentejoula}
\begin{itemize}
\item {Grp. gram.:f.}
\end{itemize}
Pequenina chapa circular de metal, para enfeite de vestuário.
(Cast. \textunderscore lentejuela\textunderscore )
\section{Lentescente}
\begin{itemize}
\item {Grp. gram.:adj.}
\end{itemize}
\begin{itemize}
\item {Proveniência:(Lat. \textunderscore lentescens\textunderscore )}
\end{itemize}
Languinhento; pegajoso.
\section{Lentescer}
\begin{itemize}
\item {Grp. gram.:v. t.  e  i.}
\end{itemize}
O mesmo que \textunderscore lentar\textunderscore .
\section{Lenteza}
\begin{itemize}
\item {Grp. gram.:f.}
\end{itemize}
O mesmo que \textunderscore lentidão\textunderscore .
\section{Lenticão}
\begin{itemize}
\item {Grp. gram.:m.}
\end{itemize}
\begin{itemize}
\item {Utilização:Prov.}
\end{itemize}
Excrescência nas espigas do centeio.
(Cp. \textunderscore lentícula\textunderscore )
\section{Lenticela}
\begin{itemize}
\item {Grp. gram.:f.}
\end{itemize}
\begin{itemize}
\item {Utilização:Bot.}
\end{itemize}
Mancha vermelha e oval, na casca dos vegetaes.
(Cp. \textunderscore lentícula\textunderscore )
\section{Lenticella}
\begin{itemize}
\item {Grp. gram.:f.}
\end{itemize}
\begin{itemize}
\item {Utilização:Bot.}
\end{itemize}
Mancha vermelha e oval, na casca dos vegetaes.
(Cp. \textunderscore lentícula\textunderscore )
\section{Lentícula}
\begin{itemize}
\item {Grp. gram.:f.}
\end{itemize}
\begin{itemize}
\item {Proveniência:(Lat. \textunderscore lenticula\textunderscore )}
\end{itemize}
Pequena lente.
O mesmo que \textunderscore lenticella\textunderscore .
\section{Lenticular}
\begin{itemize}
\item {Grp. gram.:adj.}
\end{itemize}
\begin{itemize}
\item {Grp. gram.:M.}
\end{itemize}
\begin{itemize}
\item {Proveniência:(Lat. \textunderscore lenticularis\textunderscore )}
\end{itemize}
Que tem forma de \textunderscore lente\textunderscore ^2.
Que tem a fórma de lentilha.
Instrumento, para furar o casco dos animaes.
\section{Lentidão}
\begin{itemize}
\item {Grp. gram.:f.}
\end{itemize}
\begin{itemize}
\item {Proveniência:(Lat. \textunderscore lentitudo\textunderscore )}
\end{itemize}
Qualidade ou estado de lento; demora; vagar.
Ligeira húmidade.
\section{Lentiforme}
\begin{itemize}
\item {Grp. gram.:adj.}
\end{itemize}
\begin{itemize}
\item {Proveniência:(De \textunderscore lente\textunderscore ^2 + \textunderscore fórma\textunderscore )}
\end{itemize}
O mesmo que \textunderscore lenticular\textunderscore .
\section{Lentigem}
\begin{itemize}
\item {Grp. gram.:f.}
\end{itemize}
\begin{itemize}
\item {Proveniência:(Lat. \textunderscore lentigo\textunderscore )}
\end{itemize}
Mancha da pelle, ephélide lentiforme; sarda.
\section{Lentiginoso}
\begin{itemize}
\item {Grp. gram.:adj.}
\end{itemize}
\begin{itemize}
\item {Proveniência:(Lat. \textunderscore lentiginosus\textunderscore )}
\end{itemize}
Que tem lentigens; sardento.
\section{Lentigo}
\begin{itemize}
\item {Grp. gram.:m.}
\end{itemize}
(V.lentigem)
\section{Lentígrado}
\begin{itemize}
\item {Grp. gram.:adj.}
\end{itemize}
\begin{itemize}
\item {Proveniência:(Do lat. \textunderscore lentus\textunderscore  + \textunderscore gradi\textunderscore )}
\end{itemize}
Que caminha lentamente.
\section{Lentilha}
\begin{itemize}
\item {Grp. gram.:f.}
\end{itemize}
\begin{itemize}
\item {Proveniência:(Lat. \textunderscore lenticula\textunderscore )}
\end{itemize}
Planta leguminosa, (\textunderscore ervum lens\textunderscore ).
\section{Lentilhão}
(V.lenticão)
\section{Lentilheira}
\begin{itemize}
\item {Grp. gram.:f.}
\end{itemize}
O mesmo que \textunderscore lentilha\textunderscore . Cf. B. Pereira, \textunderscore Prosódia\textunderscore , vb. \textunderscore chamesyce\textunderscore .
\section{Lentilhoso}
\begin{itemize}
\item {Grp. gram.:adj.}
\end{itemize}
Que abunda em lentilhas.
\section{Lentisca}
\begin{itemize}
\item {Grp. gram.:adj. f.}
\end{itemize}
\begin{itemize}
\item {Utilização:Prov.}
\end{itemize}
\begin{itemize}
\item {Utilização:trasm.}
\end{itemize}
Diz-se de uma variedade de azeitona.
\section{Lentiscal}
\begin{itemize}
\item {Grp. gram.:m.}
\end{itemize}
Terreno, em que crescem lentiscos.
\section{Lentisco}
\begin{itemize}
\item {Grp. gram.:m.}
\end{itemize}
\begin{itemize}
\item {Proveniência:(Lat. \textunderscore lentiscus\textunderscore )}
\end{itemize}
O mesmo quo \textunderscore aroeira\textunderscore .
\section{Lentisqueira}
\begin{itemize}
\item {Grp. gram.:f.}
\end{itemize}
O mesmo que \textunderscore lentiscal\textunderscore .
O mesmo que \textunderscore lentisco\textunderscore .
Variedade de oliveiras, (\textunderscore olea europaea\textunderscore , Lin.).
\section{Lento}
\begin{itemize}
\item {Grp. gram.:adj.}
\end{itemize}
\begin{itemize}
\item {Grp. gram.:Adv.}
\end{itemize}
\begin{itemize}
\item {Proveniência:(Lat. \textunderscore lentus\textunderscore )}
\end{itemize}
Que é pegajoso ou viscoso; molle.
Froixo.
Levemente humedecido.
Orvalhado.
Flexível; brando.
Que se move devagar; vagaroso; preguiçoso.
Que se prolonga.
Que apparece a pouco e pouco.
Fraco ou espaçado, (falando-se das pulsações).
O mesmo que \textunderscore lentamente\textunderscore .
\section{Lentor}
\begin{itemize}
\item {Grp. gram.:m.}
\end{itemize}
\begin{itemize}
\item {Proveniência:(Lat. \textunderscore lentor\textunderscore )}
\end{itemize}
O mesmo que \textunderscore lentidão\textunderscore .
\section{Lentrisca}
\begin{itemize}
\item {Utilização:Prov.}
\end{itemize}
O mesmo que \textunderscore lentisca\textunderscore .
\section{Lentrisqueira}
\begin{itemize}
\item {Grp. gram.:f.}
\end{itemize}
Prov. Oliveira, que dá lentrisca.
\section{Lentura}
\begin{itemize}
\item {Grp. gram.:f.}
\end{itemize}
O mesmo que \textunderscore lentidão\textunderscore .
Ligeira humidade.
Orvalho; relento.
\section{Leôa}
\begin{itemize}
\item {Grp. gram.:f.}
\end{itemize}
\begin{itemize}
\item {Utilização:Fig.}
\end{itemize}
\begin{itemize}
\item {Proveniência:(De \textunderscore leão\textunderscore )}
\end{itemize}
A fêmea do leão.
Mulher de maus instintos.
Mulher garrida.
\section{Leonado}
\begin{itemize}
\item {Grp. gram.:adj.}
\end{itemize}
O mesmo que \textunderscore aleonado\textunderscore . Cf. Vieira, XI, 258.
\section{Leônculo}
\begin{itemize}
\item {Grp. gram.:m.}
\end{itemize}
Leão pequeno. Cf. Camillo, \textunderscore Ag. em Palheiro\textunderscore , 75.
\section{Leoneira}
\begin{itemize}
\item {Grp. gram.:f.}
\end{itemize}
Esconderijo de leões; jaula para leões.
\section{Leonês}
\begin{itemize}
\item {Grp. gram.:adj.}
\end{itemize}
\begin{itemize}
\item {Grp. gram.:M.}
\end{itemize}
\begin{itemize}
\item {Proveniência:(Do cast. \textunderscore Leon\textunderscore , n. p.)}
\end{itemize}
Relativo á cidade ou ao antigo reino de Leão.
Habitante, de Leão.
\section{Leonesa}
\begin{itemize}
\item {Grp. gram.:f.}
\end{itemize}
\begin{itemize}
\item {Utilização:Ant.}
\end{itemize}
O mesmo que \textunderscore leôa\textunderscore .
\section{Leónico}
\begin{itemize}
\item {Grp. gram.:adj.}
\end{itemize}
Relativo a leão.
\section{Leonino}
\begin{itemize}
\item {Grp. gram.:adj.}
\end{itemize}
\begin{itemize}
\item {Utilização:Fig.}
\end{itemize}
\begin{itemize}
\item {Proveniência:(Lat. \textunderscore leoninus\textunderscore )}
\end{itemize}
Relativo ou semelhante a leão.
Próprio do leão.
Pérfido; desleal.
\section{Leonino}
\begin{itemize}
\item {Grp. gram.:adj.}
\end{itemize}
Diz-se do verso latino, em que duas cesuras rimam, uma com a outra, ou em que a sýllaba da cesura rima com a última do verso; e dizia-se da rima, que se estendia desde a última até á antepenúltima sýllaba.
(Or. incerta. Cita-se um \textunderscore Leonino\textunderscore , que no século XIV fez versos daquella espécie)
\section{Leontíase}
\begin{itemize}
\item {Grp. gram.:f.}
\end{itemize}
\begin{itemize}
\item {Utilização:Med.}
\end{itemize}
\begin{itemize}
\item {Proveniência:(Gr. \textunderscore leontiasis\textunderscore )}
\end{itemize}
Elephantíase tuberculosa da face.
\section{Leontodonte}
\begin{itemize}
\item {Grp. gram.:m.}
\end{itemize}
\begin{itemize}
\item {Proveniência:(Do gr. \textunderscore leon\textunderscore , \textunderscore leontos\textunderscore  + \textunderscore odous\textunderscore , \textunderscore odontos\textunderscore )}
\end{itemize}
Gênero de plantas compostas.
\section{Leontofono}
\begin{itemize}
\item {Grp. gram.:m.}
\end{itemize}
\begin{itemize}
\item {Proveniência:(Gr. \textunderscore leontophonos\textunderscore )}
\end{itemize}
Nome, que os antigos deram a um pequeno animal, de cuja urina se dizia que era venenosa para o leão.
\section{Leontophono}
\begin{itemize}
\item {Grp. gram.:m.}
\end{itemize}
\begin{itemize}
\item {Proveniência:(Gr. \textunderscore leontophonos\textunderscore )}
\end{itemize}
Nome, que os antigos deram a um pequeno animal, de cuja urina se dizia que era venenosa para o leão.
\section{Leontopódio}
\begin{itemize}
\item {Grp. gram.:m.}
\end{itemize}
\begin{itemize}
\item {Proveniência:(Do gr. \textunderscore leon\textunderscore , \textunderscore leontos\textunderscore  + \textunderscore pous\textunderscore , \textunderscore podos\textunderscore )}
\end{itemize}
Gênero de plantas synanthéreas.
\section{Leonuro}
\begin{itemize}
\item {Grp. gram.:m.}
\end{itemize}
\begin{itemize}
\item {Proveniência:(Do gr. \textunderscore leon\textunderscore  + \textunderscore oura\textunderscore )}
\end{itemize}
Planta, conhecida também por \textunderscore cordão de san-francisco\textunderscore .
\section{Lepantho}
\begin{itemize}
\item {Grp. gram.:m.}
\end{itemize}
Gênero de orchídeas.
\section{Lepanto}
\begin{itemize}
\item {Grp. gram.:m.}
\end{itemize}
Gênero de orquídeas.
\section{Leopardo}
\begin{itemize}
\item {Grp. gram.:m.}
\end{itemize}
\begin{itemize}
\item {Utilização:Fig.}
\end{itemize}
\begin{itemize}
\item {Proveniência:(Lat. \textunderscore leopardus\textunderscore )}
\end{itemize}
Quadrúpede carnívoro, de pelle mosqueada.
A nação inglesa.
\section{Leopoldínia}
\begin{itemize}
\item {Grp. gram.:f.}
\end{itemize}
\begin{itemize}
\item {Proveniência:(De \textunderscore Leopold\textunderscore , n. p.)}
\end{itemize}
Gênero de plantas das margens do Amazonas.
\section{Lepas}
\begin{itemize}
\item {Grp. gram.:m.}
\end{itemize}
\begin{itemize}
\item {Proveniência:(Lat. \textunderscore lepas\textunderscore )}
\end{itemize}
Concha univalve, que adhere aos rochedos.
\section{Lepes}
\begin{itemize}
\item {Grp. gram.:m.}
\end{itemize}
\begin{itemize}
\item {Utilização:Gír.}
\end{itemize}
Moéda de déz reis.
\textunderscore Café\textunderscore  ou \textunderscore botequim de lepes\textunderscore , botequim ordinário, onde se vende café, a 10 reis cada chícara.
\section{Lepicena}
\begin{itemize}
\item {Grp. gram.:f.}
\end{itemize}
\begin{itemize}
\item {Utilização:Bot.}
\end{itemize}
A gluma exterior das gramíneas.
\section{Lapidadênia}
\begin{itemize}
\item {Grp. gram.:f.}
\end{itemize}
\begin{itemize}
\item {Proveniência:(Do gr. \textunderscore lepis\textunderscore , \textunderscore lepidos\textunderscore  + \textunderscore eden\textunderscore )}
\end{itemize}
Árvore laurínea da Índia.
\section{Lepidamente}
\begin{itemize}
\item {Grp. gram.:adv.}
\end{itemize}
De modo lépido.
Promptamente.
\section{Lepidíneas}
\begin{itemize}
\item {Grp. gram.:f. pl.}
\end{itemize}
\begin{itemize}
\item {Utilização:Bot.}
\end{itemize}
\begin{itemize}
\item {Proveniência:(De \textunderscore lepídio\textunderscore )}
\end{itemize}
Tríbo de plantas crucíferas.
\section{Lepídio}
\begin{itemize}
\item {Grp. gram.:m.}
\end{itemize}
Planta, o mesmo que \textunderscore erva-pimenteira\textunderscore .
\section{Lépido}
\begin{itemize}
\item {Grp. gram.:adj.}
\end{itemize}
\begin{itemize}
\item {Utilização:Pop.}
\end{itemize}
\begin{itemize}
\item {Proveniência:(Lat. \textunderscore lepidus\textunderscore )}
\end{itemize}
Jovial; gracioso; que graceja.
Ligeiro; prompto; lesto.
\section{Lépido...}
\begin{itemize}
\item {Grp. gram.:pref.}
\end{itemize}
\begin{itemize}
\item {Proveniência:(Do gr. \textunderscore lepis\textunderscore , \textunderscore lepidos\textunderscore )}
\end{itemize}
(designativo de \textunderscore escama\textunderscore )
\section{Lèpidocarpo}
\begin{itemize}
\item {Grp. gram.:adj.}
\end{itemize}
\begin{itemize}
\item {Proveniência:(Do gr. \textunderscore lepis\textunderscore  + \textunderscore karpos\textunderscore )}
\end{itemize}
Que tem frutos escamosos.
\section{Lepidócero}
\begin{itemize}
\item {Grp. gram.:adj.}
\end{itemize}
\begin{itemize}
\item {Utilização:Zool.}
\end{itemize}
\begin{itemize}
\item {Proveniência:(Do gr. \textunderscore lepis\textunderscore  + \textunderscore keras\textunderscore )}
\end{itemize}
Que tem pequenas escamas nas antennas.
\section{Lepidodendro}
\begin{itemize}
\item {Grp. gram.:m.}
\end{itemize}
\begin{itemize}
\item {Proveniência:(Do gr. \textunderscore lepis\textunderscore , \textunderscore lepidos\textunderscore  + \textunderscore dendron\textunderscore )}
\end{itemize}
Gênero de vegetaes fósseis.
\section{Lepidoídeo}
\begin{itemize}
\item {Grp. gram.:adj.}
\end{itemize}
\begin{itemize}
\item {Proveniência:(Do gr. \textunderscore lepis\textunderscore  + \textunderscore eidos\textunderscore )}
\end{itemize}
Semelhante a escamas.
\section{Lepidólitho}
\begin{itemize}
\item {Grp. gram.:m.}
\end{itemize}
\begin{itemize}
\item {Proveniência:(Do gr. \textunderscore lepis\textunderscore , \textunderscore lepidos\textunderscore  + \textunderscore lithos\textunderscore )}
\end{itemize}
Substância mineral escamosa.
\section{Lepidólito}
\begin{itemize}
\item {Grp. gram.:m.}
\end{itemize}
\begin{itemize}
\item {Proveniência:(Do gr. \textunderscore lepis\textunderscore , \textunderscore lepidos\textunderscore  + \textunderscore lithos\textunderscore )}
\end{itemize}
Substância mineral escamosa.
\section{Lepidomelano}
\begin{itemize}
\item {Grp. gram.:m.}
\end{itemize}
\begin{itemize}
\item {Utilização:Miner.}
\end{itemize}
Variedade de biotite negra.
\section{Lepidóptero}
\begin{itemize}
\item {Grp. gram.:m.  e  adj.}
\end{itemize}
\begin{itemize}
\item {Proveniência:(Do gr. \textunderscore lepis\textunderscore , \textunderscore lepidos\textunderscore  + \textunderscore pteron\textunderscore )}
\end{itemize}
Diz-se de uma classe de insectos, que passam por metamorphoses completas, desde o estado de ovo ao de borboleta.
\section{Lepidopterologia}
\begin{itemize}
\item {Grp. gram.:f.}
\end{itemize}
\begin{itemize}
\item {Proveniência:(Do gr. \textunderscore lepis\textunderscore  + \textunderscore pteron\textunderscore  + \textunderscore logos\textunderscore )}
\end{itemize}
Parte da Zoologia, que trata dos lepidópteros.
\section{Lepidopterológico}
\begin{itemize}
\item {Grp. gram.:adj.}
\end{itemize}
Relativo á lepidopterologia.
\section{Lepidopterologista}
\begin{itemize}
\item {Grp. gram.:m.}
\end{itemize}
Aquelle que trata de lepidópteros, ou que é versado em lepidopterologia.
\section{Lepidosáurios}
\begin{itemize}
\item {fónica:sau}
\end{itemize}
\begin{itemize}
\item {Grp. gram.:m. pl.}
\end{itemize}
\begin{itemize}
\item {Proveniência:(Do gr. \textunderscore lepis\textunderscore  + \textunderscore saura\textunderscore )}
\end{itemize}
Subclasse de reptis.
\section{Lepidosereia}
\begin{itemize}
\item {fónica:se}
\end{itemize}
\begin{itemize}
\item {Grp. gram.:f.}
\end{itemize}
\begin{itemize}
\item {Proveniência:(De \textunderscore lépido...\textunderscore  + \textunderscore sereia\textunderscore )}
\end{itemize}
Gênero de animaes, que parecem formar a transição entre os peixes e os reptis.
\section{Lepidosirenos}
\begin{itemize}
\item {fónica:si}
\end{itemize}
\begin{itemize}
\item {Grp. gram.:m. pl.}
\end{itemize}
Gênero de peixes, cuja única espécie é o caramuru, (\textunderscore lepidosiren paradoxa\textunderscore ).
\section{Lepidosperma}
\begin{itemize}
\item {Grp. gram.:f.}
\end{itemize}
Gênero de plantas cyperáceas.
\section{Lepidossáurios}
\begin{itemize}
\item {Grp. gram.:m. pl.}
\end{itemize}
\begin{itemize}
\item {Proveniência:(Do gr. \textunderscore lepis\textunderscore  + \textunderscore saura\textunderscore )}
\end{itemize}
Subclasse de reptis.
\section{Lepidossereia}
\begin{itemize}
\item {Grp. gram.:f.}
\end{itemize}
\begin{itemize}
\item {Proveniência:(De \textunderscore lépido...\textunderscore  + \textunderscore sereia\textunderscore )}
\end{itemize}
Gênero de animaes, que parecem formar a transição entre os peixes e os reptis.
\section{Lepidossirenos}
\begin{itemize}
\item {Grp. gram.:m. pl.}
\end{itemize}
Gênero de peixes, cuja única espécie é o caramuru, (\textunderscore lepidosiren paradoxa\textunderscore ).
\section{Lepíptero}
\begin{itemize}
\item {Grp. gram.:m.}
\end{itemize}
\begin{itemize}
\item {Proveniência:(Do gr. \textunderscore lepis\textunderscore  + \textunderscore pteron\textunderscore )}
\end{itemize}
Gênero de peixes acanthopterýgios.
\section{Lépis}
\begin{itemize}
\item {Grp. gram.:m.}
\end{itemize}
\begin{itemize}
\item {Utilização:Gír.}
\end{itemize}
O mesmo que \textunderscore lepes\textunderscore . Cf. Camillo, \textunderscore Mar. da Fonte\textunderscore , 388.
\section{Lepisacantho}
\begin{itemize}
\item {Grp. gram.:m.}
\end{itemize}
Gênero de peixes acanthopterýgios.
\section{Lepisacanto}
\begin{itemize}
\item {Grp. gram.:m.}
\end{itemize}
Gênero de peixes acantopterígios.
\section{Lepisantho}
\begin{itemize}
\item {Grp. gram.:m.}
\end{itemize}
Gênero de plantas sapindáceas.
\section{Lepisanto}
\begin{itemize}
\item {Grp. gram.:m.}
\end{itemize}
Gênero de plantas sapindáceas.
\section{Lepisma}
\begin{itemize}
\item {Grp. gram.:m.}
\end{itemize}
\begin{itemize}
\item {Utilização:Bot.}
\end{itemize}
Escama membranosa, na base do ovário de certas plantas.
\section{Lepista}
\begin{itemize}
\item {Grp. gram.:f.}
\end{itemize}
\begin{itemize}
\item {Proveniência:(Lat. \textunderscore lepista\textunderscore )}
\end{itemize}
Espécie de vaso para beber, nos templos dos Gregos e Romanos.
\section{Lepístoma}
\begin{itemize}
\item {Grp. gram.:f.}
\end{itemize}
Gênero de plantas asclepiádeas.
\section{Lepóide}
\begin{itemize}
\item {Grp. gram.:m.}
\end{itemize}
\begin{itemize}
\item {Proveniência:(Do gr. \textunderscore lepis\textunderscore  + \textunderscore eidos\textunderscore )}
\end{itemize}
Pequena crosta, que se produz nas faces dos velhos, especialmente dos que menos cuidam da limpeza da pelle.
É também conhecida por \textunderscore funcho\textunderscore .
\section{Leporária}
\begin{itemize}
\item {Grp. gram.:adj. f.}
\end{itemize}
\begin{itemize}
\item {Proveniência:(De \textunderscore laporário\textunderscore )}
\end{itemize}
Diz-se de uma espécie de águia. Cf. B. Pereira, \textunderscore Prosódia\textunderscore , vb. \textunderscore melanaetos\textunderscore .
\section{Leporário}
\begin{itemize}
\item {Grp. gram.:adj.}
\end{itemize}
\begin{itemize}
\item {Utilização:Des.}
\end{itemize}
\begin{itemize}
\item {Proveniência:(Lat. \textunderscore leporarius\textunderscore )}
\end{itemize}
Relativo a lebre.
Que tem côr de lebre.
Dizia-se, entre os Romanos, de uma espécie de uva muito apreciada. Cf. Castilho, \textunderscore Geórg.\textunderscore , 81.
\section{Lepóride}
\begin{itemize}
\item {Grp. gram.:m.}
\end{itemize}
\begin{itemize}
\item {Proveniência:(T. fr.)}
\end{itemize}
Animal hýbrido, procedente do cóito de coêlho com lebre.
\section{Leporídeos}
\begin{itemize}
\item {Grp. gram.:m. pl.}
\end{itemize}
\begin{itemize}
\item {Proveniência:(Do lat. \textunderscore lepus\textunderscore , \textunderscore leporis\textunderscore  + gr. \textunderscore eidos\textunderscore )}
\end{itemize}
Família de mammíferos, que tem por typo o gênero lebre.
\section{Lepórido}
\begin{itemize}
\item {Grp. gram.:m.}
\end{itemize}
O mesmo ou melhor que \textunderscore lepóride\textunderscore .
\section{Leporino}
\begin{itemize}
\item {Grp. gram.:adj.}
\end{itemize}
\begin{itemize}
\item {Proveniência:(Lat. \textunderscore leporinus\textunderscore )}
\end{itemize}
Relativo á lebre.
\section{Lepra}
\begin{itemize}
\item {Grp. gram.:f.}
\end{itemize}
\begin{itemize}
\item {Utilização:Fig.}
\end{itemize}
\begin{itemize}
\item {Proveniência:(Lat. \textunderscore lepra\textunderscore )}
\end{itemize}
O mesmo que \textunderscore elephantíase\textunderscore  ou \textunderscore morpheia\textunderscore .
Doença de pelle, caracterizada por pequenas protuberâncias sólidas, cercadas de manchas roxas e luzidias.
Impureza, que os metaes adquirem na terra.
Vício ou costume condemnável, que domina uma classe ou grande porção de pessôas.
\section{Leprologia}
\begin{itemize}
\item {Grp. gram.:f.}
\end{itemize}
\begin{itemize}
\item {Proveniência:(De \textunderscore leprólogo\textunderscore )}
\end{itemize}
Tratado médico á cêrca da lepra.
\section{Leprólogo}
\begin{itemize}
\item {Grp. gram.:m.}
\end{itemize}
\begin{itemize}
\item {Proveniência:(Do gr. \textunderscore lepra\textunderscore  + \textunderscore logos\textunderscore )}
\end{itemize}
Aquelle que é perito em leprologia.
\section{Leprona}
\begin{itemize}
\item {Grp. gram.:m.}
\end{itemize}
Tumor, produzido pela lepra.
\section{Leprosaria}
\begin{itemize}
\item {Grp. gram.:f.}
\end{itemize}
Hospital de leprosos; gafaria.
\section{Leproso}
\begin{itemize}
\item {Grp. gram.:adj.}
\end{itemize}
\begin{itemize}
\item {Utilização:Ext.}
\end{itemize}
\begin{itemize}
\item {Utilização:Fig.}
\end{itemize}
\begin{itemize}
\item {Grp. gram.:M.}
\end{itemize}
Que tem lepra.
Repugnante, que inspira nojo.
Vicioso.
Aquelle que tem lepra.
\section{Leprosório}
\begin{itemize}
\item {Grp. gram.:m.}
\end{itemize}
\begin{itemize}
\item {Utilização:Des.}
\end{itemize}
\begin{itemize}
\item {Proveniência:(De \textunderscore leproso\textunderscore )}
\end{itemize}
Hospital, onde os leprosos se abrigam e são tratados. Cf. Viterbo, \textunderscore Elucid.\textunderscore 
\section{Leptálea}
\begin{itemize}
\item {Grp. gram.:f.}
\end{itemize}
\begin{itemize}
\item {Proveniência:(Do gr. \textunderscore leptaleos\textunderscore )}
\end{itemize}
Planta crucífera da Pérsia e da Sibéria.
\section{Leptinite}
\begin{itemize}
\item {Grp. gram.:f.}
\end{itemize}
\begin{itemize}
\item {Utilização:Miner.}
\end{itemize}
Espécie de feldspatho granuloso.
\section{Leptíntico}
\begin{itemize}
\item {Grp. gram.:adj.}
\end{itemize}
\begin{itemize}
\item {Utilização:Med.}
\end{itemize}
\begin{itemize}
\item {Utilização:ant.}
\end{itemize}
\begin{itemize}
\item {Proveniência:(Gr. \textunderscore leptuntikos\textunderscore )}
\end{itemize}
Dizia-se do processo ou medicamento, que servia para attenuar, ou abrandar, ou adelgaçar.
\section{Lepto}
\begin{itemize}
\item {Grp. gram.:m.}
\end{itemize}
\begin{itemize}
\item {Proveniência:(Do gr. \textunderscore leptos\textunderscore )}
\end{itemize}
Animálculo aracnídeo.
\section{Leptocárdeos}
\begin{itemize}
\item {Grp. gram.:m. pl.}
\end{itemize}
\begin{itemize}
\item {Utilização:Zool.}
\end{itemize}
Uma das seis subdivisões dos peixes, segundo Müller.
\section{Leptocardianos}
\begin{itemize}
\item {Grp. gram.:m. pl.}
\end{itemize}
Uma das três subclasses dos peixes, segundo Clauss.
\section{Leptoclase}
\begin{itemize}
\item {Grp. gram.:f.}
\end{itemize}
\begin{itemize}
\item {Proveniência:(Do gr. \textunderscore leptos\textunderscore  + \textunderscore klao\textunderscore )}
\end{itemize}
Pequena fractura natural de rocha, uma das fórmas da lithoclase.
\section{Leptodonte}
\begin{itemize}
\item {Grp. gram.:adj.}
\end{itemize}
\begin{itemize}
\item {Utilização:Zool.}
\end{itemize}
\begin{itemize}
\item {Grp. gram.:M.}
\end{itemize}
\begin{itemize}
\item {Proveniência:(Do gr. \textunderscore leptos\textunderscore  + \textunderscore odous\textunderscore , \textunderscore odonthos\textunderscore )}
\end{itemize}
Que tem dentes miúdos.
Gênero de musgos.
\section{Leptofila}
\begin{itemize}
\item {Grp. gram.:f.}
\end{itemize}
\begin{itemize}
\item {Utilização:Bot.}
\end{itemize}
\begin{itemize}
\item {Proveniência:(Do gr. \textunderscore leptos\textunderscore  + \textunderscore phullon\textunderscore )}
\end{itemize}
Planta, que tem fôlhas delgadas.
\section{Leptologia}
\begin{itemize}
\item {Grp. gram.:f.}
\end{itemize}
\begin{itemize}
\item {Proveniência:(Do gr. \textunderscore leptos\textunderscore  + \textunderscore logos\textunderscore )}
\end{itemize}
Discurso subtil; modo de discorrer minuciosamente.
\section{Leptomeningite}
\begin{itemize}
\item {Grp. gram.:f.}
\end{itemize}
\begin{itemize}
\item {Utilização:Med.}
\end{itemize}
Inflammação da piamáter.
\section{Leptomórfico}
\begin{itemize}
\item {Grp. gram.:adj.}
\end{itemize}
\begin{itemize}
\item {Utilização:Geol.}
\end{itemize}
\begin{itemize}
\item {Proveniência:(Do gr. \textunderscore leptos\textunderscore  + \textunderscore morphe\textunderscore )}
\end{itemize}
Diz-se do cristal miúdo.
\section{Leptomórphico}
\begin{itemize}
\item {Grp. gram.:adj.}
\end{itemize}
\begin{itemize}
\item {Utilização:Geol.}
\end{itemize}
\begin{itemize}
\item {Proveniência:(Do gr. \textunderscore leptos\textunderscore  + \textunderscore morphe\textunderscore )}
\end{itemize}
Diz-se do crystal miúdo.
\section{Leptophylla}
\begin{itemize}
\item {Grp. gram.:f.}
\end{itemize}
\begin{itemize}
\item {Utilização:Bot.}
\end{itemize}
\begin{itemize}
\item {Proveniência:(Do gr. \textunderscore leptos\textunderscore  + \textunderscore phullon\textunderscore )}
\end{itemize}
Planta, que tem fôlhas delgadas.
\section{Leptorima}
\begin{itemize}
\item {fónica:ri}
\end{itemize}
\begin{itemize}
\item {Grp. gram.:f.}
\end{itemize}
\begin{itemize}
\item {Proveniência:(Do gr. \textunderscore leptos\textunderscore  + \textunderscore rima\textunderscore )}
\end{itemize}
Producção marinha, espécie do esponja friável.
\section{Leptorrima}
\begin{itemize}
\item {Grp. gram.:f.}
\end{itemize}
\begin{itemize}
\item {Proveniência:(Do gr. \textunderscore leptos\textunderscore  + \textunderscore rima\textunderscore )}
\end{itemize}
Producção marinha, espécie do esponja friável.
\section{Leptospérmeas}
\begin{itemize}
\item {Grp. gram.:f. pl.}
\end{itemize}
\begin{itemize}
\item {Utilização:Bot.}
\end{itemize}
\begin{itemize}
\item {Proveniência:(Do gr. \textunderscore leptos\textunderscore  + \textunderscore sperma\textunderscore )}
\end{itemize}
Tríbo de plantas myrtáceas, no systema de De-Candolle.
\section{Leptotério}
\begin{itemize}
\item {Grp. gram.:m.}
\end{itemize}
\begin{itemize}
\item {Proveniência:(Do gr. \textunderscore leptos\textunderscore  + \textunderscore therion\textunderscore )}
\end{itemize}
Gênero de ruminantes fósseis.
\section{Leptothério}
\begin{itemize}
\item {Grp. gram.:m.}
\end{itemize}
\begin{itemize}
\item {Proveniência:(Do gr. \textunderscore leptos\textunderscore  + \textunderscore therion\textunderscore )}
\end{itemize}
Gênero de ruminantes fósseis.
\section{Lepturo}
\begin{itemize}
\item {Grp. gram.:m.}
\end{itemize}
\begin{itemize}
\item {Proveniência:(Do gr. \textunderscore leptos\textunderscore  + \textunderscore oura\textunderscore )}
\end{itemize}
Insecto coléoptero lignívoro.
Gênero de plantas gramíneas.
\section{Leptýntico}
\begin{itemize}
\item {Grp. gram.:adj.}
\end{itemize}
\begin{itemize}
\item {Utilização:Med.}
\end{itemize}
\begin{itemize}
\item {Utilização:ant.}
\end{itemize}
\begin{itemize}
\item {Proveniência:(Gr. \textunderscore leptuntikos\textunderscore )}
\end{itemize}
Dizia-se do processo ou medicamento, que servia para attenuar, ou abrandar, ou adelgaçar.
\section{Leque}
\begin{itemize}
\item {Grp. gram.:m.}
\end{itemize}
\begin{itemize}
\item {Utilização:Fig.}
\end{itemize}
Abanico de pano ou papel com varetas.
Objecto, que tem a apparência de leque.
Espécie de pólypo.
(Provém de \textunderscore Léqúios\textunderscore . Cp. \textunderscore Léquios\textunderscore .)
\section{Leque}
\begin{itemize}
\item {Grp. gram.:m.}
\end{itemize}
O mesmo que \textunderscore laque\textunderscore .
\section{Lequéssia}
\begin{itemize}
\item {Grp. gram.:f.}
\end{itemize}
\begin{itemize}
\item {Utilização:Bras. de Goiás}
\end{itemize}
Bebedeira.
Vadiação.
\section{Léquios}
\begin{itemize}
\item {Grp. gram.:m. pl.}
\end{itemize}
Habitantes de um archipélago, a que deram êsse nome, e que é situado entre o mar da China e o mar da Coreia.
\section{Lêr}
\begin{itemize}
\item {Grp. gram.:v. t.}
\end{itemize}
\begin{itemize}
\item {Grp. gram.:V. i.}
\end{itemize}
\begin{itemize}
\item {Utilização:Fam.}
\end{itemize}
\begin{itemize}
\item {Utilização:Fam.}
\end{itemize}
\begin{itemize}
\item {Proveniência:(Do lat. \textunderscore legere\textunderscore )}
\end{itemize}
Percorrer com a vista e conhecer (letras), reunindo estas em palavras.
Pronunciar em voz alta, recitar: \textunderscore lêr versos\textunderscore .
Vêr e estudar (coisa escrita): \textunderscore lêr um capítulo de história\textunderscore .
Decifrar: \textunderscore lêr uma inscripção\textunderscore .
Perceber.
Explicar como professor: \textunderscore lêr Theologia\textunderscore .
Vêr e interpretar o que está escrito: \textunderscore a pequena já sabe lêr\textunderscore .
Devanear; disparatar: \textunderscore estás a lêr!\textunderscore 
\textunderscore Lêr de cadeira\textunderscore , conhecer perfeitamente uma matéria e poder dar lições a respeito della.
\section{Lerca}
\begin{itemize}
\item {Grp. gram.:f.}
\end{itemize}
\begin{itemize}
\item {Utilização:Pop.}
\end{itemize}
\begin{itemize}
\item {Utilização:Prov.}
\end{itemize}
\begin{itemize}
\item {Utilização:minh.}
\end{itemize}
Vaca muito magra.
Mulher magra.
\section{Lerdaço}
\begin{itemize}
\item {Grp. gram.:adj.}
\end{itemize}
\begin{itemize}
\item {Utilização:Pop.}
\end{itemize}
\begin{itemize}
\item {Proveniência:(De \textunderscore lerdo\textunderscore )}
\end{itemize}
Parvo; pacóvio; pateta.
\section{Lerdo}
\begin{itemize}
\item {fónica:lêr}
\end{itemize}
\begin{itemize}
\item {Grp. gram.:adj.}
\end{itemize}
Tardio nos movimentos; froixo.
Estúpido; acanhado.
Grosseiro.
(Cast. \textunderscore lerdo\textunderscore )
\section{Lereia}
\begin{itemize}
\item {Grp. gram.:f.}
\end{itemize}
\begin{itemize}
\item {Utilização:Bras}
\end{itemize}
Conversa sem utilidade.
(Corr. de \textunderscore léria\textunderscore ?)
\section{Léria}
\begin{itemize}
\item {Grp. gram.:f.}
\end{itemize}
\begin{itemize}
\item {Utilização:Pop.}
\end{itemize}
\begin{itemize}
\item {Grp. gram.:M.}
\end{itemize}
\begin{itemize}
\item {Utilização:Prov.}
\end{itemize}
Falácia; arenga.
Trica.
Lengalenga.
Indivíduo, que fala muito, mas que não faz nada útil.
Palerma, asno.
\section{Lerica}
\begin{itemize}
\item {Grp. gram.:f.}
\end{itemize}
\begin{itemize}
\item {Utilização:Prov.}
\end{itemize}
\begin{itemize}
\item {Utilização:beir.}
\end{itemize}
Planta, que nasce nas searas e produz umas sementinhas pretas; ervilhaca.
\section{Lerna}
\begin{itemize}
\item {Grp. gram.:f.}
\end{itemize}
\begin{itemize}
\item {Utilização:Fig.}
\end{itemize}
\begin{itemize}
\item {Proveniência:(De \textunderscore Lerna\textunderscore , n. p.)}
\end{itemize}
Poço.
Abysmo:«\textunderscore sou uma lerna de desaventuras\textunderscore ». \textunderscore Eufrosina\textunderscore , 297.
Grande porção.
\section{Lero}
\begin{itemize}
\item {Grp. gram.:adj.}
\end{itemize}
\begin{itemize}
\item {Utilização:Prov.}
\end{itemize}
\begin{itemize}
\item {Utilização:alg.}
\end{itemize}
Vivo, esperto.
\section{Lerquinhas}
\begin{itemize}
\item {Grp. gram.:f.}
\end{itemize}
\begin{itemize}
\item {Utilização:Prov.}
\end{itemize}
\begin{itemize}
\item {Utilização:minh.}
\end{itemize}
\begin{itemize}
\item {Proveniência:(De \textunderscore lerca\textunderscore )}
\end{itemize}
Mulher magra.
\section{Lerta}
\begin{itemize}
\item {Grp. gram.:f.}
\end{itemize}
(Cp. \textunderscore alerta\textunderscore )
\section{Lés}
\begin{itemize}
\item {Grp. gram.:m.}
\end{itemize}
Us. na loc. pop. \textunderscore lés a lés\textunderscore , que quer dizer \textunderscore de uma a outra banda\textunderscore , \textunderscore de lado a lado\textunderscore .
\section{Lesador}
\begin{itemize}
\item {Grp. gram.:m.}
\end{itemize}
Aquelle que lesa.
\section{Lesa-fradaria}
\begin{itemize}
\item {Grp. gram.:f.}
\end{itemize}
\begin{itemize}
\item {Utilização:Deprec.}
\end{itemize}
\textunderscore Crime de lesa-fradaria\textunderscore , crime contra o respeito ou acatamento que os frades exigem:«\textunderscore incorreo no enorme crime de lesa-fradaria\textunderscore ». Filinto, IX, 98.
\section{Lesa-majestade}
\begin{itemize}
\item {Grp. gram.:f.}
\end{itemize}
\begin{itemize}
\item {Proveniência:(De \textunderscore lesa\textunderscore  + \textunderscore majestade\textunderscore )}
\end{itemize}
Us. na loc. \textunderscore crime de lesa-majestade\textunderscore , crime contra pessôas reaes ou contra o poder supremo da nação.
\section{Lesante}
\begin{itemize}
\item {Grp. gram.:adj.}
\end{itemize}
\begin{itemize}
\item {Grp. gram.:M.}
\end{itemize}
Que lesa.
Aquelle que lesa ou prejudica ou damnifica. Cf. V. Ferrer, \textunderscore Direito Nat.\textunderscore , 87.
\section{Lesão}
\begin{itemize}
\item {Grp. gram.:f.}
\end{itemize}
\begin{itemize}
\item {Proveniência:(Lat. \textunderscore laesío\textunderscore )}
\end{itemize}
Acto ou effeito de lesar.
Damno; prejuízo.
Alteração ou perturbação nas funcções ou textura dos órgãos do um indivíduo.
\textunderscore Lesão cardíaca\textunderscore , endocardite chrónica; apêrto ou insufficiência das válvulas cardíacas.
\section{Lesa-poesia}
\begin{itemize}
\item {Grp. gram.:f.}
\end{itemize}
\textunderscore Crime de lesa-poesia\textunderscore , attentado contra as leis ou regras da poesia, contra o bom gôsto poético. Cf. Camillo, \textunderscore Myst. de Lisb.\textunderscore , I, 149.
\section{Lesar}
\begin{itemize}
\item {Grp. gram.:v. t.}
\end{itemize}
\begin{itemize}
\item {Grp. gram.:V. i.}
\end{itemize}
\begin{itemize}
\item {Utilização:Bras. do N}
\end{itemize}
\begin{itemize}
\item {Proveniência:(De \textunderscore leso\textunderscore )}
\end{itemize}
Molestar; contundir; ferir.
Offender o crédito, a reputação ou os interesses de.
Tornar-se idiota, pateta.
\section{Lesbíaco}
\begin{itemize}
\item {Grp. gram.:adj.}
\end{itemize}
\begin{itemize}
\item {Proveniência:(De \textunderscore Lesbos\textunderscore , n. p.)}
\end{itemize}
Relativo á ilha de Lesbos.
Em poética, dizia-se do metro ou medida dos versos sáphicos; lýrico.
\section{Lesbianismo}
\begin{itemize}
\item {Grp. gram.:m.}
\end{itemize}
\begin{itemize}
\item {Proveniência:(De \textunderscore lesbiano\textunderscore )}
\end{itemize}
Um dos vícios sensuaes contra a natureza.
Aberração do instinto sexual.
\section{Lésbiano}
\begin{itemize}
\item {Grp. gram.:adj.}
\end{itemize}
\begin{itemize}
\item {Utilização:Fig.}
\end{itemize}
O mesmo que \textunderscore lesbíaco\textunderscore .
Dissoluto.
\section{Lésbico}
\begin{itemize}
\item {Grp. gram.:adj.}
\end{itemize}
O mesmo que \textunderscore lésbio\textunderscore .
\section{Lésbio}
\begin{itemize}
\item {Grp. gram.:adj.}
\end{itemize}
\begin{itemize}
\item {Grp. gram.:M.}
\end{itemize}
O mesmo que \textunderscore lesbíaco\textunderscore .
Habitante de Lesbos.
\section{Leseira}
\begin{itemize}
\item {Grp. gram.:f.}
\end{itemize}
\begin{itemize}
\item {Utilização:Bras. de Pernambuco}
\end{itemize}
\begin{itemize}
\item {Proveniência:(De \textunderscore leso\textunderscore )}
\end{itemize}
Palermice, toleima.
\section{Lésguio}
\begin{itemize}
\item {Grp. gram.:m.}
\end{itemize}
Língua caucásica, que não tem literatura.
\section{Lesim}
\begin{itemize}
\item {Grp. gram.:m.}
\end{itemize}
\begin{itemize}
\item {Proveniência:(Do rad. do lat. \textunderscore laesio\textunderscore )}
\end{itemize}
Veio da madeira.
Pequeno fio ou sulco natural de algumas pedras e mármores.
\section{Lesinzado}
\begin{itemize}
\item {Grp. gram.:adj.}
\end{itemize}
\begin{itemize}
\item {Utilização:Prov.}
\end{itemize}
Cheio de lesins.
\section{Lesivo}
\begin{itemize}
\item {Grp. gram.:adj.}
\end{itemize}
Que causa lesão.
Que lesa.
\section{Lesma}
\begin{itemize}
\item {fónica:lês}
\end{itemize}
\begin{itemize}
\item {Grp. gram.:f.}
\end{itemize}
\begin{itemize}
\item {Utilização:pop.}
\end{itemize}
\begin{itemize}
\item {Utilização:Fig.}
\end{itemize}
\begin{itemize}
\item {Utilização:Prov.}
\end{itemize}
\begin{itemize}
\item {Utilização:trasm.}
\end{itemize}
\begin{itemize}
\item {Utilização:Ant.}
\end{itemize}
Mollusco gasterópode.
Pessôa indolente; pessôa insípida.
Pessôa magra.
Coisa muito mimosa:«\textunderscore doeu-vos muito a picada, minha lesma?\textunderscore »M. Bernárdez, referindo-se, numa espécie de jaculatória, ao menino Deus circuncidado.
\section{Lesmida}
\begin{itemize}
\item {Grp. gram.:f.}
\end{itemize}
\begin{itemize}
\item {Utilização:Prov.}
\end{itemize}
\begin{itemize}
\item {Utilização:minh.}
\end{itemize}
\begin{itemize}
\item {Proveniência:(De \textunderscore lesma\textunderscore )}
\end{itemize}
Mulher presumida e nada sympáthica.
\section{Lés-nordeste}
\begin{itemize}
\item {Grp. gram.:m.}
\end{itemize}
\begin{itemize}
\item {Proveniência:(De \textunderscore léste\textunderscore  + \textunderscore nordeste\textunderscore )}
\end{itemize}
Vento de entre nordeste e léste.
\section{Leso}
\begin{itemize}
\item {Grp. gram.:adj.}
\end{itemize}
\begin{itemize}
\item {Utilização:Bras. de Pernambuco}
\end{itemize}
\begin{itemize}
\item {Proveniência:(Lat. \textunderscore laesus\textunderscore )}
\end{itemize}
Confuso; ferido.
Paralýtico.
Offendido; violado.
Apalermado, atoleimado.
\section{Lessole}
\begin{itemize}
\item {Grp. gram.:m.}
\end{itemize}
Pano, com que os indígenas ambaquistas cobrem as partes pudendas.
\section{Lés-sueste}
\begin{itemize}
\item {Grp. gram.:m.}
\end{itemize}
\begin{itemize}
\item {Proveniência:(De \textunderscore léste\textunderscore  + \textunderscore sueste\textunderscore )}
\end{itemize}
O mesmo que \textunderscore essueste\textunderscore .
\section{Lestada}
\begin{itemize}
\item {Grp. gram.:f.}
\end{itemize}
Vento forte de léste.
Vento persistente, que sopra de léste.
\section{Léste}
\begin{itemize}
\item {Grp. gram.:m.}
\end{itemize}
\begin{itemize}
\item {Proveniência:(Do fr. \textunderscore l'est\textunderscore )}
\end{itemize}
O mesmo que \textunderscore éste\textunderscore ; oriente; nascente; levante.
Vento, que sopra do lado do nascente.
\section{Léstes}
\begin{itemize}
\item {Grp. gram.:adj.}
\end{itemize}
\begin{itemize}
\item {Utilização:Des.}
\end{itemize}
(V.lesto)Cf. \textunderscore Eufrosina\textunderscore , 334.
\section{Lèstia}
\begin{itemize}
\item {Grp. gram.:f.}
\end{itemize}
Vento de léste. Cf. B. Pato, \textunderscore Livro do Monte\textunderscore .
\section{Lesto}
\begin{itemize}
\item {Grp. gram.:adj.}
\end{itemize}
\begin{itemize}
\item {Proveniência:(Do al. \textunderscore listig\textunderscore ?)}
\end{itemize}
Ligeiro; ágil.
Rápido.
Activo; desembaraçado.
Aliviado.
\section{Lestras}
\begin{itemize}
\item {Grp. gram.:f. pl.}
\end{itemize}
Espécie de junco odorífero, (\textunderscore anthoxanthum amarum\textunderscore , Brot.).
\section{Léo}
\begin{itemize}
\item {Grp. gram.:m.}
\end{itemize}
\begin{itemize}
\item {Utilização:Pop.}
\end{itemize}
\begin{itemize}
\item {Grp. gram.:Loc. adv.}
\end{itemize}
\begin{itemize}
\item {Proveniência:(Do lat. \textunderscore levis\textunderscore )}
\end{itemize}
Ócio; tuna.
\textunderscore Ao léu\textunderscore , sem chapéu; nuamente.
\section{Lestrigões}
\begin{itemize}
\item {Grp. gram.:m. pl.}
\end{itemize}
\begin{itemize}
\item {Proveniência:(Lat. \textunderscore lestrygones\textunderscore )}
\end{itemize}
Povo da Itália meridional, que os antigos poétas nos apresentavam como anthropóphagos.
\section{Lestrygões}
\begin{itemize}
\item {Grp. gram.:m. pl.}
\end{itemize}
\begin{itemize}
\item {Proveniência:(Lat. \textunderscore lestrygones\textunderscore )}
\end{itemize}
Povo da Itália meridional, que os antigos poétas nos apresentavam como anthropóphagos.
\section{Letal}
\begin{itemize}
\item {Grp. gram.:adj.}
\end{itemize}
\begin{itemize}
\item {Proveniência:(Lat. \textunderscore letalis\textunderscore )}
\end{itemize}
Relativo á morte; mortal; lúgubre; fatídico.
\section{Letalidade}
\begin{itemize}
\item {Grp. gram.:f.}
\end{itemize}
\begin{itemize}
\item {Utilização:Neol.}
\end{itemize}
\begin{itemize}
\item {Proveniência:(Lat. \textunderscore letalitas\textunderscore )}
\end{itemize}
Qualidade daquillo que é letal.
Conjunto de óbitos; mortalidade.
\section{Letalmente}
\begin{itemize}
\item {Grp. gram.:adv.}
\end{itemize}
De modo letal.
Mortalmente.
\section{Letão}
\begin{itemize}
\item {Grp. gram.:m.}
\end{itemize}
Dialecto dos Letões.
\section{Letargia}
\begin{itemize}
\item {Grp. gram.:f.}
\end{itemize}
\begin{itemize}
\item {Utilização:Fig.}
\end{itemize}
\begin{itemize}
\item {Proveniência:(Lat. \textunderscore lethargia\textunderscore )}
\end{itemize}
Sono profundo, em que parece suspensa a circulação e a respiração.
Apatia; prostração moral.
\section{Letargiar}
\begin{itemize}
\item {Grp. gram.:v. t.}
\end{itemize}
Causar letargo a.
Lançar ou prostrar em letargo. Cf. Camillo, \textunderscore Scen. da Hora Fin.\textunderscore , 22.
\section{Letargicamente}
\begin{itemize}
\item {Grp. gram.:adv.}
\end{itemize}
De modo letárgico.
Á maneira de letargo.
\section{Letárgico}
\begin{itemize}
\item {Grp. gram.:adj.}
\end{itemize}
\begin{itemize}
\item {Grp. gram.:M.}
\end{itemize}
\begin{itemize}
\item {Proveniência:(Lat. \textunderscore lethargicus\textunderscore )}
\end{itemize}
Relativo a letargia.
Dormente.
Preguiçoso; indolente.
Aquele que caíu em letargia.
\section{Letargo}
\begin{itemize}
\item {Grp. gram.:m.}
\end{itemize}
\begin{itemize}
\item {Utilização:Fig.}
\end{itemize}
\begin{itemize}
\item {Proveniência:(Lat. \textunderscore lethargus\textunderscore )}
\end{itemize}
O mesmo que \textunderscore letargia\textunderscore .
Torpor.
Indolência.
Inércia; apatia.
Incerteza.
Esquecimento.
\section{Lêtera}
\begin{itemize}
\item {Grp. gram.:f.}
\end{itemize}
\begin{itemize}
\item {Utilização:Ant.}
\end{itemize}
O mesmo que \textunderscore letra\textunderscore . Cf. R. Pina \textunderscore Chrón. de D. João II\textunderscore , (passim).
\section{Leterado}
\begin{itemize}
\item {Grp. gram.:m.}
\end{itemize}
\begin{itemize}
\item {Utilização:Ant.}
\end{itemize}
O mesmo que \textunderscore letrado\textunderscore . Cf. R. Pina, \textunderscore Chrón. de D. João II\textunderscore , c. XX.
\section{Leteu}
\begin{itemize}
\item {Grp. gram.:adj.}
\end{itemize}
\begin{itemize}
\item {Utilização:Poét.}
\end{itemize}
\begin{itemize}
\item {Proveniência:(Lat. \textunderscore lethaeus\textunderscore )}
\end{itemize}
Relativo ao Lethes.
Infernal.
\section{Lethargia}
\begin{itemize}
\item {Grp. gram.:f.}
\end{itemize}
\begin{itemize}
\item {Utilização:Fig.}
\end{itemize}
\begin{itemize}
\item {Proveniência:(Lat. \textunderscore lethargia\textunderscore )}
\end{itemize}
Somno profundo, em que parece suspensa a circulação e a respiração.
Apathia; prostração moral.
\section{Lethargiar}
\begin{itemize}
\item {Grp. gram.:v. t.}
\end{itemize}
Causar lethargo a.
Lançar ou prostrar em lethargo. Cf. Camillo, \textunderscore Scen. da Hora Fin.\textunderscore , 22.
\section{Lethargicamente}
\begin{itemize}
\item {Grp. gram.:adv.}
\end{itemize}
De modo lethárgico.
Á maneira de lethargo.
\section{Lethárgico}
\begin{itemize}
\item {Grp. gram.:adj.}
\end{itemize}
\begin{itemize}
\item {Grp. gram.:M.}
\end{itemize}
\begin{itemize}
\item {Proveniência:(Lat. \textunderscore lethargicus\textunderscore )}
\end{itemize}
Relativo a lethargia.
Dormente.
Preguiçoso; indolente.
Aquelle que caiu em lethargia.
\section{Lethargo}
\begin{itemize}
\item {Grp. gram.:m.}
\end{itemize}
\begin{itemize}
\item {Utilização:Fig.}
\end{itemize}
\begin{itemize}
\item {Proveniência:(Lat. \textunderscore lethargus\textunderscore )}
\end{itemize}
O mesmo que \textunderscore lethargia\textunderscore .
Torpor.
Indolência.
Inércia; apathia.
Incerteza.
Esquecimento.
\section{Letheu}
\begin{itemize}
\item {Grp. gram.:adj.}
\end{itemize}
\begin{itemize}
\item {Utilização:Poét.}
\end{itemize}
\begin{itemize}
\item {Proveniência:(Lat. \textunderscore lethaeus\textunderscore )}
\end{itemize}
Relativo ao Lethes.
Infernal.
\section{Lethro}
\begin{itemize}
\item {Grp. gram.:m.}
\end{itemize}
Gênero de insectos coleópteros, da fam. dos lamellicórneos.
\section{Letícia}
\begin{itemize}
\item {Grp. gram.:f.}
\end{itemize}
\begin{itemize}
\item {Utilização:Poét.}
\end{itemize}
\begin{itemize}
\item {Proveniência:(Lat. \textunderscore laeticia\textunderscore )}
\end{itemize}
Ledice, alegria.
\section{Letícia}
\begin{itemize}
\item {Grp. gram.:f.}
\end{itemize}
\begin{itemize}
\item {Proveniência:(De \textunderscore Letícia\textunderscore , n. p.)}
\end{itemize}
Planeta telescópico, descoberto em 1856.
\section{Lético}
\begin{itemize}
\item {Grp. gram.:adj.}
\end{itemize}
\begin{itemize}
\item {Grp. gram.:M.}
\end{itemize}
Relativo aos Letões ou ao seu dialecto.
O dialecto dos Letões.
\section{Letífero}
\begin{itemize}
\item {Grp. gram.:adj.}
\end{itemize}
\begin{itemize}
\item {Proveniência:(Lat. \textunderscore letifer\textunderscore )}
\end{itemize}
O mesmo que \textunderscore letal\textunderscore .
\section{Letificante}
\begin{itemize}
\item {Grp. gram.:adj.}
\end{itemize}
\begin{itemize}
\item {Proveniência:(Lat. \textunderscore laetificans\textunderscore )}
\end{itemize}
O mesmo que \textunderscore letífico\textunderscore ^2.
\section{Letificar}
\begin{itemize}
\item {Grp. gram.:v. t.}
\end{itemize}
\begin{itemize}
\item {Proveniência:(Lat. \textunderscore laetificare\textunderscore )}
\end{itemize}
Tornar alegre ou ledo.
\section{Letífico}
\begin{itemize}
\item {Grp. gram.:adj.}
\end{itemize}
\begin{itemize}
\item {Utilização:Poét.}
\end{itemize}
\begin{itemize}
\item {Proveniência:(Lat. \textunderscore letificus\textunderscore )}
\end{itemize}
O mesmo que \textunderscore letal\textunderscore .
\section{Letífico}
\begin{itemize}
\item {Grp. gram.:adj.}
\end{itemize}
\begin{itemize}
\item {Utilização:Poét.}
\end{itemize}
\begin{itemize}
\item {Proveniência:(Lat. laetificus)}
\end{itemize}
Que produz ledice; que letifica.
\section{Letões}
\begin{itemize}
\item {Grp. gram.:m. pl.}
\end{itemize}
Povos do noroéste da Rússia.
\section{Letomania}
\begin{itemize}
\item {Grp. gram.:f.}
\end{itemize}
\begin{itemize}
\item {Utilização:Des.}
\end{itemize}
\begin{itemize}
\item {Proveniência:(Do lat. \textunderscore letum\textunderscore  + \textunderscore mania\textunderscore )}
\end{itemize}
Monomania do suicídio.
\section{Letos}
\begin{itemize}
\item {Grp. gram.:m. pl.}
\end{itemize}
Bárbaros do Norte, os quaes, tendo entrado nos domínios romanos, tiveram concessões de terras, sob a condição de servirem no exército.
(B. lat. \textunderscore leti\textunderscore )
\section{Letra}
\begin{itemize}
\item {fónica:lê}
\end{itemize}
\begin{itemize}
\item {Grp. gram.:f.}
\end{itemize}
\begin{itemize}
\item {Utilização:Prov.}
\end{itemize}
\begin{itemize}
\item {Utilização:fam.}
\end{itemize}
\begin{itemize}
\item {Grp. gram.:Loc. adv.}
\end{itemize}
\begin{itemize}
\item {Grp. gram.:Pl.}
\end{itemize}
\begin{itemize}
\item {Proveniência:(Do lat. \textunderscore litera\textunderscore )}
\end{itemize}
Cada um dos caracteres do alphabeto.
Fórma de escrever os caracteres alphabéticos: \textunderscore a Dora tem bonita letra\textunderscore .
Inscripção.
O som representado por cada um dos caracteres alphabéticos.
Aquillo que está escrito.
Sentido claramente expresso pelo que se escreve.
Versos, correspondentes a uma cantiga ou música.
Papel representativo de moéda: \textunderscore sacar uma letra\textunderscore .
Léria, bazófia. (Colhido nas Caldas da Raínha)
\textunderscore Letra redonda\textunderscore , letra de imprensa.
\textunderscore Letra morta\textunderscore , preceito escrito, que não chegou a cumprir-se ou que deixou de se cumprir.
\textunderscore Á letra\textunderscore , literalmente; rigorosamente; com exactidão.
Carta: \textunderscore cá recebi as suas letras\textunderscore .
Literatura: \textunderscore tornou-se notável nas letras portuguesas\textunderscore .
\section{Letrache}
\begin{itemize}
\item {Grp. gram.:m.}
\end{itemize}
Apparelho de pesca, usado no Guadiana.
\section{Letradal}
\begin{itemize}
\item {Grp. gram.:adj.}
\end{itemize}
\begin{itemize}
\item {Utilização:P. us.}
\end{itemize}
Relativo a letrado.
Próprio de letrado. Cf. Cortesão, \textunderscore Subs.\textunderscore 
\section{Letradete}
\begin{itemize}
\item {fónica:dê}
\end{itemize}
\begin{itemize}
\item {Grp. gram.:adj.}
\end{itemize}
Um tanto letrado. Cf. Filinto, XII, 264.
\section{Letradice}
\begin{itemize}
\item {Grp. gram.:f.}
\end{itemize}
\begin{itemize}
\item {Utilização:Deprec.}
\end{itemize}
Presumpção de letrado.
Prosápia.
Bacharelice. Cf. Camillo, \textunderscore Narcóticos\textunderscore , II, 112.
\section{Letrado}
\begin{itemize}
\item {Grp. gram.:adj.}
\end{itemize}
\begin{itemize}
\item {Grp. gram.:M.}
\end{itemize}
\begin{itemize}
\item {Proveniência:(Lat. \textunderscore literatus\textunderscore )}
\end{itemize}
Que é versado em letras; erudito.
Literato.
Jurisconsulto.
\section{Letradura}
\begin{itemize}
\item {Grp. gram.:f.}
\end{itemize}
O mesmo que \textunderscore literatura\textunderscore .
\section{Letramento}
\begin{itemize}
\item {Grp. gram.:m.}
\end{itemize}
\begin{itemize}
\item {Utilização:Ant.}
\end{itemize}
\begin{itemize}
\item {Proveniência:(De \textunderscore letra\textunderscore )}
\end{itemize}
O mesmo que \textunderscore escrita\textunderscore .
\section{Letrear}
\begin{itemize}
\item {Grp. gram.:v. t.}
\end{itemize}
O mesmo que \textunderscore deletrear\textunderscore .
\section{Letreiro}
\begin{itemize}
\item {Grp. gram.:m.}
\end{itemize}
\begin{itemize}
\item {Proveniência:(De \textunderscore letra\textunderscore )}
\end{itemize}
Legenda; rótulo; inscripção: \textunderscore os letreiros das ruas\textunderscore .
\section{Letria}
\begin{itemize}
\item {Grp. gram.:f.}
\end{itemize}
O mesmo que \textunderscore aletria\textunderscore .
Certos ornatos em olaria.
\section{Letrilha}
\begin{itemize}
\item {Grp. gram.:f.}
\end{itemize}
\begin{itemize}
\item {Utilização:Des.}
\end{itemize}
\begin{itemize}
\item {Proveniência:(De \textunderscore letra\textunderscore )}
\end{itemize}
Pequena poesia para canto; coplas.
\section{Letro}
\begin{itemize}
\item {Grp. gram.:m.}
\end{itemize}
Gênero de insectos coleópteros, da fam. dos lamelicórneos.
\section{Letrudo}
\begin{itemize}
\item {Grp. gram.:m.  e  adj.}
\end{itemize}
\begin{itemize}
\item {Utilização:Chul.}
\end{itemize}
O mesmo que \textunderscore letrado\textunderscore .
\section{Letsómia}
\begin{itemize}
\item {Grp. gram.:f.}
\end{itemize}
\begin{itemize}
\item {Proveniência:(De \textunderscore Letsom\textunderscore , n. p.)}
\end{itemize}
Gênero de plantas peruvianas.
\section{Lettão}
\begin{itemize}
\item {Grp. gram.:m.}
\end{itemize}
Dialecto dos Lettões.
\section{Léttico}
\begin{itemize}
\item {Grp. gram.:adj.}
\end{itemize}
\begin{itemize}
\item {Grp. gram.:M.}
\end{itemize}
Relativo aos Lettões ou ao seu dialecto.
O dialecto dos Lettões.
\section{Lettões}
\begin{itemize}
\item {Grp. gram.:m. pl.}
\end{itemize}
Povos do noroéste da Rússia.
\section{Léu}
\begin{itemize}
\item {Grp. gram.:m.}
\end{itemize}
\begin{itemize}
\item {Utilização:Pop.}
\end{itemize}
\begin{itemize}
\item {Grp. gram.:Loc. adv.}
\end{itemize}
\begin{itemize}
\item {Proveniência:(Do lat. \textunderscore levis\textunderscore )}
\end{itemize}
Ócio; tuna.
\textunderscore Ao léu\textunderscore , sem chapéu; nuamente.
\section{Leucacanta}
\begin{itemize}
\item {Proveniência:(Lat. \textunderscore leucacantha\textunderscore )}
\end{itemize}
\textunderscore f.\textunderscore 
Designação antiga do pilriteiro.
\section{Leucacantha}
\begin{itemize}
\item {Grp. gram.:f.}
\end{itemize}
\begin{itemize}
\item {Proveniência:(Lat. \textunderscore leucacantha\textunderscore )}
\end{itemize}
Designação antiga do pilriteiro.
\section{Leucacantho}
\begin{itemize}
\item {Grp. gram.:m.}
\end{itemize}
\begin{itemize}
\item {Proveniência:(Lat. \textunderscore leucacantha\textunderscore )}
\end{itemize}
Designação antiga do pilriteiro.
\section{Leucacanto}
\begin{itemize}
\item {Grp. gram.:m.}
\end{itemize}
\begin{itemize}
\item {Proveniência:(Lat. \textunderscore leucacantha\textunderscore )}
\end{itemize}
Designação antiga do pilriteiro.
\section{Leucânea}
\begin{itemize}
\item {Grp. gram.:f.}
\end{itemize}
\begin{itemize}
\item {Proveniência:(Do rad. do gr. \textunderscore leukos\textunderscore )}
\end{itemize}
Gênero de insectos lepidópteros nocturnos.
\section{Leucantho}
\begin{itemize}
\item {Grp. gram.:adj.}
\end{itemize}
\begin{itemize}
\item {Utilização:Bot.}
\end{itemize}
\begin{itemize}
\item {Proveniência:(Do gr. \textunderscore leukos\textunderscore  + \textunderscore anthos\textunderscore )}
\end{itemize}
Que tem ou produz flôres brancas.
\section{Leucanto}
\begin{itemize}
\item {Grp. gram.:adj.}
\end{itemize}
\begin{itemize}
\item {Utilização:Bot.}
\end{itemize}
\begin{itemize}
\item {Proveniência:(Do gr. \textunderscore leukos\textunderscore  + \textunderscore anthos\textunderscore )}
\end{itemize}
Que tem ou produz flôres brancas.
\section{Leucemia}
\begin{itemize}
\item {Grp. gram.:f.}
\end{itemize}
\begin{itemize}
\item {Proveniência:(Do gr. \textunderscore leukos\textunderscore  + \textunderscore haima\textunderscore )}
\end{itemize}
Doença, caracterizada principalmente pelo aumento dos glóbulos brancos do sangue.
\section{Leuchtenbergite}
\begin{itemize}
\item {Grp. gram.:f.}
\end{itemize}
\begin{itemize}
\item {Utilização:Geol.}
\end{itemize}
\begin{itemize}
\item {Proveniência:(De \textunderscore Leuchtenberg\textunderscore , n. p.)}
\end{itemize}
Espécie de chlorite amarela.
\section{Lêucico}
\begin{itemize}
\item {Grp. gram.:adj.}
\end{itemize}
\begin{itemize}
\item {Proveniência:(Do gr. \textunderscore leukos\textunderscore )}
\end{itemize}
Diz-se de um ácido, extrahido da leucina.
\section{Leucina}
\begin{itemize}
\item {Grp. gram.:f.}
\end{itemize}
\begin{itemize}
\item {Proveniência:(Do gr. \textunderscore leukos\textunderscore )}
\end{itemize}
Substância branca e crystallina, fusível, solúvel na água.
Princípio, que existe no tecido pulmonar e no sangue.
\section{Leucisco}
\begin{itemize}
\item {Grp. gram.:m.}
\end{itemize}
\begin{itemize}
\item {Proveniência:(Do gr. \textunderscore leukos\textunderscore )}
\end{itemize}
Gênero de peixes, a que pertence o bordalo e o esqualo.
\section{Leucite}
\begin{itemize}
\item {Grp. gram.:f.}
\end{itemize}
\begin{itemize}
\item {Proveniência:(Do gr. \textunderscore leukos\textunderscore )}
\end{itemize}
Espécie de felspathoide, de lustre vítreo e côr branca ou acinzentada.
\section{Leucítica}
\begin{itemize}
\item {Grp. gram.:adj. f.}
\end{itemize}
\begin{itemize}
\item {Proveniência:(De \textunderscore leucite\textunderscore )}
\end{itemize}
Diz-se de uma variedade de lava. Cf. Flaviense, \textunderscore Diccion. Geog.\textunderscore 
\section{Leucitito}
\begin{itemize}
\item {Grp. gram.:m.}
\end{itemize}
Variedade de basalto, em que predomina a leucite.
\section{Leucito}
\begin{itemize}
\item {Grp. gram.:m.}
\end{itemize}
\begin{itemize}
\item {Proveniência:(Do gr. \textunderscore leukos\textunderscore )}
\end{itemize}
Grânulo esphérico ou ovóide, com actividade própria, e contido na céllula vegetal.
\section{Leucitofiro}
\begin{itemize}
\item {Grp. gram.:m.}
\end{itemize}
\begin{itemize}
\item {Utilização:Miner.}
\end{itemize}
Espécie de rocha basáltica, em que predomina a leucite.
\section{Leucitophyro}
\begin{itemize}
\item {Grp. gram.:m.}
\end{itemize}
\begin{itemize}
\item {Utilização:Miner.}
\end{itemize}
Espécie de rocha basáltica, em que predomina a leucite.
\section{Leuco...}
\begin{itemize}
\item {Grp. gram.:pref.}
\end{itemize}
\begin{itemize}
\item {Proveniência:(Do gr. \textunderscore leukos\textunderscore )}
\end{itemize}
(designativo de \textunderscore branco\textunderscore )
\section{Leucocarpo}
\begin{itemize}
\item {Grp. gram.:adj.}
\end{itemize}
\begin{itemize}
\item {Utilização:Bot.}
\end{itemize}
\begin{itemize}
\item {Grp. gram.:M. pl.}
\end{itemize}
\begin{itemize}
\item {Proveniência:(Do gr. \textunderscore leukos\textunderscore  + \textunderscore karpos\textunderscore )}
\end{itemize}
Que dá frutos brancos.
Gênero de plantas mexicanas.
\section{Leucocéfalo}
\begin{itemize}
\item {Grp. gram.:adj.}
\end{itemize}
\begin{itemize}
\item {Utilização:Zool.}
\end{itemize}
\begin{itemize}
\item {Proveniência:(Do gr. \textunderscore leukos\textunderscore  + \textunderscore kephale\textunderscore )}
\end{itemize}
Que tem cabeça branca.
\section{Leucocéphalo}
\begin{itemize}
\item {Grp. gram.:adj.}
\end{itemize}
\begin{itemize}
\item {Utilização:Zool.}
\end{itemize}
\begin{itemize}
\item {Proveniência:(Do gr. \textunderscore leukos\textunderscore  + \textunderscore kephale\textunderscore )}
\end{itemize}
Que tem cabeça branca.
\section{Leucochryso}
\begin{itemize}
\item {Grp. gram.:m.}
\end{itemize}
\begin{itemize}
\item {Proveniência:(Lat. \textunderscore leucochrysus\textunderscore )}
\end{itemize}
Designação antiga de uma pedra preciosa, branca e transparente.
\section{Leucocitário}
\begin{itemize}
\item {Grp. gram.:adj.}
\end{itemize}
Relativo a leucocito.
\section{Leucocitemia}
\begin{itemize}
\item {Grp. gram.:f.}
\end{itemize}
\begin{itemize}
\item {Proveniência:(Do gr. \textunderscore leukos\textunderscore  + \textunderscore kutos\textunderscore  + \textunderscore haima\textunderscore )}
\end{itemize}
Estado mórbido, caracterizado pela deminuição dos glóbulos vermelhos do sangue e aumento considerável dos brancos.
\section{Leucocitêmico}
\begin{itemize}
\item {Grp. gram.:adj.}
\end{itemize}
Relativo á leucocitemia.
\section{Leucocito}
\begin{itemize}
\item {Grp. gram.:m.}
\end{itemize}
\begin{itemize}
\item {Proveniência:(Do gr. \textunderscore leukos\textunderscore  + \textunderscore kutos\textunderscore )}
\end{itemize}
Glóbulo branco do sangue; linfa.
\section{Leucocitose}
\begin{itemize}
\item {Grp. gram.:f.}
\end{itemize}
\begin{itemize}
\item {Proveniência:(De \textunderscore leucocito\textunderscore )}
\end{itemize}
Aumento mórbido dos glóbulos brancos do sangue.
\section{Leucócomo}
\begin{itemize}
\item {Grp. gram.:adj.}
\end{itemize}
\begin{itemize}
\item {Proveniência:(Lat. \textunderscore leucocomus\textunderscore )}
\end{itemize}
Que tem cabellos brancos.
Que tem folhas brancas, (falando-se de plantas).
\section{Leucocriso}
\begin{itemize}
\item {Grp. gram.:m.}
\end{itemize}
\begin{itemize}
\item {Proveniência:(Lat. \textunderscore leucochrysus\textunderscore )}
\end{itemize}
Designação antiga de uma pedra preciosa, branca e transparente.
\section{Leucocytário}
\begin{itemize}
\item {Grp. gram.:adj.}
\end{itemize}
Relativo a leucocyto.
\section{Leucocythemia}
\begin{itemize}
\item {Grp. gram.:f.}
\end{itemize}
\begin{itemize}
\item {Proveniência:(Do gr. \textunderscore leukos\textunderscore  + \textunderscore kutos\textunderscore  + \textunderscore haima\textunderscore )}
\end{itemize}
Estado mórbido, caracterizado pela deminuição dos glóbulos vermelhos do sangue e aumento considerável dos brancos.
\section{Leucocythêmico}
\begin{itemize}
\item {Grp. gram.:adj.}
\end{itemize}
Relativo á leucocythemia.
\section{Leucocyto}
\begin{itemize}
\item {Grp. gram.:m.}
\end{itemize}
\begin{itemize}
\item {Proveniência:(Do gr. \textunderscore leukos\textunderscore  + \textunderscore kutos\textunderscore )}
\end{itemize}
Glóbulo branco do sangue; lympha.
\section{Leucocytose}
\begin{itemize}
\item {Grp. gram.:f.}
\end{itemize}
\begin{itemize}
\item {Proveniência:(De \textunderscore leucocyto\textunderscore )}
\end{itemize}
Aumento mórbido dos glóbulos brancos do sangue.
\section{Leucodonte}
\begin{itemize}
\item {Grp. gram.:adj.}
\end{itemize}
\begin{itemize}
\item {Utilização:Zool.}
\end{itemize}
\begin{itemize}
\item {Proveniência:(Do gr. \textunderscore leukos\textunderscore  + \textunderscore odous\textunderscore , \textunderscore odontos\textunderscore )}
\end{itemize}
Que tem dentadura branca.
\section{Leucofilo}
\begin{itemize}
\item {Grp. gram.:m.}
\end{itemize}
Gênero de plantas escrofularíneas.
\section{Leucoflegmasia}
\begin{itemize}
\item {Grp. gram.:f.}
\end{itemize}
\begin{itemize}
\item {Proveniência:(Do gr. \textunderscore leukos\textunderscore  + \textunderscore phlegma\textunderscore )}
\end{itemize}
Um dos nomes do anasarca ou da hidropisia subcutânea.
\section{Leucoflegmásico}
\begin{itemize}
\item {Grp. gram.:adj.}
\end{itemize}
Relativo á leucoflegmasia.
\section{Leucografia}
\begin{itemize}
\item {Grp. gram.:f.}
\end{itemize}
\begin{itemize}
\item {Proveniência:(Do gr. \textunderscore leukos\textunderscore  + \textunderscore graphein\textunderscore )}
\end{itemize}
Tratado sôbre o albinismo.
\section{Leucografite}
\begin{itemize}
\item {Grp. gram.:f.}
\end{itemize}
\begin{itemize}
\item {Proveniência:(Do gr. \textunderscore leukos\textunderscore  + \textunderscore graphein\textunderscore . Cp. \textunderscore graphite\textunderscore )}
\end{itemize}
Espécie de pedra branca, facilmente dissolúvel na água, e de que, nalguns países, se servem as engomadeiras para dar brilho á roupa que engomam.
\section{Leucographia}
\begin{itemize}
\item {Grp. gram.:f.}
\end{itemize}
\begin{itemize}
\item {Proveniência:(Do gr. \textunderscore leukos\textunderscore  + \textunderscore graphein\textunderscore )}
\end{itemize}
Tratado sôbre o albinismo.
\section{Leucographite}
\begin{itemize}
\item {Grp. gram.:f.}
\end{itemize}
\begin{itemize}
\item {Proveniência:(Do gr. \textunderscore leukos\textunderscore  + \textunderscore graphein\textunderscore . Cp. \textunderscore graphite\textunderscore )}
\end{itemize}
Espécie de pedra branca, facilmente dissolúvel na água, e de que, nalguns países, se servem as engomadeiras para dar brilho á roupa que engomam.
\section{Leucolena}
\begin{itemize}
\item {Grp. gram.:f.}
\end{itemize}
\begin{itemize}
\item {Proveniência:(Do gr. \textunderscore leukos\textunderscore  + \textunderscore laina\textunderscore )}
\end{itemize}
Gênero de plantas umbellíferas da Austrália.
\section{Leucólito}
\begin{itemize}
\item {Grp. gram.:adj.}
\end{itemize}
\begin{itemize}
\item {Utilização:Chím.}
\end{itemize}
\begin{itemize}
\item {Proveniência:(Do gr. \textunderscore leukos\textunderscore  + \textunderscore lutos\textunderscore )}
\end{itemize}
Diz-se do metal, que fórma sal branco ou incolor com um ácido incolor.
\section{Leucólyto}
\begin{itemize}
\item {Grp. gram.:adj.}
\end{itemize}
\begin{itemize}
\item {Utilização:Chím.}
\end{itemize}
\begin{itemize}
\item {Proveniência:(Do gr. \textunderscore leukos\textunderscore  + \textunderscore lutos\textunderscore )}
\end{itemize}
Diz-se do metal, que fórma sal branco ou incolor com um ácido incolor.
\section{Leucoma}
\begin{itemize}
\item {Grp. gram.:m.}
\end{itemize}
\begin{itemize}
\item {Proveniência:(Gr. \textunderscore leukoma\textunderscore )}
\end{itemize}
Mancha branca na córnea transparente.
\section{Leucomaína}
\begin{itemize}
\item {Grp. gram.:f.}
\end{itemize}
\begin{itemize}
\item {Proveniência:(De \textunderscore leucoma\textunderscore )}
\end{itemize}
Nome genérico de vários alcaloides, que se formam no organismo dos animaes, durante a vida.
\section{Leucopathia}
\begin{itemize}
\item {Grp. gram.:f.}
\end{itemize}
\begin{itemize}
\item {Proveniência:(Do gr. \textunderscore leukos\textunderscore  + \textunderscore pathos\textunderscore )}
\end{itemize}
O mesmo que \textunderscore albinismo\textunderscore .
\section{Leucopáthico}
\begin{itemize}
\item {Grp. gram.:adj.}
\end{itemize}
Relativo á leucopathia.
\section{Leucopatia}
\begin{itemize}
\item {Grp. gram.:f.}
\end{itemize}
\begin{itemize}
\item {Proveniência:(Do gr. \textunderscore leukos\textunderscore  + \textunderscore pathos\textunderscore )}
\end{itemize}
O mesmo que \textunderscore albinismo\textunderscore .
\section{Leucopático}
\begin{itemize}
\item {Grp. gram.:adj.}
\end{itemize}
Relativo á leucopathia.
\section{Leucopenia}
\begin{itemize}
\item {Grp. gram.:f.}
\end{itemize}
\begin{itemize}
\item {Utilização:Med.}
\end{itemize}
\begin{itemize}
\item {Proveniência:(Do gr. \textunderscore leukos\textunderscore , branco + \textunderscore penia\textunderscore , pobreza)}
\end{itemize}
Deminuição do número dos glóbulos brancos do sangue.
\section{Leucopétalo}
\begin{itemize}
\item {Grp. gram.:adj.}
\end{itemize}
\begin{itemize}
\item {Utilização:Bot.}
\end{itemize}
\begin{itemize}
\item {Proveniência:(Do gr. \textunderscore leukos\textunderscore  + \textunderscore petalon\textunderscore )}
\end{itemize}
Que tem fôlhas brancas.
\section{Leucophlegmasia}
\begin{itemize}
\item {Grp. gram.:f.}
\end{itemize}
\begin{itemize}
\item {Proveniência:(Do gr. \textunderscore leukos\textunderscore  + \textunderscore phlegma\textunderscore )}
\end{itemize}
Um dos nomes do anasarca ou da hydropisia subcutânea.
\section{Leucophlegmásico}
\begin{itemize}
\item {Grp. gram.:adj.}
\end{itemize}
Relativo á leucophlegmasia.
\section{Leucophyllo}
\begin{itemize}
\item {Grp. gram.:m.}
\end{itemize}
Gênero de plantas escrofularíneas.
\section{Leucópode}
\begin{itemize}
\item {Grp. gram.:adj.}
\end{itemize}
\begin{itemize}
\item {Utilização:Bot.}
\end{itemize}
\begin{itemize}
\item {Proveniência:(Do gr. \textunderscore leukos\textunderscore  + \textunderscore pous\textunderscore , \textunderscore podos\textunderscore )}
\end{itemize}
Diz-se dos cogumelos de pé branco.
\section{Leucóptero}
\begin{itemize}
\item {Grp. gram.:adj.}
\end{itemize}
\begin{itemize}
\item {Utilização:Zool.}
\end{itemize}
\begin{itemize}
\item {Proveniência:(Do gr. \textunderscore leukos\textunderscore  + \textunderscore pteron\textunderscore )}
\end{itemize}
Que tem asas brancas.
\section{Leucorreia}
\begin{itemize}
\item {Grp. gram.:f.}
\end{itemize}
\begin{itemize}
\item {Proveniência:(Do gr. \textunderscore leukos\textunderscore  + \textunderscore rhein\textunderscore )}
\end{itemize}
Corrimento branco da vagina ou do útero, conhecido vulgarmente por flôres brancas.
\section{Leucorreico}
\begin{itemize}
\item {Grp. gram.:adj.}
\end{itemize}
Relativo á leucorreia.
\section{Leucorrhéa}
\begin{itemize}
\item {Grp. gram.:f.}
\end{itemize}
\begin{itemize}
\item {Proveniência:(Do gr. \textunderscore leukos\textunderscore  + \textunderscore rhein\textunderscore )}
\end{itemize}
Corrimento branco da vagina ou do útero, conhecido vulgarmente por flôres brancas.
\section{Leucorrheia}
\begin{itemize}
\item {Grp. gram.:f.}
\end{itemize}
\begin{itemize}
\item {Proveniência:(Do gr. \textunderscore leukos\textunderscore  + \textunderscore rhein\textunderscore )}
\end{itemize}
Corrimento branco da vagina ou do útero, conhecido vulgarmente por flôres brancas.
\section{Leucorrheico}
\begin{itemize}
\item {Grp. gram.:adj.}
\end{itemize}
Relativo á leucorrheia.
\section{Leucose}
\begin{itemize}
\item {Grp. gram.:f.}
\end{itemize}
\begin{itemize}
\item {Proveniência:(Do gr. \textunderscore leukos\textunderscore )}
\end{itemize}
Designação genérica das doenças, que atacam os vasos lympháticos.
\section{Leucósia}
\begin{itemize}
\item {Grp. gram.:f.}
\end{itemize}
\begin{itemize}
\item {Proveniência:(Do gr. \textunderscore leukos\textunderscore )}
\end{itemize}
Gênero de crustáceos decápodes.
\section{Leucoterapia}
\begin{itemize}
\item {Grp. gram.:f.}
\end{itemize}
\begin{itemize}
\item {Utilização:Med.}
\end{itemize}
\begin{itemize}
\item {Proveniência:(Do gr. \textunderscore leukos\textunderscore  + \textunderscore therapeia\textunderscore )}
\end{itemize}
Tratamento, que consiste em provocar a leucocytose por meio de uma substância chímica.
\section{Leucotherapia}
\begin{itemize}
\item {Grp. gram.:f.}
\end{itemize}
\begin{itemize}
\item {Utilização:Med.}
\end{itemize}
\begin{itemize}
\item {Proveniência:(Do gr. \textunderscore leukos\textunderscore  + \textunderscore therapeia\textunderscore )}
\end{itemize}
Tratamento, que consiste em provocar a leucocytose por meio de uma substância chímica.
\section{Leucrócota}
\begin{itemize}
\item {Grp. gram.:m.}
\end{itemize}
\begin{itemize}
\item {Proveniência:(Lat. \textunderscore leucrocota\textunderscore )}
\end{itemize}
Animal feroz da Índia, alludido pelos antigos, mas cuja identidade é hoje diffícil reconhecer.
\section{Leutrite}
\begin{itemize}
\item {Grp. gram.:f.}
\end{itemize}
\begin{itemize}
\item {Utilização:Miner.}
\end{itemize}
Marga calcária e arenosa, que, friccionada, produz na escuridão uma viva luz phosphórica.
\section{Leva}
\begin{itemize}
\item {Grp. gram.:f.}
\end{itemize}
\begin{itemize}
\item {Utilização:Pop.}
\end{itemize}
\begin{itemize}
\item {Utilização:Náut.}
\end{itemize}
\begin{itemize}
\item {Proveniência:(De \textunderscore levar\textunderscore )}
\end{itemize}
Acto de levantar âncora para navegar.
Magote: \textunderscore chegou uma leva de presos\textunderscore .
Recrutamento.
Andadura.
Cabo delgado, que passa por um furo, feito no costado do navio e vai prender-se no sapatilho dos arganéus das portas.
\section{Leva-arriba!}
\begin{itemize}
\item {Grp. gram.:interj.}
\end{itemize}
(para mandar levantar, ou para fazer acordar)
\section{Levação}
\begin{itemize}
\item {Grp. gram.:f.}
\end{itemize}
\begin{itemize}
\item {Proveniência:(Lat. \textunderscore levatio\textunderscore )}
\end{itemize}
Tumor maligno; inchaço; bubão; íngua.
O mesmo que \textunderscore elevação\textunderscore  ou \textunderscore altura\textunderscore .
\textunderscore Levação do pólo\textunderscore , altura do pólo, altitude. Cf. \textunderscore Roteiro do Mar-Vermelho\textunderscore , (passim).
\section{Levada}
\begin{itemize}
\item {Grp. gram.:f.}
\end{itemize}
\begin{itemize}
\item {Utilização:Bras. do N}
\end{itemize}
\begin{itemize}
\item {Utilização:Bras. do N}
\end{itemize}
\begin{itemize}
\item {Utilização:Ant.}
\end{itemize}
\begin{itemize}
\item {Grp. gram.:Loc. adv.}
\end{itemize}
Acto de levar.
Corrente de água, de ordinário procedente de um rio, e que vai regando campos, ou dando movimento a moinhos, fábricas, etc.
Cascata.
Collina; elevação de terreno.
Golpe ou córte, que se faz nas orelhas das reses, para as marcar.
Golpe, bote. Cf. G. Vicente, I, 230.
\textunderscore De levada\textunderscore , sem persistência, de levante. Cf. Castilho, \textunderscore D. Quixote\textunderscore , 45.
\section{Levadente}
\begin{itemize}
\item {Grp. gram.:m.}
\end{itemize}
\begin{itemize}
\item {Utilização:Pop.}
\end{itemize}
\begin{itemize}
\item {Proveniência:(De \textunderscore levar\textunderscore  + \textunderscore dente\textunderscore )}
\end{itemize}
Mordedura.
Reprehensão.
\section{Levadia}
\begin{itemize}
\item {Grp. gram.:f.}
\end{itemize}
\begin{itemize}
\item {Utilização:Ant.}
\end{itemize}
\begin{itemize}
\item {Utilização:us. ainda na Baía}
\end{itemize}
\begin{itemize}
\item {Proveniência:(Do rad. de \textunderscore levar\textunderscore )}
\end{itemize}
Movimento agitado do mar.
\section{Levadiça}
\begin{itemize}
\item {Grp. gram.:f.}
\end{itemize}
\begin{itemize}
\item {Proveniência:(De \textunderscore levadiço\textunderscore )}
\end{itemize}
Ponte, que se póde levantar ou baixar facilmente.
\section{Levadiço}
\begin{itemize}
\item {Grp. gram.:adj.}
\end{itemize}
\begin{itemize}
\item {Proveniência:(Do rad. de \textunderscore levar\textunderscore )}
\end{itemize}
Móvel, movediço; que facilmente se levanta ou abaixa.
\section{Levadigas}
\begin{itemize}
\item {Grp. gram.:f. pl.}
\end{itemize}
\begin{itemize}
\item {Utilização:Ant.}
\end{itemize}
Chamava-se dôr de levadigas a dôr aguda ou pontada, que apparecia debaixo do braço ou junto da virilha e precedia ou acompanhava a levação ou bubão, nos casos da peste negra do século XIV:«\textunderscore dôr de levadigas te consumam...\textunderscore »Herculano, \textunderscore Lendas\textunderscore , I, 66.
(Cp. \textunderscore levação\textunderscore )
\section{Levadinho}
\begin{itemize}
\item {Grp. gram.:adj.}
\end{itemize}
Expressão pop. de realce, por \textunderscore levado\textunderscore : \textunderscore é levadinho da breca\textunderscore .
\section{Levadio}
\begin{itemize}
\item {Grp. gram.:adj.}
\end{itemize}
\begin{itemize}
\item {Proveniência:(Do rad. de \textunderscore levar\textunderscore )}
\end{itemize}
Diz-se do telhado, formado de telhas soltas.
\section{Levado}
\begin{itemize}
\item {Grp. gram.:adj.}
\end{itemize}
\begin{itemize}
\item {Utilização:Ant.}
\end{itemize}
\begin{itemize}
\item {Proveniência:(De \textunderscore levar\textunderscore )}
\end{itemize}
Que já vai alto, (falando-se do sol).
\section{Levadoira}
\begin{itemize}
\item {Grp. gram.:f.}
\end{itemize}
\begin{itemize}
\item {Utilização:T. da Nazareth}
\end{itemize}
\begin{itemize}
\item {Proveniência:(De \textunderscore levar\textunderscore )}
\end{itemize}
Pequena embarcação, com apparelho, para tirar carga de outra.
Um dos pescadores, encarregados do levantamento das rêdes.
\section{Levador}
\begin{itemize}
\item {Grp. gram.:adj.}
\end{itemize}
\begin{itemize}
\item {Grp. gram.:M.}
\end{itemize}
Que leva ou conduz.
Aquelle que conduz ou transporta.
\section{Levadoura}
\begin{itemize}
\item {Grp. gram.:f.}
\end{itemize}
\begin{itemize}
\item {Utilização:T. da Nazareth}
\end{itemize}
\begin{itemize}
\item {Proveniência:(De \textunderscore levar\textunderscore )}
\end{itemize}
Pequena embarcação, com apparelho, para tirar carga de outra.
Um dos pescadores, encarregados do levantamento das rêdes.
\section{Levadura}
\begin{itemize}
\item {Grp. gram.:f.}
\end{itemize}
(Corr. de \textunderscore levedura\textunderscore )
\section{Levadurina}
\begin{itemize}
\item {Grp. gram.:f.}
\end{itemize}
\begin{itemize}
\item {Proveniência:(De \textunderscore levadura\textunderscore )}
\end{itemize}
Medicamento novo, contra furúnculos, anthrazes, etc.
\section{Levagante}
\begin{itemize}
\item {Grp. gram.:m.}
\end{itemize}
Crustáceo decápode, marítimo, um pouco mais pequeno que a lagosta e munido de duas fortes torqueses nos braços (\textunderscore homarus vulgaris\textunderscore ).
(Cp. cast. \textunderscore lobogante\textunderscore )
\section{Levamento}
\begin{itemize}
\item {Grp. gram.:m.}
\end{itemize}
\begin{itemize}
\item {Utilização:Des.}
\end{itemize}
Acto de levar.
Furto.
\section{Levandeira}
\begin{itemize}
\item {Grp. gram.:f.}
\end{itemize}
\begin{itemize}
\item {Utilização:Prov.}
\end{itemize}
\begin{itemize}
\item {Utilização:trasm.}
\end{itemize}
O mesmo que \textunderscore levandisca\textunderscore .
\section{Levandisca}
\begin{itemize}
\item {Grp. gram.:f.}
\end{itemize}
\begin{itemize}
\item {Utilização:Prov.}
\end{itemize}
Ave, o mesmo que \textunderscore lavandisca\textunderscore .
(Cp. \textunderscore lavandisca\textunderscore )
\section{Levantada}
\begin{itemize}
\item {Grp. gram.:f.}
\end{itemize}
Acto de levantar da cama.
Acto de levantar.
\section{Levantadiço}
\begin{itemize}
\item {Grp. gram.:adj.}
\end{itemize}
\begin{itemize}
\item {Proveniência:(Do rad. de \textunderscore levantar\textunderscore )}
\end{itemize}
Que costuma insubordinar-se; indisciplinado.
Inquieto; turbulento.
Leviano.
\section{Levantado}
\begin{itemize}
\item {Grp. gram.:adj.}
\end{itemize}
\begin{itemize}
\item {Proveniência:(De \textunderscore levantar\textunderscore )}
\end{itemize}
Que tem cabeça leve.
Que é estroina ou doidivanas.
\section{Levantador}
\begin{itemize}
\item {Grp. gram.:adj.}
\end{itemize}
\begin{itemize}
\item {Grp. gram.:M.}
\end{itemize}
\begin{itemize}
\item {Utilização:Anat.}
\end{itemize}
Que levanta.
Que excita; que amotina ou que revolta.
Músculo, com que se levanta alguma parte do corpo.
Instrumento, para levantar do cérebro ossos fracturados do crânio.
\section{Leicéstria}
\begin{itemize}
\item {Grp. gram.:f.}
\end{itemize}
Gênero de plantas caprifoliáceas.
\section{Levantadura}
\begin{itemize}
\item {Grp. gram.:f.}
\end{itemize}
Acto de levantar.
Accréscimo, refôrço.
Insubordinação; revolta.
\section{Levantamento}
\begin{itemize}
\item {Grp. gram.:m.}
\end{itemize}
Acto de levantar.
Accréscimo, refôrço.
Insubordinação; revolta.
\section{Levantante}
\begin{itemize}
\item {Grp. gram.:adj.}
\end{itemize}
\begin{itemize}
\item {Utilização:Heráld.}
\end{itemize}
\begin{itemize}
\item {Proveniência:(De \textunderscore levantar\textunderscore )}
\end{itemize}
Representado em pé.
\section{Levantar}
\begin{itemize}
\item {Grp. gram.:v. t.}
\end{itemize}
\begin{itemize}
\item {Utilização:Fig.}
\end{itemize}
\begin{itemize}
\item {Grp. gram.:V. i.}
\end{itemize}
\begin{itemize}
\item {Grp. gram.:M.}
\end{itemize}
Pôr em pé.
Pôr alto; erguer.
Tornar mais alto.
Arvorar.
Tornar erecto.
Arrancar.
Aprestar, apparelhar.
Ennobrecer.
Exaltar.
Sublimar.
Excitar, revoltar.
Aumentar.
Reforçar: \textunderscore medicamento, que levanta as fôrças\textunderscore .
Originar: \textunderscore levantar uma calúmnia\textunderscore .
Cobrar, arrecadar.
Pôr fim a.
Pôr em fuga.
Afastar.
Fundar, edificar: \textunderscore levantar um prédio\textunderscore .
Suggerir.
Subir de preço.
Crescer.
Pôr-se mais alto.
Acto de levantar.
\section{Levante}
\begin{itemize}
\item {Grp. gram.:m.}
\end{itemize}
\begin{itemize}
\item {Utilização:Prov.}
\end{itemize}
\begin{itemize}
\item {Utilização:T. de Turquel}
\end{itemize}
\begin{itemize}
\item {Grp. gram.:Loc. adv.}
\end{itemize}
\begin{itemize}
\item {Proveniência:(Lat. \textunderscore levans\textunderscore , \textunderscore levantis\textunderscore )}
\end{itemize}
Acto de levantar.
A banda do horizonte, donde parece surgir o sol; oriente.
Região asiática, banhada pelo Mediterrâneo.
Povos dessa região.
Motim.
Tempo, immediato a um período chuvoso.
\textunderscore De levante\textunderscore , sem persistência.
Sem descanso.
Irreflectidamente.
Prestes a partir: \textunderscore estou de levante\textunderscore .
\section{Levântico}
\begin{itemize}
\item {Grp. gram.:adj.}
\end{itemize}
O mesmo que \textunderscore levantino\textunderscore .
\section{Levantina}
\begin{itemize}
\item {Grp. gram.:f.}
\end{itemize}
\begin{itemize}
\item {Utilização:Des.}
\end{itemize}
Estôfo de seda ordinária.
\section{Levantino}
\begin{itemize}
\item {Grp. gram.:adj.}
\end{itemize}
\begin{itemize}
\item {Grp. gram.:M.}
\end{itemize}
Relativo ao Levante ou aos povos do Levante.
Aquelle que habita no Levante.
\section{Levantisco}
\begin{itemize}
\item {Grp. gram.:m.}
\end{itemize}
\begin{itemize}
\item {Utilização:Ant.}
\end{itemize}
Homem do Levante; levantino.
\section{Levanto}
\textunderscore m.\textunderscore Acto de fazer levantar a caça.
O mesmo que \textunderscore levante\textunderscore .
\section{Levar}
\begin{itemize}
\item {Grp. gram.:v. t.}
\end{itemize}
\begin{itemize}
\item {Grp. gram.:V. i.}
\end{itemize}
\begin{itemize}
\item {Utilização:Fam.}
\end{itemize}
\begin{itemize}
\item {Grp. gram.:Loc.}
\end{itemize}
\begin{itemize}
\item {Utilização:fam.}
\end{itemize}
\begin{itemize}
\item {Grp. gram.:Loc.}
\end{itemize}
\begin{itemize}
\item {Utilização:pop.}
\end{itemize}
\begin{itemize}
\item {Grp. gram.:Loc.}
\end{itemize}
\begin{itemize}
\item {Utilização:pop.}
\end{itemize}
\begin{itemize}
\item {Proveniência:(Lat. \textunderscore levare\textunderscore )}
\end{itemize}
Transportar, levantando ou sustendo: \textunderscore levar um filho ao collo\textunderscore .
Arrastar, impellir: \textunderscore os bois levam o carro\textunderscore .
Desviar.
Supportar: \textunderscore levar muito trabalho\textunderscore .
Convencer.
Desvanecer; expungir: \textunderscore o tempo leva as inscripções\textunderscore .
Receber.
Auferir.
Exigir como paga ou preço: \textunderscore o pintor leva dez tostões por dia\textunderscore .
Gastar, consumir.
Tomar, engulir. Trajar: \textunderscore o velho leveva um capote escuro\textunderscore .
Conduzir.
Expulsar.
Conter.
Têr capacidade para: \textunderscore esta garrafa leva um litro\textunderscore .
Elevar.
Têr comsigo, sêr dotado de.
Indicar a direcção de alguma coisa.
Apanhar pancadas, receber castigo.
\textunderscore Levar a melhor\textunderscore , têr vantagem sôbre outro ou outros; sair vencedor.
\textunderscore Levar para baixo\textunderscore , sêr sovado ou maltratado.
\textunderscore Levar para o seu tabaco\textunderscore , apanhar pancadaria.
\section{Leve}
\begin{itemize}
\item {Grp. gram.:adj.}
\end{itemize}
\begin{itemize}
\item {Grp. gram.:Adv.}
\end{itemize}
\begin{itemize}
\item {Grp. gram.:Loc. adv.}
\end{itemize}
Que pesa pouco.
Tênue.
Que não é grave; que não é importante: \textunderscore leves incômmodos\textunderscore .
Simples; delicado.
Ligeiro, indistinto: \textunderscore uma leve aragem\textunderscore .
Alliviado.
\textunderscore Têr mão leve\textunderscore  ou \textunderscore mãos leves\textunderscore , estar sempre prompto para bater.
Levemente.
\textunderscore De leve\textunderscore  ou \textunderscore ao de leve\textunderscore , levemente; levianamente; superficialmente. (Lat. \textunderscore levis\textunderscore )
\section{Levedação}
\begin{itemize}
\item {Grp. gram.:f.}
\end{itemize}
Acto de levedar.
\section{Levedadura}
\begin{itemize}
\item {Grp. gram.:f.}
\end{itemize}
O mesmo que \textunderscore levedura\textunderscore .
\section{Levedar}
\begin{itemize}
\item {Grp. gram.:v. t.}
\end{itemize}
\begin{itemize}
\item {Grp. gram.:V. i.}
\end{itemize}
\begin{itemize}
\item {Proveniência:(Do lat. hyp. \textunderscore levitare\textunderscore , de \textunderscore levis\textunderscore )}
\end{itemize}
Tornar lêvedo.
Tornar-se lêvedo.
\section{Lêvedo}
\begin{itemize}
\item {Grp. gram.:adj.}
\end{itemize}
\begin{itemize}
\item {Proveniência:(Do rad. de \textunderscore levedar\textunderscore )}
\end{itemize}
Que se fermentou.
Que aumentou de volume, (falando-se da massa fermentada).
\section{Levedura}
\begin{itemize}
\item {Grp. gram.:f.}
\end{itemize}
\begin{itemize}
\item {Proveniência:(De \textunderscore lêvedo\textunderscore )}
\end{itemize}
O mesmo que \textunderscore fermento\textunderscore .
\section{Levedurina}
\begin{itemize}
\item {Grp. gram.:f.}
\end{itemize}
Levedura de cerveja, medicinal.
(Fórma preferível a \textunderscore levadurina\textunderscore )
\section{Leveiro}
\begin{itemize}
\item {Grp. gram.:adj.}
\end{itemize}
\begin{itemize}
\item {Proveniência:(De \textunderscore leve\textunderscore )}
\end{itemize}
Pouco pesado: \textunderscore canastra leveira\textunderscore .
Comportável:«\textunderscore suave tristeza, que faz os homens melhores e o fardo da vida mais leveiro.\textunderscore »Camillo, \textunderscore Noites de Insómn.\textunderscore , X, 17.
Que tem pouca fôrça:«\textunderscore são leveiras de mais as minhas mãos para sustentarem as balanças.\textunderscore »Camillo, \textunderscore Estrêll. Fun.\textunderscore , pról.
\section{Levemente}
\begin{itemize}
\item {Grp. gram.:adv.}
\end{itemize}
\begin{itemize}
\item {Utilização:Ant.}
\end{itemize}
De modo leve; levianamente; superficialmente; vagamente.
O mesmo que \textunderscore facilmente\textunderscore . Cf. Pant. de Aveiro, \textunderscore Itiner.\textunderscore , 32, (3.^a ed.).
\section{Leves}
\begin{itemize}
\item {Grp. gram.:m. pl.}
\end{itemize}
\begin{itemize}
\item {Proveniência:(De \textunderscore leve\textunderscore )}
\end{itemize}
Pulmões de ave.
Bofes.
\section{Levez}
\begin{itemize}
\item {Grp. gram.:f.}
\end{itemize}
O mesmo que \textunderscore leveza\textunderscore . Cf. Filinto, VII, 138.
\section{Leveza}
\begin{itemize}
\item {Grp. gram.:f.}
\end{itemize}
Qualidade de leve.
Leviandade; falta de tino ou de reflexão.
\section{Leviandade}
\begin{itemize}
\item {Grp. gram.:f.}
\end{itemize}
Qualidade de leviano; falta de tino ou de reflexão.
Imprudência; acto leviano.
\section{Leviano}
\begin{itemize}
\item {Grp. gram.:adj.}
\end{itemize}
\begin{itemize}
\item {Utilização:port}
\end{itemize}
\begin{itemize}
\item {Utilização:Bras}
\end{itemize}
\begin{itemize}
\item {Utilização:ant.}
\end{itemize}
\begin{itemize}
\item {Proveniência:(De \textunderscore leve\textunderscore )}
\end{itemize}
Que julga de leve.
Que reflecte pouco.
Inconsiderado; precipitado: \textunderscore palavras levianas\textunderscore .
Imprudente.
Que denota pouco siso.
Que não tem seriedade ou que procede reprehensivelmente.
Leve; que tem pequena carga:«\textunderscore a canôa vem muito descarregada e leviana.\textunderscore »LaCerda e Almeida, \textunderscore Diário de Viagem\textunderscore , 71.
\section{Leviatão}
\begin{itemize}
\item {Grp. gram.:m.}
\end{itemize}
\begin{itemize}
\item {Proveniência:(Lat. da \textunderscore Vulgata\textunderscore , \textunderscore levisthan\textunderscore )}
\end{itemize}
Grande monstro marinho, aludido na \textunderscore Biblia\textunderscore .
\section{Leviathão}
\begin{itemize}
\item {Grp. gram.:m.}
\end{itemize}
\begin{itemize}
\item {Proveniência:(Lat. da \textunderscore Vulgata\textunderscore , \textunderscore levisthan\textunderscore )}
\end{itemize}
Grande monstro marinho, alludido na \textunderscore Biblia\textunderscore .
\section{Levidade}
\begin{itemize}
\item {Grp. gram.:m.}
\end{itemize}
\begin{itemize}
\item {Utilização:Fig.}
\end{itemize}
\begin{itemize}
\item {Proveniência:(Lat. \textunderscore levitas\textunderscore )}
\end{itemize}
O mesmo que \textunderscore leveza\textunderscore .
Agilidade.
\section{Levidão}
\begin{itemize}
\item {Grp. gram.:f.}
\end{itemize}
\begin{itemize}
\item {Utilização:Fig.}
\end{itemize}
\begin{itemize}
\item {Proveniência:(De \textunderscore leve\textunderscore )}
\end{itemize}
O mesmo que \textunderscore levidade\textunderscore .
Leviandade.
\section{Levigação}
\begin{itemize}
\item {Grp. gram.:f.}
\end{itemize}
\begin{itemize}
\item {Proveniência:(Lat. \textunderscore levigatio\textunderscore )}
\end{itemize}
Acto de reduzir a pó por meio de pórphyro.
\section{Levigado}
\begin{itemize}
\item {Grp. gram.:adj.}
\end{itemize}
Liso, macio:«\textunderscore ...ovos, não tão finos e levigados como os ordinários.\textunderscore »Frei Caetano Brandão, \textunderscore Pastoraes\textunderscore .
\section{Levigar}
\begin{itemize}
\item {Grp. gram.:v. t.}
\end{itemize}
\begin{itemize}
\item {Proveniência:(Lat. \textunderscore levigare\textunderscore )}
\end{itemize}
Sujeitar á levigação.
\section{Levípede}
\begin{itemize}
\item {Grp. gram.:adj.}
\end{itemize}
\begin{itemize}
\item {Utilização:Poét.}
\end{itemize}
\begin{itemize}
\item {Proveniência:(Do lat. \textunderscore levipes\textunderscore , \textunderscore levipedis\textunderscore )}
\end{itemize}
Que tem pé leve; que anda com presteza.
\section{Levirato}
\begin{itemize}
\item {Grp. gram.:m.}
\end{itemize}
\begin{itemize}
\item {Proveniência:(Do lat. \textunderscore levir\textunderscore , cunhado)}
\end{itemize}
Obrigação, que a lei de Moisés impunha ao irmão de um defunto, de casar com a viúva dêste.
\section{Levirostro}
\begin{itemize}
\item {fónica:rós}
\end{itemize}
\begin{itemize}
\item {Grp. gram.:adj.}
\end{itemize}
\begin{itemize}
\item {Utilização:Zool.}
\end{itemize}
\begin{itemize}
\item {Grp. gram.:M. pl.}
\end{itemize}
\begin{itemize}
\item {Proveniência:(Do lat. \textunderscore levis\textunderscore  + \textunderscore rostrum\textunderscore )}
\end{itemize}
Que tem bico leve.
Família de aves trepadôras, de bico leve.
\section{Levirrostro}
\begin{itemize}
\item {Grp. gram.:adj.}
\end{itemize}
\begin{itemize}
\item {Utilização:Zool.}
\end{itemize}
\begin{itemize}
\item {Grp. gram.:M. pl.}
\end{itemize}
\begin{itemize}
\item {Proveniência:(Do lat. \textunderscore levis\textunderscore  + \textunderscore rostrum\textunderscore )}
\end{itemize}
Que tem bico leve.
Família de aves trepadôras, de bico leve.
\section{Levísia}
\begin{itemize}
\item {Grp. gram.:f.}
\end{itemize}
Gênero de plantas portuláceas.
\section{Levistónia}
\begin{itemize}
\item {Grp. gram.:f.}
\end{itemize}
Gênero de plantas.
\section{Levita}
\begin{itemize}
\item {Grp. gram.:m.}
\end{itemize}
\begin{itemize}
\item {Utilização:Ext.}
\end{itemize}
\begin{itemize}
\item {Proveniência:(Lat. \textunderscore levita\textunderscore )}
\end{itemize}
Membro da tríbo de Levi, entre os Hebreus.
Sacerdote hebreu do templo de Jerusalém.
Diácono.
Sacerdote.
\section{Levita}
\begin{itemize}
\item {Grp. gram.:f.}
\end{itemize}
\begin{itemize}
\item {Utilização:Irón.}
\end{itemize}
O mesmo que \textunderscore sobrecasaca\textunderscore , mais vulgarmente \textunderscore labita\textunderscore . Cf. Garrett, \textunderscore Helena\textunderscore , VII.
(Cast. \textunderscore levita\textunderscore )
\section{Levitação}
\begin{itemize}
\item {Grp. gram.:f.}
\end{itemize}
Acto de levitar-se.
\section{Levitar-se}
\begin{itemize}
\item {Grp. gram.:v. p.}
\end{itemize}
\begin{itemize}
\item {Utilização:Neol.}
\end{itemize}
\begin{itemize}
\item {Proveniência:(Do lat. \textunderscore levare\textunderscore )}
\end{itemize}
Erguer-se (alguém) acima do solo, nas experiências mágicas, sem que nada visível o sustenha ou suspenda.
\section{Levítico}
\begin{itemize}
\item {Grp. gram.:adj.}
\end{itemize}
\begin{itemize}
\item {Grp. gram.:M.}
\end{itemize}
Relativo aos levitas.
Terceiro livro do \textunderscore Pentateuco\textunderscore , que contém as leis dos levitas e as regras dos sacrifícios.
\section{Levítico}
\begin{itemize}
\item {Grp. gram.:m.}
\end{itemize}
Planta umbellífera medicinal, (\textunderscore lingusticum leviticum\textunderscore ).
\section{Levitonário}
\begin{itemize}
\item {Grp. gram.:m.}
\end{itemize}
Túnica, o mesmo que \textunderscore lebetona\textunderscore .
(B. lat. \textunderscore levitonarius\textunderscore )
\section{Levogiro}
\begin{itemize}
\item {Grp. gram.:adj.}
\end{itemize}
\begin{itemize}
\item {Utilização:Phýs.}
\end{itemize}
\begin{itemize}
\item {Proveniência:(Do lat. \textunderscore laevus\textunderscore  + \textunderscore gyrare\textunderscore )}
\end{itemize}
Diz-se da substância, que desvia para a esquerda o plano da polarização.
\section{Levogyro}
\begin{itemize}
\item {Grp. gram.:adj.}
\end{itemize}
\begin{itemize}
\item {Utilização:Phýs.}
\end{itemize}
\begin{itemize}
\item {Proveniência:(Do lat. \textunderscore laevus\textunderscore  + \textunderscore gyrare\textunderscore )}
\end{itemize}
Diz-se da substância, que desvia para a esquerda o plano da polarização.
\section{Levubo}
\begin{itemize}
\item {Grp. gram.:m.}
\end{itemize}
Árvore do Congo.
\section{Levulose}
\begin{itemize}
\item {Grp. gram.:f.}
\end{itemize}
Açúcar, contido no mel e na maior parte das frutas, exceptuando, entre outras, as uvas.
\section{Lexa-prem}
\begin{itemize}
\item {Grp. gram.:f.}
\end{itemize}
O mesmo que \textunderscore leixa-pren\textunderscore . Cf. Simões Dias, \textunderscore Theor. da Compos. Liter\textunderscore .
\section{Lexical}
\begin{itemize}
\item {fónica:csi}
\end{itemize}
\begin{itemize}
\item {Grp. gram.:adj.}
\end{itemize}
Relativo ao léxico.
Relativo aos vocábulos de um idioma.
\section{Léxico}
\begin{itemize}
\item {fónica:csi}
\end{itemize}
\begin{itemize}
\item {Grp. gram.:m.}
\end{itemize}
\begin{itemize}
\item {Utilização:Ant.}
\end{itemize}
\begin{itemize}
\item {Proveniência:(Gr. \textunderscore lexikon\textunderscore )}
\end{itemize}
Dicionário de línguas clássicas antigas.
Dicionário.
Dicionário das fórmas raras ou difíceis, peculiares a certos autores.
\section{Lexicografia}
\begin{itemize}
\item {fónica:csi}
\end{itemize}
\begin{itemize}
\item {Grp. gram.:f.}
\end{itemize}
Ciência ou estudo, que tem por objecto as palavras que devem constituír um léxico.
(Cp. \textunderscore lexicógrafo\textunderscore )
\section{Lexicograficamente}
\begin{itemize}
\item {fónica:csi}
\end{itemize}
\begin{itemize}
\item {Grp. gram.:adv.}
\end{itemize}
De modo lexicográfico; em forma de dicionário.
\section{Lexicográfico}
\begin{itemize}
\item {fónica:csi}
\end{itemize}
\begin{itemize}
\item {Grp. gram.:adj.}
\end{itemize}
Relativo á lexicografia.
\section{Lexicógrafo}
\begin{itemize}
\item {fónica:csi}
\end{itemize}
\begin{itemize}
\item {Grp. gram.:m.}
\end{itemize}
\begin{itemize}
\item {Proveniência:(Do gr. \textunderscore lexicon\textunderscore  + \textunderscore graphein\textunderscore )}
\end{itemize}
O mesmo que \textunderscore dicionarista\textunderscore .
\section{Lexicographia}
\begin{itemize}
\item {fónica:csi}
\end{itemize}
\begin{itemize}
\item {Grp. gram.:f.}
\end{itemize}
Sciência ou estudo, que tem por objecto as palavras que devem constituír um léxico.
(Cp. \textunderscore lexicógrapho\textunderscore )
\section{Lexicographicamente}
\begin{itemize}
\item {fónica:csi}
\end{itemize}
\begin{itemize}
\item {Grp. gram.:adv.}
\end{itemize}
De modo lexicográphico; em forma de diccionário.
\section{Lexicográphico}
\begin{itemize}
\item {fónica:csi}
\end{itemize}
\begin{itemize}
\item {Grp. gram.:adj.}
\end{itemize}
Relativo á lexicographia.
\section{Lexicógrapho}
\begin{itemize}
\item {fónica:csi}
\end{itemize}
\begin{itemize}
\item {Grp. gram.:m.}
\end{itemize}
\begin{itemize}
\item {Proveniência:(Do gr. \textunderscore lexicon\textunderscore  + \textunderscore graphein\textunderscore )}
\end{itemize}
O mesmo que \textunderscore diccionarista\textunderscore .
\section{Lexicologia}
\begin{itemize}
\item {fónica:csi}
\end{itemize}
\begin{itemize}
\item {Grp. gram.:f.}
\end{itemize}
Parte da grammática, que trata especialmente das palavras, consideradas em relação ao seu valor, á sua formação e a tudo que é necessário para a organização de um léxico.
(Cp. \textunderscore lexicólogo\textunderscore )
\section{Lexicológico}
\begin{itemize}
\item {fónica:csi}
\end{itemize}
\begin{itemize}
\item {Grp. gram.:adj.}
\end{itemize}
Relativo á lexicologia.
\section{Lexicologista}
\begin{itemize}
\item {fónica:csi}
\end{itemize}
\begin{itemize}
\item {Grp. gram.:m.}
\end{itemize}
Aquelle que se occupa de lexicologia.
\section{Lexicólogo}
\begin{itemize}
\item {fónica:csi}
\end{itemize}
\begin{itemize}
\item {Grp. gram.:m.}
\end{itemize}
\begin{itemize}
\item {Proveniência:(Do gr. \textunderscore lexicon\textunderscore  + \textunderscore logos\textunderscore )}
\end{itemize}
Aquelle que se dedica á lexicologia; diccionarista.
\section{Léxicon}
\begin{itemize}
\item {fónica:csi}
\end{itemize}
\begin{itemize}
\item {Grp. gram.:m.}
\end{itemize}
(V.léxico)
\section{Lexilogia}
\begin{itemize}
\item {fónica:csi}
\end{itemize}
\textunderscore f.\textunderscore  (e der.)
O mesmo que \textunderscore lexiologia\textunderscore , etc.
\section{Lexiologia}
\begin{itemize}
\item {fónica:csi}
\end{itemize}
\begin{itemize}
\item {Grp. gram.:f.}
\end{itemize}
\begin{itemize}
\item {Proveniência:(Do gr. \textunderscore lexis\textunderscore  + \textunderscore logos\textunderscore )}
\end{itemize}
Sciência das palavras, consideradas nos seus elementos de formação; lexicologia.
\section{Lexiológico}
\begin{itemize}
\item {fónica:csi}
\end{itemize}
\begin{itemize}
\item {Grp. gram.:adj.}
\end{itemize}
Relativo á lexiologia.
\section{Lexipireto}
\begin{itemize}
\item {fónica:csi}
\end{itemize}
\begin{itemize}
\item {Grp. gram.:adj.}
\end{itemize}
\begin{itemize}
\item {Proveniência:(Lat. \textunderscore lexipyretos\textunderscore )}
\end{itemize}
O mesmo que \textunderscore febrífugo\textunderscore .
\section{Lexipyreto}
\begin{itemize}
\item {fónica:csi}
\end{itemize}
\begin{itemize}
\item {Grp. gram.:adj.}
\end{itemize}
\begin{itemize}
\item {Proveniência:(Lat. \textunderscore lexipyretos\textunderscore )}
\end{itemize}
O mesmo que \textunderscore febrífugo\textunderscore .
\section{Leycéstria}
\begin{itemize}
\item {Grp. gram.:f.}
\end{itemize}
Gênero de plantas caprifoliáceas.
\section{Lézaro}
\begin{itemize}
\item {Grp. gram.:adj.}
\end{itemize}
\begin{itemize}
\item {Utilização:Prov.}
\end{itemize}
\begin{itemize}
\item {Utilização:minh.}
\end{itemize}
Paralýtico.
Entrèvado.
(Por \textunderscore lázaro\textunderscore ?)
\section{Lezema}
\begin{itemize}
\item {Grp. gram.:f.}
\end{itemize}
(?):«\textunderscore ...e fará juramento de lezema\textunderscore ». J. Pedro Ribeiro, \textunderscore Dissert. Chrón.\textunderscore , 303.
\section{Lezira}
\begin{itemize}
\item {Grp. gram.:f.}
\end{itemize}
O mesmo ou melhor que \textunderscore lezíria\textunderscore .
\section{Lezirão}
\begin{itemize}
\item {Grp. gram.:m.}
\end{itemize}
\begin{itemize}
\item {Utilização:T. do Ribatejo}
\end{itemize}
\begin{itemize}
\item {Grp. gram.:Adj.}
\end{itemize}
Grande lezira.
Terreno, inundado por um rio, e em que se semeia arroz.
Grande trato de terreno inculto, nas margens do Tejo.
Diz-se do carro grande de quatro rodas, usado nas lezírias.
\section{Lezíria}
\begin{itemize}
\item {Grp. gram.:f.}
\end{itemize}
\begin{itemize}
\item {Proveniência:(Do ár. \textunderscore al-jazair\textunderscore )}
\end{itemize}
Terreno alagadiço, na margem dos rios.
Margens, que os rios alagam na enchente.
\section{Lhama}
\begin{itemize}
\item {Grp. gram.:f.}
\end{itemize}
Tecido de fios de prata ou de oiro, ou de cobre prateado ou doirado.
(Cast. \textunderscore llama\textunderscore , chamma)
\section{Lhama}
\begin{itemize}
\item {Grp. gram.:f.}
\end{itemize}
Ruminante, o mesmo que \textunderscore lama\textunderscore ^2.
(Do peruv., por intermédio do cast.)
\section{Lhanamente}
\begin{itemize}
\item {Grp. gram.:adv.}
\end{itemize}
De modo lhano; com lhaneza.
\section{Lhandras}
\begin{itemize}
\item {Grp. gram.:f. pl.}
\end{itemize}
O mesmo que \textunderscore liandras\textunderscore .
\section{Lhaneza}
\begin{itemize}
\item {Grp. gram.:f. pl.}
\end{itemize}
Qualidade do que é lhano.
Affabilidade.
Simplicidade.
Lisura.
\section{Lhano}
\begin{itemize}
\item {Grp. gram.:adj.}
\end{itemize}
Sincero; despretensioso; amável.
(Cast. \textunderscore llano\textunderscore )
\section{Lhanura}
\begin{itemize}
\item {Grp. gram.:f.}
\end{itemize}
\begin{itemize}
\item {Utilização:Des.}
\end{itemize}
O mesmo que \textunderscore lhaneza\textunderscore .
Planura.
(Cast. \textunderscore llanura\textunderscore )
\section{Lhe}
\begin{itemize}
\item {Grp. gram.:pron.}
\end{itemize}
\begin{itemize}
\item {Proveniência:(Do lat. \textunderscore illi\textunderscore )}
\end{itemize}
A elle, a ella; a si; o, a.--Algumas vezes se encontra nos clássicos, em vez de \textunderscore lhe\textunderscore .
\section{Lhiçada}
\begin{itemize}
\item {Grp. gram.:f.}
\end{itemize}
\begin{itemize}
\item {Utilização:Prov.}
\end{itemize}
\begin{itemize}
\item {Utilização:trasm.}
\end{itemize}
Embrulhada; intrigalhada.
(Por \textunderscore enlíçada\textunderscore , de \textunderscore enliçar\textunderscore )
\section{Li}
\begin{itemize}
\item {Grp. gram.:m.}
\end{itemize}
Medida itinerária da China.
\section{Lia}
\begin{itemize}
\item {Grp. gram.:f.}
\end{itemize}
Bagaço, de que se faz a água-pé.
Fezes; bôrras.
\section{Lĩa}
\begin{itemize}
\item {Grp. gram.:f.}
\end{itemize}
\begin{itemize}
\item {Utilização:Ant.}
\end{itemize}
O mesmo que \textunderscore linha\textunderscore  (de geração).
\section{Liabo}
\begin{itemize}
\item {Grp. gram.:m.}
\end{itemize}
Gênero de plantas synanthéreas.
\section{Liaça}
\begin{itemize}
\item {Grp. gram.:f.}
\end{itemize}
\begin{itemize}
\item {Proveniência:(Do rad. de \textunderscore liar\textunderscore )}
\end{itemize}
Feixe de palhas, em que se envolvem os vidros, para se não partirem com o transporte.
\section{Liação}
\begin{itemize}
\item {Grp. gram.:f.}
\end{itemize}
Acto ou effeito de liar.
\section{Liáculo}
\begin{itemize}
\item {Grp. gram.:m.}
\end{itemize}
\begin{itemize}
\item {Proveniência:(Lat. \textunderscore liaculum\textunderscore )}
\end{itemize}
Antigo instrumento de pedreiro, para alisar as pedras.
\section{Liadoiro}
\begin{itemize}
\item {Grp. gram.:m.}
\end{itemize}
\begin{itemize}
\item {Proveniência:(De \textunderscore liar\textunderscore )}
\end{itemize}
Pedra, que resai de uma parede, embebendo-se noutra, para a ligar e segurar.
\section{Liador}
\begin{itemize}
\item {Grp. gram.:m.}
\end{itemize}
O mesmo que \textunderscore liadoiro\textunderscore .
\section{Liadouro}
\begin{itemize}
\item {Grp. gram.:m.}
\end{itemize}
\begin{itemize}
\item {Proveniência:(De \textunderscore liar\textunderscore )}
\end{itemize}
Pedra, que resai de uma parede, embebendo-se noutra, para a ligar e segurar.
\section{Lĩagem}
\begin{itemize}
\item {Grp. gram.:f.}
\end{itemize}
\begin{itemize}
\item {Utilização:Ant.}
\end{itemize}
O mesmo que \textunderscore linhagem\textunderscore ^1.
\section{Liágora}
\begin{itemize}
\item {Grp. gram.:f.}
\end{itemize}
Gênero de polypeiros flexíveis.
\section{Liágoro}
\begin{itemize}
\item {Grp. gram.:m.}
\end{itemize}
Gênero de crustáceos decápodes dos mares do Japão.
\section{Liamba}
\begin{itemize}
\item {Grp. gram.:f.}
\end{itemize}
(V.pango)
\section{Liame}
\begin{itemize}
\item {Grp. gram.:m.}
\end{itemize}
\begin{itemize}
\item {Proveniência:(Do lat. \textunderscore ligamen\textunderscore )}
\end{itemize}
O mesmo que \textunderscore liação\textunderscore .
Aquillo que prende ou liga uma coisa ou pessoa a outra.
Cordame de navio de vela.
\section{Liamento}
\begin{itemize}
\item {Grp. gram.:m.}
\end{itemize}
\begin{itemize}
\item {Utilização:Des.}
\end{itemize}
\begin{itemize}
\item {Proveniência:(De \textunderscore liar\textunderscore )}
\end{itemize}
O mesmo que \textunderscore ligamento\textunderscore . Cf. Pant. de Aveiro, \textunderscore Itiner.\textunderscore , 130, (2.^a ed.).
\section{Liana}
\begin{itemize}
\item {Grp. gram.:f.}
\end{itemize}
\begin{itemize}
\item {Utilização:Gal}
\end{itemize}
\begin{itemize}
\item {Proveniência:(Fr. \textunderscore liane\textunderscore )}
\end{itemize}
O mesmo que \textunderscore cipó\textunderscore .
\section{Liança}
\begin{itemize}
\item {Grp. gram.:f.}
\end{itemize}
\begin{itemize}
\item {Proveniência:(De \textunderscore liar\textunderscore )}
\end{itemize}
Liação; alliança; união.
\section{Liandras}
\begin{itemize}
\item {Grp. gram.:f. pl.}
\end{itemize}
\begin{itemize}
\item {Utilização:Prov.}
\end{itemize}
\begin{itemize}
\item {Utilização:alent.}
\end{itemize}
Arcos de circulo, feitos de ferro, que reforçam as pinas dos carros, sobrepondo-se-lhes com a mesma curvatura.
\section{Lião}
\begin{itemize}
\item {Grp. gram.:m.}
\end{itemize}
\begin{itemize}
\item {Utilização:Ant.}
\end{itemize}
O mesmo que \textunderscore leão\textunderscore ^1. Cf. \textunderscore Eufrosina\textunderscore , 180.
\section{Liar}
\begin{itemize}
\item {Grp. gram.:v. t.}
\end{itemize}
O mesmo que \textunderscore ligar\textunderscore .
\section{Lias}
\begin{itemize}
\item {Grp. gram.:m.}
\end{itemize}
\begin{itemize}
\item {Proveniência:(Ingl. \textunderscore lias\textunderscore )}
\end{itemize}
Formação do calcário argilloso.
\section{Liásico}
\begin{itemize}
\item {Grp. gram.:adj.}
\end{itemize}
\begin{itemize}
\item {Utilização:Geol.}
\end{itemize}
Em que há lias; relativo a lias.
\section{Liba}
\begin{itemize}
\item {Grp. gram.:f.}
\end{itemize}
O mesmo que \textunderscore libo\textunderscore .
\section{Libação}
\begin{itemize}
\item {Grp. gram.:f.}
\end{itemize}
\begin{itemize}
\item {Grp. gram.:Pl.}
\end{itemize}
\begin{itemize}
\item {Proveniência:(Lat. \textunderscore libatio\textunderscore )}
\end{itemize}
Acto de libar.
Ceremónia religiosa entre os pagãos, que consistia em provar vinho ou outro liquído e entorná-lo, em honra de uma divindade.
Muitos copos de vinho, bebidos mais por prazer ou para se fazerem brindes, do que por necessidade.
\section{Libambo}
\begin{itemize}
\item {Grp. gram.:m.}
\end{itemize}
\begin{itemize}
\item {Utilização:Bras}
\end{itemize}
\begin{itemize}
\item {Utilização:Bras. do N}
\end{itemize}
Cadeia de ferro, com que se prende pelo pescoço um lote de condemnados, quando saem das prisões para serviço.
Grupo de pessôas; rancho; turba.
(Do quimbundo)
\section{Libame}
\begin{itemize}
\item {Grp. gram.:m.}
\end{itemize}
\begin{itemize}
\item {Proveniência:(Lat. \textunderscore libamen\textunderscore )}
\end{itemize}
Parte da offerta nos sacrifícios pagãos, a qual, depois de consagrada, se derramava no fogo.
Libação.
\section{Libamento}
\begin{itemize}
\item {Grp. gram.:m.}
\end{itemize}
O mesmo que \textunderscore libação\textunderscore .
\section{Libanião}
\begin{itemize}
\item {Grp. gram.:m.}
\end{itemize}
\begin{itemize}
\item {Proveniência:(Do lat. \textunderscore libanus\textunderscore )}
\end{itemize}
Antigo collýrio, em que entrava o incenso.
\section{Libânio}
\begin{itemize}
\item {Grp. gram.:m.}
\end{itemize}
\begin{itemize}
\item {Proveniência:(Do lat. \textunderscore libanius\textunderscore )}
\end{itemize}
Espécie de videira, conhecida dos antigos, á qual attribuíam cheiro do incenso.
\section{Libanomancia}
\begin{itemize}
\item {Grp. gram.:f.}
\end{itemize}
\begin{itemize}
\item {Proveniência:(Do gr. \textunderscore libanos\textunderscore  + \textunderscore manteia\textunderscore )}
\end{itemize}
Adivinhação, que os Pagãos tiravam do incenso.
\section{Libanomântico}
\begin{itemize}
\item {Grp. gram.:adj.}
\end{itemize}
Relativo á libanomancia.
\section{Libar}
\begin{itemize}
\item {Grp. gram.:v. t.}
\end{itemize}
\begin{itemize}
\item {Grp. gram.:V. i.}
\end{itemize}
\begin{itemize}
\item {Proveniência:(Lat. \textunderscore libare\textunderscore )}
\end{itemize}
Beber.
Chupar.
Gozar.
Entornar o vinho da taça, depois de o provar, (ceremónia dos sacrifícios pagãos).
\section{Libata}
\begin{itemize}
\item {Grp. gram.:f.}
\end{itemize}
Grupo de casas, pertencentes a uma família, em África.
O mesmo que \textunderscore senzala\textunderscore .
\section{Libatório}
\begin{itemize}
\item {Grp. gram.:m.}
\end{itemize}
\begin{itemize}
\item {Proveniência:(Lat. \textunderscore libatorium\textunderscore )}
\end{itemize}
Vaso, em que se fazia o libame, nos sacrifícios pagãos.
\section{Libau}
\begin{itemize}
\item {Grp. gram.:m.}
\end{itemize}
Espécie de grande lontra africana, malhada de branco. Cf. Serpa Pinto, I, 299.
\section{Libela}
\begin{itemize}
\item {Grp. gram.:f.}
\end{itemize}
\begin{itemize}
\item {Proveniência:(Lat. \textunderscore libella\textunderscore )}
\end{itemize}
Pequena moéda de prata, entre os Romanos, a qual valia a décima parte do denário.
\section{Lial}
\begin{itemize}
\item {Grp. gram.:adj.}
\end{itemize}
\begin{itemize}
\item {Grp. gram.:M.}
\end{itemize}
\begin{itemize}
\item {Proveniência:(Do lat. \textunderscore legalis\textunderscore )}
\end{itemize}
Conforme com a lei.
Digno.
Honesto.
Sincero.
Fiel: \textunderscore espôsa lial\textunderscore .
Antiga moeda de prata, que na Índia portuguesa valia 12 reis.
Moéda, correspondente a 10 reis, em tempo de D. João I.
\section{Libelinha}
\begin{itemize}
\item {Grp. gram.:f.}
\end{itemize}
Insecto ortóptero, o mesmo que \textunderscore donzelinha\textunderscore , (\textunderscore libellula virgo\textunderscore ).
\section{Libelista}
\begin{itemize}
\item {Grp. gram.:m.}
\end{itemize}
\begin{itemize}
\item {Utilização:Fig.}
\end{itemize}
Aquele que faz libelo.
Aquele que formúla acusações.
\section{Libella}
\begin{itemize}
\item {Grp. gram.:f.}
\end{itemize}
\begin{itemize}
\item {Proveniência:(Lat. \textunderscore libella\textunderscore )}
\end{itemize}
Pequena moéda de prata, entre os Romanos, a qual valia a décima parte do denário.
\section{Libellinha}
\begin{itemize}
\item {Grp. gram.:f.}
\end{itemize}
Insecto orthóptero, o mesmo que \textunderscore donzellinha\textunderscore , (\textunderscore libellula virgo\textunderscore ).
\section{Libellista}
\begin{itemize}
\item {Grp. gram.:m.}
\end{itemize}
\begin{itemize}
\item {Utilização:Fig.}
\end{itemize}
Aquelle que faz libello.
Aquelle que formúla accusações.
\section{Libello}
\begin{itemize}
\item {Grp. gram.:m.}
\end{itemize}
\begin{itemize}
\item {Utilização:Ext.}
\end{itemize}
\begin{itemize}
\item {Proveniência:(Lat. \textunderscore libellus\textunderscore )}
\end{itemize}
Exposição articulada do que se pretende provar contra um réu.
Artigo ou escrito, que envolve accusação de alguém.
\section{Libéllula}
\begin{itemize}
\item {Grp. gram.:f.}
\end{itemize}
Designação scientífica da \textunderscore libellinha\textunderscore .
\section{Libéllulo}
\begin{itemize}
\item {Grp. gram.:m.}
\end{itemize}
O mesmo que \textunderscore libéllula\textunderscore .
\section{Libelo}
\begin{itemize}
\item {Grp. gram.:m.}
\end{itemize}
\begin{itemize}
\item {Utilização:Ext.}
\end{itemize}
\begin{itemize}
\item {Proveniência:(Lat. \textunderscore libellus\textunderscore )}
\end{itemize}
Exposição articulada do que se pretende provar contra um réu.
Artigo ou escrito, que envolve acusação de alguém.
\section{Libélula}
\begin{itemize}
\item {Grp. gram.:f.}
\end{itemize}
Designação científica da \textunderscore libelinha\textunderscore .
\section{Libelulo}
\begin{itemize}
\item {Grp. gram.:m.}
\end{itemize}
O mesmo que \textunderscore libélula\textunderscore .
\section{Libentemente}
\begin{itemize}
\item {Grp. gram.:adv.}
\end{itemize}
\begin{itemize}
\item {Proveniência:(Do lat. \textunderscore libens\textunderscore , \textunderscore libentis\textunderscore )}
\end{itemize}
De bôa vontade; voluntariamente.
\section{Libentíssimo}
\begin{itemize}
\item {Grp. gram.:adj.}
\end{itemize}
\begin{itemize}
\item {Proveniência:(Lat. \textunderscore libentissimus\textunderscore )}
\end{itemize}
Muito bem disposto a favor de alguém.
\section{Líber}
\begin{itemize}
\item {Grp. gram.:m.}
\end{itemize}
\begin{itemize}
\item {Utilização:Bot.}
\end{itemize}
\begin{itemize}
\item {Proveniência:(Lat. \textunderscore liber\textunderscore )}
\end{itemize}
A camada cortical, mais próxima do alburno; entrecasca.
\section{Liberação}
\begin{itemize}
\item {Grp. gram.:f.}
\end{itemize}
\begin{itemize}
\item {Proveniência:(Lat. \textunderscore liberatio\textunderscore )}
\end{itemize}
Extincção de uma obrigação ou dívida.
\section{Liberal}
\begin{itemize}
\item {Grp. gram.:adj.}
\end{itemize}
\begin{itemize}
\item {Grp. gram.:M.}
\end{itemize}
\begin{itemize}
\item {Proveniência:(Lat. \textunderscore liberalis\textunderscore )}
\end{itemize}
Generoso; que gosta de dar; franco.
Partidário do princípio da liberdade política e civil.
Que tem ideias avançadas em sociologia.
Próprio de homem livre.
Sectário da liberdade, em política.
\section{Liberalão}
\begin{itemize}
\item {Grp. gram.:m.}
\end{itemize}
\begin{itemize}
\item {Utilização:Deprec.}
\end{itemize}
Aquelle que alardeia ridiculamente de liberal.
\section{Liberalengo}
\begin{itemize}
\item {Grp. gram.:adj.}
\end{itemize}
\begin{itemize}
\item {Utilização:Deprec.}
\end{itemize}
O mesmo que \textunderscore liberalesco\textunderscore . Cf. Júl. Dinis, \textunderscore Fidalgos\textunderscore .
\section{Liberalesco}
\begin{itemize}
\item {fónica:lês}
\end{itemize}
\begin{itemize}
\item {Grp. gram.:adj.}
\end{itemize}
\begin{itemize}
\item {Utilização:Deprec.}
\end{itemize}
Relativo ao partido liberal.
\section{Liberaleza}
\begin{itemize}
\item {Grp. gram.:f.}
\end{itemize}
\begin{itemize}
\item {Utilização:Ant.}
\end{itemize}
O mesmo que \textunderscore liberalidade\textunderscore .
\section{Liberalidade}
\begin{itemize}
\item {Grp. gram.:f.}
\end{itemize}
\begin{itemize}
\item {Proveniência:(Lat. \textunderscore liberalitas\textunderscore )}
\end{itemize}
Qualidade de quem é liberal ou generoso; generosidade.
\section{Liberalismo}
\begin{itemize}
\item {Grp. gram.:m.}
\end{itemize}
\begin{itemize}
\item {Proveniência:(De \textunderscore liberal\textunderscore )}
\end{itemize}
Systema dos partidários da liberdade política e civil.
Profissão de princípios liberaes.
\section{Liberalista}
\begin{itemize}
\item {Grp. gram.:adj.}
\end{itemize}
\begin{itemize}
\item {Grp. gram.:M.}
\end{itemize}
\begin{itemize}
\item {Proveniência:(De \textunderscore liberal\textunderscore )}
\end{itemize}
Relativo ao liberalismo.
Partidário do liberalismo.
\section{Liberalizar}
\begin{itemize}
\item {Grp. gram.:v. t.}
\end{itemize}
\begin{itemize}
\item {Proveniência:(De \textunderscore liberal\textunderscore )}
\end{itemize}
Dar com liberalidade; larguear; prodigalizar.
\section{Liberalmente}
\begin{itemize}
\item {Grp. gram.:adv.}
\end{itemize}
De modo liberal; com generosidade.
\section{Liberanga}
\begin{itemize}
\item {Grp. gram.:m.}
\end{itemize}
\begin{itemize}
\item {Utilização:Deprec.}
\end{itemize}
O mesmo que \textunderscore liberalão\textunderscore .
\section{Liberar}
\begin{itemize}
\item {Grp. gram.:v. t.}
\end{itemize}
\begin{itemize}
\item {Utilização:Jur.}
\end{itemize}
\begin{itemize}
\item {Proveniência:(Lat. \textunderscore liberare\textunderscore )}
\end{itemize}
Tornar livre ou quite.
Desobrigar.
Entregar ao tomador de acções cédulas que as representam, até que as acções se passem definitivamente.
\section{Liberasta}
\begin{itemize}
\item {Grp. gram.:m.}
\end{itemize}
\begin{itemize}
\item {Utilização:Deprec.}
\end{itemize}
O mesmo que \textunderscore liberalão\textunderscore .
\section{Liberativo}
\begin{itemize}
\item {Grp. gram.:adj.}
\end{itemize}
\begin{itemize}
\item {Proveniência:(Do lat. \textunderscore liberatus\textunderscore )}
\end{itemize}
Que liberta, que desobriga.
Libertador.
\section{Liberatório}
\begin{itemize}
\item {Grp. gram.:adj.}
\end{itemize}
\begin{itemize}
\item {Proveniência:(Do lat. \textunderscore liberatus\textunderscore )}
\end{itemize}
Relativo a liberação.
Próprio para liberar ou representar valores pecuniários: \textunderscore o valor liberatório das notas de banco é-lhes dado pelo Govêrno\textunderscore .
\section{Liberdade}
\begin{itemize}
\item {Grp. gram.:f.}
\end{itemize}
\begin{itemize}
\item {Proveniência:(Lat. \textunderscore libertas\textunderscore )}
\end{itemize}
Condição do homem que póde dispor de si ou que não é propriedade de outrem.
Poder de fazer ou deixar de fazer uma coisa.
Livre arbítrio.
Faculdade de praticar tudo aquillo que não é prohibido por lei.
Conjunto ou personificação das ideias liberaes: \textunderscore amor á liberdade\textunderscore .
Permissão.
Immunidade.
Deliberação.
Ousadia: \textunderscore tómo a liberdade de lhe fazer um pedido\textunderscore .
Conjunto dos direitos, garantidos ao cidadão pela lei fundamental do Estado.
\section{Liberdoso}
\begin{itemize}
\item {Grp. gram.:adj.}
\end{itemize}
\begin{itemize}
\item {Utilização:P. us.}
\end{itemize}
Em que há liberdade; que revela liberdade.
\section{Liberiano}
\begin{itemize}
\item {Grp. gram.:adj.}
\end{itemize}
\begin{itemize}
\item {Grp. gram.:M.}
\end{itemize}
Relativo á Libéria.
Aquelle que é natural da Libéria.
\section{Liberne}
\begin{itemize}
\item {Grp. gram.:m.}
\end{itemize}
\begin{itemize}
\item {Utilização:Prov.}
\end{itemize}
\begin{itemize}
\item {Utilização:alent.}
\end{itemize}
Animal, que suppõe sêr filho de lobo e raposa e que é muito temido. (Colhido em Odemira)
\section{Libérrimo}
\begin{itemize}
\item {Grp. gram.:adj.}
\end{itemize}
\begin{itemize}
\item {Proveniência:(Lat. \textunderscore liberrimus\textunderscore )}
\end{itemize}
Muito livre.
\section{Libertação}
\begin{itemize}
\item {Grp. gram.:f.}
\end{itemize}
Acto de libertar.
\section{Libertador}
\begin{itemize}
\item {Grp. gram.:adj.}
\end{itemize}
\begin{itemize}
\item {Grp. gram.:M.}
\end{itemize}
Que liberta.
Que dá liberdade.
Aquelle que dá liberdade, que torna livre.
\section{Libertar}
\begin{itemize}
\item {Grp. gram.:v. t.}
\end{itemize}
\begin{itemize}
\item {Proveniência:(De \textunderscore liberto\textunderscore )}
\end{itemize}
Tornar livre; dar liberdade a.
Tornar quite, desobrigar.
\section{Libertário}
\begin{itemize}
\item {Grp. gram.:m.}
\end{itemize}
\begin{itemize}
\item {Utilização:Neol.}
\end{itemize}
\begin{itemize}
\item {Proveniência:(De \textunderscore libertar\textunderscore )}
\end{itemize}
O mesmo que \textunderscore anarchista\textunderscore .
\section{Libértia}
\begin{itemize}
\item {Grp. gram.:f.}
\end{itemize}
\begin{itemize}
\item {Proveniência:(De \textunderscore Libert\textunderscore , n. p.)}
\end{itemize}
Gênero de plantas irídeas.
\section{Liberticida}
\begin{itemize}
\item {Grp. gram.:m. ,  f.  e  adj.}
\end{itemize}
\begin{itemize}
\item {Proveniência:(Do lat. \textunderscore libertas\textunderscore  + \textunderscore caedere\textunderscore )}
\end{itemize}
Pessôa, que destroi ou procura destruír as liberdades ou immunidades de um país.
\section{Liberticídio}
\begin{itemize}
\item {Grp. gram.:m.}
\end{itemize}
Destruição da liberdade política de um país.
(Cp. \textunderscore liberticida\textunderscore )
\section{Libertinagem}
\begin{itemize}
\item {Grp. gram.:f.}
\end{itemize}
Vida de libertino; devassidão.
\section{Libertinamente}
\begin{itemize}
\item {Grp. gram.:adv.}
\end{itemize}
De modo libertino; licenciosamente.
\section{Libertinidade}
\begin{itemize}
\item {Grp. gram.:f.}
\end{itemize}
O mesmo que \textunderscore libertinagem\textunderscore .
\section{Libertino}
\begin{itemize}
\item {Grp. gram.:m.  e  adj.}
\end{itemize}
\begin{itemize}
\item {Proveniência:(Lat. \textunderscore libertinus\textunderscore )}
\end{itemize}
Devasso; dissoluto.
Ímpio.
\section{Libertista}
\begin{itemize}
\item {Grp. gram.:m.}
\end{itemize}
\begin{itemize}
\item {Proveniência:(Do lat. \textunderscore libertas\textunderscore )}
\end{itemize}
Partidário da doutrina do livre arbítrio.
\section{Liberto}
\begin{itemize}
\item {Grp. gram.:adj.}
\end{itemize}
\begin{itemize}
\item {Proveniência:(Lat. \textunderscore libertus\textunderscore )}
\end{itemize}
Dizia-se do escravo que foi libertado.
Livre; desopprimido.
\section{Líbi}
\begin{itemize}
\item {Grp. gram.:m.}
\end{itemize}
Linhaça de Mindanau, nas ilhas Filippinas.
\section{Líbico}
\begin{itemize}
\item {Grp. gram.:adj.}
\end{itemize}
\begin{itemize}
\item {Grp. gram.:M.}
\end{itemize}
\begin{itemize}
\item {Proveniência:(Lat. \textunderscore libycus\textunderscore )}
\end{itemize}
Relativo á Líbia.
Vento do sudoéste, segundo a antiga náutica.
\section{Libidinagem}
\begin{itemize}
\item {Grp. gram.:f.}
\end{itemize}
Vida de libidinoso.
Sensualidade.
(Cp. \textunderscore libidinoso\textunderscore )
\section{Libidinosamente}
\begin{itemize}
\item {Grp. gram.:adv.}
\end{itemize}
De modo libidinoso.
Voluptuosamente; com appetites sensuaes.
\section{Libidinoso}
\begin{itemize}
\item {Grp. gram.:adj.}
\end{itemize}
\begin{itemize}
\item {Grp. gram.:M.}
\end{itemize}
\begin{itemize}
\item {Proveniência:(Lat. \textunderscore libidinosus\textunderscore )}
\end{itemize}
Que sente vivos desejos de prazer.
Que tem appetites sensuaes.
Relativo a sensualidade; voluptuoso; lascivo.
Incontinente.
Individuo lascivo, dissoluto.
\section{Libínia}
\begin{itemize}
\item {Grp. gram.:f.}
\end{itemize}
Gênero de crustáceos decápodes.
\section{Líbio}
\begin{itemize}
\item {Grp. gram.:adj.}
\end{itemize}
\begin{itemize}
\item {Proveniência:(Lat. \textunderscore libyus\textunderscore )}
\end{itemize}
O mesmo que \textunderscore líbico\textunderscore .
\section{Libitina}
\begin{itemize}
\item {Grp. gram.:f.}
\end{itemize}
\begin{itemize}
\item {Utilização:Poét.}
\end{itemize}
\begin{itemize}
\item {Proveniência:(Lat. \textunderscore libitina\textunderscore )}
\end{itemize}
A morte.
\section{Libitinário}
\begin{itemize}
\item {Grp. gram.:m.}
\end{itemize}
\begin{itemize}
\item {Proveniência:(Lat. \textunderscore libitinarius\textunderscore )}
\end{itemize}
Funccionário que, em Roma, presidia ás ceremónias fúnebres.
\section{Líbito}
\begin{itemize}
\item {Grp. gram.:m.}
\end{itemize}
\begin{itemize}
\item {Proveniência:(Lat. \textunderscore libitum\textunderscore )}
\end{itemize}
Arbítrio.
Aquillo que apraz; talante.
\section{Libo}
\begin{itemize}
\item {Grp. gram.:m.}
\end{itemize}
\begin{itemize}
\item {Proveniência:(Lat. \textunderscore libum\textunderscore )}
\end{itemize}
Bolo, feito de farinha, queijo, mel, azeite e ovos, que se offerecia aos deuses, entre os Romanos.
\section{Libó}
\begin{itemize}
\item {Grp. gram.:m.}
\end{itemize}
Arvoreta santhomense, de raíz medicinal.
\section{Libolos}
\begin{itemize}
\item {Grp. gram.:m. pl.}
\end{itemize}
Tríbo conguesa das margens do Gango.
\section{Libongo}
\begin{itemize}
\item {Grp. gram.:m.}
\end{itemize}
Pequena moéda africana.
Espécie de pano, com que os Europeus traficam na costa da África.
\section{Libónia}
\begin{itemize}
\item {Grp. gram.:f.}
\end{itemize}
Planta dos jardins no Brasil.
\section{Libra}
\begin{itemize}
\item {Grp. gram.:f.}
\end{itemize}
\begin{itemize}
\item {Proveniência:(Lat. \textunderscore libra\textunderscore )}
\end{itemize}
O mesmo que \textunderscore arrátel\textunderscore .
Antigo pêso de 16 onças, nas pharmácias.
Moéda de oiro inglesa; esterlina.
Antiga moéda portuguesa, que começou a correr em tempo de Affonso I.
\section{Libração}
\begin{itemize}
\item {Grp. gram.:f.}
\end{itemize}
\begin{itemize}
\item {Proveniência:(Lat. \textunderscore libratio\textunderscore )}
\end{itemize}
Acto ou effeito de librar.
Oscillação de um corpo que procura o equilíbrio.
Oscillação apparente da Lua.
\section{Librame}
\begin{itemize}
\item {Grp. gram.:m.}
\end{itemize}
\begin{itemize}
\item {Utilização:Fam.}
\end{itemize}
Grande porção de libras, (moédas).
\section{Librar}
\begin{itemize}
\item {Grp. gram.:v. t.}
\end{itemize}
\begin{itemize}
\item {Grp. gram.:V. p.}
\end{itemize}
\begin{itemize}
\item {Proveniência:(Lat. \textunderscore librare\textunderscore )}
\end{itemize}
Equilibrar.
Suspender.
Basear.
Estar suspenso no ar; pairar.
\section{Libratório}
\begin{itemize}
\item {Grp. gram.:adj.}
\end{itemize}
\begin{itemize}
\item {Proveniência:(De \textunderscore librar\textunderscore )}
\end{itemize}
Em que há oscillação: \textunderscore movimento libratório\textunderscore .
\section{Libré}
\begin{itemize}
\item {Grp. gram.:f.}
\end{itemize}
\begin{itemize}
\item {Utilização:Chul.}
\end{itemize}
\begin{itemize}
\item {Utilização:Fig.}
\end{itemize}
\begin{itemize}
\item {Proveniência:(Do fr. \textunderscore livrée\textunderscore )}
\end{itemize}
Uniforme de criados, em casas nobres.
Farda, uniforme.
Fato.
Aspecto.
\section{Libréa}
\begin{itemize}
\item {Grp. gram.:f.}
\end{itemize}
\begin{itemize}
\item {Utilização:Des.}
\end{itemize}
O mesmo que \textunderscore libré\textunderscore .
\section{Libreia}
\begin{itemize}
\item {Grp. gram.:f.}
\end{itemize}
\begin{itemize}
\item {Utilização:Des.}
\end{itemize}
O mesmo que \textunderscore libré\textunderscore .
\section{Libretista}
\begin{itemize}
\item {Grp. gram.:m.}
\end{itemize}
Aquelle que escreve libreto.
\section{Libreto}
\begin{itemize}
\item {fónica:brê}
\end{itemize}
\begin{itemize}
\item {Grp. gram.:m.}
\end{itemize}
\begin{itemize}
\item {Proveniência:(It. \textunderscore libretto\textunderscore )}
\end{itemize}
Palavras ou letra de uma ópera.
Argumento ou exposição da acção e episódios, que servem de base a uma ópera.
\section{Libréu}
\begin{itemize}
\item {Grp. gram.:m.}
\end{itemize}
\begin{itemize}
\item {Utilização:Ant.}
\end{itemize}
O mesmo que \textunderscore lebréu\textunderscore . Cf. \textunderscore Peregrinação\textunderscore , CXXIV.
\section{Librina}
\begin{itemize}
\item {Grp. gram.:f.}
\end{itemize}
\begin{itemize}
\item {Utilização:Bras. do N}
\end{itemize}
\begin{itemize}
\item {Proveniência:(De \textunderscore librinar\textunderscore )}
\end{itemize}
O mesmo que \textunderscore chuvisco\textunderscore .
\section{Librinar}
\begin{itemize}
\item {Grp. gram.:v. i.}
\end{itemize}
\begin{itemize}
\item {Utilização:Bras. do N}
\end{itemize}
Cair chuva miúda, chuviscar.
\section{Libripende}
\begin{itemize}
\item {Grp. gram.:m.}
\end{itemize}
\begin{itemize}
\item {Proveniência:(Lat. \textunderscore libripens\textunderscore )}
\end{itemize}
Aquelle que pesava e pagava o sôldo ás tropas romanas, antes que em Roma houvesse moéda, isto é, antes das guerras do Pyrrho.
\section{Libro}
\begin{itemize}
\item {Grp. gram.:m.}
\end{itemize}
\begin{itemize}
\item {Utilização:Bot.}
\end{itemize}
O mesmo que \textunderscore líber\textunderscore :«\textunderscore ...entre a casca e o libro\textunderscore ». Castilho, \textunderscore Escav. Poét.\textunderscore , 22.
\section{Libua}
\begin{itemize}
\item {Grp. gram.:f.}
\end{itemize}
O mesmo que \textunderscore sabra\textunderscore .
\section{Liburna}
\begin{itemize}
\item {Grp. gram.:f.}
\end{itemize}
\begin{itemize}
\item {Proveniência:(Lat. \textunderscore liburna\textunderscore )}
\end{itemize}
Pequena e ligeira embarcação, usada pelos antigos Romanos; espécie de bergantim.
\section{Libúrneo}
\begin{itemize}
\item {Grp. gram.:adj.}
\end{itemize}
Relativo a liburno.
\section{Liburno}
\begin{itemize}
\item {Grp. gram.:m.}
\end{itemize}
\begin{itemize}
\item {Proveniência:(Lat. \textunderscore liburnus\textunderscore )}
\end{itemize}
Escravo que, em Roma, transportava a cadeirinha ou liteira de pessôas nobres ou ricas.
\section{Líbyco}
\begin{itemize}
\item {Grp. gram.:adj.}
\end{itemize}
\begin{itemize}
\item {Grp. gram.:M.}
\end{itemize}
\begin{itemize}
\item {Proveniência:(Lat. \textunderscore libycus\textunderscore )}
\end{itemize}
Relativo á Líbya.
Vento do sudoéste, segundo a antiga náutica.
\section{Líbyo}
\begin{itemize}
\item {Grp. gram.:adj.}
\end{itemize}
\begin{itemize}
\item {Proveniência:(Lat. \textunderscore libyus\textunderscore )}
\end{itemize}
O mesmo que \textunderscore líbyco\textunderscore .
\section{Liça}
\begin{itemize}
\item {Grp. gram.:f.}
\end{itemize}
\begin{itemize}
\item {Utilização:Fig.}
\end{itemize}
\begin{itemize}
\item {Proveniência:(Do b. lat. \textunderscore licia\textunderscore )}
\end{itemize}
Lugar, destinado a torneios, justas, etc.
Luta; briga.
Torneio.
Lugar, em que se debatem questões importantes.
Objecto de discussões graves.
\section{Liça}
\begin{itemize}
\item {Grp. gram.:f.}
\end{itemize}
Nome do mugem negrão, quando novo.
\section{Liça}
\begin{itemize}
\item {Grp. gram.:f.}
\end{itemize}
\begin{itemize}
\item {Utilização:Bras}
\end{itemize}
Peça de máquinas de tecidos, semelhante a um pente fechado, e feita de arame ou cordão, para erguer os fios.
(Cp. \textunderscore liço\textunderscore )
\section{Liçada}
\begin{itemize}
\item {Grp. gram.:f.}
\end{itemize}
\begin{itemize}
\item {Utilização:Prov.}
\end{itemize}
\begin{itemize}
\item {Utilização:trasm.}
\end{itemize}
O mesmo que \textunderscore lhiçada\textunderscore .
\section{Licanço}
\begin{itemize}
\item {Grp. gram.:m.}
\end{itemize}
O quatro de paus, no jôgo do truque.
\section{Licanço}
\begin{itemize}
\item {Grp. gram.:m.}
\end{itemize}
\begin{itemize}
\item {Utilização:Pop.}
\end{itemize}
O mesmo que \textunderscore licranço\textunderscore .
\section{Lição}
\begin{itemize}
\item {Grp. gram.:f.}
\end{itemize}
\begin{itemize}
\item {Proveniência:(Lat. \textunderscore lectio\textunderscore )}
\end{itemize}
Acto de lêr: \textunderscore recommenda-se a lição dos clássicos\textunderscore .
Exposição doutrinária, feita por lente ou professor: \textunderscore assistir a uma lição de Botânica\textunderscore .
Ponto ou assumpto, que um discípulo deve estudar, por indicação do professor: \textunderscore estudar a lição\textunderscore .
Variante de palavra ou passagem de uma obra.
Experiência; exemplo: \textunderscore aquillo serviu-me de lição\textunderscore .
Preceito.
Reprehensão; punição: \textunderscore apanhou bôa lição\textunderscore .
\section{Liçaróes}
\begin{itemize}
\item {Grp. gram.:m.}
\end{itemize}
\begin{itemize}
\item {Proveniência:(Do rad. de \textunderscore liço\textunderscore )}
\end{itemize}
Travessas, que seguram os liços.
\section{Liçaróis}
\begin{itemize}
\item {Grp. gram.:m.}
\end{itemize}
\begin{itemize}
\item {Proveniência:(Do rad. de \textunderscore liço\textunderscore )}
\end{itemize}
Travessas, que seguram os liços.
\section{Licate}
\begin{itemize}
\item {Grp. gram.:m.}
\end{itemize}
\begin{itemize}
\item {Utilização:Ant.}
\end{itemize}
O mesmo que \textunderscore alicate\textunderscore .
\section{Lice}
\begin{itemize}
\item {Grp. gram.:f.}
\end{itemize}
O mesmo que \textunderscore liça\textunderscore ^1. Cf. Filinto, XVII, 161; Garrett, \textunderscore Romanceiro\textunderscore , II, XLIII.
\section{Licença}
\begin{itemize}
\item {Grp. gram.:f.}
\end{itemize}
\begin{itemize}
\item {Utilização:Fig.}
\end{itemize}
\begin{itemize}
\item {Proveniência:(Do lat. \textunderscore licentia\textunderscore )}
\end{itemize}
Permissão.
Faculdade.
Liberdade.
Autorização, consentimento.
Vida dissoluta; desregramento.
\section{Licenciado}
\begin{itemize}
\item {Grp. gram.:adj.}
\end{itemize}
\begin{itemize}
\item {Grp. gram.:M.}
\end{itemize}
\begin{itemize}
\item {Proveniência:(Do lat. \textunderscore licentiatus\textunderscore )}
\end{itemize}
Que tem o título universitário, immediatamente anterior ao de doutor.
Aquelle que tem o grau do licenciatura.
\section{Licenciamento}
\begin{itemize}
\item {Grp. gram.:m.}
\end{itemize}
Acto ou effeito de licenciar.
\section{Licenciar}
\begin{itemize}
\item {Grp. gram.:v. t.}
\end{itemize}
\begin{itemize}
\item {Grp. gram.:V. p.}
\end{itemize}
\begin{itemize}
\item {Proveniência:(Lat. \textunderscore licentiare\textunderscore )}
\end{itemize}
Dar licença a.
Dispensar do serviço: \textunderscore licenciar soldados\textunderscore .
Despedir; isentar.
Tomar o grau de licenciado.
\section{Licenciatura}
\begin{itemize}
\item {Grp. gram.:f.}
\end{itemize}
\begin{itemize}
\item {Proveniência:(De \textunderscore licenciar\textunderscore )}
\end{itemize}
O mesmo que \textunderscore licenciamento\textunderscore .
Grau de licenciado.
Acto do conferir êste grau.
\section{Licenciosamente}
\begin{itemize}
\item {Grp. gram.:adv.}
\end{itemize}
De modo licencioso; dissolutamente.
\section{Licenciosidade}
\begin{itemize}
\item {Grp. gram.:f.}
\end{itemize}
Qualidade de licencioso.
\section{Licencioso}
\begin{itemize}
\item {Grp. gram.:adj.}
\end{itemize}
\begin{itemize}
\item {Proveniência:(Lat. \textunderscore licentiosus\textunderscore )}
\end{itemize}
Que usa de demasiada licença; desregrado.
Sensual; libertino.
\section{Líchen}
\begin{itemize}
\item {fónica:quen}
\end{itemize}
\begin{itemize}
\item {Grp. gram.:m.}
\end{itemize}
\begin{itemize}
\item {Grp. gram.:Pl.}
\end{itemize}
\begin{itemize}
\item {Proveniência:(Lat. \textunderscore lichen\textunderscore )}
\end{itemize}
Classe de plantas cryptogâmicas, cuja vida é interrompida pela estiagem, e que formam a transição das algas para os cogumelos.
Espécie de impigem no rosto.
Líchenes.
\section{Lichenáceas}
\begin{itemize}
\item {fónica:que}
\end{itemize}
\begin{itemize}
\item {Grp. gram.:f. pl.}
\end{itemize}
Família de plantas, que comprehende os líchenes.
\section{Lichêneas}
\begin{itemize}
\item {fónica:quê}
\end{itemize}
\begin{itemize}
\item {Grp. gram.:f. pl.}
\end{itemize}
(V.lichenáceas)
\section{Lichênico}
\begin{itemize}
\item {fónica:quê}
\end{itemize}
\begin{itemize}
\item {Grp. gram.:adj.}
\end{itemize}
Diz-se de um ácido, que se descobriu em certos líchenes.
\section{Lichenina}
\begin{itemize}
\item {fónica:que}
\end{itemize}
\begin{itemize}
\item {Grp. gram.:f.}
\end{itemize}
\begin{itemize}
\item {Proveniência:(De \textunderscore líchen\textunderscore )}
\end{itemize}
Fécula extrahída de certas liehenáceas. Cf. \textunderscore Techn. Rur.\textunderscore , 400.
\section{Lichenographia}
\begin{itemize}
\item {fónica:que}
\end{itemize}
\begin{itemize}
\item {Grp. gram.:f.}
\end{itemize}
Parte da Botânica, que trata especialmente dos líchenes.
\section{Lichenográphico}
\begin{itemize}
\item {fónica:que}
\end{itemize}
\begin{itemize}
\item {Grp. gram.:adj.}
\end{itemize}
Relativo á lichenographia.
\section{Lichino}
\begin{itemize}
\item {Grp. gram.:m.}
\end{itemize}
\begin{itemize}
\item {Proveniência:(Do lat. \textunderscore licinium\textunderscore )}
\end{itemize}
Torcida medicamentosa, que se embebe nas feridas profundas.
\section{Liciatório}
\begin{itemize}
\item {Grp. gram.:m.}
\end{itemize}
\begin{itemize}
\item {Proveniência:(Lat. \textunderscore liciatorium\textunderscore )}
\end{itemize}
Pente, por onde correm os fios da teia; pente de tear.
\section{Licínia}
\begin{itemize}
\item {Grp. gram.:f.}
\end{itemize}
Variedade de azeitona, conhecida dos antigos, e que era a mesma que chamamos \textunderscore cordovesa\textunderscore .
\section{Licitação}
\begin{itemize}
\item {Grp. gram.:f.}
\end{itemize}
\begin{itemize}
\item {Proveniência:(Lat. \textunderscore licitatio\textunderscore )}
\end{itemize}
Acto de licitar.
\section{Licitador}
\begin{itemize}
\item {Grp. gram.:m.  e  adj.}
\end{itemize}
\begin{itemize}
\item {Proveniência:(Lat. \textunderscore licitator\textunderscore )}
\end{itemize}
O que licita.
\section{Licitamente}
\begin{itemize}
\item {Grp. gram.:adv.}
\end{itemize}
De modo lícito; legalmente.
\section{Licitante}
\begin{itemize}
\item {Grp. gram.:m.  e  adj.}
\end{itemize}
\begin{itemize}
\item {Proveniência:(Lat. \textunderscore licitans\textunderscore )}
\end{itemize}
O mesmo que \textunderscore licitador\textunderscore .
\section{Licitar}
\begin{itemize}
\item {Grp. gram.:v. i.}
\end{itemize}
\begin{itemize}
\item {Grp. gram.:V. t.}
\end{itemize}
\begin{itemize}
\item {Proveniência:(Lat. \textunderscore licitari\textunderscore )}
\end{itemize}
Offerecer um lanço ou uma quantia, para obter em almoéda ou partilha judicial a adjudicação do que se vende em hasta pública ou do que se vende a quem mais dér.
Pôr em arrematação ou partilha.
Offerecer lanço sôbre.
\section{Lícito}
\begin{itemize}
\item {Grp. gram.:adj.}
\end{itemize}
\begin{itemize}
\item {Grp. gram.:M.}
\end{itemize}
\begin{itemize}
\item {Proveniência:(Lat. \textunderscore licitus\textunderscore )}
\end{itemize}
Conforme com a lei; que não é prohibido por lei; permittido: \textunderscore actos lícitos\textunderscore .
Aquillo que é permittido, aquillo que é justo.
\section{Liço}
\begin{itemize}
\item {Grp. gram.:m.}
\end{itemize}
\begin{itemize}
\item {Proveniência:(Do lat. \textunderscore licium\textunderscore )}
\end{itemize}
Cada um dos fios entre duas travessas, através dos quaes passa a urdidura do tear, e que, elevando-se ou abaixando a cada passagem da lançadeira, determinam o tecido com o fio que sái desta.
\section{Liconde}
\begin{itemize}
\item {Grp. gram.:m.}
\end{itemize}
O mesmo que \textunderscore licondo\textunderscore .
\section{Licondo}
\begin{itemize}
\item {Grp. gram.:m.}
\end{itemize}
Árvore angolense, cuja casca é fibrosa e têxtil.
O mesmo que \textunderscore liconte\textunderscore ?
\section{Liconte}
\begin{itemize}
\item {Grp. gram.:m.}
\end{itemize}
O mesmo que \textunderscore imbondeiro\textunderscore .
Tecido grosseiro, feito de filamentos do líber do imbondeiro ou adansónia.
\section{Licor}
\begin{itemize}
\item {Grp. gram.:m.}
\end{itemize}
\begin{itemize}
\item {Utilização:Restrict.}
\end{itemize}
\begin{itemize}
\item {Proveniência:(Lat. \textunderscore liquor\textunderscore )}
\end{itemize}
Qualquer líquido.
Humor.
Bebida espirituosa e açucarada.
Líquido alcoólico, composto em pharmácia.
\section{Licoreira}
\begin{itemize}
\item {Grp. gram.:f.}
\end{itemize}
Utensílio, que contém garrafa ou outra vasilha, e copos, para licor.
\section{Licoreiro}
\begin{itemize}
\item {Grp. gram.:m.}
\end{itemize}
Utensílio, que contém garrafa ou outra vasilha, e copos, para licor.
\section{Licorista}
\begin{itemize}
\item {Grp. gram.:m.}
\end{itemize}
Fabricante ou vendedor de licores.
\section{Licorne}
\begin{itemize}
\item {Grp. gram.:m.}
\end{itemize}
Animal fantástico, que figura nos brasões e que é representado com um só corno.
Constellação meridional.
Gênero de molluscos, quo têm um dente na borda da abertura da concha.
(Corr. de \textunderscore unicorne\textunderscore )
\section{Licórnio}
\begin{itemize}
\item {Grp. gram.:m.}
\end{itemize}
(V.unicórnio)
\section{Licoroso}
\begin{itemize}
\item {Grp. gram.:adj.}
\end{itemize}
Que tem propriedades de licor; que é espirituoso e aromático: \textunderscore vinho licoroso\textunderscore . Cf. \textunderscore Techn. Rur.\textunderscore , 205 e 243.
\section{Licranço}
\begin{itemize}
\item {Grp. gram.:m.}
\end{itemize}
Pequeno reptil, um pouco semelhante á vibora, mas sem a cabeça chata, (\textunderscore amphisbaena cinerea\textunderscore , Vandelli)--Os diccionaristas, incluindo o próprio Moraes, definem erradamente \textunderscore licranço\textunderscore , que não é \textunderscore lacrau\textunderscore , nem é venenoso, nem é privado de olhos, ainda que os tem pequenissimos.
\section{Lictor}
\begin{itemize}
\item {Grp. gram.:m.}
\end{itemize}
\begin{itemize}
\item {Proveniência:(Lat. \textunderscore lictor\textunderscore )}
\end{itemize}
Antigo official que, entre os Romanos, ia adeante dos cônsules ou do ditador, levando uma machadinha sôbre um feixe do varas.
\section{Lictório}
\begin{itemize}
\item {Grp. gram.:adj.}
\end{itemize}
\begin{itemize}
\item {Proveniência:(Lat. \textunderscore lictorius\textunderscore )}
\end{itemize}
Relativo ao lictor.
Próprio do lictor.
\section{Lida}
\begin{itemize}
\item {Grp. gram.:f.}
\end{itemize}
Acto ou effeito de lidar.
Azáfama; faina; trabalho.
\section{Lidador}
\begin{itemize}
\item {Grp. gram.:m.  e  adj.}
\end{itemize}
O que lida.
Batalhador.
\section{Lidage}
\begin{itemize}
\item {Grp. gram.:f.}
\end{itemize}
\begin{itemize}
\item {Utilização:Prov.}
\end{itemize}
\begin{itemize}
\item {Utilização:trasm.}
\end{itemize}
\begin{itemize}
\item {Proveniência:(De \textunderscore lidar\textunderscore )}
\end{itemize}
O mesmo que \textunderscore lida\textunderscore .
\section{Lidar}
\begin{itemize}
\item {Grp. gram.:v. i.}
\end{itemize}
\begin{itemize}
\item {Grp. gram.:V. t.}
\end{itemize}
\begin{itemize}
\item {Proveniência:(De \textunderscore lide\textunderscore )}
\end{itemize}
Lutar; combater.
Trabalhar.
Afadigar-se.
Reagir.
Combater com.
Provocar para torneio.
Correr ou farpear (toiros).
\section{Liddyte}
\begin{itemize}
\item {Grp. gram.:f.}
\end{itemize}
Substância explosiva, de effeito análogo ao da melinite, e de que recentemente têm feito uso os Ingleses para carregar granadas.
(Do ingl.)
\section{Lide}
\begin{itemize}
\item {Grp. gram.:f.}
\end{itemize}
\begin{itemize}
\item {Proveniência:(Do lat. \textunderscore lis\textunderscore , \textunderscore litis\textunderscore )}
\end{itemize}
Contenda; luta; combate.
Lida.
Questão judicial.
Questão.
Toireio.
\section{Lideira}
\begin{itemize}
\item {Grp. gram.:f.}
\end{itemize}
\begin{itemize}
\item {Utilização:Prov.}
\end{itemize}
\begin{itemize}
\item {Utilização:trasm.}
\end{itemize}
O mesmo que \textunderscore lida\textunderscore .
\section{Lidiador}
\begin{itemize}
\item {Grp. gram.:adj.}
\end{itemize}
\begin{itemize}
\item {Utilização:Des.}
\end{itemize}
Que lida, que briga.
Que representa, travando lide:«\textunderscore as legiões... co'a cruz... as lidiadoras águias órnão de Rómulo\textunderscore ». Filinto, XVI, 336.
(Por \textunderscore lideador\textunderscore , de \textunderscore lide\textunderscore )
\section{Lidimamente}
\begin{itemize}
\item {Grp. gram.:adv.}
\end{itemize}
\begin{itemize}
\item {Proveniência:(De \textunderscore lídimo\textunderscore )}
\end{itemize}
O mesmo que \textunderscore legitimamente\textunderscore .
\section{Lídimo}
\begin{itemize}
\item {Grp. gram.:adj.}
\end{itemize}
O mesmo que \textunderscore legítimo\textunderscore ; authêntico; puro vernáculo: \textunderscore português lídimo\textunderscore .
\section{Lidite}
\begin{itemize}
\item {Grp. gram.:f.}
\end{itemize}
Substância explosiva, de efeito análogo ao da melinite, e de que recentemente têm feito uso os Ingleses para carregar granadas.
(Do ingl.)
\section{Lidmeia}
\begin{itemize}
\item {Grp. gram.:f.}
\end{itemize}
Espécie de antílope africano.
\section{Lido}
\begin{itemize}
\item {Grp. gram.:adj.}
\end{itemize}
\begin{itemize}
\item {Proveniência:(De \textunderscore ler\textunderscore )}
\end{itemize}
Que tem conhecimentos obtidos pela leitura; sabedor: \textunderscore é homem muito lido\textunderscore .
\section{Lido}
\begin{itemize}
\item {Grp. gram.:m.}
\end{itemize}
Espécie de colono ou servo, de categoria superior, nas tríbos germânicas da Idade-Média. Cf. Herculano, \textunderscore Hist. de Port.\textunderscore , II, 249.
(B. lat. \textunderscore liti\textunderscore )
\section{Lidroso}
\begin{itemize}
\item {Grp. gram.:adj.}
\end{itemize}
Sujo, (falando-se da lan que reveste os testículos do carneiro).
(Corr. de \textunderscore ludroso\textunderscore )
\section{Lienal}
\begin{itemize}
\item {Grp. gram.:adj.}
\end{itemize}
\begin{itemize}
\item {Proveniência:(Do lat. \textunderscore lien\textunderscore , baço)}
\end{itemize}
Relativo ao baço.
\section{Lienite}
\begin{itemize}
\item {Grp. gram.:adj.}
\end{itemize}
\begin{itemize}
\item {Proveniência:(Do lat. \textunderscore lien\textunderscore )}
\end{itemize}
Inflammação do baço.
\section{Lienteria}
\begin{itemize}
\item {Grp. gram.:f.}
\end{itemize}
\begin{itemize}
\item {Proveniência:(Do gr. \textunderscore leinteria\textunderscore )}
\end{itemize}
Soltura ou diarreia, em que os alimentos são expellidos antes da completa digestão.
\section{Lientérico}
\begin{itemize}
\item {Grp. gram.:adj.}
\end{itemize}
\begin{itemize}
\item {Grp. gram.:M.}
\end{itemize}
Relativo á lienteria.
Que padece lienteria.
Aquello que soffre lienteria.
\section{Lierne}
\begin{itemize}
\item {Grp. gram.:m.}
\end{itemize}
\begin{itemize}
\item {Proveniência:(Fr. \textunderscore lierne\textunderscore )}
\end{itemize}
Nervura nas abóbadas góticas ou ogivaes, em fórma de cruz.
\section{Liga}
\begin{itemize}
\item {Grp. gram.:f.}
\end{itemize}
\begin{itemize}
\item {Utilização:Prov.}
\end{itemize}
\begin{itemize}
\item {Utilização:alent.}
\end{itemize}
Acto ou effeito de ligar.
Alliança; união; pacto.
Intimidade.
Combinação chimica de dois ou mais metaes.
Mistura.
Espécie de trança ou tira, feita com duas agulhas de meia.
Fita estreita, com que se cinge a meia á perna.
O mesmo que \textunderscore saburra\textunderscore .
\section{Liga}
\begin{itemize}
\item {Grp. gram.:f.}
\end{itemize}
\begin{itemize}
\item {Utilização:T. da Bairrada}
\end{itemize}
O mesmo que \textunderscore lia\textunderscore , depósito ou fezes do vinho.
\section{Ligá}
\begin{itemize}
\item {Grp. gram.:m.}
\end{itemize}
\begin{itemize}
\item {Utilização:Bras}
\end{itemize}
\begin{itemize}
\item {Proveniência:(De \textunderscore ligar\textunderscore ?)}
\end{itemize}
Coiro de boi, com que se resguardam da chuva as cargas dos animaes.
\section{Ligação}
\begin{itemize}
\item {Grp. gram.:f.}
\end{itemize}
\begin{itemize}
\item {Utilização:Mús.}
\end{itemize}
\begin{itemize}
\item {Grp. gram.:Pl.}
\end{itemize}
\begin{itemize}
\item {Proveniência:(Lat. \textunderscore ligatio\textunderscore )}
\end{itemize}
Acto ou effeito de ligar.
Coherência; connexão.
Execução de muitas notas com uma só arcada.
Curvas, traçadas e ligadas no papel, para exercício de quem apprende a escrever.
\section{Ligadura}
\begin{itemize}
\item {Grp. gram.:f.}
\end{itemize}
\begin{itemize}
\item {Proveniência:(Lat. \textunderscore ligatura\textunderscore )}
\end{itemize}
Acção de ligar; liga; ligação.
Atilho, atadura.
\section{Ligal}
\begin{itemize}
\item {Grp. gram.:m.}
\end{itemize}
\begin{itemize}
\item {Utilização:Bras}
\end{itemize}
Coiro de boí, com que se cobre a carga da bêsta.
O mesmo que \textunderscore ligá\textunderscore .
\section{Ligame}
\begin{itemize}
\item {Grp. gram.:m.}
\end{itemize}
\begin{itemize}
\item {Proveniência:(Lat. \textunderscore ligamen\textunderscore )}
\end{itemize}
Ligação; connexão; nexo.
Impedimento matrimonial.
\section{Ligâmen}
\begin{itemize}
\item {Grp. gram.:m.}
\end{itemize}
\begin{itemize}
\item {Proveniência:(Lat. \textunderscore ligamen\textunderscore )}
\end{itemize}
Ligação; connexão; nexo.
Impedimento matrimonial.
\section{Ligamento}
\begin{itemize}
\item {Grp. gram.:m.}
\end{itemize}
\begin{itemize}
\item {Proveniência:(Lat. \textunderscore ligamentum\textunderscore )}
\end{itemize}
Acto de ligar.
Liga; ligadura.
Parte fibrosa, que liga órgãos contíguos.
Parte que liga as duas válvulas da concha.
Barro ou outra substância, com que se ligam os materiaes de uma construcção.
\section{Ligamentoso}
\begin{itemize}
\item {Grp. gram.:adj.}
\end{itemize}
Análogo a ligamento; fibroso.
\section{Liga-osso}
\begin{itemize}
\item {Grp. gram.:m.}
\end{itemize}
Planta urticácea do Brasil.
\section{Ligar}
\begin{itemize}
\item {Grp. gram.:v. t.}
\end{itemize}
\begin{itemize}
\item {Grp. gram.:V. i.}
\end{itemize}
\begin{itemize}
\item {Proveniência:(Lat. \textunderscore ligare\textunderscore )}
\end{itemize}
Apertar com corda ou outro objecto flexível.
Dar nó em.
Enlaçar.
Tornar adherente.
Misturar; reunir.
Combinar chimicamente.
Juntar, unir.
Estabelecer connexão entre.
Encadear.
Juntar-se; misturar-se.
Fazer liga.
\section{Ligas-verdes}
\begin{itemize}
\item {Grp. gram.:f. pl.}
\end{itemize}
Espécie de bailado mirandês.
\section{Ligatura}
\begin{itemize}
\item {Grp. gram.:f.}
\end{itemize}
O mesmo que \textunderscore ligadura\textunderscore .
Conjunto de coisas ligadas. Cf. Garrett, \textunderscore Romanceiro\textunderscore , II, 156.
\section{Ligeira}
\begin{itemize}
\item {Grp. gram.:f.}
\end{itemize}
\begin{itemize}
\item {Utilização:Bras. do N}
\end{itemize}
\begin{itemize}
\item {Utilização:Prov.}
\end{itemize}
\begin{itemize}
\item {Utilização:dur.}
\end{itemize}
\begin{itemize}
\item {Utilização:minh.}
\end{itemize}
Espécie de chicote, com que os vaqueiros açoitam os cavallos.
Corda, com que se prende a um fueiro do carro o chifre do boi novo que se quere dirigir e amansar.
Corda, com que os pedreiros seguram os paus que sustentam o calabre de içar pedras.
\section{Ligeiramente}
\begin{itemize}
\item {Grp. gram.:adv.}
\end{itemize}
De modo ligeiro; á pressa.
De leve; suavemente.
\section{Ligeireza}
\begin{itemize}
\item {Grp. gram.:f.}
\end{itemize}
Qualidade daquelle ou daquillo que é ligeiro.
Brevidade; rapidez.
Superficialidade; leviandade.
\section{Ligeirias}
\begin{itemize}
\item {Grp. gram.:f. pl.}
\end{itemize}
\begin{itemize}
\item {Utilização:Ant.}
\end{itemize}
\begin{itemize}
\item {Proveniência:(De \textunderscore ligeiro\textunderscore )}
\end{itemize}
Chocarrices. Cf. Frei Fortun., \textunderscore Inéd.\textunderscore , 309.
\section{Ligeirice}
\begin{itemize}
\item {Grp. gram.:f.}
\end{itemize}
O mesmo que \textunderscore ligeireza\textunderscore . Cf. Camillo, \textunderscore Narcóticos\textunderscore , II, 15.
\section{Ligeiro}
\begin{itemize}
\item {Grp. gram.:adj.}
\end{itemize}
\begin{itemize}
\item {Grp. gram.:Adv.}
\end{itemize}
\begin{itemize}
\item {Grp. gram.:M.}
\end{itemize}
\begin{itemize}
\item {Utilização:Bras}
\end{itemize}
\begin{itemize}
\item {Utilização:Prov.}
\end{itemize}
\begin{itemize}
\item {Utilização:extrem.}
\end{itemize}
Leve.
Desembaraçado.
Rápido; que anda com rapidez: \textunderscore barco ligeiro\textunderscore .
Esperto.
Veloz; repentino.
Tênue; delgado: \textunderscore ligeiro fio de seda\textunderscore .
Transparente: \textunderscore tecido ligeiro\textunderscore .
Vago: \textunderscore ligeiras apparências\textunderscore .
Leviano.
Pouco importante: \textunderscore ligeiras noções de Poética\textunderscore .
Ligeiramente.
Remador de igarité.
O carneiro, que serve de guia a um rebanho por um atalho.
(Talvez do fr. \textunderscore léger\textunderscore )
\section{Ligeu}
\begin{itemize}
\item {Grp. gram.:m.}
\end{itemize}
\begin{itemize}
\item {Proveniência:(Do gr. \textunderscore lugaios\textunderscore )}
\end{itemize}
Gênero de insectos hemípteros.
\section{Lígio}
\begin{itemize}
\item {Grp. gram.:adj.}
\end{itemize}
Dizia-se do indivíduo, que estava addido ou ligado ao seu Príncipe, para em tudo o servir; e daquelle que, tendo recebido terras do soberano, ficava por isso mais obrigado a servi-lo, na paz e na guerra.
(B. lat. \textunderscore ligius\textunderscore )
\section{Lígneo}
\begin{itemize}
\item {Grp. gram.:adj.}
\end{itemize}
\begin{itemize}
\item {Proveniência:(Lat. \textunderscore ligneus\textunderscore )}
\end{itemize}
Lenhoso.
\section{Lignificar-se}
\begin{itemize}
\item {Grp. gram.:v. p.}
\end{itemize}
\begin{itemize}
\item {Proveniência:(Do lat. \textunderscore lignum\textunderscore  + \textunderscore facere\textunderscore )}
\end{itemize}
Formar lenha ou madeira, (falando-se dos vegetaes).
\section{Ligniforme}
\begin{itemize}
\item {Grp. gram.:adj.}
\end{itemize}
\begin{itemize}
\item {Proveniência:(Do lat. \textunderscore lignum\textunderscore  + \textunderscore forma\textunderscore )}
\end{itemize}
Que tem a natureza ou apparência de madeira.
\section{Lignita}
\begin{itemize}
\item {Grp. gram.:f.}
\end{itemize}
\begin{itemize}
\item {Proveniência:(Do lat. \textunderscore lignum\textunderscore )}
\end{itemize}
Carvão fóssil, matéria carbonosa, que conserva muitas vezes a fórma das plantas que lhe deram origem.
\section{Lignite}
\begin{itemize}
\item {Grp. gram.:f.}
\end{itemize}
\begin{itemize}
\item {Proveniência:(Do lat. \textunderscore lignum\textunderscore )}
\end{itemize}
Carvão fóssil, matéria carbonosa, que conserva muitas vezes a fórma das plantas que lhe deram origem.
\section{Lignito}
\begin{itemize}
\item {Grp. gram.:m.}
\end{itemize}
\begin{itemize}
\item {Proveniência:(Do lat. \textunderscore lignum\textunderscore )}
\end{itemize}
Carvão fóssil, matéria carbonosa, que conserva muitas vezes a fórma das plantas que lhe deram origem.
\section{Lignívoro}
\begin{itemize}
\item {Grp. gram.:adj.}
\end{itemize}
\begin{itemize}
\item {Grp. gram.:Pl.}
\end{itemize}
\begin{itemize}
\item {Proveniência:(Do lat. \textunderscore lignum\textunderscore  + \textunderscore vorare\textunderscore )}
\end{itemize}
Que rói madeira.
O mesmo que [[xylóphagos|xylóphago]].
\section{Lígula}
\begin{itemize}
\item {Grp. gram.:f.}
\end{itemize}
\begin{itemize}
\item {Proveniência:(Lat. \textunderscore ligula\textunderscore )}
\end{itemize}
Pequena lâmina vegetal na base das folhas das gramíneas.
Gênero de vermes intestinaes.
Lábio inferior dos insectos.
Gênero de molluscos.
Medida de capacidade para líquidos, entre os Romanos.
\section{Liguláceo}
\begin{itemize}
\item {Grp. gram.:adj.}
\end{itemize}
\begin{itemize}
\item {Utilização:Bot.}
\end{itemize}
Relativo ou semelhante a lígula.
\section{Ligulado}
\begin{itemize}
\item {Grp. gram.:adj.}
\end{itemize}
\begin{itemize}
\item {Utilização:Bot.}
\end{itemize}
Que tem lígulas.
\section{Ligúleas}
\begin{itemize}
\item {Grp. gram.:f. pl.}
\end{itemize}
\begin{itemize}
\item {Utilização:Bot.}
\end{itemize}
\begin{itemize}
\item {Proveniência:(De \textunderscore lígula\textunderscore )}
\end{itemize}
Secção das synanthéreas, no systema de Goertner.
\section{Ligulífero}
\begin{itemize}
\item {Grp. gram.:adj.}
\end{itemize}
\begin{itemize}
\item {Utilização:Bot.}
\end{itemize}
\begin{itemize}
\item {Proveniência:(Do lat. \textunderscore ligula\textunderscore  + \textunderscore ferre\textunderscore )}
\end{itemize}
O mesmo que \textunderscore ligulado\textunderscore .
\section{Ligulifloro}
\begin{itemize}
\item {Grp. gram.:adj.}
\end{itemize}
\begin{itemize}
\item {Utilização:Bot.}
\end{itemize}
\begin{itemize}
\item {Proveniência:(Do lat. \textunderscore ligula\textunderscore  + \textunderscore flos\textunderscore , \textunderscore floris\textunderscore )}
\end{itemize}
Que tem flôres liguladas.
\section{Liguliforme}
\begin{itemize}
\item {Grp. gram.:adj.}
\end{itemize}
\begin{itemize}
\item {Utilização:Bot.}
\end{itemize}
\begin{itemize}
\item {Proveniência:(De \textunderscore lígula\textunderscore  + \textunderscore fórma\textunderscore )}
\end{itemize}
Que tem fórma de lígula.
\section{Lígulo}
\begin{itemize}
\item {Grp. gram.:m.}
\end{itemize}
\begin{itemize}
\item {Utilização:Bot.}
\end{itemize}
Espécie de lingueta, que nasce do bôrdo livre da baínha da fôlha, nas gramíneas.
O mesmo que \textunderscore lígula\textunderscore .
\section{Liguloso}
\begin{itemize}
\item {Grp. gram.:adj.}
\end{itemize}
\begin{itemize}
\item {Utilização:Bot.}
\end{itemize}
O mesmo que \textunderscore ligulado\textunderscore .
\section{Lígures}
\begin{itemize}
\item {Grp. gram.:m. pl.}
\end{itemize}
\begin{itemize}
\item {Proveniência:(Lat. \textunderscore ligures\textunderscore )}
\end{itemize}
Habitantes da Ligúria.
\section{Ligúrico}
\begin{itemize}
\item {Grp. gram.:adj.}
\end{itemize}
Relativo ao Lígures ou á Ligúria.
\section{Ligurino}
\begin{itemize}
\item {Grp. gram.:adj.}
\end{itemize}
\begin{itemize}
\item {Grp. gram.:M.}
\end{itemize}
\begin{itemize}
\item {Proveniência:(Lat. \textunderscore ligurinus\textunderscore )}
\end{itemize}
Relativo á Ligúria.
Habitante da Ligúria.
\section{Ligústica}
\begin{itemize}
\item {Grp. gram.:m.}
\end{itemize}
Planta e fruto medicinal, (\textunderscore ligusticum levisticum\textunderscore , Lin.).
\section{Ligústico}
\begin{itemize}
\item {Grp. gram.:m.}
\end{itemize}
O mesmo que \textunderscore ligústica\textunderscore . Cf. \textunderscore Desengano da Med.\textunderscore , 48.
\section{Ligustrina}
\begin{itemize}
\item {Grp. gram.:f.}
\end{itemize}
Substância amarga, extrahida da casca do ligustro.
\section{Ligustro}
\begin{itemize}
\item {Grp. gram.:m.}
\end{itemize}
\begin{itemize}
\item {Proveniência:(Lat. \textunderscore ligustrum\textunderscore )}
\end{itemize}
O mesmo que \textunderscore alfena\textunderscore .
\section{Lijonja}
\begin{itemize}
\item {Grp. gram.:f.}
\end{itemize}
\begin{itemize}
\item {Utilização:Des.}
\end{itemize}
O mesmo que \textunderscore losango\textunderscore .
\section{Lila}
\begin{itemize}
\item {Grp. gram.:f.}
\end{itemize}
Espécie de tecido antigo, fabricado em Lille.
\section{Lilá}
\begin{itemize}
\item {Grp. gram.:m.}
\end{itemize}
\begin{itemize}
\item {Proveniência:(Do ár. \textunderscore lilac\textunderscore )}
\end{itemize}
Planta oleácea, (\textunderscore syringa\textunderscore ), vulgarmente conhecida pela fórma francesa \textunderscore lilás\textunderscore .
Côr arroxeada, semelhante á da flôr do lilá.
\section{Liláceas}
\begin{itemize}
\item {Grp. gram.:f. pl.}
\end{itemize}
\begin{itemize}
\item {Proveniência:(De \textunderscore liláceo\textunderscore )}
\end{itemize}
Família de plantas, que têm por typo o lilá.
\section{Liláceo}
\begin{itemize}
\item {Grp. gram.:adj.}
\end{itemize}
Semelhante ou relativo a lilá.
\section{Lilacina}
\begin{itemize}
\item {Grp. gram.:f.}
\end{itemize}
Corpo crystallizável e amargo, que se tira dos frutos verdes e das fôlhas de lilá.
\section{Lilailas}
\begin{itemize}
\item {Grp. gram.:f. pl.}
\end{itemize}
\begin{itemize}
\item {Utilização:Prov.}
\end{itemize}
\begin{itemize}
\item {Utilização:trasm.}
\end{itemize}
O mesmo que [[tretas|treta]].
\section{Lilaileiro}
\begin{itemize}
\item {Grp. gram.:m.}
\end{itemize}
\begin{itemize}
\item {Utilização:Prov.}
\end{itemize}
\begin{itemize}
\item {Utilização:trasm.}
\end{itemize}
Aquelle que anda sempre com lilailas.
\section{Lilás}
\begin{itemize}
\item {Grp. gram.:m.}
\end{itemize}
(V.lilá)
\section{Liliáceas}
\begin{itemize}
\item {Grp. gram.:f. pl.}
\end{itemize}
\begin{itemize}
\item {Proveniência:(De \textunderscore liliáceo\textunderscore )}
\end{itemize}
Famílias de plantas, que tem por typo o lírio.
\section{Liliáceo}
\begin{itemize}
\item {Grp. gram.:adj.}
\end{itemize}
\begin{itemize}
\item {Proveniência:(Do lat. \textunderscore lilium\textunderscore )}
\end{itemize}
Que é relativo ou semelhante ao lírio.
\section{Lilial}
\begin{itemize}
\item {Grp. gram.:adj.}
\end{itemize}
Relativo ao lílio; próprio do lílio.
\section{Lilifloro}
\begin{itemize}
\item {Grp. gram.:adj.}
\end{itemize}
\begin{itemize}
\item {Proveniência:(Do lat. \textunderscore lilium\textunderscore  + \textunderscore flos\textunderscore , \textunderscore floris\textunderscore )}
\end{itemize}
Que tem fôlhas semelhantes ás do lírio.
\section{Liliforme}
\begin{itemize}
\item {Grp. gram.:adj.}
\end{itemize}
\begin{itemize}
\item {Proveniência:(Do lat. \textunderscore lilium\textunderscore  + \textunderscore forma\textunderscore )}
\end{itemize}
Que tem fórma de um lírio.
\section{Lilíneas}
\begin{itemize}
\item {Grp. gram.:f. pl.}
\end{itemize}
(V.liliáceas)
\section{Lilinete}
\begin{itemize}
\item {fónica:nê}
\end{itemize}
\begin{itemize}
\item {Grp. gram.:m.}
\end{itemize}
Tecido, espécie de lila, mas menos encorpado.
\section{Lílio}
\begin{itemize}
\item {Grp. gram.:m.}
\end{itemize}
\begin{itemize}
\item {Utilização:Des.}
\end{itemize}
\begin{itemize}
\item {Proveniência:(Lat. \textunderscore lilium\textunderscore )}
\end{itemize}
O mesmo que \textunderscore lírio\textunderscore . Cf. Pant. de Aveiro, \textunderscore Itiner.\textunderscore , 293, (2.^a ed.).
\section{Liliputiano}
\begin{itemize}
\item {Grp. gram.:m.  e  adj.}
\end{itemize}
\begin{itemize}
\item {Utilização:Deprec.}
\end{itemize}
\begin{itemize}
\item {Proveniência:(De \textunderscore Liliput\textunderscore , n. p.)}
\end{itemize}
Homem pequeno.
Insignificante:«\textunderscore os nossos literatinhos liliputianos...\textunderscore »Castilho, \textunderscore Mont'Alverne\textunderscore .
\section{Lilla}
\begin{itemize}
\item {Grp. gram.:f.}
\end{itemize}
Espécie de tecido antigo, fabricado em Lille.
\section{Lillinete}
\begin{itemize}
\item {Grp. gram.:m.}
\end{itemize}
Tecido, espécie de lilla, mas menos encorpado.
\section{Lima}
\begin{itemize}
\item {Grp. gram.:f.}
\end{itemize}
\begin{itemize}
\item {Utilização:Fig.}
\end{itemize}
\begin{itemize}
\item {Utilização:Zool.}
\end{itemize}
\begin{itemize}
\item {Proveniência:(Lat. \textunderscore lima\textunderscore )}
\end{itemize}
Instrumento de metal, com asperezas ou picados, para desbastar ou raspar metaes ou pedra.
Instrumento de escultor:«\textunderscore a lima de Fídias...\textunderscore »Vieira.
\textunderscore Lima direita\textunderscore , a que tem faces planas, adelgaçando para a ponta.
\textunderscore Lima parallela\textunderscore , a que tem igual largura em todo o comprimento.
\textunderscore Lima de meia cana\textunderscore , a que tem uma das faces convexa.
\textunderscore Lima redonda\textunderscore , a que tem fórma cónica.
\textunderscore Lima triangular\textunderscore , a que tem três quinas e serve especialmente para afiar serras.
Aquillo que serve para polir ou aperfeiçoar.
Aperfeiçoamento: \textunderscore está dando a última lima aos seus versos\textunderscore .
Retoque.
Aquillo que corrói ou gasta: \textunderscore a lima do tempo\textunderscore .
Mollúsco lamellibrânchio, que se encontra em todos os mares.
\section{Lima}
\begin{itemize}
\item {Grp. gram.:f.}
\end{itemize}
Fruto da limeira; limeira.
(Ár. \textunderscore lima\textunderscore )
\section{Lima}
\begin{itemize}
\item {Grp. gram.:f.}
\end{itemize}
Peixe de Portugal.
\section{Lima}
\begin{itemize}
\item {Utilização:ant.}
\end{itemize}
\begin{itemize}
\item {Utilização:Gír.}
\end{itemize}
O mesmo que \textunderscore limosa\textunderscore .
\section{Lima}
\begin{itemize}
\item {Grp. gram.:f.}
\end{itemize}
Acto de \textunderscore limar\textunderscore ^3.
Qualidade da água que se emprega em limar^3.
\section{Limação}
\begin{itemize}
\item {Grp. gram.:f.}
\end{itemize}
\begin{itemize}
\item {Proveniência:(Lat. \textunderscore limatio\textunderscore )}
\end{itemize}
O mesmo que \textunderscore limadura\textunderscore .
\section{Limacídeo}
\begin{itemize}
\item {Grp. gram.:adj.}
\end{itemize}
\begin{itemize}
\item {Grp. gram.:M. pl.}
\end{itemize}
\begin{itemize}
\item {Proveniência:(Do lat. \textunderscore limax\textunderscore  + gr. \textunderscore eidos\textunderscore )}
\end{itemize}
Relativo ou semelhante á lesma.
Família de molluscos gasterópodes, que tem por typo a lesma.
\section{Limadamente}
\begin{itemize}
\item {Grp. gram.:adv.}
\end{itemize}
\begin{itemize}
\item {Proveniência:(De \textunderscore limar\textunderscore ^1)}
\end{itemize}
Polidamente; com correcção.
\section{Limadeira}
\begin{itemize}
\item {Grp. gram.:f.}
\end{itemize}
Mollusco acéphalo.
\section{Lima-de-umbigo}
\begin{itemize}
\item {Grp. gram.:f.}
\end{itemize}
\begin{itemize}
\item {Utilização:Bras. do N}
\end{itemize}
Variedade de lima^2.
\section{Limador}
\begin{itemize}
\item {Grp. gram.:adj.}
\end{itemize}
\begin{itemize}
\item {Grp. gram.:M.}
\end{itemize}
Que lima.
Que pule.
Que aperfeiçôa.
Aquelle que lima ou aperfeiçôa.
\section{Limadura}
\begin{itemize}
\item {Grp. gram.:f.}
\end{itemize}
\begin{itemize}
\item {Utilização:Fig.}
\end{itemize}
\begin{itemize}
\item {Proveniência:(Do lat. \textunderscore limatura\textunderscore )}
\end{itemize}
Acto ou effeito de limar^1.
Aperfeiçoamento.
\section{Limagem}
\begin{itemize}
\item {Grp. gram.:f.}
\end{itemize}
O mesmo que \textunderscore limadura\textunderscore .
Tempo que se gasta em limar^1.
\section{Limalha}
\begin{itemize}
\item {Grp. gram.:f.}
\end{itemize}
\begin{itemize}
\item {Proveniência:(De \textunderscore limar\textunderscore ^1)}
\end{itemize}
Partículas, que se separam de um corpo que se lima.
Metal pulverizado por meio da limagem.
\section{Limão}
\begin{itemize}
\item {Grp. gram.:m.}
\end{itemize}
\begin{itemize}
\item {Proveniência:(Do ár. \textunderscore leimon\textunderscore )}
\end{itemize}
Fruto do limoeiro.
Variedade de maçan.
\section{Limão}
\begin{itemize}
\item {Grp. gram.:m.}
\end{itemize}
\begin{itemize}
\item {Utilização:Prov.}
\end{itemize}
Cada uma das peças lateraes de um carro, nas quaes se encaixam os fueiros.
\section{Limão-doce}
\begin{itemize}
\item {Grp. gram.:m.}
\end{itemize}
\begin{itemize}
\item {Utilização:Bras. de Piauí}
\end{itemize}
O mesmo que \textunderscore lima\textunderscore ^2.
\section{Limãozinho}
\begin{itemize}
\item {Grp. gram.:m.}
\end{itemize}
Nome de dois arbustos brasileiros.
\section{Limar}
\begin{itemize}
\item {Grp. gram.:v. t.}
\end{itemize}
\begin{itemize}
\item {Proveniência:(Lat. \textunderscore limare\textunderscore )}
\end{itemize}
Desgastar ou polir com lima.
Polir, aperfeiçoar, (no sentido próp. e fig.).
\section{Limar}
\begin{itemize}
\item {Grp. gram.:v. t.}
\end{itemize}
\begin{itemize}
\item {Proveniência:(De \textunderscore limão\textunderscore )}
\end{itemize}
Temperar com azeite e limão.
\section{Limar}
\begin{itemize}
\item {Grp. gram.:v. i.}
\end{itemize}
\begin{itemize}
\item {Proveniência:(De \textunderscore limo\textunderscore )}
\end{itemize}
Diz-se da água, que corre sem interrupção pelos lameiros, ao contrário da água de rega.
\section{Limatão}
\begin{itemize}
\item {Grp. gram.:m.}
\end{itemize}
Lima grande, quadrada ou redonda.
Haste pyramidal de aço, de secção quadrada ou circular e superfície áspera, destinada ao alargamento de furos, em Mecânica.
(Cast. \textunderscore limatón\textunderscore )
\section{Limbera}
\begin{itemize}
\item {Grp. gram.:f.}
\end{itemize}
Árvore de Damão.
\section{Límbia}
\begin{itemize}
\item {Grp. gram.:adj. f.}
\end{itemize}
\begin{itemize}
\item {Utilização:Prov.}
\end{itemize}
\begin{itemize}
\item {Utilização:trasm.}
\end{itemize}
Que está no limbo, (falando-se da péla, no jôgo).
\section{Límbico}
\begin{itemize}
\item {Grp. gram.:adj.}
\end{itemize}
Relativo ao limbo.
\section{Limbífero}
\begin{itemize}
\item {Grp. gram.:adj.}
\end{itemize}
\begin{itemize}
\item {Proveniência:(Do lat. \textunderscore limbus\textunderscore  + \textunderscore ferre\textunderscore )}
\end{itemize}
Que tem limbo ou rebôrdo colorido.
\section{Limbifloras}
\begin{itemize}
\item {Grp. gram.:f. pl.}
\end{itemize}
\begin{itemize}
\item {Proveniência:(De \textunderscore limbo\textunderscore  + \textunderscore flôr\textunderscore )}
\end{itemize}
Ordem de plantas, que abrange as primuláceas e as gencianáceas.
\section{Limbo}
\begin{itemize}
\item {Grp. gram.:m.}
\end{itemize}
\begin{itemize}
\item {Utilização:Prov.}
\end{itemize}
\begin{itemize}
\item {Utilização:trasm.}
\end{itemize}
\begin{itemize}
\item {Proveniência:(Lat. \textunderscore limbus\textunderscore )}
\end{itemize}
Extremidade; fímbria; rebôrdo.
Parte livre das sépalas e pétalas.
Lugar, em que, segundo a crença christan, estão as almas das crianças que morrem sem baptismo, e em que estavam as almas dos justos, fallecidos antes da vinda de Christo.
Risca que, ao jôgo da péla, se faz numa parede, e na qual se não perde nem se ganha, se lhe bate a pela.
\section{Limbumbo}
\begin{itemize}
\item {Grp. gram.:m.}
\end{itemize}
Pequeno peixe africano.
\section{Limeira}
\begin{itemize}
\item {Grp. gram.:f.}
\end{itemize}
\begin{itemize}
\item {Proveniência:(De \textunderscore lima\textunderscore ^2)}
\end{itemize}
Árvore auranciácea, (\textunderscore citrus limetta aurária\textunderscore ).
Planta rutácea, (\textunderscore citrus dulcis\textunderscore ).
\section{Limenarca}
\begin{itemize}
\item {Grp. gram.:m.}
\end{itemize}
\begin{itemize}
\item {Proveniência:(Lat. \textunderscore limenarcha\textunderscore )}
\end{itemize}
Governador de um pôrto, na antiga Grécia.
Entre os Romanos, commandante das tropas, que guardavam a fronteira.
\section{Limenarcha}
\begin{itemize}
\item {Grp. gram.:m.}
\end{itemize}
\begin{itemize}
\item {Proveniência:(Lat. \textunderscore limenarcha\textunderscore )}
\end{itemize}
Governador de um pôrto, na antiga Grécia.
Entre os Romanos, commandante das tropas, que guardavam a fronteira.
\section{Limenho}
\begin{itemize}
\item {Grp. gram.:adj.}
\end{itemize}
\begin{itemize}
\item {Grp. gram.:M.}
\end{itemize}
Relativo a Lima, cidade americana.
Aquelle que é natural de Lima.
\section{Limento}
\begin{itemize}
\item {Grp. gram.:m.}
\end{itemize}
\begin{itemize}
\item {Proveniência:(De \textunderscore limo\textunderscore )}
\end{itemize}
Peixe, com a fórma e a côr da taínha.
\section{Limexilo}
\begin{itemize}
\item {fónica:csi}
\end{itemize}
\begin{itemize}
\item {Grp. gram.:m.}
\end{itemize}
\begin{itemize}
\item {Proveniência:(Do gr. \textunderscore lume\textunderscore  + \textunderscore xulon\textunderscore )}
\end{itemize}
Gênero de insectos coleópteros pentâmeros.
\section{Limiar}
\begin{itemize}
\item {Grp. gram.:m.}
\end{itemize}
\begin{itemize}
\item {Proveniência:(Lat. \textunderscore liminaris\textunderscore )}
\end{itemize}
Pedra ou peça de madeira, que, collocada transversalmente, constitue a parte superior ou inferior de uma porta ou portal.
Soleira; patamar junto á porta.
Portal; entrada.
\section{Limiforme}
\begin{itemize}
\item {Grp. gram.:adj.}
\end{itemize}
\begin{itemize}
\item {Proveniência:(De \textunderscore lima\textunderscore ^1 + \textunderscore fórma\textunderscore )}
\end{itemize}
Áspero como a lima.
\section{Liminar}
\begin{itemize}
\item {Grp. gram.:m.}
\end{itemize}
\begin{itemize}
\item {Utilização:Des.}
\end{itemize}
O mesmo que \textunderscore limiar\textunderscore .
\section{Liminarca}
\begin{itemize}
\item {Grp. gram.:m.}
\end{itemize}
(V.limenarca)
\section{Limitação}
\begin{itemize}
\item {Grp. gram.:f.}
\end{itemize}
\begin{itemize}
\item {Proveniência:(Lat. \textunderscore limitatio\textunderscore )}
\end{itemize}
Acto ou effeito de limitar.
Restricção; modificação.
\section{Limitadamente}
\begin{itemize}
\item {Grp. gram.:adv.}
\end{itemize}
De modo limitado; com restricção.
\section{Limitar}
\begin{itemize}
\item {Grp. gram.:v. t.}
\end{itemize}
\begin{itemize}
\item {Grp. gram.:V. i.}
\end{itemize}
\begin{itemize}
\item {Proveniência:(Lat. \textunderscore limitare\textunderscore )}
\end{itemize}
Pôr limite a.
Demarcar.
Restringir.
Marcar, estipular.
Confinar.
\section{Limitativamente}
\begin{itemize}
\item {Grp. gram.:adv.}
\end{itemize}
De modo limitativo; exclusivamente.
\section{Limitativo}
\begin{itemize}
\item {Grp. gram.:adj.}
\end{itemize}
\begin{itemize}
\item {Proveniência:(De \textunderscore limitar\textunderscore )}
\end{itemize}
Que serve de limite.
\section{Limite}
\begin{itemize}
\item {Grp. gram.:m.}
\end{itemize}
\begin{itemize}
\item {Proveniência:(Do lat. \textunderscore limes\textunderscore , \textunderscore limites\textunderscore )}
\end{itemize}
Linha, que estrema terrenos próximos ou contíguos.
Marco.
Fronteira.
Extremo; termo; confins; meta.
\section{Limítrofe}
\begin{itemize}
\item {Grp. gram.:adj.}
\end{itemize}
\begin{itemize}
\item {Proveniência:(Lat. \textunderscore limitrophus\textunderscore )}
\end{itemize}
Contíguo á fronteira de uma região.
Confinante.
\section{Limítrophe}
\begin{itemize}
\item {Grp. gram.:adj.}
\end{itemize}
\begin{itemize}
\item {Proveniência:(Lat. \textunderscore limitrophus\textunderscore )}
\end{itemize}
Contíguo á fronteira de uma região.
Confinante.
\section{Limnantáceas}
\begin{itemize}
\item {Grp. gram.:f. pl.}
\end{itemize}
O mesmo ou melhor que \textunderscore limnânteas\textunderscore .
\section{Limnânteas}
\begin{itemize}
\item {Grp. gram.:f. pl.}
\end{itemize}
Família de plantas, que tem por tipo o limnanto.
\section{Limnânteo}
\begin{itemize}
\item {Grp. gram.:adj.}
\end{itemize}
Relativo ou semelhante ao limnanto.
\section{Limnantháceas}
\begin{itemize}
\item {Grp. gram.:f. pl.}
\end{itemize}
O mesmo ou melhor que \textunderscore limnântheas\textunderscore .
\section{Limnântheas}
\begin{itemize}
\item {Grp. gram.:f. pl.}
\end{itemize}
Família de plantas, que tem por typo o limnantho.
\section{Limnântheo}
\begin{itemize}
\item {Grp. gram.:adj.}
\end{itemize}
Relativo ou semelhante ao limnantho.
\section{Limnantho}
\begin{itemize}
\item {Grp. gram.:m.}
\end{itemize}
\begin{itemize}
\item {Proveniência:(Do gr. \textunderscore limne\textunderscore  + \textunderscore anthos\textunderscore )}
\end{itemize}
Gênero de bellas plantas annuaes, que crescem naturalmente em terrenos húmidos ou pantanosos.
\section{Limnanto}
\begin{itemize}
\item {Grp. gram.:m.}
\end{itemize}
\begin{itemize}
\item {Proveniência:(Do gr. \textunderscore limne\textunderscore  + \textunderscore anthos\textunderscore )}
\end{itemize}
Gênero de bellas plantas annuaes, que crescem naturalmente em terrenos húmidos ou pantanosos.
\section{Limnar}
\begin{itemize}
\item {Grp. gram.:m.}
\end{itemize}
\begin{itemize}
\item {Utilização:Ant.}
\end{itemize}
(Contr. de \textunderscore liminar\textunderscore )
\section{Limnímetro}
\begin{itemize}
\item {Grp. gram.:m.}
\end{itemize}
\begin{itemize}
\item {Proveniência:(Do gr. \textunderscore limne\textunderscore  + \textunderscore metron\textunderscore )}
\end{itemize}
Instrumento, para medir o nível dos lagos.
\section{Limnita}
\begin{itemize}
\item {Grp. gram.:f.}
\end{itemize}
\begin{itemize}
\item {Proveniência:(Do gr. \textunderscore limne\textunderscore )}
\end{itemize}
Variedade de pedra, com uns veios ou traços, que lhe dão o aspecto de mappa.
\section{Limnófilo}
\begin{itemize}
\item {Grp. gram.:adj.}
\end{itemize}
\begin{itemize}
\item {Grp. gram.:M. pl.}
\end{itemize}
\begin{itemize}
\item {Proveniência:(Do gr. \textunderscore limne\textunderscore  + \textunderscore philos\textunderscore )}
\end{itemize}
Que habita nas águas.
Família de moluscos gasterópodes.
\section{Limnologia}
\begin{itemize}
\item {Grp. gram.:f.}
\end{itemize}
\begin{itemize}
\item {Proveniência:(Do gr. \textunderscore limne\textunderscore  + \textunderscore logos\textunderscore )}
\end{itemize}
Tratado, á cêrca dos lagos e águas estagnadas.
\section{Limnóphilo}
\begin{itemize}
\item {Grp. gram.:adj.}
\end{itemize}
\begin{itemize}
\item {Grp. gram.:M. pl.}
\end{itemize}
\begin{itemize}
\item {Proveniência:(Do gr. \textunderscore limne\textunderscore  + \textunderscore philos\textunderscore )}
\end{itemize}
Que habita nas águas.
Família de molluscos gasterópodes.
\section{Limo}
\begin{itemize}
\item {Grp. gram.:m.}
\end{itemize}
\begin{itemize}
\item {Utilização:Fig.}
\end{itemize}
\begin{itemize}
\item {Proveniência:(Lat. \textunderscore limus\textunderscore )}
\end{itemize}
Planta, da fam. das algas, (\textunderscore conferva rivularis\textunderscore ).
Lama; immundície.
Aquillo que é baixo, immundo.
\section{Limoada}
\begin{itemize}
\item {Grp. gram.:f.}
\end{itemize}
Pancada com limão.
Limonada.
\section{Limoado}
\begin{itemize}
\item {Grp. gram.:adj.}
\end{itemize}
Que tem a côr do limão, ou côr amarelo-clara.
\section{Limoal}
\begin{itemize}
\item {Grp. gram.:m.}
\end{itemize}
\begin{itemize}
\item {Proveniência:(De \textunderscore limão\textunderscore )}
\end{itemize}
Pomar de limoeiros.
\section{Limoctonia}
\begin{itemize}
\item {Grp. gram.:f.}
\end{itemize}
\begin{itemize}
\item {Proveniência:(Do gr. \textunderscore limos\textunderscore  + \textunderscore ktonos\textunderscore )}
\end{itemize}
Morte por falta de alimento.
\section{Limo-de-manta}
\begin{itemize}
\item {Grp. gram.:m.}
\end{itemize}
Uma das espécies de limo, que apparecem nas salinas.
\section{Limoeiro}
\begin{itemize}
\item {Grp. gram.:m.}
\end{itemize}
\begin{itemize}
\item {Proveniência:(De \textunderscore limão\textunderscore )}
\end{itemize}
Planta auranciácea, do gênero laranjeira.
Planta rutácea, (\textunderscore citrus medica\textunderscore ).
\section{Limões}
\begin{itemize}
\item {Grp. gram.:m. pl.}
\end{itemize}
\begin{itemize}
\item {Utilização:Prov.}
\end{itemize}
\begin{itemize}
\item {Utilização:alent.}
\end{itemize}
O mesmo que [[chedas|cheda]].
\section{Limo-letria}
\begin{itemize}
\item {Grp. gram.:m.}
\end{itemize}
\begin{itemize}
\item {Proveniência:(De \textunderscore limo\textunderscore  + \textunderscore aletria\textunderscore )}
\end{itemize}
Uma das espécies de limo, que apparecem nas salinas.
\section{Limonada}
\begin{itemize}
\item {Grp. gram.:f.}
\end{itemize}
\begin{itemize}
\item {Utilização:Ext.}
\end{itemize}
Bebida refrigerante, em que entra limão ou ácido cítrico.
Bebida refrigerante.
\section{Limonadeira}
\begin{itemize}
\item {Grp. gram.:f.}
\end{itemize}
Mulher, que vende limonada.
\section{Limonadeiro}
\begin{itemize}
\item {Grp. gram.:m.}
\end{itemize}
Fabricante ou vendedor de limonadas.
\section{Limonado}
\begin{itemize}
\item {Grp. gram.:adj.}
\end{itemize}
O mesmo que \textunderscore limoado\textunderscore .
\section{Limonete}
\begin{itemize}
\item {fónica:nê}
\end{itemize}
\begin{itemize}
\item {Grp. gram.:m.}
\end{itemize}
O mesmo que \textunderscore lúcia-lima\textunderscore .
\section{Limónia}
\begin{itemize}
\item {Grp. gram.:f.}
\end{itemize}
\begin{itemize}
\item {Proveniência:(Do gr. \textunderscore limonios\textunderscore )}
\end{itemize}
Gênero de insectos coleópteros pentâmeros.
\section{Limonina}
\begin{itemize}
\item {Grp. gram.:f.}
\end{itemize}
\begin{itemize}
\item {Utilização:Chím.}
\end{itemize}
Princípio amargo, descoberto por Bernays nas sementes do limão.
\section{Limonita}
\begin{itemize}
\item {Grp. gram.:f.}
\end{itemize}
\begin{itemize}
\item {Proveniência:(De \textunderscore limão\textunderscore )}
\end{itemize}
Óxydo de ferro hydratado, de côr castanha, amarelado.
\section{Limonite}
\begin{itemize}
\item {Grp. gram.:f.}
\end{itemize}
\begin{itemize}
\item {Proveniência:(De \textunderscore limão\textunderscore )}
\end{itemize}
Óxydo de ferro hydratado, de côr castanha, amarelado.
\section{Limosa}
\begin{itemize}
\item {Grp. gram.:f.}
\end{itemize}
\begin{itemize}
\item {Utilização:Gír.}
\end{itemize}
Camisa.
\section{Limosidade}
\begin{itemize}
\item {Grp. gram.:f.}
\end{itemize}
Qualidade de limoso.
Porção de limos.
\section{Limosina}
\begin{itemize}
\item {Grp. gram.:f.}
\end{itemize}
\begin{itemize}
\item {Proveniência:(Fr. \textunderscore limousine\textunderscore )}
\end{itemize}
Espécie de automóvel fechado, no gênero dos cupés, o com espelhos lateraes.
Capota de automóvel.
\section{Limosino}
\begin{itemize}
\item {Grp. gram.:adj.}
\end{itemize}
\begin{itemize}
\item {Grp. gram.:M.}
\end{itemize}
\begin{itemize}
\item {Proveniência:(Fr. \textunderscore limousin\textunderscore )}
\end{itemize}
Relativo a Limoges, em França.
Habitante de Limoges.
Dialecto dos Limosinos.
\section{Limoso}
\begin{itemize}
\item {Grp. gram.:adj.}
\end{itemize}
Que tem limos.
\section{Limote}
\begin{itemize}
\item {Grp. gram.:m.}
\end{itemize}
Lima de três quinas, representando um triângulo equilátero.
\section{Limpa}
\begin{itemize}
\item {Grp. gram.:f.}
\end{itemize}
\begin{itemize}
\item {Utilização:Bras. do N}
\end{itemize}
\begin{itemize}
\item {Proveniência:(De \textunderscore limpar\textunderscore )}
\end{itemize}
O mesmo que \textunderscore alimpa\textunderscore .
Parte da charneca, onde não cresce mato; clareira.
Acto de mondar ou cortar ervas damninhas em terreno cultivado.
\section{Limpa-botas}
\begin{itemize}
\item {Grp. gram.:m.}
\end{itemize}
\begin{itemize}
\item {Utilização:Fam.}
\end{itemize}
O mesmo que \textunderscore engraxador\textunderscore .
\section{Limpa-calhas}
\begin{itemize}
\item {Grp. gram.:m.}
\end{itemize}
Instrumento, com que se limpam as calhas ou carris, para a passagem de carros americanos ou de carros eléctricos.
Aquelle que trabalha com êsse instrumento.
\section{Limpa-candeeiros}
\begin{itemize}
\item {Grp. gram.:m.}
\end{itemize}
O mesmo que \textunderscore lampianista\textunderscore .
\section{Limpação}
\begin{itemize}
\item {Grp. gram.:f.}
\end{itemize}
O mesmo que \textunderscore limpadela\textunderscore .
\section{Limpa-chaminés}
\textunderscore m.\textunderscore 
Objecto, com que se limpam chaminés de fogões, de candeeiros, etc.
Indivíduo que limpa chaminés de cozinhas.
\section{Limpadeira}
\begin{itemize}
\item {Grp. gram.:f.}
\end{itemize}
Colhér muito estreita e de cabo comprido, com a qual se limpam os furos que a broca faz na pedra.
\section{Limpadela}
\begin{itemize}
\item {Grp. gram.:f.}
\end{itemize}
Acto ou effeito de limpar.
\section{Limpador}
\begin{itemize}
\item {Grp. gram.:adj.}
\end{itemize}
\begin{itemize}
\item {Grp. gram.:M.}
\end{itemize}
Que limpa.
Aquelle que limpa.
Máquina de joeirar trigo.
\section{Limpadura}
\begin{itemize}
\item {Grp. gram.:f.}
\end{itemize}
\begin{itemize}
\item {Grp. gram.:Pl.}
\end{itemize}
\begin{itemize}
\item {Utilização:Prov.}
\end{itemize}
\begin{itemize}
\item {Utilização:beir.}
\end{itemize}
O mesmo que \textunderscore limpadela\textunderscore .
Alimpadura.
O que sobeja da comida nos pratos.
O mesmo que [[rabeiras|rabeira]].
\section{Limpalho}
\begin{itemize}
\item {Grp. gram.:m. pl.}
\end{itemize}
\begin{itemize}
\item {Utilização:Agr.}
\end{itemize}
\begin{itemize}
\item {Proveniência:(De \textunderscore limpar\textunderscore )}
\end{itemize}
Restos de cereaes, que ficam no celleiro, depois de extrahidos os melhores bagos. Cf. \textunderscore Gazeta dos Lavr.\textunderscore , I, 17.
\section{Limpamente}
\begin{itemize}
\item {Grp. gram.:adv.}
\end{itemize}
\begin{itemize}
\item {Proveniência:(De \textunderscore limpo\textunderscore )}
\end{itemize}
Com limpeza.
\section{Limpamento}
\begin{itemize}
\item {Grp. gram.:m.}
\end{itemize}
O mesmo que \textunderscore limpeza\textunderscore .
\section{Limpante}
\begin{itemize}
\item {Grp. gram.:m.}
\end{itemize}
\begin{itemize}
\item {Utilização:T. de Villa-Viçosa}
\end{itemize}
Rodilho, para limpar talheres, loiça, etc.
\section{Limpa-queixos}
\begin{itemize}
\item {Grp. gram.:m.}
\end{itemize}
\begin{itemize}
\item {Utilização:Prov.}
\end{itemize}
\begin{itemize}
\item {Utilização:trasm.}
\end{itemize}
\begin{itemize}
\item {Utilização:fam.}
\end{itemize}
O mesmo que \textunderscore bofetada\textunderscore .
\section{Limpar}
\begin{itemize}
\item {Grp. gram.:v. i.}
\end{itemize}
\begin{itemize}
\item {Utilização:Prov.}
\end{itemize}
\begin{itemize}
\item {Utilização:trasm.}
\end{itemize}
\begin{itemize}
\item {Utilização:Bras. do N}
\end{itemize}
\begin{itemize}
\item {Grp. gram.:V. i.}
\end{itemize}
\begin{itemize}
\item {Proveniência:(Do lat. \textunderscore limpidare\textunderscore )}
\end{itemize}
Tornar limpo, puro.
Assear.
Varrer.
Enxugar.
Expungir.
Subtrahir.
Desbastar.
Arrotear.
Desramar.
Joeirar.
Tirar ou ganhar tudo a.
Dar segunda espadelada a (o linho).
Mondar; sachar.
Desanuvear-se, (falando-se do tempo).
Perder a lanugem, (falando-se dos frutos).
E diz-se das árvores, cujas flôres vão perdendo os verticillos exteriores.
\section{Limpa-trilho}
\begin{itemize}
\item {Grp. gram.:m.}
\end{itemize}
\begin{itemize}
\item {Utilização:Bras}
\end{itemize}
Construcção accessória, á frente das locomotivas, para desviar do trilho qualquer obstáculo.
\section{Limpa-vias}
\begin{itemize}
\item {Grp. gram.:m.}
\end{itemize}
O mesmo que \textunderscore limpa-calhas\textunderscore .
\section{Limpeza}
\begin{itemize}
\item {Grp. gram.:f.}
\end{itemize}
\begin{itemize}
\item {Utilização:Prov.}
\end{itemize}
\begin{itemize}
\item {Utilização:minh.}
\end{itemize}
Qualidade de limpo, de asseado, de puro.
Coisa limpa ou perfeita.
Perfeição.
Roupas de casa; bragal.
\section{Limpidez}
\begin{itemize}
\item {Grp. gram.:f.}
\end{itemize}
Qualidade daquillo que é limpido; transparência; nitidez.
\section{Límpido}
\begin{itemize}
\item {Grp. gram.:adj.}
\end{itemize}
\begin{itemize}
\item {Proveniência:(Lat. \textunderscore limpidus\textunderscore )}
\end{itemize}
Nítido; puro.
Transparente: \textunderscore água límpida\textunderscore .
Sereno; desanuveado: \textunderscore atmosphera límpida\textunderscore .
Limpo.
Viçoso.
Polido.
Ingênuo.
Sonoro.
\section{Límpio}
\begin{itemize}
\item {Grp. gram.:adj.}
\end{itemize}
\begin{itemize}
\item {Utilização:Ant.}
\end{itemize}
O mesmo que \textunderscore límpido\textunderscore .
\section{Limpíssimo}
\begin{itemize}
\item {Grp. gram.:adj.}
\end{itemize}
Muito limpo. Cf. \textunderscore Luz e Calor\textunderscore , 453.
\section{Limpo}
\begin{itemize}
\item {Grp. gram.:adj.}
\end{itemize}
\begin{itemize}
\item {Proveniência:(Do lat. \textunderscore limpidus\textunderscore )}
\end{itemize}
Que não tem impurezas ou manchas: \textunderscore fato limpo\textunderscore .
Nítido; puro.
Que não tem mistura.
Isento.
Aperfeiçoado; bem feito: \textunderscore trabalho limpo\textunderscore .
Claro, evidente.
\section{Limposo}
\begin{itemize}
\item {Grp. gram.:adj.}
\end{itemize}
\begin{itemize}
\item {Utilização:Neol.}
\end{itemize}
\begin{itemize}
\item {Proveniência:(De \textunderscore limpar\textunderscore )}
\end{itemize}
Que se occupa muito da limpeza e asseio.«\textunderscore ...limposos até a fúria do esmeril.\textunderscore »Ortigão, \textunderscore Hollanda\textunderscore , 97.
\section{Lináceas}
\begin{itemize}
\item {Grp. gram.:f. pl.}
\end{itemize}
\begin{itemize}
\item {Proveniência:(De \textunderscore lináceo\textunderscore )}
\end{itemize}
Família de plantas, que tem por typo o linho.
\section{Lináceo}
\begin{itemize}
\item {Grp. gram.:adj.}
\end{itemize}
\begin{itemize}
\item {Proveniência:(Do lat. \textunderscore linum\textunderscore )}
\end{itemize}
Relativo ou semelhante ao linho.
\section{Linagem}
\begin{itemize}
\item {Grp. gram.:f.}
\end{itemize}
\begin{itemize}
\item {Utilização:Ant.}
\end{itemize}
O mesmo que \textunderscore linhagem\textunderscore ^2.
\section{Linaloés}
\begin{itemize}
\item {Grp. gram.:m.}
\end{itemize}
Madeira aromática da Índia, (\textunderscore lignum, aloes\textunderscore ).
\section{Linária}
\begin{itemize}
\item {Grp. gram.:f.}
\end{itemize}
\begin{itemize}
\item {Proveniência:(Lat. \textunderscore linaria\textunderscore )}
\end{itemize}
Planta, o mesmo que \textunderscore valverde\textunderscore .
\section{Linarita}
\begin{itemize}
\item {Grp. gram.:f.}
\end{itemize}
\begin{itemize}
\item {Proveniência:(De \textunderscore Linares\textunderscore , n. p.)}
\end{itemize}
Sulfato de chumbo e cobre.
\section{Lincumba}
\begin{itemize}
\item {Grp. gram.:f.}
\end{itemize}
Pequeno peixe africano. Cf. Serpa Pinto, I, 298.
\section{Linda}
\begin{itemize}
\item {Grp. gram.:f.}
\end{itemize}
\begin{itemize}
\item {Proveniência:(De \textunderscore lindar\textunderscore )}
\end{itemize}
Estrema; raia, limite.
\section{Lindaço}
\begin{itemize}
\item {Grp. gram.:adj.}
\end{itemize}
\begin{itemize}
\item {Utilização:Bras. do S}
\end{itemize}
Muito lindo.
\section{Linda-flôr}
\begin{itemize}
\item {Grp. gram.:f.}
\end{itemize}
Planta brasileira, semelhante ao malmequer.
A flôr dessa planta.
\section{Lindamente}
\begin{itemize}
\item {Grp. gram.:adv.}
\end{itemize}
De modo lindo; excellentemente; perfeitamente: \textunderscore portou-se lindamente\textunderscore .
\section{Lindar}
\begin{itemize}
\item {Grp. gram.:v. t.}
\end{itemize}
\begin{itemize}
\item {Proveniência:(Do lat. \textunderscore limitare\textunderscore )}
\end{itemize}
Pôr lindas ou balisas em.
Demarcar, estremar.
\section{Linde}
\begin{itemize}
\item {Grp. gram.:m.}
\end{itemize}
\begin{itemize}
\item {Utilização:Ant.}
\end{itemize}
\begin{itemize}
\item {Proveniência:(Do lat. \textunderscore limis\textunderscore , \textunderscore limitis\textunderscore )}
\end{itemize}
O mesmo ou melhor que \textunderscore linda\textunderscore .
\section{Lindeira}
\begin{itemize}
\item {Grp. gram.:f.}
\end{itemize}
\begin{itemize}
\item {Proveniência:(De \textunderscore lindar\textunderscore )}
\end{itemize}
Ombreira da porta.
Vêrga superior da porta ou janela.
\section{Lindembérgia}
\begin{itemize}
\item {Grp. gram.:f.}
\end{itemize}
Gênero de plantas escrofularíneas.
\section{Lindérnia}
\begin{itemize}
\item {Grp. gram.:f.}
\end{itemize}
\begin{itemize}
\item {Proveniência:(De \textunderscore Lindern\textunderscore , n. p.)}
\end{itemize}
Gênero de plantas da Europa central.
\section{Lindeza}
\begin{itemize}
\item {Grp. gram.:f.}
\end{itemize}
Qualidade de lindo.
Belleza; formosura.
Aquelle ou aquillo que é lindo.
\section{Lindo}
\begin{itemize}
\item {Grp. gram.:adj.}
\end{itemize}
\begin{itemize}
\item {Proveniência:(Do lat. \textunderscore limpidus\textunderscore )}
\end{itemize}
O mesmo que \textunderscore bello\textunderscore ^1.
Agradável: \textunderscore tempo lindo\textunderscore .
Delicado; primoroso: \textunderscore procedimento lindo\textunderscore .
\section{Lindo-pardo}
\begin{itemize}
\item {Grp. gram.:m.}
\end{itemize}
Variedade de maçan pequena e pardacenta.
\section{Lindote}
\begin{itemize}
\item {Grp. gram.:adj.}
\end{itemize}
\begin{itemize}
\item {Utilização:Fam.}
\end{itemize}
Um tanto lindo.
\section{Líndsea}
\begin{itemize}
\item {Grp. gram.:f.}
\end{itemize}
Gênero de fêtos.
\section{Lindseíte}
\begin{itemize}
\item {Grp. gram.:f.}
\end{itemize}
\begin{itemize}
\item {Utilização:Miner.}
\end{itemize}
Mineral análogo ao ferro oxydado.
\section{Lineado}
\begin{itemize}
\item {Grp. gram.:adj.}
\end{itemize}
\begin{itemize}
\item {Utilização:Bot.}
\end{itemize}
\begin{itemize}
\item {Proveniência:(Do lat. \textunderscore linea\textunderscore )}
\end{itemize}
Diz-se das fôlhas, que apresentam linhas finas e parallelas, de côr differente da do fundo.
\section{Lineagem}
\begin{itemize}
\item {Grp. gram.:f.}
\end{itemize}
\begin{itemize}
\item {Utilização:Ant.}
\end{itemize}
O mesmo que \textunderscore linhagem\textunderscore ^1.
\section{Lineal}
\begin{itemize}
\item {Grp. gram.:adj.}
\end{itemize}
\begin{itemize}
\item {Proveniência:(Lat. \textunderscore linealis\textunderscore )}
\end{itemize}
O mesmo que \textunderscore linear\textunderscore .
\section{Lineamento}
\begin{itemize}
\item {Grp. gram.:m.}
\end{itemize}
\begin{itemize}
\item {Grp. gram.:Pl.}
\end{itemize}
\begin{itemize}
\item {Proveniência:(Lat. \textunderscore lineamentum\textunderscore )}
\end{itemize}
Traço, producção de uma linha.
Feições physionómicas.
Delineamento, esbôço.
Rudimentos.
\section{Linear}
\begin{itemize}
\item {Grp. gram.:adj.}
\end{itemize}
\begin{itemize}
\item {Proveniência:(Lat. \textunderscore linearis\textunderscore )}
\end{itemize}
Relativo a linhas: desenho linear.
Semelhante a uma linha: \textunderscore verme linear\textunderscore .
\section{Líneo}
\begin{itemize}
\item {Grp. gram.:adj.}
\end{itemize}
\begin{itemize}
\item {Utilização:Poét.}
\end{itemize}
\begin{itemize}
\item {Proveniência:(Lat. \textunderscore lineus\textunderscore )}
\end{itemize}
Relativo ao linho.
\section{Lineolar}
\begin{itemize}
\item {Grp. gram.:adj.}
\end{itemize}
\begin{itemize}
\item {Utilização:Bot.}
\end{itemize}
\begin{itemize}
\item {Proveniência:(Do lat. \textunderscore lineola\textunderscore )}
\end{itemize}
Diz-se dos órgãos vegetaes, em que se notam linhas, ou que têm a apparência de linha ou traço.
\section{Linete}
\begin{itemize}
\item {fónica:nê}
\end{itemize}
\begin{itemize}
\item {Grp. gram.:m.}
\end{itemize}
(V.lillinete)
\section{Linga}
\begin{itemize}
\item {Grp. gram.:f.}
\end{itemize}
\begin{itemize}
\item {Proveniência:(Do pers. \textunderscore lenguer\textunderscore ?)}
\end{itemize}
Cadeia de corda que, cingindo um fardo, se prende a uma roldana para o levantar.
\section{Linga}
\begin{itemize}
\item {Grp. gram.:m.}
\end{itemize}
Representação dos órgãos sexuaes do homem ou da mulher, sýmbolo do poder gerador, adorado na Índia.
(Do conc.)
\section{Lingada}
\begin{itemize}
\item {Grp. gram.:f.}
\end{itemize}
\begin{itemize}
\item {Proveniência:(De \textunderscore lingar\textunderscore )}
\end{itemize}
Objectos, que se lingam de uma vez.
\section{Lingam}
\begin{itemize}
\item {Grp. gram.:m.}
\end{itemize}
(V. \textunderscore linga\textunderscore ^2)
\section{Lingar}
\begin{itemize}
\item {Grp. gram.:v. t.}
\end{itemize}
Cingir de linga.
Levantar com linga.
\section{Língoa}
\begin{itemize}
\item {Grp. gram.:f.}
\end{itemize}
\begin{itemize}
\item {Grp. gram.:M.}
\end{itemize}
\begin{itemize}
\item {Utilização:Fam.}
\end{itemize}
\begin{itemize}
\item {Proveniência:(Lat. \textunderscore lingua\textunderscore )}
\end{itemize}
Principal órgão do sentido do gôsto, que concorre para a deglutição e para a fala e que é composto principalmente de um músculo revestido de uma membrana mucosa.
Linguagem, voz.
Idioma.
Complexo de regras de um idioma.
Nome de vários objectos, que têm semelhança ou analogia com o órgão da língua.
Tromba dos insectos lepidópteros.
Nome de um peixe, nas costas de Portugal, (\textunderscore synaptura lusitanica\textunderscore , Capello).
Elemento, que entra na formação do nome de várias plantas.
Intérprete.
\textunderscore Dar com a língoa nos dentes\textunderscore , revelar segrêdo.
\textunderscore Dar á língoa\textunderscore , tagarelar, sêr indiscreto.
\textunderscore As língoas do mundo\textunderscore , a maledicência.
\textunderscore Pagar pela língoa\textunderscore , soffrer os effeitos de actos ou acções, que levianamente se prepararam ou se louvaram.
\textunderscore Lingoa de trapos\textunderscore , pessôa, que fala confusamente.
\section{Lingote}
\begin{itemize}
\item {Grp. gram.:m.}
\end{itemize}
\begin{itemize}
\item {Proveniência:(Fr. \textunderscore lingot\textunderscore )}
\end{itemize}
Pequena barra de secção trapezoidal, que tem de comprimento 0^m,35 e de largura 0^m,75. Uma das duas fórmas, com que o estanho se apresenta no commércio, sendo a outra em lâminas.
\section{Lingoteira}
\begin{itemize}
\item {Grp. gram.:f.}
\end{itemize}
\begin{itemize}
\item {Proveniência:(Do fr. \textunderscore lingotière\textunderscore )}
\end{itemize}
Molde, para fazer barras de metal.
Molde, para se consolidarem e tomarem certa fórma alguns saes.
\section{Lingragem}
\begin{itemize}
\item {Grp. gram.:f.}
\end{itemize}
\begin{itemize}
\item {Utilização:ant.}
\end{itemize}
\begin{itemize}
\item {Utilização:Pleb.}
\end{itemize}
O mesmo que \textunderscore linguagem\textunderscore . Cf. Sim. Machado, f. 69.
\section{Língua}
\begin{itemize}
\item {Grp. gram.:f.}
\end{itemize}
\begin{itemize}
\item {Grp. gram.:M.}
\end{itemize}
\begin{itemize}
\item {Utilização:Fam.}
\end{itemize}
\begin{itemize}
\item {Proveniência:(Lat. \textunderscore lingua\textunderscore )}
\end{itemize}
Principal órgão do sentido do gôsto, que concorre para a deglutição e para a fala e que é composto principalmente de um músculo revestido de uma membrana mucosa.
Linguagem, voz.
Idioma.
Complexo de regras de um idioma.
Nome de vários objectos, que têm semelhança ou analogia com o órgão da língua.
Tromba dos insectos lepidópteros.
Nome de um peixe, nas costas de Portugal, (\textunderscore synaptura lusitanica\textunderscore , Capello).
Elemento, que entra na formação do nome de várias plantas.
Intérprete.
\textunderscore Dar com a língua nos dentes\textunderscore , revelar segrêdo.
\textunderscore Dar á língua\textunderscore , tagarelar, sêr indiscreto.
\textunderscore As línguas do mundo\textunderscore , a maledicência.
\textunderscore Pagar pela língua\textunderscore , soffrer os effeitos de actos ou acções, que levianamente se prepararam ou se louvaram.
\textunderscore Lingua de trapos\textunderscore , pessôa, que fala confusamente.
\section{Língua-de-boi}
\begin{itemize}
\item {Grp. gram.:f.}
\end{itemize}
O mesmo que \textunderscore ajuga\textunderscore .
\section{Língua-cervina}
\begin{itemize}
\item {Grp. gram.:f.}
\end{itemize}
O mesmo que \textunderscore escolopendra\textunderscore .
\section{Língua-de-gallinha}
\begin{itemize}
\item {Grp. gram.:f.}
\end{itemize}
\begin{itemize}
\item {Utilização:Bras}
\end{itemize}
Espécie de anileira.
\section{Língua-de-mulato}
\begin{itemize}
\item {Grp. gram.:f.}
\end{itemize}
\begin{itemize}
\item {Utilização:T. do Maranhão}
\end{itemize}
Fatia torrada de pão doce.
\section{Língua-de-onça}
\begin{itemize}
\item {Grp. gram.:f.}
\end{itemize}
Pequena e mimosa planta africana, de fôlhas radicaes, cordiformes, e flôres miúdas, em corymbos.
\section{Língua-de-ovelha}
\begin{itemize}
\item {Grp. gram.:f.}
\end{itemize}
Variedade de festuca, que se dá em terrenos pobres, mas estrumados, emquanto as outras variedades só se dão nos lameiros.
\section{Língua-de-serpente}
\begin{itemize}
\item {Grp. gram.:f.}
\end{itemize}
O mesmo que \textunderscore ophioglosso\textunderscore .
\section{Língua-de-serpentina}
\begin{itemize}
\item {Grp. gram.:f.}
\end{itemize}
Planta da serra de Sintra.
\section{Língua-de-vaca}
\begin{itemize}
\item {Grp. gram.:f.}
\end{itemize}
\begin{itemize}
\item {Utilização:Bras}
\end{itemize}
Peixe de Portugal.
Planta portulácea, também conhecida por \textunderscore maria-gomes\textunderscore .
Nome de outras diversas plantas.
\section{Linguado}
\begin{itemize}
\item {Grp. gram.:m.}
\end{itemize}
\begin{itemize}
\item {Utilização:Pop.}
\end{itemize}
\begin{itemize}
\item {Utilização:Gír.}
\end{itemize}
\begin{itemize}
\item {Utilização:Gír.}
\end{itemize}
\begin{itemize}
\item {Grp. gram.:Adj.}
\end{itemize}
\begin{itemize}
\item {Utilização:Heráld.}
\end{itemize}
\begin{itemize}
\item {Proveniência:(Do lat. \textunderscore lingulatus\textunderscore )}
\end{itemize}
Grande tira de papel, em que de ordinário se escreve o que se destina á imprensa.
Lâmina comprida.
Peixe pleuronecto, (\textunderscore pleuronectis solea\textunderscore ).
Barra de ferro fundido; gusa.
Grande língua.
Letra commercial.
Bolsa do dinheiro.
Diz-se do animal que, no campo do escudo, apresenta a língua tingida de esmalte.
\section{Lineano}
\begin{itemize}
\item {Grp. gram.:adj.}
\end{itemize}
Relativo a Lineu.
\section{Lineia}
\begin{itemize}
\item {Grp. gram.:f.}
\end{itemize}
\begin{itemize}
\item {Proveniência:(De \textunderscore Lineu\textunderscore , n. p.)}
\end{itemize}
Formosa planta ornamental, da fam. das caprifoliáceas.
\section{Linguagem}
\begin{itemize}
\item {Grp. gram.:f.}
\end{itemize}
\begin{itemize}
\item {Grp. gram.:Pl.}
\end{itemize}
\begin{itemize}
\item {Utilização:Gram.}
\end{itemize}
Emprego da língua, para a expressão dos pensamentos ou sentimentos.
Expressão dos pensamentos e sentimentos por palavras.
Qualquer systema de sinaes, empregados para a expressão do pensamento: \textunderscore linguagem mímica\textunderscore .
Idioma ou dialecto de uma nação ou região.
Tudo que exprime sensações ou ideias: \textunderscore a linguagem dos olhos\textunderscore .
Estilo.
Conjugações dos verbos.
\section{Linguajar}
\begin{itemize}
\item {Grp. gram.:v. i.}
\end{itemize}
\begin{itemize}
\item {Utilização:Neol.}
\end{itemize}
\begin{itemize}
\item {Proveniência:(De \textunderscore linguagem\textunderscore )}
\end{itemize}
Dar á lingua; falar. Cf. Pacheco da Silva, \textunderscore Promptuário\textunderscore , 22.
\section{Lingual}
\begin{itemize}
\item {Grp. gram.:adj.}
\end{itemize}
Relativo á língua.
\section{Linguana}
\begin{itemize}
\item {Grp. gram.:f.}
\end{itemize}
Planta leguminosa de Cabo-Verde.
\section{Linguarada}
\begin{itemize}
\item {Grp. gram.:f.}
\end{itemize}
Palavrão indecoroso ou atrevido.
Linguagem petulante e chula. Cf. Filinto, X, 135.
\section{Linguarado}
\begin{itemize}
\item {Grp. gram.:adj.}
\end{itemize}
\begin{itemize}
\item {Utilização:P. us.}
\end{itemize}
O mesmo que \textunderscore linguareiro\textunderscore .
\section{Linguarão}
\begin{itemize}
\item {Grp. gram.:m.}
\end{itemize}
O mesmo que \textunderscore linguareiro\textunderscore :«\textunderscore ...mais cautelosos e menos linguarões...\textunderscore »Corvo, \textunderscore Anno na Côrte\textunderscore , III, 39.
\section{Linguaraz}
\begin{itemize}
\item {Grp. gram.:m.  e  adj.}
\end{itemize}
\begin{itemize}
\item {Proveniência:(Do rad. de \textunderscore língua\textunderscore )}
\end{itemize}
Linguareiro; maldizente.
\section{Linguareiro}
\begin{itemize}
\item {Grp. gram.:m.  e  adj.}
\end{itemize}
\begin{itemize}
\item {Proveniência:(Do rad. de \textunderscore língua\textunderscore )}
\end{itemize}
Falador; chocalheiro.
\section{Linguarejar}
\begin{itemize}
\item {Grp. gram.:v. i.}
\end{itemize}
\begin{itemize}
\item {Utilização:Fam.}
\end{itemize}
Tagarelar; parolar; dar á língua.
\section{Linguarice}
\begin{itemize}
\item {Grp. gram.:f.}
\end{itemize}
Tagarelice.
\section{Linguário}
\begin{itemize}
\item {Grp. gram.:m.}
\end{itemize}
\begin{itemize}
\item {Proveniência:(Lat. \textunderscore linguarium\textunderscore )}
\end{itemize}
Multa ou castigo que, entre os antigos, se impunha aos que empregavam linguagem indecorosa ou effensiva.
\section{Linguarudo}
\begin{itemize}
\item {Grp. gram.:adj.}
\end{itemize}
\begin{itemize}
\item {Utilização:Pop.}
\end{itemize}
O mesmo que \textunderscore linguareiro\textunderscore .
\section{Linguás}
\begin{itemize}
\item {Grp. gram.:m. pl.}
\end{itemize}
\begin{itemize}
\item {Utilização:Bras}
\end{itemize}
Tríbo de aborigenes, que dominou em Mato-Grosso.
\section{Lingua-sirvina}
\begin{itemize}
\item {Grp. gram.:f.}
\end{itemize}
Planta da serra de Sintra.
\section{Lingueirão}
\begin{itemize}
\item {fónica:gu-ei}
\end{itemize}
\begin{itemize}
\item {Grp. gram.:m.}
\end{itemize}
Língua grande.
\section{Lingueirão}
\begin{itemize}
\item {fónica:gu-ei}
\end{itemize}
\begin{itemize}
\item {Grp. gram.:m.}
\end{itemize}
Mollúsco acéphalo, de concha bivalve.
Pequeno peixe marítimo.
(Corr. de \textunderscore longueirão\textunderscore . Cp. \textunderscore longueirão\textunderscore )
\section{Lingueirão-de-canudo}
\begin{itemize}
\item {Grp. gram.:m.}
\end{itemize}
Mollúsco, o mesmo que \textunderscore lingueirão\textunderscore ^2.
\section{Lingueta}
\begin{itemize}
\item {fónica:gu-ê}
\end{itemize}
\begin{itemize}
\item {Grp. gram.:f.}
\end{itemize}
Pequena língua.
Fiel da balança.
Parede entre duas chaminés.
Rampa de caes.
Lámina, que em certos instrumentos e máquinas é movida pelo ar ou pela água.
Parte móvel da fechadura, que a chave faz entrar na chapatesta.
Belho.
Appêndice da corolla de algumas plantas synanthéreas.
Ligadura ou compressa.
\section{Linguete}
\begin{itemize}
\item {fónica:gu-ê}
\end{itemize}
\begin{itemize}
\item {Grp. gram.:m.}
\end{itemize}
\begin{itemize}
\item {Proveniência:(De \textunderscore língua\textunderscore )}
\end{itemize}
Peça de ferro ou madeira, que se embebe nas rodas do cabrestante, para que êste não desande.
\section{Linguiça}
\begin{itemize}
\item {fónica:gu-i}
\end{itemize}
\begin{itemize}
\item {Grp. gram.:f.}
\end{itemize}
\begin{itemize}
\item {Utilização:Prov.}
\end{itemize}
\begin{itemize}
\item {Utilização:trasm.}
\end{itemize}
\begin{itemize}
\item {Proveniência:(Do lat. hyp. \textunderscore lucanicia\textunderscore , de \textunderscore lucanica\textunderscore , linguiça)}
\end{itemize}
Espécie de chouriço delgado.
O mesmo que \textunderscore murra\textunderscore ^1.
\section{Linguice}
\begin{itemize}
\item {fónica:gu-i}
\end{itemize}
\begin{itemize}
\item {Grp. gram.:f.}
\end{itemize}
\begin{itemize}
\item {Utilização:Ant.}
\end{itemize}
O mesmo que \textunderscore linguiça\textunderscore . Cf. \textunderscore Eufrosina\textunderscore , act. I, sc. 3.
\section{Linguifero}
\begin{itemize}
\item {fónica:gu-i}
\end{itemize}
\begin{itemize}
\item {Grp. gram.:adj.}
\end{itemize}
\begin{itemize}
\item {Proveniência:(Do lat. \textunderscore lingua\textunderscore  + \textunderscore ferre\textunderscore )}
\end{itemize}
Que tem língua, ou órgãos em fórma de língua.
\section{Linguiforme}
\begin{itemize}
\item {fónica:gu-i}
\end{itemize}
\begin{itemize}
\item {Grp. gram.:adj.}
\end{itemize}
\begin{itemize}
\item {Proveniência:(De \textunderscore língua\textunderscore  + \textunderscore fórma\textunderscore )}
\end{itemize}
Que tem fórma de língua.
\section{Linguista}
\begin{itemize}
\item {fónica:gu-i}
\end{itemize}
\begin{itemize}
\item {Grp. gram.:m.}
\end{itemize}
\begin{itemize}
\item {Proveniência:(De \textunderscore língua\textunderscore )}
\end{itemize}
Aquelle que é versado em linguística.
\section{Linguística}
\begin{itemize}
\item {fónica:gu-i}
\end{itemize}
\begin{itemize}
\item {Grp. gram.:f.}
\end{itemize}
\begin{itemize}
\item {Proveniência:(De \textunderscore linguístico\textunderscore )}
\end{itemize}
Estudo das línguas ou idiomas, nas suas relações e nos seus princípios.
Sciência dos factos da linguagem espontânea ou popular.
\section{Linguístico}
\begin{itemize}
\item {fónica:gu-i}
\end{itemize}
\begin{itemize}
\item {Grp. gram.:adj.}
\end{itemize}
\begin{itemize}
\item {Proveniência:(De \textunderscore linguista\textunderscore )}
\end{itemize}
Relativo a linguística.
\section{Língula}
\begin{itemize}
\item {Grp. gram.:f.}
\end{itemize}
\begin{itemize}
\item {Proveniência:(Lat. \textunderscore lingula\textunderscore  = \textunderscore ligula\textunderscore )}
\end{itemize}
Antiga espada romana, longa e estreita.
Espátula dos arúspices.
Gênero de molluscos, de concha bivalve.
\section{Lingulado}
\begin{itemize}
\item {Grp. gram.:adj.}
\end{itemize}
\begin{itemize}
\item {Proveniência:(De \textunderscore lingula\textunderscore )}
\end{itemize}
Que tem a fórma de uma pequena língua.
\section{Lingumoeno}
\begin{itemize}
\item {fónica:mo-e}
\end{itemize}
\begin{itemize}
\item {Grp. gram.:m.}
\end{itemize}
Pequeno peixe africano. Cf. Serpa Pinto, I, 298.
\section{Linguneta}
\begin{itemize}
\item {fónica:nê}
\end{itemize}
\begin{itemize}
\item {Grp. gram.:f.}
\end{itemize}
\begin{itemize}
\item {Utilização:Prov.}
\end{itemize}
\begin{itemize}
\item {Utilização:minh.}
\end{itemize}
Tirante espalmado, que de um lado se liga por anéis de ferro, móveis, ao temão do vessadoiro, e do outro á trentoira.
\section{Linguo-palatal}
\begin{itemize}
\item {Grp. gram.:adj.}
\end{itemize}
\begin{itemize}
\item {Utilização:Gram.}
\end{itemize}
Que se pronuncia, encostando-se a língua ao céu da bôca.
\section{Linha}
\begin{itemize}
\item {Grp. gram.:f.}
\end{itemize}
\begin{itemize}
\item {Utilização:Ext.}
\end{itemize}
\begin{itemize}
\item {Utilização:Náut.}
\end{itemize}
\begin{itemize}
\item {Utilização:Náut.}
\end{itemize}
\begin{itemize}
\item {Grp. gram.:Loc.}
\end{itemize}
\begin{itemize}
\item {Utilização:fam.}
\end{itemize}
\begin{itemize}
\item {Grp. gram.:Pl.}
\end{itemize}
\begin{itemize}
\item {Proveniência:(Lat. \textunderscore linea\textunderscore )}
\end{itemize}
Fio de linho.
Fio de algodão, seda, etc.: \textunderscore novelo de linhas\textunderscore .
Fio de metal, para as communicações telegráphicas.
Cordão.
Barbante, com um anzol na extremidade, para pescar peixe miúdo.
Estrema, balisa, raia.
Fileira: \textunderscore colloquem-se em linha\textunderscore .
Categoria.
A extensão, considerada com uma só dimensão ou comprimento.
Traço: \textunderscore duas linhas parallelas\textunderscore .
Lineamento.
Cada um dos traços horizontaes de uma pauta de música.
Trave horizontal, em que assentam as pernas da asna.
Duodécima segunda parte de uma pollegada.
Pequeno sinal gráphico, que se colloca á direita de uma letra, para a distinguir de outra, (em Álgebra).
Serviço de transportes, entre dois pontos por determinada via: \textunderscore esta mercadoria veio pela linha do Sul\textunderscore .
Direcção: \textunderscore seguiu em linha recta\textunderscore .
Procedimento.
Série de graus ou gerações de uma família.
Estrada: \textunderscore linha férrea\textunderscore .
\textunderscore Linha de barca\textunderscore , linha, graduada por meio de nós, que se prende á barquinha, instrumento para medir a velocidade do navio.
\textunderscore Linha de água\textunderscore , secção que descreve a superfície da água á roda do navio.
\textunderscore Têr a linha\textunderscore , têr o aprumo ou a gravidade, que convém a certas posições sociaes.
Carta: \textunderscore estimo que estas linhas o encontrem restabelecido\textunderscore .
\section{Linhaça}
\begin{itemize}
\item {Grp. gram.:f.}
\end{itemize}
Semente do linho.
\section{Linhaça-vermelha}
\begin{itemize}
\item {Grp. gram.:f.}
\end{itemize}
Nome que, em Vianna, se dá ao pintarroxo.
\section{Linhada}
\begin{itemize}
\item {Grp. gram.:f.}
\end{itemize}
\begin{itemize}
\item {Utilização:Ant.}
\end{itemize}
(Corr. de \textunderscore ninhada\textunderscore )
\section{Linhagem}
\begin{itemize}
\item {Grp. gram.:f.}
\end{itemize}
Tecido grosso de linho.
\section{Linhagem}
\begin{itemize}
\item {Utilização:Fig.}
\end{itemize}
\begin{itemize}
\item {Proveniência:(De \textunderscore linha\textunderscore )}
\end{itemize}
\textunderscore f.\textunderscore  (\textunderscore m.\textunderscore  em alguns clássicos)
Série de gerações.
Genealogia.
Condição social.
\section{Linhagista}
\begin{itemize}
\item {Grp. gram.:m.}
\end{itemize}
\begin{itemize}
\item {Proveniência:(De \textunderscore linhagem\textunderscore ^2)}
\end{itemize}
Aquelle que se dedica a investigações genealógicas.
\section{Linhajudo}
\begin{itemize}
\item {Grp. gram.:adj.}
\end{itemize}
Que trata de linhagens ou genealogias. Cf. Camillo, \textunderscore Corja\textunderscore , 51.
\section{Linhal}
\begin{itemize}
\item {Grp. gram.:m.}
\end{itemize}
Terreno, semeado de linho.
\section{Linhar}
\begin{itemize}
\item {Grp. gram.:m.}
\end{itemize}
\begin{itemize}
\item {Utilização:Prov.}
\end{itemize}
\begin{itemize}
\item {Utilização:trasm.}
\end{itemize}
O mesmo que \textunderscore linhal\textunderscore . Cf. G. Vicente.
O mesmo que \textunderscore coirela\textunderscore  ou \textunderscore belga\textunderscore .
\section{Linharão}
\begin{itemize}
\item {Grp. gram.:m.}
\end{itemize}
\begin{itemize}
\item {Utilização:Prov.}
\end{itemize}
\begin{itemize}
\item {Utilização:trasm.}
\end{itemize}
Linho grosso.
\section{Linharice}
\begin{itemize}
\item {Grp. gram.:f.}
\end{itemize}
\begin{itemize}
\item {Utilização:Prov.}
\end{itemize}
\begin{itemize}
\item {Utilização:minh.}
\end{itemize}
Terra, em que se ceifa linho.
\textunderscore Milho da linharice\textunderscore , milho semeado na terra que deu linho.
\section{Linhavão}
\begin{itemize}
\item {Grp. gram.:m.}
\end{itemize}
\begin{itemize}
\item {Proveniência:(De \textunderscore linha\textunderscore . Cp. \textunderscore alinhavar\textunderscore )}
\end{itemize}
Apparelho de linha e anzol, usado pelos pescadores da costa do Algarve.
\section{Linheira}
\begin{itemize}
\item {Grp. gram.:f.}
\end{itemize}
Mulher, que prepara ou asseda o linho para se vender.
Mulher, que vende linho.
(Cp. \textunderscore linheiro\textunderscore )
\section{Linheiro}
\begin{itemize}
\item {Grp. gram.:m.}
\end{itemize}
Aquelle que prepara e asseda o linho, para se fiar.
Aquelle que vende linho ou linhas.
Planta, o mesmo que \textunderscore linho\textunderscore .
\section{Linhita}
\begin{itemize}
\item {Grp. gram.:f.}
\end{itemize}
(V.lignita)
\section{Linhite}
\begin{itemize}
\item {Grp. gram.:f.}
\end{itemize}
(V.lignita)
\section{Linho}
\begin{itemize}
\item {Grp. gram.:m.}
\end{itemize}
\begin{itemize}
\item {Proveniência:(Lat. \textunderscore linum\textunderscore )}
\end{itemize}
Planta linácea, cuja haste produz um fio que serve para a fabricação de tecidos e rendas.
Tecido de linho: \textunderscore comprar dois metros de linho\textunderscore .
\section{Linhol}
\begin{itemize}
\item {Grp. gram.:m.}
\end{itemize}
\begin{itemize}
\item {Proveniência:(De \textunderscore linho\textunderscore )}
\end{itemize}
Fio, com que os sapateiros cosem o calçado e que também serve para coser lona.
\section{Linhol}
\begin{itemize}
\item {Grp. gram.:m.}
\end{itemize}
\begin{itemize}
\item {Utilização:T. de Pinhel}
\end{itemize}
\begin{itemize}
\item {Proveniência:(De \textunderscore linha\textunderscore )}
\end{itemize}
Systema de empar, prendendo-se as varas em linha.
\section{Linhoso}
\begin{itemize}
\item {Grp. gram.:adj.}
\end{itemize}
Que tem a natureza do linho.
\section{Linhote}
\begin{itemize}
\item {Grp. gram.:m.}
\end{itemize}
\begin{itemize}
\item {Proveniência:(De \textunderscore linha\textunderscore )}
\end{itemize}
Trave, que vai de uma parede a outra, para as segurar.
\section{Linifício}
\begin{itemize}
\item {Grp. gram.:m.}
\end{itemize}
\begin{itemize}
\item {Proveniência:(Lat. \textunderscore linificium\textunderscore )}
\end{itemize}
Trabalho em obras de linho; obra de linho.
\section{Linígero}
\begin{itemize}
\item {Grp. gram.:adj.}
\end{itemize}
\begin{itemize}
\item {Proveniência:(Lat. \textunderscore liniger\textunderscore )}
\end{itemize}
Que tem linho.
Que anda vestido de linho.
\section{Linimentar}
\begin{itemize}
\item {Grp. gram.:v. t.}
\end{itemize}
\begin{itemize}
\item {Utilização:Fig.}
\end{itemize}
\begin{itemize}
\item {Proveniência:(De \textunderscore linimento\textunderscore )}
\end{itemize}
Fazer fricções a; friccionar.
Applicar linimento a. Cf. Camillo, \textunderscore Noites de Insómn.\textunderscore , III, 64.
Acalmar, suavizar. Cf. Camillo, \textunderscore Volcões\textunderscore , 79.
\section{Linimento}
\begin{itemize}
\item {Grp. gram.:m.}
\end{itemize}
\begin{itemize}
\item {Proveniência:(Lat. \textunderscore linimentum\textunderscore )}
\end{itemize}
Medicamento untuoso, destinado a fricções.
\section{Linina}
\begin{itemize}
\item {Grp. gram.:f.}
\end{itemize}
\begin{itemize}
\item {Proveniência:(Do lat. \textunderscore linum\textunderscore )}
\end{itemize}
Substância crystallina, que se extrái do linho.
\section{Linisco}
\begin{itemize}
\item {Grp. gram.:m.}
\end{itemize}
\begin{itemize}
\item {Proveniência:(Gr. \textunderscore liniskos\textunderscore )}
\end{itemize}
Gênero de helminthos.
\section{Linneano}
\begin{itemize}
\item {Grp. gram.:adj.}
\end{itemize}
Relativo a Linneu.
\section{Linneia}
\begin{itemize}
\item {Grp. gram.:f.}
\end{itemize}
\begin{itemize}
\item {Proveniência:(De \textunderscore Linneu\textunderscore , n. p.)}
\end{itemize}
Formosa planta ornamental, da fam. das caprifoliáceas.
\section{Lino}
\begin{itemize}
\item {Grp. gram.:m.}
\end{itemize}
\begin{itemize}
\item {Proveniência:(Gr. \textunderscore linos\textunderscore )}
\end{itemize}
Canto de dôr, para chorar a morte da vegetação, na antiga poesia grega.
\section{Linoleato}
\begin{itemize}
\item {Grp. gram.:m.}
\end{itemize}
\begin{itemize}
\item {Proveniência:(Do lat. \textunderscore linum\textunderscore  + \textunderscore oleum\textunderscore )}
\end{itemize}
Combinação do ácido linoleico com uma base.
\section{Linoleico}
\begin{itemize}
\item {Grp. gram.:adj.}
\end{itemize}
\begin{itemize}
\item {Proveniência:(De \textunderscore linum\textunderscore  lat. + \textunderscore oleico\textunderscore )}
\end{itemize}
Diz-se de um ácico oleico, que se encontra nas sementes do linho.
\section{Linólico}
\begin{itemize}
\item {Grp. gram.:adj.}
\end{itemize}
O mesmo que \textunderscore linoleico\textunderscore .
\section{Linótipa}
\begin{itemize}
\item {Grp. gram.:f.}
\end{itemize}
O mesmo que \textunderscore linótipo\textunderscore .
\section{Linotipar}
\begin{itemize}
\item {Grp. gram.:v.}
\end{itemize}
\begin{itemize}
\item {Utilização:t. Typ.}
\end{itemize}
Compor em linótipo.
\section{Linótipo}
\begin{itemize}
\item {Grp. gram.:m.}
\end{itemize}
\begin{itemize}
\item {Utilização:Typ.}
\end{itemize}
\begin{itemize}
\item {Proveniência:(Do ingl. \textunderscore line of type\textunderscore )}
\end{itemize}
Máquina de composição tipográfica e de fundição de caracteres por linhas.
\section{Linótypa}
\begin{itemize}
\item {Grp. gram.:f.}
\end{itemize}
O mesmo que \textunderscore linótypo\textunderscore .
\section{Linotypar}
\begin{itemize}
\item {Grp. gram.:v.}
\end{itemize}
\begin{itemize}
\item {Utilização:t. Typ.}
\end{itemize}
Compor em linótypo.
\section{Linotypista}
\begin{itemize}
\item {Grp. gram.:m.}
\end{itemize}
Aquelle que trabalha com linótypo.
\section{Linótypo}
\begin{itemize}
\item {Grp. gram.:m.}
\end{itemize}
\begin{itemize}
\item {Utilização:Typ.}
\end{itemize}
\begin{itemize}
\item {Proveniência:(Do ingl. \textunderscore line of type\textunderscore )}
\end{itemize}
Máquina de composição typográphica e de fundição de caracteres por linhas.
\section{Lintel}
\begin{itemize}
\item {Grp. gram.:m.}
\end{itemize}
\begin{itemize}
\item {Proveniência:(Lat. hyp. \textunderscore limitellum\textunderscore )}
\end{itemize}
O mesmo que \textunderscore dintel\textunderscore .
\section{Linterna}
\begin{itemize}
\item {Grp. gram.:f.}
\end{itemize}
\begin{itemize}
\item {Utilização:Pop.}
\end{itemize}
O mesmo que \textunderscore lanterna\textunderscore . Cf. F. Manuel, \textunderscore Fid. Aprendiz\textunderscore .
(Cp. gall. \textunderscore linterna\textunderscore )
\section{Lio}
\begin{itemize}
\item {Grp. gram.:m.}
\end{itemize}
Atilho.
Feixe, mólho.
\section{Liocarpo}
\begin{itemize}
\item {Grp. gram.:adj.}
\end{itemize}
\begin{itemize}
\item {Utilização:Bot.}
\end{itemize}
\begin{itemize}
\item {Proveniência:(Do gr. \textunderscore leios\textunderscore  + \textunderscore karpos\textunderscore )}
\end{itemize}
Que tem frutos lisos.
\section{Liocéfalo}
\begin{itemize}
\item {Grp. gram.:adj.}
\end{itemize}
\begin{itemize}
\item {Proveniência:(Do gr. \textunderscore leios\textunderscore  + \textunderscore kephale\textunderscore )}
\end{itemize}
Que tem cabeça lisa.
\section{Liocéphalo}
\begin{itemize}
\item {Grp. gram.:adj.}
\end{itemize}
\begin{itemize}
\item {Proveniência:(Do gr. \textunderscore leios\textunderscore  + \textunderscore kephale\textunderscore )}
\end{itemize}
Que tem cabeça lisa.
\section{Liócomo}
\begin{itemize}
\item {Grp. gram.:adj.}
\end{itemize}
\begin{itemize}
\item {Proveniência:(Do gr. \textunderscore leios\textunderscore  + \textunderscore kome\textunderscore )}
\end{itemize}
Que tem cabellos lisos ou corredios.
\section{Liodermo}
\begin{itemize}
\item {Grp. gram.:adj.}
\end{itemize}
\begin{itemize}
\item {Utilização:Zool.}
\end{itemize}
\begin{itemize}
\item {Proveniência:(Do gr. \textunderscore leios\textunderscore  + \textunderscore derma\textunderscore )}
\end{itemize}
Que tem pelle lisa.
Que tem nus os tegumentos exteriores.
\section{Liofilo}
\begin{itemize}
\item {Grp. gram.:adj.}
\end{itemize}
\begin{itemize}
\item {Utilização:Bot.}
\end{itemize}
\begin{itemize}
\item {Proveniência:(Do gr. \textunderscore leios\textunderscore  + \textunderscore phullon\textunderscore )}
\end{itemize}
Que tem fôlhas lisas.
\section{Liomioma}
\begin{itemize}
\item {Grp. gram.:m.}
\end{itemize}
\begin{itemize}
\item {Proveniência:(Do gr. \textunderscore leios\textunderscore  + \textunderscore mux\textunderscore )}
\end{itemize}
Myoma de fibras lisas.
\section{Liomyoma}
\begin{itemize}
\item {Grp. gram.:m.}
\end{itemize}
\begin{itemize}
\item {Proveniência:(Do gr. \textunderscore leios\textunderscore  + \textunderscore mux\textunderscore )}
\end{itemize}
Myoma de fibras lisas.
\section{Liophyllo}
\begin{itemize}
\item {Grp. gram.:adj.}
\end{itemize}
\begin{itemize}
\item {Utilização:Bot.}
\end{itemize}
\begin{itemize}
\item {Proveniência:(Do gr. \textunderscore leios\textunderscore  + \textunderscore phullon\textunderscore )}
\end{itemize}
Que tem fôlhas lisas.
\section{Liópode}
\begin{itemize}
\item {Grp. gram.:adj.}
\end{itemize}
\begin{itemize}
\item {Utilização:Zool.}
\end{itemize}
\begin{itemize}
\item {Proveniência:(Do gr. \textunderscore leios\textunderscore  + \textunderscore pous\textunderscore , \textunderscore podos\textunderscore )}
\end{itemize}
Que tem lisa a planta do pé.
\section{Lioque}
\begin{itemize}
\item {Grp. gram.:m.}
\end{itemize}
(?):«\textunderscore ...pousava descuidoso sôbre um lioque:\textunderscore »\textunderscore Ms.\textunderscore  do séc. XVI.
\section{Liospermo}
\begin{itemize}
\item {Grp. gram.:adj.}
\end{itemize}
\begin{itemize}
\item {Utilização:Bot.}
\end{itemize}
\begin{itemize}
\item {Proveniência:(Do gr. \textunderscore leios\textunderscore  + \textunderscore sperma\textunderscore )}
\end{itemize}
Que tem lisas as sementes ou grãos.
\section{Lióstomo}
\begin{itemize}
\item {Grp. gram.:m.}
\end{itemize}
\begin{itemize}
\item {Proveniência:(Do gr. \textunderscore leios\textunderscore  + \textunderscore stoma\textunderscore )}
\end{itemize}
Gênero de molluscos fósseis.
Gênero de peixes acanthopterýgios.
\section{Liótricho}
\begin{itemize}
\item {fónica:co}
\end{itemize}
\begin{itemize}
\item {Grp. gram.:adj.}
\end{itemize}
\begin{itemize}
\item {Proveniência:(Do gr. \textunderscore leios\textunderscore  + \textunderscore trix\textunderscore , \textunderscore trikhos\textunderscore )}
\end{itemize}
O mesmo que \textunderscore liócomo\textunderscore .
\section{Liótrico}
\begin{itemize}
\item {Grp. gram.:adj.}
\end{itemize}
\begin{itemize}
\item {Proveniência:(Do gr. \textunderscore leios\textunderscore  + \textunderscore trix\textunderscore , \textunderscore trikhos\textunderscore )}
\end{itemize}
O mesmo que \textunderscore liócomo\textunderscore .
\section{Liotula}
\begin{itemize}
\item {Grp. gram.:f.}
\end{itemize}
\begin{itemize}
\item {Proveniência:(Do gr. \textunderscore leios\textunderscore , \textunderscore leiotos\textunderscore  + \textunderscore oula\textunderscore )}
\end{itemize}
Planta umbellífera do Egypto.
\section{Lioz}
\begin{itemize}
\item {Grp. gram.:adj.}
\end{itemize}
Diz-se de uma pedra calcária, branca e dura.
\section{Lipa}
\begin{itemize}
\item {Grp. gram.:f.}
\end{itemize}
Espécie de tanga, usada pelos Timorenses.
\section{Lipanina}
\begin{itemize}
\item {Grp. gram.:f.}
\end{itemize}
\begin{itemize}
\item {Utilização:Pharm.}
\end{itemize}
Combinação de azeite e ácido oleico, succedâneo do óleo de fígado de bacalhau.
\section{Lipária}
\begin{itemize}
\item {Grp. gram.:f.}
\end{itemize}
\begin{itemize}
\item {Proveniência:(Do gr. \textunderscore liparos\textunderscore )}
\end{itemize}
Gênero de arbustos leguminosos.
\section{Liparite}
\begin{itemize}
\item {Grp. gram.:f.}
\end{itemize}
\begin{itemize}
\item {Utilização:Miner.}
\end{itemize}
Nome genérico de rochas modernas, de vária textura.
\section{Líparo}
\begin{itemize}
\item {Grp. gram.:m.}
\end{itemize}
\begin{itemize}
\item {Proveniência:(Gr. \textunderscore liparos\textunderscore )}
\end{itemize}
Gênero de mammíferos australianos.
Gênero de insectos coleópteros.
\section{Liparocele}
\begin{itemize}
\item {Grp. gram.:m.}
\end{itemize}
\begin{itemize}
\item {Proveniência:(Do gr. \textunderscore liparos\textunderscore  + \textunderscore kele\textunderscore )}
\end{itemize}
Tumor gordurento.
\section{Liparolado}
\begin{itemize}
\item {Grp. gram.:adj.}
\end{itemize}
\begin{itemize}
\item {Utilização:Pharm.}
\end{itemize}
Diz-se das preparações, conhecidas geralmente pelo nome de pomadas.
\section{Liparóleo}
\begin{itemize}
\item {Grp. gram.:m.}
\end{itemize}
\begin{itemize}
\item {Proveniência:(Do gr. \textunderscore liparos\textunderscore )}
\end{itemize}
Designação genérica de qualquer preparado pharmacêutico, em que entra banha ou outra gordura com outras substâncias medicamentosas.
\section{Lípase}
\begin{itemize}
\item {Grp. gram.:f.}
\end{itemize}
\begin{itemize}
\item {Utilização:Chím.}
\end{itemize}
\begin{itemize}
\item {Proveniência:(Do gr. \textunderscore lipos\textunderscore , gordura)}
\end{itemize}
Fermento lipolýtico, que dissolve as gorduras, saponificando-as.
\section{Lipate}
\begin{itemize}
\item {Grp. gram.:m.}
\end{itemize}
Gargantilha cafreal de déz fios de contas de vidro.
\section{Lipes}
\begin{itemize}
\item {Grp. gram.:adj.}
\end{itemize}
Diz-se de uma pedra, cujo nome scientífico é vitríolo azul.
\section{Lipitude}
\begin{itemize}
\item {Grp. gram.:f.}
\end{itemize}
\begin{itemize}
\item {Utilização:Med.}
\end{itemize}
\begin{itemize}
\item {Proveniência:(Lat. \textunderscore lippitudo\textunderscore )}
\end{itemize}
Estado remeloso dos olhos.
\section{Lipograma}
\begin{itemize}
\item {Grp. gram.:m.}
\end{itemize}
\begin{itemize}
\item {Proveniência:(Do gr. \textunderscore leipein\textunderscore  + \textunderscore gramma\textunderscore )}
\end{itemize}
Composição literária, feita com o propósito de empregar nela uma ou mais letras do alfabeto.
\section{Lipogramático}
\begin{itemize}
\item {Grp. gram.:adj.}
\end{itemize}
Relativo ao lipograma.
\section{Lipogramatista}
\begin{itemize}
\item {Grp. gram.:m.}
\end{itemize}
Aquele que faz lipogramas.
\section{Lipogramma}
\begin{itemize}
\item {Grp. gram.:m.}
\end{itemize}
\begin{itemize}
\item {Proveniência:(Do gr. \textunderscore leipein\textunderscore  + \textunderscore gramma\textunderscore )}
\end{itemize}
Composição literária, feita com o propósito de empregar nella uma ou mais letras do alphabeto.
\section{Lipogrammático}
\begin{itemize}
\item {Grp. gram.:adj.}
\end{itemize}
Relativo ao lipogramma.
\section{Lipogrammatista}
\begin{itemize}
\item {Grp. gram.:m.}
\end{itemize}
Aquelle que faz lipogrammas.
\section{Lipoide}
\begin{itemize}
\item {Grp. gram.:adj.}
\end{itemize}
\begin{itemize}
\item {Proveniência:(Do gr. \textunderscore lipos\textunderscore  + \textunderscore eidos\textunderscore )}
\end{itemize}
Semelhante á gordura; que tem a apparência de gordura.
\section{Lipolítico}
\begin{itemize}
\item {Grp. gram.:adj.}
\end{itemize}
Relativo á lípólise.
\section{Lipólyse}
\begin{itemize}
\item {Grp. gram.:f.}
\end{itemize}
\begin{itemize}
\item {Utilização:Physiol.}
\end{itemize}
\begin{itemize}
\item {Proveniência:(Do gr. \textunderscore lipos\textunderscore  + \textunderscore lusis\textunderscore )}
\end{itemize}
Desdobramento das gorduras dos alimentos em ácidos e sabões, no decurso da digestão intestinal.
\section{Lipolýtico}
\begin{itemize}
\item {Grp. gram.:adj.}
\end{itemize}
Relativo á lípólyse.
\section{Lipoma}
\begin{itemize}
\item {Grp. gram.:m.}
\end{itemize}
\begin{itemize}
\item {Proveniência:(Do gr. \textunderscore lipos\textunderscore )}
\end{itemize}
Tumor adiposo.
\section{Lipomatoso}
\begin{itemize}
\item {Grp. gram.:adj.}
\end{itemize}
Que é da natureza do lipoma.
\section{Lipopsiquia}
\begin{itemize}
\item {Grp. gram.:f.}
\end{itemize}
\begin{itemize}
\item {Proveniência:(Gr. \textunderscore leipopsukhia\textunderscore )}
\end{itemize}
O mesmo que \textunderscore lipotimia\textunderscore .
\section{Lipopsychia}
\begin{itemize}
\item {fónica:qui}
\end{itemize}
\begin{itemize}
\item {Grp. gram.:f.}
\end{itemize}
\begin{itemize}
\item {Proveniência:(Gr. \textunderscore leipopsukhia\textunderscore )}
\end{itemize}
O mesmo que \textunderscore lipothymia\textunderscore .
\section{Lipote}
\begin{itemize}
\item {Grp. gram.:m.}
\end{itemize}
Porção de contas de barro vidrado, que corriam como moéda em Moçambique, e equivaliam a déz mites.
\section{Lipothymia}
\begin{itemize}
\item {Grp. gram.:f.}
\end{itemize}
\begin{itemize}
\item {Utilização:Med.}
\end{itemize}
\begin{itemize}
\item {Proveniência:(Gr. \textunderscore leipothumia\textunderscore )}
\end{itemize}
Desfallecimento.
Perda dos sentidos.
\section{Lipotimia}
\begin{itemize}
\item {Grp. gram.:f.}
\end{itemize}
\begin{itemize}
\item {Utilização:Med.}
\end{itemize}
\begin{itemize}
\item {Proveniência:(Gr. \textunderscore leipothumia\textunderscore )}
\end{itemize}
Desfalecimento.
Perda dos sentidos.
\section{Lipparite}
\begin{itemize}
\item {Grp. gram.:f.}
\end{itemize}
\begin{itemize}
\item {Utilização:Miner.}
\end{itemize}
Nome genérico de rochas modernas, de vária textura.
\section{Lippitude}
\begin{itemize}
\item {Grp. gram.:f.}
\end{itemize}
\begin{itemize}
\item {Utilização:Med.}
\end{itemize}
\begin{itemize}
\item {Proveniência:(Lat. \textunderscore lippitudo\textunderscore )}
\end{itemize}
Estado remeloso dos olhos.
\section{Lipiria}
\begin{itemize}
\item {Grp. gram.:f.}
\end{itemize}
\begin{itemize}
\item {Proveniência:(Do gr. \textunderscore leipein\textunderscore  + \textunderscore pur\textunderscore )}
\end{itemize}
Antiga designação de uma febre, caracterizada por intenso calor interno, sem elevação de temperatura no exterior.
\section{Liposo}
\begin{itemize}
\item {Grp. gram.:adj.}
\end{itemize}
\begin{itemize}
\item {Proveniência:(Do lat. \textunderscore lippus\textunderscore )}
\end{itemize}
Que tem remelas; remeloso.
\section{Lipposo}
\begin{itemize}
\item {Grp. gram.:adj.}
\end{itemize}
\begin{itemize}
\item {Proveniência:(Do lat. \textunderscore lippus\textunderscore )}
\end{itemize}
Que tem remelas; remeloso.
\section{Lipuria}
\begin{itemize}
\item {Grp. gram.:f.}
\end{itemize}
\begin{itemize}
\item {Utilização:Med.}
\end{itemize}
\begin{itemize}
\item {Proveniência:(Do gr. \textunderscore lipus\textunderscore  + \textunderscore ouron\textunderscore )}
\end{itemize}
Excesso de gordura nas urinas, não em fórma de emulsão, como na chyluria, mas em fórma de gotas, mais ou menos volumosas.
\section{Lipyria}
\begin{itemize}
\item {Grp. gram.:f.}
\end{itemize}
\begin{itemize}
\item {Proveniência:(Do gr. \textunderscore leipein\textunderscore  + \textunderscore pur\textunderscore )}
\end{itemize}
Antiga designação de uma febre, caracterizada por intenso calor interno, sem elevação de temperatura no exterior.
\section{Liquação}
\begin{itemize}
\item {Grp. gram.:f.}
\end{itemize}
\begin{itemize}
\item {Proveniência:(Lat. \textunderscore liquatio\textunderscore )}
\end{itemize}
Separação, por meio da fusão, de metaes que hajam formado liga.
Separação de substâncias heterogêneas liquefeitas.
\section{Liquefacção}
\begin{itemize}
\item {Grp. gram.:f.}
\end{itemize}
\begin{itemize}
\item {Proveniência:(Lat. \textunderscore liquefactio\textunderscore )}
\end{itemize}
Acto de liquefazer.
Estado daquillo que se tornou líquido.
\section{Liquefacto}
\begin{itemize}
\item {fónica:cu-e}
\end{itemize}
\begin{itemize}
\item {Grp. gram.:adj.}
\end{itemize}
O mesmo que \textunderscore liquefeito\textunderscore .
\section{Liquefazer}
\begin{itemize}
\item {Grp. gram.:v. t.}
\end{itemize}
\begin{itemize}
\item {Proveniência:(Do lat. \textunderscore liquefacere\textunderscore )}
\end{itemize}
Tornar líquido.
\section{Liquefeito}
\begin{itemize}
\item {Grp. gram.:adj.}
\end{itemize}
\begin{itemize}
\item {Proveniência:(Do lat. \textunderscore liquefactus\textunderscore )}
\end{itemize}
Reduzido a líquido.
Derretido.
\section{Líquen}
\begin{itemize}
\item {Grp. gram.:m.}
\end{itemize}
\begin{itemize}
\item {Grp. gram.:Pl.}
\end{itemize}
\begin{itemize}
\item {Proveniência:(Lat. \textunderscore lichen\textunderscore )}
\end{itemize}
Classe de plantas criptogâmicas, cuja vida é interrompida pela estiagem, e que formam a transição das algas para os cogumelos.
Espécie de impigem no rosto.
Líchenes.
\section{Liquenáceas}
\begin{itemize}
\item {Grp. gram.:f. pl.}
\end{itemize}
Família de plantas, que compreende os líquenes.
\section{Liquêneas}
\begin{itemize}
\item {Grp. gram.:f. pl.}
\end{itemize}
(V.liquenáceas)
\section{Liquênico}
\begin{itemize}
\item {Grp. gram.:adj.}
\end{itemize}
Diz-se de um ácido, que se descobriu em certos líquenes.
\section{Liquenina}
\begin{itemize}
\item {Grp. gram.:f.}
\end{itemize}
\begin{itemize}
\item {Proveniência:(De \textunderscore líquen\textunderscore )}
\end{itemize}
Fécula extraída de certas liquenáceas. Cf. \textunderscore Techn. Rur.\textunderscore , 400.
\section{Liquenografia}
\begin{itemize}
\item {Grp. gram.:f.}
\end{itemize}
Parte da Botânica, que trata especialmente dos lichenes.
\section{Liquenográfico}
\begin{itemize}
\item {Grp. gram.:adj.}
\end{itemize}
Relativo á liquenografia.
\section{Liques}
\begin{itemize}
\item {Grp. gram.:m.}
\end{itemize}
O cinco de oiros, no jôgo do truque.
O jôgo do truque.
\section{Liquescer}
\begin{itemize}
\item {Grp. gram.:v. i.}
\end{itemize}
\begin{itemize}
\item {Proveniência:(Lat. \textunderscore liquescere\textunderscore )}
\end{itemize}
Tornar-se líquido.
\section{Liquidação}
\begin{itemize}
\item {Grp. gram.:f.}
\end{itemize}
Acto ou effeito de liquidar.
Apuramento de contas.
Operação commercial, que consiste no pagamento do passivo e distribuição do activo pelos sócios da respectiva casa ou empresa.
O mesmo que \textunderscore liquefacção\textunderscore , (tratando-se de gases).
\section{Liquidador}
\begin{itemize}
\item {Grp. gram.:m.  e  adj.}
\end{itemize}
O que liquida.
\section{Liquidâmbar}
\begin{itemize}
\item {Grp. gram.:m.}
\end{itemize}
\begin{itemize}
\item {Proveniência:(De \textunderscore líquido\textunderscore  + \textunderscore âmbar\textunderscore )}
\end{itemize}
Gênero de árvores, em que se distingue uma, (\textunderscore liquidambar copalmum\textunderscore ), que, por meio de incisão, produz um suco resinoso, conhecido por liquidâmbar ou âmbar líquido.
\section{Liquidamente}
\begin{itemize}
\item {Grp. gram.:adv.}
\end{itemize}
\begin{itemize}
\item {Utilização:Fig.}
\end{itemize}
De modo líquido.
Com clareza; com evidência.
\section{Liquidando}
\begin{itemize}
\item {Grp. gram.:adj.}
\end{itemize}
Que se há de liquidar.
\section{Liquidante}
\begin{itemize}
\item {Grp. gram.:m.}
\end{itemize}
\begin{itemize}
\item {Utilização:Bras}
\end{itemize}
\begin{itemize}
\item {Proveniência:(De \textunderscore liquidar\textunderscore )}
\end{itemize}
Indivíduo, encarregado de liquidação judicial, em massa fallida. Cf. \textunderscore Jornal do Comm.\textunderscore , do Rio, de 2-IV-901.
\section{Liquidar}
\begin{itemize}
\item {Grp. gram.:v. t.}
\end{itemize}
\begin{itemize}
\item {Utilização:Fig.}
\end{itemize}
\begin{itemize}
\item {Grp. gram.:V. i.}
\end{itemize}
\begin{itemize}
\item {Proveniência:(De \textunderscore líquido\textunderscore )}
\end{itemize}
Verificar; averiguar; apurar.
Terminar ou encerrar transacções commerciaes, fazendo pagamento aos credores e repartindo o activo entre os sócios da respectiva empresa.
\section{Liquidatário}
\begin{itemize}
\item {Grp. gram.:m.  e  adj.}
\end{itemize}
O mesmo que \textunderscore liquidador\textunderscore .
\section{Liquidável}
\begin{itemize}
\item {Grp. gram.:adj.}
\end{itemize}
Que se póde liquidar.
\section{Liquidez}
\begin{itemize}
\item {Grp. gram.:f.}
\end{itemize}
Qualidade ou estado daquillo que é líquido.
\section{Liquidificação}
\begin{itemize}
\item {Grp. gram.:f.}
\end{itemize}
Acto de liquidificar.
\section{Liquidificador}
\begin{itemize}
\item {Grp. gram.:adj.}
\end{itemize}
Que liquidifica.
\section{Liquidificante}
\begin{itemize}
\item {Grp. gram.:adj.}
\end{itemize}
Que liquidifica ou promove a liquidificação.
\section{Liquidificar}
\begin{itemize}
\item {Grp. gram.:v. t.}
\end{itemize}
\begin{itemize}
\item {Proveniência:(Do lat. \textunderscore liquidus\textunderscore  + \textunderscore facere\textunderscore )}
\end{itemize}
O mesmo que \textunderscore liquefazer\textunderscore .
\section{Liquidificável}
\begin{itemize}
\item {Grp. gram.:adj.}
\end{itemize}
Que se póde liquidificar.
\section{Líquido}
\begin{itemize}
\item {Grp. gram.:adj.}
\end{itemize}
\begin{itemize}
\item {Utilização:Gram.}
\end{itemize}
\begin{itemize}
\item {Utilização:Fig.}
\end{itemize}
\begin{itemize}
\item {Grp. gram.:M.}
\end{itemize}
\begin{itemize}
\item {Proveniência:(Lat. \textunderscore liquidus\textunderscore )}
\end{itemize}
Que flue ou corre, tendendo sempre a nivelar-se e a tomar a fórma do vaso em que se encerra.
Viscoso.
Diz-se das consoantes \textunderscore l\textunderscore  e \textunderscore r\textunderscore  e ainda do \textunderscore m\textunderscore  e do \textunderscore n\textunderscore , que se juntam a outras com grande facilidade.
Liquidado.
Apurado; verificado.
Qualquer substância líquida.
Bebida ou alimento líquido: \textunderscore abusar dos líquidos\textunderscore .
\section{Liquómetro}
\begin{itemize}
\item {fónica:cu-ó}
\end{itemize}
\begin{itemize}
\item {Grp. gram.:m.}
\end{itemize}
\begin{itemize}
\item {Proveniência:(Do lat. \textunderscore liquor\textunderscore  + gr. \textunderscore metron\textunderscore )}
\end{itemize}
Pequeno instrumento, para determinar a fôrça alcoólica de certos líquidos.
\section{Liquor}
\begin{itemize}
\item {Grp. gram.:m.}
\end{itemize}
(V.licor)
\section{Lira}
\begin{itemize}
\item {Grp. gram.:f.}
\end{itemize}
Unidade monetária na Itália, correspondente a 180 reis.
\section{Lirão}
\begin{itemize}
\item {Grp. gram.:m.}
\end{itemize}
Peixe de Portugal.
\section{Liré}
\begin{itemize}
\item {Grp. gram.:m.}
\end{itemize}
\begin{itemize}
\item {Utilização:Gír.}
\end{itemize}
Vinho.
\section{Líria}
\begin{itemize}
\item {Grp. gram.:f.}
\end{itemize}
\begin{itemize}
\item {Utilização:Prov.}
\end{itemize}
\begin{itemize}
\item {Utilização:minh.}
\end{itemize}
O mesmo que \textunderscore lia\textunderscore  (do vinho).
\section{Líria}
\begin{itemize}
\item {Grp. gram.:f.}
\end{itemize}
\begin{itemize}
\item {Utilização:Prov.}
\end{itemize}
\begin{itemize}
\item {Utilização:trasm.}
\end{itemize}
Doença das crianças de peito, a qual, segundo parece, as ataca na bôca.
\section{Lírias}
\begin{itemize}
\item {Grp. gram.:f.}
\end{itemize}
O mesmo que \textunderscore lárias\textunderscore .
\section{Lirino}
\begin{itemize}
\item {Grp. gram.:adj.}
\end{itemize}
\begin{itemize}
\item {Utilização:Ant.}
\end{itemize}
Dizia-se de certo unguento, feito com fôlhas de lírio.
\section{Lírio}
\begin{itemize}
\item {Grp. gram.:m.}
\end{itemize}
\begin{itemize}
\item {Grp. gram.:Adj.}
\end{itemize}
\begin{itemize}
\item {Proveniência:(Do lat. \textunderscore lilium\textunderscore )}
\end{itemize}
Gênero de plantas bulbosas, que apresentam, sôbre uma haste delgada e longa, flôres avelludadas.
A flôr do lírio branco.
A brancura do lírio.
Variedade de peixe-espada.
Ferro de três pontas, que se collocava nos fossos das fortificações, para ferir quem ali caísse.
Que é próprio do lírio; relativo ao lírio. Cf. Filinto, VII, 119.
\section{Lírio-cárdeno}
\begin{itemize}
\item {Grp. gram.:m.}
\end{itemize}
Planta medicinal, o mesmo que \textunderscore íris\textunderscore .
\section{Lírio-convalle}
\begin{itemize}
\item {Grp. gram.:m.}
\end{itemize}
Planta medicinal, (\textunderscore lillium convallium\textunderscore ). Cf. \textunderscore Desengano da Med.\textunderscore , 49.
\section{Lírio-de-água}
\begin{itemize}
\item {Grp. gram.:m.}
\end{itemize}
Planta nympheácea, medicinal, (\textunderscore nymphea odorata\textunderscore ).
\section{Liriodendrina}
\begin{itemize}
\item {Grp. gram.:f.}
\end{itemize}
\begin{itemize}
\item {Proveniência:(De \textunderscore liriodendro\textunderscore )}
\end{itemize}
Substância amarga e balsâmica, extrahida da casca da tulipa.
\section{Liriodendro}
\begin{itemize}
\item {Grp. gram.:m.}
\end{itemize}
\begin{itemize}
\item {Proveniência:(Do gr. \textunderscore leirion\textunderscore  + \textunderscore dendron\textunderscore )}
\end{itemize}
Nome scientífico da tulipa.
\section{Lírio-dos-tintureiros}
\begin{itemize}
\item {Grp. gram.:m.}
\end{itemize}
Espécie de reseda, (\textunderscore reseda luteola\textunderscore , Lin.).
\section{Lírio-férreo}
\begin{itemize}
\item {Grp. gram.:m.}
\end{itemize}
Peixe de Portugal.
\section{Lirioide}
\begin{itemize}
\item {Grp. gram.:adj.}
\end{itemize}
\begin{itemize}
\item {Proveniência:(Do gr. \textunderscore leirion\textunderscore  + \textunderscore eidos\textunderscore )}
\end{itemize}
O mesmo que \textunderscore liliforme\textunderscore .
\section{Liro}
\begin{itemize}
\item {Grp. gram.:m.}
\end{itemize}
Peixe dos Açores.
(Corr. de \textunderscore lírio\textunderscore ? V. \textunderscore lírio\textunderscore  e \textunderscore lírio-férreo\textunderscore )
\section{Liró}
\begin{itemize}
\item {Grp. gram.:adj.}
\end{itemize}
\begin{itemize}
\item {Utilização:Fam.}
\end{itemize}
Vestido com apuro; casquilho; muito enfeitado.
\section{Lis}
\begin{itemize}
\item {Grp. gram.:m.}
\end{itemize}
\begin{itemize}
\item {Proveniência:(Do fr. \textunderscore lis\textunderscore )}
\end{itemize}
Lírio, Cp. \textunderscore flôr-de-lis\textunderscore .
\section{Lisamente}
\begin{itemize}
\item {Grp. gram.:adv.}
\end{itemize}
De modo liso.
\section{Lisbôa}
\begin{itemize}
\item {Grp. gram.:f.}
\end{itemize}
\begin{itemize}
\item {Utilização:Prov.}
\end{itemize}
\begin{itemize}
\item {Utilização:alent.}
\end{itemize}
Variedade de alface.
\section{Lisboano}
\begin{itemize}
\item {Grp. gram.:m.  e  adj.}
\end{itemize}
\begin{itemize}
\item {Utilização:T. de Monção}
\end{itemize}
O mesmo que \textunderscore lisboês\textunderscore .
\section{Lisboês}
\begin{itemize}
\item {Grp. gram.:m.  e  adj.}
\end{itemize}
\begin{itemize}
\item {Utilização:Ant.}
\end{itemize}
O mesmo que \textunderscore lisbonense\textunderscore . Cf. Balth. Telles, \textunderscore Chrón. da Comp. de Jesus\textunderscore , liv. I, cap. 9.
\section{Lisboêta}
\begin{itemize}
\item {Grp. gram.:m.  e  f.}
\end{itemize}
\begin{itemize}
\item {Grp. gram.:Adj.}
\end{itemize}
Pessôa, natural de Lisbôa.
Relativo a Lisbôa, próprio de Lisbôa: \textunderscore costumes lisboêtas\textunderscore .
\section{Lisboetismo}
\begin{itemize}
\item {fónica:bo-e}
\end{itemize}
\begin{itemize}
\item {Grp. gram.:m.}
\end{itemize}
Modos ou hábitos de lisboêta.
\section{Lisbonense}
\begin{itemize}
\item {Grp. gram.:adj.}
\end{itemize}
\begin{itemize}
\item {Grp. gram.:M.}
\end{itemize}
Relativo a Lisbôa.
Aquelle que é natural de Lisbôa.
\section{Lisbonês}
\begin{itemize}
\item {Grp. gram.:m.  e  adj.}
\end{itemize}
(V.lisbonense)
\section{Lisbonina}
\begin{itemize}
\item {Grp. gram.:f.}
\end{itemize}
\begin{itemize}
\item {Utilização:Ant.}
\end{itemize}
Peça de oiro.
(Cp. \textunderscore lisbonino\textunderscore )
\section{Lisbonino}
\begin{itemize}
\item {Grp. gram.:adj.}
\end{itemize}
(V.lisbonense)
\section{Liseirão}
\begin{itemize}
\item {Grp. gram.:adj.}
\end{itemize}
\begin{itemize}
\item {Utilização:ant.}
\end{itemize}
\begin{itemize}
\item {Utilização:Pop.}
\end{itemize}
\begin{itemize}
\item {Proveniência:(De \textunderscore liso\textunderscore )}
\end{itemize}
Sincero, bonacheirão, franco.
\section{Lisma}
\begin{itemize}
\item {Grp. gram.:f.}
\end{itemize}
\begin{itemize}
\item {Proveniência:(Fr. \textunderscore lisme\textunderscore )}
\end{itemize}
Direito, que os Franceses pagavam aos Argelinos e Tunesinos pela pesca do coral.
\section{Lismar}
\begin{itemize}
\item {Grp. gram.:v. t.}
\end{itemize}
\begin{itemize}
\item {Utilização:Prov.}
\end{itemize}
\begin{itemize}
\item {Utilização:minh.}
\end{itemize}
Tirar o lismo a (enguias).
\section{Lismo}
\begin{itemize}
\item {Grp. gram.:m.}
\end{itemize}
\begin{itemize}
\item {Utilização:Prov.}
\end{itemize}
\begin{itemize}
\item {Utilização:minh.}
\end{itemize}
Matéria viscosa, que recobre o corpo dos peixes.
\section{Liso}
\begin{itemize}
\item {Grp. gram.:adj.}
\end{itemize}
\begin{itemize}
\item {Utilização:Bras. do N}
\end{itemize}
\begin{itemize}
\item {Proveniência:(Do gr. \textunderscore lissos\textunderscore , seg. Körting)}
\end{itemize}
Que tem superfície plana ou sem asperezas.
Macio.
Chato.
Que não tem pregas nem ornatos.
Lhano; franco.
Diz-se dos animaes, que têm pêlo avermelhado e fino.
\section{Lisonja}
\begin{itemize}
\item {Grp. gram.:f.}
\end{itemize}
\begin{itemize}
\item {Utilização:Fig.}
\end{itemize}
Acto ou effeito de lisonjear.
Mimo, afago.
\section{Lisonja}
\begin{itemize}
\item {Grp. gram.:f.}
\end{itemize}
\begin{itemize}
\item {Utilização:Heráld.}
\end{itemize}
Espaço aberto, em fórma de parallelogramo:«\textunderscore seus olhos, fugindo pelas lisonjas das janellas...\textunderscore »Alencar, \textunderscore Minas de Prata\textunderscore , III, 379.
Ornato de brasão, em fórma de losango, num dos ângulos do escudo.
(Cp. \textunderscore losanja\textunderscore )
\section{Lisonjado}
\begin{itemize}
\item {Grp. gram.:m.}
\end{itemize}
\begin{itemize}
\item {Utilização:Heráld.}
\end{itemize}
Campo, formado de lisonjas, alternando-se a côr com o metal. Cf. L. Ribeiro, \textunderscore Trat. de Armaria\textunderscore .
\section{Lisonjar}
\textunderscore v. t.\textunderscore  (e der.)
O mesmo que \textunderscore lisonjear\textunderscore , etc.
\section{Lisonjaria}
\begin{itemize}
\item {Grp. gram.:f.}
\end{itemize}
\begin{itemize}
\item {Proveniência:(De \textunderscore lisonja\textunderscore )}
\end{itemize}
Acto ou hábito de lisonjear.
\section{Lisonjeador}
\begin{itemize}
\item {Grp. gram.:adj.}
\end{itemize}
\begin{itemize}
\item {Grp. gram.:M.}
\end{itemize}
Que lisonjeia; que satisfaz o amor-próprio.
Aquelle que lisonjeia.
\section{Lisonjear}
\begin{itemize}
\item {Grp. gram.:v. t.}
\end{itemize}
\begin{itemize}
\item {Proveniência:(De \textunderscore lisonja\textunderscore )}
\end{itemize}
Elogiar com excesso e affectação.
Adular.
Bajular.
Agradar a; tornar satisfeito: \textunderscore êsse resultado lisonjeia-me\textunderscore .
\section{Lisonjeiramente}
\begin{itemize}
\item {Grp. gram.:adv.}
\end{itemize}
De modo lisonjeiro; com lisonja.
\section{Lisonjeiro}
\begin{itemize}
\item {Grp. gram.:m.  e  adj.}
\end{itemize}
O mesmo que \textunderscore lisonjeador\textunderscore .
\section{Lisonjeria}
\begin{itemize}
\item {Grp. gram.:f.}
\end{itemize}
(V.lisonjaria). Cf. G. Vicente, I, 331.
\section{Lissadeira}
\begin{itemize}
\item {Grp. gram.:f.}
\end{itemize}
\begin{itemize}
\item {Utilização:Gal}
\end{itemize}
\begin{itemize}
\item {Proveniência:(Do fr. \textunderscore lisser\textunderscore )}
\end{itemize}
Apparelho, para alisar e lustrar, nas fábricas de fiação. Cf. \textunderscore Inquér. Industr.\textunderscore , III, 66.
\section{Lissótricho}
\begin{itemize}
\item {fónica:co}
\end{itemize}
\begin{itemize}
\item {Grp. gram.:adj.}
\end{itemize}
\begin{itemize}
\item {Proveniência:(Do gr. \textunderscore lissos\textunderscore  + \textunderscore trix\textunderscore , \textunderscore trikhos\textunderscore )}
\end{itemize}
Que tem cabello liso ou corredio.
\section{Lissótrico}
\begin{itemize}
\item {Grp. gram.:adj.}
\end{itemize}
\begin{itemize}
\item {Proveniência:(Do gr. \textunderscore lissos\textunderscore  + \textunderscore trix\textunderscore , \textunderscore trikhos\textunderscore )}
\end{itemize}
Que tem cabello liso ou corredio.
\section{Lista}
\begin{itemize}
\item {Grp. gram.:f.}
\end{itemize}
\begin{itemize}
\item {Proveniência:(Do ant. alt. al. \textunderscore lista\textunderscore )}
\end{itemize}
Tira comprida e estreita.
Listra.
Relação, rol: \textunderscore uma lista de assinantes\textunderscore .
Esteira de embarcação.
\section{Listã}
\begin{itemize}
\item {Grp. gram.:f.}
\end{itemize}
Variedade, de uva, o mesmo que \textunderscore aceitan\textunderscore .
\section{Listan}
\begin{itemize}
\item {Grp. gram.:f.}
\end{itemize}
Variedade, de uva, o mesmo que \textunderscore aceitan\textunderscore .
\section{Listan-vermelha}
\begin{itemize}
\item {Grp. gram.:f.}
\end{itemize}
Casta de uva arroxeada ou côr de jacinto.
\section{Listão}
\begin{itemize}
\item {Grp. gram.:m.}
\end{itemize}
\begin{itemize}
\item {Grp. gram.:Adj.}
\end{itemize}
\begin{itemize}
\item {Proveniência:(De \textunderscore lista\textunderscore )}
\end{itemize}
Lista grande; faixa.
Esteira de embarcação.
Régua de carpinteiro.
Diz-se do toiro, que tem no dorso uma listra de côr differente da do resto do corpo.
\section{Listário}
\begin{itemize}
\item {Grp. gram.:m.}
\end{itemize}
\begin{itemize}
\item {Utilização:Bras}
\end{itemize}
\begin{itemize}
\item {Utilização:ant.}
\end{itemize}
\begin{itemize}
\item {Proveniência:(De \textunderscore lista\textunderscore )}
\end{itemize}
Feitor, encarregado de registar o número e o pêso dos diamantes achados.
\section{Listel}
\begin{itemize}
\item {Grp. gram.:m.}
\end{itemize}
\begin{itemize}
\item {Proveniência:(Do it. \textunderscore listello\textunderscore )}
\end{itemize}
Moldura, que acompanha outra maior ou separa as caneluras de uma columna.
\section{Listelão}
\begin{itemize}
\item {Grp. gram.:m.}
\end{itemize}
\begin{itemize}
\item {Proveniência:(De \textunderscore listel\textunderscore )}
\end{itemize}
Grande moldura, quadrada e lisa.
\section{Listelo}
\begin{itemize}
\item {Grp. gram.:m.}
\end{itemize}
(V.listel)
\section{Líster}
\begin{itemize}
\item {Grp. gram.:m.}
\end{itemize}
\begin{itemize}
\item {Proveniência:(Do ing. \textunderscore leicester\textunderscore )}
\end{itemize}
Bovídeo, aperfeiçoado, no condado de Leicester.
\section{Listerina}
\begin{itemize}
\item {Grp. gram.:f.}
\end{itemize}
\begin{itemize}
\item {Utilização:Bras}
\end{itemize}
Medicamento prophylático e desinfectante.
\section{Listra}
\begin{itemize}
\item {Grp. gram.:f.}
\end{itemize}
Risca num tecido, de côr differente da dêste.
Lista.
(Alter. de \textunderscore lista\textunderscore )
\section{Listrana}
\begin{itemize}
\item {Grp. gram.:f.}
\end{itemize}
\begin{itemize}
\item {Utilização:Prov.}
\end{itemize}
\begin{itemize}
\item {Utilização:trasm.}
\end{itemize}
Rapariga descarada; lambitana.
\section{Listrão}
\begin{itemize}
\item {Grp. gram.:m.}
\end{itemize}
(V.listra)
Variedade de uva, na Ilha da Madeira.
\section{Listrar}
\begin{itemize}
\item {Grp. gram.:v. t.}
\end{itemize}
Entremear de listras; ornar de listras.
\section{Lisura}
\begin{itemize}
\item {Grp. gram.:f.}
\end{itemize}
\begin{itemize}
\item {Utilização:Fig.}
\end{itemize}
Qualidade de liso.
Macieza.
Planura.
Lhaneza, sinceridade; franqueza.
\section{Litação}
\begin{itemize}
\item {Grp. gram.:f.}
\end{itemize}
\begin{itemize}
\item {Proveniência:(Lat. \textunderscore litatio\textunderscore )}
\end{itemize}
Acto de litar.
\section{Litagogo}
\begin{itemize}
\item {Grp. gram.:adj.}
\end{itemize}
\begin{itemize}
\item {Utilização:Med.}
\end{itemize}
\begin{itemize}
\item {Proveniência:(Do gr. \textunderscore lithos\textunderscore  + \textunderscore agogos\textunderscore )}
\end{itemize}
Dizia-se das substâncias, que tinham a propriedade de expulsar os cálculos da bexiga.
\section{Litania}
\begin{itemize}
\item {Grp. gram.:f.}
\end{itemize}
\begin{itemize}
\item {Proveniência:(Lat. \textunderscore litania\textunderscore )}
\end{itemize}
O mesmo que \textunderscore ladaínha\textunderscore .
\section{Litantraz}
\begin{itemize}
\item {Grp. gram.:m.}
\end{itemize}
\begin{itemize}
\item {Proveniência:(Do gr. \textunderscore lithos\textunderscore  + \textunderscore anthrax\textunderscore )}
\end{itemize}
Espécie de carvão bituminoso.
\section{Litão}
\begin{itemize}
\item {Grp. gram.:m.}
\end{itemize}
\begin{itemize}
\item {Utilização:Ant.}
\end{itemize}
O mesmo que \textunderscore leitão\textunderscore ^1, peixe.
Cação sêco. Cf. \textunderscore Tombo do Estado da Índia\textunderscore , 248, nos \textunderscore Subsídios\textunderscore  de Felner.
\section{Litar}
\begin{itemize}
\item {Grp. gram.:v. t.}
\end{itemize}
\begin{itemize}
\item {Grp. gram.:V. i.}
\end{itemize}
\begin{itemize}
\item {Proveniência:(Lat. \textunderscore litare\textunderscore )}
\end{itemize}
Offerecer (sacrifício).
Sacrificar com bons preságios.
Obter bom preságio.
Têr bons indícios. Cf. Castilho, \textunderscore Fastos\textunderscore , II, 175.
\section{Litargírio}
\begin{itemize}
\item {Grp. gram.:m.}
\end{itemize}
\begin{itemize}
\item {Proveniência:(Do gr. \textunderscore lithos\textunderscore  + \textunderscore argorus\textunderscore )}
\end{itemize}
Designação antiga do protóxido de chumbo semi-vítreo.
\section{Litargo}
\begin{itemize}
\item {Grp. gram.:m.}
\end{itemize}
O mesmo que \textunderscore litargírio\textunderscore .
\section{Lite}
\begin{itemize}
\item {Grp. gram.:f.}
\end{itemize}
\begin{itemize}
\item {Utilização:Ant.}
\end{itemize}
O mesmo que \textunderscore lide\textunderscore . Cf. \textunderscore Viriato Trág.\textunderscore , IV, 13.
\section{Liteira}
\begin{itemize}
\item {Grp. gram.:f.}
\end{itemize}
\begin{itemize}
\item {Proveniência:(Do lat. \textunderscore lectarius\textunderscore )}
\end{itemize}
Cadeirinha portátil e coberta, sustentada por dois varaes compridos e conduzida por duas bêstas, uma atrás e outra adeante.
\section{Liteira}
\begin{itemize}
\item {Grp. gram.:f.}
\end{itemize}
Tecido de estopa e lan, tinto de preto, que se usava no vestuário das camponesas alentejanas. Cf. Rev. \textunderscore Tradição\textunderscore , IV, 154.
(Cp. \textunderscore liteiro\textunderscore )
\section{Liteireiro}
\begin{itemize}
\item {Grp. gram.:m.}
\end{itemize}
Aquelle que guia a liteira.
\section{Liteiro}
\begin{itemize}
\item {Grp. gram.:m.}
\end{itemize}
\begin{itemize}
\item {Utilização:Prov.}
\end{itemize}
\begin{itemize}
\item {Utilização:trasm.}
\end{itemize}
Conjunto de dois ou três lençoes, estendidos perpendicularmente e presos a uma varella, para impedir que o centeio salte da eira.
(Relaciona-se com o lat. \textunderscore lectarius\textunderscore ?)
\section{Literal}
\begin{itemize}
\item {Grp. gram.:adj.}
\end{itemize}
\begin{itemize}
\item {Proveniência:(Lat. \textunderscore literalis\textunderscore )}
\end{itemize}
Que é conforme á letra ou a um texto: \textunderscore traducção literal\textunderscore .
Sujeito ao rigor das palavras.
Rigoroso.
Restricto.
Terminante, claro.
Expresso por meio de letras.
\section{Literalmente}
\begin{itemize}
\item {Grp. gram.:adv.}
\end{itemize}
De modo literal; á letra.
\section{Literareiro}
\begin{itemize}
\item {Grp. gram.:adj.}
\end{itemize}
\begin{itemize}
\item {Utilização:Deprec.}
\end{itemize}
O mesmo que \textunderscore literário\textunderscore . Cf. Camillo, \textunderscore Noites de Insómn.\textunderscore , VI, 90.
\section{Literariamente}
\begin{itemize}
\item {Grp. gram.:adv.}
\end{itemize}
De modo literário.
Relativamente ás bellas-letras.
\section{Literário}
\begin{itemize}
\item {Grp. gram.:adj.}
\end{itemize}
\begin{itemize}
\item {Proveniência:(Lat. \textunderscore literarius\textunderscore )}
\end{itemize}
Relativo a letras.
Concernente á literatura.
Que diz respeito a conhecimentos humanos, adquiridos pelo estudo ou pela leitura: \textunderscore progressos literários\textunderscore .
\section{Literata}
\begin{itemize}
\item {Grp. gram.:f.}
\end{itemize}
Mulher, que compõe obras literárias; escritora.
(Cp. \textunderscore literato\textunderscore )
\section{Literataço}
\begin{itemize}
\item {Grp. gram.:m.}
\end{itemize}
\begin{itemize}
\item {Utilização:Deprec.}
\end{itemize}
\begin{itemize}
\item {Proveniência:(De \textunderscore literato\textunderscore )}
\end{itemize}
Literato pretensioso.
\section{Literatagem}
\begin{itemize}
\item {Grp. gram.:f.}
\end{itemize}
\begin{itemize}
\item {Utilização:Deprec.}
\end{itemize}
\begin{itemize}
\item {Proveniência:(De \textunderscore literato\textunderscore )}
\end{itemize}
A classe dos literatos; os literatos ridículos. Cf. Camillo, \textunderscore Narcót.\textunderscore , II, 15.
\section{Literatejar}
\begin{itemize}
\item {Grp. gram.:v. i.}
\end{itemize}
\begin{itemize}
\item {Proveniência:(De \textunderscore literato\textunderscore )}
\end{itemize}
Fazer literatura ordinária.
Sêr literato ridículo. Cf. Camillo, \textunderscore Crit. do Canc. Al.\textunderscore , p. VIII.
\section{Literatelho}
\begin{itemize}
\item {fónica:tê}
\end{itemize}
\begin{itemize}
\item {Grp. gram.:m.}
\end{itemize}
O mesmo que \textunderscore literatiço\textunderscore .
\section{Literatice}
\begin{itemize}
\item {Grp. gram.:f.}
\end{itemize}
\begin{itemize}
\item {Utilização:Deprec.}
\end{itemize}
\begin{itemize}
\item {Proveniência:(De \textunderscore literato\textunderscore )}
\end{itemize}
Qualidade de literato ridículo ou piegas.
Literatura ridícula.
\section{Literatiço}
\begin{itemize}
\item {Grp. gram.:m.}
\end{itemize}
\begin{itemize}
\item {Utilização:Deprec.}
\end{itemize}
\begin{itemize}
\item {Grp. gram.:Adj.}
\end{itemize}
\begin{itemize}
\item {Proveniência:(De \textunderscore literato\textunderscore )}
\end{itemize}
Literato reles, ordinário. Cf. Camillo, \textunderscore Crit. do Cancion.\textunderscore , 41.
Mediocremente letrado.
\section{Literatiqueiro}
\begin{itemize}
\item {Grp. gram.:m.}
\end{itemize}
\begin{itemize}
\item {Utilização:Deprec.}
\end{itemize}
Literato reles. Cf. Pacheco, \textunderscore Promptuário\textunderscore , 9.
\section{Literatismo}
\begin{itemize}
\item {Grp. gram.:m.}
\end{itemize}
\begin{itemize}
\item {Utilização:Neol.}
\end{itemize}
Mania de literato; literatice. Cf. Ortigão, \textunderscore Hollanda\textunderscore , 290.
\section{Literato}
\begin{itemize}
\item {Grp. gram.:adj.}
\end{itemize}
\begin{itemize}
\item {Grp. gram.:M.}
\end{itemize}
\begin{itemize}
\item {Proveniência:(Lat. \textunderscore literatus\textunderscore )}
\end{itemize}
O mesmo que \textunderscore letrado\textunderscore .
Aquelle que possue muitos conhecimentos de literatura.
Aquelle que cultiva distintamente a literatura.
Aquelle que escreve em público, sobre algum ramo de literatura.
Escritor.
\section{Literatura}
\begin{itemize}
\item {Grp. gram.:f.}
\end{itemize}
\begin{itemize}
\item {Proveniência:(Lat. \textunderscore literatura\textunderscore )}
\end{itemize}
Conhecimento das bellas-letras.
Conjunto do producções literárias de uma nação, de uma região ou de uma época.
Os homens de letras.
Arte de fazer composições literárias.
\section{Lithagogo}
\begin{itemize}
\item {Grp. gram.:adj.}
\end{itemize}
\begin{itemize}
\item {Utilização:Med.}
\end{itemize}
\begin{itemize}
\item {Proveniência:(Do gr. \textunderscore lithos\textunderscore  + \textunderscore agogos\textunderscore )}
\end{itemize}
Dizia-se das substâncias, que tinham a propriedade de expulsar os cálculos da bexiga.
\section{Lithanthraz}
\begin{itemize}
\item {Grp. gram.:m.}
\end{itemize}
\begin{itemize}
\item {Proveniência:(Do gr. \textunderscore lithos\textunderscore  + \textunderscore anthrax\textunderscore )}
\end{itemize}
Espécie de carvão bituminoso.
\section{Lithargo}
\begin{itemize}
\item {Grp. gram.:m.}
\end{itemize}
O mesmo que \textunderscore lithargýrio\textunderscore .
\section{Lithargýrio}
\begin{itemize}
\item {Grp. gram.:m.}
\end{itemize}
\begin{itemize}
\item {Proveniência:(Do gr. \textunderscore lithos\textunderscore  + \textunderscore argorus\textunderscore )}
\end{itemize}
Designação antiga do protóxydo de chumbo semi-vítreo.
\section{Líthia}
\begin{itemize}
\item {Grp. gram.:f.}
\end{itemize}
O mesmo que \textunderscore lithina\textunderscore .
\section{Lithíase}
\begin{itemize}
\item {Grp. gram.:f.}
\end{itemize}
\begin{itemize}
\item {Proveniência:(Gr. \textunderscore lithiasis\textunderscore )}
\end{itemize}
Formação de pedra nas vias urinárias.
Concreção pedregosa nas pálpebras.
\section{Lithíasis}
\begin{itemize}
\item {Grp. gram.:f.}
\end{itemize}
\begin{itemize}
\item {Proveniência:(Gr. \textunderscore lithiasis\textunderscore )}
\end{itemize}
Formação de pedra nas vias urinárias.
Concreção pedregosa nas pálpebras.
\section{Líthico}
\begin{itemize}
\item {Grp. gram.:adj.}
\end{itemize}
\begin{itemize}
\item {Proveniência:(Do gr. \textunderscore lithos\textunderscore )}
\end{itemize}
Relativo a pedra.
Dizia-se de um ácido, que hoje se chama \textunderscore úrico\textunderscore .
\section{Lithina}
\begin{itemize}
\item {Grp. gram.:f.}
\end{itemize}
Óxydo de líthio.
\section{Lithinado}
\begin{itemize}
\item {Grp. gram.:adj.}
\end{itemize}
Em que há lithina; que contém lithina: \textunderscore visto que as águas dos Cucos são as mais lithinadas de Portugal...\textunderscore 
\section{Lithínico}
\begin{itemize}
\item {Grp. gram.:adj.}
\end{itemize}
Relativo á lithina.
\section{Lithinífero}
\begin{itemize}
\item {Grp. gram.:adj.}
\end{itemize}
\begin{itemize}
\item {Proveniência:(De \textunderscore lithina\textunderscore  + lat. \textunderscore ferre\textunderscore )}
\end{itemize}
Que contém lithina.
\section{Líthio}
\begin{itemize}
\item {Grp. gram.:m.}
\end{itemize}
\begin{itemize}
\item {Proveniência:(Do gr. \textunderscore lithos\textunderscore )}
\end{itemize}
Metal branco e dúctil, que constitue a base da lithina.
\section{Lithizonte}
\begin{itemize}
\item {Grp. gram.:m.}
\end{itemize}
\begin{itemize}
\item {Proveniência:(Do gr. \textunderscore lithizon\textunderscore )}
\end{itemize}
Pedra preciosa da India, espécie de granada.
\section{Litho...}
\begin{itemize}
\item {Grp. gram.:pref.}
\end{itemize}
\begin{itemize}
\item {Proveniência:(Do gr. \textunderscore lithos\textunderscore )}
\end{itemize}
(designativo de \textunderscore pedra\textunderscore )
\section{Lithocálamo}
\begin{itemize}
\item {Grp. gram.:m.}
\end{itemize}
\begin{itemize}
\item {Proveniência:(Do gr. \textunderscore lithos\textunderscore  + \textunderscore kalamos\textunderscore )}
\end{itemize}
Haste fóssil de cana.
\section{Lithocarpo}
\begin{itemize}
\item {Grp. gram.:m.}
\end{itemize}
\begin{itemize}
\item {Proveniência:(Do gr. \textunderscore lithos\textunderscore  + \textunderscore karpos\textunderscore )}
\end{itemize}
Fruto fóssil.
\section{Lithochromia}
\begin{itemize}
\item {Grp. gram.:f.}
\end{itemize}
\begin{itemize}
\item {Proveniência:(Do gr. \textunderscore lithos\textunderscore  + \textunderscore khroma\textunderscore )}
\end{itemize}
Imitação da pintura a óleo por meio da lithographia.
\section{Lithochrómico}
\begin{itemize}
\item {Grp. gram.:adj.}
\end{itemize}
Relativo á lithochromia.
\section{Lithochromista}
\begin{itemize}
\item {Grp. gram.:m.}
\end{itemize}
Aquelle que trabalha em lithochromia.
\section{Lithoclase}
\begin{itemize}
\item {Grp. gram.:f.}
\end{itemize}
\begin{itemize}
\item {Proveniência:(Do gr. \textunderscore lithos\textunderscore  + \textunderscore klao\textunderscore )}
\end{itemize}
Fractura natural de rocha.
\section{Lithoclastia}
\begin{itemize}
\item {Grp. gram.:f.}
\end{itemize}
\begin{itemize}
\item {Proveniência:(De \textunderscore lithoclasto\textunderscore )}
\end{itemize}
Processo de reduzir a fragmentos os cálculos da bexiga.
\section{Lithoclasto}
\begin{itemize}
\item {Grp. gram.:m.}
\end{itemize}
\begin{itemize}
\item {Proveniência:(Do gr. \textunderscore lithos\textunderscore  + \textunderscore klao\textunderscore )}
\end{itemize}
Instrumento cirúrgico, empregado em lithoclastia.
\section{Lithocolla}
\begin{itemize}
\item {Grp. gram.:f.}
\end{itemize}
\begin{itemize}
\item {Proveniência:(De \textunderscore litho\textunderscore  + \textunderscore colla\textunderscore )}
\end{itemize}
Espécie de betume, em que entra pó de pedra e que serve para soldar pedras.
\section{Lithodendro}
\begin{itemize}
\item {Grp. gram.:m.}
\end{itemize}
\begin{itemize}
\item {Proveniência:(Do gr. \textunderscore lithos\textunderscore  + \textunderscore dendron\textunderscore )}
\end{itemize}
Polypo fóssil.
\section{Lithodonte}
\begin{itemize}
\item {Grp. gram.:m.}
\end{itemize}
\begin{itemize}
\item {Proveniência:(Do gr. \textunderscore lithos\textunderscore  + \textunderscore odous\textunderscore , \textunderscore odontos\textunderscore )}
\end{itemize}
Gênero de molluscos, de sabor apimentado.
\section{Lithodiályse}
\begin{itemize}
\item {Grp. gram.:f.}
\end{itemize}
\begin{itemize}
\item {Proveniência:(De \textunderscore litho...\textunderscore  + \textunderscore diályse\textunderscore )}
\end{itemize}
Qualquer processo de dissolver os cálculos vesicaes.
\section{Lithoféllico}
\begin{itemize}
\item {Grp. gram.:adj.}
\end{itemize}
\begin{itemize}
\item {Proveniência:(Do gr. \textunderscore lithos\textunderscore  + lat. \textunderscore fel\textunderscore )}
\end{itemize}
Diz-se de um ácido, que se encontra nos bezoares orientaes.
\section{Lithofellínico}
\begin{itemize}
\item {Grp. gram.:adj.}
\end{itemize}
O mesmo que \textunderscore lithoféllico\textunderscore .
\section{Lithogenesia}
\begin{itemize}
\item {Grp. gram.:f.}
\end{itemize}
\begin{itemize}
\item {Proveniência:(Do gr. \textunderscore lithos\textunderscore  + \textunderscore genesis\textunderscore )}
\end{itemize}
Investigação das leis que presidem á formação das pedras.
\section{Lithogenésico}
\begin{itemize}
\item {Grp. gram.:adj.}
\end{itemize}
Relativo a lithogenesia.
\section{Lithoglyphia}
\begin{itemize}
\item {Grp. gram.:f.}
\end{itemize}
\begin{itemize}
\item {Proveniência:(De \textunderscore lithóglypho\textunderscore )}
\end{itemize}
Arte de gravar sôbre pedra.
\section{Lithoglýphico}
\begin{itemize}
\item {Grp. gram.:adj.}
\end{itemize}
Relativo á lithoglyphia.
\section{Lithoglypho}
\begin{itemize}
\item {Grp. gram.:m.}
\end{itemize}
\begin{itemize}
\item {Proveniência:(Do gr. \textunderscore lithos\textunderscore  + \textunderscore glyphein\textunderscore )}
\end{itemize}
Que pratíca a lithoglyphia.
\section{Lithographar}
\begin{itemize}
\item {Grp. gram.:v. t.}
\end{itemize}
\begin{itemize}
\item {Proveniência:(De \textunderscore lithógrapho\textunderscore )}
\end{itemize}
Imprimir pelo processo lithográphico.
\section{Lithographia}
\begin{itemize}
\item {Grp. gram.:f.}
\end{itemize}
\begin{itemize}
\item {Proveniência:(De \textunderscore lithógrapho\textunderscore )}
\end{itemize}
Processo para reproduzir em papel, por meio de impressão, aquillo que estava desenhado ou escrito sôbre uma pedra especial.
Officina de lithógrapho.
Fôlha ou estampa, impressa lithographicamente.
\section{Lithográphico}
\begin{itemize}
\item {Grp. gram.:adj.}
\end{itemize}
Relativo á lithographia.
Diz-se de uma espécie de pedra, que é calcário compacto, de grão fino homogêneo, e que se emprega em lithographia.
\section{Lithógrapho}
\begin{itemize}
\item {Grp. gram.:m.}
\end{itemize}
\begin{itemize}
\item {Proveniência:(Do gr. \textunderscore lithos\textunderscore  + \textunderscore graphein\textunderscore )}
\end{itemize}
Aquelle que imprime ou desenha lithographicamente.
\section{Lithoide}
\begin{itemize}
\item {Grp. gram.:adj.}
\end{itemize}
\begin{itemize}
\item {Proveniência:(Do gr. \textunderscore lithos\textunderscore  + \textunderscore eidos\textunderscore )}
\end{itemize}
Que tem o carácter ou a apparência da pedra.
\section{Litholábio}
\begin{itemize}
\item {Grp. gram.:m.}
\end{itemize}
\begin{itemize}
\item {Proveniência:(Do gr. \textunderscore lithos\textunderscore  + \textunderscore tabein\textunderscore )}
\end{itemize}
Instrumento cirúrgico, com que se póde aprehender um cálculo urinário na bexiga.
\section{Lithólatra}
\begin{itemize}
\item {Grp. gram.:m.}
\end{itemize}
\begin{itemize}
\item {Proveniência:(Do gr. \textunderscore lithos\textunderscore  + \textunderscore latrein\textunderscore )}
\end{itemize}
Aquelle que adora a pedra.
\section{Litholatria}
\begin{itemize}
\item {Grp. gram.:f.}
\end{itemize}
\begin{itemize}
\item {Proveniência:(De \textunderscore lithólatra\textunderscore )}
\end{itemize}
Culto da pedra.
\section{Lithologia}
\begin{itemize}
\item {Grp. gram.:f.}
\end{itemize}
\begin{itemize}
\item {Proveniência:(Do gr. \textunderscore lithos\textunderscore  + \textunderscore logos\textunderscore )}
\end{itemize}
Tratado das massas rochosas, consideradas como seres especiaes e em relação á sua fórma geral e á sua disposição no globo terrestre.
\section{Lithologista}
\begin{itemize}
\item {Grp. gram.:m.}
\end{itemize}
Naturalista, que se occupa de lithologia.
\section{Lithólogo}
\begin{itemize}
\item {Grp. gram.:m.}
\end{itemize}
\begin{itemize}
\item {Proveniência:(Do gr. \textunderscore lithos\textunderscore  + \textunderscore logos\textunderscore )}
\end{itemize}
Aquelle que é perito em lithologia.
\section{Litholysia}
\begin{itemize}
\item {Grp. gram.:f.}
\end{itemize}
\begin{itemize}
\item {Proveniência:(Do gr. \textunderscore lithos\textunderscore  + \textunderscore lusis\textunderscore )}
\end{itemize}
Dissolução de cálculos vesicaes, por meio de substâncias injectadas.
\section{Lithomancia}
\begin{itemize}
\item {Grp. gram.:f.}
\end{itemize}
\begin{itemize}
\item {Proveniência:(Do gr. \textunderscore lithos\textunderscore  + \textunderscore manteia\textunderscore )}
\end{itemize}
Adivinhação, que os antigos suppunham tirar do som de certas pedras preciosas, atiradas umas contra outras.
\section{Lithotríptico}
\begin{itemize}
\item {Grp. gram.:adj.}
\end{itemize}
\begin{itemize}
\item {Proveniência:(Do gr. \textunderscore lithos\textunderscore  + \textunderscore tribein\textunderscore )}
\end{itemize}
Diz-se das substâncias, que se suppunham capazes de partir ou dissolver os cálculos urinários.
\section{Lithómetro}
\begin{itemize}
\item {Grp. gram.:m.}
\end{itemize}
\begin{itemize}
\item {Proveniência:(Do gr. \textunderscore lithos\textunderscore  + \textunderscore metron\textunderscore )}
\end{itemize}
Instrumento, para medir pedra.
\section{Lithóphago}
\begin{itemize}
\item {Grp. gram.:adj.}
\end{itemize}
\begin{itemize}
\item {Proveniência:(Do gr. \textunderscore lithos\textunderscore  + \textunderscore phagein\textunderscore )}
\end{itemize}
Diz-se dos molluscos que, introduzindo-se nos rochedos, alli permanecem adherentes ás superfícies pétreas.
\section{Lithophania}
\begin{itemize}
\item {Grp. gram.:f.}
\end{itemize}
\begin{itemize}
\item {Proveniência:(Do gr. \textunderscore lithos\textunderscore  + \textunderscore phano\textunderscore )}
\end{itemize}
Processo berlinês de produzir desenhos em placas de porcelana não esmaltada.
\section{Lithóphilo}
\begin{itemize}
\item {Grp. gram.:adj.}
\end{itemize}
\begin{itemize}
\item {Grp. gram.:M. pl.}
\end{itemize}
\begin{itemize}
\item {Proveniência:(Do gr. \textunderscore lithos\textunderscore  + \textunderscore philos\textunderscore )}
\end{itemize}
Que cresce nos rochedos.
Gênero de insectos coleópteros.
\section{Lithophyllo}
\begin{itemize}
\item {Grp. gram.:m.}
\end{itemize}
\begin{itemize}
\item {Proveniência:(Do gr. \textunderscore lithos\textunderscore  + \textunderscore phullon\textunderscore )}
\end{itemize}
Fôlha fóssil.
\section{Lithóphyto}
\begin{itemize}
\item {Grp. gram.:m.}
\end{itemize}
\begin{itemize}
\item {Proveniência:(Do gr. \textunderscore lithos\textunderscore  + \textunderscore phuton\textunderscore )}
\end{itemize}
Polypeiro pedregoso.
\section{Lithória}
\begin{itemize}
\item {Grp. gram.:f.}
\end{itemize}
Gênero de reptis batrácios.
\section{Lithorina}
\begin{itemize}
\item {Grp. gram.:f.}
\end{itemize}
Gênero de molluscos.
\section{Lithoscópio}
\begin{itemize}
\item {Grp. gram.:m.}
\end{itemize}
\begin{itemize}
\item {Utilização:Med.}
\end{itemize}
\begin{itemize}
\item {Proveniência:(Do gr. \textunderscore lithos\textunderscore  + \textunderscore skopein\textunderscore )}
\end{itemize}
Apparelho, para observar os cálculos da bexiga.
\section{Lithospermo}
\begin{itemize}
\item {Grp. gram.:adj.}
\end{itemize}
\begin{itemize}
\item {Grp. gram.:M.}
\end{itemize}
Que tem sementes duras e pedregosas.
Designação scientífica da erva-das-sete-sangrias; diospyro.
\section{Lithosphera}
\begin{itemize}
\item {Grp. gram.:f.}
\end{itemize}
\begin{itemize}
\item {Proveniência:(Do gr. \textunderscore lithos\textunderscore  + \textunderscore sphaira\textunderscore )}
\end{itemize}
A parte sólida do globo terrestre.
\section{Lithostroto}
\begin{itemize}
\item {Grp. gram.:m.}
\end{itemize}
\begin{itemize}
\item {Proveniência:(Gr. \textunderscore lithostroton\textunderscore )}
\end{itemize}
Pavimento de pedras variegadas, formando mosaico.
\section{Lithotomia}
\begin{itemize}
\item {Grp. gram.:f.}
\end{itemize}
Operação, com que se extrahem cálculos urinários, por meio de uma incisão no collo ou nas paredes da bexiga.
Antigamente, operação para partir cálculos urinários, feita uma incisão na bexiga.
(Cp. \textunderscore lithótomo\textunderscore )
\section{Lithotomista}
\begin{itemize}
\item {Grp. gram.:m.}
\end{itemize}
Cirurgião, que se dedica especialmente á pratica da lithotomia.
\section{Lithótomo}
\begin{itemize}
\item {Grp. gram.:m.}
\end{itemize}
\begin{itemize}
\item {Proveniência:(Do gr. \textunderscore lithos\textunderscore  + \textunderscore temnein\textunderscore )}
\end{itemize}
Antigamente, instrumento com que se partiam os cálculos urinários, depois de aberta a bexiga.
Hoje, instrumento, com que se faz a incisão da bexiga, para extrahir os cálculos urinários.
\section{Lithotrícia}
\begin{itemize}
\item {Grp. gram.:f.}
\end{itemize}
\begin{itemize}
\item {Proveniência:(T. hybr., do gr. \textunderscore lithos\textunderscore  + lat. \textunderscore tritus\textunderscore , de \textunderscore terere\textunderscore )}
\end{itemize}
Operação cirúrgica, com que se partem na bexiga ou na urethra cálculos urinários.
\section{Lithotripsia}
\begin{itemize}
\item {Grp. gram.:f.}
\end{itemize}
\begin{itemize}
\item {Proveniência:(Do gr. \textunderscore lithos\textunderscore  + \textunderscore tripsis\textunderscore )}
\end{itemize}
O mesmo que \textunderscore lithotrícia\textunderscore .
\section{Lithotritor}
\begin{itemize}
\item {Grp. gram.:m.}
\end{itemize}
\begin{itemize}
\item {Proveniência:(Do gr. \textunderscore lithos\textunderscore  + lat. \textunderscore tritor\textunderscore )}
\end{itemize}
Apparelho, com que se faz a lithotrícia.
\section{Lithotypographia}
\begin{itemize}
\item {Grp. gram.:f.}
\end{itemize}
\begin{itemize}
\item {Proveniência:(De \textunderscore litho...\textunderscore  + \textunderscore typographia\textunderscore )}
\end{itemize}
Arte de reproduzir, por meio da pedra litográphica, um impresso ou uma gravura.
\section{Lithóxylo}
\begin{itemize}
\item {fónica:csi}
\end{itemize}
\begin{itemize}
\item {Grp. gram.:m.}
\end{itemize}
\begin{itemize}
\item {Proveniência:(Do gr. \textunderscore lithos\textunderscore  + \textunderscore xulon\textunderscore )}
\end{itemize}
Vegetal, que se transformou em sílex, ágata ou outra pedra.
\section{Lithuânico}
\begin{itemize}
\item {Grp. gram.:m.}
\end{itemize}
Língua dos Lithuanos; o lithuano. Cf. Latino, \textunderscore Elogios\textunderscore , 73 e 89.
\section{Lithuânio}
\begin{itemize}
\item {Grp. gram.:m.}
\end{itemize}
O mesmo que \textunderscore lithuano\textunderscore .
\section{Lithuano}
\begin{itemize}
\item {Grp. gram.:m.}
\end{itemize}
Habitante da Lithuânia.
Lingua dos Lithuanos.
\section{Lithuria}
\begin{itemize}
\item {Grp. gram.:f.}
\end{itemize}
\begin{itemize}
\item {Utilização:Med.}
\end{itemize}
\begin{itemize}
\item {Proveniência:(Do gr. \textunderscore lithos\textunderscore  + \textunderscore ouron\textunderscore )}
\end{itemize}
Doença das areias ou cálculos na bexiga.
\section{Lítia}
\begin{itemize}
\item {Grp. gram.:f.}
\end{itemize}
O mesmo que \textunderscore litina\textunderscore .
\section{Litícine}
\begin{itemize}
\item {Grp. gram.:m.}
\end{itemize}
\begin{itemize}
\item {Proveniência:(Lat. \textunderscore liticen\textunderscore )}
\end{itemize}
Tocador de lítuo.
\section{Lítico}
\begin{itemize}
\item {Grp. gram.:adj.}
\end{itemize}
\begin{itemize}
\item {Proveniência:(Do gr. \textunderscore lithos\textunderscore )}
\end{itemize}
Relativo a pedra.
Dizia-se de um ácido, que hoje se chama \textunderscore úrico\textunderscore .
\section{Litigação}
\begin{itemize}
\item {Grp. gram.:f.}
\end{itemize}
\begin{itemize}
\item {Proveniência:(Lat. \textunderscore litigatio\textunderscore )}
\end{itemize}
Acto de litigar.
\section{Litigante}
\begin{itemize}
\item {Grp. gram.:adj.}
\end{itemize}
\begin{itemize}
\item {Grp. gram.:M.}
\end{itemize}
\begin{itemize}
\item {Proveniência:(Lat. \textunderscore litigans\textunderscore )}
\end{itemize}
Relativo a litígio.
Que litiga.
Aquelle que litiga.
\section{Litigar}
\begin{itemize}
\item {Grp. gram.:v. i.}
\end{itemize}
\begin{itemize}
\item {Grp. gram.:V. t.}
\end{itemize}
\begin{itemize}
\item {Proveniência:(Lat. \textunderscore litigare\textunderscore )}
\end{itemize}
Têr litígio, demanda, questões.
Pleitear; contestar. Cf. Filinto, VI, 110.
\section{Litigável}
\begin{itemize}
\item {Grp. gram.:adj.}
\end{itemize}
\begin{itemize}
\item {Proveniência:(De \textunderscore litigar\textunderscore )}
\end{itemize}
Sôbre que póde haver litígio.
Discutível; contestável. Cf. Camillo, \textunderscore Estrêl. Fun.\textunderscore , 21.
\section{Litigiar}
\begin{itemize}
\item {Grp. gram.:v. t.}
\end{itemize}
Tornar litigioso.
Fazer litígio sôbre; pleitear:«\textunderscore esclarecer o nascimento de António seria litigiar-lhe o dote.\textunderscore »Camillo, \textunderscore Caveira\textunderscore , 335.
\section{Litígio}
\begin{itemize}
\item {Grp. gram.:m.}
\end{itemize}
\begin{itemize}
\item {Proveniência:(Lat. \textunderscore litigium\textunderscore )}
\end{itemize}
Questão.
Pendência; demanda judicial.
\section{Litigiosamente}
\begin{itemize}
\item {Grp. gram.:adv.}
\end{itemize}
De modo litigioso.
Por meio de litígio.
\section{Litigioso}
\begin{itemize}
\item {Grp. gram.:adj.}
\end{itemize}
\begin{itemize}
\item {Proveniência:(Lat. \textunderscore litigiosus\textunderscore )}
\end{itemize}
Que é objecto de litígio; relativo a litígio.
Inclinado a demandas.
\section{Litina}
\begin{itemize}
\item {Grp. gram.:f.}
\end{itemize}
Óxido de lítio.
\section{Litinado}
\begin{itemize}
\item {Grp. gram.:adj.}
\end{itemize}
Em que há litina; que contém litina: \textunderscore visto que as águas dos Cucos são as mais litinadas de Portugal...\textunderscore 
\section{Litínico}
\begin{itemize}
\item {Grp. gram.:adj.}
\end{itemize}
Relativo á litina.
\section{Litinífero}
\begin{itemize}
\item {Grp. gram.:adj.}
\end{itemize}
\begin{itemize}
\item {Proveniência:(De \textunderscore lithina\textunderscore  + lat. \textunderscore ferre\textunderscore )}
\end{itemize}
Que contém litina.
\section{Lítio}
\begin{itemize}
\item {Grp. gram.:m.}
\end{itemize}
\begin{itemize}
\item {Proveniência:(Do gr. \textunderscore lithos\textunderscore )}
\end{itemize}
Metal branco e dúctil, que constitue a base da litina.
\section{Litisconsorte}
\begin{itemize}
\item {Grp. gram.:m.  e  f.}
\end{itemize}
\begin{itemize}
\item {Proveniência:(Do lat. \textunderscore lis\textunderscore , \textunderscore litis\textunderscore  + \textunderscore consors\textunderscore , \textunderscore consortis\textunderscore )}
\end{itemize}
Pessôa que, juntamente com outra, demanda alguém ou é parte em juízo.
\section{Litispendência}
\begin{itemize}
\item {Grp. gram.:f.}
\end{itemize}
\begin{itemize}
\item {Proveniência:(Do lat. \textunderscore lis\textunderscore , \textunderscore litis\textunderscore  + \textunderscore pendere\textunderscore )}
\end{itemize}
Decurso de um processo judicial.
Tempo que dura um processo.
\section{Litizonte}
\begin{itemize}
\item {Grp. gram.:m.}
\end{itemize}
\begin{itemize}
\item {Proveniência:(Do gr. \textunderscore lithizon\textunderscore )}
\end{itemize}
Pedra preciosa da India, espécie de granada.
\section{Lito...}
\begin{itemize}
\item {Grp. gram.:pref.}
\end{itemize}
\begin{itemize}
\item {Proveniência:(Do gr. \textunderscore lithos\textunderscore )}
\end{itemize}
(designativo de \textunderscore pedra\textunderscore )
\section{Litocálamo}
\begin{itemize}
\item {Grp. gram.:m.}
\end{itemize}
\begin{itemize}
\item {Proveniência:(Do gr. \textunderscore lithos\textunderscore  + \textunderscore kalamos\textunderscore )}
\end{itemize}
Haste fóssil de cana.
\section{Litocarpo}
\begin{itemize}
\item {Grp. gram.:m.}
\end{itemize}
\begin{itemize}
\item {Proveniência:(Do gr. \textunderscore lithos\textunderscore  + \textunderscore karpos\textunderscore )}
\end{itemize}
Fruto fóssil.
\section{Litocromia}
\begin{itemize}
\item {Grp. gram.:f.}
\end{itemize}
\begin{itemize}
\item {Proveniência:(Do gr. \textunderscore lithos\textunderscore  + \textunderscore khroma\textunderscore )}
\end{itemize}
Imitação da pintura a óleo por meio da litografia.
\section{Litocrómico}
\begin{itemize}
\item {Grp. gram.:adj.}
\end{itemize}
Relativo á litocromia.
\section{Litocromista}
\begin{itemize}
\item {Grp. gram.:m.}
\end{itemize}
Aquele que trabalha em litocromia.
\section{Litoclase}
\begin{itemize}
\item {Grp. gram.:f.}
\end{itemize}
\begin{itemize}
\item {Proveniência:(Do gr. \textunderscore lithos\textunderscore  + \textunderscore klao\textunderscore )}
\end{itemize}
Fractura natural de rocha.
\section{Litoclastia}
\begin{itemize}
\item {Grp. gram.:f.}
\end{itemize}
\begin{itemize}
\item {Proveniência:(De \textunderscore litoclasto\textunderscore )}
\end{itemize}
Processo de reduzir a fragmentos os cálculos da bexiga.
\section{Litoclasto}
\begin{itemize}
\item {Grp. gram.:m.}
\end{itemize}
\begin{itemize}
\item {Proveniência:(Do gr. \textunderscore lithos\textunderscore  + \textunderscore klao\textunderscore )}
\end{itemize}
Instrumento cirúrgico, empregado em litoclastia.
\section{Litocola}
\begin{itemize}
\item {Grp. gram.:f.}
\end{itemize}
\begin{itemize}
\item {Proveniência:(De \textunderscore litho\textunderscore  + \textunderscore colla\textunderscore )}
\end{itemize}
Espécie de betume, em que entra pó de pedra e que serve para soldar pedras.
\section{Litodendro}
\begin{itemize}
\item {Grp. gram.:m.}
\end{itemize}
\begin{itemize}
\item {Proveniência:(Do gr. \textunderscore lithos\textunderscore  + \textunderscore dendron\textunderscore )}
\end{itemize}
Polipo fóssil.
\section{Litodonte}
\begin{itemize}
\item {Grp. gram.:m.}
\end{itemize}
\begin{itemize}
\item {Proveniência:(Do gr. \textunderscore lithos\textunderscore  + \textunderscore odous\textunderscore , \textunderscore odontos\textunderscore )}
\end{itemize}
Gênero de moluscos, de sabor apimentado.
\section{Litodiálise}
\begin{itemize}
\item {Grp. gram.:f.}
\end{itemize}
\begin{itemize}
\item {Proveniência:(De \textunderscore lito...\textunderscore  + \textunderscore diálise\textunderscore )}
\end{itemize}
Qualquer processo de dissolver os cálculos vesicaes.
\section{Litófago}
\begin{itemize}
\item {Grp. gram.:adj.}
\end{itemize}
\begin{itemize}
\item {Proveniência:(Do gr. \textunderscore lithos\textunderscore  + \textunderscore phagein\textunderscore )}
\end{itemize}
Diz-se dos moluscos que, introduzindo-se nos rochedos, ali permanecem aderentes ás superfícies pétreas.
\section{Litofania}
\begin{itemize}
\item {Grp. gram.:f.}
\end{itemize}
\begin{itemize}
\item {Proveniência:(Do gr. \textunderscore lithos\textunderscore  + \textunderscore phano\textunderscore )}
\end{itemize}
Processo berlinês de produzir desenhos em placas de porcelana não esmaltada.
\section{Litofélico}
\begin{itemize}
\item {Grp. gram.:adj.}
\end{itemize}
\begin{itemize}
\item {Proveniência:(Do gr. \textunderscore lithos\textunderscore  + lat. \textunderscore fel\textunderscore )}
\end{itemize}
Diz-se de um ácido, que se encontra nos bezoares orientaes.
\section{Litofelínico}
\begin{itemize}
\item {Grp. gram.:adj.}
\end{itemize}
O mesmo que \textunderscore litofélico\textunderscore .
\section{Litófilo}
\begin{itemize}
\item {Grp. gram.:adj.}
\end{itemize}
\begin{itemize}
\item {Grp. gram.:M. pl.}
\end{itemize}
\begin{itemize}
\item {Proveniência:(Do gr. \textunderscore lithos\textunderscore  + \textunderscore philos\textunderscore )}
\end{itemize}
Que cresce nos rochedos.
Gênero de insectos coleópteros.
\section{Litofilo}
\begin{itemize}
\item {Grp. gram.:m.}
\end{itemize}
\begin{itemize}
\item {Proveniência:(Do gr. \textunderscore lithos\textunderscore  + \textunderscore phullon\textunderscore )}
\end{itemize}
Fôlha fóssil.
\section{Litófito}
\begin{itemize}
\item {Grp. gram.:m.}
\end{itemize}
\begin{itemize}
\item {Proveniência:(Do gr. \textunderscore lithos\textunderscore  + \textunderscore phuton\textunderscore )}
\end{itemize}
Polipeiro pedregoso.
\section{Litogenesia}
\begin{itemize}
\item {Grp. gram.:f.}
\end{itemize}
\begin{itemize}
\item {Proveniência:(Do gr. \textunderscore lithos\textunderscore  + \textunderscore genesis\textunderscore )}
\end{itemize}
Investigação das leis que presidem á formação das pedras.
\section{Litogenésico}
\begin{itemize}
\item {Grp. gram.:adj.}
\end{itemize}
Relativo a litogenesia.
\section{Litoglifia}
\begin{itemize}
\item {Grp. gram.:f.}
\end{itemize}
\begin{itemize}
\item {Proveniência:(De \textunderscore litóglifo\textunderscore )}
\end{itemize}
Arte de gravar sôbre pedra.
\section{Litoglífico}
\begin{itemize}
\item {Grp. gram.:adj.}
\end{itemize}
Relativo á litoglifia.
\section{Litoglifo}
\begin{itemize}
\item {Grp. gram.:m.}
\end{itemize}
\begin{itemize}
\item {Proveniência:(Do gr. \textunderscore lithos\textunderscore  + \textunderscore glyphein\textunderscore )}
\end{itemize}
Que pratíca a litoglifia.
\section{Litografar}
\begin{itemize}
\item {Grp. gram.:v. t.}
\end{itemize}
\begin{itemize}
\item {Proveniência:(De \textunderscore litógrafo\textunderscore )}
\end{itemize}
Imprimir pelo processo litográfico.
\section{Litografia}
\begin{itemize}
\item {Grp. gram.:f.}
\end{itemize}
\begin{itemize}
\item {Proveniência:(De \textunderscore litógrafo\textunderscore )}
\end{itemize}
Processo para reproduzir em papel, por meio de impressão, aquilo que estava desenhado ou escrito sôbre uma pedra especial.
Oficina de litógrafo.
Fôlha ou estampa, impressa litograficamente.
\section{Litográfico}
\begin{itemize}
\item {Grp. gram.:adj.}
\end{itemize}
Relativo á litografia.
Diz-se de uma espécie de pedra, que é calcário compacto, de grão fino homogêneo, e que se emprega em litografia.
\section{Litógrafo}
\begin{itemize}
\item {Grp. gram.:m.}
\end{itemize}
\begin{itemize}
\item {Proveniência:(Do gr. \textunderscore lithos\textunderscore  + \textunderscore graphein\textunderscore )}
\end{itemize}
Aquele que imprime ou desenha litograficamente.
\section{Litoide}
\begin{itemize}
\item {Grp. gram.:adj.}
\end{itemize}
\begin{itemize}
\item {Proveniência:(Do gr. \textunderscore lithos\textunderscore  + \textunderscore eidos\textunderscore )}
\end{itemize}
Que tem o carácter ou a aparência da pedra.
\section{Litolábio}
\begin{itemize}
\item {Grp. gram.:m.}
\end{itemize}
\begin{itemize}
\item {Proveniência:(Do gr. \textunderscore lithos\textunderscore  + \textunderscore tabein\textunderscore )}
\end{itemize}
Instrumento cirúrgico, com que se póde aprehender um cálculo urinário na bexiga.
\section{Litólatra}
\begin{itemize}
\item {Grp. gram.:m.}
\end{itemize}
\begin{itemize}
\item {Proveniência:(Do gr. \textunderscore lithos\textunderscore  + \textunderscore latrein\textunderscore )}
\end{itemize}
Aquele que adora a pedra.
\section{Litolatria}
\begin{itemize}
\item {Grp. gram.:f.}
\end{itemize}
\begin{itemize}
\item {Proveniência:(De \textunderscore litólatra\textunderscore )}
\end{itemize}
Culto da pedra.
\section{Litolisia}
\begin{itemize}
\item {Grp. gram.:f.}
\end{itemize}
\begin{itemize}
\item {Proveniência:(Do gr. \textunderscore lithos\textunderscore  + \textunderscore lusis\textunderscore )}
\end{itemize}
Dissolução de cálculos vesicaes, por meio de substâncias injectadas.
\section{Litologia}
\begin{itemize}
\item {Grp. gram.:f.}
\end{itemize}
\begin{itemize}
\item {Proveniência:(Do gr. \textunderscore lithos\textunderscore  + \textunderscore logos\textunderscore )}
\end{itemize}
Tratado das massas rochosas, consideradas como seres especiaes e em relação á sua fórma geral e á sua disposição no globo terrestre.
\section{Litologista}
\begin{itemize}
\item {Grp. gram.:m.}
\end{itemize}
Naturalista, que se ocupa de litologia.
\section{Litólogo}
\begin{itemize}
\item {Grp. gram.:m.}
\end{itemize}
\begin{itemize}
\item {Proveniência:(Do gr. \textunderscore lithos\textunderscore  + \textunderscore logos\textunderscore )}
\end{itemize}
Aquele que é perito em litologia.
\section{Litomancia}
\begin{itemize}
\item {Grp. gram.:f.}
\end{itemize}
\begin{itemize}
\item {Proveniência:(Do gr. \textunderscore lithos\textunderscore  + \textunderscore manteia\textunderscore )}
\end{itemize}
Adivinhação, que os antigos supunham tirar do som de certas pedras preciosas, atiradas umas contra outras.
\section{Litómetro}
\begin{itemize}
\item {Grp. gram.:m.}
\end{itemize}
\begin{itemize}
\item {Proveniência:(Do gr. \textunderscore lithos\textunderscore  + \textunderscore metron\textunderscore )}
\end{itemize}
Instrumento, para medir pedra.
\section{Litonde}
\begin{itemize}
\item {Grp. gram.:m.}
\end{itemize}
Árvore intertropical, da fam. das artocárpeas, (\textunderscore ficus elastica\textunderscore ).
\section{Litontríptico}
\begin{itemize}
\item {Grp. gram.:adj.}
\end{itemize}
\begin{itemize}
\item {Proveniência:(Do gr. \textunderscore lithos\textunderscore  + \textunderscore tribein\textunderscore )}
\end{itemize}
Diz-se das substâncias, que se supunham capazes de partir ou dissolver os cálculos urinários.
\section{Litoral}
\begin{itemize}
\item {Grp. gram.:adj.}
\end{itemize}
\begin{itemize}
\item {Grp. gram.:M.}
\end{itemize}
\begin{itemize}
\item {Proveniência:(Lat. \textunderscore litoralis\textunderscore )}
\end{itemize}
Relativo á beira-mar.
Terreno banhado pelo mar ou situado á beira do mar.
\section{Litóreo}
\begin{itemize}
\item {Grp. gram.:adj.}
\end{itemize}
\begin{itemize}
\item {Utilização:Poét.}
\end{itemize}
\begin{itemize}
\item {Proveniência:(Lat. \textunderscore litoreus\textunderscore )}
\end{itemize}
O mesmo que \textunderscore litoral\textunderscore .
\section{Litória}
\begin{itemize}
\item {Grp. gram.:f.}
\end{itemize}
Gênero de reptis batrácios.
\section{Litorina}
\begin{itemize}
\item {Grp. gram.:f.}
\end{itemize}
\begin{itemize}
\item {Proveniência:(Do lat. \textunderscore litus\textunderscore , \textunderscore litoris\textunderscore )}
\end{itemize}
Gênero de molluscos gasterópodes.
\section{Litoscópio}
\begin{itemize}
\item {Grp. gram.:m.}
\end{itemize}
\begin{itemize}
\item {Utilização:Med.}
\end{itemize}
\begin{itemize}
\item {Proveniência:(Do gr. \textunderscore lithos\textunderscore  + \textunderscore skopein\textunderscore )}
\end{itemize}
Aparelho, para observar os cálculos da bexiga.
\section{Litosfera}
\begin{itemize}
\item {Grp. gram.:f.}
\end{itemize}
\begin{itemize}
\item {Proveniência:(Do gr. \textunderscore lithos\textunderscore  + \textunderscore sphaira\textunderscore )}
\end{itemize}
A parte sólida do globo terrestre.
\section{Litospermo}
\begin{itemize}
\item {Grp. gram.:adj.}
\end{itemize}
\begin{itemize}
\item {Grp. gram.:M.}
\end{itemize}
Que tem sementes duras e pedregosas.
Designação científica da erva-das-sete-sangrias; diospiro.
\section{Litostroto}
\begin{itemize}
\item {Grp. gram.:m.}
\end{itemize}
\begin{itemize}
\item {Proveniência:(Gr. \textunderscore lithostroton\textunderscore )}
\end{itemize}
Pavimento de pedras variegadas, formando mosaico.
\section{Litotes}
\begin{itemize}
\item {Grp. gram.:f.}
\end{itemize}
\begin{itemize}
\item {Proveniência:(Gr. \textunderscore litotes\textunderscore , de \textunderscore litos\textunderscore , pequeno)}
\end{itemize}
Emprêgo figurado de uma expressão, que diz pouco para dar a entender mais.
\section{Litotipografia}
\begin{itemize}
\item {Grp. gram.:f.}
\end{itemize}
\begin{itemize}
\item {Proveniência:(De \textunderscore lito...\textunderscore  + \textunderscore tipografia\textunderscore )}
\end{itemize}
Arte de reproduzir, por meio da pedra litográfica, um impresso ou uma gravura.
\section{Litotomia}
\begin{itemize}
\item {Grp. gram.:f.}
\end{itemize}
Operação, com que se extrahem cálculos urinários, por meio de uma incisão no colo ou nas paredes da bexiga.
Antigamente, operação para partir cálculos urinários, feita uma incisão na bexiga.
(Cp. \textunderscore litótomo\textunderscore )
\section{Litotomista}
\begin{itemize}
\item {Grp. gram.:m.}
\end{itemize}
Cirurgião, que se dedica especialmente á pratica da litotomia.
\section{Litótomo}
\begin{itemize}
\item {Grp. gram.:m.}
\end{itemize}
\begin{itemize}
\item {Proveniência:(Do gr. \textunderscore lithos\textunderscore  + \textunderscore temnein\textunderscore )}
\end{itemize}
Antigamente, instrumento com que se partiam os cálculos urinários, depois de aberta a bexiga.
Hoje, instrumento, com que se faz a incisão da bexiga, para extrair os cálculos urinários.
\section{Litotrícia}
\begin{itemize}
\item {Grp. gram.:f.}
\end{itemize}
\begin{itemize}
\item {Proveniência:(T. hybr., do gr. \textunderscore lithos\textunderscore  + lat. \textunderscore tritus\textunderscore , de \textunderscore terere\textunderscore )}
\end{itemize}
Operação cirúrgica, com que se partem na bexiga ou na uretra cálculos urinários.
\section{Litotripsia}
\begin{itemize}
\item {Grp. gram.:f.}
\end{itemize}
\begin{itemize}
\item {Proveniência:(Do gr. \textunderscore lithos\textunderscore  + \textunderscore tripsis\textunderscore )}
\end{itemize}
O mesmo que \textunderscore litotrícia\textunderscore .
\section{Litotritor}
\begin{itemize}
\item {Grp. gram.:m.}
\end{itemize}
\begin{itemize}
\item {Proveniência:(Do gr. \textunderscore lithos\textunderscore  + lat. \textunderscore tritor\textunderscore )}
\end{itemize}
Aparelho, com que se faz a litotrícia.
\section{Litóxilo}
\begin{itemize}
\item {Grp. gram.:m.}
\end{itemize}
\begin{itemize}
\item {Proveniência:(Do gr. \textunderscore lithos\textunderscore  + \textunderscore xulon\textunderscore )}
\end{itemize}
Vegetal, que se transformou em sílex, ágata ou outra pedra.
\section{Litro}
\begin{itemize}
\item {Grp. gram.:m.}
\end{itemize}
Unidade das medidas de capacidade, correspondente ao espaço de um centímetro cúbico.
(B. lat. \textunderscore litra\textunderscore )
\section{Littreísta}
\begin{itemize}
\item {Grp. gram.:m.}
\end{itemize}
Positivista, partidário de Littré. Cf. Camillo, \textunderscore Narcót.\textunderscore , I, 188.
\section{Lituânico}
\begin{itemize}
\item {Grp. gram.:m.}
\end{itemize}
Língua dos Lituanos; o lituano. Cf. Latino, \textunderscore Elogios\textunderscore , 73 e 89.
\section{Lituânio}
\begin{itemize}
\item {Grp. gram.:m.}
\end{itemize}
O mesmo que \textunderscore lituano\textunderscore .
\section{Lituano}
\begin{itemize}
\item {Grp. gram.:m.}
\end{itemize}
Habitante da Lituânia.
Lingua dos Lituanos.
\section{Lítuo}
\begin{itemize}
\item {Grp. gram.:m.}
\end{itemize}
\begin{itemize}
\item {Proveniência:(Lat. \textunderscore lituus\textunderscore )}
\end{itemize}
Bastão, recurvado na extremidade superior e usado pelos áugures.
Instrumento de sôpro, espécie de clarim, recurvado como aquelle bastão, e que, nas guerras dos Romanos, servia para dar o sinal de combate.
\section{Litura}
\begin{itemize}
\item {Grp. gram.:f.}
\end{itemize}
\begin{itemize}
\item {Proveniência:(Lat. \textunderscore litura\textunderscore )}
\end{itemize}
Parte de um escrito, tornada illegível por se haver expungido ou riscado.
\section{Liturgia}
\begin{itemize}
\item {Grp. gram.:f.}
\end{itemize}
\begin{itemize}
\item {Proveniência:(Gr. \textunderscore leitourgia\textunderscore )}
\end{itemize}
Complexo das ceremónias ecclesiásticas.
Rito.
\section{Liturgicamente}
\begin{itemize}
\item {Grp. gram.:adv.}
\end{itemize}
Segundo os preceitos litúrgicos.
\section{Litúrgico}
\begin{itemize}
\item {Grp. gram.:adj.}
\end{itemize}
Relativo á liturgia.
\section{Liturgista}
\begin{itemize}
\item {Grp. gram.:m.}
\end{itemize}
Aquelle que é versado em liturgia.
\section{Liturgo}
\begin{itemize}
\item {Grp. gram.:m.}
\end{itemize}
\begin{itemize}
\item {Proveniência:(Lat. \textunderscore liturgus\textunderscore )}
\end{itemize}
Escravo publico ou escravo do Estado, entre, os antigos Romanos.
\section{Lituria}
\begin{itemize}
\item {Grp. gram.:f.}
\end{itemize}
\begin{itemize}
\item {Utilização:Med.}
\end{itemize}
\begin{itemize}
\item {Proveniência:(Do gr. \textunderscore lithos\textunderscore  + \textunderscore ouron\textunderscore )}
\end{itemize}
Doença das areias ou cálculos na bexiga.
\section{Lível}
\begin{itemize}
\item {Grp. gram.:m.}
\end{itemize}
\begin{itemize}
\item {Proveniência:(Do lat. \textunderscore libellum\textunderscore )}
\end{itemize}
O mesmo ou melhor que \textunderscore nível\textunderscore .
\section{Livél}
\begin{itemize}
\item {Grp. gram.:m.}
\end{itemize}
\begin{itemize}
\item {Proveniência:(Do lat. \textunderscore libellum\textunderscore )}
\end{itemize}
O mesmo ou melhor que \textunderscore nível\textunderscore .
\section{Livelar}
\textunderscore v. t.\textunderscore  (e der.)
O mesmo ou melhor que \textunderscore nivelar\textunderscore :«\textunderscore todas as condições se livelavam onde elle apparecia\textunderscore ». Herculano, \textunderscore Eurico\textunderscore .
\section{Lívia}
\begin{itemize}
\item {Grp. gram.:f.}
\end{itemize}
Gênero de insectos hemípteros.
\section{Lívico}
\begin{itemize}
\item {Grp. gram.:m.}
\end{itemize}
Língua uralo-altaica, vernácula na Rússia.
\section{Lividez}
\begin{itemize}
\item {Grp. gram.:f.}
\end{itemize}
Qualidade ou estado daquelle ou daquillo que é lívido.
\section{Lívido}
\begin{itemize}
\item {Grp. gram.:adj.}
\end{itemize}
\begin{itemize}
\item {Proveniência:(Lat. \textunderscore lividus\textunderscore )}
\end{itemize}
Que tem côr de chumbo, entre negro e azul.
Que tem côr cadavérica.
Extremamente pállido.
\section{Livistona}
\begin{itemize}
\item {Grp. gram.:f.}
\end{itemize}
Espécie de palmeira.
\section{Livoniano}
\begin{itemize}
\item {Grp. gram.:adj.}
\end{itemize}
\begin{itemize}
\item {Grp. gram.:M.}
\end{itemize}
Relativo á Livónia.
Habitante da Livónia.
\section{Livónio}
\begin{itemize}
\item {Grp. gram.:m.}
\end{itemize}
Língua uralo-altaica, do grupo ugro-finlandês.
\section{Livor}
\begin{itemize}
\item {Grp. gram.:m.}
\end{itemize}
\begin{itemize}
\item {Proveniência:(Lat. \textunderscore livor\textunderscore )}
\end{itemize}
O mesmo que \textunderscore lividez\textunderscore .
\section{Livra}
\begin{itemize}
\item {Grp. gram.:f.}
\end{itemize}
\begin{itemize}
\item {Utilização:Des.}
\end{itemize}
O mesmo que \textunderscore libra\textunderscore .
\section{Livração}
\begin{itemize}
\item {Grp. gram.:f.}
\end{itemize}
\begin{itemize}
\item {Utilização:Des.}
\end{itemize}
O mesmo que \textunderscore livramento\textunderscore .
\section{Livrador}
\begin{itemize}
\item {Grp. gram.:adj.}
\end{itemize}
\begin{itemize}
\item {Grp. gram.:M.}
\end{itemize}
Que livra.
Aquelle que livra; libertador.
\section{Livralhada}
\begin{itemize}
\item {Grp. gram.:f.}
\end{itemize}
\begin{itemize}
\item {Utilização:Fam.}
\end{itemize}
Montão de livros; porção de livros.
\section{Livramento}
\begin{itemize}
\item {Grp. gram.:m.}
\end{itemize}
Acto ou effeito de livrar.
\section{Livrança}
\begin{itemize}
\item {Grp. gram.:f.}
\end{itemize}
\begin{itemize}
\item {Proveniência:(De \textunderscore livrar\textunderscore )}
\end{itemize}
Livramento.
Cédula ou ordem escrita, para pagamento.
\section{Livrar}
\begin{itemize}
\item {Grp. gram.:v. t.}
\end{itemize}
\begin{itemize}
\item {Utilização:Prov.}
\end{itemize}
\begin{itemize}
\item {Utilização:minh.}
\end{itemize}
\begin{itemize}
\item {Utilização:Ant.}
\end{itemize}
\begin{itemize}
\item {Grp. gram.:V. p.}
\end{itemize}
\begin{itemize}
\item {Utilização:Prov.}
\end{itemize}
\begin{itemize}
\item {Utilização:minh.}
\end{itemize}
\begin{itemize}
\item {Proveniência:(Do lat. \textunderscore liberare\textunderscore )}
\end{itemize}
Dar liberdade a; fazer livre.
Salvar; preservar: \textunderscore livrar da peste\textunderscore .
Defender.
Tirar de difficuldades.
Desembaraçar, despejar (uma coisa).
Resolver, decidir.
Dar á luz, parir. Cp. \textunderscore delivramento\textunderscore .
\section{Livraria}
\begin{itemize}
\item {Grp. gram.:f.}
\end{itemize}
\begin{itemize}
\item {Utilização:Pop.}
\end{itemize}
\begin{itemize}
\item {Utilização:Gír.}
\end{itemize}
Reunião de livros, dispostos ordenadamente.
Bibliotheca.
Estabelecimento, em que se vendem livros.
Collecção de obras de certos autores ou sôbre determinado assumpto.
Grande porção de livros.
Repertório de cantigas.
\section{Livre}
\begin{itemize}
\item {Grp. gram.:adj.}
\end{itemize}
\begin{itemize}
\item {Grp. gram.:Adv.}
\end{itemize}
\begin{itemize}
\item {Proveniência:(Do lat. \textunderscore liber\textunderscore )}
\end{itemize}
Que tem liberdade.
Que dispõe de si; independente.
Solto.
Absolvido: \textunderscore o réu ficou livre\textunderscore .
Desembaraçado: \textunderscore livre de cuidados\textunderscore .
Franqueado.
Que não está preso pelos laços do matrimónio.
Licencioso.
Dissoluto.
Com liberdade.
\section{Livre-câmbio}
\begin{itemize}
\item {Grp. gram.:m.}
\end{itemize}
Permutação internacional de valores, productos ou mercadorias, independentemente de impostos aduaneiros.
\section{Livre-cambismo}
\begin{itemize}
\item {Grp. gram.:m.}
\end{itemize}
Systema dos que preconizam o livre-câmbio.
\section{Livre-cambista}
\begin{itemize}
\item {Grp. gram.:m.}
\end{itemize}
Partidário do livre-câmbio.
\section{Livreco}
\begin{itemize}
\item {Grp. gram.:m.}
\end{itemize}
\begin{itemize}
\item {Utilização:Deprec.}
\end{itemize}
\begin{itemize}
\item {Proveniência:(Do rad. de \textunderscore livro\textunderscore )}
\end{itemize}
Pequeno livro; livro reles.
\section{Livre-cultista}
\begin{itemize}
\item {Grp. gram.:m.}
\end{itemize}
Partidário do livre-cultismo.
\section{Livre-cultismo}
\begin{itemize}
\item {Grp. gram.:m.}
\end{itemize}
Systema ou doutrina da liberdade de cultos.
\section{Livre-culto}
\begin{itemize}
\item {Grp. gram.:m.}
\end{itemize}
O mesmo que \textunderscore livre-cultismo\textunderscore .
\section{Livreiro}
\begin{itemize}
\item {Grp. gram.:m.}
\end{itemize}
Negociante de livros.
\section{Livremente}
\begin{itemize}
\item {Grp. gram.:adv.}
\end{itemize}
De modo livre.
Espontaneamente; sem coacção.
\section{Livre-roda}
\begin{itemize}
\item {Grp. gram.:f.}
\end{itemize}
Roda da bicycleta, que póde continuar a mover-se, ainda que os pedaes estejam parados.
\section{Livre-rodagem}
\begin{itemize}
\item {Grp. gram.:f.}
\end{itemize}
Acto de rodar a bicycleta, conservando-se parados os pedaes.
\section{Livre-rodar}
\begin{itemize}
\item {Grp. gram.:m.}
\end{itemize}
O mesmo que \textunderscore livre-rodagem\textunderscore .
\section{Livrete}
\begin{itemize}
\item {fónica:vrê}
\end{itemize}
\begin{itemize}
\item {Grp. gram.:m.}
\end{itemize}
Pequeno livro; caderno; caderneta; registo.
\section{Livreto}
\begin{itemize}
\item {fónica:vrê}
\end{itemize}
\begin{itemize}
\item {Grp. gram.:m.}
\end{itemize}
O mesmo ou melhor que \textunderscore libreto\textunderscore . Cf. Rui Barb., \textunderscore Réplica\textunderscore , 157.
\section{Livre-troca}
\begin{itemize}
\item {Grp. gram.:f.}
\end{itemize}
O mesmo que \textunderscore livre-câmbio\textunderscore .
\section{Livricho}
\begin{itemize}
\item {Grp. gram.:m.}
\end{itemize}
O mesmo que \textunderscore livrilho\textunderscore .
\section{Livrilho}
\begin{itemize}
\item {Grp. gram.:m.}
\end{itemize}
\begin{itemize}
\item {Proveniência:(Do lat. \textunderscore liber\textunderscore )}
\end{itemize}
A mais interior da casca dos vegetes, adherente ao alburno.
\section{Livro}
\begin{itemize}
\item {Grp. gram.:m.}
\end{itemize}
\begin{itemize}
\item {Utilização:Fig.}
\end{itemize}
\begin{itemize}
\item {Grp. gram.:Loc.}
\end{itemize}
\begin{itemize}
\item {Utilização:pop.}
\end{itemize}
\begin{itemize}
\item {Proveniência:(Do lat. \textunderscore liber\textunderscore )}
\end{itemize}
Reunião de cadernos manuscritos ou impressos e cosidos ordenadamente.
Composição literária ou scientífica, mais extensa que um folheto e constituindo volume.
Aquillo que ensina ou instrue como um livro.
Um dos estômagos dos ruminantes.
Reunião de estampas, cosidas, formando volume.
\textunderscore Livro de quarenta fôlhas\textunderscore , baralho de cartas.
\section{Livrório}
\begin{itemize}
\item {Grp. gram.:m.}
\end{itemize}
\begin{itemize}
\item {Utilização:Deprec.}
\end{itemize}
Livro grande, mas de pouco merecimento; cartapácio. Cf. Camillo, \textunderscore Noites de Insómn.\textunderscore , IX, 65.
\section{Livruxada}
\begin{itemize}
\item {Grp. gram.:f.}
\end{itemize}
\begin{itemize}
\item {Utilização:Burl.}
\end{itemize}
\begin{itemize}
\item {Proveniência:(Do rad. de \textunderscore livro\textunderscore )}
\end{itemize}
Grande porção de livros; livralhada.
\section{Lixa}
\begin{itemize}
\item {Grp. gram.:f.}
\end{itemize}
\begin{itemize}
\item {Utilização:Ext.}
\end{itemize}
\begin{itemize}
\item {Utilização:Bras. do Pará}
\end{itemize}
\begin{itemize}
\item {Proveniência:(Do cast. \textunderscore lija\textunderscore )}
\end{itemize}
Peixe do gênero esqualo, cuja pelle escabrosa serve para polir madeira, metaes, etc.
A pelle dêste peixe.
Papel que tem adherente uma camada de areia, e que serve para polir madeira ou metal.
Simbaiba.
Variedade de maçan doce, de pelle áspera e escura.
\section{Lixação}
\begin{itemize}
\item {Grp. gram.:f.}
\end{itemize}
\begin{itemize}
\item {Proveniência:(Lat. \textunderscore lixatio\textunderscore )}
\end{itemize}
O mesmo que \textunderscore lixiviação\textunderscore .
\section{Lixa-de-lei}
\begin{itemize}
\item {Grp. gram.:f.}
\end{itemize}
(V. \textunderscore barroso\textunderscore , peixe)
\section{Lixa-de-pau}
\begin{itemize}
\item {Grp. gram.:f.}
\end{itemize}
Peixe plagióstomo, negro-violáceo, de pequenas escamas e focinho curto.
\section{Lixar}
\begin{itemize}
\item {Grp. gram.:v. t.}
\end{itemize}
\begin{itemize}
\item {Grp. gram.:V. p.}
\end{itemize}
\begin{itemize}
\item {Utilização:Gír.}
\end{itemize}
Polir ou desgastar com lixa.
Têr cóito.
\section{Lixeiro}
\begin{itemize}
\item {Grp. gram.:m.}
\end{itemize}
\begin{itemize}
\item {Utilização:Bras}
\end{itemize}
Indivíduo, encarregado da conducção do lixo.
\section{Lixento}
\begin{itemize}
\item {Grp. gram.:adj.}
\end{itemize}
O mesmo que \textunderscore lixoso\textunderscore .
\section{Lixívia}
\begin{itemize}
\item {Grp. gram.:f.}
\end{itemize}
\begin{itemize}
\item {Proveniência:(Lat. \textunderscore lixivia\textunderscore )}
\end{itemize}
Água em que se ferve cinza, e que serve para lavagem de roupa; barrela.
\section{Lixiviação}
\begin{itemize}
\item {Grp. gram.:f.}
\end{itemize}
Acto ou effeito de lixiviar.
Operação chímica, que, por meio de lavagem, separa de certas substâncias os saes nella contidos.
\section{Lixiviador}
\begin{itemize}
\item {Grp. gram.:m.}
\end{itemize}
Apparelho para lixiviar. Cf. \textunderscore Inquér. Industr.\textunderscore , parte I, 233.
\section{Lixiviar}
\begin{itemize}
\item {Grp. gram.:v. t.}
\end{itemize}
\begin{itemize}
\item {Proveniência:(De \textunderscore lixívia\textunderscore )}
\end{itemize}
Applicar barreira a.
Separar, por lavagem os saes de.
\section{Lixívio}
\begin{itemize}
\item {Grp. gram.:adj.}
\end{itemize}
\begin{itemize}
\item {Utilização:Des.}
\end{itemize}
O mesmo que \textunderscore lixivioso\textunderscore .
\section{Lixivioso}
\begin{itemize}
\item {Grp. gram.:adj.}
\end{itemize}
Semelhante á lixívia.
\section{Lixo}
\begin{itemize}
\item {Grp. gram.:m.}
\end{itemize}
\begin{itemize}
\item {Utilização:Fig.}
\end{itemize}
\begin{itemize}
\item {Proveniência:(Do lat. \textunderscore lix\textunderscore  ou \textunderscore lixa\textunderscore )}
\end{itemize}
Aquilo que se varre, para tornar limpa uma casa, um móvel, qualquer objecto.
Sujidade; immundície.
Ralé.
\section{Lixoso}
\begin{itemize}
\item {Grp. gram.:adj.}
\end{itemize}
Que tem lixo.
\section{Lizar}
\begin{itemize}
\item {Grp. gram.:v. t.}
\end{itemize}
Voltar num banho de tinta (qualquer tecido ou meada)
\section{Lizo}
\textunderscore adj.\textunderscore  (e der.)
(V. \textunderscore liso\textunderscore , etc.)--\textunderscore Lizo\textunderscore  seria preferível, se a etym. do voc. fôsse o latim \textunderscore licium\textunderscore , como alguns suppõem.
\section{Ló}
\begin{itemize}
\item {Grp. gram.:m.}
\end{itemize}
\begin{itemize}
\item {Utilização:Náut.}
\end{itemize}
\begin{itemize}
\item {Utilização:Ant.}
\end{itemize}
\begin{itemize}
\item {Proveniência:(Do ingl. \textunderscore loaf\textunderscore )}
\end{itemize}
O lado do vento.
Parte em que se amuram as velas do navio.
\textunderscore Navegar de ló\textunderscore , navegar contra o vento, quási á bolina. Cf. \textunderscore Roteiro do Mar-Vermelho\textunderscore , (passim).
Espécie de escumilha.
\textunderscore Pão de ló\textunderscore , pão fofo de farinha, açúcar e ovos.
\section{Lo}
\begin{itemize}
\item {fónica:lu}
\end{itemize}
\begin{itemize}
\item {Grp. gram.:pron.}
\end{itemize}
\begin{itemize}
\item {Grp. gram.:Art.}
\end{itemize}
\begin{itemize}
\item {Utilização:mad}
\end{itemize}
\begin{itemize}
\item {Utilização:Ant.}
\end{itemize}
Que, em vêz de \textunderscore o\textunderscore , se appõe encliticamente ás pessôas verbaes de terminação em \textunderscore r\textunderscore , \textunderscore s\textunderscore  ou \textunderscore z\textunderscore , determinando a quéda dessas consoantes:«\textunderscore tu applaude-lo; elle fê-lo; vais vesti-lo;\textunderscore ».
O mesmo que \textunderscore o\textunderscore ^2.
\section{Lôa}
\begin{itemize}
\item {Grp. gram.:f.}
\end{itemize}
\begin{itemize}
\item {Utilização:Ant.}
\end{itemize}
\begin{itemize}
\item {Utilização:Prov.}
\end{itemize}
\begin{itemize}
\item {Proveniência:(De \textunderscore loar\textunderscore )}
\end{itemize}
Apologia.
Discurso em louvor de alguém.
Elogio.
Composição poética em louvor da Virgem ou dos santos.
Prólogo de uma composição dramática.
Mentira; parlenda: \textunderscore deixa-te de lôas\textunderscore .
\section{Loador}
\begin{itemize}
\item {Grp. gram.:m.}
\end{itemize}
\begin{itemize}
\item {Utilização:Ant.}
\end{itemize}
\begin{itemize}
\item {Proveniência:(De \textunderscore loar\textunderscore )}
\end{itemize}
Aquelle que louva; lisonjeador. Cf. \textunderscore Cancion. da Vaticana\textunderscore .
\section{Loangos}
\begin{itemize}
\item {Grp. gram.:m. pl.}
\end{itemize}
Cacongos do território francês, ao norte do Congo português.
\section{Loante}
\begin{itemize}
\item {Grp. gram.:m.}
\end{itemize}
\begin{itemize}
\item {Utilização:Des.}
\end{itemize}
\begin{itemize}
\item {Proveniência:(De \textunderscore loar\textunderscore )}
\end{itemize}
Aquelle que louva, aquelle que faz lôas. Cf. Garrett, \textunderscore Viagens\textunderscore , I, 43.
\section{Loão}
\begin{itemize}
\item {Grp. gram.:m.}
\end{itemize}
Espécie de planta ou erva. Cf. B. Pereira, \textunderscore Prosódia\textunderscore , vb. \textunderscore anacaebra\textunderscore .
\section{Loar}
\begin{itemize}
\item {Grp. gram.:v. t.}
\end{itemize}
\begin{itemize}
\item {Utilização:Ant.}
\end{itemize}
O mesmo que \textunderscore louvar\textunderscore :«\textunderscore Dixeron tudo a llorar: Loemos santa maria...\textunderscore »\textunderscore Cancion. da Vaticana\textunderscore .
\section{Loasa}
\begin{itemize}
\item {Grp. gram.:f.}
\end{itemize}
\begin{itemize}
\item {Proveniência:(T. inventado arbitrariamente por Adanson)}
\end{itemize}
Gênero de plantas trepadeiras, de pêlos ásperos e urticantes.
\section{Loáseas}
\begin{itemize}
\item {Grp. gram.:f. pl.}
\end{itemize}
Fam. de plantas trepadeiras, que tem por typo a \textunderscore loasa\textunderscore .
\section{Lôba}
\begin{itemize}
\item {Grp. gram.:f.}
\end{itemize}
\begin{itemize}
\item {Utilização:Des.}
\end{itemize}
\begin{itemize}
\item {Proveniência:(Lat. \textunderscore lupa\textunderscore )}
\end{itemize}
Fêmea do lôbo.
O mesmo que \textunderscore prostituta\textunderscore . Cf. Camões, \textunderscore Odes\textunderscore .
\section{Lôba}
\begin{itemize}
\item {Grp. gram.:f.}
\end{itemize}
\begin{itemize}
\item {Proveniência:(Do fr. \textunderscore l'aube\textunderscore )}
\end{itemize}
Batina ecclesiástica.
\section{Lôba}
\begin{itemize}
\item {Grp. gram.:f.}
\end{itemize}
\begin{itemize}
\item {Utilização:Prov.}
\end{itemize}
\begin{itemize}
\item {Utilização:alent.}
\end{itemize}
Terreno, junto das oliveiras, que tem de sêr cavado na occasião das lavras, por não dever alli chegar a charrua ou o arado.
\section{Lóba}
\begin{itemize}
\item {Grp. gram.:f.}
\end{itemize}
Tumor, o mesmo que \textunderscore antecoração\textunderscore .
(Cp. \textunderscore lóbo\textunderscore )
\section{Lobacho}
\begin{itemize}
\item {Grp. gram.:m.}
\end{itemize}
Pequeno lôbo.
\section{Lobado}
\begin{itemize}
\item {Grp. gram.:adj.}
\end{itemize}
\begin{itemize}
\item {Grp. gram.:M.}
\end{itemize}
Dividido em lóbos ou lóbulos.
O mesmo que \textunderscore antecoração\textunderscore .
\section{Lobagante}
\begin{itemize}
\item {Grp. gram.:m.}
\end{itemize}
Crustáceo decápode, marítimo, um pouco mais pequeno que a lagosta e munido de duas fortes torqueses nos braços (\textunderscore homarus vulgaris\textunderscore ).
(Cp. cast. \textunderscore lobogante\textunderscore )
\section{Lobal}
\begin{itemize}
\item {Grp. gram.:adj.}
\end{itemize}
\begin{itemize}
\item {Utilização:Fig.}
\end{itemize}
Relativo a lôbo: \textunderscore sanha lobal\textunderscore .
Sanguinário.
\section{Lobão}
\begin{itemize}
\item {Grp. gram.:m.}
\end{itemize}
\begin{itemize}
\item {Proveniência:(De \textunderscore lóbo\textunderscore )}
\end{itemize}
Tumor no peito dos cavallos.
\section{Lobato}
\begin{itemize}
\item {Grp. gram.:m.}
\end{itemize}
O mesmo que \textunderscore lobacho\textunderscore .
(Cf. cast. \textunderscore lobato\textunderscore )
\section{Lobaz}
\begin{itemize}
\item {Grp. gram.:m.}
\end{itemize}
\begin{itemize}
\item {Proveniência:(Do lat. \textunderscore lupax\textunderscore )}
\end{itemize}
Grande lôbo.
\section{Lobecão}
\begin{itemize}
\item {Grp. gram.:m.}
\end{itemize}
\begin{itemize}
\item {Proveniência:(De \textunderscore lobo\textunderscore  + \textunderscore cão\textunderscore )}
\end{itemize}
Animal da raça do cão e lobo.
\section{Lobeiro}
\begin{itemize}
\item {Grp. gram.:adj.}
\end{itemize}
\begin{itemize}
\item {Utilização:Prov.}
\end{itemize}
\begin{itemize}
\item {Grp. gram.:M.}
\end{itemize}
\begin{itemize}
\item {Proveniência:(Do lat. \textunderscore luparius\textunderscore )}
\end{itemize}
Que é bom caçador de lôbos: \textunderscore cão lobeiro\textunderscore .
Lobal.
Diz-se do cavallo, cujos pêlos são amarelos na base e pretos na ponta.
Diz-se da manta, para cama ou para agasalho, de côres variadas e listrada.
Caçador de lôbos.
\section{Lobeiro}
\begin{itemize}
\item {Grp. gram.:adj.}
\end{itemize}
Diz se de uma espécie de trigo rijo.
\section{Lobeiro}
\begin{itemize}
\item {Grp. gram.:adj.}
\end{itemize}
\begin{itemize}
\item {Utilização:Prov.}
\end{itemize}
\begin{itemize}
\item {Utilização:alg.}
\end{itemize}
Favorável; agradável.
\section{Lobélia}
\begin{itemize}
\item {Grp. gram.:f.}
\end{itemize}
\begin{itemize}
\item {Proveniência:(De \textunderscore Lobel\textunderscore , n. p.)}
\end{itemize}
Gênero de plantas herbáceas, que contém um suco cáustico e venenoso, (\textunderscore lobelia syphilitica\textunderscore , Lin.).
\section{Lobeliáceas}
\begin{itemize}
\item {Grp. gram.:f. pl.}
\end{itemize}
\begin{itemize}
\item {Proveniência:(De \textunderscore lobeliáceo\textunderscore )}
\end{itemize}
Família de plantas, que tem por typo a lobélia.
\section{Lobeliáceo}
\begin{itemize}
\item {Grp. gram.:adj.}
\end{itemize}
Relativo ou semelhante á lobélia.
\section{Lobelina}
\begin{itemize}
\item {Grp. gram.:f.}
\end{itemize}
Alcaloide, descoberto na lobélia.
\section{Lobélio}
\begin{itemize}
\item {Grp. gram.:m.}
\end{itemize}
Segmento de círculo, inscrito em certas ogivas, formando festão ou uma fôlha, traçada por vários círculos que se interceptam.
(Relaciona-se com \textunderscore lobélia\textunderscore ?)
\section{Lobeno}
\begin{itemize}
\item {Grp. gram.:adj.}
\end{itemize}
\begin{itemize}
\item {Utilização:Ant.}
\end{itemize}
Relativo a lobo.
Lobal.
\textunderscore Manto lobeno\textunderscore , manto para agasalho.
(Cp. cast. \textunderscore lobezno\textunderscore )
\section{Lobete}
\begin{itemize}
\item {fónica:bê}
\end{itemize}
\begin{itemize}
\item {Grp. gram.:m.}
\end{itemize}
\begin{itemize}
\item {Utilização:Prov.}
\end{itemize}
\begin{itemize}
\item {Utilização:minh.}
\end{itemize}
O mesmo que \textunderscore lobeto\textunderscore .
\section{Lobeto}
\begin{itemize}
\item {fónica:bê}
\end{itemize}
\begin{itemize}
\item {Grp. gram.:m.}
\end{itemize}
Peça de ferro que, num moínho, está ligada ao eixo da pedra e encaixa no rodízio.
\section{Lobinho}
\begin{itemize}
\item {Grp. gram.:m.}
\end{itemize}
Pequeno lobo.
\section{Lobinho}
\begin{itemize}
\item {Grp. gram.:m.}
\end{itemize}
\begin{itemize}
\item {Proveniência:(De \textunderscore lóbo\textunderscore )}
\end{itemize}
Cisto sub-cutâneo.
\section{Lobishomem}
\begin{itemize}
\item {Grp. gram.:m.}
\end{itemize}
Homem que, segundo a crendice popular, se transforma em lobo ou em outro animal.
(Tem-se supposto que o vocábulo é composto do \textunderscore lobo\textunderscore  e \textunderscore homem\textunderscore , mas não se explica a sýllaba \textunderscore bis\textunderscore )
\section{Lóbo}
\begin{itemize}
\item {Grp. gram.:m.}
\end{itemize}
\begin{itemize}
\item {Proveniência:(Gr. \textunderscore lobos\textunderscore )}
\end{itemize}
Parte arredondada e saliente de um órgão.
Espécie de jôgo popular.
\section{Lôbo}
\begin{itemize}
\item {Grp. gram.:m.}
\end{itemize}
\begin{itemize}
\item {Utilização:Fig.}
\end{itemize}
\begin{itemize}
\item {Grp. gram.:Loc.}
\end{itemize}
\begin{itemize}
\item {Utilização:fam.}
\end{itemize}
\begin{itemize}
\item {Proveniência:(Do lat. \textunderscore lupus\textunderscore )}
\end{itemize}
Animal selvagem e carnívoro, do gênero cão, (\textunderscore canis lupus\textunderscore ).
Constellação austral.
Máquina, composta do um tambor, guarnecido de grossos dentes de ferro ou aço, e que serve para abrir a lan nas fábricas de lanifícios.
Homem sanguinário.
\textunderscore Meter-se na bôca do lôbo\textunderscore , caír em arriosca.
\section{Lôbo-cerval}
\begin{itemize}
\item {Grp. gram.:m.}
\end{itemize}
Nome vulgar do lynce de Portugal, (\textunderscore lynx guardiana\textunderscore ).
\section{Lobogante}
\begin{itemize}
\item {Grp. gram.:m.}
\end{itemize}
Crustáceo decápode, marítimo, um pouco mais pequeno que a lagosta e munido de duas fortes torqueses nos braços (\textunderscore homarus vulgaris\textunderscore ).
(Cp. cast. \textunderscore lobogante\textunderscore )
\section{Lôbo-lôba}
\begin{itemize}
\item {Grp. gram.:f.}
\end{itemize}
Planta brasileira.
\section{Lobo-marinho}
\begin{itemize}
\item {Grp. gram.:m.}
\end{itemize}
Nome vulgar da phoca.
\section{Lôbo-tigre}
\begin{itemize}
\item {Grp. gram.:m.}
\end{itemize}
Quadrúpede felino da Ásia e da África, um pouco menor que a panthera.
\section{Lobregar}
\begin{itemize}
\item {Grp. gram.:v. t.}
\end{itemize}
Tornar lôbrego; escurecer. Cf. Arn. Gama, \textunderscore Motim\textunderscore , 389.
\section{Lôbrego}
\begin{itemize}
\item {Grp. gram.:adj.}
\end{itemize}
Medonho.
Escuro.
Cavernoso.
(Corr. de \textunderscore lúgubre\textunderscore , com metáth.)
\section{Lobrigador}
\begin{itemize}
\item {Grp. gram.:m.  e  adj.}
\end{itemize}
O que lobriga.
\section{Lobrigar}
\begin{itemize}
\item {Grp. gram.:v. t.}
\end{itemize}
\begin{itemize}
\item {Proveniência:(Do lat. \textunderscore lubricare\textunderscore ?)}
\end{itemize}
Vêr a custo; entrever.
Vêr ao longe.
Vêr casualmente.
\section{Lobulado}
\begin{itemize}
\item {Grp. gram.:adj.}
\end{itemize}
\begin{itemize}
\item {Proveniência:(De \textunderscore lóbulo\textunderscore )}
\end{itemize}
O mesmo que \textunderscore lobado\textunderscore .
\section{Lobular}
\begin{itemize}
\item {Grp. gram.:adj.}
\end{itemize}
Lobado; que tem a natureza do lóbulo.
\section{Lóbulo}
\begin{itemize}
\item {Grp. gram.:m.}
\end{itemize}
\begin{itemize}
\item {Utilização:Bot.}
\end{itemize}
\begin{itemize}
\item {Proveniência:(De \textunderscore lóbo\textunderscore )}
\end{itemize}
Pequeno lóbo.
Rudimento de folha que, nas plantas monocotyledóneas, se desenvolve ás vezes em sentido opposto ao cotylédone.
\section{Lobuloso}
\begin{itemize}
\item {Grp. gram.:adj.}
\end{itemize}
Que tem lóbulos; dividido em lóbulos.
\section{Lobuno}
\begin{itemize}
\item {Grp. gram.:adj.}
\end{itemize}
\begin{itemize}
\item {Utilização:Bras. do S}
\end{itemize}
\begin{itemize}
\item {Proveniência:(Do rad. de \textunderscore lobo\textunderscore )}
\end{itemize}
Diz-se do cavallo, que tem a côr do lôbo.
\section{Loca}
\begin{itemize}
\item {Grp. gram.:f.}
\end{itemize}
\begin{itemize}
\item {Proveniência:(Do lat. \textunderscore locus\textunderscore ?)}
\end{itemize}
Esconderijo do peixe sob uma laja, debaixo de água.
Toca, furna.
\section{Locação}
\begin{itemize}
\item {Grp. gram.:f.}
\end{itemize}
\begin{itemize}
\item {Utilização:Bras}
\end{itemize}
\begin{itemize}
\item {Proveniência:(Lat. \textunderscore locatio\textunderscore )}
\end{itemize}
Aluguer; arrendamento.
Installação, collocação. Cf. \textunderscore País\textunderscore , do Rio, de 9-V-901.
\section{Locador}
\begin{itemize}
\item {Grp. gram.:m.}
\end{itemize}
\begin{itemize}
\item {Proveniência:(Do lat. \textunderscore locator\textunderscore )}
\end{itemize}
Aquelle que dá de aluguer ou de arrendamento.
\section{Locafa}
\begin{itemize}
\item {Grp. gram.:f.}
\end{itemize}
\begin{itemize}
\item {Utilização:Ant.}
\end{itemize}
\begin{itemize}
\item {Proveniência:(Do ár. \textunderscore lacaha\textunderscore )}
\end{itemize}
Ajuntamento de pessôas.
\section{Locaia}
\begin{itemize}
\item {Grp. gram.:f.}
\end{itemize}
Variedade de uva minhota.
Alvarelhão.
\section{Local}
\begin{itemize}
\item {Grp. gram.:adj.}
\end{itemize}
\begin{itemize}
\item {Grp. gram.:M.}
\end{itemize}
\begin{itemize}
\item {Grp. gram.:F.}
\end{itemize}
\begin{itemize}
\item {Proveniência:(Lat. \textunderscore localis\textunderscore )}
\end{itemize}
Relativo a lugar determinado: \textunderscore interesses locaes\textunderscore .
Lugar.
Ponto ou sítio, relativo a um acontecimento: \textunderscore no local do suicídio\textunderscore .
Notícia, dada por um periódico, relativa á localidade em que êste se publica.
\section{Localidade}
\begin{itemize}
\item {Grp. gram.:f.}
\end{itemize}
\begin{itemize}
\item {Proveniência:(Lat. \textunderscore localitas\textunderscore )}
\end{itemize}
Espaço limitado ou determinado.
Povoação.
\section{Localismo}
\begin{itemize}
\item {Grp. gram.:m.}
\end{itemize}
\begin{itemize}
\item {Proveniência:(De \textunderscore local\textunderscore )}
\end{itemize}
Defesa systemática de interesses locaes.
Paixão pelas conveniências de uma localidade; bairrismo. Cf. Oliv. Martins, \textunderscore Camões\textunderscore , 128.
\section{Localista}
\begin{itemize}
\item {Grp. gram.:m.}
\end{itemize}
\begin{itemize}
\item {Proveniência:(De \textunderscore local\textunderscore )}
\end{itemize}
Redactor da secção noticiosa de um jornal.
Aquelle que publica as noticias de uma localidade.
\section{Localização}
\begin{itemize}
\item {Grp. gram.:f.}
\end{itemize}
Acto ou effeito de localizar.
\section{Localizar}
\begin{itemize}
\item {Grp. gram.:v. t.}
\end{itemize}
Tornar local.
Pôr em lugar certo.
Imaginar como existente num lugar: \textunderscore a lenda localizou na Ásia o paraíso terreal\textunderscore .
\section{Localmente}
\begin{itemize}
\item {Grp. gram.:adv.}
\end{itemize}
De modo local.
\section{Locanda}
\begin{itemize}
\item {Grp. gram.:f.}
\end{itemize}
\begin{itemize}
\item {Proveniência:(Lat. \textunderscore locandas\textunderscore )}
\end{itemize}
Tasca; baiúca; taberna.
Tenda.
\section{Locandeira}
\begin{itemize}
\item {Grp. gram.:f.}
\end{itemize}
Mulher, que tem locanda.
Mulher de locandeiro.
\section{Locandeiro}
\begin{itemize}
\item {Grp. gram.:m.}
\end{itemize}
Aquelle que tem locanda.
Aquelle que tomou de arrendamento algum prédio; locatário.
\section{Loção}
\begin{itemize}
\item {Grp. gram.:f.}
\end{itemize}
\begin{itemize}
\item {Proveniência:(Do lat. \textunderscore lotio\textunderscore )}
\end{itemize}
Acto de lavar uma parte do corpo, passando por esta um pano ou uma esponja, embebida em líquido.
Lavagem, com que se separam de uma substância insolúvel as partes heterogêneas.
\section{Locar}
\begin{itemize}
\item {Grp. gram.:v. t.}
\end{itemize}
\begin{itemize}
\item {Utilização:Des.}
\end{itemize}
\begin{itemize}
\item {Proveniência:(Lat. \textunderscore locare\textunderscore )}
\end{itemize}
Dar de aluguer ou de arrendamento.
Meter em loca.
\section{Lóçara}
\begin{itemize}
\item {Grp. gram.:f.}
\end{itemize}
\begin{itemize}
\item {Utilização:Prov.}
\end{itemize}
\begin{itemize}
\item {Utilização:trasm.}
\end{itemize}
Burzigada.
Qualquer coisa espapaçada.
\section{Locário}
\begin{itemize}
\item {Grp. gram.:m.}
\end{itemize}
\begin{itemize}
\item {Proveniência:(Lat. \textunderscore locarius\textunderscore )}
\end{itemize}
Aquelle que, entre os Romanos, alugava as cadeiras dos espectáculos ou negociava as entradas nos theatros.
\section{Locatária}
\begin{itemize}
\item {Grp. gram.:f.}
\end{itemize}
Mulher, que tomou de aluguer qualquer coisa.
Mulher, que tomou de arrendamento um prédio.
Arrendatária; inquilina.
Mulher de locatário.
\section{Locatário}
\begin{itemize}
\item {Grp. gram.:m.}
\end{itemize}
\begin{itemize}
\item {Proveniência:(Lat. \textunderscore locatarius\textunderscore )}
\end{itemize}
Aquelle que tomou alguma coisa do aluguer.
Aquelle que tomou de arrendamento um prédio.
Inquilino; arrendatário.
\section{Locativo}
\begin{itemize}
\item {Grp. gram.:adj.}
\end{itemize}
\begin{itemize}
\item {Utilização:Gram.}
\end{itemize}
\begin{itemize}
\item {Proveniência:(Do lat. \textunderscore locatus\textunderscore )}
\end{itemize}
Resultante da locação.
Que exprime relação de lugar, (falando-se do caso de alguns nomes, no sanscr. e no lat.).
\section{Locello}
\begin{itemize}
\item {Grp. gram.:m.}
\end{itemize}
\begin{itemize}
\item {Utilização:Ant.}
\end{itemize}
\begin{itemize}
\item {Proveniência:(Lat. \textunderscore locellus\textunderscore )}
\end{itemize}
Lugarzinho.
Cova.
Sepulcro pequeno e humilde.
\section{Locelo}
\begin{itemize}
\item {Grp. gram.:m.}
\end{itemize}
\begin{itemize}
\item {Utilização:Ant.}
\end{itemize}
\begin{itemize}
\item {Proveniência:(Lat. \textunderscore locellus\textunderscore )}
\end{itemize}
Lugarzinho.
Cova.
Sepulcro pequeno e humilde.
\section{Locengue}
\begin{itemize}
\item {Grp. gram.:m.}
\end{itemize}
Nome africano do um reptil sáurio, (\textunderscore varanus\textunderscore ).
\section{Lochial}
\begin{itemize}
\item {fónica:qui}
\end{itemize}
\begin{itemize}
\item {Grp. gram.:adj.}
\end{itemize}
Relativo aos lóchios.
\section{Lóchios}
\begin{itemize}
\item {fónica:qui}
\end{itemize}
\begin{itemize}
\item {Grp. gram.:m. pl.}
\end{itemize}
\begin{itemize}
\item {Proveniência:(Do gr. \textunderscore lokheia\textunderscore )}
\end{itemize}
Evacuação sanguinolenta, que succede aos partos.
\section{Locho}
\begin{itemize}
\item {fónica:co}
\end{itemize}
\begin{itemize}
\item {Grp. gram.:m.}
\end{itemize}
\begin{itemize}
\item {Proveniência:(Gr. \textunderscore lokhos\textunderscore )}
\end{itemize}
Fila de 16 homens, unidade fundamental da phalange macedónica.
\section{Locionar}
\begin{itemize}
\item {Grp. gram.:v. t.}
\end{itemize}
\begin{itemize}
\item {Utilização:Neol.}
\end{itemize}
\begin{itemize}
\item {Proveniência:(Do lat. \textunderscore lotio\textunderscore )}
\end{itemize}
Applicar loção a.
\section{Loco}
\begin{itemize}
\item {fónica:lô}
\end{itemize}
\begin{itemize}
\item {Grp. gram.:m.}
\end{itemize}
Arbusto plumbagíneo do Brasil.
Alguns escrevem \textunderscore louco\textunderscore . Cf. B.C. Rubim, \textunderscore Voc. Bras.\textunderscore 
\section{Loco}
\begin{itemize}
\item {Grp. gram.:m.}
\end{itemize}
\begin{itemize}
\item {Proveniência:(Gr. \textunderscore lokhos\textunderscore )}
\end{itemize}
Fila de 16 homens, unidade fundamental da phalange macedónica.
\section{Locomobilidade}
\begin{itemize}
\item {Grp. gram.:f.}
\end{itemize}
Qualidade daquillo que é locomóvel.
\section{Locomoção}
\begin{itemize}
\item {Grp. gram.:f.}
\end{itemize}
\begin{itemize}
\item {Proveniência:(Do lat. \textunderscore locus\textunderscore  + \textunderscore motio\textunderscore )}
\end{itemize}
Acto ou effeito de transportar ou de se transportar, de um lugar para o outro.
\section{Locomotiva}
\begin{itemize}
\item {Grp. gram.:f.}
\end{itemize}
\begin{itemize}
\item {Proveniência:(De \textunderscore locomotivo\textunderscore )}
\end{itemize}
Máquina de vapor, que opéra a tracção das carruagens dos caminhos de ferro.
\section{Locomotival}
\begin{itemize}
\item {Grp. gram.:adj.}
\end{itemize}
\begin{itemize}
\item {Utilização:bras}
\end{itemize}
\begin{itemize}
\item {Utilização:Neol.}
\end{itemize}
Relativo a locomotiva.
\section{Locomotividade}
\begin{itemize}
\item {Grp. gram.:f.}
\end{itemize}
\begin{itemize}
\item {Proveniência:(De \textunderscore locomotivo\textunderscore )}
\end{itemize}
Qualidade da locomoção, inherente aos animaes.
\section{Locomotivo}
\begin{itemize}
\item {Grp. gram.:adj.}
\end{itemize}
\begin{itemize}
\item {Proveniência:(Do lat. \textunderscore locus\textunderscore  + \textunderscore motivus\textunderscore )}
\end{itemize}
Relativo a locomoção.
\section{Locomotor}
\begin{itemize}
\item {Grp. gram.:adj.}
\end{itemize}
\begin{itemize}
\item {Proveniência:(Do lat. \textunderscore locus\textunderscore  + \textunderscore motor\textunderscore )}
\end{itemize}
Que opéra a locomoção.
\section{Locomotora}
\begin{itemize}
\item {Grp. gram.:f.}
\end{itemize}
(Fórma talvez preferível a \textunderscore locomotiva\textunderscore )
\section{Locomotriz}
\begin{itemize}
\item {Grp. gram.:adj.}
\end{itemize}
(Flexão \textunderscore fem.\textunderscore  de \textunderscore locomotor\textunderscore )
\section{Locomóvel}
\begin{itemize}
\item {Grp. gram.:adj.}
\end{itemize}
\begin{itemize}
\item {Grp. gram.:F.}
\end{itemize}
\begin{itemize}
\item {Proveniência:(Do lat. \textunderscore locus\textunderscore  + \textunderscore mobilis\textunderscore )}
\end{itemize}
Que póde deslocar-se.
Máquina de vapor, montada sôbre rodas.
\section{Locomover-se}
\begin{itemize}
\item {Grp. gram.:v. p.}
\end{itemize}
\begin{itemize}
\item {Utilização:Neol.}
\end{itemize}
\begin{itemize}
\item {Proveniência:(Do lat. \textunderscore locus\textunderscore  + \textunderscore movere\textunderscore )}
\end{itemize}
Deslocar-se, passar de um ponto para outro.
\section{Locondi}
\begin{itemize}
\item {Grp. gram.:m.}
\end{itemize}
Arvoreta indiana.
\section{Loco-tenência}
\begin{itemize}
\item {Grp. gram.:f.}
\end{itemize}
O mesmo que \textunderscore lugar-tenência\textunderscore . Cf. Camillo, \textunderscore Mar. da Fonte\textunderscore , 233.
\section{Locotenente}
\begin{itemize}
\item {Grp. gram.:m.}
\end{itemize}
O mesmo que \textunderscore lugar-tenente\textunderscore .
\section{Locução}
\begin{itemize}
\item {Grp. gram.:f.}
\end{itemize}
\begin{itemize}
\item {Proveniência:(Do lat. \textunderscore locutio\textunderscore )}
\end{itemize}
Maneira especial de falar.
Elocução.
Linguagem.
Phrase.
Expressão.
Válvula, na parte superior do instrumento órgão.
\section{Loculado}
\begin{itemize}
\item {Grp. gram.:adj.}
\end{itemize}
Divididos em lóculos.
\section{Loculamento}
\begin{itemize}
\item {Grp. gram.:m.}
\end{itemize}
\begin{itemize}
\item {Proveniência:(Lat. \textunderscore loculamentum\textunderscore )}
\end{itemize}
O mesmo que \textunderscore lóculo\textunderscore .
\section{Locular}
\begin{itemize}
\item {Grp. gram.:adj.}
\end{itemize}
\begin{itemize}
\item {Utilização:Bot.}
\end{itemize}
\begin{itemize}
\item {Proveniência:(Lat. \textunderscore locularis\textunderscore )}
\end{itemize}
Que tem lóculos, separados por septos.
\section{Loculicida}
\begin{itemize}
\item {Grp. gram.:adj.}
\end{itemize}
\begin{itemize}
\item {Proveniência:(Do lat. \textunderscore loculus\textunderscore  + \textunderscore caedere\textunderscore )}
\end{itemize}
Diz-se do fruto, cujos septos se abrem pelo meio.
\section{Lóculo}
\begin{itemize}
\item {Grp. gram.:m.}
\end{itemize}
\begin{itemize}
\item {Proveniência:(Lat. \textunderscore loculus\textunderscore )}
\end{itemize}
Pequena cavidade.
\section{Loculoso}
\begin{itemize}
\item {Grp. gram.:adj.}
\end{itemize}
Que tem lóculos.
\section{Locupletar}
\begin{itemize}
\item {Grp. gram.:v. t.}
\end{itemize}
\begin{itemize}
\item {Proveniência:(Lat. \textunderscore locupletare\textunderscore )}
\end{itemize}
Tornar rico.
Saciar.
\section{Locurana}
\begin{itemize}
\item {Grp. gram.:f.}
\end{itemize}
\begin{itemize}
\item {Utilização:Bras}
\end{itemize}
Árvore silvestre, cuja madeira é empregada em construcções navaes.
\section{Locuri}
\begin{itemize}
\item {Grp. gram.:m.}
\end{itemize}
\begin{itemize}
\item {Utilização:Bras}
\end{itemize}
Árvore silvestre, cuja madeira se applica a frechaes e vigotas.
\section{Locusta}
\begin{itemize}
\item {Grp. gram.:f.}
\end{itemize}
\begin{itemize}
\item {Proveniência:(Lat. \textunderscore locusta\textunderscore )}
\end{itemize}
Nome scientífico do gafanhoto.
\section{Locusta}
\begin{itemize}
\item {Grp. gram.:f.}
\end{itemize}
Gênero de árvores da Guiana inglesa, (\textunderscore hymenoca combaril\textunderscore ).
\section{Locusta}
\begin{itemize}
\item {Grp. gram.:f.}
\end{itemize}
O mesmo que \textunderscore espigueta\textunderscore .
\section{Locustário}
\begin{itemize}
\item {Grp. gram.:adj.}
\end{itemize}
\begin{itemize}
\item {Grp. gram.:M. pl.}
\end{itemize}
Semelhante á locusta^1.
Família de insectos orthópteros, que tem por typo a locusta^1.
\section{Locusticida}
\begin{itemize}
\item {Grp. gram.:m.}
\end{itemize}
Solução concentrada de arseniato de soda e açúcar, destinada á destruição de gafanhotos. Cf. Rev. \textunderscore Portugal em África\textunderscore , XVII, 148.
\section{Locutório}
\begin{itemize}
\item {Grp. gram.:m.}
\end{itemize}
\begin{itemize}
\item {Proveniência:(Do lat. \textunderscore locutus\textunderscore )}
\end{itemize}
Lugar, separado por grades, através das quaes as religiosas ou pessôas recolhidas falam a quem as procura.
\section{Lodaçal}
\begin{itemize}
\item {Grp. gram.:m.}
\end{itemize}
\begin{itemize}
\item {Utilização:Fig.}
\end{itemize}
\begin{itemize}
\item {Proveniência:(Do rad. de \textunderscore lôdo\textunderscore )}
\end{itemize}
Lugar, em que há muito lôdo.
Atoleiro.
Vida desregrada.
Lugar aviltante.
\section{Lodacento}
\begin{itemize}
\item {Grp. gram.:adj.}
\end{itemize}
O mesmo que \textunderscore lodoso\textunderscore .
\section{Lodam}
\begin{itemize}
\item {Grp. gram.:m.}
\end{itemize}
\begin{itemize}
\item {Proveniência:(Do gr. \textunderscore lotos\textunderscore )}
\end{itemize}
Nome de várias plantas nympheáceas, especialmente da espécie nenúfar.
Árvore cuja madeira se applica em construcções, (\textunderscore celtis australis\textunderscore , Lin.).
\section{Lódão}
\begin{itemize}
\item {Grp. gram.:m.}
\end{itemize}
\begin{itemize}
\item {Proveniência:(Do gr. \textunderscore lotos\textunderscore )}
\end{itemize}
Nome de várias plantas nympheáceas, especialmente da espécie nenúfar.
Árvore cuja madeira se applica em construcções, (\textunderscore celtis australis\textunderscore , Lin.).
\section{Lodeira}
\begin{itemize}
\item {Grp. gram.:f.}
\end{itemize}
\begin{itemize}
\item {Proveniência:(De \textunderscore lôdo\textunderscore )}
\end{itemize}
Lameiro; terreno apaülado.
\section{Lodeiro}
\begin{itemize}
\item {Grp. gram.:m.}
\end{itemize}
Lugar, em que há muito lodo; atoleiro.
\section{Lodícula}
\begin{itemize}
\item {Grp. gram.:f.}
\end{itemize}
\begin{itemize}
\item {Proveniência:(Lat. \textunderscore lodicula\textunderscore )}
\end{itemize}
Invólucro interior da flôr das gramíneas.
\section{Lôdo}
\begin{itemize}
\item {Grp. gram.:m.}
\end{itemize}
\begin{itemize}
\item {Utilização:Fig.}
\end{itemize}
\begin{itemize}
\item {Utilização:Gír.}
\end{itemize}
\begin{itemize}
\item {Grp. gram.:Pl.}
\end{itemize}
\begin{itemize}
\item {Proveniência:(Do lat. \textunderscore lutum\textunderscore )}
\end{itemize}
Terra, misturada com restos ou detritos vegetaes ou animaes, no fundo das águas.
Lama.
Degradamento; baixeza.
Oiro.
O mesmo que [[lamas|lama:1]] das nascentes de águas mineraes.
\section{Lódo}
\begin{itemize}
\item {Grp. gram.:m.}
\end{itemize}
O mesmo que \textunderscore lódão\textunderscore .
\section{Lodoso}
\begin{itemize}
\item {Grp. gram.:adj.}
\end{itemize}
Que tem lodo.
Enlameado; sujo.
\section{Loduso}
\begin{itemize}
\item {Grp. gram.:m.}
\end{itemize}
\begin{itemize}
\item {Utilização:Gír.}
\end{itemize}
\begin{itemize}
\item {Proveniência:(De \textunderscore lôdo\textunderscore )}
\end{itemize}
Ourives.
\section{Loenda}
\begin{itemize}
\item {Grp. gram.:f.}
\end{itemize}
\begin{itemize}
\item {Utilização:Ant.}
\end{itemize}
O mesmo que \textunderscore louvor\textunderscore .
Composição poética, de carácter religioso, na Idade-Média.
(Cp. \textunderscore loar\textunderscore )
\section{Loendral}
\begin{itemize}
\item {Grp. gram.:m.}
\end{itemize}
Lugar, onde crescem loendros.
\section{Loendreira}
\begin{itemize}
\item {Grp. gram.:f.}
\end{itemize}
O mesmo que \textunderscore loendro\textunderscore .
\section{Loendro}
\begin{itemize}
\item {Grp. gram.:m.}
\end{itemize}
\begin{itemize}
\item {Proveniência:(Do b. lat. \textunderscore lorandrum\textunderscore )}
\end{itemize}
Arbusto apocýneo, (\textunderscore nerium oleander\textunderscore ).
\section{Loengo}
\begin{itemize}
\item {Grp. gram.:m.}
\end{itemize}
(V.mifongo)
\section{Loèsnoroéste}
\begin{itemize}
\item {Grp. gram.:m.}
\end{itemize}
\begin{itemize}
\item {Utilização:Des.}
\end{itemize}
O mesmo que \textunderscore oèsnoroéste\textunderscore . Cf. \textunderscore Peregrinação\textunderscore , CLV.
\section{Loèssudoéste}
\begin{itemize}
\item {Grp. gram.:m.}
\end{itemize}
\begin{itemize}
\item {Utilização:Des.}
\end{itemize}
O mesmo que \textunderscore oèssudoéste\textunderscore . Cf. \textunderscore Peregrinação\textunderscore , CLXXIX.
\section{Loéste}
\begin{itemize}
\item {Grp. gram.:m.}
\end{itemize}
\begin{itemize}
\item {Utilização:Des.}
\end{itemize}
O mesmo que \textunderscore oéste\textunderscore . Cf. \textunderscore Peregrinação\textunderscore , CLV.
\section{Lofo}
\begin{itemize}
\item {fónica:lô}
\end{itemize}
\begin{itemize}
\item {Grp. gram.:adj.}
\end{itemize}
\begin{itemize}
\item {Utilização:Gír.}
\end{itemize}
Que é pateta.
\section{Logâneas}
\begin{itemize}
\item {Grp. gram.:f. pl.}
\end{itemize}
(V.loganiáceas)
\section{Logânia}
\begin{itemize}
\item {Grp. gram.:f.}
\end{itemize}
Gênero de plantas australianas.
\section{Loganiáceas}
\begin{itemize}
\item {Grp. gram.:f. pl.}
\end{itemize}
Família de plantas, que tem por typo a logânia.
\section{Logar}
\textunderscore m.\textunderscore  (e der.)
(V. \textunderscore lugar\textunderscore , etc.)
\section{Logaríthmico}
\begin{itemize}
\item {Grp. gram.:adj.}
\end{itemize}
Relativo aos logarithmos.
\section{Logarithmo}
\begin{itemize}
\item {Grp. gram.:m.}
\end{itemize}
\begin{itemize}
\item {Utilização:Mathem.}
\end{itemize}
\begin{itemize}
\item {Proveniência:(Do gr. \textunderscore logos\textunderscore  + \textunderscore arithmos\textunderscore )}
\end{itemize}
Expoente da potência, a que é preciso elevar um número constante, chamado base, para achar um número proposto.
\section{Logarítmico}
\begin{itemize}
\item {Grp. gram.:adj.}
\end{itemize}
Relativo aos logaritmos.
\section{Logaritmo}
\begin{itemize}
\item {Grp. gram.:m.}
\end{itemize}
\begin{itemize}
\item {Utilização:Mathem.}
\end{itemize}
\begin{itemize}
\item {Proveniência:(Do gr. \textunderscore logos\textunderscore  + \textunderscore arithmos\textunderscore )}
\end{itemize}
Expoente da potência, a que é preciso elevar um número constante, chamado base, para achar um número proposto.
\section{Loge}
\begin{itemize}
\item {Grp. gram.:f.}
\end{itemize}
(V.loja)
\section{Lógea}
\begin{itemize}
\item {Grp. gram.:f.}
\end{itemize}
(V.loja)
\section{Lógica}
\begin{itemize}
\item {Grp. gram.:f.}
\end{itemize}
\begin{itemize}
\item {Utilização:Pop.}
\end{itemize}
\begin{itemize}
\item {Proveniência:(Lat. \textunderscore logica\textunderscore )}
\end{itemize}
Sciência, que tem por objecto os processos do raciocínio ou as regras para o descobrimento e exposição da verdade.
Coherência, ligação de ideias.
Raciocínio.
Ardil; lábia.
\section{Logical}
\begin{itemize}
\item {Grp. gram.:adj.}
\end{itemize}
\begin{itemize}
\item {Utilização:Des.}
\end{itemize}
\begin{itemize}
\item {Proveniência:(De \textunderscore lógica\textunderscore )}
\end{itemize}
O mesmo que \textunderscore lógico\textunderscore . Cf. \textunderscore Eufrosina\textunderscore , 189.
\section{Logicamente}
\begin{itemize}
\item {Grp. gram.:adv.}
\end{itemize}
De modo lógico.
Coherentemente.
Consequentemente.
\section{Logicar}
\begin{itemize}
\item {Grp. gram.:v. i.}
\end{itemize}
\begin{itemize}
\item {Utilização:Fam.}
\end{itemize}
\begin{itemize}
\item {Proveniência:(De \textunderscore lógica\textunderscore )}
\end{itemize}
Discorrer logicamente.
Raciocinar.
Alardear lógica. Cf. Camillo, \textunderscore Canc. Al.\textunderscore , 347.
\section{Lógico}
\begin{itemize}
\item {Grp. gram.:adj.}
\end{itemize}
\begin{itemize}
\item {Utilização:Gram.}
\end{itemize}
\begin{itemize}
\item {Grp. gram.:M.}
\end{itemize}
\begin{itemize}
\item {Utilização:Burl.}
\end{itemize}
\begin{itemize}
\item {Proveniência:(Do gr. \textunderscore logikos\textunderscore )}
\end{itemize}
Relativo a lógica.
Conforme ás regras da lógica.
Coherente.
Diz-se da anályse, que recai, não na fórma e natureza das palavras, como a anályse grammatical, mas nas partes da oração ou proposição.
Aquelle que estuda ou sabe lógica.
Indivíduo finório, manhoso.
\section{Logística}
\begin{itemize}
\item {Grp. gram.:f.}
\end{itemize}
\begin{itemize}
\item {Proveniência:(De \textunderscore logístico\textunderscore )}
\end{itemize}
Designação antiga de uma parte da Álgebra, na qual se tratava das primeiras operações, isto é, da somma, da subtracção, etc.
\section{Logístico}
\begin{itemize}
\item {Grp. gram.:adj.}
\end{itemize}
\begin{itemize}
\item {Utilização:Mathem.}
\end{itemize}
\begin{itemize}
\item {Grp. gram.:M. pl.}
\end{itemize}
\begin{itemize}
\item {Proveniência:(Gr. \textunderscore logistikos\textunderscore )}
\end{itemize}
Diz-se dos logarithmos, em que zero é o logarithmo correspondente ao número 3.600.
Antiga seita médica, que, sem desprezar a experiência, se baseava na razão e nas theorias.
\section{Logo}
\begin{itemize}
\item {Grp. gram.:adv.}
\end{itemize}
\begin{itemize}
\item {Grp. gram.:M.}
\end{itemize}
\begin{itemize}
\item {Utilização:ant.}
\end{itemize}
\begin{itemize}
\item {Proveniência:(Do lat. \textunderscore locus\textunderscore )}
\end{itemize}
Em seguida; immediatamente: \textunderscore respondi-lhe logo\textunderscore .
Breve; daqui a pouco: \textunderscore eu logo venho\textunderscore .
Portanto, por consequencia: \textunderscore mentiste; logo, és indigno...\textunderscore 
Morada, residência.
Lugar. Cf. \textunderscore Eufrosina\textunderscore , 224.
\section{Logo...}
\begin{itemize}
\item {Grp. gram.:pref.}
\end{itemize}
\begin{itemize}
\item {Proveniência:(Do gr. \textunderscore logos\textunderscore )}
\end{itemize}
(designativo de \textunderscore palavra\textunderscore )
\section{Logografia}
\begin{itemize}
\item {Grp. gram.:f.}
\end{itemize}
Arte de escrever tão depressa como se fala; estenografia.
(Cp. \textunderscore logógrafo\textunderscore )
\section{Logógrafo}
\begin{itemize}
\item {Grp. gram.:m.}
\end{itemize}
\begin{itemize}
\item {Proveniência:(Do gr. \textunderscore logos\textunderscore  + \textunderscore graphein\textunderscore )}
\end{itemize}
Nome, que se deu aos primeiros escritores gregos.
Autor de um glossário.
Aquele que escreve tão depressa como se fala.
\section{Logographia}
\begin{itemize}
\item {Grp. gram.:f.}
\end{itemize}
Arte de escrever tão depressa como se fala; estenographia.
(Cp. \textunderscore logógrapho\textunderscore )
\section{Logógrapho}
\begin{itemize}
\item {Grp. gram.:m.}
\end{itemize}
\begin{itemize}
\item {Proveniência:(Do gr. \textunderscore logos\textunderscore  + \textunderscore graphein\textunderscore )}
\end{itemize}
Nome, que se deu aos primeiros escritores gregos.
Autor de um glossário.
Aquelle que escreve tão depressa como se fala.
\section{Logogrífico}
\begin{itemize}
\item {Grp. gram.:adj.}
\end{itemize}
Relativo ao logogrifo.
\section{Logogrifo}
\begin{itemize}
\item {Grp. gram.:m.}
\end{itemize}
\begin{itemize}
\item {Utilização:Fig.}
\end{itemize}
\begin{itemize}
\item {Proveniência:(Do gr. \textunderscore logos\textunderscore  + \textunderscore griphos\textunderscore )}
\end{itemize}
Espécie de enigma, em que as letras ou sílabas da palavra, que serve de conceito, formam outras palavras, definindo-se estas.
Coisa obscura, enigmática.
\section{Logogríphico}
\begin{itemize}
\item {Grp. gram.:adj.}
\end{itemize}
Relativo ao logogripho.
\section{Logogripho}
\begin{itemize}
\item {Grp. gram.:m.}
\end{itemize}
\begin{itemize}
\item {Utilização:Fig.}
\end{itemize}
\begin{itemize}
\item {Proveniência:(Do gr. \textunderscore logos\textunderscore  + \textunderscore griphos\textunderscore )}
\end{itemize}
Espécie de enigma, em que as letras ou sýllabas da palavra, que serve de conceito, formam outras palavras, definindo-se estas.
Coisa obscura, enigmática.
\section{Logomachia}
\begin{itemize}
\item {fónica:qui}
\end{itemize}
\begin{itemize}
\item {Grp. gram.:f.}
\end{itemize}
\begin{itemize}
\item {Proveniência:(Lat. \textunderscore logomachia\textunderscore )}
\end{itemize}
Discussão sôbre o sentido ou origem de uma palavra ou palavras; questão de palavras.
\section{Logomáchico}
\begin{itemize}
\item {fónica:qui}
\end{itemize}
\begin{itemize}
\item {Grp. gram.:adj.}
\end{itemize}
Relativo á logomachia.
\section{Logomania}
\begin{itemize}
\item {Grp. gram.:f.}
\end{itemize}
\begin{itemize}
\item {Proveniência:(Lat. \textunderscore logomania\textunderscore )}
\end{itemize}
Amor excessivo ás letras ou ao estudo.
\section{Logomaquia}
\begin{itemize}
\item {Grp. gram.:f.}
\end{itemize}
\begin{itemize}
\item {Proveniência:(Lat. \textunderscore logomachia\textunderscore )}
\end{itemize}
Discussão sôbre o sentido ou origem de uma palavra ou palavras; questão de palavras.
\section{Logomáquico}
\begin{itemize}
\item {Grp. gram.:adj.}
\end{itemize}
Relativo á logomaquia.
\section{Logorreia}
\begin{itemize}
\item {Grp. gram.:f.}
\end{itemize}
\begin{itemize}
\item {Utilização:Deprec.}
\end{itemize}
\begin{itemize}
\item {Proveniência:(Do gr. \textunderscore logos\textunderscore  + \textunderscore rhein\textunderscore )}
\end{itemize}
Fluência de palavras; grande verbosidade.
\section{Logorrhéa}
\begin{itemize}
\item {Grp. gram.:f.}
\end{itemize}
\begin{itemize}
\item {Utilização:Deprec.}
\end{itemize}
\begin{itemize}
\item {Proveniência:(Do gr. \textunderscore logos\textunderscore  + \textunderscore rhein\textunderscore )}
\end{itemize}
Fluência de palavras; grande verbosidade.
\section{Logorrheia}
\begin{itemize}
\item {Grp. gram.:f.}
\end{itemize}
\begin{itemize}
\item {Utilização:Deprec.}
\end{itemize}
\begin{itemize}
\item {Proveniência:(Do gr. \textunderscore logos\textunderscore  + \textunderscore rhein\textunderscore )}
\end{itemize}
Fluência de palavras; grande verbosidade.
\section{Logo-tenente}
\begin{itemize}
\item {Grp. gram.:m.}
\end{itemize}
\begin{itemize}
\item {Utilização:Ant.}
\end{itemize}
O mesmo que \textunderscore lugar-tenente\textunderscore .
\section{Logra}
\begin{itemize}
\item {Grp. gram.:f.}
\end{itemize}
\begin{itemize}
\item {Utilização:T. do Faial}
\end{itemize}
Embarcação de três mastros, um pouco semelhante a certos lugres.
(Por \textunderscore lugra\textunderscore , de \textunderscore lugre\textunderscore ^2?)
\section{Logração}
\begin{itemize}
\item {Grp. gram.:f.}
\end{itemize}
Acto ou effeito de lograr.
Lôgro.
Engano gracioso.
Engano.
\section{Logradeira}
\begin{itemize}
\item {Grp. gram.:f.  e  adj.}
\end{itemize}
Mulher, que logra alguém, que é trapaceira.
\section{Logradoiro}
\begin{itemize}
\item {Grp. gram.:m.}
\end{itemize}
\begin{itemize}
\item {Proveniência:(De \textunderscore lograr\textunderscore )}
\end{itemize}
Aquillo que póde sêr logrado.
Terreno, contíguo a uma habitação, e que serve para estrumeira ou outro uso.
Rocio.
Terreno público, ou pastagem para os gados de uma povoação ou região.
Maninho.
\section{Logrador}
\begin{itemize}
\item {Grp. gram.:m.  e  adj.}
\end{itemize}
Aquelle que logra; homem burlador, trapaceiro.
\section{Logrador}
\begin{itemize}
\item {Grp. gram.:m.}
\end{itemize}
\begin{itemize}
\item {Utilização:Bras. do Ceará}
\end{itemize}
Parte de uma fazenda de criação, onde se trata de gado e principalmente de vacas feridas.
(Corr. de \textunderscore logradoiro\textunderscore )
\section{Logradouro}
\begin{itemize}
\item {Grp. gram.:m.}
\end{itemize}
\begin{itemize}
\item {Proveniência:(De \textunderscore lograr\textunderscore )}
\end{itemize}
Aquillo que póde sêr logrado.
Terreno, contíguo a uma habitação, e que serve para estrumeira ou outro uso.
Rocio.
Terreno público, ou pastagem para os gados de uma povoação ou região.
Maninho.
\section{Logramento}
\begin{itemize}
\item {Grp. gram.:m.}
\end{itemize}
Acto de lograr.
\section{Logrão}
\begin{itemize}
\item {Grp. gram.:m.}
\end{itemize}
\begin{itemize}
\item {Proveniência:(Do lat. \textunderscore lucrio\textunderscore )}
\end{itemize}
Aquelle que logra; burlador; intrujão.
Indivíduo interesseiro, ganancioso. Cf. B. Pereira, \textunderscore Prosódia\textunderscore , vb. \textunderscore lucrio\textunderscore .
\section{Lograr}
\begin{itemize}
\item {Grp. gram.:v. t.}
\end{itemize}
\begin{itemize}
\item {Proveniência:(Do lat. \textunderscore lucrari\textunderscore )}
\end{itemize}
Fruir; possuír.
Tirar lucro de.
Enganar.
Burlar; intrujar.
Gracejar com, mentindo.
Conseguir; obter: \textunderscore lograr triunfos\textunderscore .
\section{Logrativo}
\begin{itemize}
\item {Grp. gram.:adj.}
\end{itemize}
Que logra; trapaceiro.
\textunderscore Mulher logrativa\textunderscore , a que procura agradar, que é galanteadora. Cf. Filinto e Camillo.
\section{Logreiro}
\begin{itemize}
\item {Grp. gram.:m.}
\end{itemize}
\begin{itemize}
\item {Utilização:Ant.}
\end{itemize}
\begin{itemize}
\item {Proveniência:(De \textunderscore lograr\textunderscore , se não é alter. de \textunderscore lucreiro\textunderscore , de \textunderscore lucro\textunderscore )}
\end{itemize}
O mesmo que \textunderscore usurário\textunderscore .
\section{Logreiro}
\begin{itemize}
\item {Grp. gram.:adj.}
\end{itemize}
\begin{itemize}
\item {Proveniência:(De \textunderscore lôgro\textunderscore )}
\end{itemize}
Que logra ou burla.
Que faz caír em engano.
Manhoso. Cf. Camillo, \textunderscore Noites de Insómn.\textunderscore , V, 22; Filinto, VII, 249; XII, 97.
\section{Lôgro}
\begin{itemize}
\item {Grp. gram.:m.}
\end{itemize}
\begin{itemize}
\item {Utilização:Des.}
\end{itemize}
\begin{itemize}
\item {Utilização:Pop.}
\end{itemize}
Acto ou effeito de lograr.
Engano propositado, contra alguém.
Burla.
Lucro:«\textunderscore pôr cabedaes a lôgro...\textunderscore »Camillo, \textunderscore Ôlho de Vidro\textunderscore , 96.
Pulha, partida ou peça de entrudo; engano jocoso.
\section{Lóia}
\begin{itemize}
\item {Grp. gram.:f.}
\end{itemize}
\begin{itemize}
\item {Utilização:Ant.}
\end{itemize}
Manilha de oiro massiço no Oriente. Cf. \textunderscore Peregrinação\textunderscore , XXII.
\section{Loiça}
\begin{itemize}
\item {Grp. gram.:f.}
\end{itemize}
\begin{itemize}
\item {Utilização:Prov.}
\end{itemize}
\begin{itemize}
\item {Utilização:Pop.}
\end{itemize}
\begin{itemize}
\item {Grp. gram.:Pl.}
\end{itemize}
\begin{itemize}
\item {Proveniência:(Do lat. \textunderscore luteus\textunderscore )}
\end{itemize}
Productos de cerâmica.
Barro, porcelana ou outras substâncias análogas, manufacturadas por oleiro, para serviço de mesa especialmente.
Vasilhame.
Chocalho para o pescoço do gado.
Coisa excellente.
Depósito geral das águas, que devem alimentar a salina, e que abrange o caldeiro, a caldeira e o pejo.
\section{Loiçaria}
\begin{itemize}
\item {Grp. gram.:f.}
\end{itemize}
Estabelecimento, onde se vende loiça.
Loiças, conjunto de loiças.
\section{Loiceira}
\begin{itemize}
\item {Grp. gram.:f.}
\end{itemize}
Vendedora de loiça.
Guarda-loiça.
(Cp. \textunderscore loiceiro\textunderscore )
\section{Loiceiro}
\begin{itemize}
\item {Grp. gram.:m.}
\end{itemize}
\begin{itemize}
\item {Utilização:Prov.}
\end{itemize}
\begin{itemize}
\item {Utilização:Prov.}
\end{itemize}
\begin{itemize}
\item {Utilização:Prov.}
\end{itemize}
\begin{itemize}
\item {Utilização:minh.}
\end{itemize}
Fabricante ou negociante de loiça.
Vasilha de adega.
Utensílio, formado de um tronco vertical, com galhos, para nestes se pendurar a loiça da cozinha.
Taboleiro, em que se põe loiça.
Armário para loiça; guarda-loiça.
\section{Lóio}
\begin{itemize}
\item {Grp. gram.:m.}
\end{itemize}
\begin{itemize}
\item {Grp. gram.:Adj.}
\end{itemize}
\begin{itemize}
\item {Grp. gram.:M.}
\end{itemize}
\begin{itemize}
\item {Proveniência:(De \textunderscore Eloi\textunderscore  (santo), n. p., seg. Car. Michaëlis)}
\end{itemize}
Planta, da fam. das compostas, (\textunderscore centaurea cyanus\textunderscore ).
Azulado.
Tirante a azul.
Pertencente á Ordem de San-João Evangelista, em que o hábito dos frades era azulado.
Frade dessa Ordem.
\section{Lóio}
\begin{itemize}
\item {Grp. gram.:adj.}
\end{itemize}
\begin{itemize}
\item {Utilização:Pop.}
\end{itemize}
Ignorante; basbaque; leigo em qualquer assumpto.
(Relaciona-se com \textunderscore lóio\textunderscore ^1?)
\section{Loira}
\begin{itemize}
\item {Grp. gram.:f.}
\end{itemize}
\begin{itemize}
\item {Utilização:Fam.}
\end{itemize}
\begin{itemize}
\item {Grp. gram.:M.}
\end{itemize}
\begin{itemize}
\item {Utilização:Pop.}
\end{itemize}
\begin{itemize}
\item {Proveniência:(De \textunderscore loiro\textunderscore ^1)}
\end{itemize}
Mulher, que tem o cabello loiro.
Libra esterlina.
Homem bonacheirão, simplório.
\section{Loira}
\begin{itemize}
\item {Grp. gram.:f.}
\end{itemize}
O mesmo que \textunderscore lura\textunderscore .
\section{Loiraça}
\begin{itemize}
\item {Grp. gram.:m.  e  f.}
\end{itemize}
\begin{itemize}
\item {Utilização:Fam.}
\end{itemize}
\begin{itemize}
\item {Proveniência:(Do rad. de \textunderscore loiro\textunderscore ^1)}
\end{itemize}
Pessôa simplória.
Pessôa, que tem o cabello loiro.
\section{Loirar}
\begin{itemize}
\item {Grp. gram.:v. t.  e  i.}
\end{itemize}
O mesmo que \textunderscore loirejar\textunderscore .
\section{Loirecente}
\begin{itemize}
\item {Grp. gram.:adj.}
\end{itemize}
Que loirece.
\section{Loirecer}
\begin{itemize}
\item {Grp. gram.:v. t.  e  i.}
\end{itemize}
O mesmo que \textunderscore loirejar\textunderscore .
\section{Loireira}
\begin{itemize}
\item {Grp. gram.:f.  e  adj.}
\end{itemize}
Diz-se da mulher, que deseja agradar a todos.
Provocante, seductora. Cf. F. Manuel, \textunderscore Carta de Guia de Casados\textunderscore ; Filinto, IV, 272; Camillo, \textunderscore Regicida\textunderscore , 50.
\section{Loireira}
\begin{itemize}
\item {Grp. gram.:f.}
\end{itemize}
Casta de uva branca do Minho.
\section{Loireiral}
\begin{itemize}
\item {Grp. gram.:m.}
\end{itemize}
Lugar, onde crescem loireiros.
\section{Loireiro}
\begin{itemize}
\item {Grp. gram.:m.}
\end{itemize}
\begin{itemize}
\item {Proveniência:(Do lat. \textunderscore laurarius\textunderscore )}
\end{itemize}
Árvore monopétala, sempre, verde, que produz umas bagas escuras e amargas.
\section{Loirejante}
\begin{itemize}
\item {Grp. gram.:adj.}
\end{itemize}
Que loireja. Cf. Camillo, \textunderscore Myst. de Lisb.\textunderscore , II, 106.
\section{Loirejar}
\begin{itemize}
\item {Grp. gram.:v. t.}
\end{itemize}
\begin{itemize}
\item {Grp. gram.:V. i.}
\end{itemize}
Tornar loiro.
Tornar se loiro; apresentar-se ou mostrar-se loiro.
Amarelecer: \textunderscore o trigo já loireja\textunderscore .
\section{Loirejo}
\begin{itemize}
\item {Grp. gram.:m.}
\end{itemize}
Acto de loirejar.
Côr loira ou amarela.
\section{Loirela}
\begin{itemize}
\item {Grp. gram.:f.}
\end{itemize}
\begin{itemize}
\item {Proveniência:(De \textunderscore loiro\textunderscore ^1)}
\end{itemize}
Casta de uva preta, na região do Doiro.
Variedade de oliveira.
\section{Loiro}
\begin{itemize}
\item {Grp. gram.:adj.}
\end{itemize}
\begin{itemize}
\item {Grp. gram.:M.}
\end{itemize}
\begin{itemize}
\item {Proveniência:(Do lat. \textunderscore aureus\textunderscore ? O \textunderscore l\textunderscore  de \textunderscore loiro\textunderscore  poderia vir por intermédio do fr. \textunderscore l'or\textunderscore , como o port. \textunderscore lôba\textunderscore ^2 veio do fr. \textunderscore l'aube\textunderscore , e o port. \textunderscore léste\textunderscore  do fr. \textunderscore l'est\textunderscore )}
\end{itemize}
Que tem côr média entre o doirado e o castanho claro, como as espigas maduras do trigo.
Homem de cabello loiro.
\section{Loiro}
\begin{itemize}
\item {Grp. gram.:m.}
\end{itemize}
\begin{itemize}
\item {Grp. gram.:Pl.}
\end{itemize}
\begin{itemize}
\item {Proveniência:(Do lat. \textunderscore laurus\textunderscore )}
\end{itemize}
O mesmo que \textunderscore loireiro\textunderscore .
Folhas de loireiro.
Lauréis; triumphos; glória.
\section{Loiro}
\begin{itemize}
\item {Grp. gram.:m.}
\end{itemize}
\begin{itemize}
\item {Utilização:Pop.}
\end{itemize}
\begin{itemize}
\item {Proveniência:(Do mal. \textunderscore nori\textunderscore )}
\end{itemize}
O mesmo que \textunderscore papagaio\textunderscore .
\section{Loisa}
\begin{itemize}
\item {Grp. gram.:f.}
\end{itemize}
\begin{itemize}
\item {Proveniência:(Do lat. hyp. \textunderscore lausa\textunderscore ?)}
\end{itemize}
Lâmina de pedra.
Ardósia.
Lápide, que cobre uma sepultura.
Lura.
Armadilha de pedra, para os pássaros; loisão.
\section{Loisã}
\begin{itemize}
\item {Grp. gram.:adj.}
\end{itemize}
Dizia-se da terra, em que há muitas loisas.
\section{Loisador}
\begin{itemize}
\item {Grp. gram.:m.}
\end{itemize}
\begin{itemize}
\item {Utilização:Bras}
\end{itemize}
O encarregado de limpar ou preparar loisas.
\section{Loisan}
\begin{itemize}
\item {Grp. gram.:adj.}
\end{itemize}
Dizia-se da terra, em que há muitas loisas.
\section{Loisão}
\begin{itemize}
\item {Grp. gram.:m.}
\end{itemize}
Armadilha, o mesmo que \textunderscore loisa\textunderscore .
\section{Loisas}
\begin{itemize}
\item {Grp. gram.:f. pl.}
\end{itemize}
Us. na loc. \textunderscore coisas e loisas\textunderscore , diversas coisas, assumptos vários.
\section{Loiseira}
\begin{itemize}
\item {Grp. gram.:f.}
\end{itemize}
Lugar, donde se extrai loisa.
\section{Loiseiro}
\begin{itemize}
\item {Grp. gram.:m.}
\end{itemize}
Aquelle que extrai loisas da respectiva rocha.
Aquelle que trabalha em loisa.
\section{Loisífero}
\begin{itemize}
\item {Grp. gram.:adj.}
\end{itemize}
\begin{itemize}
\item {Proveniência:(De \textunderscore loisa\textunderscore  + lat. \textunderscore ferre\textunderscore )}
\end{itemize}
Diz-se, do terreno, em que há loisas.
\section{Loisinha}
\begin{itemize}
\item {Grp. gram.:f.}
\end{itemize}
\begin{itemize}
\item {Utilização:Prov.}
\end{itemize}
\begin{itemize}
\item {Proveniência:(De \textunderscore loisa\textunderscore )}
\end{itemize}
O mesmo que \textunderscore xisto\textunderscore ^1.
\section{Loisinho}
\begin{itemize}
\item {Grp. gram.:adj.}
\end{itemize}
\begin{itemize}
\item {Proveniência:(De \textunderscore loisa\textunderscore )}
\end{itemize}
Diz-se do terreno ou rocha, em que o xisto apparece sob a fórma laminar, como a loisa. Cf. P. Carvalho, \textunderscore Corogr. Port.\textunderscore , I, 46 e 477.
\section{Loja}
\begin{itemize}
\item {Grp. gram.:f.}
\end{itemize}
\begin{itemize}
\item {Utilização:Bras. dos sertões do N}
\end{itemize}
\begin{itemize}
\item {Proveniência:(Do it. \textunderscore loggia\textunderscore )}
\end{itemize}
Casa térrea.
Pavimento térreo de, uma casa.
Casa para venda de mercadorias.
Officina.
Habitação assobradada, ao rés do chão.
Casa de associação maçónica.
Designação do ânus da cavalgadura.
\section{Loje}
\begin{itemize}
\item {Grp. gram.:f.}
\end{itemize}
\begin{itemize}
\item {Utilização:Prov.}
\end{itemize}
\begin{itemize}
\item {Utilização:trasm.}
\end{itemize}
Córte do gado.
\section{Lojeca}
\begin{itemize}
\item {Grp. gram.:f.}
\end{itemize}
\begin{itemize}
\item {Proveniência:(De \textunderscore loja\textunderscore )}
\end{itemize}
Pequena loja; baiúca; locanda.
\section{Lojeiro}
\begin{itemize}
\item {Grp. gram.:m.}
\end{itemize}
O mesmo que \textunderscore lojista\textunderscore . C. F. Borges, \textunderscore Diccion. Jur.\textunderscore  (Us. ainda no Algarve)
\section{Lojista}
\begin{itemize}
\item {Grp. gram.:m.  e  f.}
\end{itemize}
\begin{itemize}
\item {Proveniência:(De loja)}
\end{itemize}
Pessôa, que tem loja para commércio.
\section{Lolé}
\begin{itemize}
\item {Grp. gram.:interj.}
\end{itemize}
Estribilho, em algumas canções populares.
Chiste, graça.
(Do cigano \textunderscore lolé\textunderscore , jumento)
\section{Loligídeo}
\begin{itemize}
\item {Grp. gram.:adj.}
\end{itemize}
\begin{itemize}
\item {Grp. gram.:M. pl.}
\end{itemize}
\begin{itemize}
\item {Proveniência:(Do lat. \textunderscore loligo\textunderscore  + gr. \textunderscore eidos\textunderscore )}
\end{itemize}
Relativo ou semelhante á lula ou chôco.
Família de molluscos, que tem por typo a lula.
\section{Lólio}
\begin{itemize}
\item {Grp. gram.:m.}
\end{itemize}
\begin{itemize}
\item {Proveniência:(Lat. \textunderscore lolium\textunderscore )}
\end{itemize}
Designação scientífica do joio.
\section{Lomária}
\begin{itemize}
\item {Grp. gram.:f.}
\end{itemize}
Gênero de fêtos.
\section{Lomátia}
\begin{itemize}
\item {Grp. gram.:f.}
\end{itemize}
Gênero de plantas proteáceas.
\section{Lomba}
\begin{itemize}
\item {Grp. gram.:f.}
\end{itemize}
\begin{itemize}
\item {Utilização:Prov.}
\end{itemize}
\begin{itemize}
\item {Utilização:trasm.}
\end{itemize}
\begin{itemize}
\item {Proveniência:(Do lat. \textunderscore lumbus\textunderscore )}
\end{itemize}
Cumeeira.
Lombada de serra.
Montículo de areia ou terra, formado pelo vento; médo.
Preguiça.
\section{Lombada}
\begin{itemize}
\item {Grp. gram.:f.}
\end{itemize}
\begin{itemize}
\item {Proveniência:(De \textunderscore lombo\textunderscore )}
\end{itemize}
Lomba prolongada.
Dorso do boi.
Costas do livro.
\section{Lombar}
\begin{itemize}
\item {Grp. gram.:adj.}
\end{itemize}
Relativo ao lombo: \textunderscore dôres lombares\textunderscore .
\section{Lombarda}
\begin{itemize}
\item {Grp. gram.:f.  e  adj.}
\end{itemize}
\begin{itemize}
\item {Utilização:Ant.}
\end{itemize}
Espécie de capa, usada nas recepções da côrte.
Espécie de couve.
(Fem. de \textunderscore lombardo\textunderscore ^1)
\section{Lombarda}
\textunderscore f.\textunderscore  (e der.) \textunderscore Des.\textunderscore 
O mesmo que \textunderscore bombarda\textunderscore , etc.
\section{Lombardismo}
\begin{itemize}
\item {Grp. gram.:m.}
\end{itemize}
Maneira de falar dos Lombardos.
Locução privativa da Lombárdia.
\section{Lombardito}
\begin{itemize}
\item {Grp. gram.:adj.}
\end{itemize}
Diz-se do toiro, que é um tanto lombardo.
(Do \textunderscore lombardo\textunderscore ^2)
\section{Lombardo}
\begin{itemize}
\item {Grp. gram.:adj.}
\end{itemize}
\begin{itemize}
\item {Grp. gram.:M. pl.}
\end{itemize}
Relativo á Lombárdia.
Diz-se de uma espécie de couve.
Povos da Lombárdia.
\section{Lombardo}
\begin{itemize}
\item {Grp. gram.:adj.}
\end{itemize}
Diz-se do toiro negro com o lombo acastanhado.
(Corr. de \textunderscore lompardo\textunderscore , provavelmente. V. \textunderscore lompardo\textunderscore )
\section{Lombear}
\begin{itemize}
\item {Grp. gram.:v. i.}
\end{itemize}
\begin{itemize}
\item {Utilização:Bras. do S}
\end{itemize}
Diz-se do cavallo arisco que torce o lombo, quando montado.
Diz-se da sella, quando fere o lombo do animal.
\section{Lombeira}
\begin{itemize}
\item {Grp. gram.:f.}
\end{itemize}
\begin{itemize}
\item {Utilização:Prov.}
\end{itemize}
\begin{itemize}
\item {Utilização:Bras}
\end{itemize}
\begin{itemize}
\item {Utilização:trasm.}
\end{itemize}
\begin{itemize}
\item {Proveniência:(De \textunderscore lombo\textunderscore )}
\end{itemize}
Quebrantamento de fôrças.
Molleza do corpo.
\section{Lombeiro}
\begin{itemize}
\item {Grp. gram.:adj.}
\end{itemize}
\begin{itemize}
\item {Grp. gram.:M.}
\end{itemize}
\begin{itemize}
\item {Proveniência:(De \textunderscore lombo\textunderscore )}
\end{itemize}
O mesmo que \textunderscore lombar\textunderscore .
Coiro do lombo de certos animaes.
\section{Lombelo}
\begin{itemize}
\item {fónica:bê}
\end{itemize}
\begin{itemize}
\item {Grp. gram.:m.}
\end{itemize}
\begin{itemize}
\item {Utilização:Prov.}
\end{itemize}
\begin{itemize}
\item {Utilização:trasm.}
\end{itemize}
\begin{itemize}
\item {Proveniência:(De \textunderscore lombo\textunderscore )}
\end{itemize}
Nome vulgar de um dos músculos, que se inserem na columna vertebral do gado bovino.
Cada um dos dois pedaços compridos de carne, que se tiram dos lados do lombo do porco e que também se chamam coêlhos.
\section{Lombilheiro}
\begin{itemize}
\item {Grp. gram.:m.}
\end{itemize}
Fabricante de lombilhos.
\section{Lombilho}
\begin{itemize}
\item {Grp. gram.:m.}
\end{itemize}
\begin{itemize}
\item {Utilização:Bras. do S}
\end{itemize}
\begin{itemize}
\item {Proveniência:(De \textunderscore lombo\textunderscore )}
\end{itemize}
Parte dos arreios, que póde substituir o sellim.
\section{Lombinho}
\begin{itemize}
\item {Grp. gram.:m.}
\end{itemize}
O mesmo que \textunderscore lombelo\textunderscore .
\section{Lombo}
\begin{itemize}
\item {Grp. gram.:m.}
\end{itemize}
\begin{itemize}
\item {Utilização:Pop.}
\end{itemize}
\begin{itemize}
\item {Proveniência:(Do lat. \textunderscore lumbus\textunderscore )}
\end{itemize}
Parte carnosa que está do cada lado da espinha dorsal.
Costas.
Lombada de livro.
Lombada; lomba.
Superfície convexa da telha.
\section{Lombo-abdominal}
\begin{itemize}
\item {Grp. gram.:adj.}
\end{itemize}
Relativo aos rins e ao abdome.
\section{Lombo-umeral}
\begin{itemize}
\item {Grp. gram.:adj.}
\end{itemize}
Relativo aos rins e ás espáduas.
\section{Lombrical}
\begin{itemize}
\item {Grp. gram.:adj.}
\end{itemize}
\begin{itemize}
\item {Proveniência:(Do lat. \textunderscore lumbricus\textunderscore )}
\end{itemize}
Relativo ou semelhante a lombriga.
\section{Lombricita}
\begin{itemize}
\item {Grp. gram.:f.}
\end{itemize}
\begin{itemize}
\item {Proveniência:(Do lat. \textunderscore lumbricus\textunderscore )}
\end{itemize}
Petrificação, com a fórma de lombriga.
\section{Lombricoide}
\begin{itemize}
\item {Grp. gram.:adj.}
\end{itemize}
\begin{itemize}
\item {Grp. gram.:M.}
\end{itemize}
\begin{itemize}
\item {Proveniência:(Do lat. \textunderscore lumbricus\textunderscore  + gr. \textunderscore eidos\textunderscore )}
\end{itemize}
Lombrical.
Lombriga.
\section{Lombriga}
\begin{itemize}
\item {Grp. gram.:f.}
\end{itemize}
\begin{itemize}
\item {Proveniência:(Do lat. \textunderscore lumbricus\textunderscore )}
\end{itemize}
Verme intestinal, do gênero das ascárides.
Gênero de anélidos, que tem por typo a minhoca.
\section{Lombrigar}
\begin{itemize}
\item {Grp. gram.:v. t.}
\end{itemize}
(Por \textunderscore lobrigar\textunderscore ). Cf. Garrett, \textunderscore Viagens\textunderscore .
\section{Lombrigueira}
\begin{itemize}
\item {Grp. gram.:f.}
\end{itemize}
\begin{itemize}
\item {Proveniência:(De \textunderscore lombriga\textunderscore . V. \textunderscore guaxinguba\textunderscore )}
\end{itemize}
Nome, que os Portugueses dão a árvore, que os indígenas do Brasil chamam guaxinguba.
\section{Lombudo}
\begin{itemize}
\item {Grp. gram.:adj.}
\end{itemize}
Que tem bons lombos.
\section{Lomear}
\begin{itemize}
\item {Grp. gram.:v. t.}
\end{itemize}
\begin{itemize}
\item {Utilização:Ant.}
\end{itemize}
O mesmo que \textunderscore alomear\textunderscore .
\section{Lomedro}
\begin{itemize}
\item {fónica:mê}
\end{itemize}
\begin{itemize}
\item {Utilização:Prov.}
\end{itemize}
\begin{itemize}
\item {Utilização:minh.}
\end{itemize}
\begin{itemize}
\item {Utilização:Prov.}
\end{itemize}
\begin{itemize}
\item {Utilização:trasm.}
\end{itemize}
A parte da coxa, que fica por cima do joelho.
Nádegas.
\section{Lomentáceas}
\begin{itemize}
\item {Grp. gram.:f. pl.}
\end{itemize}
\begin{itemize}
\item {Utilização:Bot.}
\end{itemize}
\begin{itemize}
\item {Proveniência:(De \textunderscore lomentáceo\textunderscore )}
\end{itemize}
O mesmo que cesalpíneas.
\section{Lomentáceo}
\begin{itemize}
\item {Grp. gram.:adj.}
\end{itemize}
\begin{itemize}
\item {Utilização:Bot.}
\end{itemize}
\begin{itemize}
\item {Proveniência:(Do lat. \textunderscore lomentum\textunderscore )}
\end{itemize}
Que é cortado por articulações, de espaço a espaço, (falando-se de frutos ou fôlhas das leguminosas)
\section{Lompardo}
\begin{itemize}
\item {Grp. gram.:adj.}
\end{itemize}
\begin{itemize}
\item {Proveniência:(De \textunderscore lombo\textunderscore  + \textunderscore pardo\textunderscore )}
\end{itemize}
Diz-se do toiro, que tem o lombo pardo e mais escuro que o resto do corpo.
\section{Lona}
\begin{itemize}
\item {Grp. gram.:f.}
\end{itemize}
Tecido grosseiro e forte, de que se fazem toldos, velas de navios, etc.
(Talvez de \textunderscore Olonne\textunderscore , n. p. geogr. em França)
\section{Lona}
\begin{itemize}
\item {Grp. gram.:f.}
\end{itemize}
\begin{itemize}
\item {Utilização:Burl.}
\end{itemize}
\begin{itemize}
\item {Utilização:Prov.}
\end{itemize}
\begin{itemize}
\item {Grp. gram.:M.}
\end{itemize}
\begin{itemize}
\item {Utilização:Prov.}
\end{itemize}
\begin{itemize}
\item {Utilização:trasm.}
\end{itemize}
Léria, palanfrório.
Mentira, lôa.
Bisbórria; troca-tintas.
\section{Lonca}
\begin{itemize}
\item {Grp. gram.:f.}
\end{itemize}
\begin{itemize}
\item {Utilização:Bras. do S}
\end{itemize}
Pedaço de coiro lonqueado.
\section{Londera-angundo}
\begin{itemize}
\item {Grp. gram.:m.}
\end{itemize}
Ave pernalta da África occidental.
\section{Londo}
\begin{itemize}
\item {Grp. gram.:m.}
\end{itemize}
\begin{itemize}
\item {Utilização:Ant.}
\end{itemize}
Certa renda ou foro.
\section{Londré}
\begin{itemize}
\item {Grp. gram.:m.}
\end{itemize}
Espécie de charuto. Cf. Camillo, Narcót., II, 15.
\section{Londres}
\begin{itemize}
\item {Grp. gram.:m.}
\end{itemize}
\begin{itemize}
\item {Proveniência:(De \textunderscore Londres\textunderscore , n. p.)}
\end{itemize}
Espécie de tecido antigo, fabricado em Londres.
\section{Londrês}
\begin{itemize}
\item {Grp. gram.:m.}
\end{itemize}
Habitante de Londres. Cf. Camillo, Narcót., II, 15.
\section{Londrino}
\begin{itemize}
\item {Grp. gram.:adj.}
\end{itemize}
\begin{itemize}
\item {Grp. gram.:M.}
\end{itemize}
Relativo a Londres.
Fabricado ou feito em Londres: \textunderscore queijo londrino\textunderscore .
Habitante de Londres.
\section{Longa}
\begin{itemize}
\item {Grp. gram.:f.}
\end{itemize}
\begin{itemize}
\item {Utilização:Des.}
\end{itemize}
\begin{itemize}
\item {Utilização:Ant.}
\end{itemize}
\begin{itemize}
\item {Proveniência:(De \textunderscore longo\textunderscore )}
\end{itemize}
Nota musical, que valia duas breves.
Em versificação latina, a sýllaba longa.
Instrumento de cordas, citado por Fernão Lopes e outros clássicos.
\section{Longada}
\begin{itemize}
\item {Grp. gram.:f.}
\end{itemize}
\begin{itemize}
\item {Utilização:Des.}
\end{itemize}
\begin{itemize}
\item {Proveniência:(De \textunderscore longado\textunderscore )}
\end{itemize}
Acto de ir para longe.
Afastamento.
Viagem:«\textunderscore que ei boime per hi fora de longada.\textunderscore »Antiga canção anón.
\section{Longadamente}
\begin{itemize}
\item {Grp. gram.:adv.}
\end{itemize}
\begin{itemize}
\item {Utilização:Ant.}
\end{itemize}
\begin{itemize}
\item {Proveniência:(Do rad. de \textunderscore longo\textunderscore )}
\end{itemize}
Por muito tempo.
\section{Longado}
\begin{itemize}
\item {Grp. gram.:adj.}
\end{itemize}
\begin{itemize}
\item {Utilização:Des.}
\end{itemize}
\begin{itemize}
\item {Proveniência:(De \textunderscore longo\textunderscore )}
\end{itemize}
Que dura muito.
Muito extenso.
\section{Longaes}
\begin{itemize}
\item {Grp. gram.:f.}
\end{itemize}
Antiga variedade de pêra, hoje, desconhecida por aquelle nome.
(Cp. \textunderscore longal\textunderscore )
\section{Longais}
\begin{itemize}
\item {Grp. gram.:f.}
\end{itemize}
Antiga variedade de pêra, hoje, desconhecida por aquelle nome.
(Cp. \textunderscore longal\textunderscore )
\section{Longal}
\begin{itemize}
\item {Grp. gram.:adj.}
\end{itemize}
\begin{itemize}
\item {Utilização:Prov.}
\end{itemize}
\begin{itemize}
\item {Utilização:trasm.}
\end{itemize}
\begin{itemize}
\item {Utilização:Prov.}
\end{itemize}
\begin{itemize}
\item {Utilização:beir.}
\end{itemize}
\begin{itemize}
\item {Proveniência:(De \textunderscore longo\textunderscore )}
\end{itemize}
Longo, comprido.
Diz-se de uma variedade de azeitona, também conhecida por \textunderscore conserva\textunderscore , ou \textunderscore sevilhana\textunderscore , ou \textunderscore regalona\textunderscore , ou \textunderscore santulhana\textunderscore .
Diz-se de uma espécie de castanha.
\section{Longamente}
\begin{itemize}
\item {Grp. gram.:adv.}
\end{itemize}
\begin{itemize}
\item {Proveniência:(De \textunderscore longo\textunderscore )}
\end{itemize}
Extensamente; por muito tempo.
\section{Longana}
\begin{itemize}
\item {Grp. gram.:f.}
\end{itemize}
Planta sapindácea do Brasil, (\textunderscore euphoria longana\textunderscore ).
\section{Longanimamente}
\begin{itemize}
\item {Grp. gram.:adv.}
\end{itemize}
De modo longânime; com generosidade.
\section{Longânime}
\begin{itemize}
\item {Grp. gram.:adj.}
\end{itemize}
\begin{itemize}
\item {Proveniência:(Lat. \textunderscore longanimis\textunderscore )}
\end{itemize}
Que tem grandeza de ânimo.
Corajoso; resignado.
Generoso.
\section{Longanimemente}
\begin{itemize}
\item {Grp. gram.:adv.}
\end{itemize}
De modo longânime; com generosidade.
\section{Longanimidade}
\begin{itemize}
\item {Grp. gram.:f.}
\end{itemize}
\begin{itemize}
\item {Proveniência:(Lat. \textunderscore longanimitas\textunderscore )}
\end{itemize}
Qualidade de quem é longânime.
\section{Longânimo}
\begin{itemize}
\item {Grp. gram.:adj.}
\end{itemize}
(V.longânime)
\section{Longarela}
\begin{itemize}
\item {Grp. gram.:m.  e  f.}
\end{itemize}
\begin{itemize}
\item {Utilização:Chul.}
\end{itemize}
\begin{itemize}
\item {Proveniência:(Do rad. de \textunderscore longo\textunderscore )}
\end{itemize}
Pessôa muito alta e delgada.
\section{Longariça}
\begin{itemize}
\item {Grp. gram.:f.}
\end{itemize}
\begin{itemize}
\item {Utilização:Ant.}
\end{itemize}
\begin{itemize}
\item {Proveniência:(De \textunderscore longo\textunderscore . Cp. cast. \textunderscore longaniza\textunderscore )}
\end{itemize}
O mesmo que \textunderscore linguiça\textunderscore .
\section{Longarina}
\begin{itemize}
\item {Grp. gram.:f.}
\end{itemize}
\begin{itemize}
\item {Proveniência:(De \textunderscore longo\textunderscore )}
\end{itemize}
Cada uma das duas vigas, em que assenta o tabuleiro das pontes.
Cp. \textunderscore longrina\textunderscore .
\section{Longarino}
\begin{itemize}
\item {Grp. gram.:m.}
\end{itemize}
O mesmo que \textunderscore longarina\textunderscore .
\section{Longe}
\begin{itemize}
\item {Grp. gram.:adv.}
\end{itemize}
\begin{itemize}
\item {Grp. gram.:Loc. adv.}
\end{itemize}
\begin{itemize}
\item {Grp. gram.:M. pl.}
\end{itemize}
\begin{itemize}
\item {Utilização:Fig.}
\end{itemize}
\begin{itemize}
\item {Grp. gram.:Adj.}
\end{itemize}
\begin{itemize}
\item {Proveniência:(Lat. \textunderscore longe\textunderscore )}
\end{itemize}
A grande distância de uma época ou de um lugar.
\textunderscore De longe em longe\textunderscore , raramente, com grandes intervallos.
Objectos, que se representam numa tela como distantes.
Grande distância.
Semelhança, ares: \textunderscore dá uns longes do pai\textunderscore .
Suspeita.
Distânte: \textunderscore andei lá por longes terras\textunderscore . J. de Lemos, \textunderscore Lua de Londres\textunderscore .
\section{Longemente}
\begin{itemize}
\item {Grp. gram.:adv.}
\end{itemize}
Até longe; até grande distância.
Amplamente:«\textunderscore ...hervas que parecião longemente se estenderem...\textunderscore »Filinto, \textunderscore D. Man.\textunderscore , I, 325.
\section{Longerão}
\begin{itemize}
\item {Grp. gram.:m.}
\end{itemize}
\begin{itemize}
\item {Utilização:Bras}
\end{itemize}
\begin{itemize}
\item {Proveniência:(Fr. \textunderscore longeron\textunderscore )}
\end{itemize}
Viga, sobre que assenta um apparelho ou máquina. Cf. \textunderscore Diário Official\textunderscore , do Brasil, de 21-IV-901.
\section{Longevidade}
\begin{itemize}
\item {Grp. gram.:f.}
\end{itemize}
\begin{itemize}
\item {Proveniência:(Lat. \textunderscore longaevitas\textunderscore )}
\end{itemize}
Qualidade de quem é longevo.
\section{Longevo}
\begin{itemize}
\item {Grp. gram.:adj.}
\end{itemize}
\begin{itemize}
\item {Utilização:Poét.}
\end{itemize}
\begin{itemize}
\item {Proveniência:(Lat. \textunderscore longaevus\textunderscore )}
\end{itemize}
Que dura muito; que tem muita idade; macróbio.
\section{Longicaule}
\begin{itemize}
\item {Grp. gram.:adj.}
\end{itemize}
\begin{itemize}
\item {Utilização:Bot.}
\end{itemize}
\begin{itemize}
\item {Proveniência:(Do lat. \textunderscore longus\textunderscore  + \textunderscore caulis\textunderscore )}
\end{itemize}
Que tem haste longa.
\section{Longicórneo}
\begin{itemize}
\item {Grp. gram.:adj.}
\end{itemize}
\begin{itemize}
\item {Utilização:Zool.}
\end{itemize}
\begin{itemize}
\item {Grp. gram.:M. pl.}
\end{itemize}
\begin{itemize}
\item {Proveniência:(De \textunderscore longo\textunderscore  + \textunderscore corno\textunderscore )}
\end{itemize}
Que tem longos os cornos ou as antennas.
Família de insectos coleópteros tetrâmeros.
\section{Longilobado}
\begin{itemize}
\item {Grp. gram.:adj.}
\end{itemize}
\begin{itemize}
\item {Utilização:Bot.}
\end{itemize}
\begin{itemize}
\item {Proveniência:(De \textunderscore longo\textunderscore  + \textunderscore lobado\textunderscore )}
\end{itemize}
Dividido em lóbulos alongados.
\section{Longímano}
\begin{itemize}
\item {Grp. gram.:adj.}
\end{itemize}
\begin{itemize}
\item {Proveniência:(Lat. \textunderscore longimanus\textunderscore )}
\end{itemize}
Que tem mãos longas.
\section{Longimetria}
\begin{itemize}
\item {Grp. gram.:f.}
\end{itemize}
\begin{itemize}
\item {Proveniência:(De \textunderscore longo\textunderscore  + gr. \textunderscore metron\textunderscore )}
\end{itemize}
Arte de medir as distâncias, por meio da trigonometria.
\section{Longina}
\begin{itemize}
\item {Grp. gram.:f.}
\end{itemize}
Insecto díptero do Brasil.
\section{Longinquamente}
\begin{itemize}
\item {Grp. gram.:adv.}
\end{itemize}
Em lugar longinquo.
Para longe.
\section{Longinquidade}
\begin{itemize}
\item {fónica:cu-i}
\end{itemize}
\begin{itemize}
\item {Grp. gram.:f.}
\end{itemize}
\begin{itemize}
\item {Proveniência:(De \textunderscore longínquo\textunderscore )}
\end{itemize}
Distância grande.
Afastamento.
\section{Longínquo}
\begin{itemize}
\item {Grp. gram.:adj.}
\end{itemize}
\begin{itemize}
\item {Proveniência:(Lat. \textunderscore longinquus\textunderscore )}
\end{itemize}
Que vem de longe: \textunderscore sons longínquos\textunderscore .
Que está distante da nossa vista ou dos nossos ouvidos: \textunderscore povos longínquos\textunderscore .
Afastado, remoto.
Porvindoiro.
\section{Longipalpo}
\begin{itemize}
\item {Grp. gram.:adj.}
\end{itemize}
\begin{itemize}
\item {Utilização:Zool.}
\end{itemize}
\begin{itemize}
\item {Proveniência:(De \textunderscore longo\textunderscore  + \textunderscore palpo\textunderscore )}
\end{itemize}
Que tem palpos longos.
\section{Longípede}
\begin{itemize}
\item {Grp. gram.:adj.}
\end{itemize}
\begin{itemize}
\item {Proveniência:(Lat. \textunderscore longipes\textunderscore )}
\end{itemize}
Que tem pés compridos.
\section{Longipene}
\begin{itemize}
\item {Grp. gram.:adj.}
\end{itemize}
\begin{itemize}
\item {Utilização:Zool.}
\end{itemize}
\begin{itemize}
\item {Grp. gram.:M. pl.}
\end{itemize}
\begin{itemize}
\item {Proveniência:(De \textunderscore longo\textunderscore  + \textunderscore penna\textunderscore )}
\end{itemize}
Que tem pennas compridas.
Família de aves palmípedes.
\section{Longipenne}
\begin{itemize}
\item {Grp. gram.:adj.}
\end{itemize}
\begin{itemize}
\item {Utilização:Zool.}
\end{itemize}
\begin{itemize}
\item {Grp. gram.:M. pl.}
\end{itemize}
\begin{itemize}
\item {Proveniência:(De \textunderscore longo\textunderscore  + \textunderscore penna\textunderscore )}
\end{itemize}
Que tem pennas compridas.
Família de aves palmípedes.
\section{Lofo}
\begin{itemize}
\item {Grp. gram.:m.}
\end{itemize}
\begin{itemize}
\item {Proveniência:(Gr. \textunderscore lophos\textunderscore )}
\end{itemize}
Peixe, de movimentos extravagantes, que vive nas profundidades do Pacífico.
\section{Lofobrânquios}
\begin{itemize}
\item {Grp. gram.:m. pl.}
\end{itemize}
\begin{itemize}
\item {Utilização:Zool.}
\end{itemize}
Ordem de peixes, que têm por tipo o hipocampo.
(Po gr. \textunderscore lophos\textunderscore  + \textunderscore brankhia\textunderscore )
\section{Lofócomo}
\begin{itemize}
\item {Grp. gram.:adj.}
\end{itemize}
\begin{itemize}
\item {Proveniência:(Do gr. \textunderscore lophos\textunderscore  + \textunderscore tome\textunderscore )}
\end{itemize}
Que tem o cabelo erriçado ou em fórma de penacho.
\section{Lofófito}
\begin{itemize}
\item {Grp. gram.:m.}
\end{itemize}
\begin{itemize}
\item {Proveniência:(Do gr. \textunderscore lophos\textunderscore  + \textunderscore phuton\textunderscore )}
\end{itemize}
Planta brasileira, parasita de certas árvores.
\section{Lofospérmia}
\begin{itemize}
\item {Grp. gram.:f.}
\end{itemize}
Gênero de plantas de jardim.
\section{Lofote}
\begin{itemize}
\item {Grp. gram.:m.}
\end{itemize}
\begin{itemize}
\item {Proveniência:(Do gr. \textunderscore lophos\textunderscore )}
\end{itemize}
Gênero de peixes acantopterígios.
O mesmo que \textunderscore lofo\textunderscore ^2?
\section{Longipétalo}
\begin{itemize}
\item {Grp. gram.:adj.}
\end{itemize}
\begin{itemize}
\item {Utilização:Bot.}
\end{itemize}
\begin{itemize}
\item {Proveniência:(De \textunderscore longo\textunderscore  + \textunderscore pétala\textunderscore )}
\end{itemize}
Que tem pétalas longas.
\section{Longirostro}
\begin{itemize}
\item {fónica:rós}
\end{itemize}
\begin{itemize}
\item {Grp. gram.:adj.}
\end{itemize}
\begin{itemize}
\item {Utilização:Zool.}
\end{itemize}
\begin{itemize}
\item {Grp. gram.:M. pl.}
\end{itemize}
\begin{itemize}
\item {Proveniência:(De \textunderscore longo\textunderscore  + \textunderscore rostro\textunderscore )}
\end{itemize}
Que tem bico comprido.
Família de aves pernaltas.
\section{Longirrostro}
\begin{itemize}
\item {Grp. gram.:adj.}
\end{itemize}
\begin{itemize}
\item {Utilização:Zool.}
\end{itemize}
\begin{itemize}
\item {Grp. gram.:M. pl.}
\end{itemize}
\begin{itemize}
\item {Proveniência:(De \textunderscore longo\textunderscore  + \textunderscore rostro\textunderscore )}
\end{itemize}
Que tem bico comprido.
Família de aves pernaltas.
\section{Longíssimo}
\begin{itemize}
\item {Grp. gram.:adv.}
\end{itemize}
Muito longe.
\section{Longitarso}
\begin{itemize}
\item {Grp. gram.:adj.}
\end{itemize}
\begin{itemize}
\item {Utilização:Anat.}
\end{itemize}
\begin{itemize}
\item {Proveniência:(De \textunderscore longo\textunderscore  + \textunderscore tarso\textunderscore )}
\end{itemize}
Que tem o tarso longo.
\section{Longitroante}
\begin{itemize}
\item {Grp. gram.:adj.}
\end{itemize}
\begin{itemize}
\item {Proveniência:(De \textunderscore longe\textunderscore  ou \textunderscore longo\textunderscore  + \textunderscore troar\textunderscore )}
\end{itemize}
Que trôa ao longe.
Que rebôa por muito tempo. Cf. Castilho, \textunderscore Fastos\textunderscore , I, 91.
\section{Longitude}
\begin{itemize}
\item {Grp. gram.:f.}
\end{itemize}
\begin{itemize}
\item {Utilização:Ext.}
\end{itemize}
\begin{itemize}
\item {Proveniência:(Lat. \textunderscore longitudo\textunderscore )}
\end{itemize}
Arco do Equador, comprehendido entre o meridiano inicial e o meridiano de qualquer lugar.
Ângulo diedro, formado pelo plano do meridiano de um lugar com o plano do meridiano inicial ou principal.
O mesmo que \textunderscore distância\textunderscore .
\section{Longitudinal}
\begin{itemize}
\item {Grp. gram.:adj.}
\end{itemize}
\begin{itemize}
\item {Proveniência:(Lat. \textunderscore longitudinalis\textunderscore )}
\end{itemize}
Extenso em comprimento.
Collocado ao comprido.
Que está na direcção do comprimento de um objecto ou órgão, ou no sentido do eixo principal de um órgão.
\section{Longitudinalmente}
\begin{itemize}
\item {Grp. gram.:adv.}
\end{itemize}
De modo longitudinal.
No sentido do comprimento.
\section{Longo}
\begin{itemize}
\item {Grp. gram.:adj.}
\end{itemize}
\begin{itemize}
\item {Grp. gram.:Loc. prep.}
\end{itemize}
\begin{itemize}
\item {Grp. gram.:Loc. adv.}
\end{itemize}
\begin{itemize}
\item {Utilização:ant.}
\end{itemize}
\begin{itemize}
\item {Proveniência:(Lat. \textunderscore longus\textunderscore )}
\end{itemize}
Extenso, no sentido do comprimento: \textunderscore rua longa\textunderscore .
Que dura muito: \textunderscore vida longa\textunderscore .
\textunderscore Ao longo de\textunderscore , por toda a extensão, segundo o comprimento, de.
\textunderscore Á longa\textunderscore , ao longe.
\section{Longobardo}
\begin{itemize}
\item {Grp. gram.:adj.  e  m.}
\end{itemize}
(V. \textunderscore lombardo\textunderscore ^1)
\section{Longórvia}
\begin{itemize}
\item {Grp. gram.:f.}
\end{itemize}
\begin{itemize}
\item {Utilização:Prov.}
\end{itemize}
\begin{itemize}
\item {Utilização:trasm.}
\end{itemize}
\begin{itemize}
\item {Proveniência:(De \textunderscore longo\textunderscore )}
\end{itemize}
Mulher alta e magra.
\section{Longrina}
\begin{itemize}
\item {Grp. gram.:f.}
\end{itemize}
\begin{itemize}
\item {Proveniência:(Fr. \textunderscore longrine\textunderscore )}
\end{itemize}
Viga, sôbre que se pregam as travessas dos carris de ferro.
Peça comprida, que se sobrepõe longitudinalmente a uma estacaria. Cf. F. Lapa, \textunderscore Phys.\textunderscore , 70. Cp. \textunderscore longarina\textunderscore .
\section{Longueça}
\begin{itemize}
\item {fónica:guê}
\end{itemize}
\begin{itemize}
\item {Grp. gram.:f.}
\end{itemize}
\begin{itemize}
\item {Utilização:Ant.}
\end{itemize}
O mesmo que \textunderscore longueza\textunderscore .--Us. por Damião de Goes.
\section{Longueirão}
\begin{itemize}
\item {Grp. gram.:m.}
\end{itemize}
\begin{itemize}
\item {Grp. gram.:Adj.}
\end{itemize}
O mesmo ou melhor que \textunderscore lingueirão\textunderscore .
Muito longe. Cf. B. Pereira, Prosódia, vb. \textunderscore longurio\textunderscore .
(Cp. lat. \textunderscore longurio\textunderscore )
\section{Longueiro}
\begin{itemize}
\item {Grp. gram.:adj.}
\end{itemize}
\begin{itemize}
\item {Utilização:Ant.}
\end{itemize}
\begin{itemize}
\item {Utilização:Ant.}
\end{itemize}
O mesmo que \textunderscore longo\textunderscore .
Demorado. Cf. \textunderscore Aulegrafia\textunderscore , 118.
\section{Longuere}
\begin{itemize}
\item {Grp. gram.:m.}
\end{itemize}
Nome africano de um reptil sáurio.
\section{Longueza}
\begin{itemize}
\item {Grp. gram.:f.}
\end{itemize}
\begin{itemize}
\item {Utilização:Des.}
\end{itemize}
O mesmo que \textunderscore longuidão\textunderscore .
\section{Longuiça}
\begin{itemize}
\item {Grp. gram.:f.}
\end{itemize}
\begin{itemize}
\item {Utilização:alg.}
\end{itemize}
\begin{itemize}
\item {Utilização:Bras}
\end{itemize}
O mesmo que \textunderscore longuriça\textunderscore .
\section{Longuidão}
\begin{itemize}
\item {Grp. gram.:f.}
\end{itemize}
Qualidade daquillo que é longo; comprimento.--Us. por Castilho.
\section{Longura}
\begin{itemize}
\item {Grp. gram.:f.}
\end{itemize}
\begin{itemize}
\item {Utilização:Fig.}
\end{itemize}
\begin{itemize}
\item {Proveniência:(De \textunderscore longo\textunderscore )}
\end{itemize}
O mesmo que \textunderscore longuidão\textunderscore .
Delonga.
\section{Longuriça}
\begin{itemize}
\item {Grp. gram.:f.}
\end{itemize}
\begin{itemize}
\item {Utilização:Prov.}
\end{itemize}
\begin{itemize}
\item {Utilização:alg.}
\end{itemize}
O mesmo que \textunderscore chouriça\textunderscore .
(Cp. \textunderscore linguiça\textunderscore )
\section{Lonicera}
\begin{itemize}
\item {Grp. gram.:f.}
\end{itemize}
\begin{itemize}
\item {Proveniência:(De \textunderscore Lonicer\textunderscore , n. p.)}
\end{itemize}
Nome scientífico da madresilva.
\section{Loniceráceas}
\begin{itemize}
\item {Grp. gram.:f. pl.}
\end{itemize}
O mesmo ou melhor que \textunderscore lonicéreas\textunderscore .
\section{Lonicéreas}
\begin{itemize}
\item {Grp. gram.:f. pl.}
\end{itemize}
\begin{itemize}
\item {Proveniência:(De \textunderscore lonicéreo\textunderscore )}
\end{itemize}
Família de plantas, o mesmo que \textunderscore caprifoliáceas\textunderscore , que tem por typo a madresilva ou a lonicera.
\section{Lonicéreo}
\begin{itemize}
\item {Grp. gram.:adj.}
\end{itemize}
Relativo ou semelhante á lonicera.
\section{Lonjura}
\begin{itemize}
\item {Grp. gram.:f.}
\end{itemize}
\begin{itemize}
\item {Utilização:Pop.}
\end{itemize}
\begin{itemize}
\item {Proveniência:(De \textunderscore longe\textunderscore )}
\end{itemize}
Grande distância: \textunderscore cheguei agora de Coímbra, que lonjura! dia, e meio de viagem! A lonjura de Paris é muito maior que a de Madrid\textunderscore .
\section{Lonquear}
\begin{itemize}
\item {Grp. gram.:v. t.}
\end{itemize}
\begin{itemize}
\item {Utilização:Bras. do S}
\end{itemize}
\begin{itemize}
\item {Proveniência:(De \textunderscore lonca\textunderscore )}
\end{itemize}
Raspar o pêlo a (uma rês), sem ferir o coiro.
\section{Lontra}
\begin{itemize}
\item {Grp. gram.:f.}
\end{itemize}
\begin{itemize}
\item {Grp. gram.:M.}
\end{itemize}
\begin{itemize}
\item {Utilização:Prov.}
\end{itemize}
\begin{itemize}
\item {Utilização:trasm.}
\end{itemize}
\begin{itemize}
\item {Proveniência:(Do b. lat. \textunderscore luntria\textunderscore )}
\end{itemize}
Pequeno quadrúpede carnívoro, da fam. das martas.
Pescador dos rios, afamado.
\section{Looque}
\begin{itemize}
\item {Grp. gram.:m.}
\end{itemize}
\begin{itemize}
\item {Proveniência:(Do ár. \textunderscore looq\textunderscore )}
\end{itemize}
Medicamento líquido, consistente como um xarope espêsso e applicado em doenças de pulmão, larynge, etc.
\section{Lopa}
\begin{itemize}
\item {Grp. gram.:f.}
\end{itemize}
\begin{itemize}
\item {Utilização:T. da Áfr. Occid. Port}
\end{itemize}
Algodão, tinto de azul.
\section{Lopano}
\begin{itemize}
\item {Grp. gram.:m.}
\end{itemize}
Espécie de batel?«\textunderscore ...meteose em hũ lopano por sob a banda da galé\textunderscore ...»Fernão Lopes, \textunderscore Chrón. de D. João I\textunderscore , p. I, c. 139.
\section{Lopes}
\begin{itemize}
\item {Grp. gram.:m.}
\end{itemize}
\begin{itemize}
\item {Utilização:Prov.}
\end{itemize}
\begin{itemize}
\item {Utilização:trasm.}
\end{itemize}
O mesmo que \textunderscore có-có\textunderscore ^1.
\section{Lopho}
\begin{itemize}
\item {Grp. gram.:m.}
\end{itemize}
\begin{itemize}
\item {Proveniência:(Gr. \textunderscore lophos\textunderscore )}
\end{itemize}
Peixe, de movimentos extravagantes, que vive nas profundidades do Pacífico.
\section{Lophobrânchios}
\begin{itemize}
\item {fónica:qui}
\end{itemize}
\begin{itemize}
\item {Grp. gram.:m. pl.}
\end{itemize}
\begin{itemize}
\item {Utilização:Zool.}
\end{itemize}
Ordem de peixes, que têm por typo o hippocampo.
(Po gr. \textunderscore lophos\textunderscore  + \textunderscore brankhia\textunderscore )
\section{Lophócomo}
\begin{itemize}
\item {Grp. gram.:adj.}
\end{itemize}
\begin{itemize}
\item {Proveniência:(Do gr. \textunderscore lophos\textunderscore  + \textunderscore tome\textunderscore )}
\end{itemize}
Que tem o cabello erriçado ou em fórma de pennacho.
\section{Lophóphyto}
\begin{itemize}
\item {Grp. gram.:m.}
\end{itemize}
\begin{itemize}
\item {Proveniência:(Do gr. \textunderscore lophos\textunderscore  + \textunderscore phuton\textunderscore )}
\end{itemize}
Planta brasileira, parasita de certas árvores.
\section{Lophospérmia}
\begin{itemize}
\item {Grp. gram.:f.}
\end{itemize}
Gênero de plantas de jardim.
\section{Lophote}
\begin{itemize}
\item {Grp. gram.:m.}
\end{itemize}
\begin{itemize}
\item {Proveniência:(Do gr. \textunderscore lophos\textunderscore )}
\end{itemize}
Gênero de peixes acanthopterýgios.
O mesmo que \textunderscore lopho\textunderscore ?
\section{Loquacidade}
\begin{itemize}
\item {Grp. gram.:f.}
\end{itemize}
\begin{itemize}
\item {Proveniência:(Lat. \textunderscore loquacitas\textunderscore )}
\end{itemize}
Qualidade de quem é loquaz.
\section{Loquaz}
\begin{itemize}
\item {Grp. gram.:adj.}
\end{itemize}
\begin{itemize}
\item {Utilização:Ext.}
\end{itemize}
\begin{itemize}
\item {Utilização:Fig.}
\end{itemize}
\begin{itemize}
\item {Proveniência:(Lat. \textunderscore loquax\textunderscore )}
\end{itemize}
Que fala muito ou com facilidade; palrador.
Eloquente.
Que produz grande rumor.
\section{Loquela}
\begin{itemize}
\item {fónica:cu-é}
\end{itemize}
\begin{itemize}
\item {Grp. gram.:f.}
\end{itemize}
\begin{itemize}
\item {Utilização:Ext.}
\end{itemize}
\begin{itemize}
\item {Proveniência:(Lat. \textunderscore loquela\textunderscore )}
\end{itemize}
Faculdade de falar.
Linguagem.
Loquacidade.
\section{Loquete}
\begin{itemize}
\item {fónica:quê}
\end{itemize}
\begin{itemize}
\item {Grp. gram.:m.}
\end{itemize}
\begin{itemize}
\item {Proveniência:(Fr. \textunderscore loquet\textunderscore )}
\end{itemize}
Cadeado, embude.
\section{Loquial}
\begin{itemize}
\item {Grp. gram.:adj.}
\end{itemize}
Relativo aos lóquios.
\section{Lóquios}
\begin{itemize}
\item {Grp. gram.:m. pl.}
\end{itemize}
\begin{itemize}
\item {Proveniência:(Do gr. \textunderscore lokheia\textunderscore )}
\end{itemize}
Evacuação sanguinolenta, que succede aos partos.
\section{Lora}
\begin{itemize}
\item {Grp. gram.:f.}
\end{itemize}
\begin{itemize}
\item {Utilização:Prov.}
\end{itemize}
O mesmo que \textunderscore lura\textunderscore .
(Cp. \textunderscore laura\textunderscore )
\section{Lora}
\begin{itemize}
\item {Grp. gram.:f.}
\end{itemize}
\begin{itemize}
\item {Utilização:Bot.}
\end{itemize}
A parte vivaz e filamentosa, despida de fôlhas, em certos líchens.
\section{Lorantáceas}
\begin{itemize}
\item {Grp. gram.:f. pl.}
\end{itemize}
\begin{itemize}
\item {Proveniência:(De \textunderscore lorantáceo\textunderscore )}
\end{itemize}
Família de plantas, que tem por tipo o loranto.
\section{Lorantáceo}
\begin{itemize}
\item {Grp. gram.:adj.}
\end{itemize}
Relativo ou semelhante ao loranto.
\section{Lorantháceas}
\begin{itemize}
\item {Grp. gram.:f. pl.}
\end{itemize}
\begin{itemize}
\item {Proveniência:(De \textunderscore lorantháceo\textunderscore )}
\end{itemize}
Família de plantas, que tem por typo o lorantho.
\section{Lorantháceo}
\begin{itemize}
\item {Grp. gram.:adj.}
\end{itemize}
Relativo ou semelhante ao lorantho.
\section{Lorantho}
\begin{itemize}
\item {Grp. gram.:m.}
\end{itemize}
\begin{itemize}
\item {Proveniência:(Do gr. \textunderscore loron\textunderscore  + \textunderscore anthos\textunderscore )}
\end{itemize}
Gênero de arbustos, de fôlhas oppostas ou alternas, espêssas ou coriáceas.
\section{Loranto}
\begin{itemize}
\item {Grp. gram.:m.}
\end{itemize}
\begin{itemize}
\item {Proveniência:(Do gr. \textunderscore loron\textunderscore  + \textunderscore anthos\textunderscore )}
\end{itemize}
Gênero de arbustos, de fôlhas opostas ou alternas, espêssas ou coriáceas.
\section{Lorário}
\begin{itemize}
\item {Grp. gram.:m.}
\end{itemize}
\begin{itemize}
\item {Proveniência:(Lat. \textunderscore lorarius\textunderscore )}
\end{itemize}
Aquelle que, entre os Romanos, azorragava os escravos.
\section{Lorca}
\begin{itemize}
\item {Grp. gram.:f.}
\end{itemize}
\begin{itemize}
\item {Utilização:Prov.}
\end{itemize}
\begin{itemize}
\item {Utilização:dur.}
\end{itemize}
Lura de coelho.
Buraco, cova: \textunderscore tenho uma lorca num dente\textunderscore .
\section{Lorcado}
\begin{itemize}
\item {Grp. gram.:adj.}
\end{itemize}
\begin{itemize}
\item {Proveniência:(De \textunderscore lorcar\textunderscore )}
\end{itemize}
Que tem lorca.
\section{Lorcar}
\begin{itemize}
\item {Grp. gram.:v. t.}
\end{itemize}
Abrir lorca em.
\section{Lorcha}
\begin{itemize}
\item {Grp. gram.:f.}
\end{itemize}
Pequena e ligeira embarcação, usada em Macau, o mesmo que \textunderscore lanteia\textunderscore . Cf. \textunderscore Peregrinação\textunderscore , XL e LXIII.
\section{Lorcha}
\begin{itemize}
\item {Grp. gram.:f.}
\end{itemize}
Fruto da África austro-central.
\section{Lord}
\begin{itemize}
\item {Grp. gram.:m.}
\end{itemize}
\begin{itemize}
\item {Utilização:Pop.}
\end{itemize}
\begin{itemize}
\item {Proveniência:(Ingl. \textunderscore lord\textunderscore )}
\end{itemize}
Título honorífico em Inglaterra.
Membro da câmara aristocrática do parlamento inglês.
Indivíduo rico.
Aquelle que vive com ostentação.
\section{Lorde}
\begin{itemize}
\item {Grp. gram.:m.}
\end{itemize}
\begin{itemize}
\item {Utilização:Pop.}
\end{itemize}
\begin{itemize}
\item {Proveniência:(Ingl. \textunderscore lord\textunderscore )}
\end{itemize}
Título honorífico em Inglaterra.
Membro da câmara aristocrática do parlamento inglês.
Indivíduo rico.
Aquelle que vive com ostentação.
\section{Lordose}
\begin{itemize}
\item {Grp. gram.:f.}
\end{itemize}
\begin{itemize}
\item {Proveniência:(Gr. \textunderscore lordosis\textunderscore )}
\end{itemize}
Encurvamento da columna vertebral para deante.
\section{Lorga}
\begin{itemize}
\item {Grp. gram.:f.}
\end{itemize}
O mesmo que \textunderscore lura\textunderscore .
Cp. \textunderscore lora\textunderscore ^1.
\section{Loriga}
\begin{itemize}
\item {Grp. gram.:f.}
\end{itemize}
\begin{itemize}
\item {Proveniência:(Lat. \textunderscore lorica\textunderscore )}
\end{itemize}
Coiraça, antiga saia de malha com lâminas de metal.
\section{Lorigado}
\begin{itemize}
\item {Grp. gram.:adj.}
\end{itemize}
\begin{itemize}
\item {Proveniência:(Do lat. \textunderscore loricatus\textunderscore )}
\end{itemize}
Revestido de loriga. Cf. Castilho, \textunderscore Fastos\textunderscore , I, 129.
\section{Lorigão}
\begin{itemize}
\item {Grp. gram.:m.}
\end{itemize}
\begin{itemize}
\item {Proveniência:(De \textunderscore loriga\textunderscore )}
\end{itemize}
Grande saia de malha. Cf. Herculano, \textunderscore Lendas e Narr.\textunderscore , II, 75.
\section{Lóris}
\begin{itemize}
\item {Grp. gram.:m.}
\end{itemize}
Espécie de lêmur indiano.(V.lêmur)
\section{Loro}
\begin{itemize}
\item {Grp. gram.:m.}
\end{itemize}
\begin{itemize}
\item {Proveniência:(Lat. \textunderscore lorum\textunderscore )}
\end{itemize}
Correia dupla, que sustenta o estribo.
Parte da cabeça das aves entre a base do bico e os olhos.
Peça da boca de alguns insectos.
Filamento de certos líchens.
\section{Lorota}
\begin{itemize}
\item {Grp. gram.:f.}
\end{itemize}
\begin{itemize}
\item {Utilização:Bras}
\end{itemize}
História mal contada.
Parola.
Bazófia.
(Por \textunderscore lerota\textunderscore , de \textunderscore léria\textunderscore ?)
\section{Lorpa}
\begin{itemize}
\item {fónica:lôr}
\end{itemize}
\begin{itemize}
\item {Grp. gram.:m.  e  adj.}
\end{itemize}
Imbecil; parvo; grosseiro.
\section{Lorpice}
\begin{itemize}
\item {Grp. gram.:f.}
\end{itemize}
Qualidade ou acção de lorpa.
\section{Losanga}
\begin{itemize}
\item {Grp. gram.:f.}
\end{itemize}
\begin{itemize}
\item {Utilização:Des.}
\end{itemize}
O mesmo que \textunderscore losanja\textunderscore .
\section{Losango}
\begin{itemize}
\item {Grp. gram.:m.}
\end{itemize}
Parallelogrammo, cujos quatro lados são iguaes, sem que os ângulos sejam rectos.
\section{Losanja}
\begin{itemize}
\item {Grp. gram.:f.}
\end{itemize}
\begin{itemize}
\item {Utilização:Des.}
\end{itemize}
\begin{itemize}
\item {Grp. gram.:Pl.}
\end{itemize}
Figura geométrica, o mesmo que \textunderscore losango\textunderscore .
Preparado pharmacêutico, composto de differentes substâncias, encorporadas por uma substância glutinosa.
\section{Losia}
\begin{itemize}
\item {Grp. gram.:f.}
\end{itemize}
\begin{itemize}
\item {Utilização:Ant.}
\end{itemize}
O mesmo que \textunderscore adussia\textunderscore .
\section{Losna}
\begin{itemize}
\item {Grp. gram.:f.}
\end{itemize}
\begin{itemize}
\item {Proveniência:(Do lat. \textunderscore aloxanum\textunderscore )}
\end{itemize}
Nome de várias plantas, uma das quaes é o mesmo que \textunderscore absintho\textunderscore .
\section{Lóstra}
\begin{itemize}
\item {Grp. gram.:f.}
\end{itemize}
\begin{itemize}
\item {Utilização:Gír.}
\end{itemize}
\begin{itemize}
\item {Utilização:Prov.}
\end{itemize}
\begin{itemize}
\item {Utilização:minh.}
\end{itemize}
Escarro.
Bofetada.
Vergastada.
\section{Lôstra}
\begin{itemize}
\item {Grp. gram.:f.}
\end{itemize}
\begin{itemize}
\item {Utilização:Prov.}
\end{itemize}
\begin{itemize}
\item {Utilização:beir.}
\end{itemize}
Mulher desmazelada, indolente.
\section{Lota}
\begin{itemize}
\item {Grp. gram.:f.}
\end{itemize}
\begin{itemize}
\item {Utilização:Prov.}
\end{itemize}
\begin{itemize}
\item {Utilização:minh.}
\end{itemize}
\begin{itemize}
\item {Proveniência:(De \textunderscore lotar\textunderscore )}
\end{itemize}
Lugar, onde se orçam os direitos que têm de sêr pagos por pescadores.
Lugar, onde se vende peixe.
Porção do peixe que se vende.
Certo modo de vender o peixe.

Pequena quantidade. (Colhido em Varzim)
\textunderscore Vender á lota\textunderscore , marcar o vendedor um preço alto, que vai baixando, até que convém a um comprador.
\section{Lotação}
\begin{itemize}
\item {Grp. gram.:f.}
\end{itemize}
Acto ou effeito de lotar.
\section{Lotador}
\begin{itemize}
\item {Grp. gram.:m.}
\end{itemize}
Aquelle que lota.
Aquelle que faz lotes.
Apparelho que, no fabrico da pólvora, distribue esta em lotes, com propriedades iguaes.
\section{Lotar}
\begin{itemize}
\item {Grp. gram.:v. t.}
\end{itemize}
\begin{itemize}
\item {Proveniência:(De \textunderscore lote\textunderscore )}
\end{itemize}
Dividir em lotes.
Calcular.
Sortear.
Misturar (um vinho) com outro ou outros, transmittindo-lhe as qualidades dêstes.
Computar a capacidade de (uma embarcação).
\section{Lotaria}
\begin{itemize}
\item {Grp. gram.:f.}
\end{itemize}
\begin{itemize}
\item {Proveniência:(De \textunderscore lota\textunderscore )}
\end{itemize}
Jôgo de asar, por meio de bilhetes numerados ou fracções dêstes, com o fim de se obterem prêmios pecuniários, que são indicados por sorteio.
Rifa.
Espécie de jôgo de cartas.
\section{Lotárico}
\begin{itemize}
\item {Grp. gram.:adj.}
\end{itemize}
\begin{itemize}
\item {Utilização:Bras}
\end{itemize}
Relativo a lotaria: \textunderscore jôgo lotárico\textunderscore .
\section{Lotaríngios}
\begin{itemize}
\item {Grp. gram.:m. pl.}
\end{itemize}
Habitantes da Lotharíngia, antigo ducado da França. Cf. Herculano, \textunderscore Hist. de Port.\textunderscore , I, 372.
\section{Lote}
\begin{itemize}
\item {Grp. gram.:m.}
\end{itemize}
\begin{itemize}
\item {Utilização:Bras}
\end{itemize}
\begin{itemize}
\item {Proveniência:(Do gót. \textunderscore hlauts\textunderscore )}
\end{itemize}
Carla uma das partes de um todo, que se reparte por vários indivíduos.
Magote de pessôas.
Reunião de objectos, que se põem conjuntamente em leilão.
Qualidade.
Padrão.
Lotação de um navio.
Grupo de bêstas de carga, cujo número não excede ordinariamente déz.
\section{Lóteas}
\begin{itemize}
\item {Grp. gram.:f. pl.}
\end{itemize}
\begin{itemize}
\item {Proveniência:(De \textunderscore lóto\textunderscore )}
\end{itemize}
Grande tríbo de plantas leguminosas, no systema de De-Candolle.
\section{Loteria}
\begin{itemize}
\item {Grp. gram.:f.}
\end{itemize}
(V.lotaria)
\section{Lotharíngios}
\begin{itemize}
\item {Grp. gram.:m. pl.}
\end{itemize}
Habitantes da Lotharíngia, antigo ducado da França. Cf. Herculano, \textunderscore Hist. de Port.\textunderscore , I, 372.
\section{Lôto}
\begin{itemize}
\item {Grp. gram.:m.}
\end{itemize}
Jogo do asar, que se pratica com 24 cartões e pequenas espheras ou círculos que se tiram á sorte e se vão collocando ou vão sendo representados nos respectivos números, inscritos naquelles cartões, até que um dos parceiros tenha preenchido os cinco números de uma das ordens de um cartão, tendo por isso direito ao bolo ou somma das entradas dos parceiros.
Conjunto de objectos, com que se joga o lôto.
(B. lat. \textunderscore lotus\textunderscore )
\section{Lóto}
\begin{itemize}
\item {Grp. gram.:m.}
\end{itemize}
O mesmo que \textunderscore lódão\textunderscore .
\section{Lotófago}
\begin{itemize}
\item {Grp. gram.:adj.}
\end{itemize}
\begin{itemize}
\item {Proveniência:(Do gr. \textunderscore lotos\textunderscore  + \textunderscore phagein\textunderscore )}
\end{itemize}
Que se alimenta de lódão, (aludindo-se a certos indivíduos mitológicos)
\section{Lotóphago}
\begin{itemize}
\item {Grp. gram.:adj.}
\end{itemize}
\begin{itemize}
\item {Proveniência:(Do gr. \textunderscore lotos\textunderscore  + \textunderscore phagein\textunderscore )}
\end{itemize}
Que se alimenta de lódão, (alludindo-se a certos indivíduos mythológicos).
\section{Lotos}
\begin{itemize}
\item {Grp. gram.:m.}
\end{itemize}
(V.lódão)
\section{Lótus}
\begin{itemize}
\item {Grp. gram.:m.}
\end{itemize}
O mesmo que \textunderscore lódão\textunderscore .
\section{Louça}
\begin{itemize}
\item {Grp. gram.:f.}
\end{itemize}
\begin{itemize}
\item {Utilização:Prov.}
\end{itemize}
\begin{itemize}
\item {Utilização:Pop.}
\end{itemize}
\begin{itemize}
\item {Grp. gram.:Pl.}
\end{itemize}
\begin{itemize}
\item {Proveniência:(Do lat. \textunderscore luteus\textunderscore )}
\end{itemize}
Productos de cerâmica.
Barro, porcelana ou outras substâncias análogas, manufacturadas por oleiro, para serviço de mesa especialmente.
Vasilhame.
Chocalho para o pescoço do gado.
Coisa excellente.
Depósito geral das águas, que devem alimentar a salina, e que abrange o caldeiro, a caldeira e o pejo.
\section{Louçaínha}
\begin{itemize}
\item {Grp. gram.:f.}
\end{itemize}
Vestuário muito ataviado.
Objecto garrido.
Garridice; enfeite.
(Cp. \textunderscore louçaínho\textunderscore )
\section{Louçainhar}
\begin{itemize}
\item {fónica:ça-i}
\end{itemize}
\begin{itemize}
\item {Grp. gram.:v. t.}
\end{itemize}
\begin{itemize}
\item {Proveniência:(De \textunderscore louçaínho\textunderscore )}
\end{itemize}
Tornar loução.
Capitular de louçania. Cf. Filinto, XI, 196.
\section{Louçaínho}
\begin{itemize}
\item {Grp. gram.:adj.}
\end{itemize}
\begin{itemize}
\item {Proveniência:(De \textunderscore loução\textunderscore )}
\end{itemize}
Que tem louçania.
Ornado de louçaínhas; garrido.
\section{Loucamente}
\begin{itemize}
\item {Grp. gram.:adv.}
\end{itemize}
De modo louco.
Impensadamente, desvairadamente; á tôa.
\section{Louçania}
\begin{itemize}
\item {Grp. gram.:f.}
\end{itemize}
Qualidade de loução.
Enfeites.
Elegância.
(Cast. \textunderscore lozania\textunderscore )
\section{Louçanmente}
\begin{itemize}
\item {Grp. gram.:adv.}
\end{itemize}
De modo loução; com garridice.
\section{Loução}
\begin{itemize}
\item {Grp. gram.:adj.}
\end{itemize}
Garrido; que tem louçaínhas.
Elegante; garboso; bello.
(Cp. cast. \textunderscore lozano\textunderscore )
\section{Louçaria}
\begin{itemize}
\item {Grp. gram.:f.}
\end{itemize}
Estabelecimento, onde se vende louça.
Louças, conjunto de louças.
\section{Louceira}
\begin{itemize}
\item {Grp. gram.:f.}
\end{itemize}
Vendedora de louça.
Guarda-louça.
(Cp. \textunderscore louceiro\textunderscore )
\section{Louceiro}
\begin{itemize}
\item {Grp. gram.:m.}
\end{itemize}
\begin{itemize}
\item {Utilização:Prov.}
\end{itemize}
\begin{itemize}
\item {Utilização:Prov.}
\end{itemize}
\begin{itemize}
\item {Utilização:Prov.}
\end{itemize}
\begin{itemize}
\item {Utilização:minh.}
\end{itemize}
Fabricante ou negociante de louça.
Vasilha de adega.
Utensílio, formado de um tronco vertical, com galhos, para nestes se pendurar a louça da cozinha.
Taboleiro, em que se põe louça.
Armário para louça; guarda-louça.
\section{Louco}
\begin{itemize}
\item {Grp. gram.:adj.}
\end{itemize}
\begin{itemize}
\item {Grp. gram.:M.}
\end{itemize}
\begin{itemize}
\item {Proveniência:(Do lat. \textunderscore elucus\textunderscore ?)}
\end{itemize}
Que perdeu o uso da razão.
Demente.
Extravagante.
Imprudente; temerário.
Estulto.
Muito galhofeiro; estroina.
Apaixonado.
Furioso.
Aquelle que perdeu o uso da razão.
Aquelle que é extravagante.
\section{Louco}
\begin{itemize}
\item {Grp. gram.:adj.}
\end{itemize}
\begin{itemize}
\item {Utilização:Prov.}
\end{itemize}
\begin{itemize}
\item {Utilização:minh.}
\end{itemize}
O mesmo que \textunderscore viçoso\textunderscore , (falando-se de plantas).
\section{Loucura}
\begin{itemize}
\item {Grp. gram.:f.}
\end{itemize}
Estado de quem é louco.
Acto próprio de louco.
Falta de tino ou de prudência; insensatez.
Grande extravagância.
Aventura insensata.
\section{Loudel}
\begin{itemize}
\item {Grp. gram.:m.}
\end{itemize}
(V.laudel)
\section{Louletano}
\begin{itemize}
\item {Grp. gram.:adj.}
\end{itemize}
\begin{itemize}
\item {Grp. gram.:M.}
\end{itemize}
Relativo a Loulé.
Aquelle que é natural de Loulé.
\section{Louquejar}
\begin{itemize}
\item {Grp. gram.:v. i.}
\end{itemize}
\begin{itemize}
\item {Proveniência:(De \textunderscore louco\textunderscore )}
\end{itemize}
Fazer ou dizer loucuras.
Fazer imprudências.
Commeter diabruras.
\section{Louquice}
\begin{itemize}
\item {Grp. gram.:f.}
\end{itemize}
O mesmo que \textunderscore loucura\textunderscore .
\section{Loura}
\begin{itemize}
\item {Grp. gram.:f.}
\end{itemize}
\begin{itemize}
\item {Utilização:Fam.}
\end{itemize}
\begin{itemize}
\item {Grp. gram.:M.}
\end{itemize}
\begin{itemize}
\item {Utilização:Pop.}
\end{itemize}
\begin{itemize}
\item {Proveniência:(De \textunderscore louro\textunderscore ^1)}
\end{itemize}
Mulher, que tem o cabello louro.
Libra esterlina.
Homem bonacheirão, simplório.
\section{Loura}
\begin{itemize}
\item {Grp. gram.:f.}
\end{itemize}
O mesmo que \textunderscore lura\textunderscore .
\section{Louraça}
\begin{itemize}
\item {Grp. gram.:m.  e  f.}
\end{itemize}
\begin{itemize}
\item {Utilização:Fam.}
\end{itemize}
\begin{itemize}
\item {Proveniência:(Do rad. de \textunderscore louro\textunderscore ^1)}
\end{itemize}
Pessôa simplória.
Pessôa, que tem o cabello louro.
\section{Lourar}
\begin{itemize}
\item {Grp. gram.:v. t.  e  i.}
\end{itemize}
O mesmo que \textunderscore lourejar\textunderscore .
\section{Lourecente}
\begin{itemize}
\item {Grp. gram.:adj.}
\end{itemize}
Que lourece.
\section{Lourecer}
\begin{itemize}
\item {Grp. gram.:v. t.  e  i.}
\end{itemize}
O mesmo que \textunderscore lourejar\textunderscore .
\section{Loureira}
\begin{itemize}
\item {Grp. gram.:f.  e  adj.}
\end{itemize}
Diz-se da mulher, que deseja agradar a todos.
Provocante, seductora. Cf. F. Manuel, \textunderscore Carta de Guia de Casados\textunderscore ; Filinto, IV, 272; Camillo, \textunderscore Regicida\textunderscore , 50.
\section{Loureira}
\begin{itemize}
\item {Grp. gram.:f.}
\end{itemize}
Casta de uva branca do Minho.
\section{Loureiral}
\begin{itemize}
\item {Grp. gram.:m.}
\end{itemize}
Lugar, onde crescem loureiros.
\section{Loureiro}
\begin{itemize}
\item {Grp. gram.:m.}
\end{itemize}
\begin{itemize}
\item {Proveniência:(Do lat. \textunderscore laurarius\textunderscore )}
\end{itemize}
Árvore monopétala, sempre, verde, que produz umas bagas escuras e amargas.
\section{Lourejante}
\begin{itemize}
\item {Grp. gram.:adj.}
\end{itemize}
Que loureja. Cf. Camillo, \textunderscore Myst. de Lisb.\textunderscore , II, 106.
\section{Lourejar}
\begin{itemize}
\item {Grp. gram.:v. t.}
\end{itemize}
\begin{itemize}
\item {Grp. gram.:V. i.}
\end{itemize}
Tornar louro.
Tornar se louro; apresentar-se ou mostrar-se louro.
Amarelecer: \textunderscore o trigo já loureja\textunderscore .
\section{Lourejo}
\begin{itemize}
\item {Grp. gram.:m.}
\end{itemize}
Acto de lourejar.
Côr loura ou amarela.
\section{Lourela}
\begin{itemize}
\item {Grp. gram.:f.}
\end{itemize}
\begin{itemize}
\item {Proveniência:(De \textunderscore louro\textunderscore ^1)}
\end{itemize}
Casta de uva preta, na região do Douro.
Variedade de oliveira.
\section{Louro}
\begin{itemize}
\item {Grp. gram.:adj.}
\end{itemize}
\begin{itemize}
\item {Grp. gram.:M.}
\end{itemize}
\begin{itemize}
\item {Proveniência:(Do lat. \textunderscore aureus\textunderscore ? O \textunderscore l\textunderscore  de \textunderscore louro\textunderscore  poderia vir por intermédio do fr. \textunderscore l'or\textunderscore , como o port. \textunderscore lôba\textunderscore ^2 veio do fr. \textunderscore l'aube\textunderscore , e o port. \textunderscore léste\textunderscore  do fr. \textunderscore l'est\textunderscore )}
\end{itemize}
Que tem côr média entre o doirado e o castanho claro, como as espigas maduras do trigo.
Homem de cabello louro.
\section{Louro}
\begin{itemize}
\item {Grp. gram.:m.}
\end{itemize}
\begin{itemize}
\item {Grp. gram.:Pl.}
\end{itemize}
\begin{itemize}
\item {Proveniência:(Do lat. \textunderscore laurus\textunderscore )}
\end{itemize}
O mesmo que \textunderscore loureiro\textunderscore .
Folhas de loureiro.
Lauréis; triumphos; glória.
\section{Lousa}
\begin{itemize}
\item {Grp. gram.:f.}
\end{itemize}
\begin{itemize}
\item {Proveniência:(Do lat. hyp. \textunderscore lausa\textunderscore ?)}
\end{itemize}
Lâmina de pedra.
Ardósia.
Lápide, que cobre uma sepultura.
Lura.
Armadilha de pedra, para os pássaros; lousão.
\section{Lousã}
\begin{itemize}
\item {Grp. gram.:adj.}
\end{itemize}
Dizia-se da terra, em que há muitas lousas.
\section{Lousador}
\begin{itemize}
\item {Grp. gram.:m.}
\end{itemize}
\begin{itemize}
\item {Utilização:Bras}
\end{itemize}
O encarregado de limpar ou preparar lousas.
\section{Lousão}
\begin{itemize}
\item {Grp. gram.:m.}
\end{itemize}
Armadilha, o mesmo que \textunderscore lousa\textunderscore .
\section{Lousas}
\begin{itemize}
\item {Grp. gram.:f. pl.}
\end{itemize}
Us. na loc. \textunderscore coisas e lousas\textunderscore , diversas coisas, assumptos vários.
\section{Louseira}
\begin{itemize}
\item {Grp. gram.:f.}
\end{itemize}
Lugar, donde se extrai lousa.
\section{Louseiro}
\begin{itemize}
\item {Grp. gram.:m.}
\end{itemize}
Aquelle que extrai lousas da respectiva rocha.
Aquelle que trabalha em lousa.
\section{Lousiar}
\begin{itemize}
\item {Grp. gram.:v. t.}
\end{itemize}
\begin{itemize}
\item {Utilização:Ant.}
\end{itemize}
\begin{itemize}
\item {Proveniência:(Do lat. \textunderscore laus\textunderscore )}
\end{itemize}
Lisonjear, adular. Cf. Frei Fortun., \textunderscore Inéditos\textunderscore , 310.
\section{Lousífero}
\begin{itemize}
\item {Grp. gram.:adj.}
\end{itemize}
\begin{itemize}
\item {Proveniência:(De \textunderscore lousa\textunderscore  + lat. \textunderscore ferre\textunderscore )}
\end{itemize}
Diz-se, do terreno, em que há lousas.
\section{Lousinha}
\begin{itemize}
\item {Grp. gram.:f.}
\end{itemize}
\begin{itemize}
\item {Utilização:Prov.}
\end{itemize}
\begin{itemize}
\item {Proveniência:(De \textunderscore lousa\textunderscore )}
\end{itemize}
O mesmo que \textunderscore xisto\textunderscore ^1.
\section{Lousinho}
\begin{itemize}
\item {Grp. gram.:adj.}
\end{itemize}
\begin{itemize}
\item {Proveniência:(De \textunderscore lousa\textunderscore )}
\end{itemize}
Diz-se do terreno ou rocha, em que o xisto apparece sob a fórma laminar, como a lousa. Cf. P. Carvalho, \textunderscore Corogr. Port.\textunderscore , I, 46 e 477.
\section{Louva-a-deus}
\begin{itemize}
\item {Grp. gram.:m.}
\end{itemize}
\begin{itemize}
\item {Utilização:Mad}
\end{itemize}
Designação vulgar de um insecto, de côr entre verde-claro e amarelo, cujas patas deanteiras lembram mãos erguidas para rezar.
Ave, tambem conhecida por \textunderscore papinho\textunderscore .
\section{Louvação}
\begin{itemize}
\item {Grp. gram.:f.}
\end{itemize}
Acto ou effeito de louvar.
Avaliação, feita por peritos.
Escolha de louvados ou peritos.
\section{Louvadamente}
\begin{itemize}
\item {Grp. gram.:adv.}
\end{itemize}
De modo louvado; com louvor.
\section{Louvado}
\begin{itemize}
\item {Grp. gram.:adj.}
\end{itemize}
\begin{itemize}
\item {Grp. gram.:M.}
\end{itemize}
Que recebeu louvor; que foi objecto de louvor.
Pessôa, nomeada ou escolhida, para com outras, nas mesmas condições, proceder a uma avaliação.
Árbitro; perito.
\section{Louvador}
\begin{itemize}
\item {Grp. gram.:m.  e  adj.}
\end{itemize}
\begin{itemize}
\item {Proveniência:(Do lat. \textunderscore laudator\textunderscore )}
\end{itemize}
O que louva.
\section{Louvamento}
\begin{itemize}
\item {Grp. gram.:m.}
\end{itemize}
\begin{itemize}
\item {Proveniência:(De \textunderscore louvar\textunderscore )}
\end{itemize}
O mesmo que \textunderscore louvação\textunderscore .
\section{Louvaminha}
\begin{itemize}
\item {Grp. gram.:f.}
\end{itemize}
\begin{itemize}
\item {Proveniência:(Do rad. de \textunderscore louvar\textunderscore )}
\end{itemize}
Lisonja.
Louvor excessivo e affectado.
\section{Louvaminhar}
\begin{itemize}
\item {Grp. gram.:v. t.}
\end{itemize}
Dirigir louvaminhas a.
\section{Louvaminheiro}
\begin{itemize}
\item {Grp. gram.:m.  e  adj.}
\end{itemize}
O que tem o hábito de louvaminhar; adulador.
\section{Louvar}
\begin{itemize}
\item {Grp. gram.:v. t.}
\end{itemize}
\begin{itemize}
\item {Grp. gram.:V. p.}
\end{itemize}
\begin{itemize}
\item {Proveniência:(Do lat. \textunderscore laudare\textunderscore )}
\end{itemize}
Significar por palavras o mérito de; elogiar, gabar: \textunderscore louvar um artista\textunderscore .
Approvar, elogiando: \textunderscore louvar um projecto\textunderscore .
Exaltar.
Bem dizer.
Avaliar.
Calcular o valor de: \textunderscore louvaram o tapête em déz mil réis\textunderscore .
Gabar-se.
Acceitar a opinião de outrem: \textunderscore louvo-me no parecer do Sr. Doutor\textunderscore .
Indicar alguém para árbitro ou perito.
\section{Louvável}
\begin{itemize}
\item {Grp. gram.:adj.}
\end{itemize}
\begin{itemize}
\item {Proveniência:(Lat. \textunderscore laudabilis\textunderscore )}
\end{itemize}
Que se deve louvar.
Digno de louvor.
\section{Louvavelmente}
\begin{itemize}
\item {Grp. gram.:adv.}
\end{itemize}
De modo louvável.
\section{Loiola}
\begin{itemize}
\item {Grp. gram.:m.}
\end{itemize}
\begin{itemize}
\item {Utilização:Deprec.}
\end{itemize}
Jesuíta.
Hypócrita. Cf. Herculano, \textunderscore Reacção\textunderscore , 19.
\section{Loiolista}
\begin{itemize}
\item {Grp. gram.:m.}
\end{itemize}
\begin{itemize}
\item {Proveniência:(De \textunderscore Loyola\textunderscore , sobrenome do fundador da Companhia de Jesus)}
\end{itemize}
O mesmo que \textunderscore jesuíta\textunderscore .
\section{Louvinha-a-deus}
\begin{itemize}
\item {Grp. gram.:f.}
\end{itemize}
O mesmo que \textunderscore louva-a-deus\textunderscore .
\section{Louvor}
\begin{itemize}
\item {Grp. gram.:m.}
\end{itemize}
Acto de louvar.
Elogio; apologia; glorificação.
\section{Louvoura}
\begin{itemize}
\item {Grp. gram.:f.}
\end{itemize}
\begin{itemize}
\item {Utilização:Marn.}
\end{itemize}
Côdea de chloreto de sódio, que se fórma sóbre os crystallizadores, na primeira colheita do sal. Cf. \textunderscore Museu Techn.\textunderscore , 89.
\section{Lovelace}
\begin{itemize}
\item {Grp. gram.:m.}
\end{itemize}
\begin{itemize}
\item {Utilização:Fig.}
\end{itemize}
\begin{itemize}
\item {Proveniência:(De \textunderscore Lovelace\textunderscore , n. p.)}
\end{itemize}
Namorador galante.
Seductor de mulheres.
\section{Lovelaceano}
\begin{itemize}
\item {Grp. gram.:adj.}
\end{itemize}
Relativo a lovelace; próprio de lovelace. Cf. Camillo, \textunderscore Mar. da Fonte\textunderscore , 394.
\section{Lóxia}
\begin{itemize}
\item {fónica:csi}
\end{itemize}
\begin{itemize}
\item {Grp. gram.:f.}
\end{itemize}
\begin{itemize}
\item {Utilização:Zool.}
\end{itemize}
O mesmo que \textunderscore cruza-bico\textunderscore  e \textunderscore trinca-nozes\textunderscore :«\textunderscore ...a lóxia alli verás... co'o bico encruzado...\textunderscore »Bocage, \textunderscore As Plantas\textunderscore , canto I.
\section{Loxocosmo}
\begin{itemize}
\item {fónica:cso}
\end{itemize}
\begin{itemize}
\item {Grp. gram.:m.}
\end{itemize}
\begin{itemize}
\item {Proveniência:(Do gr. \textunderscore loxos\textunderscore  + \textunderscore kosmos\textunderscore )}
\end{itemize}
Instrumento, para demonstrar os phenómenos do movimento da Terra, as estações e a desigualdade dos dias.
\section{Loxodromia}
\begin{itemize}
\item {fónica:cso}
\end{itemize}
\begin{itemize}
\item {Grp. gram.:f.}
\end{itemize}
\begin{itemize}
\item {Utilização:Náut.}
\end{itemize}
\begin{itemize}
\item {Proveniência:(Do gr. \textunderscore loxos\textunderscore  + \textunderscore dromos\textunderscore )}
\end{itemize}
Linha de navegação, que corta todos os meridianos, sob o mesmo ângulo, e que, nas cartas marítimas, é representada por uma linha recta.
Curva, traçada na superficie de uma esphera, cortando todos os meridianos, sob o mesmo ângulo.
\section{Loxodrómico}
\begin{itemize}
\item {fónica:cso}
\end{itemize}
\begin{itemize}
\item {Grp. gram.:adj.}
\end{itemize}
Relativo á loxodromia.
\section{Loxodromismo}
\begin{itemize}
\item {fónica:cso}
\end{itemize}
\begin{itemize}
\item {Grp. gram.:m.}
\end{itemize}
\begin{itemize}
\item {Proveniência:(De \textunderscore loxodromia\textunderscore )}
\end{itemize}
Marcha em direcção oblíqua.
\section{Loyola}
\begin{itemize}
\item {Grp. gram.:m.}
\end{itemize}
\begin{itemize}
\item {Utilização:Deprec.}
\end{itemize}
Jesuíta.
Hypócrita. Cf. Herculano, \textunderscore Reacção\textunderscore , 19.
\section{Loyolista}
\begin{itemize}
\item {Grp. gram.:m.}
\end{itemize}
\begin{itemize}
\item {Proveniência:(De \textunderscore Loyola\textunderscore , sobrenome do fundador da Companhia de Jesus)}
\end{itemize}
O mesmo que \textunderscore jesuíta\textunderscore .
\section{Lua}
\begin{itemize}
\item {Grp. gram.:f.}
\end{itemize}
\begin{itemize}
\item {Utilização:Fig.}
\end{itemize}
\begin{itemize}
\item {Utilização:Bras}
\end{itemize}
\begin{itemize}
\item {Utilização:Pop.}
\end{itemize}
\begin{itemize}
\item {Proveniência:(Do lat. \textunderscore luna\textunderscore )}
\end{itemize}
Satéllite, que gira em volta da terra e que a illumina de noite.
Espaço de um mês: \textunderscore já tinham passado quatro luas\textunderscore .
A signa moirisca do crescente. Cf. Jac. Freire, \textunderscore D. João de Castro\textunderscore , 17.
Parte deanteira e arqueada da sella.
Nome de um peixe de Portugal.
Cio dos animaes, ferra, aluamento.
\section{Luada}
\begin{itemize}
\item {Grp. gram.:f.}
\end{itemize}
Supposta influência da Lua, segundo a crendice popular.
\section{Luambongo}
\begin{itemize}
\item {Grp. gram.:m.}
\end{itemize}
\begin{itemize}
\item {Proveniência:(T. afr.)}
\end{itemize}
Mammífero carnívoro do Congo.
\section{Luando}
\begin{itemize}
\item {Grp. gram.:m.}
\end{itemize}
Pássaro conirostro africano.
Esteira de mabu.
\section{Luar}
\begin{itemize}
\item {Grp. gram.:m.}
\end{itemize}
\begin{itemize}
\item {Proveniência:(Do lat. \textunderscore lunaris\textunderscore )}
\end{itemize}
Luz da Lua.
Espécie de jôgo popular.
\section{Luário}
\begin{itemize}
\item {Grp. gram.:m.}
\end{itemize}
\begin{itemize}
\item {Utilização:Ant.}
\end{itemize}
O mesmo que \textunderscore lunário\textunderscore .
\section{Luba}
\begin{itemize}
\item {Grp. gram.:f.}
\end{itemize}
Pequena árvore santhomense, (\textunderscore parkia intermedia\textunderscore , Oliver).
\section{Lubambeiro}
\begin{itemize}
\item {Grp. gram.:adj.}
\end{itemize}
\begin{itemize}
\item {Utilização:Bras}
\end{itemize}
\begin{itemize}
\item {Proveniência:(De \textunderscore lubambo\textunderscore )}
\end{itemize}
Desordeiro.
Intrigante.
\section{Lubambo}
\begin{itemize}
\item {Grp. gram.:m.}
\end{itemize}
\begin{itemize}
\item {Utilização:Bras}
\end{itemize}
Barulho.
Intriga, enrêdo.
\section{Lubínia}
\begin{itemize}
\item {Grp. gram.:f.}
\end{itemize}
\begin{itemize}
\item {Proveniência:(De \textunderscore Lubin\textunderscore , n. p.)}
\end{itemize}
Gênero de plantas primuláceas.
\section{Lubishomem}
\begin{itemize}
\item {Grp. gram.:m.}
\end{itemize}
O mesmo que \textunderscore lobishomem\textunderscore . Cf. Herculano, in \textunderscore Panorama\textunderscore , IV, 164.
\section{Lubobos}
\begin{itemize}
\item {Grp. gram.:m. pl.}
\end{itemize}
Tríbo independente, na margem esquerda do Cuanza.
\section{Lubricamente}
\begin{itemize}
\item {Grp. gram.:adv.}
\end{itemize}
De modo lúbrico; sensualmente.
\section{Lubricar}
\begin{itemize}
\item {Grp. gram.:v. t.}
\end{itemize}
\begin{itemize}
\item {Proveniência:(Lat. \textunderscore lubricare\textunderscore )}
\end{itemize}
Tornar lúbrico, lubrificar.
Laxar (o ventre) com um medicamento.
\section{Lubricidade}
\begin{itemize}
\item {Grp. gram.:f.}
\end{itemize}
\begin{itemize}
\item {Utilização:Fig.}
\end{itemize}
\begin{itemize}
\item {Proveniência:(Lat. \textunderscore lubricitas\textunderscore )}
\end{itemize}
Qualidade daquillo que é lúbrico.
Sensualidade; lascívia excessiva.
\section{Lúbrico}
\begin{itemize}
\item {Grp. gram.:adj.}
\end{itemize}
\begin{itemize}
\item {Utilização:Fig.}
\end{itemize}
\begin{itemize}
\item {Proveniência:(Lat. \textunderscore lubricus\textunderscore )}
\end{itemize}
Escorregadio.
Húmido ou liso, a ponto de fazer escorregar.
Húmido.
Sensual; lascivo.
\section{Lubrificação}
\begin{itemize}
\item {Grp. gram.:f.}
\end{itemize}
Acto ou effeito de lubrificar.
\section{Lubrificante}
\begin{itemize}
\item {Grp. gram.:adj.}
\end{itemize}
Que lubrifica.
\section{Lubrificar}
\begin{itemize}
\item {Grp. gram.:v. t.}
\end{itemize}
\begin{itemize}
\item {Proveniência:(Do lat. \textunderscore lubricus\textunderscore  + \textunderscore facere\textunderscore )}
\end{itemize}
Tornar lúbrico ou escorregadio.
Humedecer: \textunderscore lubrificar um maquinismo\textunderscore .
\section{Lúbrigo}
\begin{itemize}
\item {Grp. gram.:m.}
\end{itemize}
\begin{itemize}
\item {Proveniência:(Do lat. \textunderscore lubricus\textunderscore )}
\end{itemize}
Lugar ou terreno escorregadio:«\textunderscore ficaram sepultados no lúbrigo de seus sentimentos.\textunderscore »J. F. Castilho, \textunderscore Grinalda\textunderscore .
\section{Luca}
\begin{itemize}
\item {Grp. gram.:f.}
\end{itemize}
\begin{itemize}
\item {Utilização:Gír.}
\end{itemize}
Carta.
\section{Luca}
\begin{itemize}
\item {Grp. gram.:f.}
\end{itemize}
Espécie de ran, o mesmo que \textunderscore raineta\textunderscore .
\section{Luca}
\begin{itemize}
\item {Grp. gram.:f.}
\end{itemize}
Pequena ave nocturna, de rapina, a que os Franceses chamam \textunderscore petit-duc\textunderscore .--Ouve-se especialmente no Alentejo.
(Talvez de \textunderscore strix aluco\textunderscore , nome que os antigos deram a esta ave)
\section{Lucanário}
\begin{itemize}
\item {Grp. gram.:m.}
\end{itemize}
\begin{itemize}
\item {Proveniência:(Do lat. \textunderscore lux\textunderscore )}
\end{itemize}
Intervallo de duas vigas, numa construcção.
\section{Lucanda}
\begin{itemize}
\item {Grp. gram.:f.}
\end{itemize}
Árvore urticácea de Angola.
\section{Lucango}
\begin{itemize}
\item {Grp. gram.:m.}
\end{itemize}
Árvore, de Cabinda, própria para construcções.
\section{Lucão}
\begin{itemize}
\item {Grp. gram.:m.}
\end{itemize}
Rêde de pesca.
\section{Lucarna}
\begin{itemize}
\item {Grp. gram.:f.}
\end{itemize}
\begin{itemize}
\item {Proveniência:(Fr. \textunderscore lucarne\textunderscore )}
\end{itemize}
Abertura no tecto de uma casa, para dar luz ao sótão.
Trapeira.
Fresta numa parede, só para dar luz ao interior de uma casa ou de um compartimento.
\section{Lucasso}
\begin{itemize}
\item {Grp. gram.:m.}
\end{itemize}
Espécie de enxada, na Lunda.
\section{Lucena}
\begin{itemize}
\item {Grp. gram.:f.}
\end{itemize}
(V.lycena)
\section{Lucena}
\begin{itemize}
\item {Grp. gram.:f.}
\end{itemize}
\begin{itemize}
\item {Proveniência:(De \textunderscore Lucena\textunderscore , n. p.)}
\end{itemize}
Variedade de pêra.
\section{Lucente}
\begin{itemize}
\item {Grp. gram.:adj.}
\end{itemize}
\begin{itemize}
\item {Proveniência:(Lat. \textunderscore lucens\textunderscore )}
\end{itemize}
O mesmo que \textunderscore luzente\textunderscore . Cf. Pero Lopes, \textunderscore Diário da Navegação\textunderscore .
\section{Lúceres}
\begin{itemize}
\item {Grp. gram.:m. pl.}
\end{itemize}
\begin{itemize}
\item {Proveniência:(Lat. \textunderscore luceres\textunderscore )}
\end{itemize}
Uma das três centúrias dos cavalleiros romanos.
\section{Lucerna}
\begin{itemize}
\item {Grp. gram.:f.}
\end{itemize}
\begin{itemize}
\item {Utilização:Ant.}
\end{itemize}
Clarabóia; abertura, por onde entre luz num edifício. Cf. Pant. de Aveiro, \textunderscore Itiner.\textunderscore , 273, (2.^a ed.).
\section{Lucernário}
\begin{itemize}
\item {Grp. gram.:m.}
\end{itemize}
\begin{itemize}
\item {Proveniência:(Lat. \textunderscore lucernarium\textunderscore )}
\end{itemize}
Espécie de poço que, entre os primeiros Christãos, dava accesso ás catacumbas.
\section{Lucescente}
\begin{itemize}
\item {Grp. gram.:adj.}
\end{itemize}
\begin{itemize}
\item {Utilização:Poét.}
\end{itemize}
\begin{itemize}
\item {Proveniência:(Lat. \textunderscore lucescens\textunderscore )}
\end{itemize}
Que começa a brilhar; que se mostra brilhante.
\section{Lúcia-lima}
\begin{itemize}
\item {Grp. gram.:f.}
\end{itemize}
Arbusto verbenáceo, aromático, (\textunderscore lippia citriodora\textunderscore ), também conhecido por \textunderscore limonete\textunderscore , \textunderscore bella-luisa\textunderscore , \textunderscore erva-luisa\textunderscore , \textunderscore verbena\textunderscore .
\section{Lúcias}
\begin{itemize}
\item {Grp. gram.:f.}
\end{itemize}
(V.ascídias)
\section{Lucidamente}
\begin{itemize}
\item {Grp. gram.:adv.}
\end{itemize}
De modo lúcido; com clareza; de modo perceptível.
\section{Lucidar}
\begin{itemize}
\item {Grp. gram.:v. t.}
\end{itemize}
\begin{itemize}
\item {Proveniência:(De \textunderscore lúcido\textunderscore )}
\end{itemize}
Passar para papel vegetal (um desenho) por fórma que transpareçam as linhas dêste.
Reproduzir (um desenho) contra a luz e sôbre um vidro.
\section{Lucidez}
\begin{itemize}
\item {Grp. gram.:f.}
\end{itemize}
Qualidade daquillo que é lúcido; clareza: \textunderscore lucidez de ideias\textunderscore .
Perceptibilidade.
\section{Lúcido}
\begin{itemize}
\item {Grp. gram.:adj.}
\end{itemize}
\begin{itemize}
\item {Utilização:Fig.}
\end{itemize}
\begin{itemize}
\item {Proveniência:(Lat. \textunderscore lucidos\textunderscore )}
\end{itemize}
Que luz; resplandecente; fulgente.
Claro.
Pállido.
Brilhante.
Que tem penetração e clareza de intelligência.
Que revela intelligência clara; em que se mostra uso da razão: \textunderscore intervallos lúcidos\textunderscore .
\section{Lúcifer}
\begin{itemize}
\item {Grp. gram.:m.}
\end{itemize}
\begin{itemize}
\item {Proveniência:(Lat. \textunderscore lucifer\textunderscore )}
\end{itemize}
Nome, que os Latinos deram á estrella Vênus.
Satanás.
\section{Luciferário}
\begin{itemize}
\item {Grp. gram.:m.}
\end{itemize}
\begin{itemize}
\item {Proveniência:(Do lat. \textunderscore lucifer\textunderscore )}
\end{itemize}
Aquelle que leva lanterna em procissões.
\section{Luciferiano}
\begin{itemize}
\item {Grp. gram.:adj.}
\end{itemize}
\begin{itemize}
\item {Utilização:bras}
\end{itemize}
\begin{itemize}
\item {Utilização:Neol.}
\end{itemize}
O mesmo que \textunderscore luciferino\textunderscore .
\section{Luciférico}
\begin{itemize}
\item {Grp. gram.:adj.}
\end{itemize}
O mesmo que \textunderscore luciferino\textunderscore .
\section{Luciferino}
\begin{itemize}
\item {Grp. gram.:adj.}
\end{itemize}
Relativo a Lúcifer; diabólico:«\textunderscore ...ao desaforamento da vida juntavam uma soberba luciferina...\textunderscore »Sousa, \textunderscore Vida do Arceb.\textunderscore , I, 492, Cf. Camillo, \textunderscore Ôlho de Vidro\textunderscore , 159; \textunderscore Caveira\textunderscore , 64.
\section{Lucífero}
\begin{itemize}
\item {Grp. gram.:adj.}
\end{itemize}
\begin{itemize}
\item {Utilização:Poét.}
\end{itemize}
\begin{itemize}
\item {Proveniência:(Lat. \textunderscore lucifer\textunderscore )}
\end{itemize}
Que dá ou traz luz.
\section{Lucífluo}
\begin{itemize}
\item {Grp. gram.:adj.}
\end{itemize}
\begin{itemize}
\item {Proveniência:(Lat. \textunderscore lucifluus\textunderscore )}
\end{itemize}
Que flue ou corre, brilhando:«\textunderscore lucífluas torrentes...\textunderscore »Pato Moniz, \textunderscore Apparição\textunderscore , 19.
\section{Lucífugo}
\begin{itemize}
\item {Grp. gram.:adj.}
\end{itemize}
\begin{itemize}
\item {Grp. gram.:M. pl.}
\end{itemize}
\begin{itemize}
\item {Proveniência:(Lat. \textunderscore lucifugus\textunderscore )}
\end{itemize}
Que foge da luz.
Noctívago.
Família de insectos, da ordem dos coleópteros, com antennas variáveis e elytros duros.
\section{Lucilação}
\begin{itemize}
\item {Grp. gram.:f.}
\end{itemize}
Acto de lucilar.
\section{Lucilante}
\begin{itemize}
\item {Grp. gram.:adj.}
\end{itemize}
Que lucila. Cf. Camillo, \textunderscore Brasileira\textunderscore , 242.
\section{Lucilar}
\begin{itemize}
\item {Grp. gram.:v. i.}
\end{itemize}
\begin{itemize}
\item {Proveniência:(Do lat. \textunderscore lux\textunderscore , \textunderscore lucis\textunderscore )}
\end{itemize}
Brilhar escassamente.
Luzir; tremeluzir. Cf. Camillo, \textunderscore Retr. de Ricard.\textunderscore , 42.
\section{Lucilina}
\begin{itemize}
\item {Grp. gram.:f.}
\end{itemize}
\begin{itemize}
\item {Proveniência:(Do lat. \textunderscore lux\textunderscore , \textunderscore lucis\textunderscore )}
\end{itemize}
Substância, que se extrái do petróleo e que se emprega na illuminação.
\section{Luci-luzir}
\begin{itemize}
\item {Grp. gram.:v. i.}
\end{itemize}
\begin{itemize}
\item {Utilização:bras}
\end{itemize}
\begin{itemize}
\item {Utilização:Neol.}
\end{itemize}
O mesmo que \textunderscore tremeluzir\textunderscore .
Luzir a espaços, como o pyrilampo; lucilar.
\section{Lucímetro}
\begin{itemize}
\item {Grp. gram.:m.}
\end{itemize}
\begin{itemize}
\item {Proveniência:(Do lat. \textunderscore lux\textunderscore , \textunderscore lucis\textunderscore  + gr. \textunderscore metron\textunderscore )}
\end{itemize}
Apparelho, para comparar o brilho das differentes regiões do céu.
\section{Lucina}
\begin{itemize}
\item {Grp. gram.:f.}
\end{itemize}
\begin{itemize}
\item {Utilização:Poét.}
\end{itemize}
\begin{itemize}
\item {Proveniência:(Lat. \textunderscore Lucina\textunderscore , n. p.)}
\end{itemize}
A lua.
\section{Lucinha}
\begin{itemize}
\item {Grp. gram.:f.}
\end{itemize}
\begin{itemize}
\item {Proveniência:(De \textunderscore lúcio\textunderscore )}
\end{itemize}
Pequeno peixe das águas marítimas da costa de Portugal.
\section{Lucinoctes}
\begin{itemize}
\item {Grp. gram.:f. pl.}
\end{itemize}
\begin{itemize}
\item {Proveniência:(Do lat. \textunderscore lux\textunderscore  + \textunderscore nox\textunderscore )}
\end{itemize}
(V.nyctagíneas)
\section{Lúcio}
\begin{itemize}
\item {Grp. gram.:m.}
\end{itemize}
\begin{itemize}
\item {Proveniência:(Lat. \textunderscore lucius\textunderscore )}
\end{itemize}
Peixe de água dôce, da fam. de esoces.
\section{Lucipotente}
\begin{itemize}
\item {Grp. gram.:adj.}
\end{itemize}
\begin{itemize}
\item {Utilização:Poét.}
\end{itemize}
\begin{itemize}
\item {Proveniência:(Do lat. \textunderscore lux\textunderscore  + \textunderscore potens\textunderscore )}
\end{itemize}
Que esparge luz intensa, que illumina tudo. Cf. \textunderscore Agostinheida\textunderscore , 157.
\section{Lucivelo}
\begin{itemize}
\item {Grp. gram.:m.}
\end{itemize}
\begin{itemize}
\item {Utilização:bras}
\end{itemize}
\begin{itemize}
\item {Utilização:Neol.}
\end{itemize}
\begin{itemize}
\item {Proveniência:(Do lat. \textunderscore lux\textunderscore , \textunderscore lucis\textunderscore  + \textunderscore velum\textunderscore )}
\end{itemize}
O mesmo que \textunderscore quebra-luz\textunderscore .
\section{Luco}
\begin{itemize}
\item {Grp. gram.:m.}
\end{itemize}
\begin{itemize}
\item {Utilização:Des.}
\end{itemize}
\begin{itemize}
\item {Proveniência:(Lat. \textunderscore lucus\textunderscore )}
\end{itemize}
O mesmo que \textunderscore bosque\textunderscore . Cf. Castilho, \textunderscore Fastos\textunderscore , I, 123.
\section{Luco}
\begin{itemize}
\item {Grp. gram.:m.}
\end{itemize}
Espécie de cereaes, cultivada na África e na Índia.
\section{Luco}
\begin{itemize}
\item {Grp. gram.:m.}
\end{itemize}
\begin{itemize}
\item {Utilização:T. da Áfr. Or. Port}
\end{itemize}
Colhér de pau.
\section{Lucrar}
\begin{itemize}
\item {Grp. gram.:v. t.}
\end{itemize}
\begin{itemize}
\item {Grp. gram.:V. i.}
\end{itemize}
\begin{itemize}
\item {Proveniência:(Lat. \textunderscore lucrari\textunderscore )}
\end{itemize}
Aproveitar.
Têr como vantagem.
Tirar lucros; têr interesse.
\section{Lucrativamente}
\begin{itemize}
\item {Grp. gram.:adv.}
\end{itemize}
De modo lucrativo.
\section{Lucrativo}
\begin{itemize}
\item {Grp. gram.:adj.}
\end{itemize}
\begin{itemize}
\item {Proveniência:(Lat. \textunderscore lucrativus\textunderscore )}
\end{itemize}
Que dá lucro ou vantagens.
Gratuito.
\section{Lucreciamente}
\begin{itemize}
\item {Grp. gram.:adv.}
\end{itemize}
Á maneira de Lucrécia, matrona romana.
Pudicamente. Cf. Camillo, \textunderscore Corja\textunderscore , 203.
\section{Lucro}
\begin{itemize}
\item {Grp. gram.:m.}
\end{itemize}
\begin{itemize}
\item {Proveniência:(Lat. \textunderscore lucrum\textunderscore )}
\end{itemize}
Utilidade, vantagem.
Interesse; ganho líquido.
Proveito, vantagem gratuita.
\section{Lucroso}
\begin{itemize}
\item {Grp. gram.:adj.}
\end{itemize}
O mesmo que \textunderscore lucrativo\textunderscore .
Que dá lucro; que contém lucro. Cf. Filinto, IX, 31.
\section{Luctífero}
\begin{itemize}
\item {Grp. gram.:adj.}
\end{itemize}
\begin{itemize}
\item {Proveniência:(Lat. \textunderscore luctifer\textunderscore )}
\end{itemize}
Que causa luto; que produz calamidade; desastroso; funesto.
\section{Luctífico}
\begin{itemize}
\item {Grp. gram.:adj.}
\end{itemize}
\begin{itemize}
\item {Utilização:Poét.}
\end{itemize}
\begin{itemize}
\item {Proveniência:(Lat. \textunderscore luctificus\textunderscore )}
\end{itemize}
O mesmo que \textunderscore luctífero\textunderscore . Cf. F. Barreto, \textunderscore Eneida\textunderscore , VII, 76.
\section{Luctísono}
\begin{itemize}
\item {fónica:so}
\end{itemize}
\begin{itemize}
\item {Grp. gram.:adj.}
\end{itemize}
\begin{itemize}
\item {Utilização:Poét.}
\end{itemize}
\begin{itemize}
\item {Proveniência:(Lat. \textunderscore luctisonus\textunderscore )}
\end{itemize}
Que tem um tom lúgubre.
\section{Luctíssono}
\begin{itemize}
\item {Grp. gram.:adj.}
\end{itemize}
\begin{itemize}
\item {Utilização:Poét.}
\end{itemize}
\begin{itemize}
\item {Proveniência:(Lat. \textunderscore luctisonus\textunderscore )}
\end{itemize}
Que tem um tom lúgubre.
\section{Lucto}
\textunderscore m.\textunderscore  (e der.)
O mesmo que \textunderscore luto\textunderscore ^1, etc.
\section{Lucubração}
\begin{itemize}
\item {Grp. gram.:f.}
\end{itemize}
\begin{itemize}
\item {Utilização:Ext.}
\end{itemize}
\begin{itemize}
\item {Proveniência:(Lat. \textunderscore lucubratio\textunderscore )}
\end{itemize}
Acto de lucubrar.
Estudo prolongado ou outro trabalho, feito de noite.
Vigília.
Meditação grave.
Estudo profundo.
\section{Lucubrar}
\begin{itemize}
\item {Grp. gram.:v. i.}
\end{itemize}
\begin{itemize}
\item {Utilização:Ext.}
\end{itemize}
\begin{itemize}
\item {Grp. gram.:V. t.}
\end{itemize}
\begin{itemize}
\item {Proveniência:(Lat. \textunderscore lucubrare\textunderscore )}
\end{itemize}
Trabalhar á luz, de noite.
Passar as noites, estudando.
Dedicar-se a longos trabalhos intellectuaes.
Fazer com trabalho, durante a noite ou durante noites.
Estudar ou apprender, trabalhando dedicada e assiduamente. Cf. Camillo, \textunderscore Narcót.\textunderscore , II, 8.
\section{Lúcula}
\begin{itemize}
\item {Grp. gram.:f.}
\end{itemize}
\begin{itemize}
\item {Proveniência:(Do lat. \textunderscore lux\textunderscore )}
\end{itemize}
Ruga luminosa, que se cruza com outras na superfície do Sol.
\section{Luculentamente}
\begin{itemize}
\item {Grp. gram.:adv.}
\end{itemize}
De modo luculento.
\section{Luculento}
\begin{itemize}
\item {Grp. gram.:adj.}
\end{itemize}
\begin{itemize}
\item {Utilização:Poét.}
\end{itemize}
\begin{itemize}
\item {Proveniência:(Lat. \textunderscore luculentus\textunderscore )}
\end{itemize}
Brilhante; esplêndido; illuminado.
\section{Luculiano}
\begin{itemize}
\item {Grp. gram.:adj.}
\end{itemize}
\begin{itemize}
\item {Proveniência:(Lat. \textunderscore lucullianus\textunderscore )}
\end{itemize}
Relativo a luculo, próprio de um luculo.
Magnificente, (falando-se de um jantar ou banquete).
\section{Luculliano}
\begin{itemize}
\item {Grp. gram.:adj.}
\end{itemize}
\begin{itemize}
\item {Proveniência:(Lat. \textunderscore lucullianus\textunderscore )}
\end{itemize}
Relativo a lucullo, próprio de um lucullo.
Magnificente, (falando-se de um jantar ou banquete).
\section{Lucullo}
\begin{itemize}
\item {Grp. gram.:m.}
\end{itemize}
\begin{itemize}
\item {Utilização:Ext.}
\end{itemize}
\begin{itemize}
\item {Proveniência:(De \textunderscore Luculto\textunderscore , n. p.)}
\end{itemize}
Homem rico, que dá grandes banquetes e faz ostentação do seu luxo e da sua opulência.
\section{Luculo}
\begin{itemize}
\item {Grp. gram.:m.}
\end{itemize}
\begin{itemize}
\item {Utilização:Ext.}
\end{itemize}
\begin{itemize}
\item {Proveniência:(De \textunderscore Luculto\textunderscore , n. p.)}
\end{itemize}
Homem rico, que dá grandes banquetes e faz ostentação do seu luxo e da sua opulência.
\section{Ludâmbulo}
\begin{itemize}
\item {Grp. gram.:adj.}
\end{itemize}
\begin{itemize}
\item {Utilização:bras}
\end{itemize}
\begin{itemize}
\item {Utilização:Neol.}
\end{itemize}
\begin{itemize}
\item {Proveniência:(Do lat. \textunderscore ludus\textunderscore  + \textunderscore ambulare\textunderscore )}
\end{itemize}
O mesmo que \textunderscore turista\textunderscore . Cf. \textunderscore Renascença\textunderscore , do Rio, n.^o 10.
\section{Ludgeriana}
\begin{itemize}
\item {Grp. gram.:f.}
\end{itemize}
Planta cinchonácea.
\section{Ludião}
\begin{itemize}
\item {Grp. gram.:m.}
\end{itemize}
\begin{itemize}
\item {Proveniência:(Lat. \textunderscore ludio\textunderscore )}
\end{itemize}
Figurinha, que fluctua numa garrafa cheia de água, e que serve para demonstrar a theoria da aerostação.
\section{Ludibriante}
\begin{itemize}
\item {Grp. gram.:adj.}
\end{itemize}
Que ludíbria.
\section{Ludibriar}
\begin{itemize}
\item {Grp. gram.:v. t.}
\end{itemize}
Tratar com ludíbrio; zombar de.
\section{Ludibriável}
\begin{itemize}
\item {Grp. gram.:adj.}
\end{itemize}
Que merece sêr ludibriado. Cf. Rui Barb., \textunderscore Réplica\textunderscore , 157.
\section{Ludíbrio}
\begin{itemize}
\item {Grp. gram.:m.}
\end{itemize}
\begin{itemize}
\item {Proveniência:(Lat. \textunderscore ludibrium\textunderscore )}
\end{itemize}
Acto de escarnecer alguém.
Desprêzo.
Objecto de zombaria.
\section{Ludibrioso}
\begin{itemize}
\item {Grp. gram.:adj.}
\end{itemize}
\begin{itemize}
\item {Proveniência:(Lat. \textunderscore ludibriosus\textunderscore )}
\end{itemize}
Em que há ludíbrio.
Que dirige ludíbrios a alguém.
\section{Lúdrico}
\begin{itemize}
\item {Grp. gram.:adj.}
\end{itemize}
\begin{itemize}
\item {Proveniência:(Lat. \textunderscore ludicrus\textunderscore )}
\end{itemize}
Relativo a divertimentos ou a espectáculos públicos:«\textunderscore pompas, ao mesmo tempo religiosas e lúdricas\textunderscore ». Herculano, \textunderscore Cister\textunderscore , II, 79.
\section{Lúdio}
\begin{itemize}
\item {Grp. gram.:m.}
\end{itemize}
\begin{itemize}
\item {Proveniência:(Lat. \textunderscore ludio\textunderscore )}
\end{itemize}
Figurinha, que fluctua numa garrafa cheia de água, e que serve para demonstrar a theoria da aerostação.
\section{Ludo}
\begin{itemize}
\item {Grp. gram.:m.}
\end{itemize}
\begin{itemize}
\item {Utilização:P. us.}
\end{itemize}
\begin{itemize}
\item {Proveniência:(Lat. \textunderscore ludus\textunderscore )}
\end{itemize}
Jôgo, brinco.
Briga de athletas:«\textunderscore iam a ver os ludos dos gladiadores\textunderscore ». Castilho, \textunderscore Metam.\textunderscore , p. XXXII. Cf. Odorico Mendes, \textunderscore Eneida\textunderscore ; Filinto, XIV, 36, 69 e 79.
\section{Ludomania}
\begin{itemize}
\item {Grp. gram.:f.}
\end{itemize}
\begin{itemize}
\item {Proveniência:(Do lat. \textunderscore ludus\textunderscore  e \textunderscore mania\textunderscore )}
\end{itemize}
Mania dos divertimentos, dos desportes.
\section{Ludreiro}
\begin{itemize}
\item {Grp. gram.:f.}
\end{itemize}
\begin{itemize}
\item {Proveniência:(De \textunderscore ludro\textunderscore )}
\end{itemize}
O mesmo que \textunderscore lodeiro\textunderscore ; atoleiro; lamaçal; charco.
\section{Ludrento}
\begin{itemize}
\item {Grp. gram.:adj.}
\end{itemize}
O mesmo que \textunderscore ludro\textunderscore .
\section{Ludro}
\begin{itemize}
\item {Grp. gram.:adj.}
\end{itemize}
\begin{itemize}
\item {Proveniência:(Do lat. hyp. \textunderscore lutulum\textunderscore , de \textunderscore lutum\textunderscore )}
\end{itemize}
Sujo, (falando-se da lan, antes de preparada).
Turvo, (falando-se de um líquido, impregnado de substâncias estranhas, como a enxurrada, a água em que se lavaram objectos sujos, etc.).
\section{Ludroso}
\begin{itemize}
\item {Grp. gram.:adj.}
\end{itemize}
O mesmo que \textunderscore ludro\textunderscore ; churdo.
\section{Lueta}
\begin{itemize}
\item {fónica:ê}
\end{itemize}
\begin{itemize}
\item {Grp. gram.:f.}
\end{itemize}
\begin{itemize}
\item {Utilização:Ant.}
\end{itemize}
Lua nova.
\section{Lufa}
\begin{itemize}
\item {Grp. gram.:f.}
\end{itemize}
\begin{itemize}
\item {Utilização:Fig.}
\end{itemize}
\begin{itemize}
\item {Proveniência:(T. onom.)}
\end{itemize}
Ventania.
Afan, azáfama.
Vela de navio, ou contracção dessa vela sob a acção do vento.
\section{Lufa}
\begin{itemize}
\item {Grp. gram.:f.}
\end{itemize}
\begin{itemize}
\item {Proveniência:(Do ár. \textunderscore louff\textunderscore )}
\end{itemize}
Gênero de plantas cucurbitáceas.
\section{Lufada}
\begin{itemize}
\item {Grp. gram.:f.}
\end{itemize}
\begin{itemize}
\item {Proveniência:(De \textunderscore lufa\textunderscore . Fr. João de Sousa indica a etym. do ár. \textunderscore lafaha\textunderscore )}
\end{itemize}
Rajada de vento.
\section{Lufa-lufa}
\begin{itemize}
\item {Grp. gram.:f.}
\end{itemize}
\begin{itemize}
\item {Utilização:Pop.}
\end{itemize}
Azáfama; grande afan, grande pressa.
(Cp. \textunderscore lufa\textunderscore ^1)
\section{Lufar}
\begin{itemize}
\item {Grp. gram.:v. i.}
\end{itemize}
\begin{itemize}
\item {Proveniência:(De lufa)}
\end{itemize}
Soprar com fôrça (o vento).
Offegar.
\section{Luffa}
\begin{itemize}
\item {Grp. gram.:f.}
\end{itemize}
\begin{itemize}
\item {Proveniência:(Do ár. \textunderscore louff\textunderscore )}
\end{itemize}
Gênero de plantas cucurbitáceas.
\section{Lugar}
\begin{itemize}
\item {Grp. gram.:m.}
\end{itemize}
\begin{itemize}
\item {Utilização:Mathem.}
\end{itemize}
\begin{itemize}
\item {Utilização:Astron.}
\end{itemize}
\begin{itemize}
\item {Grp. gram.:Loc. adv.}
\end{itemize}
\begin{itemize}
\item {Utilização:Ant.}
\end{itemize}
Espaço, occupado por um corpo.
Espaço, em que está alguém ou alguma coisa.
Espaço, independentemente do que póde conter.
Pequena povoação, localidade.
Ordem, posição, classe: \textunderscore em primeiro lugar...\textunderscore 
Ponto de observação.
Circunstâncias especiaes de alguém: \textunderscore eu, no teu lugar, não faria isso\textunderscore .
Passagem de um livro, trecho: \textunderscore lugares selectos de bons prosadores\textunderscore .
Ensejo, lazer: \textunderscore dar lugar a queixas\textunderscore .
O objecto ou espaço em que alguém se senta habitualmente ou destinado ao assento de alguém.
Destino.
Pequeno estabelecimento para venda de hortaliças, frutas, etc.
Superfície sólida, que contém os differentes pontos que são próprios para resolver uma questão indeterminada.
Ponto no espaço, a que corresponde um astro.
\textunderscore De lugar\textunderscore , de passagem; ao mesmo tempo.
(Cp. cast. \textunderscore lugar\textunderscore )
\section{Lugareiro}
\begin{itemize}
\item {Grp. gram.:adj.}
\end{itemize}
\begin{itemize}
\item {Utilização:Prov.}
\end{itemize}
\begin{itemize}
\item {Utilização:trasm.}
\end{itemize}
\begin{itemize}
\item {Proveniência:(De \textunderscore lugar\textunderscore )}
\end{itemize}
Relativo a uma terra ou lugar: \textunderscore expressões lugareiras\textunderscore .
Vulgar, popular, (falando-se de música).
\section{Lugarejo}
\begin{itemize}
\item {Grp. gram.:m.}
\end{itemize}
Pequeno lugar.
Casal.
Aldeia.
\section{Lugarete}
\begin{itemize}
\item {fónica:garê}
\end{itemize}
\begin{itemize}
\item {Grp. gram.:m.}
\end{itemize}
Lugar pequeno.
Lugarejo. Cf. Pant. de Aveiro, \textunderscore Itiner.\textunderscore , 65, (2.^a ed.).
\section{Lugar-tenência}
\begin{itemize}
\item {Grp. gram.:f.}
\end{itemize}
Qualidade ou categoria de lugar-tenente.
\section{Lugar-tenente}
\begin{itemize}
\item {Grp. gram.:m.}
\end{itemize}
\begin{itemize}
\item {Proveniência:(De \textunderscore lugar\textunderscore  + \textunderscore tenente\textunderscore )}
\end{itemize}
Aquelle que provisoriamente desempenha funcções de outrem.
\section{Lugdunense}
\begin{itemize}
\item {Grp. gram.:adj.}
\end{itemize}
\begin{itemize}
\item {Proveniência:(Lat. \textunderscore lugdunensis\textunderscore )}
\end{itemize}
Relativo a Lião, em França.
\section{Lugente}
\begin{itemize}
\item {Grp. gram.:adj.}
\end{itemize}
\begin{itemize}
\item {Proveniência:(Lat. \textunderscore lugens\textunderscore )}
\end{itemize}
Plangente, lastimoso; lúgubre. Cf. Camillo, \textunderscore Mar. da Fonte\textunderscore , 171.
\section{Lugre}
\begin{itemize}
\item {Grp. gram.:m.}
\end{itemize}
Pássaro conirostro, espécie de pintasilgo esverdeado, (\textunderscore chrysomitris spinus\textunderscore , Lin.).
\section{Lugre}
\begin{itemize}
\item {Grp. gram.:m.}
\end{itemize}
\begin{itemize}
\item {Proveniência:(Do ingl. \textunderscore lugger\textunderscore )}
\end{itemize}
Navio mercante, com vários systemas de mastreação.
\section{Lúgubre}
\begin{itemize}
\item {Grp. gram.:adj.}
\end{itemize}
\begin{itemize}
\item {Proveniência:(Lat. \textunderscore lugubris\textunderscore )}
\end{itemize}
Relativo a luto.
Que produz luto.
Que indica luto.
Fúnebre.
Pavoroso.
Escuro.
Medonho; triste; funesto.
\section{Lugubremente}
\begin{itemize}
\item {Grp. gram.:adv.}
\end{itemize}
De modo lúgubre.
\section{Lugubridade}
\begin{itemize}
\item {Grp. gram.:f.}
\end{itemize}
Qualidade de lúgubre.
\section{Luia}
\begin{itemize}
\item {Grp. gram.:f.}
\end{itemize}
Arvore angolense de Malange.
\section{Luina}
\begin{itemize}
\item {Grp. gram.:m.}
\end{itemize}
\begin{itemize}
\item {Grp. gram.:Pl.}
\end{itemize}
Uma das três linguas que se falam no Baroce, em África.
Tríbo africana do Alto Zambeze.
\section{Luir}
\begin{itemize}
\item {Grp. gram.:v. t.}
\end{itemize}
\begin{itemize}
\item {Utilização:Des.}
\end{itemize}
\begin{itemize}
\item {Proveniência:(Lat. \textunderscore luere\textunderscore )}
\end{itemize}
Purificar, expurgar.
Expiar.
Pagar.
\section{Luís}
\begin{itemize}
\item {Grp. gram.:m.}
\end{itemize}
Moéda de oiro, usada em França desde Luís XIII.
\section{Luisiana}
\begin{itemize}
\item {fónica:lu-i}
\end{itemize}
\begin{itemize}
\item {Grp. gram.:f.}
\end{itemize}
Variedade de videira americana.
\section{Luita}
\begin{itemize}
\item {Grp. gram.:f.}
\end{itemize}
\begin{itemize}
\item {Utilização:Ant.}
\end{itemize}
O mesmo que \textunderscore luta\textunderscore .
\section{Luitar}
\begin{itemize}
\item {Grp. gram.:v. i.}
\end{itemize}
\begin{itemize}
\item {Utilização:Ant.}
\end{itemize}
O mesmo que \textunderscore lutar\textunderscore ^1.
\section{Luito}
\begin{itemize}
\item {Grp. gram.:m.}
\end{itemize}
\begin{itemize}
\item {Utilização:Ant.}
\end{itemize}
O mesmo que \textunderscore luto\textunderscore ^1.
\section{Lujanja}
\begin{itemize}
\item {Grp. gram.:f.}
\end{itemize}
Ave da África occidental.
\section{Lula}
\begin{itemize}
\item {Grp. gram.:f.}
\end{itemize}
\begin{itemize}
\item {Proveniência:(Do lat. \textunderscore lunula\textunderscore )}
\end{itemize}
Mollusco, da ordem dos acetabulíferos decápodes, (\textunderscore calmar communis\textunderscore ).
\section{Lulão}
\begin{itemize}
\item {Grp. gram.:m.}
\end{itemize}
\begin{itemize}
\item {Utilização:T. de Varzim}
\end{itemize}
\begin{itemize}
\item {Proveniência:(De \textunderscore lula\textunderscore )}
\end{itemize}
Peixe maritimo da costa, (\textunderscore motella tricirsata\textunderscore ).
\section{Lulismo}
\begin{itemize}
\item {Grp. gram.:m.}
\end{itemize}
Sistema de filosofia mística, preconizado por Lulo, no século XIV.
\section{Lullismo}
\begin{itemize}
\item {Grp. gram.:m.}
\end{itemize}
Systema de philosophia mýstica, preconizado por Lullo, no século XIV.
\section{Lulundo}
\begin{itemize}
\item {Grp. gram.:m.}
\end{itemize}
Língua africana; o mesmo que \textunderscore lundês\textunderscore ? Cf. Capello e Ivens, I, 14.
\section{Lumache}
\begin{itemize}
\item {Grp. gram.:m.}
\end{itemize}
\begin{itemize}
\item {Utilização:Ant.}
\end{itemize}
\begin{itemize}
\item {Proveniência:(Do lat. \textunderscore limax\textunderscore )}
\end{itemize}
Espécie de concha, que servia de ornato e de moéda em vários pontos da África.
Caurim.
\section{Lumachela}
\begin{itemize}
\item {Grp. gram.:f.}
\end{itemize}
\begin{itemize}
\item {Proveniência:(It. \textunderscore lumachella\textunderscore )}
\end{itemize}
Espécie de mármore, composto pela aglomeração de grande número de conchas.
\section{Lumachella}
\begin{itemize}
\item {Grp. gram.:f.}
\end{itemize}
\begin{itemize}
\item {Proveniência:(It. \textunderscore lumachella\textunderscore )}
\end{itemize}
Espécie de mármore, composto pela agglomeração de grande número de conchas.
\section{Lumaréo}
\begin{itemize}
\item {Grp. gram.:m.}
\end{itemize}
\begin{itemize}
\item {Proveniência:(Do rad. de \textunderscore lume\textunderscore )}
\end{itemize}
Fogueira.
Labareda.
Fogacho.
\section{Lumaréu}
\begin{itemize}
\item {Grp. gram.:m.}
\end{itemize}
\begin{itemize}
\item {Proveniência:(Do rad. de \textunderscore lume\textunderscore )}
\end{itemize}
Fogueira.
Labareda.
Fogacho.
\section{Lumbágico}
\begin{itemize}
\item {Grp. gram.:adj.}
\end{itemize}
Relativo ao lumbago.
\section{Lumbago}
\begin{itemize}
\item {Grp. gram.:m.}
\end{itemize}
\begin{itemize}
\item {Proveniência:(Lat. \textunderscore lumbago\textunderscore )}
\end{itemize}
Dôr forte e súbita na região lombar.
\section{Lumbo}
\begin{itemize}
\item {Grp. gram.:m.}
\end{itemize}
Ave aquática, espécie de mergulhão.
\section{Lumbôa}
\begin{itemize}
\item {Grp. gram.:f.}
\end{itemize}
(V.quibôa)
\section{Lumbombo}
\begin{itemize}
\item {Grp. gram.:m.}
\end{itemize}
(V.dizombole)
\section{Lumbrical}
\begin{itemize}
\item {Grp. gram.:adj.}
\end{itemize}
O mesmo ou melhor que \textunderscore lombrical\textunderscore .
\section{Lumbricário}
\begin{itemize}
\item {Grp. gram.:adj.}
\end{itemize}
O mesmo ou melhor que \textunderscore lombrical\textunderscore .
\section{Lumbricida}
\begin{itemize}
\item {Grp. gram.:adj.}
\end{itemize}
\begin{itemize}
\item {Proveniência:(Do lat. \textunderscore lumbricus\textunderscore  + \textunderscore caedere\textunderscore )}
\end{itemize}
Que mata lombrigas; anti-helmínthico.
\section{Lume}
\begin{itemize}
\item {Grp. gram.:m.}
\end{itemize}
\begin{itemize}
\item {Utilização:Fig.}
\end{itemize}
\begin{itemize}
\item {Utilização:Prov.}
\end{itemize}
\begin{itemize}
\item {Utilização:minh.}
\end{itemize}
\begin{itemize}
\item {Utilização:Prov.}
\end{itemize}
\begin{itemize}
\item {Utilização:Prov.}
\end{itemize}
\begin{itemize}
\item {Utilização:alg.}
\end{itemize}
\begin{itemize}
\item {Proveniência:(Lat. \textunderscore lumen\textunderscore )}
\end{itemize}
Desenvolvimento de calor e de luz: \textunderscore accender o lume\textunderscore .
Substância em combustão; fogo: \textunderscore deitar papel no lume\textunderscore .
Luz.
Clarão.
Cirio, vela: \textunderscore os lumes do altar\textunderscore .
Brilhantismo.
Guia.
Perspicácia: \textunderscore o rapaz tem lume no ôlho\textunderscore .
Superfície: \textunderscore ao lume da água\textunderscore .
Parte do casco do cavallo, correspondente á parte anterior da ferradura.
Parte anterior da ferradura.
Cada uma das cavidades do eixo, em que assenta o carro.
O mesmo que phósphoro.
\textunderscore Lumes de pau\textunderscore , phósphoros de enxôfre.
\textunderscore Dar a lume\textunderscore , publicar.
\textunderscore Lume em ala\textunderscore , lume ateado.
\section{Lumeeira}
\begin{itemize}
\item {Grp. gram.:f.}
\end{itemize}
\begin{itemize}
\item {Utilização:Pop.}
\end{itemize}
\begin{itemize}
\item {Utilização:Constr.}
\end{itemize}
\begin{itemize}
\item {Proveniência:(De \textunderscore lume\textunderscore )}
\end{itemize}
Objecto, que alumia.
Castiçal.
Clarabóia.
Clarão.
Archote; fogaréu.
Pyrilampo.
O mesmo que bandeira de porta; parte superior da ombreira.
\section{Lumeeiro}
\begin{itemize}
\item {Grp. gram.:m.}
\end{itemize}
\begin{itemize}
\item {Proveniência:(De \textunderscore lume\textunderscore )}
\end{itemize}
Astro.
Luzeiro.
Fresta, para deixar entrar a luz e o ar.
Pyrilampo.
\section{Lume-prompto}
\begin{itemize}
\item {Grp. gram.:m.}
\end{itemize}
Phósphoro ordinário de madeira, cujo uso antecedeu o dos phósphoros de cera.
\section{Lume-pronto}
\begin{itemize}
\item {Grp. gram.:m.}
\end{itemize}
Fósforo ordinário de madeira, cujo uso antecedeu o dos fósforos de cera.
\section{Lúmia}
\begin{itemize}
\item {Grp. gram.:f.}
\end{itemize}
\begin{itemize}
\item {Utilização:Gír.}
\end{itemize}
Meretriz.
(Or. ind.)
\section{Lumiaco}
\begin{itemize}
\item {Grp. gram.:m.}
\end{itemize}
\begin{itemize}
\item {Utilização:Prov.}
\end{itemize}
\begin{itemize}
\item {Utilização:trasm.}
\end{itemize}
Morrão, usado na cresta das colmeias.
\section{Lumiar}
\begin{itemize}
\item {Grp. gram.:v. t.}
\end{itemize}
\begin{itemize}
\item {Utilização:Prov.}
\end{itemize}
\begin{itemize}
\item {Utilização:minh.}
\end{itemize}
Tirar de (um campo) a água do inverno.
(Cp. \textunderscore alumiar\textunderscore ^1)
\section{Lumiar}
\begin{itemize}
\item {Grp. gram.:m.}
\end{itemize}
\begin{itemize}
\item {Utilização:Ant.}
\end{itemize}
O mesmo que \textunderscore limiar\textunderscore . Cf. \textunderscore Eufrosina\textunderscore , act. I., sc. 1.
\section{Lumieira}
\begin{itemize}
\item {Grp. gram.:f.}
\end{itemize}
\begin{itemize}
\item {Utilização:Pop.}
\end{itemize}
\begin{itemize}
\item {Utilização:Constr.}
\end{itemize}
\begin{itemize}
\item {Proveniência:(De \textunderscore lume\textunderscore )}
\end{itemize}
Objecto, que alumia.
Castiçal.
Clarabóia.
Clarão.
Archote; fogaréu.
Pyrilampo.
O mesmo que bandeira de porta; parte superior da ombreira.
\section{Lumieiro}
\begin{itemize}
\item {Grp. gram.:m.}
\end{itemize}
\begin{itemize}
\item {Proveniência:(De \textunderscore lume\textunderscore )}
\end{itemize}
Astro.
Luzeiro.
Fresta, para deixar entrar a luz e o ar.
Pyrilampo.
\section{Lumilho}
\begin{itemize}
\item {Grp. gram.:m.}
\end{itemize}
\begin{itemize}
\item {Utilização:Ant.}
\end{itemize}
\begin{itemize}
\item {Proveniência:(De \textunderscore lume\textunderscore ?)}
\end{itemize}
Espécie de ponto de bordadeira.
\section{Luminador}
\begin{itemize}
\item {Grp. gram.:m.}
\end{itemize}
\begin{itemize}
\item {Utilização:Ant.}
\end{itemize}
Illuminador.
Aquelle que sabe colorir estampas, ou adornar livros com ellas. Cf. J. Castilho, \textunderscore Manuelinas\textunderscore , 53 e seg.
(Cp. \textunderscore illuminador\textunderscore )
\section{Luminar}
\begin{itemize}
\item {Grp. gram.:adj.}
\end{itemize}
\begin{itemize}
\item {Grp. gram.:M.}
\end{itemize}
\begin{itemize}
\item {Utilização:Fig.}
\end{itemize}
\begin{itemize}
\item {Proveniência:(Lat. \textunderscore luminaris\textunderscore )}
\end{itemize}
Que dá luz.
Astro.
Pessôa de grande illustração.
\section{Luminária}
\begin{itemize}
\item {Grp. gram.:f.}
\end{itemize}
\begin{itemize}
\item {Utilização:Fig.}
\end{itemize}
\begin{itemize}
\item {Grp. gram.:Pl.}
\end{itemize}
\begin{itemize}
\item {Proveniência:(De \textunderscore luminar\textunderscore )}
\end{itemize}
Aquillo que alumia.
Pequena lanterna.
Candeia; lamparina.
Homem muito illustrado.
Illuminação pública e festiva.
\section{Luminarista}
\begin{itemize}
\item {Grp. gram.:m.}
\end{itemize}
\begin{itemize}
\item {Utilização:Burl.}
\end{itemize}
Aquelle que põe luminárias. Cf. Macedo, \textunderscore Burros\textunderscore , 6.
\section{Luminatura}
\begin{itemize}
\item {Grp. gram.:f.}
\end{itemize}
\begin{itemize}
\item {Utilização:Ant.}
\end{itemize}
O mesmo que \textunderscore illuminura\textunderscore .
\section{Luminescência}
\begin{itemize}
\item {Grp. gram.:f.}
\end{itemize}
Propriedade ou qualidade de luminescente.
\section{Luminescente}
\begin{itemize}
\item {Grp. gram.:adj.}
\end{itemize}
Diz-se de vários compostos chímicos, que têm a propriedade de se tornar luminosos, sob a acção da luz ordinária ou dos raios X.
\section{Luminosamente}
\begin{itemize}
\item {Grp. gram.:adj.}
\end{itemize}
De modo luminoso.
Brilhantemente.
\section{Luminosidade}
\begin{itemize}
\item {Grp. gram.:f.}
\end{itemize}
\begin{itemize}
\item {Utilização:Hist. Nat.}
\end{itemize}
Qualidade de luminoso.
Intensidade da luz diffusa.
\section{Luminoso}
\begin{itemize}
\item {Grp. gram.:adj.}
\end{itemize}
\begin{itemize}
\item {Utilização:Fig.}
\end{itemize}
\begin{itemize}
\item {Proveniência:(Lat. \textunderscore luminosus\textunderscore )}
\end{itemize}
Que produz ou espalha luz.
Claro; brilhante: \textunderscore dia luminoso\textunderscore .
Bello.
Perspicaz, que comprehende facilmente: \textunderscore engenho luminoso\textunderscore .
\section{Lumioso}
\begin{itemize}
\item {Grp. gram.:adj.}
\end{itemize}
\begin{itemize}
\item {Utilização:Des.}
\end{itemize}
O mesmo que \textunderscore luminoso\textunderscore . Cf. A. Ferreira, \textunderscore Elegia\textunderscore , III.
\section{Luna}
\begin{itemize}
\item {Grp. gram.:f.}
\end{itemize}
\begin{itemize}
\item {Utilização:Ant.}
\end{itemize}
\begin{itemize}
\item {Proveniência:(Lat. \textunderscore luna\textunderscore )}
\end{itemize}
Espécie de argola para as orelhas.
\section{Lunação}
\begin{itemize}
\item {Grp. gram.:f.}
\end{itemize}
\begin{itemize}
\item {Proveniência:(Do lat. \textunderscore luna\textunderscore )}
\end{itemize}
Espaço, que decorre entre uma lua-nova e a lua-nova seguinte.
\section{Lunado}
\begin{itemize}
\item {Grp. gram.:adj.}
\end{itemize}
\begin{itemize}
\item {Utilização:Poét.}
\end{itemize}
\begin{itemize}
\item {Proveniência:(Do lat. \textunderscore luna\textunderscore )}
\end{itemize}
Que tem cornos, em fórma de meia-lua.
Que tem cornos:«\textunderscore a fronte bem lunada\textunderscore  (da vaca)»Castilho, \textunderscore Geórg.\textunderscore , 147.
\section{Lunanco}
\begin{itemize}
\item {Grp. gram.:adj.}
\end{itemize}
\begin{itemize}
\item {Utilização:Bras. do S}
\end{itemize}
Que tem uma anca mais alta que a outra, (falando-se do cavallo).
(Cast. \textunderscore lunanco\textunderscore )
\section{Lunanquear}
\begin{itemize}
\item {Grp. gram.:v. i.}
\end{itemize}
\begin{itemize}
\item {Utilização:Bras}
\end{itemize}
Ficar lunanco; sêr lunanco.
\section{Lunar}
\begin{itemize}
\item {Grp. gram.:adj.}
\end{itemize}
\begin{itemize}
\item {Grp. gram.:M.}
\end{itemize}
\begin{itemize}
\item {Utilização:Philol.}
\end{itemize}
\begin{itemize}
\item {Utilização:Prov.}
\end{itemize}
\begin{itemize}
\item {Utilização:trasm.}
\end{itemize}
\begin{itemize}
\item {Proveniência:(Lat. \textunderscore lunaris\textunderscore )}
\end{itemize}
Relativo á Lua: \textunderscore eclipse lunar\textunderscore .
Sinal, que apparece na pelle de alguns indivíduos, e que se attribuía á influência da Lua.
Diz-se das letras \textunderscore r\textunderscore , \textunderscore s\textunderscore , \textunderscore z\textunderscore , e \textunderscore c\textunderscore .
Cada uma das peças extremas do carro de bois, e mais ou menos em fórma de crescente. (Colhido em Miranda)
\section{Lunarejo}
\begin{itemize}
\item {Grp. gram.:adj.}
\end{itemize}
\begin{itemize}
\item {Utilização:Bras. do S}
\end{itemize}
\begin{itemize}
\item {Proveniência:(De \textunderscore lunar\textunderscore )}
\end{itemize}
Diz-se do animal, que se distingue por qualquer sinal no pêlo.
\section{Lunária}
\begin{itemize}
\item {Grp. gram.:f.}
\end{itemize}
Planta crucífera, (\textunderscore lunaria biennis rediviva\textunderscore ).
\section{Lunário}
\begin{itemize}
\item {Grp. gram.:m.}
\end{itemize}
\begin{itemize}
\item {Proveniência:(Do lat. \textunderscore luna\textunderscore )}
\end{itemize}
Calendário, em que se computa o tempo por luas.
\section{Lunático}
\begin{itemize}
\item {Grp. gram.:adj.}
\end{itemize}
\begin{itemize}
\item {Utilização:Fig.}
\end{itemize}
\begin{itemize}
\item {Grp. gram.:M.}
\end{itemize}
\begin{itemize}
\item {Proveniência:(Lat. \textunderscore lunaticus\textunderscore )}
\end{itemize}
Sujeito á influência da lua.
Maníaco; atoleimado; extravagante.
Aquelle que tem manias ou que é atoleimado.
\section{Luncúmbi}
\begin{itemize}
\item {Grp. gram.:m.}
\end{itemize}
Uma das línguas, faladas pelos indígenas da África occidental.
\section{Lunda}
\begin{itemize}
\item {Grp. gram.:adj.}
\end{itemize}
\begin{itemize}
\item {Grp. gram.:M.}
\end{itemize}
Relativo á Lunda, em África.
Indígena da Lunda.
A língua falada nesta região.
\section{Lundês}
\begin{itemize}
\item {Grp. gram.:adj.}
\end{itemize}
\begin{itemize}
\item {Grp. gram.:M.}
\end{itemize}
Relativo á Lunda, em África.
Indígena da Lunda.
A língua falada nesta região.
\section{Lundu}
\begin{itemize}
\item {Grp. gram.:m.}
\end{itemize}
\begin{itemize}
\item {Utilização:Bras. do N}
\end{itemize}
Mau humor, o mesmo que \textunderscore calundu\textunderscore .
\section{Lundu}
\begin{itemize}
\item {Grp. gram.:m.}
\end{itemize}
\begin{itemize}
\item {Utilização:Bras. do N}
\end{itemize}
\begin{itemize}
\item {Proveniência:(T. afr.)}
\end{itemize}
Dança desenvolta, própria de Pretos.
Canto ou música, correspondente a essa dança.
Zanga, amuo.
\section{Lundum}
\begin{itemize}
\item {Grp. gram.:m.}
\end{itemize}
\begin{itemize}
\item {Utilização:Bras. do N}
\end{itemize}
\begin{itemize}
\item {Proveniência:(T. afr.)}
\end{itemize}
Dança desenvolta, própria de Pretos.
Canto ou música, correspondente a essa dança.
Zanga, amuo.
\section{Lunduzeiro}
\begin{itemize}
\item {Grp. gram.:adj.}
\end{itemize}
\begin{itemize}
\item {Utilização:Bras. do N}
\end{itemize}
\begin{itemize}
\item {Proveniência:(De \textunderscore lundu\textunderscore ^2)}
\end{itemize}
Zangado.
Que se amua facilmente.
\section{Luneta}
\begin{itemize}
\item {fónica:nê}
\end{itemize}
\begin{itemize}
\item {Grp. gram.:f.}
\end{itemize}
\begin{itemize}
\item {Utilização:T. de fortificação}
\end{itemize}
\begin{itemize}
\item {Utilização:T. de curtidor}
\end{itemize}
\begin{itemize}
\item {Proveniência:(Do rad. do lat. \textunderscore luna\textunderscore )}
\end{itemize}
Utensílio, composto de um ou dois vidros ou lentes, e destinado geralmente a auxiliar a vista.
Parte da custódia, em que se segura a hóstia.
Fresta circular ou oval, para communicar luz e ar ao interior das habitações.
Círculo de aço, para medir o calibre das balas.
Parte da guilhotina, sôbre a qual se atravessa o pescoço do condemnado.
Redente com flancos.
Instrumento cortante, usado na pousagem das pelles.
\section{Lunetaría}
\begin{itemize}
\item {Grp. gram.:f.}
\end{itemize}
Estabelecimento, onde se vendem lunetas e óculos.
\section{Lunga}
\begin{itemize}
\item {Grp. gram.:f.}
\end{itemize}
Árvore de Angola.
\section{Lungungua}
\begin{itemize}
\item {Grp. gram.:f.}
\end{itemize}
Ave pernalta, o mesmo que \textunderscore quilúbio\textunderscore .
\section{Lunhaneca}
\begin{itemize}
\item {Grp. gram.:m.}
\end{itemize}
Uma das línguas faladas na província de Angola.
\section{Lunícola}
\begin{itemize}
\item {Grp. gram.:m.  e  adj.}
\end{itemize}
\begin{itemize}
\item {Proveniência:(Do lat. \textunderscore luna\textunderscore  + \textunderscore colere\textunderscore )}
\end{itemize}
Habitante da Lua.
Selenita.
\section{Luniforme}
\begin{itemize}
\item {Grp. gram.:adj.}
\end{itemize}
\begin{itemize}
\item {Proveniência:(Do lat. \textunderscore luna\textunderscore  + \textunderscore forma\textunderscore )}
\end{itemize}
Que tem fórma de meia-lua.
\section{Luni-solar}
\begin{itemize}
\item {Grp. gram.:adj.}
\end{itemize}
\begin{itemize}
\item {Proveniência:(Do lat. \textunderscore luna\textunderscore  + \textunderscore sol\textunderscore )}
\end{itemize}
Que depende da Lua e do Sol, ao mesmo tempo.
\section{Lúnula}
\begin{itemize}
\item {Grp. gram.:f.}
\end{itemize}
\begin{itemize}
\item {Proveniência:(Lat. \textunderscore lunula\textunderscore )}
\end{itemize}
Cada um dos satéllites de Júpiter ou Saturno.
Figura geométrica, composta por dois arcos convexos que se interceptam.
Mancha esbranquiçada e semi-lunar, na base da unha.
Objecto, em fórma de meia-lua.
\section{Lunulado}
\begin{itemize}
\item {Grp. gram.:adj.}
\end{itemize}
Luniforme; que tem lúnula.
\section{Lunular}
\begin{itemize}
\item {Grp. gram.:adj.}
\end{itemize}
O mesmo que \textunderscore lunulado\textunderscore .
\section{Lunulite}
\begin{itemize}
\item {Grp. gram.:f.}
\end{itemize}
Gênero de polypeiros, quási todos fósseis.
\section{Lupa}
\begin{itemize}
\item {Grp. gram.:f.}
\end{itemize}
\begin{itemize}
\item {Proveniência:(Fr. \textunderscore loupe\textunderscore )}
\end{itemize}
Tumor no joêlho de alguns animaes.
Microscópio ou lente biconvexa, que aumenta muito os objectos á vista.
\section{Lupa}
\begin{itemize}
\item {Grp. gram.:f.}
\end{itemize}
\begin{itemize}
\item {Utilização:Náut.}
\end{itemize}
\begin{itemize}
\item {Utilização:Gír.}
\end{itemize}
Uma das maneiras, com que, á fôrça de braços, se içam os escaleres aos turcos.
\textunderscore Cantar a lupa\textunderscore , vomitar.
\section{Lupada}
\begin{itemize}
\item {Grp. gram.:f.}
\end{itemize}
\begin{itemize}
\item {Utilização:Náut.}
\end{itemize}
\begin{itemize}
\item {Proveniência:(De \textunderscore lupa\textunderscore ^2)}
\end{itemize}
Cada um dos impulsos, com que, em certos casos, se iça um escaler aos turcos.
\section{Lupamba}
\begin{itemize}
\item {Grp. gram.:f.}
\end{itemize}
Ave africana de rapina.
\section{Lupanar}
\begin{itemize}
\item {Grp. gram.:m.}
\end{itemize}
\begin{itemize}
\item {Proveniência:(Lat. \textunderscore lupanar\textunderscore )}
\end{itemize}
Casa de meretrizes; alcoice; bordel; prostíbulo.
\section{Lupanário}
\begin{itemize}
\item {Grp. gram.:adj.}
\end{itemize}
\begin{itemize}
\item {Utilização:Neol.}
\end{itemize}
Relativo a lupanar.
\section{Lupanga}
\begin{itemize}
\item {Grp. gram.:f.}
\end{itemize}
Pequena espada, usada pelos Cafres.
\section{Lupante}
\begin{itemize}
\item {Grp. gram.:m.}
\end{itemize}
\begin{itemize}
\item {Utilização:Gír.}
\end{itemize}
\begin{itemize}
\item {Proveniência:(De \textunderscore lupar\textunderscore )}
\end{itemize}
Ôlho.
\section{Lupar}
\begin{itemize}
\item {Grp. gram.:v. t.}
\end{itemize}
\begin{itemize}
\item {Utilização:Gír.}
\end{itemize}
\begin{itemize}
\item {Proveniência:(De \textunderscore lupa\textunderscore ^1)}
\end{itemize}
Vêr.
\section{Lúparo}
\begin{itemize}
\item {Grp. gram.:m.}
\end{itemize}
\begin{itemize}
\item {Utilização:Prov.}
\end{itemize}
\begin{itemize}
\item {Utilização:beir.}
\end{itemize}
(Corr. de \textunderscore túpulo\textunderscore )
Rebento ou espigo de couves velhas.
\section{Lupercaes}
\begin{itemize}
\item {Grp. gram.:f. pl.}
\end{itemize}
\begin{itemize}
\item {Proveniência:(Lat. \textunderscore lupercalia\textunderscore )}
\end{itemize}
Festas romanas, em honra do deus Pan.
\section{Luperco}
\begin{itemize}
\item {Grp. gram.:m.}
\end{itemize}
\begin{itemize}
\item {Proveniência:(Lat. \textunderscore lupercus\textunderscore )}
\end{itemize}
Sacerdote de Pan, entre os Romanos.
\section{Lúpero}
\begin{itemize}
\item {Grp. gram.:m.}
\end{itemize}
\begin{itemize}
\item {Proveniência:(Gr. \textunderscore luperos\textunderscore )}
\end{itemize}
Gênero de insectos coleópteros tetrâmeros.
\section{Lúpia}
\begin{itemize}
\item {Grp. gram.:f.}
\end{itemize}
\begin{itemize}
\item {Grp. gram.:Pl.}
\end{itemize}
O mesmo que \textunderscore lobinho\textunderscore ^1.
O mesmo que \textunderscore lupa\textunderscore ^1.
\section{Lupinastro}
\begin{itemize}
\item {Grp. gram.:m.}
\end{itemize}
Variedade de trevo, (\textunderscore trifolium lupinaster\textunderscore ).
\section{Lupinina}
\begin{itemize}
\item {Grp. gram.:f.}
\end{itemize}
\begin{itemize}
\item {Proveniência:(Do lat. \textunderscore lupinus\textunderscore )}
\end{itemize}
Substância amarga, descoberta na farinha do tremoço.
\section{Lupino}
\begin{itemize}
\item {Grp. gram.:adj.}
\end{itemize}
\begin{itemize}
\item {Proveniência:(Lat. \textunderscore lupinus\textunderscore )}
\end{itemize}
Relativo a lobo.
\section{Lupino}
\begin{itemize}
\item {Grp. gram.:m.}
\end{itemize}
Planta annual brasileira.
\section{Lupinose}
\begin{itemize}
\item {Grp. gram.:f.}
\end{itemize}
Doença dos cavallos, causada pela lupinotoxina.
\section{Lupinotoxina}
\begin{itemize}
\item {fónica:csi}
\end{itemize}
\begin{itemize}
\item {Grp. gram.:f.}
\end{itemize}
\begin{itemize}
\item {Proveniência:(Do lat. \textunderscore lupinus\textunderscore )}
\end{itemize}
Alcaloide venenoso dos tremoços.
\section{Lupishomem}
\begin{itemize}
\item {Grp. gram.:m.}
\end{itemize}
(V.lobishomem)
\section{Luposo}
\begin{itemize}
\item {Grp. gram.:adj.}
\end{itemize}
\begin{itemize}
\item {Utilização:Med.}
\end{itemize}
Relativo a lúpus.
\section{Lupossa}
\begin{itemize}
\item {Grp. gram.:f.}
\end{itemize}
Planta africana, dioica, trepadora ou rasteira.
\section{Lupulina}
\begin{itemize}
\item {Grp. gram.:f.}
\end{itemize}
\begin{itemize}
\item {Proveniência:(De \textunderscore lúpulo\textunderscore )}
\end{itemize}
Espécie de lúpulo, (\textunderscore medicago lupulina\textunderscore ).
Substância balsâmica e amarga, contida no lúpulo.
\section{Lúpulo}
\begin{itemize}
\item {Grp. gram.:m.}
\end{itemize}
\begin{itemize}
\item {Proveniência:(Do lat. \textunderscore lupus\textunderscore )}
\end{itemize}
Planta trepadeira, da fam. das urticáceas, (\textunderscore humulus lupulus\textunderscore , Lin.).
\section{Lúpuro}
\begin{itemize}
\item {Grp. gram.:m.}
\end{itemize}
\begin{itemize}
\item {Utilização:Prov.}
\end{itemize}
\begin{itemize}
\item {Utilização:trasm.}
\end{itemize}
Rebento de couve, o mesmo que \textunderscore lúparo\textunderscore .
\section{Lúpus}
\begin{itemize}
\item {Grp. gram.:m.}
\end{itemize}
\begin{itemize}
\item {Utilização:Med.}
\end{itemize}
\begin{itemize}
\item {Proveniência:(Do lat. \textunderscore lupus\textunderscore )}
\end{itemize}
Inflammação cutânea, que se manifesta por pequenas protuberâncias, atacando especialmente o rosto e podendo sêr seguida de úlceras.
\section{Luque}
\begin{itemize}
\item {Grp. gram.:m.}
\end{itemize}
Planta trepadeira da ilha de San-Thomé.
\section{Luquelo}
\begin{itemize}
\item {Grp. gram.:m.}
\end{itemize}
Arbusto africano, da fam. das ampelídeas, de cachos semelhantes aos da diagalves.
\section{Lura}
\begin{itemize}
\item {Grp. gram.:f.}
\end{itemize}
\begin{itemize}
\item {Proveniência:(Lat. \textunderscore lura\textunderscore )}
\end{itemize}
Toca; lora.
Esconderijo de certos animaes.
Utensílio de barro, em que fazem criação os coêlhos domésticos.
\section{Lurar}
\begin{itemize}
\item {Grp. gram.:v. t.}
\end{itemize}
\begin{itemize}
\item {Grp. gram.:V. p.}
\end{itemize}
Fazer luras em.
Esburacar, escavar. Cf. Camillo, \textunderscore Brasileira\textunderscore , 104 e 190.
Meter-se em lura, esconder-se. Us. por Camillo.
\section{Lurda}
\begin{itemize}
\item {Grp. gram.:f.}
\end{itemize}
\begin{itemize}
\item {Utilização:Prov.}
\end{itemize}
\begin{itemize}
\item {Utilização:trasm.}
\end{itemize}
O mesmo que \textunderscore lóstra\textunderscore .
\section{Lurgo}
\begin{itemize}
\item {Grp. gram.:m.}
\end{itemize}
Pequena ave, quási toda verde.
\section{Lúria}
\begin{itemize}
\item {Grp. gram.:f.}
\end{itemize}
\begin{itemize}
\item {Utilização:Prov.}
\end{itemize}
\begin{itemize}
\item {Utilização:trasm.}
\end{itemize}
Corda grossa para apertar a carga do carro de bois.
\section{Lúrido}
\begin{itemize}
\item {Grp. gram.:adj.}
\end{itemize}
\begin{itemize}
\item {Utilização:Poét.}
\end{itemize}
\begin{itemize}
\item {Proveniência:(Lat. \textunderscore luridus\textunderscore )}
\end{itemize}
Pállido, lívido.
Escuro.
\section{Lurta}
\begin{itemize}
\item {Grp. gram.:f.}
\end{itemize}
\begin{itemize}
\item {Utilização:Prov.}
\end{itemize}
\begin{itemize}
\item {Utilização:trasm.}
\end{itemize}
\begin{itemize}
\item {Proveniência:(Do cast. \textunderscore lurte\textunderscore , alude?)}
\end{itemize}
O mesmo que \textunderscore póla\textunderscore ^1, sova.
\section{Lúrtia}
\begin{itemize}
\item {Grp. gram.:f.}
\end{itemize}
\begin{itemize}
\item {Utilização:Prov.}
\end{itemize}
\begin{itemize}
\item {Utilização:trasm.}
\end{itemize}
O mesmo que \textunderscore lurta\textunderscore .
\section{Luscar}
\begin{itemize}
\item {Grp. gram.:v. i.}
\end{itemize}
\begin{itemize}
\item {Utilização:Ant.}
\end{itemize}
Brincar, divertir-se.
(Talvez de \textunderscore lusco\textunderscore )
\section{Luscínia}
\begin{itemize}
\item {Grp. gram.:f.}
\end{itemize}
\begin{itemize}
\item {Proveniência:(Lat. \textunderscore luscinia\textunderscore )}
\end{itemize}
Gênero de pássaros insectívoros, a que pertence o rouxinol.
\section{Lusco}
\begin{itemize}
\item {Grp. gram.:adj.}
\end{itemize}
\begin{itemize}
\item {Utilização:Ext.}
\end{itemize}
\begin{itemize}
\item {Grp. gram.:Loc. adv.}
\end{itemize}
\begin{itemize}
\item {Utilização:ant.}
\end{itemize}
\begin{itemize}
\item {Proveniência:(Lat. \textunderscore luscus\textunderscore )}
\end{itemize}
Que tem só um ôlho ou vê só de um ôlho.
Vesgo.
Que não vê; cego.
«\textunderscore Entre lusco &amp; fusco\textunderscore », á hora do crepúsculo. \textunderscore Eufrosina\textunderscore , 156.
\section{Licantropia}
\begin{itemize}
\item {Grp. gram.:f.}
\end{itemize}
Doença mental, em que o enfermo se supõe transformado em lôbo.
(Cp. \textunderscore licantropo\textunderscore )
\section{Lusco-fusco}
\begin{itemize}
\item {Grp. gram.:m.}
\end{itemize}
O anoitecer; a hora do crepúsculo.--Filinto, VII, 84, diz \textunderscore lusque-fusque\textunderscore .
\section{Lusismo}
\begin{itemize}
\item {Grp. gram.:m.}
\end{itemize}
\begin{itemize}
\item {Utilização:P. us.}
\end{itemize}
\begin{itemize}
\item {Proveniência:(De \textunderscore luso\textunderscore )}
\end{itemize}
O mesmo que \textunderscore lusitanismo\textunderscore .
\section{Lusitânico}
\begin{itemize}
\item {Grp. gram.:adj.}
\end{itemize}
Relativo á Lusitânia ou aos Lusitanos. Cf. \textunderscore Lusíadas\textunderscore , IX, 58.
\section{Lusitanismo}
\begin{itemize}
\item {Grp. gram.:m.}
\end{itemize}
\begin{itemize}
\item {Utilização:Ext.}
\end{itemize}
Costume próprio de Lusitanos.
Locução vernácula portuguesa:«\textunderscore por toda a parte lhe estão pullulando lusitanismos em vocábulos...\textunderscore »Castilho, \textunderscore Primavera\textunderscore , 151.
\section{Lusitano}
\begin{itemize}
\item {Grp. gram.:adj.}
\end{itemize}
\begin{itemize}
\item {Utilização:Ext.}
\end{itemize}
\begin{itemize}
\item {Grp. gram.:M.}
\end{itemize}
\begin{itemize}
\item {Proveniência:(Lat. \textunderscore lusitanus\textunderscore )}
\end{itemize}
Relativo á Lusitânia ou aos seus habitantes.
Relativo a Portugal ou aos Portugueses.
Habitante da Lusitânia.
\section{Luso}
\begin{itemize}
\item {Grp. gram.:m.  e  adj.}
\end{itemize}
\begin{itemize}
\item {Proveniência:(De \textunderscore Luso\textunderscore , n. p. do supposto fundador da raça lusitânica)}
\end{itemize}
O mesmo que \textunderscore lusitano\textunderscore .
\section{Luso...}
Elemento, que entra na composição de várias palavras, com a designação de \textunderscore lusitano\textunderscore  ou de \textunderscore relativo a Portugal\textunderscore .
\section{Luso-africano}
\begin{itemize}
\item {Grp. gram.:adj.}
\end{itemize}
Relativo a Portugal e á África.
\section{Luso-americano}
\begin{itemize}
\item {Grp. gram.:adj.}
\end{itemize}
Relativo a Portugal e á América.
\section{Luso-andaluz}
\begin{itemize}
\item {Grp. gram.:adj.}
\end{itemize}
Diz-se de uma variedade de cavallos portugueses, de origem andaluza.
\section{Luso-árabe}
\begin{itemize}
\item {Grp. gram.:adj.}
\end{itemize}
Relativo a Lusos e Árabes.
\section{Luso-brasileiro}
\begin{itemize}
\item {Grp. gram.:adj.}
\end{itemize}
Relativo a Portugal e Brasil.
\section{Luso-britânnico}
\begin{itemize}
\item {Grp. gram.:adj.}
\end{itemize}
Relativo a Portugal e á Inglaterra.
\section{Luso-chinês}
\begin{itemize}
\item {Grp. gram.:adj.}
\end{itemize}
Relativo a Portugal e á China.
\section{Lusões}
\begin{itemize}
\item {Grp. gram.:m. pl.}
\end{itemize}
Antigo povo de Bornéu. Cf. \textunderscore Peregrinação\textunderscore , IX.
\section{Luso-espanhol}
\begin{itemize}
\item {Grp. gram.:adj.}
\end{itemize}
Relativo a Portugal e á Espanha.
\section{Luso-francês}
\begin{itemize}
\item {Grp. gram.:adj.}
\end{itemize}
Relativo a Portugal e á França.
\section{Luso-germânico}
\begin{itemize}
\item {Grp. gram.:adj.}
\end{itemize}
Relativo a Portugal e á Alemanha.
\section{Luso-hispânico}
\begin{itemize}
\item {Grp. gram.:adj.}
\end{itemize}
O mesmo que \textunderscore luso-espanhol\textunderscore .
\section{Luso-indiano}
\begin{itemize}
\item {Grp. gram.:adj.}
\end{itemize}
Relativo a Portugal e á Índia.
\section{Luso-italiano}
\begin{itemize}
\item {Grp. gram.:adj.}
\end{itemize}
Relativo a Portugal e á Italia.
\section{Lusones}
\begin{itemize}
\item {Grp. gram.:m. pl.}
\end{itemize}
\begin{itemize}
\item {Proveniência:(Lat. \textunderscore lusones\textunderscore )}
\end{itemize}
Antigo povo do interior da Espanha, junto ás nascentes do Tejo. Cf. Herculano, \textunderscore Hist. de Port.\textunderscore , I, 16.
\section{Lusório}
\begin{itemize}
\item {Grp. gram.:adj.}
\end{itemize}
\begin{itemize}
\item {Proveniência:(Lat. lusorius)}
\end{itemize}
Relativo a jôgo ou a folganças.
\section{Lusque-fusque}
\begin{itemize}
\item {Grp. gram.:m.}
\end{itemize}
O mesmo que \textunderscore lusco-fusco\textunderscore . Cf. Filinto, VII, 84.
\section{Lusquifusque}
\begin{itemize}
\item {Grp. gram.:m.}
\end{itemize}
\begin{itemize}
\item {Utilização:Ant.}
\end{itemize}
O mesmo que \textunderscore lusco-fusco\textunderscore .
\section{Lusquir-se}
\begin{itemize}
\item {Grp. gram.:v. p.}
\end{itemize}
\begin{itemize}
\item {Utilização:Prov.}
\end{itemize}
\begin{itemize}
\item {Utilização:trasm.}
\end{itemize}
Esconder-se.
\section{Lustração}
\begin{itemize}
\item {Grp. gram.:f.}
\end{itemize}
\begin{itemize}
\item {Proveniência:(Lat. \textunderscore lustratio\textunderscore )}
\end{itemize}
Acto ou effeito de lustrar.
Lavagem; purificação.
\section{Lustradeira}
\begin{itemize}
\item {Grp. gram.:f.}
\end{itemize}
\begin{itemize}
\item {Proveniência:(De \textunderscore lustrar\textunderscore )}
\end{itemize}
Apparelho, composto de cylindros de cobre perfurados, em que se enrolam os panos, nas fábricas de lanifícios, para serem lustrados por meio de vapor.
\section{Lustradela}
\begin{itemize}
\item {Grp. gram.:f.}
\end{itemize}
\begin{itemize}
\item {Proveniência:(De \textunderscore lustrar\textunderscore )}
\end{itemize}
Acto ou effeito de dar lustre em chapéus, botas, etc.
\section{Lustrador}
\begin{itemize}
\item {Grp. gram.:m.}
\end{itemize}
\begin{itemize}
\item {Grp. gram.:Adj.}
\end{itemize}
Aquelle ou aquillo que lustra.
Apparelho, para dar lustro aos grãos da pólvora.
Que lustra.
\section{Lustral}
\begin{itemize}
\item {Grp. gram.:adj.}
\end{itemize}
\begin{itemize}
\item {Proveniência:(Lat. \textunderscore lustralis\textunderscore )}
\end{itemize}
Que serve para lustrar ou para purificar: \textunderscore água lustral\textunderscore .
\section{Lustrar}
\begin{itemize}
\item {Grp. gram.:v. t.}
\end{itemize}
\begin{itemize}
\item {Grp. gram.:V. i.}
\end{itemize}
\begin{itemize}
\item {Proveniência:(Lat. \textunderscore lustrare\textunderscore )}
\end{itemize}
Tornar brilhante ou polido: \textunderscore lustrar o calçado\textunderscore .
Purificar, lavando.
Examinar minuciosamente.
Percorrer.
Tornar instruido ou culto.
Brilhar, resplandecer. Cf. Camillo, \textunderscore Caveira\textunderscore , 46.
\section{Lustre}
\begin{itemize}
\item {Grp. gram.:m.}
\end{itemize}
\begin{itemize}
\item {Utilização:Fig.}
\end{itemize}
Brilho de um objecto polido, envernizado ou engraxado.
Candelabro.
Fama, glória.
Resplendor; brilhantismo.
(Cast. \textunderscore lustre\textunderscore )
\section{Lustrilho}
\begin{itemize}
\item {Grp. gram.:m.}
\end{itemize}
\begin{itemize}
\item {Grp. gram.:M.}
\end{itemize}
\begin{itemize}
\item {Proveniência:(De \textunderscore lustre\textunderscore )}
\end{itemize}
O mesmo que \textunderscore lustrino\textunderscore .
Tecido de lan, um pouco lustroso.
\section{Lustrina}
\begin{itemize}
\item {Grp. gram.:f.}
\end{itemize}
\begin{itemize}
\item {Proveniência:(De \textunderscore lustrino\textunderscore )}
\end{itemize}
Tecido lustroso de seda, de algodão ou de lan.
\section{Lustrino}
\begin{itemize}
\item {Grp. gram.:adj.}
\end{itemize}
Que tem lustre; que é lustroso.
Diz-se da lan estambrada e luzente.
\section{Lustriverde}
\begin{itemize}
\item {Grp. gram.:adj.}
\end{itemize}
\begin{itemize}
\item {Proveniência:(De \textunderscore lustro\textunderscore ^2 + \textunderscore verde\textunderscore )}
\end{itemize}
Que tem lustro ou brilho, verdejando. Cf. Filinto, IV, 15.
\section{Lustro}
\begin{itemize}
\item {Grp. gram.:m.}
\end{itemize}
\begin{itemize}
\item {Proveniência:(Lat. \textunderscore lustrum\textunderscore )}
\end{itemize}
Período de cinco annos.
\section{Lustro}
\begin{itemize}
\item {Grp. gram.:m.}
\end{itemize}
\begin{itemize}
\item {Utilização:Pop.}
\end{itemize}
\begin{itemize}
\item {Proveniência:(De \textunderscore lustrar\textunderscore )}
\end{itemize}
O mesmo que \textunderscore polimento\textunderscore .
\section{Lustro}
\begin{itemize}
\item {Grp. gram.:m.}
\end{itemize}
\begin{itemize}
\item {Utilização:Prov.}
\end{itemize}
\begin{itemize}
\item {Utilização:trasm.}
\end{itemize}
O mesmo que \textunderscore relâmpago\textunderscore .
\section{Lustrosamente}
\begin{itemize}
\item {Grp. gram.:adv.}
\end{itemize}
De modo lustroso.
\section{Lustroso}
\begin{itemize}
\item {Grp. gram.:adj.}
\end{itemize}
\begin{itemize}
\item {Utilização:Fig.}
\end{itemize}
\begin{itemize}
\item {Proveniência:(De \textunderscore lustrar\textunderscore )}
\end{itemize}
Que tem lustre.
Em que há brilho.
Esplêndido; magnifico; notável.
\section{Luta}
\begin{itemize}
\item {Grp. gram.:f.}
\end{itemize}
\begin{itemize}
\item {Utilização:Ext.}
\end{itemize}
\begin{itemize}
\item {Proveniência:(Lat. \textunderscore lucta\textunderscore )}
\end{itemize}
Combate entre dois indivíduos, braço a braço.
Peleja; guerra.
Conflicto.
Contenda.
Esfôrço, empenho.
\section{Lutador}
\begin{itemize}
\item {Grp. gram.:m.  e  adj.}
\end{itemize}
\begin{itemize}
\item {Proveniência:(Lat. \textunderscore luctator\textunderscore )}
\end{itemize}
O que luta; athleta.
\section{Lutante}
\begin{itemize}
\item {Grp. gram.:adj.}
\end{itemize}
\begin{itemize}
\item {Proveniência:(Lat. \textunderscore luctans\textunderscore )}
\end{itemize}
Que luta.
\section{Lutar}
\begin{itemize}
\item {Grp. gram.:v. i.}
\end{itemize}
\begin{itemize}
\item {Utilização:Fig.}
\end{itemize}
\begin{itemize}
\item {Proveniência:(Lat. \textunderscore luctari\textunderscore )}
\end{itemize}
Travar luta.
Brigar; combater.
Esforçar-se; altercar.
\section{Lutar}
\begin{itemize}
\item {Grp. gram.:v. t.}
\end{itemize}
\begin{itemize}
\item {Proveniência:(De \textunderscore luto\textunderscore ^2)}
\end{itemize}
Indutar.
Tapar com a massa que se chama luto.
\section{Luteicórneo}
\begin{itemize}
\item {Grp. gram.:adj.}
\end{itemize}
\begin{itemize}
\item {Utilização:Zool.}
\end{itemize}
\begin{itemize}
\item {Proveniência:(Do lat. \textunderscore luteus\textunderscore  + \textunderscore cornu\textunderscore )}
\end{itemize}
Que tem cornos ou antennas amarelas.
\section{Lúteo-gállico}
\begin{itemize}
\item {Grp. gram.:adj.}
\end{itemize}
Diz-se de um ácido, que é o princípio còrante da noz de galha.
\section{Luteolina}
\begin{itemize}
\item {Grp. gram.:f.}
\end{itemize}
\begin{itemize}
\item {Proveniência:(Do lat. \textunderscore luteolus\textunderscore )}
\end{itemize}
Substância còrante da reseda amarela.
\section{Luteranismo}
\begin{itemize}
\item {Grp. gram.:m.}
\end{itemize}
\begin{itemize}
\item {Proveniência:(De \textunderscore luterano\textunderscore )}
\end{itemize}
Doutrina dos Luteranos.
Seita religiosa, fundada por Lutero.
\section{Luterano}
\begin{itemize}
\item {Grp. gram.:adj.}
\end{itemize}
\begin{itemize}
\item {Grp. gram.:M.}
\end{itemize}
\begin{itemize}
\item {Proveniência:(De \textunderscore Lutero\textunderscore , n. p.)}
\end{itemize}
Relativo á doutrina dos Luteranos.
Sectário de Lutero ou das suas doutrinas.
\section{Lutheranismo}
\begin{itemize}
\item {Grp. gram.:m.}
\end{itemize}
\begin{itemize}
\item {Proveniência:(De \textunderscore lutherano\textunderscore )}
\end{itemize}
Doutrina dos Lutheranos.
Seita religiosa, fundada por Luthero.
\section{Lutherano}
\begin{itemize}
\item {Grp. gram.:adj.}
\end{itemize}
\begin{itemize}
\item {Grp. gram.:M.}
\end{itemize}
\begin{itemize}
\item {Proveniência:(De \textunderscore Luthero\textunderscore , n. p.)}
\end{itemize}
Relativo á doutrina dos Lutheranos.
Sectário de Luthero ou das suas doutrinas.
\section{Luto}
\begin{itemize}
\item {Grp. gram.:m.}
\end{itemize}
\begin{itemize}
\item {Proveniência:(Lat. \textunderscore luctus\textunderscore )}
\end{itemize}
Sentimento ou pesar pela morte de alguém.
Crepe ou pano negro, que se emprega ou se traja depois do fallecimento de uma pessôa de família ou de alguém, cuja falta obrigue, official ou particularmente, a igual manifestação de sentimento.
Tristeza; morte.
\section{Luto}
\begin{itemize}
\item {Grp. gram.:m.}
\end{itemize}
\begin{itemize}
\item {Proveniência:(Lat. \textunderscore lutum\textunderscore , lodo)}
\end{itemize}
Espécie de massa, que endurece com o calor, e serve para tapar fendas e impedir a evaporação de substâncias voláteis ou gasosas.
\section{Lutoca}
\begin{itemize}
\item {Grp. gram.:f.}
\end{itemize}
\begin{itemize}
\item {Proveniência:(T. lund.)}
\end{itemize}
Arbusto africano, de fôlhas glaucas, serreadas, e flôres terminaes.
\section{Lutombo}
\begin{itemize}
\item {Grp. gram.:m.}
\end{itemize}
O mesmo que \textunderscore bordão\textunderscore ^3, árvore.
\section{Lutoso}
\begin{itemize}
\item {Grp. gram.:adj.}
\end{itemize}
\begin{itemize}
\item {Proveniência:(Lat. \textunderscore lutosus\textunderscore )}
\end{itemize}
Que tem muito lodo; lodoso.
\section{Lutulência}
\begin{itemize}
\item {Grp. gram.:f.}
\end{itemize}
Qualidade de lutulento.
\section{Lutulento}
\begin{itemize}
\item {Grp. gram.:adj.}
\end{itemize}
\begin{itemize}
\item {Proveniência:(Lat. \textunderscore lutulentus\textunderscore )}
\end{itemize}
Que tem lodo.
Lamacento.
\section{Lutumbo}
\begin{itemize}
\item {Grp. gram.:m.}
\end{itemize}
Arbusto africano, de caule herbáceo, piloso, de 12 a 20 fôlhas em cada pé, e flôres gamopétalas, com corolla côr de rosa.
\section{Lutuosa}
\begin{itemize}
\item {Grp. gram.:f.}
\end{itemize}
\begin{itemize}
\item {Utilização:Ant.}
\end{itemize}
\begin{itemize}
\item {Proveniência:(De \textunderscore lutuoso\textunderscore )}
\end{itemize}
Direito, que os donatários recebiam por morte dos seus rendeiros, e os Bispos pela vagatura de uma igreja que delles dependesse.
\section{Lutuoso}
\begin{itemize}
\item {Grp. gram.:adj.}
\end{itemize}
\begin{itemize}
\item {Utilização:Fig.}
\end{itemize}
\begin{itemize}
\item {Proveniência:(Lat. \textunderscore luctuosus\textunderscore )}
\end{itemize}
Coberto de luto.
Lúgubre; triste.
\section{Luva}
\begin{itemize}
\item {Grp. gram.:f.}
\end{itemize}
\begin{itemize}
\item {Utilização:Prov.}
\end{itemize}
\begin{itemize}
\item {Utilização:Náut.}
\end{itemize}
\begin{itemize}
\item {Grp. gram.:Pl.}
\end{itemize}
\begin{itemize}
\item {Proveniência:(Ingl. \textunderscore glove\textunderscore )}
\end{itemize}
Peça de vestuário, para cobrir a mão e cada um dos dedos separadamente.
Utensílio de crina, com que se limpam bêstas.
Bolsa ou rêde metállica com que se limpam árvores.
\textunderscore Dar de luva\textunderscore , encher (a vela) pelo bordo contrário, por descuido do timoneiro ou por salto de vento.
\textunderscore Luva de nossa-senhora\textunderscore , planta escrofularínea.
\textunderscore Atirar a luva\textunderscore , accusar, reptar, provocar.
\textunderscore Levantar a luva\textunderscore , acceitar o repto; tirar vingança.
Recompensa, brinde em reconhecimento de um serviço ou favor: \textunderscore receber bôas luvas\textunderscore .
\section{Luvaria}
\begin{itemize}
\item {Grp. gram.:f.}
\end{itemize}
Fábrica de luvas.
Estabelecimento, em que se vendem luvas.
\section{Luvas-de-santa-maria}
\begin{itemize}
\item {Grp. gram.:f. pl.}
\end{itemize}
O mesmo que \textunderscore dedaleira\textunderscore .
\section{Luveira}
\begin{itemize}
\item {Grp. gram.:f.}
\end{itemize}
Mulher, que fabríca ou vende luvas.
(Cp. \textunderscore luveiro\textunderscore )
\section{Luveiro}
\begin{itemize}
\item {Grp. gram.:m.}
\end{itemize}
Vendedor ou fabricante de luvas.
\section{Luvista}
\begin{itemize}
\item {Grp. gram.:m.}
\end{itemize}
O mesmo que \textunderscore luveiro\textunderscore .
\section{Luxação}
\begin{itemize}
\item {Grp. gram.:f.}
\end{itemize}
\begin{itemize}
\item {Proveniência:(Lat. \textunderscore luxatio\textunderscore )}
\end{itemize}
Saída da extremidade articular de um ôsso, para fóra da cavidade que lhe é própria, em virtude de um esfôrço externo ou de uma alteração orgânica.
\section{Luxado}
\begin{itemize}
\item {Grp. gram.:adj.}
\end{itemize}
\begin{itemize}
\item {Proveniência:(De \textunderscore luxar\textunderscore ^1)}
\end{itemize}
Deslocado, desarticulado.
\section{Luxar}
\begin{itemize}
\item {Grp. gram.:v. t.}
\end{itemize}
\begin{itemize}
\item {Utilização:Ant.}
\end{itemize}
\begin{itemize}
\item {Utilização:Prov.}
\end{itemize}
\begin{itemize}
\item {Utilização:alg.}
\end{itemize}
\begin{itemize}
\item {Proveniência:(Lat. \textunderscore luxare\textunderscore )}
\end{itemize}
Deslocar.
Desconjuntar.
Tirar para fóra da cavidade ou superfície própria (um osso articulado).
Desarticular.
Defecar, sujando:«\textunderscore santinho está luxado e luxou a manta toda.\textunderscore »G. Vicente, \textunderscore Auto da Lusit.\textunderscore 
Sujar, emporcalhar.
\section{Luxar}
\begin{itemize}
\item {Grp. gram.:v. t.}
\end{itemize}
\begin{itemize}
\item {Proveniência:(Do lat. \textunderscore luxari\textunderscore )}
\end{itemize}
Ostentar luxo.
Trajar luxuosamente.
\section{Luxamento}
\begin{itemize}
\item {Grp. gram.:m.}
\end{itemize}
\begin{itemize}
\item {Utilização:Ant.}
\end{itemize}
\begin{itemize}
\item {Proveniência:(De \textunderscore luxar\textunderscore ^1)}
\end{itemize}
Aviltamento.
Perversão.
\section{Luxaria}
\begin{itemize}
\item {Grp. gram.:f.}
\end{itemize}
\begin{itemize}
\item {Utilização:Bras}
\end{itemize}
Luxo demasiado. Cf. Taunay, \textunderscore Innocência\textunderscore , 45.
\section{Luxemburguês}
\begin{itemize}
\item {Grp. gram.:adj.}
\end{itemize}
\begin{itemize}
\item {Grp. gram.:M.}
\end{itemize}
Relativo ao Luxemburgo.
Habitante do Luxemburgo.
\section{Luxento}
\begin{itemize}
\item {Grp. gram.:adj.}
\end{itemize}
Que usa luxo; luxuoso.
\section{Luxeta}
\begin{itemize}
\item {fónica:xê}
\end{itemize}
\begin{itemize}
\item {Grp. gram.:f.}
\end{itemize}
Pá de mineiro.
(Cp. \textunderscore luxoso\textunderscore ^2)
\section{Luxo}
\begin{itemize}
\item {Grp. gram.:m.}
\end{itemize}
\begin{itemize}
\item {Proveniência:(Lat. \textunderscore luxus\textunderscore )}
\end{itemize}
Ostentação ou magnificência.
Profusão de ornatos; ornamento.
Viço.
Superfluidade.
\section{Luxo}
\begin{itemize}
\item {Grp. gram.:adj.}
\end{itemize}
\begin{itemize}
\item {Utilização:Prov.}
\end{itemize}
\begin{itemize}
\item {Utilização:alg.}
\end{itemize}
O mesmo que \textunderscore luxado\textunderscore .
\section{Luxoso}
\begin{itemize}
\item {Grp. gram.:adj.}
\end{itemize}
\begin{itemize}
\item {Utilização:Pop.}
\end{itemize}
O mesmo que \textunderscore luxuoso\textunderscore .
\section{Luxoso}
\begin{itemize}
\item {Grp. gram.:adj.}
\end{itemize}
\begin{itemize}
\item {Utilização:Prov.}
\end{itemize}
\begin{itemize}
\item {Utilização:alg.}
\end{itemize}
Sujo.
(Cp. \textunderscore luxar\textunderscore ^1)
\section{Luxuário}
\begin{itemize}
\item {Grp. gram.:adj.}
\end{itemize}
Relativo a luxo. Cf. Herculano, \textunderscore Quest. Pub.\textunderscore , II, 277.
\section{Luxuliana}
\begin{itemize}
\item {Grp. gram.:f.}
\end{itemize}
Espécie de granito, semelhante ao pórphyro.
\section{Luxuosamente}
\begin{itemize}
\item {Grp. gram.:adv.}
\end{itemize}
De modo luxuoso.
\section{Luxuosidade}
\begin{itemize}
\item {Grp. gram.:f.}
\end{itemize}
Qualidade de luxuoso. Cf. Castilho, \textunderscore Fastos\textunderscore , I, 179.
\section{Luxuoso}
\begin{itemize}
\item {Grp. gram.:adj.}
\end{itemize}
Que traja com luxo; que vive ostentosamente.
Que ostenta luxo.
\section{Luxúria}
\begin{itemize}
\item {Grp. gram.:f.}
\end{itemize}
\begin{itemize}
\item {Proveniência:(Lat. \textunderscore luxuria\textunderscore )}
\end{itemize}
Viço dos vegetaes. Incontinência nos animaes.
Libertinagem; corrupção.
Lascívia; sensualidade.
\section{Luxuriante}
\begin{itemize}
\item {Grp. gram.:adj.}
\end{itemize}
\begin{itemize}
\item {Proveniência:(Lat. \textunderscore luxurians\textunderscore )}
\end{itemize}
Viçoso.
Exuberante.
Luxurioso.
\section{Luxuriar}
\begin{itemize}
\item {Grp. gram.:v. i.}
\end{itemize}
\begin{itemize}
\item {Utilização:Fig.}
\end{itemize}
\begin{itemize}
\item {Proveniência:(Lat. \textunderscore luxuriari\textunderscore )}
\end{itemize}
Vicejar; desenvolver-se.
Entregar-se a licenciosidades.
\section{Luxuriosamente}
\begin{itemize}
\item {Grp. gram.:adv.}
\end{itemize}
De modo luxurioso; com sensualidade.
\section{Luxurioso}
\begin{itemize}
\item {Grp. gram.:adj.}
\end{itemize}
\begin{itemize}
\item {Utilização:Fig.}
\end{itemize}
\begin{itemize}
\item {Proveniência:(Lat. \textunderscore luxuriosus\textunderscore )}
\end{itemize}
Viçoso.
Sensual; licencioso; dissoluto.
\section{Luz}
\begin{itemize}
\item {Grp. gram.:f.}
\end{itemize}
\begin{itemize}
\item {Utilização:Fig.}
\end{itemize}
\begin{itemize}
\item {Utilização:Bras. do S}
\end{itemize}
\begin{itemize}
\item {Grp. gram.:Pl.}
\end{itemize}
\begin{itemize}
\item {Proveniência:(Lat. \textunderscore lux\textunderscore )}
\end{itemize}
Propriedade dos corpos, que determina o phenómeno da visão e se manifesta pelas côres.
Aquillo que torna os objectos visíveis.
Clarão, produzido por uma substância em ignição.
O dia.
Brilho.
Vela, candeeiro, etc., cuja torcida está inflammada: \textunderscore apagar a luz\textunderscore .
Publicidade: \textunderscore dar um livro á luz\textunderscore .
Evidência.
Illustração.
Pessôa illustrada.
Conhecimentos.
Civilização.
Os pontos de um quadro, em que o artista imita a luz.
Espaço de terreno, que um parelheiro leva de deanteira a outro.
\textunderscore Dar á luz\textunderscore , parir.
A sciência.
Progresso.
Noções, conhecimentos.
\section{Luzarda}
\begin{itemize}
\item {Grp. gram.:f.}
\end{itemize}
\begin{itemize}
\item {Utilização:Prov.}
\end{itemize}
Espécie de lâmpada com trempe, em que se faz café, se fritam ovos, etc.
\section{Luzarra}
\begin{itemize}
\item {Grp. gram.:f.}
\end{itemize}
\begin{itemize}
\item {Utilização:Prov.}
\end{itemize}
\begin{itemize}
\item {Utilização:dur.}
\end{itemize}
Bichinho pouco conhecido, também chamado \textunderscore rela\textunderscore .
\section{Luzecu}
\begin{itemize}
\item {Grp. gram.:m.}
\end{itemize}
\begin{itemize}
\item {Utilização:Pop.}
\end{itemize}
O mesmo que \textunderscore luzecuco\textunderscore .
\section{Luzecuco}
\begin{itemize}
\item {Grp. gram.:m.}
\end{itemize}
\begin{itemize}
\item {Utilização:Prov.}
\end{itemize}
\begin{itemize}
\item {Utilização:alg.}
\end{itemize}
O mesmo que \textunderscore pyrilampo\textunderscore .
\section{Luzeira}
\begin{itemize}
\item {Grp. gram.:f.}
\end{itemize}
\begin{itemize}
\item {Utilização:Ant.}
\end{itemize}
\begin{itemize}
\item {Proveniência:(De \textunderscore luz\textunderscore )}
\end{itemize}
Lâmpada.
\section{Luzeiro}
\begin{itemize}
\item {Grp. gram.:m.}
\end{itemize}
\begin{itemize}
\item {Utilização:Fig.}
\end{itemize}
\begin{itemize}
\item {Utilização:Gír.}
\end{itemize}
\begin{itemize}
\item {Grp. gram.:Pl.}
\end{itemize}
\begin{itemize}
\item {Utilização:Pop.}
\end{itemize}
\begin{itemize}
\item {Proveniência:(De \textunderscore luz\textunderscore )}
\end{itemize}
Coisa que emite luz.
Brilho.
Astro.
Homem illustre, luminar.
O mesmo que \textunderscore brilhante\textunderscore .
Os olhos.
\section{Luze-luze}
\begin{itemize}
\item {Grp. gram.:m.}
\end{itemize}
\begin{itemize}
\item {Utilização:Pop.}
\end{itemize}
O mesmo que \textunderscore pyrilampo\textunderscore .
\section{Luz-em-cu}
\begin{itemize}
\item {Grp. gram.:m.}
\end{itemize}
\begin{itemize}
\item {Utilização:Prov.}
\end{itemize}
\begin{itemize}
\item {Utilização:alent.}
\end{itemize}
Pyrilampo.
Vagalume.
\section{Luzença}
\begin{itemize}
\item {Grp. gram.:f.}
\end{itemize}
\begin{itemize}
\item {Utilização:Ant.}
\end{itemize}
\begin{itemize}
\item {Proveniência:(Do rad. de \textunderscore luzente\textunderscore )}
\end{itemize}
Acto de luzir; luz.
\section{Luzente}
\begin{itemize}
\item {Grp. gram.:adj.}
\end{itemize}
\begin{itemize}
\item {Grp. gram.:M.}
\end{itemize}
\begin{itemize}
\item {Utilização:Gír.}
\end{itemize}
\begin{itemize}
\item {Proveniência:(Do lat. \textunderscore lucens\textunderscore )}
\end{itemize}
Que luz; luminoso.
Pedra preciosa.
\section{Luzerna}
\begin{itemize}
\item {Grp. gram.:f.}
\end{itemize}
\begin{itemize}
\item {Proveniência:(Ingl. \textunderscore lucern\textunderscore )}
\end{itemize}
Nome de várias plantas leguminosas.
\section{Luzerna}
\begin{itemize}
\item {Grp. gram.:f.}
\end{itemize}
\begin{itemize}
\item {Proveniência:(Do lat. \textunderscore lucerna\textunderscore )}
\end{itemize}
Grande luz; clarão.
O mesmo que \textunderscore lucarna\textunderscore .
\section{Luzernal}
\begin{itemize}
\item {Grp. gram.:m.}
\end{itemize}
O mesmo que \textunderscore luzerneira\textunderscore .
\section{Luzerneira}
\begin{itemize}
\item {Grp. gram.:f.}
\end{itemize}
\begin{itemize}
\item {Proveniência:(De \textunderscore luzerna\textunderscore ^2)}
\end{itemize}
Terreno, em que crescem luzernas.
\section{Luzetro}
\begin{itemize}
\item {Grp. gram.:m.}
\end{itemize}
(V.maleiteira)
\section{Luzica}
\begin{itemize}
\item {Grp. gram.:m.}
\end{itemize}
\begin{itemize}
\item {Utilização:Prov.}
\end{itemize}
\begin{itemize}
\item {Utilização:minh.}
\end{itemize}
O mesmo que \textunderscore luzecu\textunderscore .
\section{Luzida}
\begin{itemize}
\item {Grp. gram.:f.}
\end{itemize}
\begin{itemize}
\item {Utilização:Gír.}
\end{itemize}
\begin{itemize}
\item {Proveniência:(De \textunderscore luzir\textunderscore )}
\end{itemize}
Festa.
\section{Luzidamente}
\begin{itemize}
\item {Grp. gram.:adv.}
\end{itemize}
De modo luzido.
Com brilho.
\section{Luzidia}
\begin{itemize}
\item {Grp. gram.:f.}
\end{itemize}
\begin{itemize}
\item {Proveniência:(De \textunderscore luzidio\textunderscore )}
\end{itemize}
Casta de uva branca do Minho.
\section{Luzidio}
\begin{itemize}
\item {Grp. gram.:adj.}
\end{itemize}
\begin{itemize}
\item {Proveniência:(Do rad. de \textunderscore luzido\textunderscore )}
\end{itemize}
Que luz muito.
Nítido; brilhante.
\section{Luzido}
\begin{itemize}
\item {Grp. gram.:adj.}
\end{itemize}
\begin{itemize}
\item {Proveniência:(De \textunderscore luzir\textunderscore )}
\end{itemize}
Cheio de luz.
Ostentoso; pomposo.
Brilhante.
\section{Luzimento}
\begin{itemize}
\item {Grp. gram.:m.}
\end{itemize}
Acto ou effeito de luzir; ostentação; esplendor: \textunderscore o luzimento da festa\textunderscore .
\section{Luzincu}
\begin{itemize}
\item {Grp. gram.:m.}
\end{itemize}
\begin{itemize}
\item {Utilização:Prov.}
\end{itemize}
\begin{itemize}
\item {Utilização:minh.}
\end{itemize}
O mesmo que \textunderscore pyrilampo\textunderscore .
(Cp. \textunderscore luz-em-cu\textunderscore )
\section{Lúzio}
\begin{itemize}
\item {Grp. gram.:m.}
\end{itemize}
Espécie de embarcação da África austral.
\section{Lúzio}
\begin{itemize}
\item {Grp. gram.:m.}
\end{itemize}
\begin{itemize}
\item {Utilização:Gír.}
\end{itemize}
\begin{itemize}
\item {Utilização:Gír.}
\end{itemize}
\begin{itemize}
\item {Proveniência:(De \textunderscore luz\textunderscore )}
\end{itemize}
O ôlho.
Lampeão.
\section{Luzir}
\begin{itemize}
\item {Grp. gram.:v. i.}
\end{itemize}
\begin{itemize}
\item {Grp. gram.:V. t.}
\end{itemize}
\begin{itemize}
\item {Utilização:Des.}
\end{itemize}
\begin{itemize}
\item {Proveniência:(Lat. \textunderscore lucere\textunderscore )}
\end{itemize}
Dar luz.
Brilhar.
Desenvolver-se; mostrar que aproveita.
Illuminar; illustrar.
\section{Luzuangua}
\begin{itemize}
\item {Grp. gram.:f.}
\end{itemize}
Árvore angolense, no Duque-de-Bragança.
\section{Lúzula}
\begin{itemize}
\item {Grp. gram.:f.}
\end{itemize}
Gênero de plantas juncáceas.
\section{Lxa}
Abrev. de \textunderscore Lisbôa\textunderscore , ou antes de \textunderscore Lixbôa\textunderscore :«\textunderscore como os infantes partiram de Lixbôa...\textunderscore »Rui de Pina, \textunderscore Chrón. de D. Duarte\textunderscore .
\section{Lycanthropia}
\begin{itemize}
\item {Grp. gram.:f.}
\end{itemize}
Doença mental, em que o enfermo se suppõe transformado em lôbo.
(Cp. \textunderscore lycanthropo\textunderscore )
\section{Licantropo}
\begin{itemize}
\item {Grp. gram.:m.}
\end{itemize}
\begin{itemize}
\item {Proveniência:(Do gr. \textunderscore lucos\textunderscore  + \textunderscore anthropos\textunderscore )}
\end{itemize}
Aquele que sofre licantropia.
\section{Liceal}
\begin{itemize}
\item {Grp. gram.:adj.}
\end{itemize}
\begin{itemize}
\item {Utilização:Neol.}
\end{itemize}
Relativo a liceu: \textunderscore o ensino liceal\textunderscore .
\section{Licena}
\begin{itemize}
\item {Grp. gram.:f.}
\end{itemize}
\begin{itemize}
\item {Proveniência:(Do gr. \textunderscore lukaina\textunderscore )}
\end{itemize}
Gênero de insectos lepidópteros diúrnos.
\section{Licenídios}
\begin{itemize}
\item {Grp. gram.:m.}
\end{itemize}
\begin{itemize}
\item {Proveniência:(Do gr. \textunderscore lukaina\textunderscore  + \textunderscore eidos\textunderscore )}
\end{itemize}
Fam. de lepidópteros.
\section{Licênios}
\begin{itemize}
\item {Grp. gram.:m. pl.}
\end{itemize}
\begin{itemize}
\item {Proveniência:(De \textunderscore licena\textunderscore )}
\end{itemize}
Tríbo de lepidópteros.
\section{Licetol}
\begin{itemize}
\item {Grp. gram.:m.}
\end{itemize}
Medicamento contra a diátese úrica.
\section{Liceu}
\begin{itemize}
\item {Grp. gram.:m.}
\end{itemize}
\begin{itemize}
\item {Utilização:Ext.}
\end{itemize}
\begin{itemize}
\item {Proveniência:(Lat. \textunderscore lyceum\textunderscore )}
\end{itemize}
Instituto oficial de instrucção secundária.
Colégio ou estabelecimento particular de instrucção secundária.
\section{Lício}
\begin{itemize}
\item {Grp. gram.:m.}
\end{itemize}
\begin{itemize}
\item {Proveniência:(Lat. \textunderscore lycium\textunderscore )}
\end{itemize}
Arbusto espinhoso, cujo suco era empregado em Medicina.
\section{Licne}
\begin{itemize}
\item {Grp. gram.:f.}
\end{itemize}
\begin{itemize}
\item {Proveniência:(Lat. \textunderscore lychnis\textunderscore )}
\end{itemize}
Pedra preciosa, muito brilhante, mencionada por Plínio.
Nome, que os Romanos deram a uma variedade de rosas, de côr muito viva.
\section{Lícnis}
\begin{itemize}
\item {Grp. gram.:f.}
\end{itemize}
\begin{itemize}
\item {Proveniência:(Lat. \textunderscore lychnis\textunderscore )}
\end{itemize}
Pedra preciosa, muito brilhante, mencionada por Plínio.
Nome, que os Romanos deram a uma variedade de rosas, de côr muito viva.
\section{Licnite}
\begin{itemize}
\item {Grp. gram.:f.}
\end{itemize}
\begin{itemize}
\item {Proveniência:(Lat. \textunderscore lychnites\textunderscore )}
\end{itemize}
Mármore, que se extrahia das pedreiras de Paros, á luz de lâmpadas.
\section{Licnítide}
\begin{itemize}
\item {Grp. gram.:f.}
\end{itemize}
\begin{itemize}
\item {Proveniência:(Lat. \textunderscore lychnitis\textunderscore )}
\end{itemize}
Planta, de que os antigos faziam mechas.
\section{Licnóbio}
\begin{itemize}
\item {Grp. gram.:m.}
\end{itemize}
\begin{itemize}
\item {Proveniência:(Lat. \textunderscore lychnobius\textunderscore )}
\end{itemize}
Aquele que vela de noite ou faz da noite dia.
\section{Licnomancia}
\begin{itemize}
\item {Grp. gram.:f.}
\end{itemize}
\begin{itemize}
\item {Proveniência:(Do gr. \textunderscore lukhnos\textunderscore  + \textunderscore manteia\textunderscore )}
\end{itemize}
Adivinhação, por meio de lâmpadas ou brandões.
\section{Licnoscopia}
\begin{itemize}
\item {Grp. gram.:f.}
\end{itemize}
\begin{itemize}
\item {Proveniência:(Do gr. \textunderscore lukhnos\textunderscore  + \textunderscore skopein\textunderscore )}
\end{itemize}
O mesmo que \textunderscore licnomancia\textunderscore .
\section{Licnuco}
\begin{itemize}
\item {Grp. gram.:m.}
\end{itemize}
\begin{itemize}
\item {Proveniência:(Lat. \textunderscore lychnuchus\textunderscore )}
\end{itemize}
Lâmpada ou lampadário, entre os Gregos e Romanos.
\section{Licodite}
\begin{itemize}
\item {Grp. gram.:f.}
\end{itemize}
Substância explosiva.
\section{Licoperdáceas}
\begin{itemize}
\item {Grp. gram.:f. pl.}
\end{itemize}
\begin{itemize}
\item {Utilização:Bot.}
\end{itemize}
Família de cogumelos, no sistema de Brogniart.
(Cp. \textunderscore licoperdo\textunderscore )
\section{Licoperdíneas}
\begin{itemize}
\item {Grp. gram.:f. pl.}
\end{itemize}
\begin{itemize}
\item {Proveniência:(De \textunderscore licoperdo\textunderscore )}
\end{itemize}
Tríbo de licoperdáceas.
\section{Licoperdo}
\begin{itemize}
\item {Grp. gram.:m.}
\end{itemize}
\begin{itemize}
\item {Proveniência:(Do gr. \textunderscore lukos\textunderscore  + \textunderscore perdein\textunderscore )}
\end{itemize}
Gênero de cogumelos.
\section{Licoperdóneas}
\begin{itemize}
\item {Grp. gram.:f. pl.}
\end{itemize}
\begin{itemize}
\item {Proveniência:(De \textunderscore licoperdo\textunderscore )}
\end{itemize}
Nome, criado por Merat, para designar um grupo de cogumelos.
\section{Licopérsico}
\begin{itemize}
\item {Grp. gram.:m.}
\end{itemize}
\begin{itemize}
\item {Proveniência:(Do gr. \textunderscore lukos\textunderscore  + \textunderscore persikon\textunderscore )}
\end{itemize}
Gênero de árvores solanáceas.
\section{Licopódeas}
\begin{itemize}
\item {Grp. gram.:f. pl.}
\end{itemize}
(V.licopodiáceas)
\section{Licopodiáceas}
\begin{itemize}
\item {Grp. gram.:f. pl.}
\end{itemize}
\begin{itemize}
\item {Proveniência:(De \textunderscore licopodiáceo\textunderscore )}
\end{itemize}
Família de plantas, que tem por tipo o licopódio.
\section{Licopodiáceo}
\begin{itemize}
\item {Grp. gram.:adj.}
\end{itemize}
Relativo ou semelhante ao licopódio.
\section{Licopodina}
\begin{itemize}
\item {Grp. gram.:f.}
\end{itemize}
\begin{itemize}
\item {Utilização:Chím.}
\end{itemize}
Princípio azotado, que se encontra no licopódio.
\section{Licopodíneas}
\begin{itemize}
\item {Grp. gram.:f. pl.}
\end{itemize}
(V.licopodiáceas)
\section{Licopódio}
\begin{itemize}
\item {Grp. gram.:m.}
\end{itemize}
\begin{itemize}
\item {Proveniência:(Do gr. \textunderscore lukos\textunderscore  + \textunderscore pous\textunderscore , \textunderscore podos\textunderscore )}
\end{itemize}
Gênero de plantas cryptògâmicas e trepadeiras, de que há duas espécies na Europa.
\section{Licópode}
\begin{itemize}
\item {Grp. gram.:m.}
\end{itemize}
O mesmo que \textunderscore licopódio\textunderscore .
\section{Licópside}
\begin{itemize}
\item {Grp. gram.:f.}
\end{itemize}
\begin{itemize}
\item {Proveniência:(Do gr. \textunderscore lukos\textunderscore  + \textunderscore opsis\textunderscore )}
\end{itemize}
Gênero de plantas borragíneas.
\section{Licorexia}
\begin{itemize}
\item {fónica:csi}
\end{itemize}
\begin{itemize}
\item {Grp. gram.:f.}
\end{itemize}
\begin{itemize}
\item {Utilização:Med.}
\end{itemize}
\begin{itemize}
\item {Proveniência:(Do gr. \textunderscore lukos\textunderscore  + \textunderscore orexis\textunderscore )}
\end{itemize}
Variedade de bulímia.
\section{Licose}
\begin{itemize}
\item {Grp. gram.:f.}
\end{itemize}
O mesmo que \textunderscore tarântula\textunderscore .
\section{Lídio}
\begin{itemize}
\item {Grp. gram.:adj.}
\end{itemize}
\begin{itemize}
\item {Grp. gram.:M.}
\end{itemize}
\begin{itemize}
\item {Proveniência:(Do lat. \textunderscore lydius\textunderscore )}
\end{itemize}
Relativo á Lídia.
Aquele que é natural da Lídia.
\section{Lince}
\begin{itemize}
\item {Grp. gram.:m.}
\end{itemize}
\begin{itemize}
\item {Proveniência:(Do gr. \textunderscore lunx\textunderscore )}
\end{itemize}
Quadrúpede carnívoro, também conhecido por \textunderscore lôbo cerval\textunderscore .
Constelação boreal.
\section{Linchagem}
\begin{itemize}
\item {Grp. gram.:f.}
\end{itemize}
Processo ou acto de linchar.
\section{Linchamento}
\begin{itemize}
\item {Grp. gram.:m.}
\end{itemize}
Acto de linchar.
\section{Linchar}
\begin{itemize}
\item {Grp. gram.:v. t.}
\end{itemize}
\begin{itemize}
\item {Proveniência:(De \textunderscore Lynch\textunderscore , n. p.)}
\end{itemize}
Justiçar ou executar summariamente, segundo o processo instituido por Lynch, nos Estados Unidos.
\section{Linfa}
\begin{itemize}
\item {Grp. gram.:f.}
\end{itemize}
\begin{itemize}
\item {Utilização:Poét.}
\end{itemize}
\begin{itemize}
\item {Proveniência:(Lat. \textunderscore lympha\textunderscore )}
\end{itemize}
Líquido branco e nutritivo, contido em certos vasos do organismo.
Humor aquoso das plantas.
Água.
\section{Linfado}
\begin{itemize}
\item {Grp. gram.:adj.}
\end{itemize}
\begin{itemize}
\item {Utilização:Des.}
\end{itemize}
\begin{itemize}
\item {Proveniência:(De \textunderscore linfa\textunderscore )}
\end{itemize}
O mesmo que \textunderscore hidrófobo\textunderscore .
\section{Linfadura}
\begin{itemize}
\item {Grp. gram.:f.}
\end{itemize}
Acto de linfar.
\section{Linfagogo}
\begin{itemize}
\item {Grp. gram.:m.}
\end{itemize}
Substância, que aumenta a produção da linfa.
\section{Linfangioma}
\begin{itemize}
\item {Grp. gram.:m.}
\end{itemize}
Tumor de vasos linfáticos.
(Do lat, \textunderscore lympha\textunderscore  + gr. \textunderscore angeion\textunderscore )
\section{Linfangite}
\begin{itemize}
\item {Grp. gram.:f.}
\end{itemize}
\begin{itemize}
\item {Proveniência:(Do lat. \textunderscore lympha\textunderscore  + gr. \textunderscore angeion\textunderscore )}
\end{itemize}
Inflamação dos vasos linfáticos.
\section{Linfar}
\begin{itemize}
\item {Grp. gram.:v. t.}
\end{itemize}
\begin{itemize}
\item {Proveniência:(De \textunderscore linfa\textunderscore )}
\end{itemize}
Misturar com água, diluir.
\section{Linfático}
\begin{itemize}
\item {Grp. gram.:adj.}
\end{itemize}
Relativo á linfa.
Que contém linfa.
Em que predomina a linfa: \textunderscore temperamento linfático\textunderscore .
\section{Linfatismo}
\begin{itemize}
\item {Grp. gram.:m.}
\end{itemize}
\begin{itemize}
\item {Utilização:Med.}
\end{itemize}
Estado linfático do organismo.
\section{Linfite}
\begin{itemize}
\item {Grp. gram.:f.}
\end{itemize}
O mesmo que \textunderscore linfangite\textunderscore .
\section{Linfocito}
\begin{itemize}
\item {Grp. gram.:m.}
\end{itemize}
O mesmo que [[doença do sono|sono]].
\section{Linfoma}
\begin{itemize}
\item {Grp. gram.:m.}
\end{itemize}
\begin{itemize}
\item {Proveniência:(Do lat. \textunderscore lympha\textunderscore )}
\end{itemize}
Tumor das glândulas linfáticas.
\section{Linforragia}
\begin{itemize}
\item {Grp. gram.:f.}
\end{itemize}
\begin{itemize}
\item {Utilização:Med.}
\end{itemize}
\begin{itemize}
\item {Proveniência:(Do lat. \textunderscore limpha\textunderscore  + gr. \textunderscore rhagein\textunderscore )}
\end{itemize}
Derramamento persistente de linfa, depois de ferido um vaso linfático.
\section{Linfose}
\begin{itemize}
\item {Grp. gram.:f.}
\end{itemize}
\begin{itemize}
\item {Utilização:Med.}
\end{itemize}
\begin{itemize}
\item {Proveniência:(De \textunderscore linfa\textunderscore )}
\end{itemize}
Acto de elaboração especial, de que resulta a linfa.
\section{Linfotomia}
\begin{itemize}
\item {Grp. gram.:f.}
\end{itemize}
\begin{itemize}
\item {Proveniência:(Do lat. \textunderscore lympha\textunderscore  + \textunderscore tome\textunderscore )}
\end{itemize}
Dissecção dos vasos linfáticos.
\section{Lionês}
\begin{itemize}
\item {Grp. gram.:adj.}
\end{itemize}
\begin{itemize}
\item {Grp. gram.:M.}
\end{itemize}
\begin{itemize}
\item {Proveniência:(De \textunderscore Lyon\textunderscore , n. p. fr.)}
\end{itemize}
Relativo a Lião.
Habitante de Lião.
\section{Lipemania}
\begin{itemize}
\item {Grp. gram.:f.}
\end{itemize}
\begin{itemize}
\item {Utilização:Med.}
\end{itemize}
\begin{itemize}
\item {Proveniência:(Do gr. \textunderscore lúpe\textunderscore  + \textunderscore mania\textunderscore )}
\end{itemize}
Espécie de alienação mental, caracterizada por tristeza profunda.
\section{Lira}
\begin{itemize}
\item {Grp. gram.:f.}
\end{itemize}
\begin{itemize}
\item {Utilização:Fig.}
\end{itemize}
\begin{itemize}
\item {Utilização:Gír.}
\end{itemize}
\begin{itemize}
\item {Proveniência:(Gr. \textunderscore lura\textunderscore )}
\end{itemize}
Antigo instrumento músico, de cordas.
Talento poético.
Arte de versejar.
Ave galinácea, (\textunderscore menura lyra\textunderscore ).
Superfície inferior da abóbada dos pilares do cérebro.
Constelação boreal.
Guitarra.
\section{Lirado}
\begin{itemize}
\item {Grp. gram.:adj.}
\end{itemize}
\begin{itemize}
\item {Utilização:Bot.}
\end{itemize}
\begin{itemize}
\item {Proveniência:(De \textunderscore lira\textunderscore )}
\end{itemize}
Diz-se das fôlhas de certas crucíferas, cujos lóbulos superiores são grandes e reunidos, em quanto os inferiores são pequenos e divididos até á nervura média.
\section{Liral}
\begin{itemize}
\item {Grp. gram.:adj.}
\end{itemize}
Dizia-se \textunderscore cravo liral\textunderscore  um instrumento músico, o mesmo que \textunderscore espineta\textunderscore .
\section{Lírica}
\begin{itemize}
\item {Grp. gram.:f.}
\end{itemize}
\begin{itemize}
\item {Proveniência:(De \textunderscore lírico\textunderscore )}
\end{itemize}
O gênero lírico da poesia.
Colecção de poesias líricas.
\section{Lírico}
\begin{itemize}
\item {Grp. gram.:adj.}
\end{itemize}
\begin{itemize}
\item {Grp. gram.:M.}
\end{itemize}
\begin{itemize}
\item {Proveniência:(De \textunderscore lira\textunderscore )}
\end{itemize}
Relativo á poesia que exprime os grandes e os delicados sentimentos pessoaes do poéta.
Relativo á poesia.
Sentimental.
Relativo a óperas: \textunderscore teatro lírico\textunderscore .
Poéta, que cultiva o gênero lírico.
\section{Liriforme}
\begin{itemize}
\item {Grp. gram.:adj.}
\end{itemize}
\begin{itemize}
\item {Proveniência:(De \textunderscore lira\textunderscore  + \textunderscore forma\textunderscore )}
\end{itemize}
Que tem fórma de lira.
\section{Lirismo}
\begin{itemize}
\item {Grp. gram.:m.}
\end{itemize}
\begin{itemize}
\item {Proveniência:(De \textunderscore lira\textunderscore )}
\end{itemize}
Qualidade de lírico, sublime ou sentimental.
Subjectivismo poético.
Entusiasmo.
\section{Lirista}
\begin{itemize}
\item {Grp. gram.:m.}
\end{itemize}
\begin{itemize}
\item {Utilização:Deprec.}
\end{itemize}
\begin{itemize}
\item {Proveniência:(De \textunderscore lira\textunderscore )}
\end{itemize}
Tocador de lira.
Poéta frívolo, banal.
\section{Lirístria}
\begin{itemize}
\item {Grp. gram.:f.}
\end{itemize}
\begin{itemize}
\item {Proveniência:(Lat. \textunderscore lyristria\textunderscore )}
\end{itemize}
A mulher que tocava lira.
\section{Lirodo}
\begin{itemize}
\item {Grp. gram.:m.}
\end{itemize}
\begin{itemize}
\item {Proveniência:(Lat. \textunderscore lyrodus\textunderscore )}
\end{itemize}
Cantador que, entre os antigos, acompanhava o seu canto com o som da lira.
O mesmo que \textunderscore citaredo\textunderscore . Cf. Castilho, \textunderscore Fastos\textunderscore , III, 207.
\section{Lise}
\begin{itemize}
\item {Grp. gram.:f.}
\end{itemize}
\begin{itemize}
\item {Utilização:Med.}
\end{itemize}
\begin{itemize}
\item {Proveniência:(Lat. \textunderscore lysis\textunderscore )}
\end{itemize}
Defervescência lenta e gradual da febre.
Terminação lenta de uma doença.
\section{Lisidina}
\begin{itemize}
\item {Grp. gram.:f.}
\end{itemize}
Composto químico, contra a doença da gota.
\section{Lisimáquia}
\begin{itemize}
\item {Grp. gram.:f.}
\end{itemize}
\begin{itemize}
\item {Proveniência:(De \textunderscore Lisimacho\textunderscore , n. p.)}
\end{itemize}
Gênero de plantas primuláceas.
\section{Lísio}
\begin{itemize}
\item {Grp. gram.:adj.}
\end{itemize}
\begin{itemize}
\item {Proveniência:(Do gr. \textunderscore lusis\textunderscore )}
\end{itemize}
Resultante de uma dissolução química.
\section{Lisofórmio}
\begin{itemize}
\item {Grp. gram.:m.}
\end{itemize}
Preparação antiséptica, em que entra o lisol e a formalina?
\section{Lisol}
\begin{itemize}
\item {Grp. gram.:m.}
\end{itemize}
Composto químico, utilizado em algumas indústrias.
\section{Lisolagem}
\begin{itemize}
\item {Grp. gram.:f.}
\end{itemize}
Emprêgo do lisol.
\section{Lissofobia}
\begin{itemize}
\item {Grp. gram.:f.}
\end{itemize}
Temor mórbido da hidrofobia.
\section{Litrariadas}
\begin{itemize}
\item {Grp. gram.:f. pl.}
\end{itemize}
\begin{itemize}
\item {Proveniência:(Do lat. \textunderscore lythrum\textunderscore )}
\end{itemize}
Família de plantas dicotiledóneas, a que pertence a salgueirinha, (\textunderscore lythrum\textunderscore ), cujas flôres tem a côr do sangue coalhado.
\section{Litro}
\begin{itemize}
\item {Grp. gram.:m.}
\end{itemize}
\begin{itemize}
\item {Proveniência:(Gr. \textunderscore luthron\textunderscore )}
\end{itemize}
Planta, de flôres vermelhas, do tipo da litraríadas.
\section{Lycanthropo}
\begin{itemize}
\item {Grp. gram.:m.}
\end{itemize}
\begin{itemize}
\item {Proveniência:(Do gr. \textunderscore lucos\textunderscore  + \textunderscore anthropos\textunderscore )}
\end{itemize}
Aquelle que sofre lycanthropia.
\section{Lyceal}
\begin{itemize}
\item {Grp. gram.:adj.}
\end{itemize}
\begin{itemize}
\item {Utilização:Neol.}
\end{itemize}
Relativo a lyceu: \textunderscore o ensino lyceal\textunderscore .
\section{Lycena}
\begin{itemize}
\item {Grp. gram.:f.}
\end{itemize}
\begin{itemize}
\item {Proveniência:(Do gr. \textunderscore lukaina\textunderscore )}
\end{itemize}
Gênero de insectos lepidópteros diúrnos.
\section{Lycenídios}
\begin{itemize}
\item {Grp. gram.:m.}
\end{itemize}
\begin{itemize}
\item {Proveniência:(Do gr. \textunderscore lukaina\textunderscore  + \textunderscore eidos\textunderscore )}
\end{itemize}
Fam. de lepidópteros.
\section{Lycênios}
\begin{itemize}
\item {Grp. gram.:m. pl.}
\end{itemize}
\begin{itemize}
\item {Proveniência:(De \textunderscore lycena\textunderscore )}
\end{itemize}
Tríbo de lepidópteros.
\section{Lycetol}
\begin{itemize}
\item {Grp. gram.:m.}
\end{itemize}
Medicamento contra a diáthese úrica.
\section{Lyceu}
\begin{itemize}
\item {Grp. gram.:m.}
\end{itemize}
\begin{itemize}
\item {Utilização:Ext.}
\end{itemize}
\begin{itemize}
\item {Proveniência:(Lat. \textunderscore lyceum\textunderscore )}
\end{itemize}
Instituto official de instrucção secundária.
Collégio ou estabelecimento particular de instrucção secundária.
\section{Lýchnis}
\begin{itemize}
\item {Grp. gram.:f.}
\end{itemize}
\begin{itemize}
\item {Proveniência:(Lat. \textunderscore lychnis\textunderscore )}
\end{itemize}
Pedra preciosa, muito brilhante, mencionada por Plínio.
Nome, que os Romanos deram a uma variedade de rosas, de côr muito viva.
\section{Lychnite}
\begin{itemize}
\item {Grp. gram.:f.}
\end{itemize}
\begin{itemize}
\item {Proveniência:(Lat. \textunderscore lychnites\textunderscore )}
\end{itemize}
Mármore, que se extrahia das pedreiras de Paros, á luz de lâmpadas.
\section{Lychnítide}
\begin{itemize}
\item {Grp. gram.:f.}
\end{itemize}
\begin{itemize}
\item {Proveniência:(Lat. \textunderscore lychnitis\textunderscore )}
\end{itemize}
Planta, de que os antigos faziam mechas.
\section{Lychnóbio}
\begin{itemize}
\item {Grp. gram.:m.}
\end{itemize}
\begin{itemize}
\item {Proveniência:(Lat. \textunderscore lychnobius\textunderscore )}
\end{itemize}
Aquelle que vela de noite ou faz da noite dia.
\section{Lychnomancia}
\begin{itemize}
\item {Grp. gram.:f.}
\end{itemize}
\begin{itemize}
\item {Proveniência:(Do gr. \textunderscore lukhnos\textunderscore  + \textunderscore manteia\textunderscore )}
\end{itemize}
Adivinhação, por meio de lâmpadas ou brandões.
\section{Lychnoscopia}
\begin{itemize}
\item {Grp. gram.:f.}
\end{itemize}
\begin{itemize}
\item {Proveniência:(Do gr. \textunderscore lukhnos\textunderscore  + \textunderscore skopein\textunderscore )}
\end{itemize}
O mesmo que \textunderscore licnomancia\textunderscore .
\section{Lychnucho}
\begin{itemize}
\item {fónica:co}
\end{itemize}
\begin{itemize}
\item {Grp. gram.:m.}
\end{itemize}
\begin{itemize}
\item {Proveniência:(Lat. \textunderscore lychnuchus\textunderscore )}
\end{itemize}
Lâmpada ou lampadário, entre os Gregos e Romanos.
\section{Lýcio}
\begin{itemize}
\item {Grp. gram.:m.}
\end{itemize}
\begin{itemize}
\item {Proveniência:(Lat. \textunderscore lycium\textunderscore )}
\end{itemize}
Arbusto espinhoso, cujo suco era empregado em Medicina.
\section{Lycodite}
\begin{itemize}
\item {Grp. gram.:f.}
\end{itemize}
Substância explosiva.
\section{Lycoperdáceas}
\begin{itemize}
\item {Grp. gram.:f. pl.}
\end{itemize}
\begin{itemize}
\item {Utilização:Bot.}
\end{itemize}
Família de cogumelos, no systema de Brogniart.
(Cp. \textunderscore lycoperdo\textunderscore )
\section{Lycoperdíneas}
\begin{itemize}
\item {Grp. gram.:f. pl.}
\end{itemize}
\begin{itemize}
\item {Proveniência:(De \textunderscore lycoperdo\textunderscore )}
\end{itemize}
Tríbo de lycoperdáceas.
\section{Lycoperdo}
\begin{itemize}
\item {Grp. gram.:m.}
\end{itemize}
\begin{itemize}
\item {Proveniência:(Do gr. \textunderscore lukos\textunderscore  + \textunderscore perdein\textunderscore )}
\end{itemize}
Gênero de cogumelos.
\section{Lycoperdóneas}
\begin{itemize}
\item {Grp. gram.:f. pl.}
\end{itemize}
\begin{itemize}
\item {Proveniência:(De \textunderscore lycoperdo\textunderscore )}
\end{itemize}
Nome, criado por Merat, para designar um grupo de cogumelos.
\section{Lycopérsico}
\begin{itemize}
\item {Grp. gram.:m.}
\end{itemize}
\begin{itemize}
\item {Proveniência:(Do gr. \textunderscore lukos\textunderscore  + \textunderscore persikon\textunderscore )}
\end{itemize}
Gênero de árvores solanáceas.
\section{Lycopódeas}
\begin{itemize}
\item {Grp. gram.:f. pl.}
\end{itemize}
(V.lycopodiáceas)
\section{Lycopodiáceas}
\begin{itemize}
\item {Grp. gram.:f. pl.}
\end{itemize}
\begin{itemize}
\item {Proveniência:(De \textunderscore lycopodiáceo\textunderscore )}
\end{itemize}
Família de plantas, que tem por typo o licopódio.
\section{Lycopodiáceo}
\begin{itemize}
\item {Grp. gram.:adj.}
\end{itemize}
Relativo ou semelhante ao lycopódio.
\section{Lycopodina}
\begin{itemize}
\item {Grp. gram.:f.}
\end{itemize}
\begin{itemize}
\item {Utilização:Chím.}
\end{itemize}
Princípio azotado, que se encontra no lycopódio.
\section{Lycopodíneas}
\begin{itemize}
\item {Grp. gram.:f. pl.}
\end{itemize}
(V.lycopodiáceas)
\section{Lycopódio}
\begin{itemize}
\item {Grp. gram.:m.}
\end{itemize}
\begin{itemize}
\item {Proveniência:(Do gr. \textunderscore lukos\textunderscore  + \textunderscore pous\textunderscore , \textunderscore podos\textunderscore )}
\end{itemize}
Gênero de plantas cryptògâmicas e trepadeiras, de que há duas espécies na Europa.
\section{Lycópode}
\begin{itemize}
\item {Grp. gram.:m.}
\end{itemize}
O mesmo que \textunderscore lycopódio\textunderscore .
\section{Lycópside}
\begin{itemize}
\item {Grp. gram.:f.}
\end{itemize}
\begin{itemize}
\item {Proveniência:(Do gr. \textunderscore lukos\textunderscore  + \textunderscore opsis\textunderscore )}
\end{itemize}
Gênero de plantas borragíneas.
\section{Lycorexia}
\begin{itemize}
\item {Grp. gram.:f.}
\end{itemize}
\begin{itemize}
\item {Utilização:Med.}
\end{itemize}
\begin{itemize}
\item {Proveniência:(Do gr. \textunderscore lukos\textunderscore  + \textunderscore orexis\textunderscore )}
\end{itemize}
Variedade de bulímia.
\section{Lycose}
\begin{itemize}
\item {Grp. gram.:f.}
\end{itemize}
O mesmo que \textunderscore tarântula\textunderscore .
\section{Lýdio}
\begin{itemize}
\item {Grp. gram.:adj.}
\end{itemize}
\begin{itemize}
\item {Grp. gram.:M.}
\end{itemize}
\begin{itemize}
\item {Proveniência:(Do lat. \textunderscore lydius\textunderscore )}
\end{itemize}
Relativo á Lýdia.
Aquelle que é natural da Lýdia.
\section{Lymexylo}
\begin{itemize}
\item {fónica:csi}
\end{itemize}
\begin{itemize}
\item {Grp. gram.:m.}
\end{itemize}
\begin{itemize}
\item {Proveniência:(Do gr. \textunderscore lume\textunderscore  + \textunderscore xulon\textunderscore )}
\end{itemize}
Gênero de insectos coleópteros pentâmeros.
\section{Lympha}
\begin{itemize}
\item {Grp. gram.:f.}
\end{itemize}
\begin{itemize}
\item {Utilização:Poét.}
\end{itemize}
\begin{itemize}
\item {Proveniência:(Lat. \textunderscore lympha\textunderscore )}
\end{itemize}
Líquido branco e nutritivo, contido em certos vasos do organismo.
Humor aquoso das plantas.
Água.
\section{Lymphado}
\begin{itemize}
\item {Grp. gram.:adj.}
\end{itemize}
\begin{itemize}
\item {Utilização:Des.}
\end{itemize}
\begin{itemize}
\item {Proveniência:(De \textunderscore lympha\textunderscore )}
\end{itemize}
O mesmo que \textunderscore hydróphobo\textunderscore .
\section{Lymphadura}
\begin{itemize}
\item {Grp. gram.:f.}
\end{itemize}
Acto de lymphar.
\section{Lymphagogo}
\begin{itemize}
\item {Grp. gram.:m.}
\end{itemize}
Substância, que aumenta a producção da lympha.
\section{Lymphangioma}
\begin{itemize}
\item {Grp. gram.:m.}
\end{itemize}
Tumor de vasos lympháticos.
(Do lat, \textunderscore lympha\textunderscore  + gr. \textunderscore angeion\textunderscore )
\section{Lymphangite}
\begin{itemize}
\item {Grp. gram.:f.}
\end{itemize}
\begin{itemize}
\item {Proveniência:(Do lat. \textunderscore lympha\textunderscore  + gr. \textunderscore angeion\textunderscore )}
\end{itemize}
Inflammação dos vasos lympháticos.
\section{Lymphar}
\begin{itemize}
\item {Grp. gram.:v. t.}
\end{itemize}
\begin{itemize}
\item {Proveniência:(De \textunderscore lympha\textunderscore )}
\end{itemize}
Misturar com água, diluir.
\section{Lymphático}
\begin{itemize}
\item {Grp. gram.:adj.}
\end{itemize}
Relativo á lympha.
Que contém lympha.
Em que predomina a lympha: \textunderscore temperamento lymphático\textunderscore .
\section{Lymphatismo}
\begin{itemize}
\item {Grp. gram.:m.}
\end{itemize}
\begin{itemize}
\item {Utilização:Med.}
\end{itemize}
Estado lymphático do organismo.
\section{Lymphite}
\begin{itemize}
\item {Grp. gram.:f.}
\end{itemize}
O mesmo que \textunderscore lymphangite\textunderscore .
\section{Lymphocyto}
\begin{itemize}
\item {Grp. gram.:m.}
\end{itemize}
O mesmo que [[doença do somno|somno]].
\section{Lymphoma}
\begin{itemize}
\item {Grp. gram.:m.}
\end{itemize}
\begin{itemize}
\item {Proveniência:(Do lat. \textunderscore lympha\textunderscore )}
\end{itemize}
Tumor das glândulas lympháticas.
\section{Lymphorrhagia}
\begin{itemize}
\item {Grp. gram.:f.}
\end{itemize}
\begin{itemize}
\item {Utilização:Med.}
\end{itemize}
\begin{itemize}
\item {Proveniência:(Do lat. \textunderscore limpha\textunderscore  + gr. \textunderscore rhagein\textunderscore )}
\end{itemize}
Derramamento persistente de lympha, depois de ferido um vaso lymphático.
\section{Lymphose}
\begin{itemize}
\item {Grp. gram.:f.}
\end{itemize}
\begin{itemize}
\item {Utilização:Med.}
\end{itemize}
\begin{itemize}
\item {Proveniência:(De \textunderscore lympha\textunderscore )}
\end{itemize}
Acto de elaboração especial, de que resulta a lympha.
\section{Lymphotomia}
\begin{itemize}
\item {Grp. gram.:f.}
\end{itemize}
\begin{itemize}
\item {Proveniência:(Do lat. \textunderscore lympha\textunderscore  + \textunderscore tome\textunderscore )}
\end{itemize}
Dissecção dos vasos lympháticos.
\section{Lynce}
\begin{itemize}
\item {Grp. gram.:m.}
\end{itemize}
\begin{itemize}
\item {Proveniência:(Do gr. \textunderscore lunx\textunderscore )}
\end{itemize}
Quadrúpede carnívoro, também conhecido por \textunderscore lôbo cerval\textunderscore .
Constellação boreal.
\section{Lynchagem}
\begin{itemize}
\item {Grp. gram.:f.}
\end{itemize}
Processo ou acto de lynchar.
\section{Lynchamento}
\begin{itemize}
\item {Grp. gram.:m.}
\end{itemize}
Acto de lynchar.
\section{Lynchar}
\begin{itemize}
\item {Grp. gram.:v. t.}
\end{itemize}
\begin{itemize}
\item {Proveniência:(De \textunderscore Lynch\textunderscore , n. p.)}
\end{itemize}
Justiçar ou executar summariamente, segundo o processo instituido por Lynch, nos Estados Unidos.
\section{Lyonês}
\begin{itemize}
\item {Grp. gram.:adj.}
\end{itemize}
\begin{itemize}
\item {Grp. gram.:M.}
\end{itemize}
\begin{itemize}
\item {Proveniência:(De \textunderscore Lyon\textunderscore , n. p. fr.)}
\end{itemize}
Relativo a Lião.
Habitante de Lião.
\section{Lypemania}
\begin{itemize}
\item {Grp. gram.:f.}
\end{itemize}
\begin{itemize}
\item {Utilização:Med.}
\end{itemize}
\begin{itemize}
\item {Proveniência:(Do gr. \textunderscore lúpe\textunderscore  + \textunderscore mania\textunderscore )}
\end{itemize}
Espécie de alienação mental, caracterizada por tristeza profunda.
\section{Lyra}
\begin{itemize}
\item {Grp. gram.:f.}
\end{itemize}
\begin{itemize}
\item {Utilização:Fig.}
\end{itemize}
\begin{itemize}
\item {Utilização:Gír.}
\end{itemize}
\begin{itemize}
\item {Proveniência:(Gr. \textunderscore lura\textunderscore )}
\end{itemize}
Antigo instrumento músico, de cordas.
Talento poético.
Arte de versejar.
Ave gallinácea, (\textunderscore menura lyra\textunderscore ).
Superfície inferior da abóbada dos pilares do cérebro.
Constellação boreal.
Guitarra.
\section{Lyrado}
\begin{itemize}
\item {Grp. gram.:adj.}
\end{itemize}
\begin{itemize}
\item {Utilização:Bot.}
\end{itemize}
\begin{itemize}
\item {Proveniência:(De \textunderscore lyra\textunderscore )}
\end{itemize}
Diz-se das fôlhas de certas crucíferas, cujos lóbulos superiores são grandes e reunidos, em quanto os inferiores são pequenos e divididos até á nervura média.
\section{Lyral}
\begin{itemize}
\item {Grp. gram.:adj.}
\end{itemize}
Dizia-se \textunderscore cravo lyral\textunderscore  um instrumento músico, o mesmo que \textunderscore espineta\textunderscore .
\section{Lýrica}
\begin{itemize}
\item {Grp. gram.:f.}
\end{itemize}
\begin{itemize}
\item {Proveniência:(De \textunderscore lýrico\textunderscore )}
\end{itemize}
O gênero lýrico da poesia.
Collecção de poesias lýricas.
\section{Lýrico}
\begin{itemize}
\item {Grp. gram.:adj.}
\end{itemize}
\begin{itemize}
\item {Grp. gram.:M.}
\end{itemize}
\begin{itemize}
\item {Proveniência:(De \textunderscore lyra\textunderscore )}
\end{itemize}
Relativo á poesia que exprime os grandes e os delicados sentimentos pessoaes do poéta.
Relativo á poesia.
Sentimental.
Relativo a óperas: \textunderscore theatro lýrico\textunderscore .
Poéta, que cultiva o gênero lýrico.
\section{Lyriforme}
\begin{itemize}
\item {Grp. gram.:adj.}
\end{itemize}
\begin{itemize}
\item {Proveniência:(De \textunderscore lyra\textunderscore  + \textunderscore forma\textunderscore )}
\end{itemize}
Que tem fórma de lyra.
\section{Lyrismo}
\begin{itemize}
\item {Grp. gram.:m.}
\end{itemize}
\begin{itemize}
\item {Proveniência:(De \textunderscore lyra\textunderscore )}
\end{itemize}
Qualidade de lýrico, sublime ou sentimental.
Subjectivismo poético.
Enthusiasmo.
\section{Lyrista}
\begin{itemize}
\item {Grp. gram.:m.}
\end{itemize}
\begin{itemize}
\item {Utilização:Deprec.}
\end{itemize}
\begin{itemize}
\item {Proveniência:(De \textunderscore lyra\textunderscore )}
\end{itemize}
Tocador de lyra.
Poéta frívolo, banal.
\section{Lyrístria}
\begin{itemize}
\item {Grp. gram.:f.}
\end{itemize}
\begin{itemize}
\item {Proveniência:(Lat. \textunderscore lyristria\textunderscore )}
\end{itemize}
A mulher que tocava lyra.
\section{Lyrodo}
\begin{itemize}
\item {Grp. gram.:m.}
\end{itemize}
\begin{itemize}
\item {Proveniência:(Lat. \textunderscore lyrodus\textunderscore )}
\end{itemize}
Cantador que, entre os antigos, acompanhava o seu canto com o som da lyra.
O mesmo que \textunderscore citharedo\textunderscore . Cf. Castilho, \textunderscore Fastos\textunderscore , III, 207.
\section{Lyse}
\begin{itemize}
\item {Grp. gram.:f.}
\end{itemize}
\begin{itemize}
\item {Utilização:Med.}
\end{itemize}
\begin{itemize}
\item {Proveniência:(Lat. \textunderscore lysis\textunderscore )}
\end{itemize}
Defervescência lenta e gradual da febre.
Terminação lenta de uma doença.
\section{Lysidina}
\begin{itemize}
\item {Grp. gram.:f.}
\end{itemize}
Composto chímico, contra a doença da gota.
\section{Lysimáchia}
\begin{itemize}
\item {fónica:qui}
\end{itemize}
\begin{itemize}
\item {Grp. gram.:f.}
\end{itemize}
\begin{itemize}
\item {Proveniência:(De \textunderscore Lysimacho\textunderscore , n. p.)}
\end{itemize}
Gênero de plantas primuláceas.
\section{Lysio}
\begin{itemize}
\item {Grp. gram.:adj.}
\end{itemize}
\begin{itemize}
\item {Proveniência:(Do gr. \textunderscore lusis\textunderscore )}
\end{itemize}
Resultante de uma dissolução chímica.
\section{Lýsis}
\begin{itemize}
\item {Grp. gram.:f.}
\end{itemize}
(V.lyse)
\section{Lysofórmio}
\begin{itemize}
\item {Grp. gram.:m.}
\end{itemize}
Preparação antiséptica, em que entra o lysol e a formalina?
\section{Lysol}
\begin{itemize}
\item {Grp. gram.:m.}
\end{itemize}
Composto chímico, utilizado em algumas indústrias.
\section{Lysolagem}
\begin{itemize}
\item {Grp. gram.:f.}
\end{itemize}
Emprêgo do lysol.
\section{Lyssophobia}
\begin{itemize}
\item {Grp. gram.:f.}
\end{itemize}
Temor mórbido da hydrophobia.
\section{Lythrariadas}
\begin{itemize}
\item {Grp. gram.:f. pl.}
\end{itemize}
\begin{itemize}
\item {Proveniência:(Do lat. \textunderscore lythrum\textunderscore )}
\end{itemize}
Família de plantas dicotyledóneas, a que pertence a salgueirinha, (\textunderscore lythrum\textunderscore ), cujas flôres tem a côr do sangue coalhado.
\section{Lythro}
\begin{itemize}
\item {Grp. gram.:m.}
\end{itemize}
\begin{itemize}
\item {Proveniência:(Gr. \textunderscore luthron\textunderscore )}
\end{itemize}
\end{document}