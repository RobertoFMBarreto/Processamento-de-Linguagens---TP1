\documentclass{article}
\usepackage[portuguese]{babel}
\title{K}
\begin{document}
Acto ou effeito de juxtapor.
\section{K}
\begin{itemize}
\item {fónica:cá,capa}
\end{itemize}
\begin{itemize}
\item {Grp. gram.:m.}
\end{itemize}
\begin{itemize}
\item {Grp. gram.:Adj.}
\end{itemize}
Esta letra tem sido considerada a undécima do nosso alphabeto, mas, de facto, é estranha ao abecedário português, e só deverá usar-se em certos vocábulos não aportuguesados, e em alguns vocábulos derivados de nomes próprios estrangeiros, em que ella entre, sendo todavia tolerável em \textunderscore kilo\textunderscore  e seus compostos, onde representa uma espécie de convenção internacional, embora destituída de base scientífica.
Que numa série occupa o undécimo lugar.--Os Latinos só empregavam o \textunderscore k\textunderscore  como letra numeral, que significava 250.
\section{Kabbfússia}
\begin{itemize}
\item {Grp. gram.:f.}
\end{itemize}
\begin{itemize}
\item {Proveniência:(De \textunderscore Kabbfuss\textunderscore , n. p.)}
\end{itemize}
Gênero de plantas compostas.
\section{Kantismo}
\begin{itemize}
\item {Grp. gram.:m.}
\end{itemize}
Systema philosóphico de Kant.
\section{Kantista}
\begin{itemize}
\item {Grp. gram.:m.}
\end{itemize}
Sectário da philosophia de Kant.
\section{Kennédya}
\begin{itemize}
\item {Grp. gram.:f.}
\end{itemize}
\begin{itemize}
\item {Proveniência:(De \textunderscore Kennedy\textunderscore , n. p.)}
\end{itemize}
Gênero de plantas leguminosas.
\section{Képi}
\begin{itemize}
\item {Grp. gram.:m.}
\end{itemize}
(V.quépi)
\section{Képler}
\begin{itemize}
\item {Grp. gram.:m.}
\end{itemize}
\begin{itemize}
\item {Proveniência:(De \textunderscore Kepler\textunderscore , n. p.)}
\end{itemize}
A quarta mancha da lua.
\section{Kepléria}
\begin{itemize}
\item {Grp. gram.:f.}
\end{itemize}
\begin{itemize}
\item {Proveniência:(De \textunderscore Kepler\textunderscore , n. p.)}
\end{itemize}
Gênero de palmeiras indianas.
\section{Kepleriano}
\begin{itemize}
\item {Grp. gram.:adj.}
\end{itemize}
Relativo a Kepler.
Diz-se especialmente das três leis de Kepler, que precederam o descobrimento da gravitação.
\section{Kera...}
\begin{itemize}
\item {Grp. gram.:pref.}
\end{itemize}
\begin{itemize}
\item {Proveniência:(Gr. \textunderscore keras\textunderscore )}
\end{itemize}
(us. em algumas palavras scientíficas, mas que deve substituir-se por \textunderscore cera...\textunderscore )
\section{Kermesse}
\begin{itemize}
\item {Grp. gram.:f.}
\end{itemize}
\begin{itemize}
\item {Proveniência:(Fr. \textunderscore kermesse\textunderscore )}
\end{itemize}
Estrangeirismo inútil, por \textunderscore feira com arraial\textunderscore , \textunderscore bazar\textunderscore .
\section{Kérria}
\begin{itemize}
\item {Grp. gram.:f.}
\end{itemize}
\begin{itemize}
\item {Proveniência:(De \textunderscore Kerr\textunderscore , n. p.)}
\end{itemize}
Gênero de plantas da tríbo das espíreas, originária do Japão, (\textunderscore herria japonica\textunderscore ).
\section{Khediva}
\begin{itemize}
\item {Grp. gram.:m.}
\end{itemize}
(V.quedive)
\section{Killíngia}
\begin{itemize}
\item {Grp. gram.:f.}
\end{itemize}
\begin{itemize}
\item {Proveniência:(De \textunderscore Killing\textunderscore , n. p.)}
\end{itemize}
Gênero de plantas, cyperáceas.
\section{Kilo}
\begin{itemize}
\item {Grp. gram.:m.}
\end{itemize}
(e seus compostos)
(V. \textunderscore quilo\textunderscore ^1, etc.)
\section{Kirganélia}
\begin{itemize}
\item {Grp. gram.:f.}
\end{itemize}
\begin{itemize}
\item {Proveniência:(De \textunderscore Kirganel\textunderscore , n. p.)}
\end{itemize}
Árvore euphorbiácea da Índia.
\section{Kleínia}
\begin{itemize}
\item {Grp. gram.:f.}
\end{itemize}
\begin{itemize}
\item {Proveniência:(De \textunderscore Klein\textunderscore , n. p.)}
\end{itemize}
Gênero de plantas compostas.
\section{Kneipista}
\begin{itemize}
\item {Grp. gram.:m.}
\end{itemize}
Sectário do systema therapêutico de Kneip.
\section{Knóxia}
\begin{itemize}
\item {Grp. gram.:f.}
\end{itemize}
\begin{itemize}
\item {Proveniência:(De \textunderscore Knox\textunderscore , n. p.)}
\end{itemize}
Gênero de plantas rubiáceas da Índia.
\section{Kóchia}
\begin{itemize}
\item {fónica:qui}
\end{itemize}
\begin{itemize}
\item {Grp. gram.:f.}
\end{itemize}
\begin{itemize}
\item {Proveniência:(De \textunderscore Koch\textunderscore , n. p.)}
\end{itemize}
Gênero de plantas chenopódeas.
\section{Koeléria}
\begin{itemize}
\item {Grp. gram.:f.}
\end{itemize}
\begin{itemize}
\item {Proveniência:(De \textunderscore Koeler\textunderscore , n. p.)}
\end{itemize}
Gênero de plantas gramíneas.
\section{Koelreutéria}
\begin{itemize}
\item {Grp. gram.:f.}
\end{itemize}
\begin{itemize}
\item {Proveniência:(De \textunderscore Koelreuter\textunderscore , n. p.)}
\end{itemize}
Gênero de plantas sapindáceas.
\section{Kólbia}
\begin{itemize}
\item {Grp. gram.:f.}
\end{itemize}
\begin{itemize}
\item {Proveniência:(De \textunderscore Kolb\textunderscore , n. p.)}
\end{itemize}
Planta trepadeira da África tropical.
\section{Kraméria}
\begin{itemize}
\item {Grp. gram.:f.}
\end{itemize}
\begin{itemize}
\item {Proveniência:(De \textunderscore Kramer\textunderscore , n. p.)}
\end{itemize}
Gênero de arbustos americanos.
\section{Krausista}
\begin{itemize}
\item {Grp. gram.:m.}
\end{itemize}
Partidário do systema philosóphico de Krause.
\section{Kúmmel}
\begin{itemize}
\item {Grp. gram.:m.}
\end{itemize}
Licôr doce e alcoólico, cujo gôsto particular é devido a uma infusão de sementes de cominho.
\section{Kúnthia}
\begin{itemize}
\item {Grp. gram.:f.}
\end{itemize}
\begin{itemize}
\item {Proveniência:(De \textunderscore Kunth\textunderscore , n. p.)}
\end{itemize}
Gênero de palmeiras dos Andes.
\section{Kúnzea}
\begin{itemize}
\item {Grp. gram.:f.}
\end{itemize}
\begin{itemize}
\item {Proveniência:(De \textunderscore Kunz\textunderscore , n. p.)}
\end{itemize}
\end{document}