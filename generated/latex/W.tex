\documentclass{article}
\usepackage[portuguese]{babel}
\title{W}
\begin{document}
Que tem vurmo.
\section{W}
\begin{itemize}
\item {Utilização:Chím.}
\end{itemize}
(o \textunderscore vê\textunderscore  dobrado)
É letra estranha ao alphabeto português; mas emprega-se geralmente em algumas palavras, derivadas de certos nomes próprios estrangeiros, especialmente ingleses e alemães.
Em palavras derivadas do alemão, sôa como \textunderscore v\textunderscore ; e, nas derivadas do inglês, como \textunderscore u\textunderscore . Em português, poderia portanto substituir-se respectivamente por essas duas letras.
Abrev. de \textunderscore tungstênio\textunderscore .
\section{Wagneriano}
\begin{itemize}
\item {fónica:va}
\end{itemize}
\begin{itemize}
\item {Grp. gram.:adj.}
\end{itemize}
Relativo a Wagner, compositor musical: \textunderscore escola wagneriana\textunderscore .
\section{Wagnerismo}
\begin{itemize}
\item {fónica:va}
\end{itemize}
\begin{itemize}
\item {Grp. gram.:m.}
\end{itemize}
Systema musical de Wagner.
Processo ou systema de Wagner, em Música.
\section{Wagnerita}
\begin{itemize}
\item {fónica:vagne}
\end{itemize}
\begin{itemize}
\item {Grp. gram.:f.}
\end{itemize}
\begin{itemize}
\item {Proveniência:(Do al. \textunderscore Wagner\textunderscore , n. p.)}
\end{itemize}
Phosphato de magnésia.
\section{Wagnerite}
\begin{itemize}
\item {fónica:vagne}
\end{itemize}
\begin{itemize}
\item {Grp. gram.:f.}
\end{itemize}
\begin{itemize}
\item {Proveniência:(Do al. \textunderscore Wagner\textunderscore , n. p.)}
\end{itemize}
Phosphato de magnésia.
\section{Wagnerito}
\begin{itemize}
\item {fónica:vagne}
\end{itemize}
\begin{itemize}
\item {Grp. gram.:f.}
\end{itemize}
\begin{itemize}
\item {Proveniência:(Do al. \textunderscore Wagner\textunderscore , n. p.)}
\end{itemize}
Phosphato de magnésia.
\section{Wahlembérgia}
\begin{itemize}
\item {fónica:valem}
\end{itemize}
\begin{itemize}
\item {Grp. gram.:f.}
\end{itemize}
\begin{itemize}
\item {Proveniência:(Do al. \textunderscore Wahlemberg\textunderscore , n. p.)}
\end{itemize}
Gênero de plantas campanuláceas.
\section{Waldstêinia}
\begin{itemize}
\item {fónica:val}
\end{itemize}
\begin{itemize}
\item {Grp. gram.:f.}
\end{itemize}
\begin{itemize}
\item {Proveniência:(Do al. \textunderscore Waldstein\textunderscore , n. p.)}
\end{itemize}
Gênero de plantas rosáceas.
\section{Walkéria}
\begin{itemize}
\item {fónica:uòlké}
\end{itemize}
\begin{itemize}
\item {Grp. gram.:f.}
\end{itemize}
\begin{itemize}
\item {Proveniência:(Do ing. \textunderscore Walker\textunderscore , n. p.)}
\end{itemize}
Gênero de musgos.
\section{Walthéria}
\begin{itemize}
\item {fónica:val}
\end{itemize}
\begin{itemize}
\item {Grp. gram.:f.}
\end{itemize}
\begin{itemize}
\item {Proveniência:(Do al. \textunderscore Walther\textunderscore , n. p.)}
\end{itemize}
Gênero de plantas buttneriáceas.
\section{Warwicita}
\begin{itemize}
\item {fónica:uaru}
\end{itemize}
\begin{itemize}
\item {Grp. gram.:f.}
\end{itemize}
\begin{itemize}
\item {Utilização:Miner.}
\end{itemize}
\begin{itemize}
\item {Proveniência:(Do ingl. \textunderscore Warwick\textunderscore , n. p.)}
\end{itemize}
Variedade de óxydo de manganés.
\section{Warwicite}
\begin{itemize}
\item {fónica:uaru}
\end{itemize}
\begin{itemize}
\item {Grp. gram.:f.}
\end{itemize}
\begin{itemize}
\item {Utilização:Miner.}
\end{itemize}
\begin{itemize}
\item {Proveniência:(Do ingl. \textunderscore Warwick\textunderscore , n. p.)}
\end{itemize}
Variedade de óxydo de manganés.
\section{Warwicito}
\begin{itemize}
\item {fónica:uaru}
\end{itemize}
\begin{itemize}
\item {Grp. gram.:f.}
\end{itemize}
\begin{itemize}
\item {Utilização:Miner.}
\end{itemize}
\begin{itemize}
\item {Proveniência:(Do ingl. \textunderscore Warwick\textunderscore , n. p.)}
\end{itemize}
Variedade de óxydo de manganés.
\section{Watsónia}
\begin{itemize}
\item {fónica:uòt}
\end{itemize}
\begin{itemize}
\item {Grp. gram.:f.}
\end{itemize}
\begin{itemize}
\item {Proveniência:(Do ingl. \textunderscore Watson\textunderscore , n. p.)}
\end{itemize}
Gênero de plantas irídeas.
\section{Wébbia}
\begin{itemize}
\item {fónica:vé}
\end{itemize}
\begin{itemize}
\item {Grp. gram.:f.}
\end{itemize}
\begin{itemize}
\item {Proveniência:(Do al. \textunderscore Webb\textunderscore , n. p.)}
\end{itemize}
Gênero de plantas synanthéreas.
\section{Wellingtónia}
\begin{itemize}
\item {fónica:uèling}
\end{itemize}
\begin{itemize}
\item {Grp. gram.:f.}
\end{itemize}
\begin{itemize}
\item {Proveniência:(Do ingl. \textunderscore Wellington\textunderscore , n. p.)}
\end{itemize}
Corpulentíssima árvore conífera da Califórnia, onde attinge mais de cem metros de altura e déz de diâmetro.
(Pertence ao gênero \textunderscore sequoia gigantea\textunderscore , Torr.)
\section{Wernerita}
\begin{itemize}
\item {fónica:ver}
\end{itemize}
\begin{itemize}
\item {Grp. gram.:f.}
\end{itemize}
\begin{itemize}
\item {Utilização:Miner.}
\end{itemize}
\begin{itemize}
\item {Proveniência:(Do al. \textunderscore Werner\textunderscore , n. p.)}
\end{itemize}
Nome de vários silicatos de alumina e cal.
\section{Wernerite}
\begin{itemize}
\item {fónica:ver}
\end{itemize}
\begin{itemize}
\item {Grp. gram.:f.}
\end{itemize}
\begin{itemize}
\item {Utilização:Miner.}
\end{itemize}
\begin{itemize}
\item {Proveniência:(Do al. \textunderscore Werner\textunderscore , n. p.)}
\end{itemize}
Nome de vários silicatos de alumina e cal.
\section{Wernerito}
\begin{itemize}
\item {fónica:ver}
\end{itemize}
\begin{itemize}
\item {Grp. gram.:f.}
\end{itemize}
\begin{itemize}
\item {Utilização:Miner.}
\end{itemize}
\begin{itemize}
\item {Proveniência:(Do al. \textunderscore Werner\textunderscore , n. p.)}
\end{itemize}
Nome de vários silicatos de alumina e cal.
\section{Wesleyano}
\begin{itemize}
\item {fónica:uès}
\end{itemize}
\begin{itemize}
\item {Grp. gram.:m.}
\end{itemize}
\begin{itemize}
\item {Proveniência:(Do ingl. \textunderscore Wesley\textunderscore , n. p.)}
\end{itemize}
Sectário do methodismo.
Methodista.
\section{Whígtia}
\begin{itemize}
\item {fónica:uíg}
\end{itemize}
\begin{itemize}
\item {Grp. gram.:f.}
\end{itemize}
\begin{itemize}
\item {Proveniência:(Do ingl. \textunderscore Wight\textunderscore , n. p.)}
\end{itemize}
Gênero de plantas escrofularíneas.
\section{Whist}
\begin{itemize}
\item {fónica:uíst}
\end{itemize}
\begin{itemize}
\item {Grp. gram.:m.}
\end{itemize}
\begin{itemize}
\item {Proveniência:(T. ingl.)}
\end{itemize}
Espécie de jôgo de cartas, análogo ao da bisca.
\section{Wiclefismo}
\begin{itemize}
\item {fónica:uí}
\end{itemize}
\begin{itemize}
\item {Grp. gram.:m.}
\end{itemize}
Doutrina religiosa do heresiarcha Wiclef, segundo o qual a Igreja Romana não é superior ás demais, o clero não deve têr bens temporaes e o desregramento dos padres faz-lhes perder todos os poderes espirituaes.
\section{Wiclefista}
\begin{itemize}
\item {fónica:uik}
\end{itemize}
\begin{itemize}
\item {Grp. gram.:m.}
\end{itemize}
Sectário do wiclefismo.
\section{Widdringtónia}
\begin{itemize}
\item {fónica:uid}
\end{itemize}
\begin{itemize}
\item {Grp. gram.:f.}
\end{itemize}
\begin{itemize}
\item {Proveniência:(Do ingl. \textunderscore Widrington\textunderscore , n. p.)}
\end{itemize}
Gênero de plantas coníferas.
\section{Wiedmânnia}
\begin{itemize}
\item {fónica:vid}
\end{itemize}
\begin{itemize}
\item {Grp. gram.:m.}
\end{itemize}
\begin{itemize}
\item {Proveniência:(Do al. \textunderscore Wiedemann\textunderscore , n. p.)}
\end{itemize}
Gênero de plantas labiadas.
\section{Wigândia}
\begin{itemize}
\item {fónica:uig}
\end{itemize}
\begin{itemize}
\item {Grp. gram.:f.}
\end{itemize}
\begin{itemize}
\item {Proveniência:(Do ing. \textunderscore Wigand\textunderscore , n. p.)}
\end{itemize}
Gênero de plantas hydroleáceas.
\section{Wilbrândia}
\begin{itemize}
\item {fónica:vilb}
\end{itemize}
\begin{itemize}
\item {Grp. gram.:f.}
\end{itemize}
\begin{itemize}
\item {Proveniência:(Do al. \textunderscore Wilbrand\textunderscore , n. p.)}
\end{itemize}
Gênero de plantas cucurbitáceas.
\section{Wilsónia}
\begin{itemize}
\item {fónica:uil}
\end{itemize}
\begin{itemize}
\item {Grp. gram.:f.}
\end{itemize}
\begin{itemize}
\item {Proveniência:(Do ingl. \textunderscore Wilson\textunderscore , n. p.)}
\end{itemize}
Gênero de plantas convolvuláceas.
\section{Wistéria}
\begin{itemize}
\item {fónica:vis}
\end{itemize}
\begin{itemize}
\item {Grp. gram.:f.}
\end{itemize}
\begin{itemize}
\item {Proveniência:(Do al. \textunderscore Wister\textunderscore , n. p.)}
\end{itemize}
Gênero de plantas leguminosas.
\section{Woodwárdia}
\begin{itemize}
\item {fónica:udu}
\end{itemize}
\begin{itemize}
\item {Grp. gram.:f.}
\end{itemize}
\begin{itemize}
\item {Proveniência:(Do ingl. \textunderscore Woodward\textunderscore , n. p.)}
\end{itemize}
Gênero de plantas polypódeas.
\section{Wríghtia}
\begin{itemize}
\item {fónica:ráitia}
\end{itemize}
\begin{itemize}
\item {Grp. gram.:f.}
\end{itemize}
\begin{itemize}
\item {Proveniência:(Do ingl. \textunderscore Wright\textunderscore , n. p.)}
\end{itemize}
\end{document}