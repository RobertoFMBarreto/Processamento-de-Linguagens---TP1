
\begin{itemize}
\item {Proveniência: }
\end{itemize}\documentclass{article}
\usepackage[portuguese]{babel}
\title{Z}
\begin{document}
Tantalato de ýttrio.
\section{Z}
\begin{itemize}
\item {fónica:zê}
\end{itemize}
\begin{itemize}
\item {Grp. gram.:m.}
\end{itemize}
Vigésima quinta e última letra do alphabeto português.
Uma das incógnitas, (em problemas de Mathemática).
Como letra numeral, valeu 2:000.
\section{Zaadona}
\begin{itemize}
\item {Grp. gram.:f.}
\end{itemize}
\begin{itemize}
\item {Utilização:Ant.}
\end{itemize}
O mesmo que \textunderscore zadona\textunderscore .
Zabaneira.
\section{Zabaneira}
\begin{itemize}
\item {Grp. gram.:f.}
\end{itemize}
\begin{itemize}
\item {Utilização:Des.}
\end{itemize}
Mulhér desavergonhada.
\section{Zabelé}
\begin{itemize}
\item {Grp. gram.:m.}
\end{itemize}
\begin{itemize}
\item {Utilização:Bras. do N}
\end{itemize}
Nambu, de corpo e pés vermelhos, e de canto melodioso.
\section{Zaborreira}
\begin{itemize}
\item {Grp. gram.:f.}
\end{itemize}
\begin{itemize}
\item {Utilização:Prov.}
\end{itemize}
\begin{itemize}
\item {Utilização:dur.}
\end{itemize}
Lugar, que ficou húmido ou lamacento nas casas de lagar, depois da fabricação do vinho ou do azeite.
Ladeira, formada por enchentes á beira do rio.
(Há quem derive o termo de \textunderscore zaburro\textunderscore ; mas parece mais provável que venha de \textunderscore bôrra\textunderscore  e de um pref. indefinido)
\section{Zabra}
\begin{itemize}
\item {Grp. gram.:f.}
\end{itemize}
Pequena embarcação, espécie de bote, na África Oriental. Cf. \textunderscore Roteiro de Vasco da Gama\textunderscore ; Goes, \textunderscore Chrón. de D. Man.\textunderscore , II, c. XVIII.
(Do ár.)
\section{Zabumba}
\begin{itemize}
\item {Grp. gram.:m.}
\end{itemize}
\begin{itemize}
\item {Utilização:Pop.}
\end{itemize}
\begin{itemize}
\item {Utilização:Burl.}
\end{itemize}
\begin{itemize}
\item {Proveniência:(T. onom.?)}
\end{itemize}
Tambor grande.
Bombo.
Grande chapéu alto.
\section{Zabumbar}
\begin{itemize}
\item {Grp. gram.:v. t.}
\end{itemize}
\begin{itemize}
\item {Grp. gram.:V. i.}
\end{itemize}
\begin{itemize}
\item {Utilização:Pop.}
\end{itemize}
\begin{itemize}
\item {Proveniência:(De \textunderscore zabumba\textunderscore )}
\end{itemize}
Atordoar. Cf. Camillo, \textunderscore Cancion. Al.\textunderscore , 475.
O mesmo que \textunderscore bater\textunderscore .
\section{Zabumbeiro}
\begin{itemize}
\item {Grp. gram.:m.}
\end{itemize}
Tocador de zabumba.
\section{Zaburreiro}
\begin{itemize}
\item {Grp. gram.:m.}
\end{itemize}
\begin{itemize}
\item {Utilização:T. da Bairrada}
\end{itemize}
Pé de milho zaburro.
\section{Zaburro}
\begin{itemize}
\item {Grp. gram.:adj.}
\end{itemize}
\begin{itemize}
\item {Utilização:Prov.}
\end{itemize}
\begin{itemize}
\item {Utilização:beir.}
\end{itemize}
\begin{itemize}
\item {Utilização:Prov.}
\end{itemize}
\begin{itemize}
\item {Utilização:beir.}
\end{itemize}
Diz-se de uma variedade de milho indiano.
Diz-se de uma espécie de milho avermelhado-escuro, cultivado em alguns pontos de Portugal, e cuja espiga não é compacta, mas formada de várias hastes irregulares, que partem da extremidade superior da cana.
Impropriamente, diz-se também, no Minho e em Trás-os-Montes, de uma variedade de milho, mais conhecido por milho das vassoiras.
Milho grosso, que se semeia basto e se corta verde, para alimento de animaes.
\textunderscore Zaburro vermelho\textunderscore . Diz-se de uma variedade de milho, espécie de sorgo, (\textunderscore andropogon sorghum\textunderscore , Brotero), talvez o mesmo que o mencionado acima, em segundo lugar.
\section{Zaca}
\begin{itemize}
\item {Grp. gram.:m.}
\end{itemize}
O mesmo que \textunderscore zaco\textunderscore .
\section{Zacintos}
\begin{itemize}
\item {Grp. gram.:m.}
\end{itemize}
\begin{itemize}
\item {Proveniência:(De \textunderscore Zacyntho\textunderscore , n, p.)}
\end{itemize}
Gênero de plantas sinantéreas.
\section{Zaco}
\begin{itemize}
\item {Grp. gram.:m.}
\end{itemize}
Supremo sacerdote, entre os Bonzos.
\section{Zacum}
\begin{itemize}
\item {Grp. gram.:m.}
\end{itemize}
Planta espinhosa da Arábia.
Fruto amargo dessa planta.
(Do ár.)
\section{Zacynthos}
\begin{itemize}
\item {Grp. gram.:m.}
\end{itemize}
\begin{itemize}
\item {Proveniência:(De \textunderscore Zacyntho\textunderscore , n. p.)}
\end{itemize}
Gênero de plantas synanthéreas.
\section{Zadona}
\begin{itemize}
\item {Grp. gram.:f.}
\end{itemize}
\begin{itemize}
\item {Utilização:Ant.}
\end{itemize}
Mulhér livre ou fôrra.
\section{Zafira}
\begin{itemize}
\item {Grp. gram.:f.}
\end{itemize}
\begin{itemize}
\item {Utilização:Ant.}
\end{itemize}
O mesmo que \textunderscore saphira\textunderscore .
\section{Zaga}
\begin{itemize}
\item {Grp. gram.:f.}
\end{itemize}
Árvore, de que se fazem azagaias; espécie de palmeira.
\section{Zagaia}
\begin{itemize}
\item {Proveniência:(De \textunderscore zaga\textunderscore )}
\end{itemize}
\textunderscore f.\textunderscore  (e der.)
O mesmo que \textunderscore azagaia\textunderscore , etc.
\section{Zagaiar}
\begin{itemize}
\item {Grp. gram.:v. t.}
\end{itemize}
O mesmo que \textunderscore azagaiar\textunderscore . Cf. Filinto, \textunderscore D. Man.\textunderscore , III, 259.
\section{Zagal}
\begin{itemize}
\item {Grp. gram.:m.}
\end{itemize}
\begin{itemize}
\item {Utilização:Prov.}
\end{itemize}
\begin{itemize}
\item {Proveniência:(Do ár. \textunderscore zagal\textunderscore )}
\end{itemize}
Pastor.
Ajudante do maioral de gado.
Mancebo forte, vigoroso.
\section{Zagala}
\begin{itemize}
\item {Grp. gram.:f.}
\end{itemize}
(Fem. de \textunderscore zagal\textunderscore )
\section{Zagalejo}
\begin{itemize}
\item {Grp. gram.:m.}
\end{itemize}
Pequeno zagal.
\section{Zagaleto}
\begin{itemize}
\item {fónica:lê}
\end{itemize}
\begin{itemize}
\item {Grp. gram.:m.}
\end{itemize}
Pequeno zagal.
\section{Zagalote}
\begin{itemize}
\item {Grp. gram.:m.}
\end{itemize}
Pequena bala para espingarda.
\section{Zagão}
\begin{itemize}
\item {Grp. gram.:m.}
\end{itemize}
\begin{itemize}
\item {Utilização:Prov.}
\end{itemize}
\begin{itemize}
\item {Utilização:beir.}
\end{itemize}
Espaço maiór ou menór, dentro de casa, ao fundo da escada principal.
(Cast. \textunderscore zaguán\textunderscore )
\section{Zagari}
\begin{itemize}
\item {Grp. gram.:m.}
\end{itemize}
\begin{itemize}
\item {Utilização:Ant.}
\end{itemize}
Espécie de pano de linho.
\section{Zagatai}
\begin{itemize}
\item {Grp. gram.:m.}
\end{itemize}
O mesmo que \textunderscore turcomano\textunderscore , língua.
\section{Zagaté}
\begin{itemize}
\item {Grp. gram.:m.}
\end{itemize}
\begin{itemize}
\item {Utilização:Pop.}
\end{itemize}
O mesmo que \textunderscore tagaté\textunderscore .
\section{Zagonal}
\begin{itemize}
\item {Grp. gram.:m.}
\end{itemize}
\begin{itemize}
\item {Utilização:Ant.}
\end{itemize}
O mesmo que \textunderscore diácono\textunderscore . Cf. S. R. Viterbo, \textunderscore Elucidário\textunderscore .
\section{Zagorrino}
\begin{itemize}
\item {Grp. gram.:m.}
\end{itemize}
O mesmo que \textunderscore zagorro\textunderscore .
\section{Zagorro}
\begin{itemize}
\item {fónica:gô}
\end{itemize}
\begin{itemize}
\item {Grp. gram.:m. adj.}
\end{itemize}
\begin{itemize}
\item {Utilização:Prov.}
\end{itemize}
\begin{itemize}
\item {Utilização:alent.}
\end{itemize}
Velhaco.
Estúrdio.
\section{Zagrão}
\begin{itemize}
\item {Grp. gram.:m.}
\end{itemize}
\begin{itemize}
\item {Utilização:Gír.}
\end{itemize}
Vinho.
\section{Zagre}
\begin{itemize}
\item {Grp. gram.:f.}
\end{itemize}
O mesmo que \textunderscore uzagre\textunderscore .
\section{Zagré}
\begin{itemize}
\item {Grp. gram.:m.}
\end{itemize}
O mesmo que \textunderscore zagrão\textunderscore .
\section{Zagu}
\begin{itemize}
\item {Grp. gram.:m.}
\end{itemize}
Árvore indiana.
\section{Zagucho}
\begin{itemize}
\item {Grp. gram.:adj.}
\end{itemize}
\begin{itemize}
\item {Utilização:Prov.}
\end{itemize}
\begin{itemize}
\item {Utilização:trasm.}
\end{itemize}
Muito vivo; muito esperto.
\section{Zagunchada}
\begin{itemize}
\item {Grp. gram.:f.}
\end{itemize}
\begin{itemize}
\item {Utilização:Fam.}
\end{itemize}
Golpe de zaguncho.
Motejo, remoque; censura.
\section{Zagunchar}
\begin{itemize}
\item {Grp. gram.:v. t.}
\end{itemize}
\begin{itemize}
\item {Utilização:Fam.}
\end{itemize}
Ferir com zaguncho.
Molestar; censurar.
Dirigir remoques a.
\section{Zaguncho}
\begin{itemize}
\item {Grp. gram.:m.}
\end{itemize}
Espécie de azagaia. Cf. \textunderscore Peregrinação\textunderscore , XXXVI.
\section{Zaida}
\begin{itemize}
\item {Grp. gram.:f.}
\end{itemize}
\begin{itemize}
\item {Proveniência:(De \textunderscore Zaida\textunderscore , n. p.)}
\end{itemize}
Insecto díptero, espécie de mosca.
\section{Zâimbo}
\begin{itemize}
\item {Grp. gram.:adj.}
\end{itemize}
Tôrto, zambro.
Que tem os olhos tortos. Cf. F. Manuel, \textunderscore Feira de Anex.\textunderscore 
(Cp. \textunderscore zambro\textunderscore )
\section{Zaimo}
\begin{itemize}
\item {Grp. gram.:m.}
\end{itemize}
Cavalleiro da milícia turca.
\section{Zaino}
\begin{itemize}
\item {Grp. gram.:adj.}
\end{itemize}
\begin{itemize}
\item {Utilização:Fig.}
\end{itemize}
\begin{itemize}
\item {Proveniência:(It. \textunderscore zaino\textunderscore )}
\end{itemize}
Diz-se do cavallo, cujo pêlo é todo castanho-escuro, sem mescla.
Que não tem malhas brancas, (falando-se do cavallo).
Que tem o pêlo negro, com pouco brilho.
Disfarçado; velhaco.
\section{Zaino}
\begin{itemize}
\item {Grp. gram.:m.}
\end{itemize}
\begin{itemize}
\item {Utilização:Prov.}
\end{itemize}
\begin{itemize}
\item {Utilização:Ant.}
\end{itemize}
Homem amancebado, libertino. Cf. S. R. Viterbo, \textunderscore Elucidário\textunderscore .
(Cp. \textunderscore zoina\textunderscore )
\section{Zaipana}
\begin{itemize}
\item {Grp. gram.:m.}
\end{itemize}
\begin{itemize}
\item {Utilização:T. do Fundão}
\end{itemize}
Burguês lorpa ou grosseiro.
Homem ingênuo e bonacheirão, patrazana.
\section{Zaira}
\begin{itemize}
\item {Grp. gram.:f.}
\end{itemize}
\begin{itemize}
\item {Proveniência:(De \textunderscore Zaira\textunderscore , n. p.)}
\end{itemize}
Insecto díptero, espécie de môsca.
\section{Zalumar}
\begin{itemize}
\item {Grp. gram.:v. i.}
\end{itemize}
\begin{itemize}
\item {Utilização:Gír. de marinheiro.}
\end{itemize}
Erguer a voz; cantar.
(Por \textunderscore celeumar\textunderscore , de \textunderscore celeuma\textunderscore ?)
\section{Zama}
\begin{itemize}
\item {Grp. gram.:m.}
\end{itemize}
\begin{itemize}
\item {Proveniência:(T. afr.)}
\end{itemize}
Espécie de feijão, na província de Moçambique.
\section{Zambaio}
\begin{itemize}
\item {Grp. gram.:m.  e  adj.}
\end{itemize}
\begin{itemize}
\item {Utilização:Prov.}
\end{itemize}
\begin{itemize}
\item {Utilização:alg.}
\end{itemize}
Zanaga, zarolho.
(Cp. \textunderscore zambro\textunderscore )
\section{Zambana}
\begin{itemize}
\item {Grp. gram.:m.}
\end{itemize}
\begin{itemize}
\item {Utilização:Prov.}
\end{itemize}
O mesmo que \textunderscore zaipana\textunderscore .
\section{Zambeta}
\begin{itemize}
\item {fónica:bê}
\end{itemize}
\begin{itemize}
\item {Grp. gram.:adj.}
\end{itemize}
\begin{itemize}
\item {Utilização:Bras}
\end{itemize}
\begin{itemize}
\item {Utilização:Bras. do N}
\end{itemize}
Zambro; cambaio.
Que tem os pés dispostos de maneira, que os calcanhares se tocam, e as pontas se desviam muito.
\section{Zambo}
\begin{itemize}
\item {Grp. gram.:m.  e  adj.}
\end{itemize}
Designação brasileira dos filhos de preto e de mulhér indígena.
Espécie de macaco da América, disforme, e muito selvagem (\textunderscore simia sphinx\textunderscore ).
\section{Zambôa}
\begin{itemize}
\item {Grp. gram.:f.}
\end{itemize}
\begin{itemize}
\item {Utilização:Bras}
\end{itemize}
\begin{itemize}
\item {Utilização:Ant.}
\end{itemize}
\begin{itemize}
\item {Utilização:Fig.}
\end{itemize}
Espécie de cidra.
O mesmo que \textunderscore gambôa\textunderscore ^1.
Pessôa estúpida.
(Cast. \textunderscore zamboa\textunderscore )
\section{Zamboeira}
\begin{itemize}
\item {Grp. gram.:f.}
\end{itemize}
Árvore, que dá zambôas.
\section{Zamborrada}
\begin{itemize}
\item {Grp. gram.:f.}
\end{itemize}
\begin{itemize}
\item {Utilização:Prov.}
\end{itemize}
\begin{itemize}
\item {Utilização:trasm.}
\end{itemize}
Bátega forte e rápida (de água).
\section{Zamboto}
\begin{itemize}
\item {fónica:bô}
\end{itemize}
\begin{itemize}
\item {Grp. gram.:m.}
\end{itemize}
\begin{itemize}
\item {Utilização:T. de Moncorvo}
\end{itemize}
O mesmo que \textunderscore jangoto\textunderscore .
\section{Zambozinos}
\begin{itemize}
\item {Grp. gram.:m. pl.}
\end{itemize}
O mesmo que \textunderscore gambozinos\textunderscore .
\section{Zambrana}
\begin{itemize}
\item {Grp. gram.:m.}
\end{itemize}
\begin{itemize}
\item {Utilização:Prov.}
\end{itemize}
O mesmo que \textunderscore zaipana\textunderscore .
\section{Zambra}
\begin{itemize}
\item {Grp. gram.:f.}
\end{itemize}
\begin{itemize}
\item {Proveniência:(Do ár. \textunderscore zamra\textunderscore )}
\end{itemize}
Espécie de dança e música moirisca, que se conservou na península hispânica.
\section{Zambra}
\begin{itemize}
\item {Grp. gram.:f.}
\end{itemize}
Embarcação, o mesmo que \textunderscore zabra\textunderscore .
\section{Zambralho}
\begin{itemize}
\item {Grp. gram.:m.}
\end{itemize}
Uma das espécies de aves, em que se cevam os falcões. Cf. Fern. Pereira, \textunderscore Caça de Altan.\textunderscore 
\section{Zambro}
\begin{itemize}
\item {Grp. gram.:adj.}
\end{itemize}
Que tem pernas tortas; cambaio.
Tôrto, (falando-se das pernas):«\textunderscore ...de pernas zambras...\textunderscore »\textunderscore Anat. Joc.\textunderscore , 7.
\section{Zambuco}
\begin{itemize}
\item {Grp. gram.:m.}
\end{itemize}
(V.sambuco)
\section{Zambugal}
\begin{itemize}
\item {Grp. gram.:m.}
\end{itemize}
Árvore brasileira, cujos frutos, do tamanho de cocos, contêm uma noz comestível.
\section{Zambujal}
\begin{itemize}
\item {Grp. gram.:m.}
\end{itemize}
\begin{itemize}
\item {Proveniência:(De \textunderscore zambujo\textunderscore )}
\end{itemize}
Terreno, onde crescem zambujeiros.
\section{Zambujeira}
\begin{itemize}
\item {Grp. gram.:f.  e  adj.}
\end{itemize}
\begin{itemize}
\item {Utilização:Prov.}
\end{itemize}
\begin{itemize}
\item {Utilização:alent.}
\end{itemize}
Diz-se de uma variedade de azeitona miúda.
\section{Zambujeiro}
\begin{itemize}
\item {Grp. gram.:m.}
\end{itemize}
\begin{itemize}
\item {Proveniência:(De \textunderscore zambujo\textunderscore )}
\end{itemize}
Árvore rhamnácea, espécie de oliveira brava.
\section{Zambujo}
\begin{itemize}
\item {Grp. gram.:m.}
\end{itemize}
O mesmo que \textunderscore zambujeiro\textunderscore .
(Cp. cast. \textunderscore acebuche\textunderscore , do ár.)
\section{Zambulha}
\begin{itemize}
\item {Grp. gram.:f.  e  adj.}
\end{itemize}
O mesmo que \textunderscore zambulheira\textunderscore .
\section{Zambulheira}
\begin{itemize}
\item {Grp. gram.:f.  e  adj.}
\end{itemize}
\begin{itemize}
\item {Utilização:Prov.}
\end{itemize}
\begin{itemize}
\item {Utilização:trasm.}
\end{itemize}
Diz-se de uma variedade de azeitona.
\section{Zamburrada}
\begin{itemize}
\item {Grp. gram.:f.}
\end{itemize}
\begin{itemize}
\item {Utilização:Prov.}
\end{itemize}
\begin{itemize}
\item {Utilização:trasm.}
\end{itemize}
Grande quantidade.
\section{Zamorano}
\begin{itemize}
\item {Grp. gram.:adj.}
\end{itemize}
\begin{itemize}
\item {Grp. gram.:M.}
\end{itemize}
Relativo a Zamora, em Espanha.
Habitante de Zamora.--Melhór fórma seria \textunderscore çamorano\textunderscore  ou \textunderscore çamorão\textunderscore .
\section{Zamori}
\begin{itemize}
\item {Grp. gram.:m.}
\end{itemize}
O mesmo que \textunderscore zamorim\textunderscore .
\section{Zamorim}
\begin{itemize}
\item {Grp. gram.:m.}
\end{itemize}
Título dos antigos soberanos de Calecute, na Índia.--Também se tem escrito \textunderscore samorim\textunderscore , mas preferível é \textunderscore çamorim\textunderscore .
\section{Zampar}
\begin{itemize}
\item {Grp. gram.:v. t.}
\end{itemize}
Comer muito, com avidez e á pressa.
Enzampar.
(Cast. \textunderscore zampar\textunderscore )
\section{Zamparina}
\begin{itemize}
\item {Grp. gram.:adj.}
\end{itemize}
\begin{itemize}
\item {Proveniência:(De \textunderscore Zamparini\textunderscore , n. p.)}
\end{itemize}
Diz-se de uma fórma de usar o chapéu, inclinado para a testa e para a orelha direita.
\section{Zamponha}
\begin{itemize}
\item {Grp. gram.:f.}
\end{itemize}
\begin{itemize}
\item {Utilização:Des.}
\end{itemize}
O mesmo que \textunderscore sanfona\textunderscore :«\textunderscore o eco das zamponhas...\textunderscore »Filinto, X, 83.
(Cast. \textunderscore zampoña\textunderscore )
\section{Zamumo}
\begin{itemize}
\item {Grp. gram.:m.}
\end{itemize}
Grande árvore medicinal da ilha de San-Thomé.
\section{Zanaga}
\begin{itemize}
\item {Grp. gram.:m. ,  f.  e  adj.}
\end{itemize}
\begin{itemize}
\item {Utilização:Pop.}
\end{itemize}
Pessôa vesga.
\section{Zanago}
\begin{itemize}
\item {Grp. gram.:adj.}
\end{itemize}
O mesmo que \textunderscore zanaga\textunderscore . Cf. Arn. Gama, \textunderscore Segr. do Abb.\textunderscore , 58.
\section{Zancarrão}
\begin{itemize}
\item {Grp. gram.:m.}
\end{itemize}
O mesmo que \textunderscore sancarrão\textunderscore :«\textunderscore o zancarrão de Mafoma é o osso de uma sua perna.\textunderscore »M. Bernardez.
\section{Zanga}
\begin{itemize}
\item {Grp. gram.:f.}
\end{itemize}
\begin{itemize}
\item {Utilização:Bras. de Minas}
\end{itemize}
Aborrecimento.
Importunação.
Aversão.
Enguiço.
Espécie de voltarete, entre dois parceiros e sem o naipe de copas. (Esta última acepção talvez seja a primitiva)
Desarranjo: \textunderscore o meu relógio tem zanga\textunderscore .
(Cast. \textunderscore zanga\textunderscore )
\section{Zanga}
\begin{itemize}
\item {Grp. gram.:f.}
\end{itemize}
\begin{itemize}
\item {Utilização:Prov.}
\end{itemize}
Peça de madeira, em fórma de cruz, com que nas eiras se acamam e seguram as paveias de trigo, centeio ou cevada, para serem debulhadas com o mangual.
Espécie de moínho caseiro, que se move a braço, geralmente desusado, mas ainda conhecido na Bairrada, e frequente em Marrocos e na Argélia.
(Do ár.)
\section{Zanga}
\begin{itemize}
\item {Grp. gram.:f.}
\end{itemize}
\begin{itemize}
\item {Utilização:Bras}
\end{itemize}
Insecto, o mesmo que \textunderscore nígua\textunderscore .
\section{Zangaburrinha}
\begin{itemize}
\item {Grp. gram.:f.}
\end{itemize}
\begin{itemize}
\item {Utilização:Bras}
\end{itemize}
Apparelho, o mesmo que \textunderscore gangorra\textunderscore ^1.
\section{Zangado}
\begin{itemize}
\item {Grp. gram.:adj.}
\end{itemize}
\begin{itemize}
\item {Proveniência:(De \textunderscore zangar\textunderscore ^1)}
\end{itemize}
Que se zanga facilmente.
Que se zangou.
Irritado.
\section{Zangador}
\begin{itemize}
\item {Grp. gram.:m.  e  adj.}
\end{itemize}
O que causa zanga.
\section{Zangalhão}
\begin{itemize}
\item {Grp. gram.:m.}
\end{itemize}
O mesmo que \textunderscore zangaralhão\textunderscore .
\section{Zangalho}
\begin{itemize}
\item {Grp. gram.:m.}
\end{itemize}
O mesmo que \textunderscore zangaralhão\textunderscore .
\section{Zangam}
\begin{itemize}
\item {Grp. gram.:m.}
\end{itemize}
\begin{itemize}
\item {Utilização:Fig.}
\end{itemize}
Espécie de abelha, que não fabríca mel, e come o que as outras fabricam, (\textunderscore vespa crabro\textunderscore ).
Parasito.
Explorador.
Importuno; zângano.
(Cast. \textunderscore zángano\textunderscore )
\section{Zângano}
\begin{itemize}
\item {Grp. gram.:m.}
\end{itemize}
Parasito.
Agiota fraudulento.
Agente de negócios particulares.
Adelo.
Bobo.
(Cast. \textunderscore zángano\textunderscore )
\section{Zângão}
\begin{itemize}
\item {Grp. gram.:m.}
\end{itemize}
\begin{itemize}
\item {Utilização:Fig.}
\end{itemize}
Espécie de abelha, que não fabríca mel, e come o que as outras fabricam, (\textunderscore vespa crabro\textunderscore ).
Parasito.
Explorador.
Importuno; zângano.
(Cast. \textunderscore zángano\textunderscore )
\section{Zangar}
\begin{itemize}
\item {Grp. gram.:v. t.}
\end{itemize}
\begin{itemize}
\item {Grp. gram.:V. p.}
\end{itemize}
Causar zanga^1 a.
Irritar-se; têr zanga.
\section{Zangar}
\begin{itemize}
\item {Grp. gram.:v. t.}
\end{itemize}
\begin{itemize}
\item {Utilização:Prov.}
\end{itemize}
\begin{itemize}
\item {Utilização:trasm.}
\end{itemize}
Transpor, saltando.
\section{Zangaralhão}
\begin{itemize}
\item {Grp. gram.:m.}
\end{itemize}
\begin{itemize}
\item {Utilização:Pop.}
\end{itemize}
O mesmo que \textunderscore trangalhadanças\textunderscore .
(Cast. \textunderscore zangarullon\textunderscore )
\section{Zangaralheiro}
\begin{itemize}
\item {Grp. gram.:m.}
\end{itemize}
\begin{itemize}
\item {Utilização:Ant.}
\end{itemize}
Aquelle que adorna igrejas ou arma e dispõe os ornatos de festas religiosas?:«\textunderscore ...cada um me quer por zangaralheiro para o seu presepio.\textunderscore »\textunderscore Anat. Joc.\textunderscore , I, 331.
\section{Zangarelha}
\begin{itemize}
\item {fónica:garê}
\end{itemize}
\begin{itemize}
\item {Grp. gram.:f.}
\end{itemize}
Nome que, em Ílhavo, se dá á tarrafa de arrastar.
\section{Zangarelho}
\begin{itemize}
\item {fónica:garê}
\end{itemize}
\begin{itemize}
\item {Grp. gram.:f.}
\end{itemize}
\begin{itemize}
\item {Utilização:Pesc.}
\end{itemize}
Rêde de um só pano, para emmalhar pescadas.
O mesmo que \textunderscore zangarelha\textunderscore .
\section{Zangarilhar}
\begin{itemize}
\item {Grp. gram.:v. i.}
\end{itemize}
\begin{itemize}
\item {Utilização:Prov.}
\end{itemize}
\begin{itemize}
\item {Utilização:trasm.}
\end{itemize}
Andar para trás e para deante; passar e tornar a passar.
\section{Zangarilheira, á}
\begin{itemize}
\item {Grp. gram.:loc. adv.}
\end{itemize}
\begin{itemize}
\item {Utilização:Prov.}
\end{itemize}
\begin{itemize}
\item {Utilização:trasm.}
\end{itemize}
Livremente, á vontade.
\section{Zangarinheiro}
\begin{itemize}
\item {Grp. gram.:m.}
\end{itemize}
O mesmo que \textunderscore sanguinheiro\textunderscore .
\section{Zangarinho}
\begin{itemize}
\item {Grp. gram.:m.}
\end{itemize}
\begin{itemize}
\item {Utilização:Prov.}
\end{itemize}
\begin{itemize}
\item {Utilização:minh.}
\end{itemize}
O mesmo que \textunderscore zangarinheiro\textunderscore .
\section{Zangarrão}
\begin{itemize}
\item {Grp. gram.:m.}
\end{itemize}
\begin{itemize}
\item {Utilização:Prov.}
\end{itemize}
\begin{itemize}
\item {Utilização:trasm.}
\end{itemize}
\begin{itemize}
\item {Utilização:T. de Moncorvo}
\end{itemize}
Homem que, nalgumas terras, anda pedindo esmola para os santos, vestido de diabo.
O mesmo que \textunderscore besoiro\textunderscore .
\section{Zangarrear}
\begin{itemize}
\item {Grp. gram.:v. t.}
\end{itemize}
Tocar viola, á maneira chula, marcando rythmo sempre do mesmo modo e com os mesmos acordes em rasgado. Cf. E. Vieira, \textunderscore Diccion. Mus.\textunderscore 
(Cast. \textunderscore zangarrear\textunderscore )
\section{Zangarreio}
\begin{itemize}
\item {Grp. gram.:m.}
\end{itemize}
Acto de zangarrear.
\section{Zango}
\begin{itemize}
\item {Grp. gram.:m.}
\end{itemize}
(Fórma pop. de \textunderscore zângão\textunderscore )
\section{Zangorriana}
\begin{itemize}
\item {Grp. gram.:f.}
\end{itemize}
(V.zangurriana)
\section{Zangorrino}
\begin{itemize}
\item {Grp. gram.:m.}
\end{itemize}
\begin{itemize}
\item {Utilização:T. da Bairrada}
\end{itemize}
Indivíduo sonso, dissimulado.
(Cp. \textunderscore zagorrino\textunderscore )
\section{Zangrinheiro}
\begin{itemize}
\item {Grp. gram.:m.}
\end{itemize}
O mesmo que \textunderscore sanguinheiro\textunderscore .
\section{Zanguerrear}
\begin{itemize}
\item {Grp. gram.:v. i.}
\end{itemize}
\begin{itemize}
\item {Utilização:Prov.}
\end{itemize}
\begin{itemize}
\item {Utilização:trasm.}
\end{itemize}
O mesmo que \textunderscore zinguerrear\textunderscore .
\section{Zanguizarra}
\begin{itemize}
\item {Grp. gram.:f.}
\end{itemize}
\begin{itemize}
\item {Utilização:Pop.}
\end{itemize}
Algazarra; tumulto.
Toque desafinado de viola.
Qualquer toque ou som estrídulo.
(Cp. \textunderscore zangarrear\textunderscore )
\section{Zangurrar}
\begin{itemize}
\item {Grp. gram.:v. i.}
\end{itemize}
\begin{itemize}
\item {Utilização:T. do Fundão}
\end{itemize}
Vadiar; mandriar.
Andar á tuna, na vida airada.
\section{Zangurriana}
\begin{itemize}
\item {Grp. gram.:f.}
\end{itemize}
\begin{itemize}
\item {Utilização:Chul.}
\end{itemize}
Embriaguez.
Cantilena monótona e persistente.
(Cp. \textunderscore zangarrear\textunderscore )
\section{Zangurrina}
\begin{itemize}
\item {Grp. gram.:f.}
\end{itemize}
O mesmo que \textunderscore zangurriana\textunderscore , bebedeira.
\section{Zanizo}
\begin{itemize}
\item {Grp. gram.:m.}
\end{itemize}
\begin{itemize}
\item {Utilização:Bras}
\end{itemize}
Planta medicinal.
\section{Zanoio}
\begin{itemize}
\item {Grp. gram.:m.  e  adj.}
\end{itemize}
O mesmo que \textunderscore zanolho\textunderscore .
\section{Zanolho}
\begin{itemize}
\item {fónica:nô}
\end{itemize}
\begin{itemize}
\item {Grp. gram.:m.  e  adj.}
\end{itemize}
O mesmo que \textunderscore zarolho\textunderscore .
\section{Zanquim}
\begin{itemize}
\item {Grp. gram.:m.}
\end{itemize}
Antiga moéda turca. Cf. F. Manuel, \textunderscore Apólogos\textunderscore .
\section{Zante}
\begin{itemize}
\item {Grp. gram.:m.}
\end{itemize}
Antiga moéda de Veneza. Cf. F. Manuel, \textunderscore Apólogos\textunderscore .
\section{Zante}
\begin{itemize}
\item {Grp. gram.:m.}
\end{itemize}
Casta de uva brasileira.
\section{Zânthio}
\begin{itemize}
\item {Grp. gram.:m.}
\end{itemize}
Planta medicinal, febrífuga, (\textunderscore zanthium spinosum\textunderscore ).
\section{Zanthoxýleas}
\begin{itemize}
\item {fónica:csi}
\end{itemize}
\begin{itemize}
\item {Grp. gram.:f. pl.}
\end{itemize}
Família de plantas, que, tem por typo o zanthóxylo.
\section{Zanthóxylo}
\begin{itemize}
\item {fónica:csi}
\end{itemize}
\begin{itemize}
\item {Grp. gram.:m.}
\end{itemize}
Gênero de plantas medicinaes dos trópicos, que alguns botânicos collocam na fam. das rutáceas, desapprovando a formação da fam. das zanthoxýleas.
\section{Zântio}
\begin{itemize}
\item {Grp. gram.:m.}
\end{itemize}
Planta medicinal, febrífuga, (\textunderscore zanthium spinosum\textunderscore ).
\section{Zantoxíleas}
\begin{itemize}
\item {fónica:csi}
\end{itemize}
\begin{itemize}
\item {Grp. gram.:f. pl.}
\end{itemize}
Família de plantas, que, tem por tipo o zantóxilo.
\section{Zantóxilo}
\begin{itemize}
\item {fónica:csi}
\end{itemize}
\begin{itemize}
\item {Grp. gram.:m.}
\end{itemize}
Gênero de plantas medicinaes dos trópicos, que alguns botânicos colocam na fam. das rutáceas, desaprovando a formação da fam. das zantoxíleas.
\section{Zanzar}
\begin{itemize}
\item {Grp. gram.:v. i.}
\end{itemize}
\begin{itemize}
\item {Utilização:Bras}
\end{itemize}
Andar ao acaso, vaguear.
(Contr. de \textunderscore zaranzar\textunderscore )
\section{Zanzarilhar}
\begin{itemize}
\item {Grp. gram.:v. i.}
\end{itemize}
O mesmo que \textunderscore zangarilhar\textunderscore .
\section{Zanzibar}
\begin{itemize}
\item {Grp. gram.:m.}
\end{itemize}
Língua banta da costa oriental da África.
\section{Zanzibarita}
\begin{itemize}
\item {Grp. gram.:adj.}
\end{itemize}
\begin{itemize}
\item {Grp. gram.:M.}
\end{itemize}
Relativo a Zanzibar.
Habitante de Zanzibar.
\section{Zanzino}
\begin{itemize}
\item {Grp. gram.:m.}
\end{itemize}
\begin{itemize}
\item {Utilização:Prov.}
\end{itemize}
\begin{itemize}
\item {Utilização:trasm.}
\end{itemize}
O mesmo que \textunderscore moscardo\textunderscore .
\section{Zanzo}
\begin{itemize}
\item {Grp. gram.:m.}
\end{itemize}
Planta malvácea do Brasil.
\section{Zão-zão}
\begin{itemize}
\item {Grp. gram.:m.}
\end{itemize}
\begin{itemize}
\item {Proveniência:(T. onom.)}
\end{itemize}
Som monótono; zum-zum:«\textunderscore o ajoujado zão-zão dos consoantes...\textunderscore »Filinto, IV, 210.
\section{Zape}
\begin{itemize}
\item {Grp. gram.:m.}
\end{itemize}
\begin{itemize}
\item {Grp. gram.:Interj.}
\end{itemize}
\begin{itemize}
\item {Proveniência:(T. onom.)}
\end{itemize}
Pancada.
Voz imitativa dessa pancada; sape.
\section{Zápete}
\begin{itemize}
\item {Grp. gram.:m.}
\end{itemize}
O quatro de paus, no jôgo do truque.
Jôgo do truque.
\section{Zapetrape}
\begin{itemize}
\item {Grp. gram.:m.}
\end{itemize}
Mãozada de gato. Cf. F. Manuel, \textunderscore F. dos Anexins\textunderscore , 170.
\section{Zapota}
\begin{itemize}
\item {Grp. gram.:f.}
\end{itemize}
O mesmo que \textunderscore sapota\textunderscore .
\section{Zapote}
\begin{itemize}
\item {Grp. gram.:m.}
\end{itemize}
O mesmo que \textunderscore sapote\textunderscore .
Fruto do zapotilheiro.
\section{Zapoteca}
\begin{itemize}
\item {Grp. gram.:m.}
\end{itemize}
\begin{itemize}
\item {Grp. gram.:Pl.}
\end{itemize}
Idioma dos Zapotecas.
Indígenas americanos, que hoje fazem parte do México, povoando o Estado de Oaxaca.
\section{Zapotilha}
\begin{itemize}
\item {Grp. gram.:f.}
\end{itemize}
Fruto do zapotilheiro, de grãos oleosos.
\section{Zapotilheiro}
\begin{itemize}
\item {Grp. gram.:m.}
\end{itemize}
Árvore americana, o mesmo que \textunderscore sapota\textunderscore , (\textunderscore phytolácea divica\textunderscore ), de que se extrai uma substância análoga á guta-percha.
\section{Zapupe}
\begin{itemize}
\item {Grp. gram.:m.}
\end{itemize}
Fibra têxtil de uma variedade de agaves, no México. Cf. \textunderscore Jorn.-do-Comm.\textunderscore , do Rio, de 10-XII-908.
\section{Zarabatana}
\begin{itemize}
\item {Grp. gram.:f.}
\end{itemize}
\begin{itemize}
\item {Proveniência:(T. ár.)}
\end{itemize}
Tubo comprido, pelo qual se impellem com o sopro setas ou bolinhas.
\section{Zarabona}
\begin{itemize}
\item {Grp. gram.:f.}
\end{itemize}
\begin{itemize}
\item {Utilização:T. da Ilha das Flôres}
\end{itemize}
\begin{itemize}
\item {Proveniência:(De \textunderscore Zerbone\textunderscore , n. p.)}
\end{itemize}
O mesmo que \textunderscore relojoeiro\textunderscore .
\section{Zaracoteia}
\begin{itemize}
\item {Grp. gram.:f.}
\end{itemize}
(V.zaragatôa)(Us. por Camillo)
\section{Zaragalhada}
\begin{itemize}
\item {Grp. gram.:f.}
\end{itemize}
\begin{itemize}
\item {Utilização:Pop.}
\end{itemize}
Alvorôto, o mesmo que \textunderscore turbamulta\textunderscore .
\section{Zaragata}
\begin{itemize}
\item {Grp. gram.:f.}
\end{itemize}
\begin{itemize}
\item {Utilização:Pop.}
\end{itemize}
O mesmo que \textunderscore zaragalhada\textunderscore .
Desordem; algazarra; confusão.
(Cast. \textunderscore zaragata\textunderscore )
\section{Zaragatôa}
\begin{itemize}
\item {Grp. gram.:f.}
\end{itemize}
\begin{itemize}
\item {Utilização:Ext.}
\end{itemize}
Nome de duas plantas plantagíneas.
Pequena esponja, na extremidade de um pauzinho, para applicar medicamentos na garganta ou fossas nasaes.
Pincel de fios de linho, para o mesmo fim.
Medicamento, que se applica com êstes objectos.
(Cast. \textunderscore zaragatona\textunderscore )
\section{Zaragota}
\begin{itemize}
\item {fónica:gô}
\end{itemize}
\begin{itemize}
\item {Grp. gram.:f.}
\end{itemize}
\begin{itemize}
\item {Utilização:P. us.}
\end{itemize}
O mesmo que \textunderscore zaragatôa\textunderscore , planta.
\section{Zarandalha}
\begin{itemize}
\item {Grp. gram.:f.}
\end{itemize}
\begin{itemize}
\item {Utilização:Ant.}
\end{itemize}
Ninharia; bagatela:«\textunderscore ...trazendo os Ethiopes frutos e carnes... que os nossos resgatavam com fatos de mui vil preço e tenuissimas zarandalhas...\textunderscore »Filinto, \textunderscore D. Man.\textunderscore , I, 67.
(Cast. \textunderscore zarandajas\textunderscore )
\section{Zaranguilhar}
\begin{itemize}
\item {Grp. gram.:v. i.}
\end{itemize}
\begin{itemize}
\item {Utilização:Prov.}
\end{itemize}
\begin{itemize}
\item {Utilização:minh.}
\end{itemize}
O mesmo que \textunderscore zangarilhar\textunderscore .
\section{Zaranza}
\begin{itemize}
\item {Grp. gram.:m. ,  f.  e  adj.}
\end{itemize}
\begin{itemize}
\item {Grp. gram.:Adj.}
\end{itemize}
\begin{itemize}
\item {Utilização:T. de Turquel}
\end{itemize}
Pessôa atabalhoada, que procede sem reflectir; doidivanas.
Bêbedo.
\section{Zaranzar}
\begin{itemize}
\item {Grp. gram.:v. i.}
\end{itemize}
\begin{itemize}
\item {Proveniência:(De \textunderscore zaranza\textunderscore )}
\end{itemize}
Andar á tôa, ao acaso; atrapalhar-se no andar ou nos movimentos.
\section{Zarão}
\begin{itemize}
\item {Grp. gram.:m.}
\end{itemize}
\begin{itemize}
\item {Utilização:Prov.}
\end{itemize}
\begin{itemize}
\item {Utilização:trasm.}
\end{itemize}
Pião grande.
\section{Zarapelho}
\begin{itemize}
\item {fónica:pê}
\end{itemize}
\begin{itemize}
\item {Grp. gram.:m.}
\end{itemize}
\begin{itemize}
\item {Utilização:Prov.}
\end{itemize}
O diabo.
\section{Zarasca}
\begin{itemize}
\item {Grp. gram.:f.}
\end{itemize}
\begin{itemize}
\item {Utilização:Prov.}
\end{itemize}
\begin{itemize}
\item {Utilização:trasm.}
\end{itemize}
Pião pequeno ou reles.
(Cp. \textunderscore zarão\textunderscore )
\section{Zarba}
\begin{itemize}
\item {Grp. gram.:f.}
\end{itemize}
\begin{itemize}
\item {Utilização:Prov.}
\end{itemize}
\begin{itemize}
\item {Utilização:trasm.}
\end{itemize}
Mata de arbustos; sebe viva.
\section{Zarca}
\begin{itemize}
\item {Grp. gram.:f.}
\end{itemize}
\begin{itemize}
\item {Proveniência:(De \textunderscore zarco\textunderscore )}
\end{itemize}
Mulhér, de olhos azues.
\section{Zarcão}
\begin{itemize}
\item {Grp. gram.:m.}
\end{itemize}
Nome vulgar do mínio.
Côr do laranja ou de tijolo, muito viva.
(Do ár.)
\section{Zarco}
\begin{itemize}
\item {Grp. gram.:adj.}
\end{itemize}
\begin{itemize}
\item {Proveniência:(Do ár. \textunderscore zarca\textunderscore )}
\end{itemize}
Que tem olhos azues claros.
Que tem malha branca em volta de um ou de ambos os olhos, (falando-se do cavallo).
\section{Zarelha}
\begin{itemize}
\item {fónica:zarê}
\end{itemize}
\begin{itemize}
\item {Grp. gram.:f.}
\end{itemize}
(Fem. de \textunderscore zarelho\textunderscore )
\section{Zarelhar}
\begin{itemize}
\item {Grp. gram.:v. i.}
\end{itemize}
\begin{itemize}
\item {Proveniência:(De \textunderscore zarelho\textunderscore )}
\end{itemize}
Intrometer-se em tudo; intrigar.
Traquinar; doidejar.
\section{Zarelho}
\begin{itemize}
\item {fónica:zarê}
\end{itemize}
\begin{itemize}
\item {Grp. gram.:m.}
\end{itemize}
\begin{itemize}
\item {Utilização:Prov.}
\end{itemize}
\begin{itemize}
\item {Utilização:minh.}
\end{itemize}
Homem metediço.
Homem ou rapaz travesso.
Zaranza.
Indivíduo, que gagueja um pouco.
Peça de arame, que segura a bandoleira á espingarda.
\section{Zargo}
\begin{itemize}
\item {Grp. gram.:adj.}
\end{itemize}
(Corr. de \textunderscore zarco\textunderscore )
\section{Zarguncho}
\textunderscore m.\textunderscore  (e der.)
O mesmo que \textunderscore zaguncho\textunderscore , etc.
Peixe de Portugal.
\section{Zaro}
\begin{itemize}
\item {Grp. gram.:m.  e  adj.}
\end{itemize}
Diz-se, no Algarve, de uma variedade de figos, redondos e pardos.
\section{Zaroca}
\begin{itemize}
\item {Grp. gram.:f.}
\end{itemize}
\begin{itemize}
\item {Utilização:Prov.}
\end{itemize}
Covazinha no chão, para o jôgo do botão ou da bugalhinha.
\section{Zarolha}
\begin{itemize}
\item {fónica:zarô}
\end{itemize}
\begin{itemize}
\item {Grp. gram.:adj. f.}
\end{itemize}
\begin{itemize}
\item {Utilização:Prov.}
\end{itemize}
\begin{itemize}
\item {Utilização:minh.}
\end{itemize}
Diz-se da roupa mal enxuta.
\section{Zarolho}
\begin{itemize}
\item {fónica:zarô}
\end{itemize}
\begin{itemize}
\item {Grp. gram.:adj.}
\end{itemize}
\begin{itemize}
\item {Utilização:Chul.}
\end{itemize}
\begin{itemize}
\item {Utilização:Bras. do N}
\end{itemize}
Vesgo.
Cego de um ôlho.
Diz-se do milho, no comêço da maturação.
\section{Zaróna}
\begin{itemize}
\item {Grp. gram.:f.}
\end{itemize}
\begin{itemize}
\item {Utilização:Prov.}
\end{itemize}
\begin{itemize}
\item {Utilização:trasm.}
\end{itemize}
Pião, o mesmo que \textunderscore zarasca\textunderscore .
\section{Zarpar}
\begin{itemize}
\item {Grp. gram.:v. t.}
\end{itemize}
\begin{itemize}
\item {Utilização:Prov.}
\end{itemize}
\begin{itemize}
\item {Utilização:minh.}
\end{itemize}
\begin{itemize}
\item {Grp. gram.:V. i.}
\end{itemize}
\begin{itemize}
\item {Utilização:Bras}
\end{itemize}
O mesmo ou melhór que \textunderscore sarpar\textunderscore .
Enganar, abusar da bôa fé de, em proveito próprio.
Fugir.
(Cast. \textunderscore zarpar\textunderscore )
\section{Zarra}
\begin{itemize}
\item {Grp. gram.:f.}
\end{itemize}
\begin{itemize}
\item {Utilização:Ant.}
\end{itemize}
O mesmo que \textunderscore jarra\textunderscore ^2.
Almotolia.
\section{Zarro}
\begin{itemize}
\item {Grp. gram.:m.}
\end{itemize}
\begin{itemize}
\item {Utilização:Náut.}
\end{itemize}
Cabo náutico, com pernadas fixas no têrço da vêrga da gávea.
O mesmo que \textunderscore tarrantana\textunderscore .
\section{Zarza}
\begin{itemize}
\item {Grp. gram.:f.}
\end{itemize}
\begin{itemize}
\item {Utilização:Bras}
\end{itemize}
\begin{itemize}
\item {Proveniência:(T. cast.)}
\end{itemize}
O mesmo que \textunderscore salsa-parrilha\textunderscore .
\section{Zarzagitânia}
\begin{itemize}
\item {Grp. gram.:f.}
\end{itemize}
\begin{itemize}
\item {Utilização:Ant.}
\end{itemize}
Espécie de pano de algodão, usado entre os Moiros. Cf. Sousa, \textunderscore Ann. de D. João III\textunderscore .
\section{Zarzuela}
\begin{itemize}
\item {Grp. gram.:f.}
\end{itemize}
Peça theatral espanhola, parte da qual é cantada.
Espécie de ópera cómica.
Opereta.
(Cast. \textunderscore zarzuela\textunderscore )
\section{Zás!}
\begin{itemize}
\item {Grp. gram.:interj.}
\end{itemize}
\begin{itemize}
\item {Proveniência:(T. onom.)}
\end{itemize}
(imitativa de pancada)
\section{Zás-catrás!}
\begin{itemize}
\item {Grp. gram.:interj.}
\end{itemize}
\begin{itemize}
\item {Utilização:T. de Alcanena}
\end{itemize}
O mesmo que \textunderscore zás-trás!\textunderscore 
\section{Zás-trás!}
\begin{itemize}
\item {Grp. gram.:interj.}
\end{itemize}
O mesmo que \textunderscore zás!\textunderscore 
\section{Zatu}
\begin{itemize}
\item {Grp. gram.:m.}
\end{itemize}
Animal cornígero do Brasil.
\section{Zavada}
\begin{itemize}
\item {Grp. gram.:adj. f.}
\end{itemize}
\begin{itemize}
\item {Utilização:Prov.}
\end{itemize}
\begin{itemize}
\item {Utilização:trasm.}
\end{itemize}
Diz-se de uma cara deslavada, sem vergonha.
(Corr. de \textunderscore deslavada\textunderscore ?)
\section{Zavaneira}
\begin{itemize}
\item {Grp. gram.:f.}
\end{itemize}
\begin{itemize}
\item {Utilização:Prov.}
\end{itemize}
\begin{itemize}
\item {Utilização:trasm.}
\end{itemize}
Bôa dona de casa, muito diligente.--É notável a semelhança com \textunderscore zabaneira\textunderscore  e a divergência de significados.
\section{Zavar}
\begin{itemize}
\item {Grp. gram.:v. i.}
\end{itemize}
\begin{itemize}
\item {Utilização:Prov.}
\end{itemize}
\begin{itemize}
\item {Utilização:trasm.}
\end{itemize}
Morder raivosamente, com frenesi.
\section{Zavra}
\begin{itemize}
\item {Grp. gram.:f.}
\end{itemize}
Embarcação, o mesmo que \textunderscore zabra\textunderscore .
\section{Zazerino}
\begin{itemize}
\item {Grp. gram.:adj.}
\end{itemize}
(V.jazerino)
\section{Zazinta}
\begin{itemize}
\item {Grp. gram.:f.}
\end{itemize}
Planta medicinal, o mesmo que \textunderscore zacintos\textunderscore .
\section{Zazo}
\begin{itemize}
\item {Grp. gram.:m.}
\end{itemize}
\begin{itemize}
\item {Utilização:Ant.}
\end{itemize}
Supremo sacerdote, entre os Japoneses.
\section{Zazyntha}
\begin{itemize}
\item {Grp. gram.:f.}
\end{itemize}
Planta medicinal, o mesmo que \textunderscore zacynthos\textunderscore .
\section{Zéa}
\begin{itemize}
\item {Grp. gram.:m.}
\end{itemize}
Nome scientífico do maís.
\section{Zebar}
\textunderscore v. t.\textunderscore  (e der)
Us. no Mínho, por \textunderscore cevar\textunderscore , etc.
\section{Zebo}
\begin{itemize}
\item {fónica:zê}
\end{itemize}
\begin{itemize}
\item {Grp. gram.:m.}
\end{itemize}
O mesmo que \textunderscore gêbo\textunderscore , espécie de boi selvagem.
(Cp. \textunderscore gêbo\textunderscore )
\section{Zebra}
\begin{itemize}
\item {fónica:zê}
\end{itemize}
\begin{itemize}
\item {Grp. gram.:f.}
\end{itemize}
\begin{itemize}
\item {Utilização:Ant.}
\end{itemize}
\begin{itemize}
\item {Utilização:Prov.}
\end{itemize}
\begin{itemize}
\item {Utilização:beir.}
\end{itemize}
\begin{itemize}
\item {Proveniência:(T. afr.)}
\end{itemize}
Variedade de equídeo africano.
O mesmo que \textunderscore vaca\textunderscore ^1 ou \textunderscore vitella\textunderscore .
Pião comprido e mal feito.
\section{Zebrado}
\begin{itemize}
\item {Grp. gram.:m.}
\end{itemize}
\begin{itemize}
\item {Proveniência:(De \textunderscore zebrar\textunderscore )}
\end{itemize}
Listras, como as das zebras.
\section{Zebraínho}
\begin{itemize}
\item {Grp. gram.:m.}
\end{itemize}
Variedade de uva do Cartaxo, o mesmo que \textunderscore sobraínho\textunderscore .
\section{Zebral}
\begin{itemize}
\item {Grp. gram.:adj.}
\end{itemize}
\begin{itemize}
\item {Utilização:Ant.}
\end{itemize}
\begin{itemize}
\item {Proveniência:(De \textunderscore zebra\textunderscore )}
\end{itemize}
Relativo a zebra.
Dizia-se de uma pedra, que servia de pêso e equivalia a uma arroba.
\section{Zebrar}
\begin{itemize}
\item {Grp. gram.:v. t.}
\end{itemize}
Listrar, dando a apparência de pelle de zebra.
\section{Zebrário}
\begin{itemize}
\item {Grp. gram.:adj.}
\end{itemize}
\begin{itemize}
\item {Utilização:Ant.}
\end{itemize}
Relativo a zebra.
O mesmo que \textunderscore bovino\textunderscore .
\section{Zebrino}
\begin{itemize}
\item {Grp. gram.:adj.}
\end{itemize}
Relativo a zebra.
\section{Zebro}
\begin{itemize}
\item {fónica:zê}
\end{itemize}
\begin{itemize}
\item {Grp. gram.:m.}
\end{itemize}
\begin{itemize}
\item {Utilização:Ant.}
\end{itemize}
O mesmo que \textunderscore boi\textunderscore  ou \textunderscore novilho\textunderscore .
(Cp. \textunderscore zebra\textunderscore )
\section{Zebróide}
\begin{itemize}
\item {Grp. gram.:adj.}
\end{itemize}
\begin{itemize}
\item {Grp. gram.:M.}
\end{itemize}
\begin{itemize}
\item {Proveniência:(De \textunderscore zebra\textunderscore  + gr. \textunderscore eidos\textunderscore )}
\end{itemize}
Semelhante á zebra.
Animal, produzido pelo cruzamento de égua e zebra.--Há um exemplar, obtido pelo barão de Paraná, no Brasil. A \textunderscore Sociedade de aclimação\textunderscore  de Paris reconheceu o nome \textunderscore zebróide\textunderscore .
\section{Zebruno}
\begin{itemize}
\item {Grp. gram.:adj.}
\end{itemize}
O mesmo ou melhór que \textunderscore sebruno\textunderscore .
\section{Zécora}
\begin{itemize}
\item {Grp. gram.:f.}
\end{itemize}
Planta, o mesmo que \textunderscore onagra\textunderscore .
\section{Zé-cuécas}
\begin{itemize}
\item {Grp. gram.:m.}
\end{itemize}
\begin{itemize}
\item {Utilização:T. do Fundão}
\end{itemize}
Sujeito inútil, pacóvio, inhenho.
\section{Zé-da-véstia}
\begin{itemize}
\item {Grp. gram.:m.}
\end{itemize}
\begin{itemize}
\item {Utilização:Pop.}
\end{itemize}
O mesmo que \textunderscore zé-quitólis\textunderscore ; zé-dos-anzóes; jagodes, joão-ninguém.
\section{Zedoária}
\begin{itemize}
\item {Grp. gram.:f.}
\end{itemize}
\begin{itemize}
\item {Proveniência:(Do ár. \textunderscore geduaron\textunderscore , segundo Sousa, \textunderscore Vestig. da Ling. Aráb.\textunderscore )}
\end{itemize}
Planta herbácea e medicinal, da fam. das amomáceas.
\section{Zé-dos-anzóes}
\begin{itemize}
\item {Grp. gram.:m.}
\end{itemize}
\begin{itemize}
\item {Utilização:Pop.}
\end{itemize}
Qualquer sujeito.
Certo sujeito; fulano.
\section{Zeduária}
\begin{itemize}
\item {Grp. gram.:f.}
\end{itemize}
O mesmo ou melhór que \textunderscore zedoária\textunderscore .
\section{Zefirino}
\begin{itemize}
\item {Grp. gram.:adj.}
\end{itemize}
Relativo ao zéfiro.
\section{Zéfiro}
\begin{itemize}
\item {Grp. gram.:m.}
\end{itemize}
\begin{itemize}
\item {Utilização:Ant.}
\end{itemize}
\begin{itemize}
\item {Proveniência:(Lat. \textunderscore zephyrus\textunderscore )}
\end{itemize}
Vento suave e fresco; aragem.
Vento do Ocidente.
\section{Zé-godes}
\begin{itemize}
\item {Grp. gram.:m.}
\end{itemize}
\begin{itemize}
\item {Utilização:Prov.}
\end{itemize}
\begin{itemize}
\item {Utilização:beir.}
\end{itemize}
O mesmo que \textunderscore jagodes\textunderscore .
\section{Zé-goélas}
\begin{itemize}
\item {Grp. gram.:m.}
\end{itemize}
\begin{itemize}
\item {Utilização:Prov.}
\end{itemize}
Indivíduo palrador.
Aquelle que fala alto ou grita muito. (Colhido na Bairrada)
\section{Zegoniar}
\begin{itemize}
\item {Grp. gram.:v. i.}
\end{itemize}
\begin{itemize}
\item {Utilização:Ant.}
\end{itemize}
Accusar falsamente alguém de mancebia ou adultério.
\section{Zegulo}
\begin{itemize}
\item {Grp. gram.:m.}
\end{itemize}
\begin{itemize}
\item {Utilização:Ant.}
\end{itemize}
Homem amancebado.
\section{Zeimão}
\begin{itemize}
\item {Grp. gram.:m.  e  adj.}
\end{itemize}
\begin{itemize}
\item {Utilização:Prov.}
\end{itemize}
\begin{itemize}
\item {Utilização:minh.}
\end{itemize}
Homem sem préstimo.
\section{Zeísmo}
\begin{itemize}
\item {Grp. gram.:m.}
\end{itemize}
\begin{itemize}
\item {Utilização:Med.}
\end{itemize}
\begin{itemize}
\item {Proveniência:(De \textunderscore zéa\textunderscore )}
\end{itemize}
Doutrina dos que attribuem a pellagra ao uso do maís adulterado.
\section{Zelação}
\begin{itemize}
\item {Grp. gram.:f.}
\end{itemize}
\begin{itemize}
\item {Utilização:Bras}
\end{itemize}
Estrêlla cadente, bólide.
\section{Zelador}
\begin{itemize}
\item {Grp. gram.:m.  e  adj.}
\end{itemize}
\begin{itemize}
\item {Grp. gram.:M.}
\end{itemize}
O que zela.
Empregado fiscal de um município.
\section{Zelandês}
\begin{itemize}
\item {Grp. gram.:adj.}
\end{itemize}
\begin{itemize}
\item {Grp. gram.:M.}
\end{itemize}
Relativo á Zelândia.
Habitante da Zelândia. Cf. Ortigão, \textunderscore Holanda\textunderscore , 68.
\section{Zelante}
\begin{itemize}
\item {Grp. gram.:adj.}
\end{itemize}
Que zela.
\section{Zelar}
\begin{itemize}
\item {Grp. gram.:v. t.}
\end{itemize}
\begin{itemize}
\item {Grp. gram.:V. i.}
\end{itemize}
\begin{itemize}
\item {Proveniência:(Lat. \textunderscore zelare\textunderscore )}
\end{itemize}
Têr zêlo por.
Tratar com zêlo, com cuidado, com ciúme.
Administrar, diligentemente.
Têr ciúmes para com.
Têr zelos ou ciúmes.
\section{Zelha}
\begin{itemize}
\item {fónica:zê}
\end{itemize}
\begin{itemize}
\item {Grp. gram.:f.}
\end{itemize}
Planta acerácea, (\textunderscore acer monspessulanum\textunderscore , Lin.).
\section{Zêlo}
\begin{itemize}
\item {Grp. gram.:m.}
\end{itemize}
\begin{itemize}
\item {Proveniência:(Lat. \textunderscore zelus\textunderscore )}
\end{itemize}
Dedicação ardente.
Affeição íntima.
Desvelo; cuidado.
Pontualidade e diligência em qualquer serviço.
\section{Zelosamente}
\begin{itemize}
\item {Grp. gram.:adv.}
\end{itemize}
De modo zeloso; com cuidado; pontualmente.
\section{Zeloso}
\begin{itemize}
\item {Grp. gram.:adj.}
\end{itemize}
Que tem zelos.
Cuidadoso.
\section{Zelote}
\begin{itemize}
\item {Grp. gram.:adj.}
\end{itemize}
\begin{itemize}
\item {Utilização:Pop.}
\end{itemize}
\begin{itemize}
\item {Proveniência:(Lat. \textunderscore zelotes\textunderscore )}
\end{itemize}
Que finge têr zêlos.
\section{Zelotipia}
\begin{itemize}
\item {Grp. gram.:f.}
\end{itemize}
\begin{itemize}
\item {Utilização:Des.}
\end{itemize}
\begin{itemize}
\item {Proveniência:(Gr. \textunderscore zelotupia\textunderscore )}
\end{itemize}
Zelos; inveja.
\section{Zelotypia}
\begin{itemize}
\item {Grp. gram.:f.}
\end{itemize}
\begin{itemize}
\item {Utilização:Des.}
\end{itemize}
\begin{itemize}
\item {Proveniência:(Gr. \textunderscore zelotupia\textunderscore )}
\end{itemize}
Zelos; inveja.
\section{Zembro}
\begin{itemize}
\item {Grp. gram.:adj.}
\end{itemize}
\begin{itemize}
\item {Utilização:Prov.}
\end{itemize}
\begin{itemize}
\item {Utilização:trasm.}
\end{itemize}
O mesmo que \textunderscore zambro\textunderscore .
\section{Zend}
\begin{itemize}
\item {Grp. gram.:m.}
\end{itemize}
\begin{itemize}
\item {Grp. gram.:Adj.}
\end{itemize}
Explicação da religião de Zoroastro.
Língua, em que Zoroastro escreveu os seus livros, e que tem analogia com o idioma persa das inscripções cuneiformes.
Relativo ao zend.
\section{Zenda}
\begin{itemize}
\item {Grp. gram.:m.}
\end{itemize}
\begin{itemize}
\item {Grp. gram.:Adj.}
\end{itemize}
O mesmo ou melhór que \textunderscore zend\textunderscore .
Explicação da religião de Zoroastro.
Língua, em que Zoroastro escreveu os seus livros, e que tem analogia com o idioma persa das inscripções cuneiformes.
Relativo ao zend.
\section{Zend-avesta}
\begin{itemize}
\item {Grp. gram.:m.}
\end{itemize}
Conjunto dos livros sagrados dos Persas.
\section{Zendicismo}
\begin{itemize}
\item {Grp. gram.:m.}
\end{itemize}
\begin{itemize}
\item {Utilização:P. us.}
\end{itemize}
O systema religioso do Zend-Avesta ou dos livros sagrados dos Persas.
\section{Zenepro}
\begin{itemize}
\item {Grp. gram.:m.}
\end{itemize}
Planta da serra de Sintra.
\section{Zenéria}
\begin{itemize}
\item {Grp. gram.:f.}
\end{itemize}
Gênero de plantas cucurbitáceas.
\section{Zenetas}
\begin{itemize}
\item {Grp. gram.:m. pl.}
\end{itemize}
Uma das tríbos árabes, que invadiram a Espanha no séc. VIII. Cf. Herculano, \textunderscore Eurico\textunderscore , XII.
(Do ár.)
\section{Zenetense}
\begin{itemize}
\item {Grp. gram.:adj.}
\end{itemize}
Relativo aos Zenetas. Cf. Herculano, \textunderscore Hist. de Port.\textunderscore , I, 484.
\section{Zengue-zengue}
\begin{itemize}
\item {Grp. gram.:m.}
\end{itemize}
\begin{itemize}
\item {Proveniência:(T. afr.)}
\end{itemize}
Majestosa árvore africana.
\section{Zeniar}
\begin{itemize}
\item {Grp. gram.:m.}
\end{itemize}
\begin{itemize}
\item {Utilização:Ant.}
\end{itemize}
O mesmo que \textunderscore azinhavre\textunderscore .
\section{Zenir}
\begin{itemize}
\item {Grp. gram.:v. i.}
\end{itemize}
\begin{itemize}
\item {Utilização:Prov.}
\end{itemize}
\begin{itemize}
\item {Utilização:trasm.}
\end{itemize}
O mesmo que \textunderscore zunir\textunderscore .
Resumbrar, (falando-se da água).
\section{Zenital}
\begin{itemize}
\item {Grp. gram.:adj.}
\end{itemize}
Relativo ao zênite.
\section{Zênite}
\begin{itemize}
\item {Grp. gram.:m.}
\end{itemize}
\begin{itemize}
\item {Utilização:Fig.}
\end{itemize}
\begin{itemize}
\item {Proveniência:(Fr. \textunderscore zenith\textunderscore )}
\end{itemize}
Ponto da esfera celeste, que, relativamente a cada lugar da terra, é encontrado pela vertical que se levanta no mesmo lugar.
Auge; o ponto mais elevado.--\textunderscore Zeníte\textunderscore  é pronúncia usada, mas errónea.
\section{Zenithal}
\begin{itemize}
\item {Grp. gram.:adj.}
\end{itemize}
Relativo ao zênithe.
\section{Zênithe}
\begin{itemize}
\item {Grp. gram.:m.}
\end{itemize}
\begin{itemize}
\item {Utilização:Fig.}
\end{itemize}
\begin{itemize}
\item {Proveniência:(Fr. \textunderscore zenith\textunderscore )}
\end{itemize}
Ponto da esphera celeste, que, relativamente a cada lugar da terra, é encontrado pela vertical que se levanta no mesmo lugar.
Auge; o ponto mais elevado.--\textunderscore Zeníte\textunderscore  é pronúncia usada, mas errónea.
\section{Zenóbia}
\begin{itemize}
\item {Grp. gram.:f.}
\end{itemize}
Gênero de plantas ericíneas.
Gênero de crustáceos.
Gênero de borboletas.
\section{Zenónico}
\begin{itemize}
\item {Grp. gram.:adj.}
\end{itemize}
Relativo ao zenonismo.
\section{Zenonismo}
\begin{itemize}
\item {Grp. gram.:m.}
\end{itemize}
Systema do philósopho estoico Zenão.
\section{Zenonista}
\begin{itemize}
\item {Grp. gram.:m.}
\end{itemize}
Sectário do zenonismo.
\section{Zenzereiro}
\begin{itemize}
\item {Grp. gram.:m.}
\end{itemize}
\begin{itemize}
\item {Utilização:Ant.}
\end{itemize}
O mesmo que \textunderscore azereiro\textunderscore .
\section{Zêo}
\begin{itemize}
\item {Grp. gram.:m.}
\end{itemize}
\begin{itemize}
\item {Utilização:Ant.}
\end{itemize}
O mesmo que \textunderscore zêlo\textunderscore . Cf. Frei Fortun., \textunderscore Inéd.\textunderscore , III, 316.
\section{Zeófago}
\begin{itemize}
\item {Grp. gram.:adj.}
\end{itemize}
\begin{itemize}
\item {Proveniência:(De \textunderscore zéa\textunderscore  + gr. \textunderscore phagein\textunderscore )}
\end{itemize}
Que se alimenta com maís.
\section{Zeolíthico}
\begin{itemize}
\item {Grp. gram.:adj.}
\end{itemize}
Relativo ao zeólitho.
\section{Zeólitho}
\begin{itemize}
\item {Grp. gram.:m.}
\end{itemize}
\begin{itemize}
\item {Proveniência:(Do gr. \textunderscore zein\textunderscore  + \textunderscore lithos\textunderscore )}
\end{itemize}
Nome de várias substâncias pedregosas que, dissolvidas pelos ácidos, tomam consistência gelatinosa.
\section{Zeolítico}
\begin{itemize}
\item {Grp. gram.:adj.}
\end{itemize}
Relativo ao zeólito.
\section{Zeólito}
\begin{itemize}
\item {Grp. gram.:m.}
\end{itemize}
\begin{itemize}
\item {Proveniência:(Do gr. \textunderscore zein\textunderscore  + \textunderscore lithos\textunderscore )}
\end{itemize}
Nome de várias substâncias pedregosas que, dissolvidas pelos ácidos, tomam consistência gelatinosa.
\section{Zeóphago}
\begin{itemize}
\item {Grp. gram.:adj.}
\end{itemize}
\begin{itemize}
\item {Proveniência:(De \textunderscore zéa\textunderscore  + gr. \textunderscore phagein\textunderscore )}
\end{itemize}
Que se alimenta com maís.
\section{Zephyrino}
\begin{itemize}
\item {Grp. gram.:adj.}
\end{itemize}
Relativo ao zéphyro.
\section{Zéphyro}
\begin{itemize}
\item {Grp. gram.:m.}
\end{itemize}
\begin{itemize}
\item {Utilização:Ant.}
\end{itemize}
\begin{itemize}
\item {Proveniência:(Lat. \textunderscore zephyrus\textunderscore )}
\end{itemize}
Vento suave e fresco; aragem.
Vento do Occidente.
\section{Zé-preira}
\begin{itemize}
\item {Grp. gram.:m.}
\end{itemize}
\begin{itemize}
\item {Utilização:Pop.}
\end{itemize}
Tocador de tambor. Cf. Júl. Dinis, \textunderscore Morgadinha\textunderscore , 83; G. Braga, \textunderscore Mal da Delphina\textunderscore , 56.
\section{Zé-quitólis}
\begin{itemize}
\item {Grp. gram.:m.}
\end{itemize}
\begin{itemize}
\item {Utilização:Prov.}
\end{itemize}
\begin{itemize}
\item {Utilização:beir.}
\end{itemize}
Homem insignificante; bisbórria; joão-ninguém.
\section{Zequim}
\begin{itemize}
\item {Grp. gram.:m.}
\end{itemize}
\begin{itemize}
\item {Utilização:Ant.}
\end{itemize}
O mesmo ou melhór que \textunderscore sequim\textunderscore .
\section{Zerbo}
\begin{itemize}
\item {Grp. gram.:m.}
\end{itemize}
(V.zirbo)
\section{Zerê}
\begin{itemize}
\item {Grp. gram.:adj.}
\end{itemize}
\begin{itemize}
\item {Utilização:Bras}
\end{itemize}
Zarolho.
\section{Zerechia}
\begin{itemize}
\item {Grp. gram.:f.}
\end{itemize}
\begin{itemize}
\item {Utilização:Prov.}
\end{itemize}
\begin{itemize}
\item {Utilização:beir.}
\end{itemize}
\begin{itemize}
\item {Utilização:Prov.}
\end{itemize}
\begin{itemize}
\item {Utilização:trasm.}
\end{itemize}
\begin{itemize}
\item {Proveniência:(T. onom.)}
\end{itemize}
O zumbido, que acompanha o vôo rápido das abêlhas.
Chiada de rapazes; balbúrdia.
\section{Zerenamora}
\begin{itemize}
\item {Grp. gram.:f.}
\end{itemize}
\begin{itemize}
\item {Utilização:Prov.}
\end{itemize}
\begin{itemize}
\item {Utilização:trasm.}
\end{itemize}
O mesmo que \textunderscore bebedeira\textunderscore .
\section{Zeribanda}
\begin{itemize}
\item {Grp. gram.:f.}
\end{itemize}
(V.sarabanda)
\section{Zeribando}
\begin{itemize}
\item {Grp. gram.:m.}
\end{itemize}
\begin{itemize}
\item {Utilização:Ant.}
\end{itemize}
Azorrague.
(Cp. \textunderscore zeribanda\textunderscore )
\section{Zero}
\begin{itemize}
\item {Grp. gram.:m.}
\end{itemize}
\begin{itemize}
\item {Utilização:Ext.}
\end{itemize}
\begin{itemize}
\item {Utilização:Fig.}
\end{itemize}
\begin{itemize}
\item {Proveniência:(It. \textunderscore zero\textunderscore )}
\end{itemize}
Cifra.
Algarismo em fórma de 0, que por si não tem valor algum, mas que, á direita de outros números, faz que êstes tenham um valor déz vezes maior.
Nada.
Ponto, em que se começam a contar os graus.
Ponto que, nos thermómetros, corresponde á temperatura do gêlo que se derrete.
Pessôa ou coisa, sem valor.
\section{Zerumba}
\begin{itemize}
\item {Grp. gram.:f.}
\end{itemize}
Uma das drogarias que recebíamos da Índia, e que provavelmente é o mesmo que \textunderscore gengibre\textunderscore . Cf. \textunderscore Not. para a Hist. e Geogr.\textunderscore , II, 393.
\section{Zerumbete}
\begin{itemize}
\item {fónica:bê}
\end{itemize}
\begin{itemize}
\item {Grp. gram.:m.}
\end{itemize}
\begin{itemize}
\item {Proveniência:(De \textunderscore zerumba\textunderscore )}
\end{itemize}
Gengibre silvestre.
\section{Zerzulho}
\begin{itemize}
\item {Grp. gram.:m.}
\end{itemize}
\begin{itemize}
\item {Utilização:Prov.}
\end{itemize}
\begin{itemize}
\item {Utilização:trasm.}
\end{itemize}
O mesmo que \textunderscore dinheiro\textunderscore .
\section{Zesto}
\begin{itemize}
\item {Grp. gram.:m.}
\end{itemize}
\begin{itemize}
\item {Proveniência:(Fr. \textunderscore zeste\textunderscore )}
\end{itemize}
A camada mais externa do limão. Cf. Ed. Magalhães, \textunderscore Hyg. Alimentar\textunderscore , I, 393.
\section{Zeta}
\begin{itemize}
\item {Grp. gram.:m.}
\end{itemize}
Nome da letra, que no alphabeto grego corresponde a \textunderscore z\textunderscore .
\section{Zetacismo}
\begin{itemize}
\item {Grp. gram.:m.}
\end{itemize}
\begin{itemize}
\item {Proveniência:(De \textunderscore zeta\textunderscore )}
\end{itemize}
Vício na pronúncia do \textunderscore z\textunderscore  ou \textunderscore s.\textunderscore 
\section{Zetética}
\begin{itemize}
\item {Grp. gram.:f.}
\end{itemize}
\begin{itemize}
\item {Proveniência:(De \textunderscore zetético\textunderscore )}
\end{itemize}
Méthodo de investigação, ou conjunto de preceitos, para a resolução de um problema.
\section{Zetético}
\begin{itemize}
\item {Grp. gram.:adj.}
\end{itemize}
\begin{itemize}
\item {Proveniência:(Gr. \textunderscore zetétikos\textunderscore )}
\end{itemize}
Relativo á zetética ou a investigações.
\section{Zeugma}
\begin{itemize}
\item {Grp. gram.:f.}
\end{itemize}
\begin{itemize}
\item {Utilização:Rhet.}
\end{itemize}
\begin{itemize}
\item {Proveniência:(Lat. \textunderscore zeugma\textunderscore )}
\end{itemize}
Figura de elocução, pela qual uma palavra, já expressa numa proposição, é subentendida em outra proposição, que com a primeira tem analogia ou relação.
\section{Zeugo}
\begin{itemize}
\item {Grp. gram.:m.}
\end{itemize}
Instrumento músico dos Gregos, formado de duas frautas reunidas.
\section{Zèzé}
\begin{itemize}
\item {Grp. gram.:m.}
\end{itemize}
O mesmo que \textunderscore tsetsé\textunderscore .
\section{Zezereiro}
\begin{itemize}
\item {Grp. gram.:m.}
\end{itemize}
O mesmo que \textunderscore zenzereiro\textunderscore . Cf. \textunderscore Port. Ant. e Mod.\textunderscore , vb. \textunderscore Preces\textunderscore .
(Provavelmente, corr. de \textunderscore azereiro\textunderscore , sob infl. de \textunderscore Zézere\textunderscore , n. p.)
\section{Zibelina}
\begin{itemize}
\item {Grp. gram.:adj.}
\end{itemize}
\begin{itemize}
\item {Grp. gram.:F.}
\end{itemize}
\begin{itemize}
\item {Proveniência:(It. \textunderscore zibellino\textunderscore )}
\end{itemize}
Diz-se de uma variedade de marta, do Norte da Ásia e da África.
Marta zibelina.
\section{Zibellina}
\begin{itemize}
\item {Grp. gram.:adj.}
\end{itemize}
\begin{itemize}
\item {Grp. gram.:F.}
\end{itemize}
\begin{itemize}
\item {Proveniência:(It. \textunderscore zibellino\textunderscore )}
\end{itemize}
Diz-se de uma variedade de marta, do Norte da Ásia e da África.
Marta zibellina.
\section{Zibeta}
\begin{itemize}
\item {Grp. gram.:m.}
\end{itemize}
\begin{itemize}
\item {Proveniência:(Do ár. \textunderscore zabád\textunderscore )}
\end{itemize}
Espécie de furão asiático.
\section{Zicha}
\begin{itemize}
\item {Grp. gram.:f.}
\end{itemize}
\begin{itemize}
\item {Utilização:Prov.}
\end{itemize}
\begin{itemize}
\item {Utilização:trasm.}
\end{itemize}
Belga comprida e estreita.
\section{Zichar}
\begin{itemize}
\item {Grp. gram.:V. i.}
\end{itemize}
\begin{itemize}
\item {Utilização:Prov.}
\end{itemize}
\begin{itemize}
\item {Utilização:trasm.}
\end{itemize}
Sair em borbotões (a água).
(Talvez corr. de \textunderscore esguichar\textunderscore )
\section{Zicho}
\begin{itemize}
\item {Grp. gram.:m.}
\end{itemize}
Acto de zichar; esguicho.
\section{Zígnia}
\begin{itemize}
\item {Grp. gram.:f.}
\end{itemize}
Planta decorativa, procedente da África.
\section{Ziguezague}
\begin{itemize}
\item {Grp. gram.:m.}
\end{itemize}
\begin{itemize}
\item {Proveniência:(Fr. \textunderscore zigzag\textunderscore )}
\end{itemize}
Série de linhas, que forma ângulos alternadamente salientes e reintrantes.
Modo de andar, descrevendo essa série de linhas.
Sinuosidade.
Trincheira, que forma voltas ou ângulos alternados, para que os sitiantes de uma praça não possam sêr batidos pelos sitiados.
Ornato, em fórma de ziguezague.
\section{Ziguezaguear}
\begin{itemize}
\item {Grp. gram.:v. i.}
\end{itemize}
\begin{itemize}
\item {Utilização:Neol.}
\end{itemize}
Fazer ziguezagues.
Andar, formando ziguezagues.
\section{Ziguezigue}
\begin{itemize}
\item {Grp. gram.:m.}
\end{itemize}
\begin{itemize}
\item {Utilização:Fig.}
\end{itemize}
\begin{itemize}
\item {Proveniência:(T. onom., com fórma correspondente no ár. e no persa)}
\end{itemize}
Brinquedo infantil, espécie de cègarrega.
Traquinas.
\section{Zíleas}
\begin{itemize}
\item {Grp. gram.:f. pl.}
\end{itemize}
Tríbo de plantas crucíferas, no sistema de De-Candolle.
\section{Zílleas}
\begin{itemize}
\item {Grp. gram.:f. pl.}
\end{itemize}
Tríbo de plantas crucíferas, no systema de De-Candolle.
\section{Zimbar}
\begin{itemize}
\item {Grp. gram.:v. i.}
\end{itemize}
\begin{itemize}
\item {Utilização:Ant.}
\end{itemize}
O mesmo que \textunderscore zumbar\textunderscore ^3.
\section{Zimbas}
\begin{itemize}
\item {Grp. gram.:m. pl.}
\end{itemize}
Antigo povo cafreal. Cf. Couto, \textunderscore Déc.\textunderscore 
\section{Zimbo}
\begin{itemize}
\item {Grp. gram.:m.}
\end{itemize}
Concha univalve que, entre os Congueses, se usa como moéda.
O mesmo que \textunderscore lumache\textunderscore .
\section{Zimboque}
\begin{itemize}
\item {Grp. gram.:m.}
\end{itemize}
\begin{itemize}
\item {Utilização:Prov.}
\end{itemize}
Chavelha do carro. (Colhido em Manhouce)
\section{Zimbório}
\begin{itemize}
\item {Grp. gram.:m.}
\end{itemize}
Parte mais alta e exterior da cúpula de um edifício.
(Cast. ant. \textunderscore cimbório\textunderscore )
\section{Zimbrada}
\begin{itemize}
\item {Grp. gram.:f.}
\end{itemize}
Acto de zimbrar.
\section{Zimbral}
\begin{itemize}
\item {Grp. gram.:m.}
\end{itemize}
Terreno, onde crescem zimbros.
\section{Zimbrão}
\begin{itemize}
\item {Grp. gram.:m.}
\end{itemize}
\begin{itemize}
\item {Proveniência:(De \textunderscore zimbro\textunderscore )}
\end{itemize}
Árvore de Santiago de Cabo-Verde, (\textunderscore juníperus communis\textunderscore ).
\section{Zimbrar}
\begin{itemize}
\item {Grp. gram.:v. t.}
\end{itemize}
\begin{itemize}
\item {Utilização:Prov.}
\end{itemize}
\begin{itemize}
\item {Utilização:trasm.}
\end{itemize}
\begin{itemize}
\item {Grp. gram.:V. i.}
\end{itemize}
Açoitar; vergastar.
Pôr bordões estirados e retesados sôbre a pelle de (um tambor), para requintar o som.
Baloiçar, arfar, da popa á prôa, (falando-se do navio).
(Cp. \textunderscore azumbrar\textunderscore  e \textunderscore zumbrir\textunderscore )
\section{Zimbreiro}
\begin{itemize}
\item {Grp. gram.:m.}
\end{itemize}
Arbusto, o mesmo que \textunderscore zimbro\textunderscore ^2. Cf. \textunderscore Port. Ant. e Mod.\textunderscore , XII, 2248.
\section{Zimbro}
\begin{itemize}
\item {Grp. gram.:m.}
\end{itemize}
Orvalho, cacimba.
\section{Zimbro}
\begin{itemize}
\item {Grp. gram.:m.}
\end{itemize}
O mesmo que \textunderscore junípero\textunderscore .
O mesmo que \textunderscore genebra\textunderscore ^1. Cf. Filinto, VIII, 253.
\section{Zina}
\begin{itemize}
\item {Grp. gram.:f.}
\end{itemize}
\begin{itemize}
\item {Utilização:Prov.}
\end{itemize}
\begin{itemize}
\item {Utilização:beir.}
\end{itemize}
\begin{itemize}
\item {Utilização:T. da Bairrada}
\end{itemize}
Auge; pino.
O maior grau de intensidade.
Raiva, fúria.
Tineta; mania: \textunderscore deu-lhe para alli na zina\textunderscore .
\section{Zinabre}
\begin{itemize}
\item {Grp. gram.:m.}
\end{itemize}
O mesmo que \textunderscore azinhavre\textunderscore .
\section{Zincagem}
\begin{itemize}
\item {Grp. gram.:f.}
\end{itemize}
Acto ou effeito de \textunderscore zincar\textunderscore .
\section{Zincar}
\begin{itemize}
\item {Grp. gram.:v. t.}
\end{itemize}
Revestir de zinco.
\section{Zíncico}
\begin{itemize}
\item {Grp. gram.:adj.}
\end{itemize}
\begin{itemize}
\item {Utilização:Chím.}
\end{itemize}
\begin{itemize}
\item {Proveniência:(De \textunderscore zinco\textunderscore )}
\end{itemize}
Que contém zinco.
Diz-se da combinação do zinco com o oxygênio, e diz-se dos sáes, formados pelo óxydo de zinco.
\section{Zinco}
\begin{itemize}
\item {Grp. gram.:m.}
\end{itemize}
\begin{itemize}
\item {Proveniência:(Fr. \textunderscore zinc\textunderscore )}
\end{itemize}
Corpo simples metállico, de uma brancura azulada, e muito usado nas indústrias.
\section{Zincografar}
\begin{itemize}
\item {Grp. gram.:v. t.}
\end{itemize}
\begin{itemize}
\item {Proveniência:(De \textunderscore zincógrafo\textunderscore )}
\end{itemize}
Gravar ou imprimir em lâminas de zinco.
\section{Zincografia}
\begin{itemize}
\item {Grp. gram.:f.}
\end{itemize}
\begin{itemize}
\item {Proveniência:(De \textunderscore zincógrafo\textunderscore )}
\end{itemize}
Arte de zincografar.
\section{Zincográfico}
\begin{itemize}
\item {Grp. gram.:adj.}
\end{itemize}
Relativo á zincografia.
\section{Zincógrafo}
\begin{itemize}
\item {Grp. gram.:m.}
\end{itemize}
\begin{itemize}
\item {Proveniência:(De \textunderscore zinco\textunderscore  + gr. \textunderscore graphein\textunderscore )}
\end{itemize}
Aquele que zincografa.
\section{Zincographar}
\begin{itemize}
\item {Grp. gram.:v. t.}
\end{itemize}
\begin{itemize}
\item {Proveniência:(De \textunderscore zincógrapho\textunderscore )}
\end{itemize}
Gravar ou imprimir em lâminas de zinco.
\section{Zincographia}
\begin{itemize}
\item {Grp. gram.:f.}
\end{itemize}
\begin{itemize}
\item {Proveniência:(De \textunderscore zincógrapho\textunderscore )}
\end{itemize}
Arte de zincographar.
\section{Zincográphico}
\begin{itemize}
\item {Grp. gram.:adj.}
\end{itemize}
Relativo á zincographia.
\section{Zincógrapho}
\begin{itemize}
\item {Grp. gram.:m.}
\end{itemize}
\begin{itemize}
\item {Proveniência:(De \textunderscore zinco\textunderscore  + gr. \textunderscore graphein\textunderscore )}
\end{itemize}
Aquelle que zincographa.
\section{Zineto}
\begin{itemize}
\item {fónica:nê}
\end{itemize}
\begin{itemize}
\item {Grp. gram.:adj.}
\end{itemize}
\begin{itemize}
\item {Utilização:T. de Turquel}
\end{itemize}
Um tanto ébrio; tocado da pinga.
\section{Zinga}
\begin{itemize}
\item {Grp. gram.:f.}
\end{itemize}
\begin{itemize}
\item {Utilização:Bras}
\end{itemize}
Vara comprida, de que se servem os canoeiros, para vencer a fôrça da corrente, quando não basta a acção dos remos.
(Corr. de \textunderscore ginga\textunderscore . V. \textunderscore ginga\textunderscore )
\section{Zingador}
\begin{itemize}
\item {Grp. gram.:m.}
\end{itemize}
\begin{itemize}
\item {Utilização:Bras}
\end{itemize}
Aquelle que zinga.
\section{Zingamocho}
\begin{itemize}
\item {fónica:mô}
\end{itemize}
\begin{itemize}
\item {Grp. gram.:m.}
\end{itemize}
\begin{itemize}
\item {Utilização:T. de Amarante}
\end{itemize}
Catavento.
Remate de um zimbório; pináculo.
Espécie de boiz.
\section{Zingar}
\begin{itemize}
\item {Grp. gram.:v. i.}
\end{itemize}
\begin{itemize}
\item {Utilização:Bras}
\end{itemize}
Manejar a zinga.(V.gingar)
\section{Zingarear}
\begin{itemize}
\item {Grp. gram.:v. i.}
\end{itemize}
Vadiar.
(Cp. \textunderscore zangurrar\textunderscore )
\section{Zíngaro}
\begin{itemize}
\item {Grp. gram.:m.}
\end{itemize}
\begin{itemize}
\item {Proveniência:(It. \textunderscore zingaro\textunderscore )}
\end{itemize}
Um dos nomes, por que se designam os ciganos.
\section{Zingiberáceas}
\begin{itemize}
\item {Grp. gram.:f. pl.}
\end{itemize}
O mesmo ou melhór que \textunderscore gengiberáceas\textunderscore .
\section{Zingração}
\begin{itemize}
\item {Grp. gram.:f.}
\end{itemize}
Acto de zingrar. Cf. \textunderscore Anat. Joc.\textunderscore , II, 458.
\section{Zingrar}
\begin{itemize}
\item {Grp. gram.:v. t.}
\end{itemize}
\begin{itemize}
\item {Grp. gram.:V. i.}
\end{itemize}
Motejar de; burlar.
Dizer motejos.
Não dar importância:«\textunderscore quando eu era amante, perneava padecente; agora que já zingro das finezas...\textunderscore »\textunderscore Anat. Joc.\textunderscore  I, 195.
\section{Zinguerrear}
\begin{itemize}
\item {Grp. gram.:v. i.}
\end{itemize}
\begin{itemize}
\item {Utilização:Prov.}
\end{itemize}
\begin{itemize}
\item {Utilização:trasm.}
\end{itemize}
Emittir um som particular, como coisa que se mova num eixo muito froixo.
(Cp. \textunderscore zangarrear\textunderscore )
\section{Zinideira}
\begin{itemize}
\item {Grp. gram.:f.}
\end{itemize}
\begin{itemize}
\item {Utilização:Prov.}
\end{itemize}
\begin{itemize}
\item {Utilização:trasm.}
\end{itemize}
Pedaço de vêrga, aguçado numa das extremidades e preso pela outra a um pau, que os rapazes agitam, para o fazerem zinir.
\section{Zinir}
\begin{itemize}
\item {Grp. gram.:v. i.}
\end{itemize}
O mesmo que \textunderscore zunir\textunderscore .
\section{Zínnia}
\begin{itemize}
\item {Grp. gram.:f.}
\end{itemize}
\begin{itemize}
\item {Proveniência:(De \textunderscore Zinn\textunderscore , n. p.)}
\end{itemize}
Gênero de plantas synanthéreas.
\section{Zinote}
\begin{itemize}
\item {Grp. gram.:m.}
\end{itemize}
\begin{itemize}
\item {Utilização:Prov.}
\end{itemize}
\begin{itemize}
\item {Utilização:trasm.}
\end{itemize}
O mesmo que [[nádegas|nádega]].
\section{Zirbeiro}
\begin{itemize}
\item {Grp. gram.:m.}
\end{itemize}
\begin{itemize}
\item {Utilização:Prov.}
\end{itemize}
\begin{itemize}
\item {Utilização:trasm.}
\end{itemize}
Local, onde se arma o ichós.
\section{Zirbo}
\begin{itemize}
\item {Grp. gram.:m.}
\end{itemize}
\begin{itemize}
\item {Proveniência:(It. \textunderscore zirbo\textunderscore )}
\end{itemize}
O mesmo que \textunderscore epíploon\textunderscore , ou \textunderscore redenho\textunderscore .
\section{Zircão}
\begin{itemize}
\item {Grp. gram.:m.}
\end{itemize}
Silicato de zircónio.
(Cast. \textunderscore zircón\textunderscore )
\section{Zircónico}
\begin{itemize}
\item {Grp. gram.:adj.}
\end{itemize}
Diz-se do óxydo de zircónio, e dos saes formados por êsse óxydo.
\section{Zircónio}
\begin{itemize}
\item {Grp. gram.:m.}
\end{itemize}
Metal escuro, que não tem aspecto metállico, em quanto se não fricciona.
\section{Zirigaita}
\begin{itemize}
\item {Grp. gram.:f.}
\end{itemize}
(V.sirigaita)
\section{Zirrar}
\begin{itemize}
\item {Grp. gram.:v. i.}
\end{itemize}
\begin{itemize}
\item {Utilização:Prov.}
\end{itemize}
\begin{itemize}
\item {Utilização:trasm.}
\end{itemize}
Fazer zirra-zirra.
\section{Zirra-zirra!}
\begin{itemize}
\item {Grp. gram.:interj.}
\end{itemize}
\begin{itemize}
\item {Utilização:Prov.}
\end{itemize}
\begin{itemize}
\item {Utilização:trasm.}
\end{itemize}
(Serve para as mondadeiras fazerem troça de algum rapaz, que lhes passe perto do campo da monda)
\section{Zirro}
\begin{itemize}
\item {Grp. gram.:m.}
\end{itemize}
O mesmo que \textunderscore gaivão\textunderscore ^1.
\section{Zizânia}
\begin{itemize}
\item {Grp. gram.:f.}
\end{itemize}
O mesmo que \textunderscore cizânia\textunderscore .
\section{Zízia}
\begin{itemize}
\item {Grp. gram.:f.}
\end{itemize}
\begin{itemize}
\item {Proveniência:(De \textunderscore Zizii\textunderscore , n. p.)}
\end{itemize}
Gênero de plantas umbellíferas.
\section{Zizífico}
\begin{itemize}
\item {Grp. gram.:adj.}
\end{itemize}
\begin{itemize}
\item {Utilização:Chím.}
\end{itemize}
\begin{itemize}
\item {Proveniência:(De \textunderscore zízifo\textunderscore )}
\end{itemize}
Diz-se de um ácido, extraido da açofeifa.
\section{Zízifo}
\begin{itemize}
\item {Grp. gram.:m.}
\end{itemize}
\begin{itemize}
\item {Proveniência:(Lat. \textunderscore zizyphum\textunderscore )}
\end{itemize}
Gênero de plantas ramnáceas, o mesmo que \textunderscore açofeifeira\textunderscore .
\section{Zizýphico}
\begin{itemize}
\item {Grp. gram.:adj.}
\end{itemize}
\begin{itemize}
\item {Utilização:Chím.}
\end{itemize}
\begin{itemize}
\item {Proveniência:(De \textunderscore zízypho\textunderscore )}
\end{itemize}
Diz-se de um ácido, extrahido da açofeifa.
\section{Zízypho}
\begin{itemize}
\item {Grp. gram.:m.}
\end{itemize}
\begin{itemize}
\item {Proveniência:(Lat. \textunderscore zizyphum\textunderscore )}
\end{itemize}
Gênero de plantas rhamnáceas, o mesmo que \textunderscore açofeifeira\textunderscore .
\section{Zoada}
\begin{itemize}
\item {Grp. gram.:f.}
\end{itemize}
Acto ou effeito de zoar.
Zumbido; zunido.
\section{Zoante}
\begin{itemize}
\item {Grp. gram.:adj.}
\end{itemize}
\begin{itemize}
\item {Utilização:Gram.}
\end{itemize}
Que zôa.
Diz-se de certas letras invogaes, que se pronunciam zoando, como \textunderscore j\textunderscore ,\textunderscore v\textunderscore  e \textunderscore z\textunderscore 
\section{Zoantários}
\begin{itemize}
\item {Grp. gram.:m. pl.}
\end{itemize}
\begin{itemize}
\item {Proveniência:(De \textunderscore zoanto\textunderscore )}
\end{itemize}
Animaes da classe dos pólipos.
\section{Zoanthários}
\begin{itemize}
\item {Grp. gram.:m. pl.}
\end{itemize}
\begin{itemize}
\item {Proveniência:(De \textunderscore zoantho\textunderscore )}
\end{itemize}
Animaes da classe dos pólypos.
\section{Zoantho}
\begin{itemize}
\item {Grp. gram.:m.}
\end{itemize}
\begin{itemize}
\item {Proveniência:(Do gr. \textunderscore zoon\textunderscore  + \textunderscore anthos\textunderscore )}
\end{itemize}
Gênero de pólypos carnudos, sem invólucro sólido.
\section{Zoanthropia}
\begin{itemize}
\item {Grp. gram.:f.}
\end{itemize}
\begin{itemize}
\item {Proveniência:(De \textunderscore zoanthropo\textunderscore )}
\end{itemize}
Doença mental, em que o enfermo se julga transformado num animal.
\section{Zoanthropo}
\begin{itemize}
\item {Grp. gram.:m.}
\end{itemize}
\begin{itemize}
\item {Proveniência:(Do gr. \textunderscore zoon\textunderscore  + \textunderscore anthropos\textunderscore )}
\end{itemize}
Indivíduo, atacado de zoanthropia.
\section{Zoanto}
\begin{itemize}
\item {Grp. gram.:m.}
\end{itemize}
\begin{itemize}
\item {Proveniência:(Do gr. \textunderscore zoon\textunderscore  + \textunderscore anthos\textunderscore )}
\end{itemize}
Gênero de pólipos carnudos, sem invólucro sólido.
\section{Zoantropia}
\begin{itemize}
\item {Grp. gram.:f.}
\end{itemize}
\begin{itemize}
\item {Proveniência:(De \textunderscore zoantropo\textunderscore )}
\end{itemize}
Doença mental, em que o enfermo se julga transformado num animal.
\section{Zoantropo}
\begin{itemize}
\item {Grp. gram.:m.}
\end{itemize}
\begin{itemize}
\item {Proveniência:(Do gr. \textunderscore zoon\textunderscore  + \textunderscore anthropos\textunderscore )}
\end{itemize}
Indivíduo, atacado de zoantropia.
\section{Zoar}
\begin{itemize}
\item {Grp. gram.:v. i.}
\end{itemize}
Têr som forte e confuso.
Zunir.
(Alter. de \textunderscore soar\textunderscore )
\section{Zoarco}
\begin{itemize}
\item {Grp. gram.:m.}
\end{itemize}
Gênero de peixes acanthopterýgios.
\section{Zodiacal}
\begin{itemize}
\item {Grp. gram.:adj.}
\end{itemize}
Relativo ao zodíaco.
\section{Zodíaco}
\begin{itemize}
\item {Grp. gram.:m.}
\end{itemize}
\begin{itemize}
\item {Proveniência:(Lat. \textunderscore zodiacus\textunderscore )}
\end{itemize}
Zona da esphera celeste, que envolve á eclíptica, e contém as doze constellações, que o Sol parece percorrer no espaço de um anno.
\section{Zoecía}
\begin{itemize}
\item {Grp. gram.:f.}
\end{itemize}
\begin{itemize}
\item {Utilização:Hist. Nat.}
\end{itemize}
\begin{itemize}
\item {Proveniência:(Do gr. \textunderscore zoon\textunderscore  + \textunderscore oikos\textunderscore )}
\end{itemize}
Habitação de pólypos.
\section{Zoeira}
\begin{itemize}
\item {Grp. gram.:f.}
\end{itemize}
\begin{itemize}
\item {Utilização:Prov.}
\end{itemize}
\begin{itemize}
\item {Utilização:Prov.}
\end{itemize}
\begin{itemize}
\item {Utilização:minh.}
\end{itemize}
O mesmo que \textunderscore zoada\textunderscore .
Valentia.
\section{Zoélas}
\begin{itemize}
\item {Grp. gram.:m. pl.}
\end{itemize}
\begin{itemize}
\item {Utilização:Ant.}
\end{itemize}
Povos antigos da comarca de Bragança.
\section{Zoga}
\begin{itemize}
\item {Grp. gram.:f.}
\end{itemize}
\begin{itemize}
\item {Utilização:Prov.}
\end{itemize}
\begin{itemize}
\item {Utilização:trasm.}
\end{itemize}
Pau de urze, com sua raíz.
\section{Zògada}
\begin{itemize}
\item {Grp. gram.:f.}
\end{itemize}
\begin{itemize}
\item {Utilização:Prov.}
\end{itemize}
\begin{itemize}
\item {Utilização:trasm.}
\end{itemize}
Pancada com zoga.
\section{Zóides}
\begin{itemize}
\item {Grp. gram.:m. pl.}
\end{itemize}
\begin{itemize}
\item {Proveniência:(Do gr. \textunderscore zoon\textunderscore  + \textunderscore eidos\textunderscore )}
\end{itemize}
Designação, dada por alguns naturalistas aos seres inferiores da escala animal; protozoários.
\section{Zoilo}
\begin{itemize}
\item {Grp. gram.:m.}
\end{itemize}
\begin{itemize}
\item {Proveniência:(De \textunderscore Zoilo\textunderscore , n. p.)}
\end{itemize}
Mau crítico; crítico invejoso.
\section{Zóina}
\begin{itemize}
\item {Grp. gram.:adj.}
\end{itemize}
\begin{itemize}
\item {Grp. gram.:F.}
\end{itemize}
\begin{itemize}
\item {Utilização:Prov.}
\end{itemize}
\begin{itemize}
\item {Utilização:minh.}
\end{itemize}
\begin{itemize}
\item {Proveniência:(Do ár. \textunderscore zania\textunderscore )}
\end{itemize}
Azoinado; estonteado.
Mulhér mal comportada; prostituta.
\section{Zoiodina}
\begin{itemize}
\item {Grp. gram.:f.}
\end{itemize}
\begin{itemize}
\item {Utilização:Chím.}
\end{itemize}
\begin{itemize}
\item {Proveniência:(Do gr. \textunderscore zoon\textunderscore  + \textunderscore iodes\textunderscore )}
\end{itemize}
Producto azotado, de bella côr de violeta.
\section{Zoipeira}
\begin{itemize}
\item {Grp. gram.:f.}
\end{itemize}
\begin{itemize}
\item {Utilização:Prov.}
\end{itemize}
\begin{itemize}
\item {Utilização:trasm.}
\end{itemize}
Mulhér gorda, desajeitada e suja.
(Cp. \textunderscore zoupeiro\textunderscore )
\section{Zoísmo}
\begin{itemize}
\item {Grp. gram.:m.}
\end{itemize}
\begin{itemize}
\item {Proveniência:(Do gr. \textunderscore zoon\textunderscore )}
\end{itemize}
Conjunto dos phenómenos da vida animal.
\section{Zola}
\begin{itemize}
\item {Grp. gram.:f.}
\end{itemize}
\begin{itemize}
\item {Utilização:Prov.}
\end{itemize}
\begin{itemize}
\item {Utilização:beir.}
\end{itemize}
O leite, que as crianças mamam.
Acto de mamar.
\section{Zolulo}
\begin{itemize}
\item {Grp. gram.:m.}
\end{itemize}
Árvore do Congo.
\section{Zombadeira}
\begin{itemize}
\item {Grp. gram.:f.  e  adj.}
\end{itemize}
\begin{itemize}
\item {Proveniência:(De \textunderscore zombar\textunderscore )}
\end{itemize}
Mulhér zombeteira.
\section{Zombado}
\begin{itemize}
\item {Grp. gram.:adj.}
\end{itemize}
\begin{itemize}
\item {Proveniência:(De \textunderscore zombar\textunderscore )}
\end{itemize}
Que foi escarnecido; de quem se zombou.
\section{Zombador}
\begin{itemize}
\item {Grp. gram.:m.  e  adj.}
\end{itemize}
O que zomba.
\section{Zombal}
\begin{itemize}
\item {Grp. gram.:m.  e  adj.}
\end{itemize}
\begin{itemize}
\item {Utilização:Ant.}
\end{itemize}
O mesmo que \textunderscore zombador\textunderscore . Cf. Castro, \textunderscore Paráphrase\textunderscore , 20, v.^o.
\section{Zombar}
\begin{itemize}
\item {Grp. gram.:v. i.}
\end{itemize}
\begin{itemize}
\item {Grp. gram.:Loc.}
\end{itemize}
\begin{itemize}
\item {Utilização:Loc. de Turquel.}
\end{itemize}
\begin{itemize}
\item {Grp. gram.:V. t.}
\end{itemize}
Escarnecer; mofar.
Ridiculizar; fazer chacota.
Tratar com ludíbrio ou vilipêndio.
\textunderscore Zombar de uma mulhér\textunderscore , seduzi-la.
Fazer zombaria de. Cf. Filinto, I, 105.
(Cast. \textunderscore zumbar\textunderscore )
\section{Zombaria}
\begin{itemize}
\item {Grp. gram.:f.}
\end{itemize}
Acto ou effeito de zombar.
\section{Zombatório}
\begin{itemize}
\item {Grp. gram.:adj.}
\end{itemize}
\begin{itemize}
\item {Proveniência:(De \textunderscore zombar\textunderscore )}
\end{itemize}
Relativo a zombaria.
Que envolve zombaria. Cf. Filinto, XII, 248.
\section{Zomba-zombando}
\begin{itemize}
\item {Grp. gram.:loc. adv.}
\end{itemize}
\begin{itemize}
\item {Proveniência:(De \textunderscore zombar\textunderscore )}
\end{itemize}
Por zombaria.
Por chalaça.
Como quem não quere.
A pouco e pouco.
\section{Zombeirão}
\begin{itemize}
\item {Grp. gram.:m.  e  adj.}
\end{itemize}
\begin{itemize}
\item {Utilização:Pop.}
\end{itemize}
O mesmo que \textunderscore zombador\textunderscore .
\section{Zombeiro}
\begin{itemize}
\item {Grp. gram.:m.  e  adj.}
\end{itemize}
\begin{itemize}
\item {Utilização:Ant.}
\end{itemize}
O que zomba ou escarnece.
\section{Zombetear}
\begin{itemize}
\item {Grp. gram.:v. i.}
\end{itemize}
O mesmo que \textunderscore zombar\textunderscore .
\section{Zombeteiro}
\begin{itemize}
\item {Grp. gram.:m.  e  adj.}
\end{itemize}
O mesmo que \textunderscore zombador\textunderscore .
\section{Zomol}
\begin{itemize}
\item {Grp. gram.:m.}
\end{itemize}
\begin{itemize}
\item {Proveniência:(Do gr. \textunderscore zomos\textunderscore )}
\end{itemize}
Suco de carne dessecada.
\section{Zomoterapia}
\begin{itemize}
\item {Grp. gram.:f.}
\end{itemize}
\begin{itemize}
\item {Proveniência:(Do gr. \textunderscore zomos\textunderscore  + \textunderscore therapeia\textunderscore )}
\end{itemize}
Tratamento terapêutico pela ingestão de carne crua.
\section{Zomotherapia}
\begin{itemize}
\item {Grp. gram.:f.}
\end{itemize}
\begin{itemize}
\item {Proveniência:(Do gr. \textunderscore zomos\textunderscore  + \textunderscore therapeia\textunderscore )}
\end{itemize}
Tratamento therapêutico pela ingestão de carne crua.
\section{Zom-zom}
\begin{itemize}
\item {Grp. gram.:m.}
\end{itemize}
\begin{itemize}
\item {Proveniência:(T. onom.)}
\end{itemize}
Som confuso e monótono, como o rasgado da viola.
\section{Zona}
\begin{itemize}
\item {Grp. gram.:f.}
\end{itemize}
\begin{itemize}
\item {Utilização:Med.}
\end{itemize}
\begin{itemize}
\item {Proveniência:(Lat. \textunderscore zona\textunderscore )}
\end{itemize}
Cinta, faixa.
Cada uma das cinco grandes divisões da esphera terrestre, que se suppõem separadas entre si por círculos parallelos ao equador.
Cada uma das partes da esphera celeste, correspondente a cada uma daquellas cinco divisões.
Qualquer região, considerada relativamente á sua temperatura.
Parte da superfície de uma esphera, entre dois planos parallelos.
Espaço de terreno ou região, caracterizado por circunstâncias particulares.
Região.
Malha, que cerca uma parte ou um órgão de um animal.
Inflammação cutânea, com erupção vesicular, rodeando o peito ou o abdome.
\section{Zona}
\begin{itemize}
\item {Grp. gram.:f.}
\end{itemize}
\begin{itemize}
\item {Utilização:Gír.}
\end{itemize}
Noite.
(Cp. \textunderscore sona\textunderscore  a \textunderscore sorna\textunderscore ^2)
\section{Zonada}
\begin{itemize}
\item {Grp. gram.:f.}
\end{itemize}
Espécie de forragem.
\section{Zonado}
\begin{itemize}
\item {Grp. gram.:adj.}
\end{itemize}
\begin{itemize}
\item {Proveniência:(De \textunderscore zona\textunderscore ^1)}
\end{itemize}
Marcado ou assinalado com listras ou vergões coloridos e concêntricos.
\section{Zonchadura}
\begin{itemize}
\item {Grp. gram.:f.}
\end{itemize}
Acto ou effeito de zonchar.
\section{Zonchar}
\begin{itemize}
\item {Grp. gram.:v. i.}
\end{itemize}
\begin{itemize}
\item {Proveniência:(De \textunderscore zoncho\textunderscore )}
\end{itemize}
Dar á bomba.
\section{Zoncho}
\begin{itemize}
\item {Grp. gram.:m.}
\end{itemize}
Alavanca, com que se faz mover o êmbolo da bomba.
(Cast. \textunderscore suncho\textunderscore )
\section{Zoníptilo}
\begin{itemize}
\item {Grp. gram.:f.}
\end{itemize}
Gênero de insectos coleópteros pentâmeros.
\section{Zonote}
\begin{itemize}
\item {Grp. gram.:m.}
\end{itemize}
Cisterna vasta e funda, que os habitantes de Iucatão abrem e revestem de ladrilho, para recolher e conservar a água da chuva contra o tempo da sêca. Cf. \textunderscore Jorn.-do-Comm.\textunderscore , do Rio, de 16-IX-900.
\section{Zontró}
\begin{itemize}
\item {Grp. gram.:m.}
\end{itemize}
Apparelho que, na Índia, serve para a destillação da sura, e é composto de duas peças de barro, que se communicam por um tubo de bambu. Cf. L. Mendes, \textunderscore Ind. Port.\textunderscore 
(Do conc.)
\section{Zonuro}
\begin{itemize}
\item {Grp. gram.:m.}
\end{itemize}
Gênero de reptís sáurios.
\section{Zonýptilo}
\begin{itemize}
\item {Grp. gram.:f.}
\end{itemize}
Gênero de insectos coleópteros pentâmeros.
\section{Zonzo}
\begin{itemize}
\item {Grp. gram.:adj.}
\end{itemize}
\begin{itemize}
\item {Utilização:Bras}
\end{itemize}
\begin{itemize}
\item {Proveniência:(T. cast.)}
\end{itemize}
Tonto; estonteado.
\section{Zonzonar}
\begin{itemize}
\item {Grp. gram.:v. i.}
\end{itemize}
Soar monotonamente, como o zom-zom da viola. Cf. A. Pimentel, \textunderscore As Al. Canções\textunderscore , 208.
(Cp. \textunderscore zom-zom\textunderscore )
\section{Zoo...}
\begin{itemize}
\item {Grp. gram.:pref.}
\end{itemize}
\begin{itemize}
\item {Proveniência:(Do gr. \textunderscore zoon\textunderscore )}
\end{itemize}
(designativo de \textunderscore animal\textunderscore )
\section{Zoobia}
\begin{itemize}
\item {Grp. gram.:f.}
\end{itemize}
\begin{itemize}
\item {Utilização:Hist. Nat.}
\end{itemize}
\begin{itemize}
\item {Proveniência:(Do gr. \textunderscore zoon\textunderscore  + \textunderscore bios\textunderscore )}
\end{itemize}
Sciência da vida.
Funccionamento dos órgãos, de que resulta a conservação do sêr animado.
\section{Zoóbio}
\begin{itemize}
\item {Grp. gram.:adj.}
\end{itemize}
\begin{itemize}
\item {Proveniência:(Do gr. \textunderscore zoon\textunderscore  + \textunderscore bios\textunderscore )}
\end{itemize}
Que vive dentro do corpo dos animaes; entozoário.
\section{Zoobiologia}
\begin{itemize}
\item {Grp. gram.:f.}
\end{itemize}
\begin{itemize}
\item {Proveniência:(Do gr. \textunderscore zoon\textunderscore  + \textunderscore bios\textunderscore  + \textunderscore logos\textunderscore )}
\end{itemize}
Sciência da vida animal.
\section{Zoocarpo}
\begin{itemize}
\item {Grp. gram.:m.}
\end{itemize}
\begin{itemize}
\item {Proveniência:(Do gr. \textunderscore zoon\textunderscore  + \textunderscore karpos\textunderscore )}
\end{itemize}
Nome, que se deu aos corpos, que depois se chamaram zoósporos.
V. \textunderscore zoósporo\textunderscore .
\section{Zoochímica}
\begin{itemize}
\item {fónica:qui}
\end{itemize}
\begin{itemize}
\item {Grp. gram.:f.}
\end{itemize}
Chímica animal.
\section{Zoochímico}
\begin{itemize}
\item {fónica:qui}
\end{itemize}
\begin{itemize}
\item {Grp. gram.:adj.}
\end{itemize}
Relativo á zoochímica.
\section{Zoochorographia}
\begin{itemize}
\item {fónica:co}
\end{itemize}
\begin{itemize}
\item {Grp. gram.:f.}
\end{itemize}
\begin{itemize}
\item {Proveniência:(De \textunderscore zoo...\textunderscore  + \textunderscore chorographia\textunderscore )}
\end{itemize}
Descripção dos animaes de determinada região.
\section{Zoochorographico}
\begin{itemize}
\item {fónica:co}
\end{itemize}
\begin{itemize}
\item {Grp. gram.:adj.}
\end{itemize}
Relativo á zoochorographia.
\section{Zoococcina}
\begin{itemize}
\item {Grp. gram.:f.}
\end{itemize}
Substância animal, que se encontra no quermes vegetal.
\section{Zoococina}
\begin{itemize}
\item {Grp. gram.:f.}
\end{itemize}
Substância animal, que se encontra no quermes vegetal.
\section{Zoocorografia}
\begin{itemize}
\item {Grp. gram.:f.}
\end{itemize}
\begin{itemize}
\item {Proveniência:(De \textunderscore zoo...\textunderscore  + \textunderscore corografia\textunderscore )}
\end{itemize}
Descripção dos animaes de determinada região.
\section{Zoocorografico}
\begin{itemize}
\item {Grp. gram.:adj.}
\end{itemize}
Relativo á zoocorografia.
\section{Zooematina}
\begin{itemize}
\item {fónica:o-e}
\end{itemize}
\begin{itemize}
\item {Grp. gram.:f.}
\end{itemize}
\begin{itemize}
\item {Utilização:Chím.}
\end{itemize}
\begin{itemize}
\item {Proveniência:(Do gr. \textunderscore zoon\textunderscore  + \textunderscore haima\textunderscore )}
\end{itemize}
Princípio còrante do sangue.
\section{Zooéthica}
\begin{itemize}
\item {Grp. gram.:f.}
\end{itemize}
\begin{itemize}
\item {Proveniência:(Do gr. \textunderscore zoon\textunderscore  + \textunderscore ethos\textunderscore )}
\end{itemize}
Tratado á cêrca dos costumes dos animaes.
\section{Zooética}
\begin{itemize}
\item {Grp. gram.:f.}
\end{itemize}
\begin{itemize}
\item {Proveniência:(Do gr. \textunderscore zoon\textunderscore  + \textunderscore ethos\textunderscore )}
\end{itemize}
Tratado á cêrca dos costumes dos animaes.
\section{Zoofagia}
\begin{itemize}
\item {Grp. gram.:f.}
\end{itemize}
\begin{itemize}
\item {Proveniência:(De \textunderscore zoófago\textunderscore )}
\end{itemize}
Qualidade de zoófago.
Voracidade, que incita os animaes a devorar a presa, antes de morta.
\section{Zoofágico}
\begin{itemize}
\item {Grp. gram.:adj.}
\end{itemize}
Relativo á zoofagia.
\section{Zoófago}
\begin{itemize}
\item {Grp. gram.:adj.}
\end{itemize}
\begin{itemize}
\item {Proveniência:(Do gr. \textunderscore zoon\textunderscore  + \textunderscore phagein\textunderscore )}
\end{itemize}
Diz-se dos animaes, que se alimentam de outros.
\section{Zoófilo}
\begin{itemize}
\item {Grp. gram.:adj.}
\end{itemize}
\begin{itemize}
\item {Grp. gram.:M.}
\end{itemize}
\begin{itemize}
\item {Proveniência:(Do gr. \textunderscore zoon\textunderscore  + \textunderscore philos\textunderscore )}
\end{itemize}
Que gosta de animaes.
Amigo dos animaes.
\section{Zoofitantráceo}
\begin{itemize}
\item {Grp. gram.:adj.}
\end{itemize}
\begin{itemize}
\item {Utilização:Miner.}
\end{itemize}
\begin{itemize}
\item {Proveniência:(Do gr. \textunderscore zoon\textunderscore  + \textunderscore phuton\textunderscore  + \textunderscore anthrax\textunderscore )}
\end{itemize}
Diz-se do carvão mineral, produzido pelos restos de animaes e vegetaes.
\section{Zoofitário}
\begin{itemize}
\item {Grp. gram.:adj.}
\end{itemize}
\begin{itemize}
\item {Grp. gram.:M. pl.}
\end{itemize}
\begin{itemize}
\item {Utilização:Hist. Nat.}
\end{itemize}
\begin{itemize}
\item {Proveniência:(De \textunderscore zoófito\textunderscore )}
\end{itemize}
Relativo aos zoófitos.
Animaes radiados, que têm todas as faculdades da animalidade, mas que estão reunidos entre si por uma parte viva que lhes é comum e se multiplicam como as plantas.
\section{Zoofítico}
\begin{itemize}
\item {Grp. gram.:adj.}
\end{itemize}
Relativo a zoófitos; que contém zoófitos.
\section{Zoófito}
\begin{itemize}
\item {Grp. gram.:m.}
\end{itemize}
\begin{itemize}
\item {Proveniência:(Do gr. \textunderscore zoon\textunderscore  + \textunderscore phuton\textunderscore )}
\end{itemize}
Nome, por que se designavam os animaes ínfimos da escala zoológica.
\section{Zoofitografia}
\begin{itemize}
\item {Grp. gram.:f.}
\end{itemize}
\begin{itemize}
\item {Proveniência:(Do gr. \textunderscore zoon\textunderscore  + \textunderscore phuton\textunderscore  + \textunderscore graphein\textunderscore )}
\end{itemize}
Parte da História Natural, em que se descrevem os zoófitos.
\section{Zoofitográfico}
\begin{itemize}
\item {Grp. gram.:adj.}
\end{itemize}
Relativo á zoofitografia.
\section{Zoofitógrafo}
\begin{itemize}
\item {Grp. gram.:m.}
\end{itemize}
Naturalista, que escreveu á cêrca da zoofitografia.
\section{Zoofitólito}
\begin{itemize}
\item {Grp. gram.:m.}
\end{itemize}
\begin{itemize}
\item {Proveniência:(Do gr. \textunderscore zoon\textunderscore  + \textunderscore phuton\textunderscore  + \textunderscore lithos\textunderscore )}
\end{itemize}
Zoófito fóssil.
\section{Zoofitologia}
\begin{itemize}
\item {Grp. gram.:f.}
\end{itemize}
Estudo científico dos zoófitos.
(Cp. \textunderscore zoofitólogo\textunderscore )
\section{Zoofitológico}
\begin{itemize}
\item {Grp. gram.:adj.}
\end{itemize}
Relativo á zoofitologia.
\section{Zoofitólogo}
\begin{itemize}
\item {Grp. gram.:m.}
\end{itemize}
\begin{itemize}
\item {Proveniência:(Do gr. \textunderscore zoon\textunderscore  + \textunderscore phutos\textunderscore  + \textunderscore logos\textunderscore )}
\end{itemize}
Naturalista, que é perito em zoofitologia.
\section{Zoofobia}
\begin{itemize}
\item {Grp. gram.:f.}
\end{itemize}
\begin{itemize}
\item {Proveniência:(Do gr. \textunderscore zoon\textunderscore  + \textunderscore phobein\textunderscore )}
\end{itemize}
Mêdo mórbido de qualquer animal.
\section{Zoófobo}
\begin{itemize}
\item {Grp. gram.:m.}
\end{itemize}
Aquele que tem zoofobia.
\section{Zoofórico}
\begin{itemize}
\item {Grp. gram.:adj.}
\end{itemize}
Relativo a zoóforo.
\section{Zoóforo}
\begin{itemize}
\item {Grp. gram.:m.}
\end{itemize}
\begin{itemize}
\item {Utilização:Archit.}
\end{itemize}
\begin{itemize}
\item {Proveniência:(Lat. \textunderscore zoophorus\textunderscore )}
\end{itemize}
Espaço, entre arquitrave e cornija, ornado antigamente com cabeças de animaes.
\section{Zoogenia}
\begin{itemize}
\item {Grp. gram.:f.}
\end{itemize}
\begin{itemize}
\item {Proveniência:(Do gr. \textunderscore zoon\textunderscore  + \textunderscore genos\textunderscore )}
\end{itemize}
Formação ou geração dos animaes.
\section{Zoogênico}
\begin{itemize}
\item {Grp. gram.:adj.}
\end{itemize}
Relativo á zoogenía.
\section{Zoogênio}
\begin{itemize}
\item {Grp. gram.:m.}
\end{itemize}
\begin{itemize}
\item {Proveniência:(Do gr. \textunderscore zoon\textunderscore  + \textunderscore genos\textunderscore )}
\end{itemize}
Substância viscosa, que se encontra nas águas thermaes.
\section{Zoogeografia}
\begin{itemize}
\item {Grp. gram.:f.}
\end{itemize}
\begin{itemize}
\item {Proveniência:(Do gr. \textunderscore zoon\textunderscore  + \textunderscore ge\textunderscore  + \textunderscore graphein\textunderscore )}
\end{itemize}
Descripção da fauna de certas regiões.
\section{Zoogeográfico}
\begin{itemize}
\item {Grp. gram.:adj.}
\end{itemize}
Relativo á zoogeografia.
\section{Zoogeographia}
\begin{itemize}
\item {Grp. gram.:f.}
\end{itemize}
\begin{itemize}
\item {Proveniência:(Do gr. \textunderscore zoon\textunderscore  + \textunderscore ge\textunderscore  + \textunderscore graphein\textunderscore )}
\end{itemize}
Descripção da fauna de certas regiões.
\section{Zoogeográphico}
\begin{itemize}
\item {Grp. gram.:adj.}
\end{itemize}
Relativo á zoogeographia.
\section{Zoogléa}
\begin{itemize}
\item {Grp. gram.:f.}
\end{itemize}
\begin{itemize}
\item {Utilização:Med.}
\end{itemize}
\begin{itemize}
\item {Proveniência:(Do gr. \textunderscore zoon\textunderscore  + \textunderscore gloia\textunderscore )}
\end{itemize}
Massa, formada por colónias de micro-organismos.
\section{Zoogleia}
\begin{itemize}
\item {Grp. gram.:f.}
\end{itemize}
\begin{itemize}
\item {Utilização:Med.}
\end{itemize}
\begin{itemize}
\item {Proveniência:(Do gr. \textunderscore zoon\textunderscore  + \textunderscore gloia\textunderscore )}
\end{itemize}
Massa, formada por colónias de micro-organismos.
\section{Zooglifito}
\begin{itemize}
\item {Grp. gram.:m.}
\end{itemize}
\begin{itemize}
\item {Utilização:Miner.}
\end{itemize}
\begin{itemize}
\item {Proveniência:(Do gr. \textunderscore zoon\textunderscore  + \textunderscore gluphein\textunderscore )}
\end{itemize}
Pedra, que apresenta vestígios de animaes.
\section{Zooglyphito}
\begin{itemize}
\item {Grp. gram.:m.}
\end{itemize}
\begin{itemize}
\item {Utilização:Miner.}
\end{itemize}
\begin{itemize}
\item {Proveniência:(Do gr. \textunderscore zoon\textunderscore  + \textunderscore gluphein\textunderscore )}
\end{itemize}
Pedra, que apresenta vestígios de animaes.
\section{Zoografar}
\begin{itemize}
\item {Grp. gram.:v. t.}
\end{itemize}
\begin{itemize}
\item {Proveniência:(De \textunderscore zoógrafo\textunderscore )}
\end{itemize}
Descrever ou desenhar (animaes).
\section{Zoografia}
\begin{itemize}
\item {Grp. gram.:f.}
\end{itemize}
\begin{itemize}
\item {Proveniência:(De \textunderscore zoógrafo\textunderscore )}
\end{itemize}
Descripção, desenho ou pintura de animaes.
Arte de zoografar.
\section{Zoográfico}
\begin{itemize}
\item {Grp. gram.:adj.}
\end{itemize}
Relativo á zoografia.
\section{Zoógrafo}
\begin{itemize}
\item {Grp. gram.:m.}
\end{itemize}
\begin{itemize}
\item {Proveniência:(Do gr. \textunderscore zoon\textunderscore  + \textunderscore graphein\textunderscore )}
\end{itemize}
Aquele que pinta ou descreve animaes.
\section{Zoographar}
\begin{itemize}
\item {Grp. gram.:v. t.}
\end{itemize}
\begin{itemize}
\item {Proveniência:(De \textunderscore zoógrapho\textunderscore )}
\end{itemize}
Descrever ou desenhar (animaes).
\section{Zoographia}
\begin{itemize}
\item {Grp. gram.:f.}
\end{itemize}
\begin{itemize}
\item {Proveniência:(De \textunderscore zoógrapho\textunderscore )}
\end{itemize}
Descripção, desenho ou pintura de animaes.
Arte de zoographar.
\section{Zoográphico}
\begin{itemize}
\item {Grp. gram.:adj.}
\end{itemize}
Relativo á zoographia.
\section{Zoógrapho}
\begin{itemize}
\item {Grp. gram.:m.}
\end{itemize}
\begin{itemize}
\item {Proveniência:(Do gr. \textunderscore zoon\textunderscore  + \textunderscore graphein\textunderscore )}
\end{itemize}
Aquelle que pinta ou descreve animaes.
\section{Zooiatra}
\begin{itemize}
\item {fónica:o-i}
\end{itemize}
\begin{itemize}
\item {Grp. gram.:m.}
\end{itemize}
Aquelle que exerce a zooiatria.
\section{Zooiatria}
\begin{itemize}
\item {fónica:o-i}
\end{itemize}
\begin{itemize}
\item {Grp. gram.:f.}
\end{itemize}
\begin{itemize}
\item {Proveniência:(Do gr. \textunderscore zoon\textunderscore  + \textunderscore iatros\textunderscore )}
\end{itemize}
Medicina veterinária.
\section{Zooiátrico}
\begin{itemize}
\item {Grp. gram.:adj.}
\end{itemize}
Relativo á zooiatria.
\section{Zooiatrologia}
\begin{itemize}
\item {fónica:o-i}
\end{itemize}
\begin{itemize}
\item {Grp. gram.:f.}
\end{itemize}
\begin{itemize}
\item {Proveniência:(Do gr. \textunderscore zoon\textunderscore  + \textunderscore iatros\textunderscore  + \textunderscore logos\textunderscore )}
\end{itemize}
Sciência médica veterinária.
Tratado á cêrca desta sciência.
\section{Zooiatrológico}
\begin{itemize}
\item {Grp. gram.:adj.}
\end{itemize}
Relativo á zooiatrologia.
\section{Zoóide}
\begin{itemize}
\item {Grp. gram.:adj.}
\end{itemize}
\begin{itemize}
\item {Grp. gram.:M. pl.}
\end{itemize}
\begin{itemize}
\item {Proveniência:(Do gr. \textunderscore zoon\textunderscore  + \textunderscore eidos\textunderscore )}
\end{itemize}
Que tem aspecto de animal, ou de uma parte de um animal.
O mesmo ou melhor que \textunderscore zóides\textunderscore .
\section{Zoólatra}
\begin{itemize}
\item {Grp. gram.:m. ,  f.  e  adj.}
\end{itemize}
Pessôa, que adora os animaes.
(Cp. \textunderscore zoolatria\textunderscore )
\section{Zoolatria}
\begin{itemize}
\item {Grp. gram.:f.}
\end{itemize}
\begin{itemize}
\item {Proveniência:(Do gr. \textunderscore zoon\textunderscore  + \textunderscore latreia\textunderscore )}
\end{itemize}
Adoração dos animaes.
\section{Zoolátrico}
\begin{itemize}
\item {Grp. gram.:adj.}
\end{itemize}
Relativo á zoolatria.
\section{Zoólea}
\begin{itemize}
\item {Grp. gram.:f.}
\end{itemize}
Gênero de insectos orthópteros.
\section{Zoolíthico}
\begin{itemize}
\item {Grp. gram.:adj.}
\end{itemize}
Relativo ao zoólitho; que tem zoólithos.
\section{Zoolithífero}
\begin{itemize}
\item {Grp. gram.:adj.}
\end{itemize}
\begin{itemize}
\item {Proveniência:(Do gr. \textunderscore zoon\textunderscore  + \textunderscore lithos\textunderscore  +lat. \textunderscore ferre\textunderscore )}
\end{itemize}
Diz-se do mineral, que contém restos de animaes fósseis.
\section{Zoólitho}
\begin{itemize}
\item {Grp. gram.:m.}
\end{itemize}
\begin{itemize}
\item {Proveniência:(Do gr. \textunderscore zoon\textunderscore  + \textunderscore lithos\textunderscore )}
\end{itemize}
Animal fóssil.
Parte de um animal petrificado.
\section{Zoolítico}
\begin{itemize}
\item {Grp. gram.:adj.}
\end{itemize}
Relativo ao zoólito; que tem zoólitos.
\section{Zoolitífero}
\begin{itemize}
\item {Grp. gram.:adj.}
\end{itemize}
\begin{itemize}
\item {Proveniência:(Do gr. \textunderscore zoon\textunderscore  + \textunderscore lithos\textunderscore  + lat. \textunderscore ferre\textunderscore )}
\end{itemize}
Diz-se do mineral, que contém restos de animaes fósseis.
\section{Zoólito}
\begin{itemize}
\item {Grp. gram.:m.}
\end{itemize}
\begin{itemize}
\item {Proveniência:(Do gr. \textunderscore zoon\textunderscore  + \textunderscore lithos\textunderscore )}
\end{itemize}
Animal fóssil.
Parte de um animal petrificado.
\section{Zoologia}
\begin{itemize}
\item {Grp. gram.:f.}
\end{itemize}
\begin{itemize}
\item {Proveniência:(De \textunderscore zoólogo\textunderscore )}
\end{itemize}
Tratado á cêrca dos animaes.
Sciência, que se occupa dos seres animados.
\section{Zoologicamente}
\begin{itemize}
\item {Grp. gram.:adj.}
\end{itemize}
De modo zoológico; segundo a Zoologia.
\section{Zoológico}
\begin{itemize}
\item {Grp. gram.:adj.}
\end{itemize}
Relativo á Zoologia.
\section{Zoologista}
\begin{itemize}
\item {Grp. gram.:m.}
\end{itemize}
Tratadista de Zoologia.
\section{Zoólogo}
\begin{itemize}
\item {Grp. gram.:m.}
\end{itemize}
\begin{itemize}
\item {Proveniência:(Do gr. \textunderscore zoon\textunderscore  + \textunderscore logos\textunderscore )}
\end{itemize}
Indivíduo, versado em Zoologia.
\section{Zoomagnético}
\begin{itemize}
\item {Grp. gram.:adj.}
\end{itemize}
Relativo ao zoomagnetismo.
\section{Zoomagnetismo}
\begin{itemize}
\item {Grp. gram.:m.}
\end{itemize}
\begin{itemize}
\item {Proveniência:(De \textunderscore zoo...\textunderscore  + \textunderscore magnetismo\textunderscore )}
\end{itemize}
Magnetismo animal.
\section{Zoomorfia}
\begin{itemize}
\item {Grp. gram.:f.}
\end{itemize}
\begin{itemize}
\item {Proveniência:(Do gr. \textunderscore zoon\textunderscore  + \textunderscore morphe\textunderscore )}
\end{itemize}
Representação dos animaes vivos, que habitam nas conchas.
Descripção da parte externa dos animaes.
\section{Zoomorfismo}
\begin{itemize}
\item {Grp. gram.:m.}
\end{itemize}
Culto religioso, que dá ás divindades a fórma de animaes.
Crença, em que os homens se podem transformar em animaes como a crença dos lobishomens, etc.
(Cp. \textunderscore zoomorfia\textunderscore )
\section{Zoomorfita}
\begin{itemize}
\item {Grp. gram.:f.}
\end{itemize}
O mesmo que \textunderscore zoomorfito\textunderscore .
\section{Zoomorfito}
\begin{itemize}
\item {Grp. gram.:m.}
\end{itemize}
O mesmo que \textunderscore zoólito\textunderscore .
\section{Zoomorfose}
\begin{itemize}
\item {Grp. gram.:f.}
\end{itemize}
História dos animaes que vivem em conchas.
(Cp. \textunderscore zoomorfia\textunderscore )
\section{Zoomorphia}
\begin{itemize}
\item {Grp. gram.:f.}
\end{itemize}
\begin{itemize}
\item {Proveniência:(Do gr. \textunderscore zoon\textunderscore  + \textunderscore morphe\textunderscore )}
\end{itemize}
Representação dos animaes vivos, que habitam nas conchas.
Descripção da parte externa dos animaes.
\section{Zoomorphismo}
\begin{itemize}
\item {Grp. gram.:m.}
\end{itemize}
Culto religioso, que dá ás divindades a fórma de animaes.
Crença, em que os homens se podem transformar em animaes como a crença dos lobishomens, etc.
(Cp. \textunderscore zoomorphia\textunderscore )
\section{Zoomorphita}
\begin{itemize}
\item {Grp. gram.:f.}
\end{itemize}
O mesmo que \textunderscore zoomorphito\textunderscore .
\section{Zoomorphito}
\begin{itemize}
\item {Grp. gram.:m.}
\end{itemize}
O mesmo que \textunderscore zoólitho\textunderscore .
\section{Zoomorphose}
\begin{itemize}
\item {Grp. gram.:f.}
\end{itemize}
História dos animaes que vivem em conchas.
(Cp. \textunderscore zoomorphia\textunderscore )
\section{Zoonato}
\begin{itemize}
\item {Grp. gram.:m.}
\end{itemize}
\begin{itemize}
\item {Utilização:Chím.}
\end{itemize}
Sal, resultante da combinação do ácido zoónico com uma base.
(Cp. \textunderscore zoónico\textunderscore )
\section{Zoónico}
\begin{itemize}
\item {Grp. gram.:adj.}
\end{itemize}
\begin{itemize}
\item {Proveniência:(Do gr. \textunderscore zoon\textunderscore )}
\end{itemize}
Diz-se de um ácido, extrahido de substâncias animaes.
\section{Zoonita}
\begin{itemize}
\item {Grp. gram.:f.}
\end{itemize}
(V.zoonito)
\section{Zoonitado}
\begin{itemize}
\item {Grp. gram.:adj.}
\end{itemize}
\begin{itemize}
\item {Utilização:Zool.}
\end{itemize}
\begin{itemize}
\item {Proveniência:(De \textunderscore zoonito\textunderscore )}
\end{itemize}
Diz-se dos animaes articulados, vermes e echinodermes.
\section{Zoonito}
\begin{itemize}
\item {Grp. gram.:m.}
\end{itemize}
\begin{itemize}
\item {Utilização:Hist. Nat.}
\end{itemize}
\begin{itemize}
\item {Proveniência:(Do gr. \textunderscore zoon\textunderscore )}
\end{itemize}
Cada um dos seres parciaes, que constituem um animal composto.
\section{Zoonomia}
\begin{itemize}
\item {Grp. gram.:f.}
\end{itemize}
\begin{itemize}
\item {Proveniência:(Do gr. \textunderscore zoon\textunderscore  + \textunderscore nomos\textunderscore )}
\end{itemize}
Conjunto das leis orgânicas dos animaes.
\section{Zoonómico}
\begin{itemize}
\item {Grp. gram.:adj.}
\end{itemize}
Relativo á zoonomia.
\section{Zoonose}
\begin{itemize}
\item {Grp. gram.:f.}
\end{itemize}
\begin{itemize}
\item {Proveniência:(Do gr. \textunderscore zoon\textunderscore )}
\end{itemize}
Designação genérica das doenças dos animaes.
\section{Zoonosologia}
\begin{itemize}
\item {Grp. gram.:f.}
\end{itemize}
\begin{itemize}
\item {Proveniência:(De \textunderscore zoo...\textunderscore  + \textunderscore nosologia\textunderscore )}
\end{itemize}
Conhecimento das doenças dos animaes.
\section{Zoonosológico}
\begin{itemize}
\item {Grp. gram.:adj.}
\end{itemize}
Relativo á zoonosologia.
\section{Zooparasito}
\begin{itemize}
\item {Grp. gram.:m.}
\end{itemize}
\begin{itemize}
\item {Proveniência:(Do gr. \textunderscore zoon\textunderscore  + \textunderscore parasitos\textunderscore )}
\end{itemize}
Parasito dos animaes.
\section{Zoopathologia}
\begin{itemize}
\item {Grp. gram.:f.}
\end{itemize}
\begin{itemize}
\item {Proveniência:(Do gr. \textunderscore zoon\textunderscore  + \textunderscore pathos\textunderscore )}
\end{itemize}
O mesmo que \textunderscore zoonosologia\textunderscore .
\section{Zoopatologia}
\begin{itemize}
\item {Grp. gram.:f.}
\end{itemize}
\begin{itemize}
\item {Proveniência:(Do gr. \textunderscore zoon\textunderscore  + \textunderscore pathos\textunderscore )}
\end{itemize}
O mesmo que \textunderscore zoonosologia\textunderscore .
\section{Zoophagia}
\begin{itemize}
\item {Grp. gram.:f.}
\end{itemize}
\begin{itemize}
\item {Proveniência:(De \textunderscore zoóphago\textunderscore )}
\end{itemize}
Qualidade de zoóphago.
Voracidade, que incita os animaes a devorar a presa, antes de morta.
\section{Zoophágico}
\begin{itemize}
\item {Grp. gram.:adj.}
\end{itemize}
Relativo á zoophagia.
\section{Zoóphago}
\begin{itemize}
\item {Grp. gram.:adj.}
\end{itemize}
\begin{itemize}
\item {Proveniência:(Do gr. \textunderscore zoon\textunderscore  + \textunderscore phagein\textunderscore )}
\end{itemize}
Diz-se dos animaes, que se alimentam de outros.
\section{Zoóphilo}
\begin{itemize}
\item {Grp. gram.:adj.}
\end{itemize}
\begin{itemize}
\item {Grp. gram.:M.}
\end{itemize}
\begin{itemize}
\item {Proveniência:(Do gr. \textunderscore zoon\textunderscore  + \textunderscore philos\textunderscore )}
\end{itemize}
Que gosta de animaes.
Amigo dos animaes.
\section{Zoophobia}
\begin{itemize}
\item {Grp. gram.:f.}
\end{itemize}
\begin{itemize}
\item {Proveniência:(Do gr. \textunderscore zoon\textunderscore  + \textunderscore phobein\textunderscore )}
\end{itemize}
Mêdo mórbido de qualquer animal.
\section{Zoóphobo}
\begin{itemize}
\item {Grp. gram.:m.}
\end{itemize}
Aquelle que tem zoophobia.
\section{Zoophórico}
\begin{itemize}
\item {Grp. gram.:adj.}
\end{itemize}
Relativo a zoóphoro.
\section{Zoóphoro}
\begin{itemize}
\item {Grp. gram.:m.}
\end{itemize}
\begin{itemize}
\item {Utilização:Archit.}
\end{itemize}
\begin{itemize}
\item {Proveniência:(Lat. \textunderscore zoophorus\textunderscore )}
\end{itemize}
Espaço, entre architrave e cornija, ornado antigamente com cabeças de animaes.
\section{Zoophytanthráceo}
\begin{itemize}
\item {Grp. gram.:adj.}
\end{itemize}
\begin{itemize}
\item {Utilização:Miner.}
\end{itemize}
\begin{itemize}
\item {Proveniência:(Do gr. \textunderscore zoon\textunderscore  + \textunderscore phuton\textunderscore  + \textunderscore anthrax\textunderscore )}
\end{itemize}
Diz-se do carvão mineral, produzido pelos restos de animaes e vegetaes.
\section{Zoophytário}
\begin{itemize}
\item {Grp. gram.:adj.}
\end{itemize}
\begin{itemize}
\item {Grp. gram.:M. pl.}
\end{itemize}
\begin{itemize}
\item {Utilização:Hist. Nat.}
\end{itemize}
\begin{itemize}
\item {Proveniência:(De \textunderscore zoóphyto\textunderscore )}
\end{itemize}
Relativo aos zoóphitos.
Animaes radiados, que têm todas as faculdades da animalidade, mas que estão reunidos entre si por uma parte viva que lhes é commum e se multiplicam como as plantas.
\section{Zoophýtico}
\begin{itemize}
\item {Grp. gram.:adj.}
\end{itemize}
Relativo a zoóphytos; que contém zoóphytos.
\section{Zoóphyto}
\begin{itemize}
\item {Grp. gram.:m.}
\end{itemize}
\begin{itemize}
\item {Proveniência:(Do gr. \textunderscore zoon\textunderscore  + \textunderscore phuton\textunderscore )}
\end{itemize}
Nome, por que se designavam os animaes ínfimos da escala zoológica.
\section{Zoophytographia}
\begin{itemize}
\item {Grp. gram.:f.}
\end{itemize}
\begin{itemize}
\item {Proveniência:(Do gr. \textunderscore zoon\textunderscore  + \textunderscore phuton\textunderscore  + \textunderscore graphein\textunderscore )}
\end{itemize}
Parte da História Natural, em que se descrevem os zoóphytos.
\section{Zoophytográphico}
\begin{itemize}
\item {Grp. gram.:adj.}
\end{itemize}
Relativo á zoophytographia.
\section{Zoophytógrapho}
\begin{itemize}
\item {Grp. gram.:m.}
\end{itemize}
Naturalista, que escreveu á cêrca da zoophytographia.
\section{Zoophytólitho}
\begin{itemize}
\item {Grp. gram.:m.}
\end{itemize}
\begin{itemize}
\item {Proveniência:(Do gr. \textunderscore zoon\textunderscore  + \textunderscore phuton\textunderscore  + \textunderscore lithos\textunderscore )}
\end{itemize}
Zoóphyto fóssil.
\section{Zoophytologia}
\begin{itemize}
\item {Grp. gram.:f.}
\end{itemize}
Estudo scientífico dos zoóphytos.
(Cp. \textunderscore zoophytólogo\textunderscore )
\section{Zoophytológico}
\begin{itemize}
\item {Grp. gram.:adj.}
\end{itemize}
Relativo á zoophytologia.
\section{Zoophytólogo}
\begin{itemize}
\item {Grp. gram.:m.}
\end{itemize}
\begin{itemize}
\item {Proveniência:(Do gr. \textunderscore zoon\textunderscore  + \textunderscore phutos\textunderscore  + \textunderscore logos\textunderscore )}
\end{itemize}
Naturalista, que é perito em zoophytologia.
\section{Zoopónica}
\begin{itemize}
\item {Grp. gram.:f.}
\end{itemize}
Parte da Zoologia, que trata das descripções e observações.
\section{Zoópside}
\begin{itemize}
\item {Grp. gram.:f.}
\end{itemize}
\begin{itemize}
\item {Proveniência:(Do gr. \textunderscore zoon\textunderscore  + \textunderscore ops\textunderscore )}
\end{itemize}
Gênero de plantas hepáthicas da Nova-Zelândia.
\section{Zooquímica}
\begin{itemize}
\item {Grp. gram.:f.}
\end{itemize}
Química animal.
\section{Zooquímico}
\begin{itemize}
\item {Grp. gram.:adj.}
\end{itemize}
Relativo á zooquímica.
\section{Zooscopia}
\begin{itemize}
\item {Grp. gram.:f.}
\end{itemize}
\begin{itemize}
\item {Proveniência:(Do gr. \textunderscore zoon\textunderscore  + \textunderscore skopein\textunderscore )}
\end{itemize}
Observação scientífica de animaes a ôlho nú, ou com auxílio de uma lente ou de um microscópio.
\section{Zooscópico}
\begin{itemize}
\item {Grp. gram.:adj.}
\end{itemize}
Relativo á zooscopia.
\section{Zoospermo}
\begin{itemize}
\item {Grp. gram.:m.}
\end{itemize}
\begin{itemize}
\item {Proveniência:(Do gr. \textunderscore zoon\textunderscore  + \textunderscore sperma\textunderscore )}
\end{itemize}
Gênero de animálculos microscópicos, que vivem no esperma dos animaes, e dos quaes se diz que a accumulação de dois mil faria o volume de um grão de areia.
\section{Zoosporângio}
\begin{itemize}
\item {Grp. gram.:m.}
\end{itemize}
\begin{itemize}
\item {Proveniência:(De \textunderscore zoósporo\textunderscore  + gr. \textunderscore angeion\textunderscore )}
\end{itemize}
Céllula, onde se desenvolve o zoósporo.
\section{Zoosporango}
\begin{itemize}
\item {Grp. gram.:m.}
\end{itemize}
\begin{itemize}
\item {Proveniência:(De \textunderscore zoósporo\textunderscore  + gr. \textunderscore angeion\textunderscore )}
\end{itemize}
Céllula, onde se desenvolve o zoósporo.
\section{Zoospóreas}
\begin{itemize}
\item {Grp. gram.:f. pl.}
\end{itemize}
Família de plantas phýceas.
(Cp. \textunderscore zoósporo\textunderscore )
\section{Zoósporo}
\begin{itemize}
\item {Grp. gram.:m.}
\end{itemize}
\begin{itemize}
\item {Utilização:Bot.}
\end{itemize}
\begin{itemize}
\item {Proveniência:(Do gr. \textunderscore zoon\textunderscore  + \textunderscore spora\textunderscore )}
\end{itemize}
Esporo que contém cílios vibráteis, em certas algas.
\section{Zootaxia}
\begin{itemize}
\item {fónica:csi}
\end{itemize}
\begin{itemize}
\item {Grp. gram.:f.}
\end{itemize}
\begin{itemize}
\item {Proveniência:(Do gr. \textunderscore zoon\textunderscore  + \textunderscore taxis\textunderscore )}
\end{itemize}
Classificação dos animaes.
\section{Zootáxico}
\begin{itemize}
\item {fónica:csi}
\end{itemize}
\begin{itemize}
\item {Grp. gram.:adj.}
\end{itemize}
Relativo á zootaxia.
\section{Zooteca}
\begin{itemize}
\item {Grp. gram.:f.}
\end{itemize}
\begin{itemize}
\item {Proveniência:(Do gr. \textunderscore zoon\textunderscore  + \textunderscore theke\textunderscore )}
\end{itemize}
Óvulo macho, nas plantas acotiledóneas.
\section{Zootechnia}
\begin{itemize}
\item {Grp. gram.:f.}
\end{itemize}
\begin{itemize}
\item {Utilização:Ant.}
\end{itemize}
\begin{itemize}
\item {Proveniência:(Do gr. \textunderscore zoon\textunderscore  + \textunderscore tekhne\textunderscore )}
\end{itemize}
Arte de criar e aperfeiçoar animaes domésticos.
Arte de empalhar e conservar animaes.
\section{Zootéchnico}
\begin{itemize}
\item {Grp. gram.:adj.}
\end{itemize}
Relativo á zootechnia.
\section{Zootecnia}
\begin{itemize}
\item {Grp. gram.:f.}
\end{itemize}
\begin{itemize}
\item {Utilização:Ant.}
\end{itemize}
\begin{itemize}
\item {Proveniência:(Do gr. \textunderscore zoon\textunderscore  + \textunderscore tekhne\textunderscore )}
\end{itemize}
Arte de criar e aperfeiçoar animaes domésticos.
Arte de empalhar e conservar animaes.
\section{Zootécnico}
\begin{itemize}
\item {Grp. gram.:adj.}
\end{itemize}
Relativo á zootecnia.
\section{Zooterapia}
\begin{itemize}
\item {Grp. gram.:f.}
\end{itemize}
O mesmo que \textunderscore zooterapêutica\textunderscore .
\section{Zooterapêutica}
\begin{itemize}
\item {Grp. gram.:f.}
\end{itemize}
\begin{itemize}
\item {Proveniência:(De \textunderscore zoo...\textunderscore  + \textunderscore terapêutica\textunderscore )}
\end{itemize}
Terapêutica dos animaes.
\section{Zooterápico}
\begin{itemize}
\item {Grp. gram.:adj.}
\end{itemize}
Relativo á zooterapia.
\section{Zootheca}
\begin{itemize}
\item {Grp. gram.:f.}
\end{itemize}
\begin{itemize}
\item {Proveniência:(Do gr. \textunderscore zoon\textunderscore  + \textunderscore theke\textunderscore )}
\end{itemize}
Óvulo macho, nas plantas acotyledóneas.
\section{Zootherapia}
\begin{itemize}
\item {Grp. gram.:f.}
\end{itemize}
O mesmo que \textunderscore zootherapêutica\textunderscore .
\section{Zootherapêutica}
\begin{itemize}
\item {Grp. gram.:f.}
\end{itemize}
\begin{itemize}
\item {Proveniência:(De \textunderscore zoo...\textunderscore  + \textunderscore therapêutica\textunderscore )}
\end{itemize}
Therapêutica dos animaes.
\section{Zootherápico}
\begin{itemize}
\item {Grp. gram.:adj.}
\end{itemize}
Relativo á zootherapia.
\section{Zoótico}
\begin{itemize}
\item {Grp. gram.:adj.}
\end{itemize}
\begin{itemize}
\item {Utilização:Miner.}
\end{itemize}
\begin{itemize}
\item {Proveniência:(Do gr. \textunderscore zoon\textunderscore )}
\end{itemize}
Diz-se da rocha ou terreno, que contém corpos organizados.
\section{Zootomia}
\begin{itemize}
\item {Grp. gram.:f.}
\end{itemize}
\begin{itemize}
\item {Proveniência:(Do gr. \textunderscore zoon\textunderscore  + \textunderscore tome\textunderscore )}
\end{itemize}
Dissecção ou anatomia dos animaes.
\section{Zigena}
\begin{itemize}
\item {Grp. gram.:f.}
\end{itemize}
\begin{itemize}
\item {Proveniência:(Lat. \textunderscore zygaena\textunderscore )}
\end{itemize}
Gênero de insectos lepidópteros.
Peixe-martelo.
\section{Zigênidos}
\begin{itemize}
\item {Grp. gram.:m. pl.}
\end{itemize}
Família de insectos, que tem por tipo a zigena.
\section{Zignêmeas}
\begin{itemize}
\item {Grp. gram.:f. pl.}
\end{itemize}
\begin{itemize}
\item {Utilização:Bot.}
\end{itemize}
\begin{itemize}
\item {Proveniência:(T. mal der. do gr. \textunderscore zugos\textunderscore  + \textunderscore nema\textunderscore )}
\end{itemize}
Tríbo de algas.
\section{Zigóbato}
\begin{itemize}
\item {Grp. gram.:m.}
\end{itemize}
\begin{itemize}
\item {Proveniência:(Do gr. \textunderscore zugos\textunderscore  + \textunderscore bates\textunderscore )}
\end{itemize}
Gênero de peixes selácios.
\section{Zigócero}
\begin{itemize}
\item {Grp. gram.:adj.}
\end{itemize}
\begin{itemize}
\item {Utilização:Zool.}
\end{itemize}
\begin{itemize}
\item {Proveniência:(Do gr. \textunderscore zugos\textunderscore  + \textunderscore keras\textunderscore )}
\end{itemize}
Que tem tentáculos em número par.
\section{Zigodactilia}
\begin{itemize}
\item {Grp. gram.:f.}
\end{itemize}
\begin{itemize}
\item {Utilização:Zool.}
\end{itemize}
Qualidade de zigodáctilo.
\section{Zigodáctilo}
\begin{itemize}
\item {Grp. gram.:adj.}
\end{itemize}
\begin{itemize}
\item {Grp. gram.:M. pl.}
\end{itemize}
\begin{itemize}
\item {Proveniência:(Do gr. \textunderscore zugos\textunderscore  + \textunderscore daktulos\textunderscore )}
\end{itemize}
Que tem dedos em número par.
Aves trepadoras.
\section{Zigofiláceas}
\begin{itemize}
\item {Grp. gram.:f. pl.}
\end{itemize}
Família de plantas, que tem por tipo o zigofilo.
(Fem. pl. de \textunderscore zigofiláceo\textunderscore )
\section{Zigofiláceo}
\begin{itemize}
\item {Grp. gram.:adj.}
\end{itemize}
Relativo ou semelhante ao zigofilo.
\section{Zigofíleas}
\begin{itemize}
\item {Grp. gram.:f. pl.}
\end{itemize}
(V.zigofiláceas)
\section{Zigoma}
\begin{itemize}
\item {Grp. gram.:m.}
\end{itemize}
\begin{itemize}
\item {Proveniência:(Gr. \textunderscore zugoma\textunderscore )}
\end{itemize}
Osso da maçan do rosto.
\section{Zigomático}
\begin{itemize}
\item {Grp. gram.:adj.}
\end{itemize}
Relativo ao zigoma.
\section{Zigopétalo}
\begin{itemize}
\item {Grp. gram.:m.}
\end{itemize}
\begin{itemize}
\item {Proveniência:(Do gr. \textunderscore zugos\textunderscore  + \textunderscore petalon\textunderscore )}
\end{itemize}
Gênero de orquídeas.
\section{Zootómico}
\begin{itemize}
\item {Grp. gram.:adj.}
\end{itemize}
Relativo á zootomia.
\section{Zootomista}
\begin{itemize}
\item {Grp. gram.:m.  e  f.}
\end{itemize}
Pessôa, que trata de zootomia.
\section{Zootrópio}
\begin{itemize}
\item {Grp. gram.:m.}
\end{itemize}
\begin{itemize}
\item {Proveniência:(Do gr. \textunderscore zoon\textunderscore  + \textunderscore trepein\textunderscore )}
\end{itemize}
Apparelho, para fazer a sýnthese das imagens chromophotográphicas.
\section{Zopeiro}
\begin{itemize}
\item {Grp. gram.:adj.}
\end{itemize}
\begin{itemize}
\item {Proveniência:(De \textunderscore zopo\textunderscore )}
\end{itemize}
O mesmo ou melhór que \textunderscore zoupeiro\textunderscore .
\section{Zopilote}
\begin{itemize}
\item {Grp. gram.:m.}
\end{itemize}
Ave de rapina do México, (\textunderscore sarcorampus papa\textunderscore ).
\section{Zopissa}
\begin{itemize}
\item {Grp. gram.:f.}
\end{itemize}
\begin{itemize}
\item {Proveniência:(Lat. \textunderscore zopissa\textunderscore )}
\end{itemize}
Alcatrão ou pez, que se tira das querenas das embarcações velhas.
\section{Zopo}
\begin{itemize}
\item {fónica:zô}
\end{itemize}
\begin{itemize}
\item {Grp. gram.:m.}
\end{itemize}
\begin{itemize}
\item {Grp. gram.:Adj.}
\end{itemize}
Homem zoupeiro.
Diz-se do indivíduo trôpego; indolente. Cf. Garrett, \textunderscore Fábulas\textunderscore , 62.
(Cast. \textunderscore zopo\textunderscore )
\section{Zorame}
\begin{itemize}
\item {Grp. gram.:m.}
\end{itemize}
\begin{itemize}
\item {Utilização:Ant.}
\end{itemize}
Capa antiga, o mesmo que \textunderscore cerome\textunderscore . Cf. Herculano, 99, 125 e 127.
\section{Zorate}
\begin{itemize}
\item {Grp. gram.:adj.}
\end{itemize}
\begin{itemize}
\item {Utilização:Prov.}
\end{itemize}
Doido; maluco.
(Der. regressiva de \textunderscore zorates\textunderscore , por \textunderscore os orates\textunderscore . Cf. Júl. Moreira, \textunderscore Estudos\textunderscore , II, 131)
\section{Zorato}
\begin{itemize}
\item {Grp. gram.:adj.}
\end{itemize}
\begin{itemize}
\item {Utilização:Prov.}
\end{itemize}
O mesmo que \textunderscore zorate\textunderscore .
\section{Zorilha}
\begin{itemize}
\item {Grp. gram.:m.}
\end{itemize}
Mammífero mustelídeo, (\textunderscore mustella zorilla\textunderscore ).
\section{Zorlitho}
\begin{itemize}
\item {Grp. gram.:m.}
\end{itemize}
Espécie de veado, (\textunderscore cervus capreolus\textunderscore ), que é o corço vulgar.
\section{Zorlito}
\begin{itemize}
\item {Grp. gram.:m.}
\end{itemize}
Espécie de veado, (\textunderscore cervus capreolus\textunderscore ), que é o corço vulgar.
\section{Zornão}
\begin{itemize}
\item {Grp. gram.:adj.}
\end{itemize}
\begin{itemize}
\item {Utilização:Prov.}
\end{itemize}
\begin{itemize}
\item {Utilização:trasm.}
\end{itemize}
\begin{itemize}
\item {Utilização:Ext.}
\end{itemize}
\begin{itemize}
\item {Proveniência:(De \textunderscore zornar\textunderscore )}
\end{itemize}
Diz-se do burro, que zurra muito, sobretudo quando avista fêmea.
Femeeiro.
\section{Zornar}
\begin{itemize}
\item {Grp. gram.:v. i.}
\end{itemize}
\begin{itemize}
\item {Utilização:Prov.}
\end{itemize}
\begin{itemize}
\item {Utilização:trasm.}
\end{itemize}
O mesmo que \textunderscore zurrar\textunderscore ^1.
(Talvez misto de \textunderscore zurrar\textunderscore  e \textunderscore ornear\textunderscore )
\section{Zorô}
\begin{itemize}
\item {Grp. gram.:m.}
\end{itemize}
\begin{itemize}
\item {Utilização:Bras. do Rio}
\end{itemize}
Iguaria de camarões e quiabo.
\section{Zoroastrianismo}
\begin{itemize}
\item {Grp. gram.:m.}
\end{itemize}
O mesmo que \textunderscore zoroastrismo\textunderscore .
\section{Zoroastriano}
\begin{itemize}
\item {Grp. gram.:adj.}
\end{itemize}
Relativo a Zoroastro ou ao zoroastrismo.
\section{Zoroástrico}
\begin{itemize}
\item {Grp. gram.:adj.}
\end{itemize}
O mesmo que \textunderscore zoroastriano\textunderscore .
\section{Zoroastrismo}
\begin{itemize}
\item {Grp. gram.:m.}
\end{itemize}
Doutrina de Zoroastro.
\section{Zorongo}
\begin{itemize}
\item {Grp. gram.:m.}
\end{itemize}
\begin{itemize}
\item {Proveniência:(T. cast.)}
\end{itemize}
Dança espanhola, cujos passos se dirigem alternadamente para deante e para trás.
\section{Zorra}
\begin{itemize}
\item {fónica:zô}
\end{itemize}
\begin{itemize}
\item {Grp. gram.:f.}
\end{itemize}
\begin{itemize}
\item {Utilização:Fig.}
\end{itemize}
\begin{itemize}
\item {Utilização:Pesc.}
\end{itemize}
Carro muito baixo, com quatro rodas, para transporte de objectos muito pesados.
Apparelho sem rodas, geralmente feito de um tronco bifurcado, para transporte de grandes pedras ou de outros objectos muito pesados.
Coisa ou pessôa muito vagarosa.
Pequena rêde de arrastar, para a pesca do caranguejo.
Planta, o mesmo que \textunderscore alopecura\textunderscore .
(Cast. \textunderscore zorra\textunderscore )
\section{Zorra}
\begin{itemize}
\item {fónica:zô}
\end{itemize}
\begin{itemize}
\item {Grp. gram.:f.}
\end{itemize}
\begin{itemize}
\item {Utilização:T. de Sabugal}
\end{itemize}
Raposa velha. Cf. G. Vicente, \textunderscore Mofina Mendes\textunderscore .
O mesmo que \textunderscore rameira\textunderscore .
\section{Zorragar}
\begin{itemize}
\item {Grp. gram.:v. t.}
\end{itemize}
\begin{itemize}
\item {Proveniência:(De \textunderscore zorrague\textunderscore )}
\end{itemize}
O mesmo que \textunderscore azorragar\textunderscore .
\section{Zorrague}
\begin{itemize}
\item {Grp. gram.:m.}
\end{itemize}
\begin{itemize}
\item {Utilização:Des.}
\end{itemize}
O mesmo que \textunderscore azorrague\textunderscore .
\section{Zorral}
\begin{itemize}
\item {Grp. gram.:m.}
\end{itemize}
Ave, o mesmo que \textunderscore estorninho\textunderscore .
(Cp. \textunderscore zorzal\textunderscore )
\section{Zorrão}
\begin{itemize}
\item {Grp. gram.:m.}
\end{itemize}
\begin{itemize}
\item {Utilização:Prov.}
\end{itemize}
\begin{itemize}
\item {Utilização:trasm.}
\end{itemize}
\begin{itemize}
\item {Proveniência:(De \textunderscore zorra\textunderscore ^1)}
\end{itemize}
Pessôa indolente.
\section{Zorreiro}
\begin{itemize}
\item {Grp. gram.:m.  e  adj.}
\end{itemize}
\begin{itemize}
\item {Proveniência:(De \textunderscore zorra\textunderscore ^1)}
\end{itemize}
Homem vagaroso, ronceiro.
\section{Zorreiro}
\begin{itemize}
\item {Grp. gram.:adj.}
\end{itemize}
\begin{itemize}
\item {Utilização:Prov.}
\end{itemize}
\begin{itemize}
\item {Utilização:alg.}
\end{itemize}
Diz-se de uma variedade de milho, de cana baixa, mas muito productivo.
\section{Zorrilho}
\begin{itemize}
\item {Grp. gram.:m.}
\end{itemize}
\begin{itemize}
\item {Utilização:Bras. do S}
\end{itemize}
Mammífero, o mesmo que \textunderscore maritacaca\textunderscore .
(Cast. \textunderscore zorrillo\textunderscore )
\section{Zorro}
\begin{itemize}
\item {fónica:zô}
\end{itemize}
\begin{itemize}
\item {Grp. gram.:m.}
\end{itemize}
\begin{itemize}
\item {Utilização:Prov.}
\end{itemize}
\begin{itemize}
\item {Utilização:trasm.}
\end{itemize}
\begin{itemize}
\item {Utilização:Prov.}
\end{itemize}
\begin{itemize}
\item {Utilização:trasm.}
\end{itemize}
\begin{itemize}
\item {Grp. gram.:Adj.}
\end{itemize}
O mesmo que \textunderscore raposo\textunderscore .
Filho bastardo.
Criança, que os pais enjeitam e expõem ou mandam expôr á porta de outrem ou em lugar ermo. Cf. Trindade Coelho, \textunderscore Meus Amores\textunderscore , 1.^a ed., 128.
Criado velho.
Manhoso ou matreiro como as raposas.
(Cp. \textunderscore zorra\textunderscore ^2)
\section{Zorro}
\begin{itemize}
\item {fónica:zô}
\end{itemize}
\begin{itemize}
\item {Grp. gram.:m.  Loc.}
\end{itemize}
\begin{itemize}
\item {Utilização:Loc. de Arganil.}
\end{itemize}
\begin{itemize}
\item {Proveniência:(De \textunderscore zorra\textunderscore ^1)}
\end{itemize}
\textunderscore Andar de zorro\textunderscore , andar de rastos.
\section{Zorros, a}
\begin{itemize}
\item {Grp. gram.:loc. adv.}
\end{itemize}
\begin{itemize}
\item {Utilização:Prov.}
\end{itemize}
De rôjo; de rastos. (Colhido na Bairrada)
(Cp. \textunderscore zorra\textunderscore ^1)
\section{Zortar}
\begin{itemize}
\item {Grp. gram.:v. i.}
\end{itemize}
\begin{itemize}
\item {Utilização:Prov.}
\end{itemize}
\begin{itemize}
\item {Utilização:trasm.}
\end{itemize}
Sair; ir-se embora.
(Alter. de \textunderscore desertar\textunderscore ?)
\section{Zorzal}
\begin{itemize}
\item {Grp. gram.:m.}
\end{itemize}
Ave, o mesmo que \textunderscore estorninho\textunderscore .
(Cast. \textunderscore zorzal\textunderscore )
\section{Zorzaleiro}
\begin{itemize}
\item {Grp. gram.:adj.}
\end{itemize}
\begin{itemize}
\item {Utilização:Ant.}
\end{itemize}
\begin{itemize}
\item {Proveniência:(De \textunderscore zorzal\textunderscore )}
\end{itemize}
Bom para caçar zorzaes, (falando-se do falcão).
\section{Zostér}
\begin{itemize}
\item {Grp. gram.:m.}
\end{itemize}
\begin{itemize}
\item {Proveniência:(Do gr. \textunderscore zoster\textunderscore )}
\end{itemize}
Faixa, zona, cinta.
\section{Zostera}
\begin{itemize}
\item {Grp. gram.:f.}
\end{itemize}
\begin{itemize}
\item {Proveniência:(Do gr. \textunderscore zoster\textunderscore )}
\end{itemize}
Planta marítima.
\section{Zostéreas}
\begin{itemize}
\item {Grp. gram.:f. pl.}
\end{itemize}
Tribo de plantas, que tem por typo a zostera.
\section{Zote}
\begin{itemize}
\item {Grp. gram.:m.  e  adj.}
\end{itemize}
Idiota; pateta.
(Cast. \textunderscore zote\textunderscore )
\section{Zoteca}
\begin{itemize}
\item {Grp. gram.:f.}
\end{itemize}
\begin{itemize}
\item {Proveniência:(Lat. \textunderscore zotheca\textunderscore )}
\end{itemize}
Gabinete esconso, nas antigas habitações romanas, para estudo, ou para nele se descansar ou dormir de dia.
\section{Zotheca}
\begin{itemize}
\item {Grp. gram.:f.}
\end{itemize}
\begin{itemize}
\item {Proveniência:(Lat. \textunderscore zotheca\textunderscore )}
\end{itemize}
Gabinete esconso, nas antigas habitações romanas, para estudo, ou para nelle se descansar ou dormir de dia.
\section{Zotismo}
\begin{itemize}
\item {Grp. gram.:m.}
\end{itemize}
Estado ou condição de zote.
\section{Zouca}
\begin{itemize}
\item {Grp. gram.:f.}
\end{itemize}
\begin{itemize}
\item {Utilização:Gír. de pedreiros.}
\end{itemize}
Coisa.
(Metáth. de \textunderscore couza\textunderscore  = \textunderscore cousa\textunderscore , \textunderscore coisa\textunderscore )
\section{Zoupar}
\begin{itemize}
\item {Grp. gram.:v. t.}
\end{itemize}
\begin{itemize}
\item {Utilização:Prov.}
\end{itemize}
\begin{itemize}
\item {Utilização:minh.}
\end{itemize}
(V.zupar)
\section{Zoupeira}
\begin{itemize}
\item {Grp. gram.:f.}
\end{itemize}
\begin{itemize}
\item {Utilização:T. de Sabugal}
\end{itemize}
O mesmo que \textunderscore rameira\textunderscore .
\section{Zoupeiro}
\begin{itemize}
\item {Grp. gram.:adj.}
\end{itemize}
\begin{itemize}
\item {Utilização:Ext.}
\end{itemize}
\begin{itemize}
\item {Proveniência:(De \textunderscore zoupo\textunderscore )}
\end{itemize}
Trôpego.
Acanhado.
Indolente; mandrião.
\section{Zoupo}
\begin{itemize}
\item {Grp. gram.:m.}
\end{itemize}
O mesmo que \textunderscore zopo\textunderscore .
\section{Zoura}
\begin{itemize}
\item {Grp. gram.:f.}
\end{itemize}
\begin{itemize}
\item {Utilização:T. de Moncorvo}
\end{itemize}
O mesmo que \textunderscore diarreia\textunderscore .
\section{Zourar}
\begin{itemize}
\item {Grp. gram.:v. i.}
\end{itemize}
\begin{itemize}
\item {Utilização:T. de Moncorvo}
\end{itemize}
Têr diarreia.
\section{Zovo}
\begin{itemize}
\item {Grp. gram.:m.}
\end{itemize}
Hippopótamo da África Oriental.
\section{Zózima}
\begin{itemize}
\item {Grp. gram.:f.}
\end{itemize}
Gênero de plantas umbellíferas.
\section{Zózimo}
\begin{itemize}
\item {Grp. gram.:m.}
\end{itemize}
Gênero de crustáceos decápodes.
\section{Zuaque}
\begin{itemize}
\item {Grp. gram.:m.}
\end{itemize}
\begin{itemize}
\item {Utilização:Prov.}
\end{itemize}
\begin{itemize}
\item {Utilização:trasm.}
\end{itemize}
O mesmo que \textunderscore zuate\textunderscore ; zinote.
\section{Zuarte}
\begin{itemize}
\item {Grp. gram.:m.}
\end{itemize}
Pano azul ou preto, de algodão.
Ganga azul.
\section{Zuate}
\begin{itemize}
\item {Grp. gram.:f.}
\end{itemize}
\begin{itemize}
\item {Utilização:Prov.}
\end{itemize}
\begin{itemize}
\item {Utilização:trasm.}
\end{itemize}
\begin{itemize}
\item {Utilização:Chul.}
\end{itemize}
O mesmo que \textunderscore ânus\textunderscore .
\section{Zuavo}
\begin{itemize}
\item {Grp. gram.:m.}
\end{itemize}
\begin{itemize}
\item {Proveniência:(Fr. \textunderscore zouave\textunderscore )}
\end{itemize}
Soldado argelino, ao serviço da França.
\section{Zucar}
\begin{itemize}
\item {Grp. gram.:v. t.}
\end{itemize}
\begin{itemize}
\item {Utilização:Pop.}
\end{itemize}
Bater, soar.
(Por \textunderscore socar\textunderscore ?)
\section{Zuche}
\begin{itemize}
\item {Grp. gram.:m.}
\end{itemize}
Serpente do Brasil.
\section{Zuco}
\begin{itemize}
\item {Grp. gram.:adj.}
\end{itemize}
\begin{itemize}
\item {Utilização:T. da Bairrada}
\end{itemize}
Aparvalhado, tolo.
Bêbedo.
Que não sabe o que faz nem o que diz.
\section{Zuído}
\begin{itemize}
\item {Grp. gram.:m.}
\end{itemize}
\begin{itemize}
\item {Utilização:Prov.}
\end{itemize}
\begin{itemize}
\item {Proveniência:(De \textunderscore zuir\textunderscore )}
\end{itemize}
Sussurro nos ouvidos; zumbido.
\section{Zuidoiro}
\begin{itemize}
\item {fónica:zu-i}
\end{itemize}
\begin{itemize}
\item {Grp. gram.:m.}
\end{itemize}
\begin{itemize}
\item {Utilização:Prov.}
\end{itemize}
\begin{itemize}
\item {Proveniência:(De \textunderscore zuir\textunderscore )}
\end{itemize}
Zuído prolongado.
\section{Zuinglianismo}
\begin{itemize}
\item {Grp. gram.:m.}
\end{itemize}
\begin{itemize}
\item {Proveniência:(De \textunderscore Zwingli\textunderscore , n. p.)}
\end{itemize}
Seita religiosa, fundada por Zwingli na Suíça, nos princípios do século XVI.
\section{Zuingliano}
\begin{itemize}
\item {Grp. gram.:m.}
\end{itemize}
\begin{itemize}
\item {Proveniência:(De \textunderscore Zwingli\textunderscore , n. p.)}
\end{itemize}
Sectário do zuinglianismo.
\section{Zuir}
\begin{itemize}
\item {Grp. gram.:v. i.}
\end{itemize}
O mesmo que \textunderscore zunir\textunderscore .
Zumbir.
(Contr. de \textunderscore zunir\textunderscore )
\section{Zulo}
\begin{itemize}
\item {Grp. gram.:adj.}
\end{itemize}
\begin{itemize}
\item {Grp. gram.:M.}
\end{itemize}
\begin{itemize}
\item {Proveniência:(T. afr.)}
\end{itemize}
Relativo aos habitantes da Zululândia.
Habitante da Zululândia.
Língua dos Zulos.
\section{Zumba!}
\begin{itemize}
\item {Grp. gram.:interj.}
\end{itemize}
\begin{itemize}
\item {Proveniência:(T. onom.)}
\end{itemize}
Voz imitativa de pancada ou quéda.
\section{Zumbaia}
\begin{itemize}
\item {Grp. gram.:f.}
\end{itemize}
Cortesia exaggerada; grande mesura; salamaleque.
(Do ár.)
\section{Zumbaiar}
\begin{itemize}
\item {Grp. gram.:v. t.}
\end{itemize}
Fazer zumbaias a; lisonjear; bajular.
\section{Zumbaieiro}
\begin{itemize}
\item {Grp. gram.:m.}
\end{itemize}
Aquelle que faz zumbaias.
\section{Zumbar}
\begin{itemize}
\item {Grp. gram.:v. i.}
\end{itemize}
O mesmo que \textunderscore zumbir\textunderscore ; fazer zum-zum; fazer ruído.
(Cast. \textunderscore zumbar\textunderscore )
\section{Zumbar}
\begin{itemize}
\item {Grp. gram.:v. t.}
\end{itemize}
\begin{itemize}
\item {Utilização:Ant.}
\end{itemize}
O mesmo que \textunderscore zumbrar\textunderscore .
\section{Zumbar}
\begin{itemize}
\item {Grp. gram.:v. i.}
\end{itemize}
\begin{itemize}
\item {Utilização:Prov.}
\end{itemize}
\begin{itemize}
\item {Proveniência:(De \textunderscore zumba\textunderscore !)}
\end{itemize}
Dar pancadas em. Cf. Castilho, \textunderscore Fausto\textunderscore , 165.
\section{Zumbi}
\begin{itemize}
\item {Grp. gram.:m.}
\end{itemize}
\begin{itemize}
\item {Utilização:Bras}
\end{itemize}
Ente phantástico que, segundo a crença popular, vagueia dentro das casas a horas mortas.
(Do quimbundo)
\section{Zumbido}
\begin{itemize}
\item {Grp. gram.:m.}
\end{itemize}
Acto ou effeito de zumbir.
Ruído especial, que se sente nos ouvidos, em virtude de qualquer indisposição pathológica, ou em virtude de um estampido, explosão, ou qualquer estrondo exterior.
\section{Zumbidor}
\begin{itemize}
\item {Grp. gram.:adj.}
\end{itemize}
Que zumbe. Cf. Castilho, \textunderscore Fastos\textunderscore , II, 85.
\section{Zumbir}
\begin{itemize}
\item {Grp. gram.:v. i.}
\end{itemize}
\begin{itemize}
\item {Proveniência:(De \textunderscore zumbo\textunderscore )}
\end{itemize}
Fazer ruído, (falando-se de insectos que esvoaçam).
Sentirem (os ouvidos) ruído especial.
Cp. \textunderscore zumbido\textunderscore .
\section{Zumbo}
\begin{itemize}
\item {Grp. gram.:m.}
\end{itemize}
\begin{itemize}
\item {Proveniência:(T. onom.)}
\end{itemize}
Ruído confuso; rumor; zunido.
\section{Zumbrar}
\begin{itemize}
\item {Grp. gram.:v. i.}
\end{itemize}
(V.azumbrar)
\section{Zumbrir-se}
\begin{itemize}
\item {Grp. gram.:v. p.}
\end{itemize}
\begin{itemize}
\item {Utilização:Fig.}
\end{itemize}
Curvar-se.
Humilhar-se.
(Cp. \textunderscore azumbrar\textunderscore )
\section{Zum-zum}
\begin{itemize}
\item {Grp. gram.:m.}
\end{itemize}
(V.zunzum)
\section{Zuncho}
\begin{itemize}
\item {Grp. gram.:adj.}
\end{itemize}
\begin{itemize}
\item {Utilização:Gír.}
\end{itemize}
Que está de acôrdo.
\section{Zundos}
\begin{itemize}
\item {Grp. gram.:m. pl.}
\end{itemize}
Uma das categorias, em que se divide o séquito do soba dos Jingas.
\section{Zunga}
\begin{itemize}
\item {Grp. gram.:f.}
\end{itemize}
\begin{itemize}
\item {Utilização:Bras}
\end{itemize}
Insecto, o mesmo que \textunderscore nígua\textunderscore .
\section{Zungão}
\begin{itemize}
\item {Grp. gram.:m.}
\end{itemize}
\begin{itemize}
\item {Utilização:Prov.}
\end{itemize}
\begin{itemize}
\item {Utilização:trasm.}
\end{itemize}
Instrumento infantil, formado de uma lasca de madeira, que se faz zungar.
\section{Zungar}
\begin{itemize}
\item {Grp. gram.:v. i.}
\end{itemize}
\begin{itemize}
\item {Utilização:Prov.}
\end{itemize}
\begin{itemize}
\item {Utilização:trasm.}
\end{itemize}
Zumbir.
Silvar.
\section{Zungo}
\begin{itemize}
\item {Grp. gram.:m.}
\end{itemize}
Árvore da ilha de San-Thomé.
\section{Zungu}
\begin{itemize}
\item {Grp. gram.:m.}
\end{itemize}
\begin{itemize}
\item {Utilização:Bras}
\end{itemize}
Casa, dividida em pequenos compartimentos, que se alugam por baixo preço a gente ordinária ou de mau procedimento.
Conjunto de pequenas habitações, á maneira de cortiço.
\section{Zunideira}
\begin{itemize}
\item {Grp. gram.:f.}
\end{itemize}
\begin{itemize}
\item {Proveniência:(De \textunderscore zunir\textunderscore )}
\end{itemize}
Pedra, sôbre que os ourives alisam o oiro.
\section{Zunido}
\begin{itemize}
\item {Grp. gram.:m.}
\end{itemize}
Sibilo.
Zumbido.
Acto ou effeito de zunir.
\section{Zunidor}
\begin{itemize}
\item {Grp. gram.:adj.}
\end{itemize}
Que zune.
\section{Zunimento}
\begin{itemize}
\item {Grp. gram.:m.}
\end{itemize}
O mesmo que \textunderscore zunido\textunderscore .
\section{Zunir}
\begin{itemize}
\item {Grp. gram.:v. i.}
\end{itemize}
Produzir som agudo, (falando-se do vento, que se escôa por fendas, ramarias a árvores, etc.).
Zumbir.
Sibilar.
Soar asperamente.
(Cp. \textunderscore zunzum\textunderscore )
\section{Zunzum}
\begin{itemize}
\item {Grp. gram.:m.}
\end{itemize}
\begin{itemize}
\item {Proveniência:(T. onom.)}
\end{itemize}
Rumor; boato.
Zumbido.
Mexerico.
\section{Zunzunar}
\begin{itemize}
\item {Grp. gram.:v. i.}
\end{itemize}
\begin{itemize}
\item {Utilização:Neol.}
\end{itemize}
Fazer zun-zum; rumorejar.
\section{Zupa!}
\begin{itemize}
\item {Grp. gram.:interj.}
\end{itemize}
\begin{itemize}
\item {Proveniência:(T. onom.)}
\end{itemize}
Voz imitativa do som produzido por marrada.
\section{Zupar}
\begin{itemize}
\item {Grp. gram.:v. t.}
\end{itemize}
\begin{itemize}
\item {Utilização:Pop.}
\end{itemize}
\begin{itemize}
\item {Utilização:Fig.}
\end{itemize}
\begin{itemize}
\item {Proveniência:(De \textunderscore zupa!\textunderscore )}
\end{itemize}
Dar marradas em.
Bater; sovar. Cf. Camillo, \textunderscore Corja\textunderscore , 182.
\section{Zurame}
\begin{itemize}
\item {Grp. gram.:m.}
\end{itemize}
\begin{itemize}
\item {Utilização:Ant.}
\end{itemize}
Capa antiga, o mesmo que \textunderscore cerome\textunderscore . Cf. Herculano, 99, 125 e 127.
\section{Zuraque}
\begin{itemize}
\item {Grp. gram.:m.}
\end{itemize}
Planta convolvulácea, (\textunderscore convolvulus tricolor\textunderscore , Lin.).
\section{Zurbada}
\begin{itemize}
\item {Grp. gram.:f.}
\end{itemize}
\begin{itemize}
\item {Utilização:Prov.}
\end{itemize}
\begin{itemize}
\item {Utilização:trasm.}
\end{itemize}
Pancada; impulso.
Bátega: \textunderscore o trigo acamou-se com aquella zurbada de água\textunderscore .
(Relaciona-se provavelmente com \textunderscore zumbrar\textunderscore )
\section{Zurra}
\begin{itemize}
\item {Grp. gram.:f.}
\end{itemize}
\begin{itemize}
\item {Utilização:Prov.}
\end{itemize}
O mesmo que \textunderscore surra\textunderscore ^1; sova, pancadas.
\section{Zurracha}
\begin{itemize}
\item {Grp. gram.:f.}
\end{itemize}
Antiga embarcação de carreira ou passagem.
(Por \textunderscore zorracha\textunderscore , de \textunderscore zorra\textunderscore ^1?)
\section{Zurrada}
\begin{itemize}
\item {Grp. gram.:f.}
\end{itemize}
Acto de zurrar; zurraria. Cf. Guerreiro, \textunderscore Diccion. de Cons.\textunderscore , 53.
\section{Zurrador}
\begin{itemize}
\item {Grp. gram.:m.  e  adj.}
\end{itemize}
O que zurra.
\section{Zurrapa}
\begin{itemize}
\item {Grp. gram.:f.}
\end{itemize}
Vinho mau ou estragado.
Água-pé.
(Cast. \textunderscore zurrapa\textunderscore )
\section{Zurrar}
\begin{itemize}
\item {Grp. gram.:v. i.}
\end{itemize}
\begin{itemize}
\item {Grp. gram.:V. t.}
\end{itemize}
\begin{itemize}
\item {Utilização:Burl.}
\end{itemize}
\begin{itemize}
\item {Proveniência:(De \textunderscore zurro\textunderscore )}
\end{itemize}
Emittir zurro; ornejar.
Proferir nesciamente, tolamente.
\section{Zurrar}
\begin{itemize}
\item {Grp. gram.:v. i.}
\end{itemize}
\begin{itemize}
\item {Utilização:T. da Bairrada}
\end{itemize}
Dar zurra ou pancadas.
Beber bem, á larga: \textunderscore haja alegria e zurrem na pinga\textunderscore .
\section{Zurraria}
\begin{itemize}
\item {Grp. gram.:f.}
\end{itemize}
\begin{itemize}
\item {Proveniência:(De \textunderscore zurrar\textunderscore )}
\end{itemize}
Muitos zurros simultâneos.
\section{Zurre!}
\begin{itemize}
\item {Grp. gram.:interj.}
\end{itemize}
\begin{itemize}
\item {Utilização:Prov.}
\end{itemize}
\begin{itemize}
\item {Utilização:beir.}
\end{itemize}
Voz imperativa, para fazer sair ou para despedir; vai-te! sume-te!
\section{Zurro}
\begin{itemize}
\item {Grp. gram.:m.}
\end{itemize}
\begin{itemize}
\item {Proveniência:(T. onom.)}
\end{itemize}
Voz do burro.
Espécie de grande cègarrega, de que se tiram sons muito fortes, conhecida especialmente no norte do país, mas que também se ouve em Lisbôa, pelo Carnaval.
\section{Zurvada}
\begin{itemize}
\item {Grp. gram.:f.}
\end{itemize}
\begin{itemize}
\item {Utilização:Prov.}
\end{itemize}
\begin{itemize}
\item {Utilização:trasm.}
\end{itemize}
Forte bátega (de chuva).
\section{Zurvanada}
\begin{itemize}
\item {Grp. gram.:f.}
\end{itemize}
\begin{itemize}
\item {Utilização:Prov.}
\end{itemize}
\begin{itemize}
\item {Utilização:trasm.}
\end{itemize}
O mesmo que \textunderscore zurvada\textunderscore .
\section{Zurzidela}
\begin{itemize}
\item {Grp. gram.:f.}
\end{itemize}
Acto ou effeito de zurzir.
Sova, tunda.
\section{Zurzidor}
\begin{itemize}
\item {Grp. gram.:adj.}
\end{itemize}
Que zurze.
\section{Zurzir}
\begin{itemize}
\item {Grp. gram.:v. t.}
\end{itemize}
Açoitar.
Maltratar; espancar.
Castigar.
Criticar severamente.
Molestar.
(Cast. \textunderscore zurcir\textunderscore )
\section{Zuzara}
\begin{itemize}
\item {Grp. gram.:f.}
\end{itemize}
Gênero de crustáceos.
\section{Zygena}
\begin{itemize}
\item {Grp. gram.:f.}
\end{itemize}
\begin{itemize}
\item {Proveniência:(Lat. \textunderscore zygaena\textunderscore )}
\end{itemize}
Gênero de insectos lepidópteros.
Peixe-martelo.
\section{Zygênidos}
\begin{itemize}
\item {Grp. gram.:m. pl.}
\end{itemize}
Família de insectos, que tem por typo a zygena.
\section{Zygnêmeas}
\begin{itemize}
\item {Grp. gram.:f. pl.}
\end{itemize}
\begin{itemize}
\item {Utilização:Bot.}
\end{itemize}
\begin{itemize}
\item {Proveniência:(T. mal der. do gr. \textunderscore zugos\textunderscore  + \textunderscore nema\textunderscore )}
\end{itemize}
Tríbo de algas.
\section{Zygóbato}
\begin{itemize}
\item {Grp. gram.:m.}
\end{itemize}
\begin{itemize}
\item {Proveniência:(Do gr. \textunderscore zugos\textunderscore  + \textunderscore bates\textunderscore )}
\end{itemize}
Gênero de peixes selácios.
\section{Zygócero}
\begin{itemize}
\item {Grp. gram.:adj.}
\end{itemize}
\begin{itemize}
\item {Utilização:Zool.}
\end{itemize}
\begin{itemize}
\item {Proveniência:(Do gr. \textunderscore zugos\textunderscore  + \textunderscore keras\textunderscore )}
\end{itemize}
Que tem tentáculos em número par.
\section{Zygodactylia}
\begin{itemize}
\item {Grp. gram.:f.}
\end{itemize}
\begin{itemize}
\item {Utilização:Zool.}
\end{itemize}
Qualidade de zygodáctylo.
\section{Zygodáctylo}
\begin{itemize}
\item {Grp. gram.:adj.}
\end{itemize}
\begin{itemize}
\item {Grp. gram.:M. pl.}
\end{itemize}
\begin{itemize}
\item {Proveniência:(Do gr. \textunderscore zugos\textunderscore  + \textunderscore daktulos\textunderscore )}
\end{itemize}
Que tem dedos em número par.
Aves trepadoras.
\section{Zygoma}
\begin{itemize}
\item {Grp. gram.:m.}
\end{itemize}
\begin{itemize}
\item {Proveniência:(Gr. \textunderscore zugoma\textunderscore )}
\end{itemize}
Osso da maçan do rosto.
\section{Zygomático}
\begin{itemize}
\item {Grp. gram.:adj.}
\end{itemize}
Relativo ao zygoma.
\section{Zygômato-auricular}
\begin{itemize}
\item {Grp. gram.:adj.}
\end{itemize}
\begin{itemize}
\item {Utilização:Anat.}
\end{itemize}
Relativo ao zygoma e á orelha.
\section{Zygômato-labial}
\begin{itemize}
\item {Grp. gram.:adj.}
\end{itemize}
\begin{itemize}
\item {Utilização:Anat.}
\end{itemize}
Relativo ao zygoma e aos lábios.
\section{Zygômato-maxillar}
\begin{itemize}
\item {Grp. gram.:adj.}
\end{itemize}
\begin{itemize}
\item {Utilização:Anat.}
\end{itemize}
Relativo ao zygoma e á maxilla.
\section{Zygopétalo}
\begin{itemize}
\item {Grp. gram.:m.}
\end{itemize}
\begin{itemize}
\item {Proveniência:(Do gr. \textunderscore zugos\textunderscore  + \textunderscore petalon\textunderscore )}
\end{itemize}
Gênero de orchídeas.
\section{Zygophylláceas}
\begin{itemize}
\item {Grp. gram.:f. pl.}
\end{itemize}
Família de plantas, que tem por typo o zygophyllo.
(Fem. pl. de \textunderscore zygophylláceo\textunderscore )
\section{Zygophylláceo}
\begin{itemize}
\item {Grp. gram.:adj.}
\end{itemize}
Relativo ou semelhante ao zygophyllo.
\section{Zygophýlleas}
\begin{itemize}
\item {Grp. gram.:f. pl.}
\end{itemize}
(V.zygophylláceas)
\section{Zigofilo}
\begin{itemize}
\item {Grp. gram.:m.}
\end{itemize}
\begin{itemize}
\item {Proveniência:(Do gr. \textunderscore zugos\textunderscore  + \textunderscore phullon\textunderscore )}
\end{itemize}
Planta vermífuga, espécie de alcaparra.
\section{Zigósporo}
\begin{itemize}
\item {Grp. gram.:m.}
\end{itemize}
\begin{itemize}
\item {Utilização:Bot.}
\end{itemize}
\begin{itemize}
\item {Proveniência:(Do gr. \textunderscore zugos\textunderscore  + \textunderscore spora\textunderscore )}
\end{itemize}
Esporo, produzido pela soldadura de dois filamentos vizinhos, em certas espécies de cogumelos e algas microscópicas.
\section{Zigóstata}
\begin{itemize}
\item {Grp. gram.:m.}
\end{itemize}
\begin{itemize}
\item {Proveniência:(Lat. \textunderscore zygostata\textunderscore )}
\end{itemize}
Verificador ou inspector de pesos e medidas, entre os antigos Gregos e Romanos.
\section{Zigóstate}
\begin{itemize}
\item {Grp. gram.:m.}
\end{itemize}
\begin{itemize}
\item {Proveniência:(Lat. \textunderscore zygostates\textunderscore )}
\end{itemize}
O mesmo que \textunderscore zigóstata\textunderscore .
\section{Zilórgano}
\begin{itemize}
\item {Grp. gram.:m.}
\end{itemize}
Antigo instrumento músico.
\section{Zímase}
\begin{itemize}
\item {Grp. gram.:f.}
\end{itemize}
\begin{itemize}
\item {Utilização:Chím.}
\end{itemize}
\begin{itemize}
\item {Proveniência:(Do gr. \textunderscore zume\textunderscore )}
\end{itemize}
Fermento solúvel, não figurado.
O mesmo que \textunderscore diástase\textunderscore .
\section{Zimeóse}
\begin{itemize}
\item {Grp. gram.:f.}
\end{itemize}
Doença dos vinhos, que os torna grossos.
O mesmo que \textunderscore zímase\textunderscore .
\section{Zímico}
\begin{itemize}
\item {Grp. gram.:adj.}
\end{itemize}
\begin{itemize}
\item {Proveniência:(Do gr. \textunderscore zume\textunderscore )}
\end{itemize}
Relativo á fermentação.
Diz-se de um ácido, que é o mesmo que ácido láctico.
\section{Zimogenia}
\begin{itemize}
\item {Grp. gram.:f.}
\end{itemize}
\begin{itemize}
\item {Proveniência:(Do gr. \textunderscore zume\textunderscore  + \textunderscore genos\textunderscore )}
\end{itemize}
Fermentação química.
\section{Zimogênico}
\begin{itemize}
\item {Grp. gram.:adj.}
\end{itemize}
Relativo á zimogenia.
Que produz fermentação.
\section{Zimógeno}
\begin{itemize}
\item {Grp. gram.:adj.}
\end{itemize}
O mesmo que \textunderscore zimogênico\textunderscore .
\section{Zimologia}
\begin{itemize}
\item {Grp. gram.:f.}
\end{itemize}
\begin{itemize}
\item {Proveniência:(Do gr. \textunderscore zume\textunderscore  + \textunderscore logos\textunderscore )}
\end{itemize}
Tratado da fermentação.
\section{Zimológico}
\begin{itemize}
\item {Grp. gram.:adj.}
\end{itemize}
Relativo á zimologia.
\section{Zimoscópio}
\begin{itemize}
\item {Grp. gram.:m.}
\end{itemize}
\begin{itemize}
\item {Proveniência:(Do gr. \textunderscore sume\textunderscore  + \textunderscore skopein\textunderscore )}
\end{itemize}
O mesmo ou melhór que \textunderscore zimosímetro.\textunderscore 
\section{Zimose}
\begin{itemize}
\item {Grp. gram.:f.}
\end{itemize}
\begin{itemize}
\item {Utilização:Chím.}
\end{itemize}
Fermento solúvel. Cf. \textunderscore Jorn.-do-Comm.\textunderscore , do Rio, de 19-IX-901.
\section{Zimosimetria}
\begin{itemize}
\item {Grp. gram.:f.}
\end{itemize}
Aplicação do zimosímetro.
\section{Zimosimétrico}
\begin{itemize}
\item {Grp. gram.:adj.}
\end{itemize}
Relativo á zimosimetria.
\section{Zimosímetro}
\begin{itemize}
\item {Grp. gram.:m.}
\end{itemize}
\begin{itemize}
\item {Proveniência:(Do gr. \textunderscore zumosis\textunderscore  + \textunderscore metron\textunderscore )}
\end{itemize}
Instrumento, para calcular o grau da fermentação de um líquido.
\section{Zimotecnia}
\begin{itemize}
\item {Grp. gram.:f.}
\end{itemize}
\begin{itemize}
\item {Proveniência:(Do gr. \textunderscore zume\textunderscore  + \textunderscore techne\textunderscore )}
\end{itemize}
Arte de excitar e dirigir a fermentação.
\section{Zimotécnico}
\begin{itemize}
\item {Grp. gram.:adj.}
\end{itemize}
Relativo á zimotecnia.
\section{Zimótico}
\begin{itemize}
\item {Grp. gram.:adj.}
\end{itemize}
\begin{itemize}
\item {Proveniência:(Gr. \textunderscore zumotikos\textunderscore )}
\end{itemize}
Próprio para a fermentação; zímico.
\section{Zito}
\begin{itemize}
\item {Grp. gram.:m.}
\end{itemize}
\begin{itemize}
\item {Proveniência:(Lat. \textunderscore zythum\textunderscore )}
\end{itemize}
Espécie de cerveja, fabricada pelos antigos Egípcios.
\section{Zitógala}
\begin{itemize}
\item {Grp. gram.:f.}
\end{itemize}
\begin{itemize}
\item {Proveniência:(Do gr. \textunderscore zuthos\textunderscore  + \textunderscore gala\textunderscore )}
\end{itemize}
Mistura de leite com cerveja, usada como bebida entre alguns povos.
\section{Zygophyllo}
\begin{itemize}
\item {Grp. gram.:m.}
\end{itemize}
\begin{itemize}
\item {Proveniência:(Do gr. \textunderscore zugos\textunderscore  + \textunderscore phullon\textunderscore )}
\end{itemize}
Planta vermífuga, espécie de alcaparra.
\section{Zygósporo}
\begin{itemize}
\item {Grp. gram.:m.}
\end{itemize}
\begin{itemize}
\item {Utilização:Bot.}
\end{itemize}
\begin{itemize}
\item {Proveniência:(Do gr. \textunderscore zugos\textunderscore  + \textunderscore spora\textunderscore )}
\end{itemize}
Esporo, produzido pela soldadura de dois filamentos vizinhos, em certas espécies de cogumelos e algas microscópicas.
\section{Zygóstata}
\begin{itemize}
\item {Grp. gram.:m.}
\end{itemize}
\begin{itemize}
\item {Proveniência:(Lat. \textunderscore zygostata\textunderscore )}
\end{itemize}
Verificador ou inspector de pesos e medidas, entre os antigos Gregos e Romanos.
\section{Zygóstate}
\begin{itemize}
\item {Grp. gram.:m.}
\end{itemize}
\begin{itemize}
\item {Proveniência:(Lat. \textunderscore zygostates\textunderscore )}
\end{itemize}
O mesmo que \textunderscore zygóstata\textunderscore .
\section{Zylórgano}
\begin{itemize}
\item {Grp. gram.:m.}
\end{itemize}
Antigo instrumento músico.
\section{Zýmase}
\begin{itemize}
\item {Grp. gram.:f.}
\end{itemize}
\begin{itemize}
\item {Utilização:Chím.}
\end{itemize}
\begin{itemize}
\item {Proveniência:(Do gr. \textunderscore zume\textunderscore )}
\end{itemize}
Fermento solúvel, não figurado.
O mesmo que \textunderscore diástase\textunderscore .
\section{Zymeóse}
\begin{itemize}
\item {Grp. gram.:f.}
\end{itemize}
Doença dos vinhos, que os torna grossos.
O mesmo que \textunderscore zýmase\textunderscore .
\section{Zýmico}
\begin{itemize}
\item {Grp. gram.:adj.}
\end{itemize}
\begin{itemize}
\item {Proveniência:(Do gr. \textunderscore zume\textunderscore )}
\end{itemize}
Relativo á fermentação.
Diz-se de um ácido, que é o mesmo que ácido láctico.
\section{Zymogenia}
\begin{itemize}
\item {Grp. gram.:f.}
\end{itemize}
\begin{itemize}
\item {Proveniência:(Do gr. \textunderscore zume\textunderscore  + \textunderscore genos\textunderscore )}
\end{itemize}
Fermentação chímica.
\section{Zymogênico}
\begin{itemize}
\item {Grp. gram.:adj.}
\end{itemize}
Relativo á zymogenia.
Que produz fermentação.
\section{Zymógeno}
\begin{itemize}
\item {Grp. gram.:adj.}
\end{itemize}
O mesmo que \textunderscore zymogênico\textunderscore .
\section{Zymologia}
\begin{itemize}
\item {Grp. gram.:f.}
\end{itemize}
\begin{itemize}
\item {Proveniência:(Do gr. \textunderscore zume\textunderscore  + \textunderscore logos\textunderscore )}
\end{itemize}
Tratado da fermentação.
\section{Zymológico}
\begin{itemize}
\item {Grp. gram.:adj.}
\end{itemize}
Relativo á zymologia.
\section{Zymoscópio}
\begin{itemize}
\item {Grp. gram.:m.}
\end{itemize}
\begin{itemize}
\item {Proveniência:(Do gr. \textunderscore sume\textunderscore  + \textunderscore skopein\textunderscore )}
\end{itemize}
O mesmo ou melhór que \textunderscore zymosímetro.\textunderscore 
\section{Zymose}
\begin{itemize}
\item {Grp. gram.:f.}
\end{itemize}
\begin{itemize}
\item {Utilização:Chím.}
\end{itemize}
Fermento solúvel. Cf. \textunderscore Jorn.-do-Comm.\textunderscore , do Rio, de 19-IX-901.
\section{Zymosimetria}
\begin{itemize}
\item {Grp. gram.:f.}
\end{itemize}
Applicação do zymosímetro.
\section{Zymosimétrico}
\begin{itemize}
\item {Grp. gram.:adj.}
\end{itemize}
Relativo á zymosimetria.
\section{Zymosímetro}
\begin{itemize}
\item {Grp. gram.:m.}
\end{itemize}
\begin{itemize}
\item {Proveniência:(Do gr. \textunderscore zumosis\textunderscore  + \textunderscore metron\textunderscore )}
\end{itemize}
Instrumento, para calcular o grau da fermentação de um líquido.
\section{Zymotechnia}
\begin{itemize}
\item {Grp. gram.:f.}
\end{itemize}
\begin{itemize}
\item {Proveniência:(Do gr. \textunderscore zume\textunderscore  + \textunderscore techne\textunderscore )}
\end{itemize}
Arte de excitar e dirigir a fermentação.
\section{Zymotéchnico}
\begin{itemize}
\item {Grp. gram.:adj.}
\end{itemize}
Relativo á zimotechnia.
\section{Zymótico}
\begin{itemize}
\item {Grp. gram.:adj.}
\end{itemize}
\begin{itemize}
\item {Proveniência:(Gr. \textunderscore zumotikos\textunderscore )}
\end{itemize}
Próprio para a fermentação; zýmico.
\section{Zytho}
\begin{itemize}
\item {Grp. gram.:m.}
\end{itemize}
\begin{itemize}
\item {Proveniência:(Lat. \textunderscore zythum\textunderscore )}
\end{itemize}
Espécie de cerveja, fabricada pelos antigos Egýpcios.
\section{Zythógala}
\begin{itemize}
\item {Grp. gram.:f.}
\end{itemize}
\begin{itemize}
\item {Proveniência:(Do gr. \textunderscore zuthos\textunderscore  + \textunderscore gala\textunderscore )}
\end{itemize}
\end{document}